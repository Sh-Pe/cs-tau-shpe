\documentclass[]{article}

% Math packages
\usepackage[usenames]{color}
\usepackage{forest}
\usepackage{ifxetex,ifluatex,amsmath,amssymb,mathrsfs,amsthm,witharrows,mathtools}
\WithArrowsOptions{displaystyle}
\renewcommand{\qedsymbol}{$\blacksquare$} % end proofs with \blacksquare. Overwrites the defualts. 
\usepackage{cancel,bm}
\usepackage[thinc]{esdiff}

% tikz
\usepackage{tikz}
\newcommand\sqw{1}
\newcommand\squ[4][1]{\fill[#4] (#2*\sqw,#3*\sqw) rectangle +(#1*\sqw,#1*\sqw);}


% code 
\usepackage{listings}
\usepackage{xcolor}

\definecolor{codegreen}{rgb}{0,0.35,0}
\definecolor{codegray}{rgb}{0.5,0.5,0.5}
\definecolor{codenumber}{rgb}{0.1,0.3,0.5}
\definecolor{codeblue}{rgb}{0,0,0.5}
\definecolor{codered}{rgb}{0.5,0.03,0.02}
\definecolor{codegray}{rgb}{0.95,0.95,0.95}

\lstdefinestyle{pythonstylesheet}{
	language=Python,
	emphstyle=\color{deepred},
	backgroundcolor=\color{codegray},
	keywordstyle=\color{deepblue}\bfseries\itshape,
	numberstyle=\scriptsize\color{codenumber},
	basicstyle=\ttfamily\footnotesize,
	breakatwhitespace=false, 
	breaklines=true, 
	captionpos=b, 
	keepspaces=true, 
	numbers=left, 
	numbersep=5pt, 
	showspaces=false,                
	showstringspaces=false,
	showtabs=false, 
	tabsize=2, 
	morekeywords={object,type,isinstance,copy,deepcopy,zip,enumerate,reversed,list,set,len,dict,tuple,range,xrange,append,execfile,real,imag,reduce,str,repr},              % Add keywords here
	keywordstyle=\color{codeblue},
	emph={__init__,__add__,__mul__,__div__,__sub__,__call__,__getitem__,__setitem__,__eq__,__ne__,__nonzero__,__rmul__,__radd__,__repr__,__str__,__get__,__truediv__,__pow__,__name__,__future__,__all__,as,assert,nonlocal,with,yield,self,True,False,None,AssertionError,ValueError},          % Custom highlighting
	emphstyle=\color{codered},
	stringstyle=\color{codegreen},
	showstringspaces=false,
	abovecaptionskip=0pt,belowcaptionskip =0pt,
	framextopmargin=-\topsep, 
}
\newcommand\pythonstyle{\lstset{pythonstylesheet}}
\newcommand\pyl[1]     {{\lstinline!#1!}}
\lstset{style=pythonstylesheet}

\usepackage[style=1,skipbelow=\topskip,skipabove=\topskip,framemethod=TikZ]{mdframed}
\definecolor{bggray}{rgb}{0.85, 0.85, 0.85}
\mdfsetup{leftmargin=0pt,rightmargin=0pt,backgroundcolor=codegray,middlelinewidth=0.5pt,skipabove=4pt,skipbelow=0pt,middlelinecolor=black,roundcorner=5}
\BeforeBeginEnvironment{lstlisting}{\begin{mdframed}\vspace{-0.4em}}
	\AfterEndEnvironment{lstlisting}{\vspace{-0.8em}\end{mdframed}}

% Deisgn
\usepackage[labelfont=bf]{caption}
\usepackage[margin=0.6in]{geometry}
\usepackage{multicol}
\usepackage[skip=4pt, indent=0pt]{parskip}
\usepackage[normalem]{ulem}
\forestset{default}
\renewcommand\labelitemi{$\bullet$}
\graphicspath{ {./} }

% Hebrew initialzing
\usepackage[bidi=basic]{babel}
\PassOptionsToPackage{no-math}{fontspec}
\babelprovide[main, import]{hebrew}
\babelfont{rm}{David CLM}
\babelfont{sf}{David CLM}
\babelfont{tt}{Monaspace Argon}
\usepackage[shortlabels]{enumitem}
\newlist{hebenum}{enumerate}{1}

% Language Shortcuts
\newcommand\en[1] {\selectlanguage{english}#1\selectlanguage{hebrew}}
\newcommand\sen   {\selectlanguage{english}}
\newcommand\she   {\selectlanguage{hebrew}}
\newcommand\del   {$ \!\! $}
\newcommand\ttt[1]{\en{\small\texttt{#1}\normalsize}}

\newcommand\npage {\vfil {\hfil \textbf{\textit{המשך בעמוד הבא}}} \hfil \vfil \pagebreak}
\newcommand\ndoc  {\dotfill \\ \vfil {\begin{center} {\textbf{\textit{שחר פרץ, 2024}} \\ \scriptsize \textit{נוצר באמצעות תוכנה חופשית בלבד}} \end{center}} \vfil	}

\newcommand{\rn}[1]{
	\textup{\uppercase\expandafter{\romannumeral#1}}
}

\makeatletter
\newcommand{\skipitems}[1]{
	\addtocounter{\@enumctr}{#1}
}
\makeatother

%! ~~~ Math shortcuts ~~~

% Letters shortcuts
\newcommand\N     {\mathbb{N}}
\newcommand\Z     {\mathbb{Z}}
\newcommand\R     {\mathbb{R}}
\newcommand\Q     {\mathbb{Q}}
\newcommand\C     {\mathbb{C}}

\newcommand\ml    {\ell}
\newcommand\mj    {\jmath}
\newcommand\mi    {\imath}

\newcommand\powerset {\mathcal{P}}
\newcommand\ps    {\mathcal{P}}
\newcommand\pc    {\mathcal{P}}
\newcommand\ac    {\mathcal{A}}
\newcommand\bc    {\mathcal{B}}
\newcommand\cc    {\mathcal{C}}
\newcommand\dc    {\mathcal{D}}
\newcommand\ec    {\mathcal{E}}
\newcommand\fc    {\mathcal{F}}
\newcommand\nc    {\mathcal{N}}
\newcommand\sca   {\mathcal{S}} % \sc is already definded
\newcommand\rca   {\mathcal{R}} % \rc is already definded

\newcommand\Si    {\Sigma}

% Logic & sets shorcuts
\newcommand\siff  {\longleftrightarrow}
\newcommand\ssiff {\leftrightarrow}
\newcommand\so    {\longrightarrow}
\newcommand\sso   {\rightarrow}

\newcommand\epsi  {\epsilon}
\newcommand\vepsi {\varepsilon}
\newcommand\vphi  {\varphi}
\newcommand\Neven {\N_{\mathrm{even}}}
\newcommand\Nodd  {\N_{\mathrm{odd }}}
\newcommand\Zeven {\Z_{\mathrm{even}}}
\newcommand\Zodd  {\Z_{\mathrm{odd }}}
\newcommand\Np    {\N_+}

% Text Shortcuts
\newcommand\open  {\big(}
\newcommand\qopen {\quad\big(}
\newcommand\close {\big)}
\newcommand\also  {\text{, }}
\newcommand\defi  {\text{ definition}}
\newcommand\defis {\text{ definitions}}
\newcommand\given {\text{given }}
\newcommand\case  {\text{if }}
\newcommand\syx   {\text{ syntax}}
\newcommand\rle   {\text{ rule}}
\newcommand\other {\text{else}}
\newcommand\set   {\ell et \text{ }}
\newcommand\ans   {\mathit{Ans.}}

% Set theory shortcuts
\newcommand\ra    {\rangle}
\newcommand\la    {\langle}

\newcommand\oto   {\leftarrow}

\newcommand\QED   {\quad\quad\mathscr{Q.E.D.}\;\;\blacksquare}
\newcommand\QEF   {\quad\quad\mathscr{Q.E.F.}}
\newcommand\eQED  {\mathscr{Q.E.D.}\;\;\blacksquare}
\newcommand\eQEF  {\mathscr{Q.E.F.}}
\newcommand\jQED  {\mathscr{Q.E.D.}}

\newcommand\dom   {\mathrm{dom}}
\newcommand\Img   {\mathrm{Im}}
\newcommand\range {\mathrm{range}}

\newcommand\trio  {\triangle}

\newcommand\rc    {\right\rceil}
\newcommand\lc    {\left\lceil}
\newcommand\rf    {\right\rfloor}
\newcommand\lf    {\left\lfloor}

\newcommand\lex   {<_{lex}}

\newcommand\az    {\aleph_0}
\newcommand\uaz   {^{\aleph_0}}
\newcommand\al    {\aleph}
\newcommand\ual   {^\aleph}
\newcommand\taz   {2^{\aleph_0}}
\newcommand\utaz  { ^{\left (2^{\aleph_0} \right )}}
\newcommand\tal   {2^{\aleph}}
\newcommand\utal  { ^{\left (2^{\aleph} \right )}}
\newcommand\ttaz  {2^{\left (2^{\aleph_0}\right )}}

\newcommand\n     {$n$־יה\ }

% Math A&B shortcuts
\newcommand\logn  {\log n}
\newcommand\cosx  {\cos x}
\newcommand\cost  {\cos \theta}
\newcommand\sinx  {\sin x}
\newcommand\sint  {\sin \theta}
\newcommand\tanx  {\tan x}
\newcommand\tant  {\tan \theta}

\newcommand\seq   {\overset{!}{=}}
\newcommand\sle   {\overset{!}{\le}}
\newcommand\sge   {\overset{!}{\ge}}
\newcommand\sll   {\overset{!}{<}}
\newcommand\sgg   {\overset{!}{>}}

\newcommand\h     {\hat}
\newcommand\ve    {\vec}
\newcommand\lv    {\overrightarrow}
\newcommand\ol    {\overline}

\newcommand\mlcm  {\mathrm{lcm}}

\DeclareMathOperator{\bin}   {bin}

\DeclareMathOperator{\sech}   {sech}
\DeclareMathOperator{\csch}   {csch}
\DeclareMathOperator{\arcsec} {arcsec}
\DeclareMathOperator{\arccot} {arcCot}
\DeclareMathOperator{\arccsc} {arcCsc}
\DeclareMathOperator{\arccosh}{arccosh}
\DeclareMathOperator{\arcsinh}{arcsinh}
\DeclareMathOperator{\arctanh}{arctanh}
\DeclareMathOperator{\arcsech}{arcsech}
\DeclareMathOperator{\arccsch}{arccsch}
\DeclareMathOperator{\arccoth}{arccoth} 

\newcommand\dx    {\,\mathrm{d}x}
\newcommand\dt    {\,\mathrm{d}t}
\newcommand\dtt   {\,\mathrm{d}\theta}
\newcommand\df    {\mathrm{d}f}
\newcommand\dfdx  {\diff{f}{x}}
\newcommand\dit   {\limhz \frac{f(x + h) - f(x)}{h}}

\newcommand\nt[1] {\frac{#1}{#1}}

\newcommand\limz  {\lim_{x \to 0}}
\newcommand\limxz {\lim_{x \to x_0}}
\newcommand\limi  {\lim_{x \to \infty}}
\newcommand\limni {\lim_{x \to - \infty}}
\newcommand\limpmi{\lim_{x \to \pm \infty}}

\newcommand\ta    {\theta}
\newcommand\ap    {\alpha}

\renewcommand\inf {\infty}
\newcommand  \ninf{-\inf}

% Combinatorics shortcuts
\newcommand\sumnk     {\sum_{k = 0}^{n}}
\newcommand\sumni     {\sum_{i = 0}^{n}}
\newcommand\sumnko    {\sum_{k = 1}^{n}}
\newcommand\sumnio    {\sum_{i = 1}^{n}}
\newcommand\sumai     {\sum_{i = 1}^{n} A_i}
\newcommand\nsum[2]   {\reflectbox{\displaystyle\sum_{\reflectbox{\scriptsize$#1$}}^{\reflectbox{\scriptsize$#2$}}}}

\newcommand\bink      {\binom{n}{k}}
\newcommand\setn      {\{a_i\}^{2n}_{i = 1}}
\newcommand\setc[1]   {\{a_i\}^{#1}_{i = 1}}

\newcommand\cupain    {\bigcup_{i = 1}^{n} A_i}
\newcommand\cupai[1]  {\bigcup_{i = 1}^{#1} A_i}
\newcommand\cupiiai   {\bigcup_{i \in I} A_i}
\newcommand\capain    {\bigcap_{i = 1}^{n} A_i}
\newcommand\capai[1]  {\bigcap_{i = 1}^{#1} A_i}
\newcommand\capiiai   {\bigcap_{i \in I} A_i}

\newcommand\xot       {x_{1, 2}}
\newcommand\ano       {a_{n - 1}}
\newcommand\ant       {a_{n - 2}}

% Other shortcuts
\newcommand\tl    {\tilde}
\newcommand\op    {^{-1}}

\newcommand\sof[1]    {\left | #1 \right |}
\newcommand\cl [1]    {\left ( #1 \right )}
\newcommand\csb[1]    {\left [ #1 \right ]}

\newcommand\bs    {\blacksquare}

%! ~~~ Document ~~~

\author{שחר פרץ}
\title{מ.מ.למדמ"ח $\sim$ עמית ווינשטין $\sim$ קודים לתיקון שגיאות}
\date{19 ליוני 2024}

\begin{document}
	\maketitle
	\section{חזרה}
	הודעה מקודית באורך $k$. \\
	נשלח הודעה באורך $m$. \\
	פונ' קידוד $E \colon \{0, 1\}^{k} \to \{0, 1\}^{m}$. \\
	יתקיים $|\Img(E)| = 2^{k}$, ונסמן $C = \Img(E)$. 
	מרחק האמינג: $\Delta(x, y) = |\{i \mid x_i \neq y_i\}|$. \\
	מרחק של קוד: $d = \Delta(C) = \min_{x \neq y \in \{0, 1\}\}} \Delta(E(x), E(y))$. \\
	מטרה: $d$ גדול ו־$n$ קטן. החסם יהיה $d \le n - k + 1$. \\
	כמה שגיאות ניתן לזהות? $d - 1$; עבור כדור האמינג סביב $E(x)$ (כל האפשרויות במרחק האמינג $d - 1$ בהכרח $E(y)$ לא יכול להיות שם ולכן בוודאות נוכל לזהות זאת) \\
	כמה שגיאות ניתן לתקן? $\lf \frac{d - 1}{2} \rf$ כי כך כדורי ה־humming לא ידעו זה בזה ($\lf \frac{d - 1 }{2} \rf + \lf \frac{d - 1}{2} \rf < d$) כלומר אין חיתוך בין הכדורים, ובהכרח יהיה אפשר לזהות בצורה כזו או אחרת לתקן את הטעויות. 
	
	\section{אלגו'ים}
	אלגו' 1: לעבור על כל $2^k $ המילים ולמצוא את המרחק האמיניג המינימלי של מילה נתונה $x$, שידוע שהוא מתחת לחסם הדרוש כדי לתקן שגיאות, כלומר באיזה כדור האמינג היא נמצאת. סיבוכיות $2^k \cdot n$. תחת ההנחה שלהפעיל את הקוד לוקח $O(1)$. 
	
	אלגו' 2: נשנה את המילים עד שנקבל משהו מינימלי. סיבוכיות: $n^{\lf \frac{d - 1}{2}\rf}$. 
	
	אלו אלגוריתמים שמתבססים על ההנחה שהכי סביר להחזיר את המילה הקרובה ביותר במרחק humming. 
	
	\section{דוג'ים}
	לדוגמה עבורים $Rap_3$ (לחזור על כל ביט 3 פעמים) יתקיים $k = 1, n = 3, d = 3$, או באופן כללי $n = 3k, d = 3$ (עבור שינוי מינמלי של אות אחת, שיגרור מרחק האמינג של 3). עוד כמה דוגמאות: 
	\begin{center}
		\begin{tabular}{|c|ccc|c|} \hline
			name & $k$ &$n$ &$d$ &תיאור  \\ \hline 
			$Rep_3$ & $k$ & $3k$ & $3$ &חזרה 3 פעמים\\ \hline 
			$Rep_t$ & $k$ & $tk$ & $t$ &חזרה $t$ פעמים \\ \hline
			$Par$ & $2$ & $3$ & $2$ &הוספת ביט זוגיות \\ \hline
			$Par$ & $k$ & $k + 1$ & $2$ & ראה לעיל \\ \hline
			משחק קלפים & $5^2$ & $6^2$ & $4$ & משהו מהשבוע שעבר \\ \hline
		\end{tabular}
	\end{center}
	
	\textit{הערה: }הוספת ביט זוגיות הוא ה־xor של כל הערכים, שמסומן ב־$\oplus$ – ההפוך לאמ"מ, או אך ורק אחד משתי האפשרויות (כמו לשאול ילד, אתה רוצה גליה או עוגת שוקולד). בפייתון נשתמש ב־\^{} בשביל xor. תכונות: 
	\[ A \oplus B = 0 \iff A = B, \ (A \oplus B)\oplus C = A \oplus (B \oplus C), \ A \oplus B = B \oplus A \]
	xor של  רצף ערכים בינאי ייצג את הזוגיות של העמודה. 
	
	וכפועל: קיסרנו, נקסר, הקסרה. 
	
	\section{Index Code}
	נבחר $k = 2^{\ell} - 1$ עבור $\ell$ כלשהו. 
	
	נקודד הודעה $x_1, \dots, x_k$ באופן הבא: 

	נגדיר: 
	\[ EC(x) = \bigoplus \big\{ \bin(i) \mid \iota i. x_i = 1 \big\} \]
	וגם (עיגול קטן מסמל שירשור)
	\[ E(x) = x ^\circ EC(X) \]
	דוג':  
	
	\sen
	\ttt{\ \ 1234567} \\
	\ttt{\ \ 0110110} \\
	\ttt{2 010} \\
	\ttt{3 011} \\
	\ttt{5 101} \\
	\ttt{6 110} \\
	\ttt{---------} \\
	\ttt{\ \ 010} 
	
	\she
	כלומר $EC(x) = 010, E(x) = 0110110010$. 
	יתקיים $n = k + \ell = k + \log(k)$. 
	
	ועבור $d$? טענה: $d \ge 2$. הסבר: אם $\Delta(x, y) \ge 2$, סיימנו. אחרת, אם $\Delta(x, y) = 1$, אזי בהכרח קיים אינדקס יחיד $i$ כך ש־$x_i \neq y_i$, ולכן $EC(x)$ ו־$EC9y)$ שונים, כלומר $\Delta(EC(x), EC(y)) \ge 2$. 
	
	הוכחה ש־$d \le 2$: נראה דוג'
	\begin{align}
		&EC(x) = 000, \ &x \colon 0000000 \\
		&EC(y) = 001, \ &y \colon 1000000
	\end{align}
	
	סה"כ $d = 2$ כי הוכחנו שני חסמים. 
	
	לא מדהים. 
	\subsection{לא מדהים, גרסה 2}
	רעיון: נשרשר את $EC(x)$ פעמיים: 
	$E(x) = x \circ EC(x) \circ EC(x)$
	עתה, יתקיים $ = 2^{\ell} - 1$, וגם $n = k + \ell = k + 2\log k$, ואחרונה $d = 3$: ניתן חסם תחתון ועליון. 
	
	$d \ge 3$: אם $\Delta(x, y) \ge 3$ סיימנו. אם $\Delta(x, y) = 1$ אזי $EC(x) \neq EC(y) \implies \Delta(E(x), E(y)) \ge 3$. אם $\Delta(x, y) = 2 $ אזי $\exists i, i'. x_i \neq y_i \land x_i' \neq y_i'$. כלומר, $EC(x) \oplus EC(y) = i \oplus i'$ כלומר $\Delta(E(x), E(y)) \ge 4$. 
	
	כיוון שני: $d \le 3$. נשתמש באותה הדוגמה. $x = 0000000$, $y = 1000000$, נקבל $E(x) = 000, EC(y) = 001$ וסה"כ נקבל מרחק $3$. 
	
	סה"כ $d = 3$. 
	
	\subsubsection{זיהוי שגיאות}
	תיאורטית, ניתן לתקן שגיאה אחת. איך נעשה זאת, באופן יעיל? 
	קלט: $y = x \circ EC_1, EC_2$, שזה שיבוש לכל היותר אחד על $x' \circ EC(x') \circ EC(x')$. נפלג למקרים: 
	\begin{itemize}
		\item $\impliedby EC_1 \neq EC_2 $ יש שגיאה ב־$EC_1$ או $EC_2$, כלומר $x$ תקין ונחזיר אותו. 
		\item $\impliedby EC_1 = EC_2$ 
		\begin{itemize}
			\item אם $EC(x) = EC_1$, אז אין שגיאה, ונחזיר את $x$. 
			\item אחרת, האינדקס ב־$x$ בו יש טעות הוא $EC(x) \oplus EC_1(x)$, ונחזיר את $x$ אחרי שנהפוך את הקלט הזה. 
		\end{itemize}
	\end{itemize}
	
	\subsection{יותר מדהים}
	רעיון: נשרשר ביט זוגיות (נקסר את כל $x$): 
	\[ E(x) = x \circ EC(x) \circ EC(x) \circ Par(x) \]
	\textbf{טענה: }$d = 4$. 
	
	\textbf{טענה: }עבור קוד עם מרחק אי־זוגי $d$, הוספת ביט זוגיות מעלה את המחק ל־$d + 1$. 
	
	\begin{proof}
		נפצל למקרים. 
		\begin{itemize}
			\item $\Delta(E(x), E(y)) \ge d + 1$: סיימנו
			\item $\Delta(E(x), E(y)) = d $: ביט הזוגיות יהיה כונה ולכן נקבל מרחק $d + 1$. 
		\end{itemize}
	\end{proof}
	
	\section{קוד Hamming}
	שמם נגזר (למיטב זכורנו של המורה) מהעובדה שכדורי ההאמינג ברדיוס $1$ מהווים ריצוף מרחק של המושלם או משהו כזה. 
	משפחת קודים של $(n, k, d) = (2^{t}, 2^{t} - 1, 3)$. הדוגמה הכי פופולארית: 
	בפרט, קוד $(7, 4, 3)$. נניח שהקלט הוא $x_3, x_5, x_6, x_7$. נוסיף את הביטים החסרים. נוסיף את הביטים: 
	\begin{align*}
		x_1 = x_3 \oplus x_5 \oplus x_7 \\
		x_2 = x_3 \oplus  x_6 \oplus x_7 \\
		x_4 = x_5 \oplus x_6 \oplus x_7
	\end{align*}
	
	זהו קוד עבורו $d = 3$. 
	\begin{proof}
		תעברו על המילים ותבדקו
	\end{proof}
	
\end{document}