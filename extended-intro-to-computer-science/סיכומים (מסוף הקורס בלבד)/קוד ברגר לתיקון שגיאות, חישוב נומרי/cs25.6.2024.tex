\documentclass[]{article}

% Math packages
\usepackage[usenames]{color}
\usepackage{forest}
\usepackage{ifxetex,ifluatex,amsmath,amssymb,mathrsfs,amsthm,witharrows,mathtools}
\WithArrowsOptions{displaystyle}
\renewcommand{\qedsymbol}{$\blacksquare$} % end proofs with \blacksquare. Overwrites the defualts. 
\usepackage{cancel,bm}
\usepackage[thinc]{esdiff}

% tikz
\usepackage{tikz}
\newcommand\sqw{1}
\newcommand\squ[4][1]{\fill[#4] (#2*\sqw,#3*\sqw) rectangle +(#1*\sqw,#1*\sqw);}


% code 
\usepackage{listings}
\usepackage{xcolor}

\definecolor{codegreen}{rgb}{0,0.35,0}
\definecolor{codegray}{rgb}{0.5,0.5,0.5}
\definecolor{codenumber}{rgb}{0.1,0.3,0.5}
\definecolor{codeblue}{rgb}{0,0,0.5}
\definecolor{codered}{rgb}{0.5,0.03,0.02}
\definecolor{codegray}{rgb}{0.96,0.96,0.96}

\lstdefinestyle{pythonstylesheet}{
	language=Python,
	emphstyle=\color{deepred},
	backgroundcolor=\color{codegray},
	keywordstyle=\color{deepblue}\bfseries\itshape,
	numberstyle=\scriptsize\color{codenumber},
	basicstyle=\ttfamily\footnotesize,
	breakatwhitespace=false, 
	breaklines=true, 
	captionpos=b, 
	keepspaces=true, 
	numbers=left, 
	numbersep=5pt, 
	showspaces=false,                
	showstringspaces=false,
	showtabs=false, 
	tabsize=4, 
	morekeywords={object,type,isinstance,copy,deepcopy,zip,enumerate,reversed,list,set,len,dict,tuple,range,xrange,append,execfile,real,imag,reduce,str,repr},              % Add keywords here
	keywordstyle=\color{codeblue},
	emph={__init__,__add__,__mul__,__div__,__sub__,__call__,__getitem__,__setitem__,__eq__,__ne__,__nonzero__,__rmul__,__radd__,__repr__,__str__,__get__,__truediv__,__pow__,__name__,__future__,__all__,as,assert,nonlocal,with,yield,self,True,False,None,AssertionError,ValueError},          % Custom highlighting
	emphstyle=\color{codered},
	stringstyle=\color{codegreen},
	showstringspaces=false,
	abovecaptionskip=0pt,belowcaptionskip =0pt,
	framextopmargin=-\topsep, 
}
\newcommand\pythonstyle{\lstset{pythonstylesheet}}
\newcommand\pyl[1]     {{\lstinline!#1!}}
\lstset{style=pythonstylesheet}

\usepackage[style=1,skipbelow=\topskip,skipabove=\topskip,framemethod=TikZ]{mdframed}
\definecolor{bggray}{rgb}{0.85, 0.85, 0.85}
\mdfsetup{leftmargin=0pt,rightmargin=0pt,backgroundcolor=codegray,middlelinewidth=0.5pt,skipabove=5pt,skipbelow=0pt,middlelinecolor=black,roundcorner=5}
\BeforeBeginEnvironment{lstlisting}{\begin{mdframed}\vspace{-0.4em}}
	\AfterEndEnvironment{lstlisting}{\vspace{-0.8em}\end{mdframed}}

% Deisgn
\usepackage[labelfont=bf]{caption}
\usepackage[margin=0.6in]{geometry}
\usepackage{multicol}
\usepackage[skip=4pt, indent=0pt]{parskip}
\usepackage[normalem]{ulem}
\forestset{default}
\renewcommand\labelitemi{$\bullet$}
\usepackage{titlesec}
\titleformat{\section}[block]
{\fontsize{15}{15}}
{\sen \dotfill \she}
{0em}
{\MakeUppercase}
\usepackage{graphicx}
\graphicspath{ {./} }

% Hebrew initialzing
\usepackage[bidi=basic]{babel}
\PassOptionsToPackage{no-math}{fontspec}
\babelprovide[main, import]{hebrew}
\babelprovide[import]{english}
\babelfont[hebrew]{rm}{David CLM}
\babelfont[hebrew]{sf}{David CLM}
\babelfont[english]{tt}{Monaspace Neon}
\usepackage[shortlabels]{enumitem}
\newlist{hebenum}{enumerate}{1}

% Language Shortcuts
\newcommand\en[1] {\begin{otherlanguage}{english}#1\end{otherlanguage}}
\newcommand\sen   {\begin{otherlanguage}{english}}
	\newcommand\she   {\end{otherlanguage}}
\newcommand\del   {$ \!\! $}
\newcommand\ttt[1]{\en{\footnotesize\texttt{#1}\normalsize}}

\newcommand\npage {\vfil {\hfil \textbf{\textit{המשך בעמוד הבא}}} \hfil \vfil \pagebreak}
\newcommand\ndoc  {\dotfill \\ \vfil {\begin{center} {\textbf{\textit{שחר פרץ, 2024}} \\ \scriptsize \textit{נוצר באמצעות תוכנה חופשית בלבד}} \end{center}} \vfil	}

\newcommand{\rn}[1]{
	\textup{\uppercase\expandafter{\romannumeral#1}}
}

\makeatletter
\newcommand{\skipitems}[1]{
	\addtocounter{\@enumctr}{#1}
}
\makeatother

%! ~~~ Math shortcuts ~~~

% Letters shortcuts
\newcommand\N     {\mathbb{N}}
\newcommand\Z     {\mathbb{Z}}
\newcommand\R     {\mathbb{R}}
\newcommand\Q     {\mathbb{Q}}
\newcommand\C     {\mathbb{C}}

\newcommand\ml    {\ell}
\newcommand\mj    {\jmath}
\newcommand\mi    {\imath}

\newcommand\powerset {\mathcal{P}}
\newcommand\ps    {\mathcal{P}}
\newcommand\pc    {\mathcal{P}}
\newcommand\ac    {\mathcal{A}}
\newcommand\bc    {\mathcal{B}}
\newcommand\cc    {\mathcal{C}}
\newcommand\dc    {\mathcal{D}}
\newcommand\ec    {\mathcal{E}}
\newcommand\fc    {\mathcal{F}}
\newcommand\nc    {\mathcal{N}}
\newcommand\sca   {\mathcal{S}} % \sc is already definded
\newcommand\rca   {\mathcal{R}} % \rc is already definded

\newcommand\Si    {\Sigma}

% Logic & sets shorcuts
\newcommand\siff  {\longleftrightarrow}
\newcommand\ssiff {\leftrightarrow}
\newcommand\so    {\longrightarrow}
\newcommand\sso   {\rightarrow}

\newcommand\epsi  {\epsilon}
\newcommand\vepsi {\varepsilon}
\newcommand\vphi  {\varphi}
\newcommand\Neven {\N_{\mathrm{even}}}
\newcommand\Nodd  {\N_{\mathrm{odd }}}
\newcommand\Zeven {\Z_{\mathrm{even}}}
\newcommand\Zodd  {\Z_{\mathrm{odd }}}
\newcommand\Np    {\N_+}

% Text Shortcuts
\newcommand\open  {\big(}
\newcommand\qopen {\quad\big(}
\newcommand\close {\big)}
\newcommand\also  {\text{, }}
\newcommand\defi  {\text{ definition}}
\newcommand\defis {\text{ definitions}}
\newcommand\given {\text{given }}
\newcommand\case  {\text{if }}
\newcommand\syx   {\text{ syntax}}
\newcommand\rle   {\text{ rule}}
\newcommand\other {\text{else}}
\newcommand\set   {\ell et \text{ }}
\newcommand\ans   {\mathit{Ans.}}

% Set theory shortcuts
\newcommand\ra    {\rangle}
\newcommand\la    {\langle}

\newcommand\oto   {\leftarrow}

\newcommand\QED   {\quad\quad\mathscr{Q.E.D.}\;\;\blacksquare}
\newcommand\QEF   {\quad\quad\mathscr{Q.E.F.}}
\newcommand\eQED  {\mathscr{Q.E.D.}\;\;\blacksquare}
\newcommand\eQEF  {\mathscr{Q.E.F.}}
\newcommand\jQED  {\mathscr{Q.E.D.}}

\newcommand\dom   {\text{dom}}
\newcommand\Img   {\text{Im}}
\newcommand\range {\text{range}}

\newcommand\trio  {\triangle}

\newcommand\rc    {\right\rceil}
\newcommand\lc    {\left\lceil}
\newcommand\rf    {\right\rfloor}
\newcommand\lf    {\left\lfloor}

\newcommand\lex   {<_{lex}}

\newcommand\az    {\aleph_0}
\newcommand\uaz   {^{\aleph_0}}
\newcommand\al    {\aleph}
\newcommand\ual   {^\aleph}
\newcommand\taz   {2^{\aleph_0}}
\newcommand\utaz  { ^{\left (2^{\aleph_0} \right )}}
\newcommand\tal   {2^{\aleph}}
\newcommand\utal  { ^{\left (2^{\aleph} \right )}}
\newcommand\ttaz  {2^{\left (2^{\aleph_0}\right )}}

\newcommand\n     {$n$־יה\ }

% Math A&B shortcuts
\newcommand\logn  {\log n}
\newcommand\cosx  {\cos x}
\newcommand\cost  {\cos \theta}
\newcommand\sinx  {\sin x}
\newcommand\sint  {\sin \theta}
\newcommand\tanx  {\tan x}
\newcommand\tant  {\tan \theta}

\newcommand\seq   {\overset{!}{=}}
\newcommand\sle   {\overset{!}{\le}}
\newcommand\sge   {\overset{!}{\ge}}
\newcommand\sll   {\overset{!}{<}}
\newcommand\sgg   {\overset{!}{>}}

\newcommand\h     {\hat}
\newcommand\ve    {\vec}
\newcommand\lv    {\overrightarrow}
\newcommand\ol    {\overline}

\newcommand\mlcm  {\mathrm{lcm}}

\DeclareMathOperator{\sech}   {sech}
\DeclareMathOperator{\csch}   {csch}
\DeclareMathOperator{\arcsec} {arcsec}
\DeclareMathOperator{\arccot} {arcCot}
\DeclareMathOperator{\arccsc} {arcCsc}
\DeclareMathOperator{\arccosh}{arccosh}
\DeclareMathOperator{\arcsinh}{arcsinh}
\DeclareMathOperator{\arctanh}{arctanh}
\DeclareMathOperator{\arcsech}{arcsech}
\DeclareMathOperator{\arccsch}{arccsch}
\DeclareMathOperator{\arccoth}{arccoth} 

\newcommand\dx    {\,\mathrm{d}x}
\newcommand\dt    {\,\mathrm{d}t}
\newcommand\dtt   {\,\mathrm{d}\theta}
\newcommand\df    {\mathrm{d}f}
\newcommand\dfdx  {\diff{f}{x}}
\newcommand\dit   {\limhz \frac{f(x + h) - f(x)}{h}}

\newcommand\nt[1] {\frac{#1}{#1}}

\newcommand\limz  {\lim_{x \to 0}}
\newcommand\limxz {\lim_{x \to x_0}}
\newcommand\limi  {\lim_{x \to \infty}}
\newcommand\limh  {\lim_{x \to 0}}
\newcommand\limni {\lim_{x \to - \infty}}
\newcommand\limpmi{\lim_{x \to \pm \infty}}

\newcommand\ta    {\theta}
\newcommand\ap    {\alpha}

\renewcommand\inf {\infty}
\newcommand  \ninf{-\inf}

% Combinatorics shortcuts
\newcommand\sumnk     {\sum_{k = 0}^{n}}
\newcommand\sumni     {\sum_{i = 0}^{n}}
\newcommand\sumnko    {\sum_{k = 1}^{n}}
\newcommand\sumnio    {\sum_{i = 1}^{n}}
\newcommand\sumai     {\sum_{i = 1}^{n} A_i}
\newcommand\nsum[2]   {\reflectbox{\displaystyle\sum_{\reflectbox{\scriptsize$#1$}}^{\reflectbox{\scriptsize$#2$}}}}

\newcommand\bink      {\binom{n}{k}}
\newcommand\setn      {\{a_i\}^{2n}_{i = 1}}
\newcommand\setc[1]   {\{a_i\}^{#1}_{i = 1}}

\newcommand\cupain    {\bigcup_{i = 1}^{n} A_i}
\newcommand\cupai[1]  {\bigcup_{i = 1}^{#1} A_i}
\newcommand\cupiiai   {\bigcup_{i \in I} A_i}
\newcommand\capain    {\bigcap_{i = 1}^{n} A_i}
\newcommand\capai[1]  {\bigcap_{i = 1}^{#1} A_i}
\newcommand\capiiai   {\bigcap_{i \in I} A_i}

\newcommand\xot       {x_{1, 2}}
\newcommand\ano       {a_{n - 1}}
\newcommand\ant       {a_{n - 2}}

% Other shortcuts
\newcommand\tl    {\tilde}
\newcommand\op    {^{-1}}

\newcommand\sof[1]    {\left | #1 \right |}
\newcommand\cl [1]    {\left ( #1 \right )}
\newcommand\csb[1]    {\left [ #1 \right ]}

\newcommand\bs    {\blacksquare}

%! ~~~ Document ~~~

\author{שחר פרץ}
\title{מ.מ.למדמ"ח $\sim$ עמית וינשטיין $\sim$ קוד ברגר וחישוב נומרי}
\date{25 ליוני 2024}

\begin{document}
	\maketitle
	\section{\sen Reminders \she}
	\sen
	\subsection{Index Code}
	
	\begin{alignat*}{9}
		&&& EC(x) = \bigoplus_{x_i = 1}i \\
		IC:  \quad &&& x \circ EC(x) & \quad d = 2\\
		IC_2:\quad &&& x \circ EC(x) \circ EC(x) & \quad d = 3\\
		IC_3:\quad &&& x \circ EC(x) \circ EC(x) \circ par(x) & \quad d = 4
	\end{alignat*}
	\subsection{Hamming Code (743)}
	The messeage: $x_3x_5x_6x_7$ ($2^{4}$ options)
	
	We adds: 
	\begin{align*}
		x_1 &= x_3 \oplus x_5 \oplus x_7\\
		x_2 &= x_3 \oplus x_6 \oplus x_7 \\
		x_4 &= x_5 \oplus x_6 \oplus x_7
	\end{align*}
	And in general, hamming code will acts as $(2^{t} - 1, 2^{t} - t - 1, 3)$
	\she
	
	ניתן להסתכל על $2^{4}$ מילות הקוד ולוודא שאכן מתקיים $d = 3$. לחילופין, אפשר לפלג ל־3 מקרים עבור השתנות של כל מספר תווים, ולהראות שזה ישנה כמות נתונה של ביטי זוגיות. לכן, ניתן לתקן ששששששגיאה בודדת. איך? 
	נניח רצו לשלוח את ההודעה $x_1x_2x_3x_4x_5x_6x_7$. בפועל, קיבלנו את $y_1y_2y_3y_4y_5y_6y_7$. נניח שקיימת לכל היותר שגיאה בודדת. נחשב: 
	\begin{align*}
		y_1' &= y_3 \oplus y_5 \oplus y_7 \\
		y_2' &= y_3 \oplus y_6 \oplus y_7 \\
		y_4' &= y_5 \oplus y_6 \oplus y_7
	\end{align*}
	
	נבדוק באילו מבין $y_1, y_2, y_4$ שונים מ־$y_1', y_2', y_4'$. נחשב את האינדקס: 
	\[ i = 4 \cdot [y_4 \oplus y_4'] + 2 \cdot [y_2 \oplus y_2'] + 1 \cdot [y_1 \oplus y_1'] \] 
	כאשר הכפל במספרים הוא כדי לייצד להתחשב בהיות המספר בינארי, וקיסור ייבדוק האם הם שונים. נתקן את ההודעה: נהפוך את $y_i$ עבור ה־$i$ שחישבנו, ונחזיר את הערכים באינדקסים $3, 5, 6, 7$. 
	
	\textbf{טענה: }עבור $j \neq i$, $x_j = y_j$, ובנוסף, $x_i \neq y_i$ ($i$ משתנה קשור, $j$ חופשי. לא מנוסח בבהירות). 
	
	אינטואיציה שאני כתבתי: אם הטעות בתיקון השגיאות, אז ישתנה איבר יחיד מבין $y_1, y_2, y_3$ ולכן הקיסור כחלק מציאת האינדקס יתחלף במיקום יחיד, ולא נשנה את $y_3, y_5, y_6, y_7$ (תבדקו ידנית). כחלק מהבנייה של $i$, נמצא יותר ביטי זוגיות יתהפכו, בנינו את הקוד בצורה כזו שהאינדקס יתאים (לדוגמא, טעות ב־$y-7$ תגרור שלוש שגיאות, ולכן נגיע לאינדקס האחרון; כן טעות באחרים יגררו שתי שגיאות שיובילו למיקום המדוייק גם כן). 
	
	אינטואיציה שהמורה כתב: כאשר ביו מתהפך, הוא משפיע על $y_1', y_2', y_4'$ בהדיוק לפי היבטים הדוקים בייצוג הבינאי של האינדקס שלו. לכן, אך יש עוד שיבוש יחיד, נתקן בדיוק אותו. 
	\section{\sen Berger Code \she}
	
	נגדיר $k = 2^{\ell} -  1$. נגדיר: 
	\[ \mathrm{Berger}(x) = x \circ \underbrace{\mathrm{bin}\cl{\text{כמות ביטי האפסים ב־$x$}}}_{EC(x)} \]
	
	\textbf{דוגמה: }$\ell = 3, \ x = 110001, EC(x) = 011$ (כי יש $3$ ביטי אפסים ב־$x$)
	
	אורך מילת קוד? $2^{\ell} - 1 + \ell$
	
	מרחק מינימל? $d = 2$. 
	
	לא מאפשר לתקן טעויות. אבל יש לו תכונה מעניינת. נניח, וידוע שרק אפסים יהפכו לאחדים ($0 \mapsto 1$). אינטואיציה: ה־$EC$ יוכל רק לגדול, וכמות האפסים ב־$x$ תוכל רק לקטון, ולכן, תחת ההנחה שהשינוי הוא $0 \mapsto 1$ בלבד, נוכל לזהות כל מספר של טעויות מהסוג הזה. נפרמל. 
	\textbf{טענה: }נוכל לזהות כל מהפס ר של טעויות חד־צצדיות שבהן $0$ הופך ל־$1$ (וזה עובד גם הפוך). 
	
	\begin{proof}
		כל טעות בחלק של $x$ מקטינה את מס' ה־$0$ ב־$x$ שאמורים להיות שווים ב־$EC(x)$. מצד שני, כל טעות כזו ב־$EC(x)$ מגדילה את מס' האפסים שאנחנו מצפים לראות ב־$x$. לכן, לא משנה כמה טעויות כאלה תהינה, נראה אי־התאמה בין $x$ המשודר ו־$EC(x)$ הישודר. 
	\end{proof}
	
	\textit{הבחנה: }אותה ההוכחה תעבוד גם עבור שגיאות בכיוון ההפוך. 
	
	\section{\sen Numeric Calculations \she}
	מיועד לביצוע חישובים בעזרת המחשב, בין עם מתמטיים ובין אם לאו. 
	\subsection{מציאת שורש}
	נתונה לנו פונקציה $f \colon \R\to \R$ (עד לכדי דיוק של \ttt{float}). רוצים למצוא את שורש של הפונ', כלומר ערך $a$ כך ש־$f(a) = 0$, או לפחות $|f(a)| < \vepsi$ עבור $\vepsi$ קטן שהוגדר מראש. בעיות: 
	
	\begin{enumerate}
		\item בעיה ראשונה: האם יש שורש? 
		\item בעיה שנייה: האם הפונקציה רציפה? 
	\end{enumerate}
	
	דוגמה: 
	\sen
	\begin{lstlisting}
lambda x: 0 if x==0.123456 else 1\end{lstlisting}
	\she
	קשה למצוא שורש כי זה די אקראי, לפונקציה כזו.
	
	הנחות מקלות: נניח שהפונקציה רציפה ובהכרח נתונים לנו $u, l$ כך ש־$f(u) \ge 0, f(l) \le 0$. תחת ההנחה הזו, ממשפט ערך הביניים, בין שתי הנקודות הללו יהיה שורש (כלומר קיימת נק' $L < a < u$ כך ש־$f(a) = 0$, בה"כ $L < u$).
	
	\textbf{אלגו': }נביע מעין חיפוש בינאי המרחב "רציף". נחשב בכל צעד את הערך $x = \frac{L + u}{2}$ ו־$f(x)$. נפלג למקרים. 
	\begin{itemize}
		\item אם $|f(x)| < \vepsi$ אז סיימנו ונחזיר את $x$. 
		\item אחרת: 
		\begin{itemize}
			\item אם $f(x) > 0$ נחליף בין $u$ ל־$x$. 
			\item אם $f(x) < 0$ נחליף בין $L$ ל־$x$. 
		\end{itemize}
	\end{itemize}
	
	פונקציות שמקבלות פונקציות אחרות כפרמטר, או מחזירות פונקציות בערך ההחזרה – נקראות high-order functions. בפייתון זה מאוד טבעי. חתימה: 
\sen
	\begin{lstlisting}
def find_root(f: Callable, L: float, u: float, epsi: float) -> float: ...
\end{lstlisting}
\she
	כדי לחשב, לדוגמה, את $\sqrt2$, נוכל להעביר לפונקציה \ttt{f = lambda x: x * x - 2} \ עם הקצוות 1 ו־2 שכן $1 < \sqrt2 < 2$.
	
	\subsection{חישוב $\bm{\pi}$}
	נתבונן במעגל היחידה, וברביוע החוסם אותו. ברביע הראשון, היחס בין שטח הריבוע ($1^2 = 1$) לבין רבע המעגל ($\pi \cdot 1 \cdot 0.25 = \frac{\pi}{4}$) וסה"כ נחלק ונמצא את יחס השטחים.  
	
	\textbf{הרעיון: }נגריל ערכים ברביעו, ונבדוק כמה מהם נמצאים בתוך רבע העיגול – 	בהינתן נק' $(x, y)$ נבדוק האם $x^{2} + y^{2} = 1$.  נדגום $n$ נקודות ונחשב כמה מתוכן נופלות בתוך רבע העיגול. נניח $k$ נפלו ברבע העיגול, נחזיר $\frac{4k}{n}$. ההתפלגות של הערכים הנכונים, תהיה התפלגות בינומית. ניסויים מהסוג הזה נקראים Monte Carlo. 
	
	\subsection{נגזרת ואינטגרל נומריים}
	\subsubsection{נגזרת}
	הגדרת נגזרת: 
	\[ f'(x) = \limh \frac{f(x + h) - f(x)}{h} \]
	נרצה פונקציה בשם \ttt{diff} שמקבלת את $f$ כקלט (ונניח גם את $h$) ומחזירה פונקציה חדשה $f'$. דוגמה: 
	\sen\begin{lstlisting}
def diff(f: Callable, h:float=.001) -> Callable: 
	return lambda x: (f(x + h) - f(x)) / h	\end{lstlisting}\she
	במקום לקחת $h$ קבוע, אפשר לנסות לזהות מתי הנגזרת משתגעת ובתהאם לכך לבחור $h$ יותר קטן. 
	\subsubsection{אינטגרל}
	הגדרה של אינטגרל (מסויים): (ציור של שטח מתחת לפונקציה). סוכמים את שטח הפונ' מעל ציר ה־$x$ פחות השטח שמתחת לציר במיקום במקטע. 
	
	ניתן לחלק את השטח למקטעים קטנים, ולסכום מלבנים ברוחב הקטן לפי ערך הפונקציה. עבור הקטע $a, b$ מחשבים את הסכום: 
	\[ h \cdot f(h) + h \cdot f(a + h) + \dots + h \cdot f(a + ih) + \dots + h \cdot f\cl{a + \lf\frac{b - a}{h}\rf h} \]
	כלומר את: 
	\[ h \cdot \sum_{i = 0}^{\lf \frac{b - a}{h} \rf}f(a + ih) \]
	קוד: 
	\sen\begin{lstlisting}
def integral(f: Callable, h: float) -> Callable: 
	return lambda a, b: h * sum(f(a + i * h) for i in range(1 + int((a - b) // h)))\end{lstlisting}\she
\end{document}