%! ~~~ Packages Setup ~~~ 
\documentclass[]{article}
\usepackage{lipsum}
\usepackage{rotating}


% Math packages
\usepackage[usenames]{color}
\usepackage{forest}
\usepackage{ifxetex,ifluatex,amssymb,amsmath,mathrsfs,amsthm,witharrows,mathtools,mathdots}
\usepackage{amsmath}
\WithArrowsOptions{displaystyle}
\renewcommand{\qedsymbol}{$\blacksquare$} % end proofs with \blacksquare. Overwrites the defualts. 
\usepackage{cancel,bm}
\usepackage[thinc]{esdiff}


% tikz
\usepackage{tikz}
\usetikzlibrary{graphs}
\newcommand\sqw{1}
\newcommand\squ[4][1]{\fill[#4] (#2*\sqw,#3*\sqw) rectangle +(#1*\sqw,#1*\sqw);}


% code 
\usepackage{algorithm2e}
\usepackage{listings}
\usepackage{xcolor}

\definecolor{codegreen}{rgb}{0,0.35,0}
\definecolor{codegray}{rgb}{0.5,0.5,0.5}
\definecolor{codenumber}{rgb}{0.1,0.3,0.5}
\definecolor{codeblue}{rgb}{0,0,0.5}
\definecolor{codered}{rgb}{0.5,0.03,0.02}
\definecolor{codegray}{rgb}{0.96,0.96,0.96}

\lstdefinestyle{pythonstylesheet}{
    language=Java,
    emphstyle=\color{deepred},
    backgroundcolor=\color{codegray},
    keywordstyle=\color{deepblue}\bfseries\itshape,
    numberstyle=\scriptsize\color{codenumber},
    basicstyle=\ttfamily\footnotesize,
    commentstyle=\color{codegreen}\itshape,
    breakatwhitespace=false, 
    breaklines=true, 
    captionpos=b, 
    keepspaces=true, 
    numbers=left, 
    numbersep=5pt, 
    showspaces=false,                
    showstringspaces=false,
    showtabs=false, 
    tabsize=4, 
    morekeywords={as,assert,nonlocal,with,yield,self,True,False,None,AssertionError,ValueError,in,else},              % Add keywords here
    keywordstyle=\color{codeblue},
    emph={var, List, Iterable, Iterator},          % Custom highlighting
    emphstyle=\color{codered},
    stringstyle=\color{codegreen},
    showstringspaces=false,
    abovecaptionskip=0pt,belowcaptionskip =0pt,
    framextopmargin=-\topsep, 
}
\newcommand\pythonstyle{\lstset{pythonstylesheet}}
\newcommand\pyl[1]     {{\lstinline!#1!}}
\lstset{style=pythonstylesheet}

\usepackage[style=1,skipbelow=\topskip,skipabove=\topskip,framemethod=TikZ]{mdframed}
\definecolor{bggray}{rgb}{0.85, 0.85, 0.85}
\mdfsetup{leftmargin=0pt,rightmargin=0pt,innerleftmargin=15pt,backgroundcolor=codegray,middlelinewidth=0.5pt,skipabove=5pt,skipbelow=0pt,middlelinecolor=black,roundcorner=5}
\BeforeBeginEnvironment{lstlisting}{\begin{mdframed}\vspace{-0.4em}}
    \AfterEndEnvironment{lstlisting}{\vspace{-0.8em}\end{mdframed}}


% Design
\usepackage[labelfont=bf]{caption}
\usepackage[margin=0.6in]{geometry}
\usepackage{multicol}
\usepackage[skip=4pt, indent=0pt]{parskip}
\usepackage[normalem]{ulem}
\forestset{default}
\renewcommand\labelitemi{$\bullet$}
\usepackage{titlesec}
\titleformat{\section}[block]
{\fontsize{15}{15}}
{\sen \dotfill (\thesection)\dotfill\she}
{0em}
{\MakeUppercase}
\usepackage{graphicx}
\graphicspath{ {./} }

\usepackage[colorlinks]{hyperref}
\definecolor{mgreen}{RGB}{25, 160, 50}
\definecolor{mblue}{RGB}{30, 60, 200}
\usepackage{hyperref}
\hypersetup{
    colorlinks=true,
    citecolor=mgreen,
    linkcolor=black,
    urlcolor=mblue,
    pdftitle={Document by Shahar Perets},
    %	pdfpagemode=FullScreen,
}
\usepackage{yfonts}
\def\gothstart#1{\noindent\smash{\lower3ex\hbox{\llap{\Huge\gothfamily#1}}}
    \parshape=3 3.1em \dimexpr\hsize-3.4em 3.4em \dimexpr\hsize-3.4em 0pt \hsize}
\def\frakstart#1{\noindent\smash{\lower3ex\hbox{\llap{\Huge\frakfamily#1}}}
    \parshape=3 1.5em \dimexpr\hsize-1.5em 2em \dimexpr\hsize-2em 0pt \hsize}



% Hebrew initialzing
\usepackage[bidi=basic]{babel}
\PassOptionsToPackage{no-math}{fontspec}
\babelprovide[main, import, Alph=letters]{hebrew}
\babelprovide[import]{english}
\babelfont[hebrew]{rm}{David CLM}
\babelfont[hebrew]{sf}{David CLM}
%\babelfont[english]{tt}{Monaspace Xenon}
\usepackage[shortlabels]{enumitem}
\newlist{hebenum}{enumerate}{1}

% Language Shortcuts
\newcommand\en[1] {\begin{otherlanguage}{english}#1\end{otherlanguage}}
\newcommand\he[1] {\she#1\sen}
\newcommand\sen   {\begin{otherlanguage}{english}}
    \newcommand\she   {\end{otherlanguage}}
\newcommand\del   {$ \!\! $}

\newcommand\npage {\vfil {\hfil \textbf{\textit{המשך בעמוד הבא}}} \hfil \vfil \pagebreak}
\newcommand\ndoc  {\dotfill \\ \vfil {\begin{center}
            {\textbf{\textit{שחר פרץ, 2025}} \\
                \scriptsize \textit{קומפל ב־}\en{\LaTeX}\,\textit{ ונוצר באמצעות תוכנה חופשית בלבד}}
    \end{center}} \vfil	}

\newcommand{\rn}[1]{
    \textup{\uppercase\expandafter{\romannumeral#1}}
}

\makeatletter
\newcommand{\skipitems}[1]{
    \addtocounter{\@enumctr}{#1}
}
\makeatother

%! ~~~ Math shortcuts ~~~

% Letters shortcuts
\newcommand\N     {\mathbb{N}}
\newcommand\Z     {\mathbb{Z}}
\newcommand\R     {\mathbb{R}}
\newcommand\Q     {\mathbb{Q}}
\newcommand\C     {\mathbb{C}}
\newcommand\E     {\mathbb{E}}
\newcommand\One   {\mathbb{I}}

\newcommand\ml    {\ell}
\newcommand\mj    {\jmath}
\newcommand\mi    {\imath}

\newcommand\powerset {\mathcal{P}}
\newcommand\ps    {\mathcal{P}}
\newcommand\pc    {\mathcal{P}}
\newcommand\ac    {\mathcal{A}}
\newcommand\bc    {\mathcal{B}}
\newcommand\cc    {\mathcal{C}}
\newcommand\dc    {\mathcal{D}}
\newcommand\ec    {\mathcal{E}}
\newcommand\fc    {\mathcal{F}}
\newcommand\nc    {\mathcal{N}}
\newcommand\vc    {\mathcal{V}} % Vance
\newcommand\sca   {\mathcal{S}} % \sc is already definded
\newcommand\rca   {\mathcal{R}} % \rc is already definded

\newcommand\prm   {\mathrm{p}}
\newcommand\arm   {\mathrm{a}} % x86
\newcommand\brm   {\mathrm{b}}
\newcommand\crm   {\mathrm{c}}
\newcommand\drm   {\mathrm{d}}
\newcommand\erm   {\mathrm{e}}
\newcommand\frm   {\mathrm{f}}
\newcommand\nrm   {\mathrm{n}}
\newcommand\vrm   {\mathrm{v}}
\newcommand\srm   {\mathrm{s}}
\newcommand\rrm   {\mathrm{r}}

\newcommand\Si    {\Sigma}

% Logic & sets shorcuts
\newcommand\siff  {\longleftrightarrow}
\newcommand\ssiff {\leftrightarrow}
\newcommand\so    {\longrightarrow}
\newcommand\sso   {\rightarrow}

\newcommand\epsi  {\epsilon}
\newcommand\vepsi {\varepsilon}
\newcommand\vphi  {\varphi}
\newcommand\Neven {\N_{\mathrm{even}}}
\newcommand\Nodd  {\N_{\mathrm{odd }}}
\newcommand\Zeven {\Z_{\mathrm{even}}}
\newcommand\Zodd  {\Z_{\mathrm{odd }}}
\newcommand\Np    {\N_+}

% Text Shortcuts
\newcommand\open  {\big(}
\newcommand\qopen {\quad\big(}
\newcommand\close {\big)}
\newcommand\also  {\mathrm{, }}
\newcommand\defis {\mathrm{ definitions}}
\newcommand\given {\mathrm{given }}
\newcommand\case  {\mathrm{if }}
\newcommand\syx   {\mathrm{ syntax}}
\newcommand\rle   {\mathrm{ rule}}
\newcommand\other {\mathrm{else}}
\newcommand\set   {\ell et \text{ }}
\newcommand\ans   {\mathscr{A}\!\mathit{nswer}}

% Set theory shortcuts
\newcommand\ra    {\rangle}
\newcommand\la    {\langle}

\newcommand\oto   {\leftarrow}

\newcommand\QED   {\quad\quad\mathscr{Q.E.D.}\;\;\blacksquare}
\newcommand\QEF   {\quad\quad\mathscr{Q.E.F.}}
\newcommand\eQED  {\mathscr{Q.E.D.}\;\;\blacksquare}
\newcommand\eQEF  {\mathscr{Q.E.F.}}
\newcommand\jQED  {\mathscr{Q.E.D.}}

\DeclareMathOperator\dom   {dom}
\DeclareMathOperator\Img   {Im}
\DeclareMathOperator\range {range}

\newcommand\trio  {\triangle}

\newcommand\rc    {\right\rceil}
\newcommand\lc    {\left\lceil}
\newcommand\rf    {\right\rfloor}
\newcommand\lf    {\left\lfloor}
\newcommand\ceil  [1] {\lc #1 \rc}
\newcommand\floor [1] {\lf #1 \rf}

\newcommand\lex   {<_{lex}}

\newcommand\az    {\aleph_0}
\newcommand\uaz   {^{\aleph_0}}
\newcommand\al    {\aleph}
\newcommand\ual   {^\aleph}
\newcommand\taz   {2^{\aleph_0}}
\newcommand\utaz  { ^{\left (2^{\aleph_0} \right )}}
\newcommand\tal   {2^{\aleph}}
\newcommand\utal  { ^{\left (2^{\aleph} \right )}}
\newcommand\ttaz  {2^{\left (2^{\aleph_0}\right )}}

\newcommand\n     {$n$־יה\ }

% Math A&B shortcuts
\newcommand\logn  {\log n}
\newcommand\logx  {\log x}
\newcommand\lnx   {\ln x}
\newcommand\cosx  {\cos x}
\newcommand\sinx  {\sin x}
\newcommand\sint  {\sin \theta}
\newcommand\tanx  {\tan x}
\newcommand\tant  {\tan \theta}
\newcommand\sex   {\sec x}
\newcommand\sect  {\sec^2}
\newcommand\cotx  {\cot x}
\newcommand\cscx  {\csc x}
\newcommand\sinhx {\sinh x}
\newcommand\coshx {\cosh x}
\newcommand\tanhx {\tanh x}

\newcommand\seq   {\overset{!}{=}}
\newcommand\slh   {\overset{LH}{=}}
\newcommand\sle   {\overset{!}{\le}}
\newcommand\sge   {\overset{!}{\ge}}
\newcommand\sll   {\overset{!}{<}}
\newcommand\sgg   {\overset{!}{>}}

\newcommand\h     {\hat}
\newcommand\ve    {\vec}
\newcommand\lv    {\overrightarrow}
\newcommand\ol    {\overline}

\newcommand\mlcm  {\mathrm{lcm}}

\DeclareMathOperator{\sech}   {sech}
\DeclareMathOperator{\csch}   {csch}
\DeclareMathOperator{\arcsec} {arcsec}
\DeclareMathOperator{\arccot} {arcCot}
\DeclareMathOperator{\arccsc} {arcCsc}
\DeclareMathOperator{\arccosh}{arccosh}
\DeclareMathOperator{\arcsinh}{arcsinh}
\DeclareMathOperator{\arctanh}{arctanh}
\DeclareMathOperator{\arcsech}{arcsech}
\DeclareMathOperator{\arccsch}{arccsch}
\DeclareMathOperator{\arccoth}{arccoth}
\DeclareMathOperator{\atant}  {atan2} 
\DeclareMathOperator{\Sp}     {span} 
\DeclareMathOperator{\sgn}    {sgn} 
\DeclareMathOperator{\row}    {Row} 
\DeclareMathOperator{\adj}    {adj} 
\DeclareMathOperator{\rk}     {rank} 
\DeclareMathOperator{\col}    {Col} 
\DeclareMathOperator{\tr}     {tr}

\newcommand\dx    {\,\mathrm{d}x}
\newcommand\dt    {\,\mathrm{d}t}
\newcommand\dtt   {\,\mathrm{d}\theta}
\newcommand\du    {\,\mathrm{d}u}
\newcommand\dv    {\,\mathrm{d}v}
\newcommand\df    {\mathrm{d}f}
\newcommand\dfdx  {\diff{f}{x}}
\newcommand\dit   {\limhz \frac{f(x + h) - f(x)}{h}}

\newcommand\nt[1] {\frac{#1}{#1}}

\newcommand\limz  {\lim_{x \to 0}}
\newcommand\limxz {\lim_{x \to x_0}}
\newcommand\limi  {\lim_{x \to \infty}}
\newcommand\limh  {\lim_{x \to 0}}
\newcommand\limni {\lim_{x \to - \infty}}
\newcommand\limpmi{\lim_{x \to \pm \infty}}

\newcommand\ta    {\theta}
\newcommand\ap    {\alpha}

\renewcommand\inf {\infty}
\newcommand  \ninf{-\inf}

% Combinatorics shortcuts
\newcommand\sumnk     {\sum_{k = 0}^{n}}
\newcommand\sumni     {\sum_{i = 0}^{n}}
\newcommand\sumnko    {\sum_{k = 1}^{n}}
\newcommand\sumnio    {\sum_{i = 1}^{n}}
\newcommand\sumai     {\sum_{i = 1}^{n} A_i}
\newcommand\nsum[2]   {\reflectbox{\displaystyle\sum_{\reflectbox{\scriptsize$#1$}}^{\reflectbox{\scriptsize$#2$}}}}

\newcommand\bink      {\binom{n}{k}}
\newcommand\setn      {\{a_i\}^{2n}_{i = 1}}
\newcommand\setc[1]   {\{a_i\}^{#1}_{i = 1}}

\newcommand\cupain    {\bigcup_{i = 1}^{n} A_i}
\newcommand\cupai[1]  {\bigcup_{i = 1}^{#1} A_i}
\newcommand\cupiiai   {\bigcup_{i \in I} A_i}
\newcommand\capain    {\bigcap_{i = 1}^{n} A_i}
\newcommand\capai[1]  {\bigcap_{i = 1}^{#1} A_i}
\newcommand\capiiai   {\bigcap_{i \in I} A_i}

\newcommand\xot       {x_{1, 2}}
\newcommand\ano       {a_{n - 1}}
\newcommand\ant       {a_{n - 2}}

% Linear Algebra
\DeclareMathOperator{\chr}     {char}
\DeclareMathOperator{\diag}    {diag}
\DeclareMathOperator{\Hom}     {Hom}
\DeclareMathOperator{\Sym}     {Sym}
\DeclareMathOperator{\Asym}    {ASym}

\newcommand\lra       {\leftrightarrow}
\newcommand\chrf      {\chr(\F)}
\newcommand\F         {\mathbb{F}}
\newcommand\co        {\colon}
\newcommand\tmat[2]   {\cl{\begin{matrix}
            #1
        \end{matrix}\, \middle\vert\, \begin{matrix}
            #2
\end{matrix}}}

\makeatletter
\newcommand\rrr[1]    {\xxrightarrow{1}{#1}}
\newcommand\rrt[2]    {\xxrightarrow{1}[#2]{#1}}
\newcommand\mat[2]    {M_{#1\times#2}}
\newcommand\gmat      {\mat{m}{n}(\F)}
\newcommand\tomat     {\, \dequad \longrightarrow}
\newcommand\pms[1]    {\begin{pmatrix}
        #1
\end{pmatrix}}

% someone's code from the internet: https://tex.stackexchange.com/questions/27545/custom-length-arrows-text-over-and-under
\makeatletter
\newlength\min@xx
\newcommand*\xxrightarrow[1]{\begingroup
    \settowidth\min@xx{$\m@th\scriptstyle#1$}
    \@xxrightarrow}
\newcommand*\@xxrightarrow[2][]{
    \sbox8{$\m@th\scriptstyle#1$}  % subscript
    \ifdim\wd8>\min@xx \min@xx=\wd8 \fi
    \sbox8{$\m@th\scriptstyle#2$} % superscript
    \ifdim\wd8>\min@xx \min@xx=\wd8 \fi
    \xrightarrow[{\mathmakebox[\min@xx]{\scriptstyle#1}}]
    {\mathmakebox[\min@xx]{\scriptstyle#2}}
    \endgroup}
\makeatother


% Greek Letters
\newcommand\ag        {\alpha}
\newcommand\bg        {\beta}
\newcommand\cg        {\gamma}
\newcommand\dg        {\delta}
\newcommand\eg        {\epsi}
\newcommand\zg        {\zeta}
\newcommand\hg        {\eta}
\newcommand\tg        {\theta}
\newcommand\ig        {\iota}
\newcommand\kg        {\keppa}
\renewcommand\lg      {\lambda}
\newcommand\og        {\omicron}
\newcommand\rg        {\rho}
\newcommand\sg        {\sigma}
\newcommand\yg        {\usilon}
\newcommand\wg        {\omega}

\newcommand\Ag        {\Alpha}
\newcommand\Bg        {\Beta}
\newcommand\Cg        {\Gamma}
\newcommand\Dg        {\Delta}
\newcommand\Eg        {\Epsi}
\newcommand\Zg        {\Zeta}
\newcommand\Hg        {\Eta}
\newcommand\Tg        {\Theta}
\newcommand\Ig        {\Iota}
\newcommand\Kg        {\Keppa}
\newcommand\Lg        {\Lambda}
\newcommand\Og        {\Omicron}
\newcommand\Rg        {\Rho}
\newcommand\Sg        {\Sigma}
\newcommand\Yg        {\Usilon}
\newcommand\Wg        {\Omega}

% Other shortcuts
\newcommand\tl    {\tilde}
\newcommand\op    {^{-1}}

\newcommand\sof[1]    {\left | #1 \right |}
\newcommand\cl [1]    {\left ( #1 \right )}
\newcommand\csb[1]    {\left [ #1 \right ]}
\newcommand\ccb[1]    {\left \{ #1 \right \}}

\newcommand\bs        {\blacksquare}
\newcommand\dequad    {\!\!\!\!\!\!}
\newcommand\dequadd   {\dequad\duquad}

\renewcommand\phi     {\varphi}

\newtheorem{Theorem}{משפט}
\theoremstyle{definition}
\newtheorem{definition}{הגדרה}
\newtheorem{Lemma}{למה}
\newtheorem{Remark}{הערה}
\newtheorem{Notion}{סימון}


\newcommand\theo  [1] {\begin{Theorem}#1\end{Theorem}}
\newcommand\defi  [1] {\begin{definition}#1\end{definition}}
\newcommand\rmark [1] {\begin{Remark}#1\end{Remark}}
\newcommand\lem   [1] {\begin{Lemma}#1\end{Lemma}}
\newcommand\noti  [1] {\begin{Notion}#1\end{Notion}}

% DS
\newcommand\limsi     {\limsup_{n \to \inf}}
\newcommand\limfi     {\liminf_{n \to \inf}}

\DeclareMathOperator\amort   {amort}
\DeclareMathOperator\worst   {worst}
\DeclareMathOperator\type    {type}
\DeclareMathOperator\cost    {cost}
\DeclareMathOperator\tim     {time}

\newcommand\dsList{
    \sFunc{List}
    \sFunc{Retrieve}
    \SetKwFunction{RetrieveFirst}{Retrieve-First}
    \SetKwFunction{RetrieveLast}{Retrieve-Last}
    \sFunc{Delete}
    \SetKwFunction{DeleteFirst}{Delete-First}
    \SetKwFunction{DeleteLast}{Delete-Last}
    \sFunc{Insert}
    \SetKwFunction{InsertFirst}{Insert-First}
    \SetKwFunction{InsertLast}{Insert-Last}
    \sFunc{Shift}
    \sFunc{Length}
    \sFunc{Concat}
    \sFunc{Plant}
    \sFunc{Split}
}
\newcommand\dsQueue{
    \sFunc{Queue}
    \sFunc{Enqueue}
    \sFunc{Head}
    \sFunc{Dequeue}
}
\newcommand\dsStack{
    \sFunc{Stack}
    \sFunc{Push}
    \sFunc{Top}
    \sFunc{Pop}
}
\newcommand\dsVector{
    \sFunc{Vector}
    \sFunc{Get}
    \sFunc{Set}
}
\newcommand\dsGraph{
    \sFunc{Graph}
    \sFunc{Edge}
    \SetKwFunction{AddEdge}{Add-Edge}
    \SetKwFunction{RemoveEdge}{Remove-Edge}
    \sFunc{InDeg} \sFunc{OutDeg}
}
\newcommand\importDs{
    \dsList
    \dsQueue
    \dsStack
    \dsVector
    \dsGraph
    \SetKwProg{Fn}{function}{ is}{end}
    \SetKwData{error}{\color{codered}error}
    \SetKwInOut{Input}{input}
    \SetKwInOut{Output}{output}
    \SetKwRepeat{Do}{do}{while}
    \SetKwData{Null}{\color{codegreen}null}
    \SetKwData{True}{\color{codeblue}true}
    \SetKwData{False}{\color{codeblue}false}
}


% Algorithems
\newcommand\sFunc [1] {\SetKwFunction{#1}{#1}}
\newcommand\sData [1] {\SetKwData{#1}{#1}}
\newcommand\sIO   [1] {\SetKwInOut{#1}{#1}}
\newcommand\ttt   [1] {\sen \texttt{#1} \she\,}
\newcommand\io    [2] {\Input{#1}\Output{#2}\BlankLine}

%! ~~~ Document ~~~

\author{שחר פרץ}
\title{\textit{מבני נתונים 15 $\sim$ מבוא להסתברות}}
\begin{document}
    \maketitle
    
    \textbf{מרצה: }עמית ווינשטיין
  
  \subsubsection*{מבוא להסתברות}
  
  אנחנו מתעניינים בהסתברות כאשר הלאוריתמים/מבני הנתונים אינם דטרמינסטיים. באלגו' הסתברותי, נתעיין בזמן הריצה 
    
    כבר היום נשתמש בזה בשביל לנתח באופן יותר מדויק quicksort. אלגוריתם לא דטרמיניסטי – ``מטיל מטבעות'', ואז מה שרוצים לשאול זה עבור כל פלט, מה אנחנו ``מצפים'' (ולא רק מה ה־worst case). ה''צפוי'' עבור הקלט, על פני הטלת המטבעות של האלג/מבנה. 
    
    נגדיר הגדרות של מבוא להסתברות: 
    
    \defi{\textit{ניסוי (Experiment):} תהליך שבו יש חוסר וודאות לגבי התוצאה}
    
    \defi{\textit{מרחב המדגם (Sample Space):} קבוצת התוצאות האפשרויות בניסוי. מסומן ב־$S$ או ב־$\Sg$. }
    
    \textbf{דוגמה. }עבור קובייה, $[6]$ מהווה את מרחב המדגם. עבור מטבע, $\{\text{עץ}, \text{פלי}\}$. שתי הדוגמאות האלו דיסקרטיות, ויש אוסף סופי של אפשרויות. לצורך העניין, הממשיים $\R$ עבור קבוצת כל הזמנים עד שהמנורה תכבה. לצורכינו, נדבר על מרחבי מדגם סופיים. 
    
    \textit{הערה. }זה קצת מוזר להגדיר דיסקרטי, כי $\N$ לרוב נחשב דיסקרטי, אך $\Q$ לא (אז עוצמות לא הגדרה טובה). זה לא משנה לצורכינו כי נעבוד רק עם מרחבי מדגם סופיים. 
    
    \defi{\textit{אירוע (Event): }ת''ק של מרחב המדגם. }
    \textbf{דוגמאות}
    \begin{multicols}{2}
        \begin{enumerate}
            \item יצא ערך זוגי
            \item הטלנו 2 קוביות ייצא אותו הערך
            \item הטעלנו 2 קוביות והסכום $\le$ 4
            \item הטעלנו קוביה ויצא 4. 
        \end{enumerate}
    \end{multicols}
    \defi{\textit{אירוע פשוט: }ת''ק בגודל $1$ ממרחב המדרגם}
    בתור קבוצות, נרצה לחתוך, לאחד, לבדוק זרות/הכלה, וכן למצא משלים (כאשר עולם הדיון הוא תמיד מרחב המדגם). לצורך הדוגמה, ביחס לדוגמאות לעיל: 
    \[ B \cap C = \{(1, 1), (2, 2)\}, \ A^{C} = \{1, 3, 5\} \quad \text{\en{etc.}} \]
    
    \defi{\textit{פונקציית ההסתברות $P$: }היא פונקציה $P \co \ps(S) \to \R$, כאשר הממשיים ההסתברות שלהם. היא מקיימת את התכונות הבאות: \\
    \begin{enumerate}
        \item \hfil $\forall E \subseteq S \co 0 \le P(E) \le 1$
        \item \hfil $P(S) = 1$
        \item \hfil $\forall E, F \subseteq S \co E \cap F = \varnothing \implies P(E \cap F) = P(E) + P(F)$
    \end{enumerate}}
        אינטואיטית, אנחנו נותנים הסתברות לכל אחת מהמאורעות שיכולים לראות. 
    \defi{\textit{הסתברות מותנה (Conditional Probability): }תסומן $P(E \mid F)$ ותוגדר להיות: 
    \[ P(E \mid F) = \frac{P(E \cap F)}{P(F)} \]
    ומוגדרת כאשר $P(F) > 0$. }
    \textit{הערה. }נוכל לשחק עם המשוואה לעיל, נקבל: $P(E \cap F) = P(F) \cdot P(E \mid F)$
    
    הגיונית, $P(E \cap F)$ מתאר את ההתסבורת של $E$ בהינתן שקרה $F$. 
        
    \textbf{דוגמה. }כד עם 8 כדורים אדומים ו־$4$ כחולים. מוציאים $22$ כדורים ללא חזרות. נגידר: $=E$הכדור הראשון יצא אדום, $=F$ הכדור השני היה אדום. אז $P(E \cap F) = P(E) \cdot P(F \mid E) = \frac{8}{12} \cdot \frac{7}{11}$. 
        
        נוכל להכליל באינדוקציה: ,
    \[ P\cl{\bigcap_{i = 1}^{n}E_i} = P(E_1) + \prod_{i = 2}^{n}P\cl{E_i \mid \bigcap^{i - 1}_{i = 1}E_i} = P(E_1) \cdot P(E_2 \mid E_1) \cdots P(E_n \mid E_1 \cap E_2 \cdots \cap E_{n - 1}) \]
        עבור חלוקה $S = F_1 \cup F_2 \cup F_3 \cup \cdots \cup F_n$ זרים בזוגות, היא חלוקה של מרחב המדגם, מתקיים בעבור אירוע $E$: 
    \[ P(E) = \sum_{i = 1}^{n}P(F_i) \cdot P(E \mid F_i) = \sum_{i = 1}^{n}P(E \cap F_i) \]
        מסקנה: 
    \[ P(E) = P(E \mid F) \cdot P(F) + P(E \mid F^{C}) \cdot P(F^{C}) = P(E \cap F) + P(E \cap F^{C}) \]
        
    \defi{\textit{מאורעות בלתי תלויים (Independent Events): }יהיו $E, F$ מאורעות, הם יקראו בלתי תלויים אם $P(E \cap F) = P(E) \cdot P(F)$. בניסוח שקול $P(E \mid F) = P(E)$. }
    
    כאשר אירועים זרים, הם ממש ממש תלויים. הסיבה: כי אם אחד קורה, השני בהכרח לא קורה. זה נותן עליו ידע. חוץ ממקרים מנוונים כמו $P(E) = 0$. 
    
    \defi{\textit{משתנה מקרי (Random Variable): }פונקציה $S \to \R$. אוהבים להשתמש באותיות $x, y$ עבור משתנים מקריים. בקיצור מ''מ. }
    
    \noti{עבור מ''מ $X$, נסמן ב־$X = x$ את ה\textit{אירוע} שבו $X$ קיבל את הערך $x$. ואז נובל לכתוב $P(X = x)$. }
    \textit{הערה: }$X = x$לאו דווקא מאורע פשוט. \\
    \textbf{דוגמה. }אם $X$ הוא סכום הטלה של $2$ קוביות, אז זהו משתנה מקרי. הוא הופך את מרחב המדגם לערך קונקרטי שאפשר להשתמש בו. 
    
    \defi{\textit{תוחלת (Expectation): }הממוצע המשוקלל של משתנה מקרי, כאשר המשקלים הם ההסתברות. כלומר: 
    \[ \E[x] = \sum_{\mathclap{x \in \dom X}} x \cdot P(X = x) \]}
    הרעיון: להגיד מה הערך ה''ממוצע'' של מ''מ. עבור $=X$ערך הטלת קובייה, מתקיים $\E[x] = 3.5$. 
    
    \theo{שתי תכונות חשובות: 
    \begin{enumerate}
        \item אפשר לחשוב על $Y$ לעיל כמו מ''מ חדש. 
        \[ \E[\underbrace{a \cdot x + b}_{Y}] = a \cdot \E[X] + b \]
        \item ``יותר מעניינת'', נכון עבור כל $X, Y$ זוג מ''מ. (שימו לב, $X + Y$ חיבור פונקציות)
        \[ \E[X + Y] = \E[X] + \E[Y] \]
    \end{enumerate}}
    התכונה השנייה מאפשר לעשות משהו מאוד נחמד – אם המשתנה המקרי שלנו מסובך במיוחד, לפרק מ''ממים למ''ממים קטנים יותר שהוא שווה לסכומם. בהמשך, לדוגמה, ניקח את המ''מ הוא כמות ההשוואות ב־quicksort, ונפרק אותו לסכום של מ''מים יותר קטנים. 
    
    \defi{\textit{מ''מ קינדיקטור: }מקבל ערך $1$ אם קרא אירוע מסויים, ו־$0$ אחרת. יסומן ב־$\mathbb{I}_E$. }
    אזי: 
    \[ \E[\One_E] = 1 \cdot P(E) + 0 \cdot P(E^{C}) = P(E) \]
    ``ועוד אפס כפול ההסתברות של אמאשלי'' – עמית על קבוצת המשלים. 
    
    \defi{\textit{התפלגות אחידה: }מרחב מדגם סופי בו כל התוצאות האפשריות בו, (בהסתברות זהה, כלומר התפלגות קבועה. }
    \textbf{דוגמה. }$X$ מתפלג אחיד בין $a, a + 1 \dots b$ אז $\forall a \le x \le b \co P(X = x) = \frac{1}{b - a + 1}, \ \E[x] = \frac{a + b}{2}$
    \defi{\textit{התפלגות כיאומטרית: }חוזרים על אירוע שמצליחים בהסתברות $P$ כמה פמעים עד ההצלחה הראשונה, והשמתנה המקרי הוא כמות הניסויים. }בעבור $X$ התפלגות גיאומטרית: 
    \[ \forall k > 0 \co P(X = k) = (1 - p)^{k + 1} \cdot p, \ P(X \ge k) = (1 - p^{k + 1}) \]
    כאשר $(1 - p)$ מתאר כשלונות, ו־$p$ את ההסתברות להצלחת $E$. אז התוחלת: 
    \[ \E[X] = \sum_{\inf}^{j = 1}k \cdot P(X = k) = \sum_{i = 1}^{\inf}P(X \ge x) = \sum_{k = 1}^{\inf}(1 - p)^{k - 1} = \sum_{k = 0}^{]inf}(1 - p)^{x} = \frac{1}{p} \]
    
    
    \subsubsection*{QuickSort}
    נסמן ב־$X$ להיות מ''מ הוא מספר ההשוואות שהאלגו' מבצע על הקלט $x_1x_2 \dots x_n$ כאשר $x_i \neq x_j$. אחרי המיון, המערך יראה: 
    \[ z_1 < z_2 < z_3 < \cdots < z_i < \cdots < z_n \]
    יהי לנו יותר קל לנתח כשנגע מה נמצא איפה. זכרו שלא משנה מה הקלט עצמו כי הבחירות של האלגו' הם הדבר החשוב. נשים לב ש־$z_i, z_j$ עלולים להיות משווים לכל היותר פעם אחת – כאשר הראשון מבינהם נבחר להיות ה־pivot. בזכות התובנה הזו, נוכל להגדיר $X = \sum X_{i, j}$ כאשר $X_{i, j}$ מתאר האם אנחנו משווי םאת $z_i$ ו־$z_j$. נוכל לדחוף $\E$ על שני אגפי המשוואה ומאדטיביות רק לחפש את $\E[X_{i, j}]$: 
    \begin{itemize}
        \item עור $j = i + 1$, מתקיים $P(i, j) = 1$. 
        \item עבור $i \ll j$ מתקיים $P(i, j)$ אינטואיטיבית קטן
        \item עבור $i = 1, j = n$ נקבל $P(i, j) = \frac{2}{n}$
        \item נראה ש־$P(i, j) = \frac{2}{j - i + 1}$ נראה הגיוני כי עובד במקרים לעיל. נוכיח את הנוסחה הזו. 
    \end{itemize}
    \begin{proof}
        נוכיח את הנוסחה באינדוקציה על אורך המערך הנוכחי. 
        
        בסיס: $n = 2$, אז $i = 1, j = 2$ בה''כ ואכן מתקיים: 
        \[ P(1, 2) = \frac{2}{2 - 1 + 1} = \frac{2}{2} = 1 \]
        צעד: נניח נוכנות עד $n - 1$, ונוכיח עבור $n$. נניח שה־pivot שנבחר הוא $z_k$. נחלק למקרים: 
        \begin{itemize}
            \item אם $i < j < k$, אז $z_i, z_j$ יהיו במערך השמאלי עם אותם האינדקסים ומה.א. נקבל $P(i, j) = \frac{2}{j - i + 1}$
            \item אם $k < i< j$ אז $z_i, z_j$ יהיו במערך השמאלי עם אינדקסים $i - k, j - k$ ולכן: 
            \[ P(i, j) = \frac{2}{(j -k) - (i - k) + 1} = \frac{2}{j - i + 1} \]
            \item אם $i \le k \le j$, נשווה רק כאשר $k = i$, או $k = j$, ואחרת לא נשווה אותם בעתיד. לכן $P(i, j) = \frac{2}{j - i + 1}$. 
        \end{itemize}
    \end{proof}
    
    \theo{$\E[x] = O(n \logn)$}\begin{proof}
        \[ \E[X] = \sum_{i < j} P(i, j) = \sum_{i = 1}^{n - 1}\sum_{j = i + 1}^{n}\frac{2}{j - i + 1} \overset{k = j - i + 1}{=} \sum_{i = 1}^{n - 1}\sum_{k = 2}^{n - i + 1} \frac{2}{k} \le 2 \cdot \sum_{i = 1}^{n - 1}\underbrace{\sum_{k = 1}^{n}\frac{1}{k}}_{H_n} \]
        ידוע שהטור ההרמוני $H_n$ מקיים $H_n \le \ln(n + 1)$. לכן: 
        \[ \le 2 \cdot n \cdot (\ln n + 1) = O(n \logn) \]
    \end{proof}
    
    
    
    

    
    
    
    \ndoc
\end{document}