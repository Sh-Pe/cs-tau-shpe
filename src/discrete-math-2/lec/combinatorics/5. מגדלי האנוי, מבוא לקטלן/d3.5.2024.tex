\documentclass[]{article}

% Math packages
\usepackage[usenames]{color}
\usepackage{forest}
\usepackage{ifxetex,ifluatex,amsmath,amssymb,mathrsfs,amsthm,witharrows,mathtools}
\WithArrowsOptions{displaystyle}
\renewcommand{\qedsymbol}{$\blacksquare$} % end proofs with \blacksquare. Overwrites the defualts. 
\usepackage{cancel,bm}

% tikz
\usepackage{tikz,pgfplots}
\newcommand\sqw{1}
\newcommand\squ[4][1]{\fill[#4] (#2*\sqw,#3*\sqw) rectangle +(#1*\sqw,#1*\sqw);}


% code 
\usepackage{listings}
\usepackage{xcolor}

\definecolor{codegreen}{rgb}{0,0.35,0}
\definecolor{codegray}{rgb}{0.5,0.5,0.5}
\definecolor{codenumber}{rgb}{0.1,0.3,0.5}
\definecolor{deepblue}{rgb}{0,0,0.5}
\definecolor{deepred}{rgb}{0.5,0.03,0.02}

\lstdefinestyle{pythonstylesheet}{
	language=Python,
	morekeywords={}
	emphstyle=\color{deepred},
	backgroundcolor=\color{white},   
	commentstyle=\color{codegreen}\itshape,
	keywordstyle=\color{deepblue}\bfseries\itshape,
	numberstyle=\tiny\color{codenumber},
	basicstyle=\ttfamily\footnotesize,
	breakatwhitespace=false, 
	breaklines=true, 
	captionpos=b, 
	keepspaces=true, 
	numbers=left, 
	numbersep=5pt, 
	showspaces=false,                
	showstringspaces=false,
	showtabs=false, 
	tabsize=2, 
	morekeywords={object,type,isinstance,copy,deepcopy,zip,enumerate,reversed,list,set,len,dict,tuple,range,xrange,append,execfile,real,imag,reduce,str,repr},              % Add keywords here
	keywordstyle=\color{deepblue},
	emph={__init__,__add__,__mul__,__div__,__sub__,__call__,__getitem__,__setitem__,__eq__,__ne__,__nonzero__,__rmul__,__radd__,__repr__,__str__,__get__,__truediv__,__pow__,__name__,__future__,__all__,as,assert,nonlocal,with,yield,self,True,False,None},          % Custom highlighting
	emphstyle=\color{deepred},
	stringstyle=\color{deepgreen},
	showstringspaces=false
}
\newcommand\pythonstyle{\lstset{pythonstylesheet}}
\newcommand\pyl[1]     {{\pythonstyle\lstinline!#1!}}
\lstset{style=pythonstylesheet}


% Deisgn
\usepackage[labelfont=bf]{caption}
\usepackage[margin=0.6in]{geometry}
\usepackage{multicol}
\usepackage[skip=4pt, indent=0pt]{parskip}
\usepackage[normalem]{ulem}
\forestset{default}
\renewcommand\labelitemi{$\bullet$}
\usepackage{titlesec}


% Hebrew initialzing
\usepackage{polyglossia}
\setmainlanguage{hebrew}
\setotherlanguage{english}
\newfontfamily\hebrewfont[Script=Hebrew, Ligatures=TeX]{David CLM}
\usepackage[shortlabels]{enumitem}
\newlist{hebenum}{enumerate}{1}
\setlist[hebenum,1]{
	labelindent=\parindent,
	label={{\hebrewfont{\protect\hebrewnumeral{\value{hebenumi}}}}.}
}

% Language Shortcuts
\newcommand\en[1] {\selectlanguage{english}#1\selectlanguage{hebrew}}
\newcommand\sen   {\selectlanguage{english}}
\newcommand\she   {\selectlanguage{hebrew}}
\newcommand\del   {$ \!\! $}
\newcommand\ttt[1]{\en{\texttt{#1}}}

\newcommand\npage {\vfil {\hfil \textbf{\textit{המשך בעמוד הבא}}} \hfil \vfil}
\newcommand\ndoc  {\dotfill \\ \vfil \hfil \textbf{\textit{שחר פרץ, 2024}} \hfil \vfil}

\newcommand{\rn}[1]{
	\textup{\uppercase\expandafter{\romannumeral#1}}
}


%! ~~~ Math shortcuts ~~~

% Letters shortcuts
\newcommand\N     {\mathbb{N}}
\newcommand\Z     {\mathbb{Z}}
\newcommand\R     {\mathbb{R}}
\newcommand\Q     {\mathbb{Q}}
\newcommand\C     {\mathbb{C}}

\newcommand\ml    {\ell}
\newcommand\mj    {\jmath}
\newcommand\mi    {\imath}

\newcommand\powerset {\mathcal{P}}
\newcommand\ps    {\mathcal{P}}
\newcommand\pc    {\mathcal{P}}
\newcommand\ac    {\mathcal{A}}
\newcommand\bc    {\mathcal{B}}
\newcommand\cc    {\mathcal{C}}
\newcommand\dc    {\mathcal{D}}
\newcommand\ec    {\mathcal{E}}
\newcommand\fc    {\mathcal{F}}
\newcommand\nc    {\mathcal{N}}
\newcommand\sca   {\mathcal{S}} % \sc is already definded
\newcommand\rca   {\mathcal{R}} % \rc is already definded

\newcommand\Si    {\Sigma}

% Logic & sets shorcuts
\newcommand\siff  {\longleftrightarrow}
\newcommand\ssiff {\leftrightarrow}
\newcommand\so    {\longrightarrow}
\newcommand\sso   {\rightarrow}

\newcommand\epsi  {\epsilon}
\newcommand\vepsi {\varepsilon}
\newcommand\vphi  {\varphi}
\newcommand\Neven {\N_{\mathrm{even}}}
\newcommand\Nodd  {\N_{\mathrm{odd }}}
\newcommand\Zeven {\Z_{\mathrm{even}}}
\newcommand\Zodd  {\Z_{\mathrm{odd }}}
\newcommand\Np    {\N_+}

% Text Shortcuts
\newcommand\open  {\big(}
\newcommand\qopen {\quad\big(}
\newcommand\close {\big)}
\newcommand\also  {\text{, }}
\newcommand\defi  {\text{ definition}}
\newcommand\defis {\text{ definitions}}
\newcommand\given {\text{given }}
\newcommand\case  {\text{if }}
\newcommand\syx   {\text{ syntax}}
\newcommand\rle   {\text{ rule}}
\newcommand\other {\text{else}}
\newcommand\set   {\ell et \text{ }}
\newcommand\ans   {\mathit{Ans.}}

% Set theory shortcuts
\newcommand\ra    {\rangle}
\newcommand\la    {\langle}

\newcommand\oto   {\leftarrow}

\newcommand\QED   {\quad\quad\mathscr{Q.E.D.}\;\;\blacksquare}
\newcommand\QEF   {\quad\quad\mathscr{Q.E.F.}}
\newcommand\eQED  {\mathscr{Q.E.D.}\;\;\blacksquare}
\newcommand\eQEF  {\mathscr{Q.E.F.}}
\newcommand\jQED  {\mathscr{Q.E.D.}}

\newcommand\dom   {\text{dom}}
\newcommand\Img   {\text{Im}}
\newcommand\range {\text{range}}

\newcommand\trio  {\triangle}

\newcommand\rc    {\right\rceil}
\newcommand\lc    {\left\lceil}
\newcommand\rf    {\right\rfloor}
\newcommand\lf    {\left\lfloor}

\newcommand\lex   {<_{lex}}

\newcommand\az    {\aleph_0}
\newcommand\uaz   {^{\aleph_0}}
\newcommand\al    {\aleph}
\newcommand\ual   {^\aleph}
\newcommand\taz   {2^{\aleph_0}}
\newcommand\utaz  { ^{\left (2^{\aleph_0} \right )}}
\newcommand\tal   {2^{\aleph}}
\newcommand\utal  { ^{\left (2^{\aleph} \right )}}
\newcommand\ttaz  {2^{\left (2^{\aleph_0}\right )}}

\newcommand\n     {$n$־יה\ }

% Math A&B shortcuts
\newcommand\logn  {\log n}
\newcommand\cosx  {\cos x}
\newcommand\cost  {\cos \theta}
\newcommand\sinx  {\sin x}
\newcommand\sint  {\sin \theta}
\newcommand\tanx  {\tan x}
\newcommand\tant  {\tan \theta}
\newcommand\dx    {\,\mathrm{d}x}

\newcommand\seq   {\overset{!}{=}}
\newcommand\sseq  {\overset{?}{=}}
\newcommand\sle   {\overset{!}{\le}}
\newcommand\sge   {\overset{!}{\ge}}
\newcommand\sll   {\overset{!}{<}}
\newcommand\sgg   {\overset{!}{>}}

\newcommand\h     {\hat}
\newcommand\ve    {\vec}
\newcommand\lv    {\overrightarrow}
\newcommand\ol    {\overline}

\newcommand\mlcm  {\mathrm{lcm}}

\newcommand\limz  {\lim_{x \to 0}}
\newcommand\limxz {\lim_{x \to x_0}}
\newcommand\limi  {\lim_{x \to \infty}}
\newcommand\limni {\lim_{x \to - \infty}}
\newcommand\limpmi{\lim_{x \to \pm \infty}}

\newcommand\ta    {\theta}
\newcommand\ap    {\alpha}

\renewcommand\inf {\infty}
\newcommand  \ninf{-\inf}

% Combinatorics shortcuts
\newcommand\sumnk     {\sum_{k = 0}^{n}}
\newcommand\sumni     {\sum_{i = 0}^{n}}
\newcommand\sumnko    {\sum_{k = 1}^{n}}
\newcommand\sumnio    {\sum_{i = 1}^{n}}
\newcommand\sumai     {\sum_{i = 1}^{n} A_i}
\newcommand\nsum[2]   {\reflectbox{\displaystyle\sum_{\reflectbox{\scriptsize$#1$}}^{\reflectbox{\scriptsize$#2$}}}}

\newcommand\bink      {\binom{n}{k}}

\newcommand\cupain    {\bigcup_{i = 1}^{n} A_i}
\newcommand\cupai[1]  {\bigcup_{i = 1}^{#1} A_i}
\newcommand\cupiiai   {\bigcup_{i \in I} A_i}

\newcommand\sof[1]    {\left | #1 \right |}
\newcommand\cl [1]    {\left ( #1 \right )}

\newcommand\xot       {x_{1, 2}}
\newcommand\ano       {a_{n - 1}}
\newcommand\ant       {a_{n - 2}}

% Other shortcuts
\newcommand\tl    {\tilde}
\newcommand\op    {^{-1}}

\newcommand\bs    {\blacksquare}

%! ~~~ Document ~~~

\author{שחר פרץ}
\title{מתמטיקה בדידה $\sim$ קומבי 7 $\sim$ נוסחאות נסיגה, המשך}
\date{3 למאי 2024}


\begin{document}
	\maketitle
	
	\section{מגדלי האנוי}
	נתונים שלושה מוטות, שיקראו $A, B, C$. סבביב המוט $A$ מסודרות $n$ שונות בגודלן, מהגדולה לקטנה. בכל פעולה, מותר להזיז טבעת אחת מהמוט עליה היא נמצאת למוט אחר, ואסור להניח טבעת מעל טבעת שקטנה ממנה. המטרה, תהיה להעביר את כל הטבעות ממוט $A$ למוט $C$. 
	
	\textbf{השאלה: }מה מספר הצעדים המינימלי הדרוש לשם כך?
	
	\textbf{פתרון: }נסמן את המספר הדרוש של צעדים ב־$h(n)$. 
	\textit{מקרה פרטי: }נראה ש־$h(3) \le 7$. הוכחה: נסמן את הטבעות ב$1, 2, 3$. צעדים: 
	\[1 \to C,  \ 2 \to B, \ 1 \to B, 3 \to C, \ 1 \to A, 2 \to C, 1 \to C \ \bs \]
	\textbf{טענה: } $h(1) = 1, \ \forall n \ge 2, \ h(n) = 2h(n - 1) + 1$. 
	\begin{proof}
		נראה $\le$: (קיום אלגו' מתאים): נעביר את $n - 1 $ הטבעות העליונות ל־$B$ ($h(n - 1)$), נעביר את הטבעת הגדולה ביותר בתחתית ל־$C$ ($1$) ואז את הערימה שב־$B$ ל־$C$ ($h(n - 1)$) וסה''כ $h(n) = 2h(n - 1) + 1 $. 
		
		נראה $\ge$: (הוכחת מינימליות): כדי להגיע לכך שבמוט $C$ תהיה הטבעת $n$, עלינו להגיע למצב שבו $C$ ריק, ועל מוט $A$ ישנה הטבעת $n$ בלבד. כלומר, עלינו להעביר את $n - 1 $ הטבעות הקטנות ל־$B$ בצורה חוקית. לכן, יהיו לפחות $h(n - 1)$ צעדים לשם כך. נוסיף לזה את הצעדים הדרושים להעבירם ל־$C$, ואת המעבר של הטבעת ה־$n$ ל־$C$, ונקבל $2h(n - 1) + 1 \le h(n)$. 
		
		עתה, נרצה למצוא נוסחה סגורה. נעשה הצבה חוזרת (לא חלק מההוכחה) כדי להבין מה קורה כאן: 
		\begin{multline*}
			h(n) = 2h(n - 1) + 1 = 2(2h(n - 2) + 1) + 1 = 2^{2} +h(n - 2) + 2 + 1 = \\
			2^{2}(2h(n - 3) + 1) + 2 + 1 = h^{2}h(n - 3) + 2^{2} + 2 + 2^{0} = \dots 2^{n - 1}h(1) + 2^{n - 2} + \dots + 2^{0} = 2^{n - 1} + \sum^{n - 1} 2^{k} = \frac{2^{n} - 1}{2 - 1} = 2^{n} - 1
		\end{multline*}
		תזכורת, של סכום סדרה הנדסית: 
		\[ \sumnk q^{k} = \frac{q^{n + 1} - 1}{q - 1} \]
		
		נחזור להוכחה. נוכיח באינדוקציה $\forall n \ge 1. h(n) = 2^{n} - 1 $. \textit{בסיס: }$h(1) \sseq 2^{1} - 1$  ונקבל $1 = 1 $ כדרוש. 
		
		\textit{צעד: }
		\[ \begin{WithArrows}[groups]
			  &h(n + 1) \Arrow{by \defi} \\
			= &2h(n) + 1 \Arrow{induction} \\
			= &2(2^{n} - 1) + 1 \\
			= &2^{n + 1} - 2 + 1 = 2^{n + 1} - 1
		\end{WithArrows} \]
	\end{proof}
	
	\npage
	
	\pagebreak
	
	\section{מספרי קטלן}
	\textbf{הגדרה: }(לא פומרלית) \underline{מבנה סוגריים מאוזן } הוא רצף של ``('' ו־$'')''$ הוא רצף בו מספר הפוצחים גדול או שווה ומספר הסוגרים, וברצף כולו מספרם שווה. 
	
	\textbf{דוגמה: }עבור רצף באורך $2 \cdot 3$, כל האפשרויות הן $((())), \ ()()(), \ (())(), \ ()(()), \ (()())$, כלומר $5$ אפשרויות. 
	
	\textbf{הגדרה שקולה פורמלית}: סדרה $\{a_i\}^{2n}_{i = 1}$, אשר עבורה $\forall. 1 \le i \le 2n. a_i \in \{-1, 1\}$, היא תקראה מאוזנת אם מתקיימים שני התנאים הבאים: 
	\begin{enumerate}
		\item $\sum_{i = 1}^{2n} a_I = 0$ (יש כמות שווה של $1$ ו־$-1$). 
		\item $\forall 1 \le k \le 2n. \sum_{i = 1}^{k}a_i \ge 0 $ (התנאי השני). 
	\end{enumerate}
	
	\textbf{הגדרה: }מספר קטלן ה־$n$־י, מסומן $\cc_n$, הוא מספר מבני הסוגריים המאוזנים באורך $2n$. 
	
	\textbf{טענה: }(נוסחת נסיגה עבור $\cc_n$): 
	\[ \cc_n = \sum_{i = 0}^{n -1} \cc_i \cdot \cc_{n - 1 - i} \]
	נגיד שהיא מסדר $1$ כי זה מגדיר אותה ואנחנו לא רוצים לתת $\inf$ איברי בסיס (הסיבה האמיתית היא שזה מספיק בשביל להגדיר את כל המשתנים האחרים אבל אני מעדיף לנסח את זה ככה). תנאי התחלה $\cc_0 = 1 $. 
	
	\begin{proof}
		נתבונן במבנה סוגריים מאוזן באורך $2n$. כמות האפשרויות היא $\cc_n$. הוא, בהכרח מתחיל בפותח, שלאחריו יהיה סוגר. ביניהם, יש מבנה סוגריים מאוזן באורך $2i$ ($0 \le i \le n -1 $). מספר האפשרויות עבורו יהיה $\cc_{i}$. עבור יתר התווים, שיצרו מבנה סוגריים מאוזן באורך $2n - 2i - 2 $ (אורך כללי, פחות אורך פינמי פחות פותח וסוגר) ולכך יש $\cc_{n - i - 1}$ אפשרויות. סה''כ מכלל החיבור מצאנו שהנוסחה עובדת. 
	\end{proof}
	
	\textbf{טענה: }(נוסחה סגורה ל־$\cc_n$):
	\[ \cc_n = \binom{2n}{n}  - \frac{2n}{n - 1} \]
	
	\begin{proof}
		נוכיח קומבינטורית. \underline{נתבונן בבעיה:} מה מספר הדרכים להגיד במישור, מהנקודה $(0, 0)$ לנקודה $(n, n)$ כאשר בכל שלב מותר למנוע ימינה או למעלה בלבד מבלי לחצות את האלכסון $y = x$. 
		
		\textit{צד שמאל: }נתאים צעד ימינה ל־$(+1)$ (או $''(``$) וצעד למעלה ל־$(-1)$ (או $'')''$). לכן מספר האפשרויות החוקיות הוא $\cc_n$.  \\
		\textbf{צד ימין: }מספר האפשרויות להגיע מ־$(0, 0)$ ל־$(n, n)$ ללא ההגבלה על האלכסון, הוא $\binom{2n }{n}$; עלינו לבצע שני $n$ צעדים, מתוכם $n$ ימינה, אז נבחר אילו מהצעדים ימינה. ניעזר בעקרון המשלים, וההסבר את למה זה עובד ייגמר ברביעי כי לנטלי אין זמן. 
		
		
	\end{proof}
	
	
	
\end{document}