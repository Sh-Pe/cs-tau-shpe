%! ~~~ Packages Setup ~~~ 
\documentclass[]{article}
\usepackage{lipsum}
\usepackage{rotating}


% Math packages
\usepackage[usenames]{color}
\usepackage{forest}
\usepackage{ifxetex,ifluatex,amssymb,amsmath,mathrsfs,amsthm,witharrows,mathtools,mathdots}
\usepackage{amsmath}
\WithArrowsOptions{displaystyle}
\renewcommand{\qedsymbol}{$\blacksquare$} % end proofs with \blacksquare. Overwrites the defualts. 
\usepackage{cancel,bm}
\usepackage[thinc]{esdiff}


% tikz
\usepackage{tikz}
\usetikzlibrary{graphs}
\newcommand\sqw{1}
\newcommand\squ[4][1]{\fill[#4] (#2*\sqw,#3*\sqw) rectangle +(#1*\sqw,#1*\sqw);}


% code 
\usepackage{algorithm2e}
\usepackage{listings}
\usepackage{xcolor}

\definecolor{codegreen}{rgb}{0,0.35,0}
\definecolor{codegray}{rgb}{0.5,0.5,0.5}
\definecolor{codenumber}{rgb}{0.1,0.3,0.5}
\definecolor{codeblue}{rgb}{0,0,0.5}
\definecolor{codered}{rgb}{0.5,0.03,0.02}
\definecolor{codegray}{rgb}{0.96,0.96,0.96}

\lstdefinestyle{pythonstylesheet}{
	language=Java,
	emphstyle=\color{deepred},
	backgroundcolor=\color{codegray},
	keywordstyle=\color{deepblue}\bfseries\itshape,
	numberstyle=\scriptsize\color{codenumber},
	basicstyle=\ttfamily\footnotesize,
	commentstyle=\color{codegreen}\itshape,
	breakatwhitespace=false, 
	breaklines=true, 
	captionpos=b, 
	keepspaces=true, 
	numbers=left, 
	numbersep=5pt, 
	showspaces=false,                
	showstringspaces=false,
	showtabs=false, 
	tabsize=4, 
	morekeywords={as,assert,nonlocal,with,yield,self,True,False,None,AssertionError,ValueError,in,else},              % Add keywords here
	keywordstyle=\color{codeblue},
	emph={var, List, Iterable, Iterator},          % Custom highlighting
	emphstyle=\color{codered},
	stringstyle=\color{codegreen},
	showstringspaces=false,
	abovecaptionskip=0pt,belowcaptionskip =0pt,
	framextopmargin=-\topsep, 
}
\newcommand\pythonstyle{\lstset{pythonstylesheet}}
\newcommand\pyl[1]     {{\lstinline!#1!}}
\lstset{style=pythonstylesheet}

\usepackage[style=1,skipbelow=\topskip,skipabove=\topskip,framemethod=TikZ]{mdframed}
\definecolor{bggray}{rgb}{0.85, 0.85, 0.85}
\mdfsetup{leftmargin=0pt,rightmargin=0pt,innerleftmargin=15pt,backgroundcolor=codegray,middlelinewidth=0.5pt,skipabove=5pt,skipbelow=0pt,middlelinecolor=black,roundcorner=5}
\BeforeBeginEnvironment{lstlisting}{\begin{mdframed}\vspace{-0.4em}}
	\AfterEndEnvironment{lstlisting}{\vspace{-0.8em}\end{mdframed}}


% Design
\usepackage[labelfont=bf]{caption}
\usepackage[margin=0.6in]{geometry}
\usepackage{multicol}
\usepackage[skip=4pt, indent=0pt]{parskip}
\usepackage[normalem]{ulem}
\forestset{default}
\renewcommand\labelitemi{$\bullet$}
\usepackage{titlesec}
\titleformat{\section}[block]
{\fontsize{15}{15}}
{\sen \dotfill (\thesection)\dotfill\she}
{0em}
{\MakeUppercase}
\usepackage{graphicx}
\graphicspath{ {./} }

\usepackage[colorlinks]{hyperref}
\definecolor{mgreen}{RGB}{25, 160, 50}
\definecolor{mblue}{RGB}{30, 60, 200}
\usepackage{hyperref}
\hypersetup{
	colorlinks=true,
	citecolor=mgreen,
	linkcolor=black,
	urlcolor=mblue,
	pdftitle={Document by Shahar Perets},
	%	pdfpagemode=FullScreen,
}
\usepackage{yfonts}
\def\gothstart#1{\noindent\smash{\lower3ex\hbox{\llap{\Huge\gothfamily#1}}}
	\parshape=3 3.1em \dimexpr\hsize-3.4em 3.4em \dimexpr\hsize-3.4em 0pt \hsize}
\def\frakstart#1{\noindent\smash{\lower3ex\hbox{\llap{\Huge\frakfamily#1}}}
	\parshape=3 1.5em \dimexpr\hsize-1.5em 2em \dimexpr\hsize-2em 0pt \hsize}



% Hebrew initialzing
\usepackage[bidi=basic]{babel}
\PassOptionsToPackage{no-math}{fontspec}
\babelprovide[main, import, Alph=letters]{hebrew}
\babelprovide[import]{english}
\babelfont[hebrew]{rm}{David CLM}
\babelfont[hebrew]{sf}{David CLM}
%\babelfont[english]{tt}{Monaspace Xenon}
\usepackage[shortlabels]{enumitem}
\newlist{hebenum}{enumerate}{1}

% Language Shortcuts
\newcommand\en[1] {\begin{otherlanguage}{english}#1\end{otherlanguage}}
\newcommand\he[1] {\she#1\sen}
\newcommand\sen   {\begin{otherlanguage}{english}}
	\newcommand\she   {\end{otherlanguage}}
\newcommand\del   {$ \!\! $}

\newcommand\npage {\vfil {\hfil \textbf{\textit{המשך בעמוד הבא}}} \hfil \vfil \pagebreak}
\newcommand\ndoc  {\dotfill \\ \vfil {\begin{center}
			{\textbf{\textit{שחר פרץ, 2025}} \\
				\scriptsize \textit{קומפל ב־}\en{\LaTeX}\,\textit{ ונוצר באמצעות תוכנה חופשית בלבד}}
	\end{center}} \vfil	}

\newcommand{\rn}[1]{
	\textup{\uppercase\expandafter{\romannumeral#1}}
}

\makeatletter
\newcommand{\skipitems}[1]{
	\addtocounter{\@enumctr}{#1}
}
\makeatother

%! ~~~ Math shortcuts ~~~

% Letters shortcuts
\newcommand\N     {\mathbb{N}}
\newcommand\Z     {\mathbb{Z}}
\newcommand\R     {\mathbb{R}}
\newcommand\Q     {\mathbb{Q}}
\newcommand\C     {\mathbb{C}}
\newcommand\One   {\mathit{1}}

\newcommand\ml    {\ell}
\newcommand\mj    {\jmath}
\newcommand\mi    {\imath}

\newcommand\powerset {\mathcal{P}}
\newcommand\ps    {\mathcal{P}}
\newcommand\pc    {\mathcal{P}}
\newcommand\ac    {\mathcal{A}}
\newcommand\bc    {\mathcal{B}}
\newcommand\cc    {\mathcal{C}}
\newcommand\dc    {\mathcal{D}}
\newcommand\ec    {\mathcal{E}}
\newcommand\fc    {\mathcal{F}}
\newcommand\nc    {\mathcal{N}}
\newcommand\oc    {\mathcal{O}}
\newcommand\vc    {\mathcal{V}} % Vance
\newcommand\sca   {\mathcal{S}} % \sc is already definded
\newcommand\rca   {\mathcal{R}} % \rc is already definded
\newcommand\zc    {\mathcal{Z}}

\newcommand\prm   {\mathrm{p}}
\newcommand\arm   {\mathrm{a}} % x86
\newcommand\brm   {\mathrm{b}}
\newcommand\crm   {\mathrm{c}}
\newcommand\drm   {\mathrm{d}}
\newcommand\erm   {\mathrm{e}}
\newcommand\frm   {\mathrm{f}}
\newcommand\nrm   {\mathrm{n}}
\newcommand\vrm   {\mathrm{v}}
\newcommand\srm   {\mathrm{s}}
\newcommand\rrm   {\mathrm{r}}

\newcommand\Si    {\Sigma}

% Logic & sets shorcuts
\newcommand\siff  {\longleftrightarrow}
\newcommand\ssiff {\leftrightarrow}
\newcommand\so    {\longrightarrow}
\newcommand\sso   {\rightarrow}

\newcommand\epsi  {\epsilon}
\newcommand\vepsi {\varepsilon}
\newcommand\vphi  {\varphi}
\newcommand\Neven {\N_{\mathrm{even}}}
\newcommand\Nodd  {\N_{\mathrm{odd }}}
\newcommand\Zeven {\Z_{\mathrm{even}}}
\newcommand\Zodd  {\Z_{\mathrm{odd }}}
\newcommand\Np    {\N_+}

% Text Shortcuts
\newcommand\open  {\big(}
\newcommand\qopen {\quad\big(}
\newcommand\close {\big)}
\newcommand\also  {\mathrm{, }}
\newcommand\defis {\mathrm{ definitions}}
\newcommand\given {\mathrm{given }}
\newcommand\case  {\mathrm{if }}
\newcommand\syx   {\mathrm{ syntax}}
\newcommand\rle   {\mathrm{ rule}}
\newcommand\other {\mathrm{else}}
\newcommand\set   {\ell et \text{ }}
\newcommand\ans   {\mathscr{A}\!\mathit{nswer}}

% Set theory shortcuts
\newcommand\ra    {\rangle}
\newcommand\la    {\langle}

\newcommand\oto   {\leftarrow}

\newcommand\QED   {\quad\quad\mathscr{Q.E.D.}\;\;\blacksquare}
\newcommand\QEF   {\quad\quad\mathscr{Q.E.F.}}
\newcommand\eQED  {\mathscr{Q.E.D.}\;\;\blacksquare}
\newcommand\eQEF  {\mathscr{Q.E.F.}}
\newcommand\jQED  {\mathscr{Q.E.D.}}

\DeclareMathOperator\dom   {dom}
\DeclareMathOperator\Img   {Im}
\DeclareMathOperator\range {range}

\newcommand\trio  {\triangle}

\newcommand\rc    {\right\rceil}
\newcommand\lc    {\left\lceil}
\newcommand\rf    {\right\rfloor}
\newcommand\lf    {\left\lfloor}
\newcommand\ceil  [1] {\lc #1 \rc}
\newcommand\floor [1] {\lf #1 \rf}

\newcommand\lex   {<_{lex}}

\newcommand\az    {\aleph_0}
\newcommand\uaz   {^{\aleph_0}}
\newcommand\al    {\aleph}
\newcommand\ual   {^\aleph}
\newcommand\taz   {2^{\aleph_0}}
\newcommand\utaz  { ^{\left (2^{\aleph_0} \right )}}
\newcommand\tal   {2^{\aleph}}
\newcommand\utal  { ^{\left (2^{\aleph} \right )}}
\newcommand\ttaz  {2^{\left (2^{\aleph_0}\right )}}

\newcommand\n     {$n$־יה\ }

% Math A&B shortcuts
\newcommand\logn  {\log n}
\newcommand\logx  {\log x}
\newcommand\lnx   {\ln x}
\newcommand\cosx  {\cos x}
\newcommand\sinx  {\sin x}
\newcommand\sint  {\sin \theta}
\newcommand\tanx  {\tan x}
\newcommand\tant  {\tan \theta}
\newcommand\sex   {\sec x}
\newcommand\sect  {\sec^2}
\newcommand\cotx  {\cot x}
\newcommand\cscx  {\csc x}
\newcommand\sinhx {\sinh x}
\newcommand\coshx {\cosh x}
\newcommand\tanhx {\tanh x}

\newcommand\seq   {\overset{!}{=}}
\newcommand\slh   {\overset{LH}{=}}
\newcommand\sle   {\overset{!}{\le}}
\newcommand\sge   {\overset{!}{\ge}}
\newcommand\sll   {\overset{!}{<}}
\newcommand\sgg   {\overset{!}{>}}

\newcommand\h     {\hat}
\newcommand\ve    {\vec}
\newcommand\lv    {\overrightarrow}
\newcommand\ol    {\overline}

\newcommand\mlcm  {\mathrm{lcm}}

\DeclareMathOperator{\sech}   {sech}
\DeclareMathOperator{\csch}   {csch}
\DeclareMathOperator{\arcsec} {arcsec}
\DeclareMathOperator{\arccot} {arcCot}
\DeclareMathOperator{\arccsc} {arcCsc}
\DeclareMathOperator{\arccosh}{arccosh}
\DeclareMathOperator{\arcsinh}{arcsinh}
\DeclareMathOperator{\arctanh}{arctanh}
\DeclareMathOperator{\arcsech}{arcsech}
\DeclareMathOperator{\arccsch}{arccsch}
\DeclareMathOperator{\arccoth}{arccoth}
\DeclareMathOperator{\atant}  {atan2} 
\DeclareMathOperator{\Sp}     {span} 
\DeclareMathOperator{\sgn}    {sgn} 
\DeclareMathOperator{\row}    {Row} 
\DeclareMathOperator{\adj}    {adj} 
\DeclareMathOperator{\rk}     {rank} 
\DeclareMathOperator{\col}    {Col} 
\DeclareMathOperator{\tr}     {tr}

\newcommand\dx    {\,\mathrm{d}x}
\newcommand\dt    {\,\mathrm{d}t}
\newcommand\dtt   {\,\mathrm{d}\theta}
\newcommand\du    {\,\mathrm{d}u}
\newcommand\dv    {\,\mathrm{d}v}
\newcommand\df    {\mathrm{d}f}
\newcommand\dfdx  {\diff{f}{x}}
\newcommand\dit   {\limhz \frac{f(x + h) - f(x)}{h}}

\newcommand\nt[1] {\frac{#1}{#1}}

\newcommand\limz  {\lim_{x \to 0}}
\newcommand\limxz {\lim_{x \to x_0}}
\newcommand\limi  {\lim_{x \to \infty}}
\newcommand\limh  {\lim_{x \to 0}}
\newcommand\limni {\lim_{x \to - \infty}}
\newcommand\limpmi{\lim_{x \to \pm \infty}}

\newcommand\ta    {\theta}
\newcommand\ap    {\alpha}

\renewcommand\inf {\infty}
\newcommand  \ninf{-\inf}

% Combinatorics shortcuts
\newcommand\sumnk     {\sum_{k = 0}^{n}}
\newcommand\sumni     {\sum_{i = 0}^{n}}
\newcommand\sumnko    {\sum_{k = 1}^{n}}
\newcommand\sumnio    {\sum_{i = 1}^{n}}
\newcommand\sumai     {\sum_{i = 1}^{n} A_i}
\newcommand\nsum[2]   {\reflectbox{\displaystyle\sum_{\reflectbox{\scriptsize$#1$}}^{\reflectbox{\scriptsize$#2$}}}}

\newcommand\bink      {\binom{n}{k}}
\newcommand\setn      {\{a_i\}^{2n}_{i = 1}}
\newcommand\setc[1]   {\{a_i\}^{#1}_{i = 1}}

\newcommand\cupain    {\bigcup_{i = 1}^{n} A_i}
\newcommand\cupai[1]  {\bigcup_{i = 1}^{#1} A_i}
\newcommand\cupiiai   {\bigcup_{i \in I} A_i}
\newcommand\capain    {\bigcap_{i = 1}^{n} A_i}
\newcommand\capai[1]  {\bigcap_{i = 1}^{#1} A_i}
\newcommand\capiiai   {\bigcap_{i \in I} A_i}

\newcommand\xot       {x_{1, 2}}
\newcommand\ano       {a_{n - 1}}
\newcommand\ant       {a_{n - 2}}

% Linear Algebra
\DeclareMathOperator{\chr}     {char}
\DeclareMathOperator{\diag}    {diag}
\DeclareMathOperator{\Hom}     {Hom}
\DeclareMathOperator{\Sym}     {Sym}
\DeclareMathOperator{\Asym}    {ASym}

\newcommand\lra       {\leftrightarrow}
\newcommand\chrf      {\chr(\F)}
\newcommand\F         {\mathbb{F}}
\newcommand\co        {\colon}
\newcommand\tmat[2]   {\cl{\begin{matrix}
			#1
		\end{matrix}\, \middle\vert\, \begin{matrix}
			#2
\end{matrix}}}

\makeatletter
\newcommand\rrr[1]    {\xxrightarrow{1}{#1}}
\newcommand\rrt[2]    {\xxrightarrow{1}[#2]{#1}}
\newcommand\mat[2]    {M_{#1\times#2}}
\newcommand\gmat      {\mat{m}{n}(\F)}
\newcommand\tomat     {\, \dequad \longrightarrow}
\newcommand\pms[1]    {\begin{pmatrix}
		#1
\end{pmatrix}}

\newcommand\norm[1]   {\left \vert \left \vert #1 \right \vert \right \vert}
\newcommand\snorm     {\left \vert \left \vert \cdot \right \vert \right \vert}
\newcommand\smut      {\left \la \cdot \mid \cdot \right \ra}
\newcommand\mut[2]    {\left \la #1 \,\middle\vert\, #2 \right \ra}

% someone's code from the internet: https://tex.stackexchange.com/questions/27545/custom-length-arrows-text-over-and-under
\makeatletter
\newlength\min@xx
\newcommand*\xxrightarrow[1]{\begingroup
	\settowidth\min@xx{$\m@th\scriptstyle#1$}
	\@xxrightarrow}
\newcommand*\@xxrightarrow[2][]{
	\sbox8{$\m@th\scriptstyle#1$}  % subscript
	\ifdim\wd8>\min@xx \min@xx=\wd8 \fi
	\sbox8{$\m@th\scriptstyle#2$} % superscript
	\ifdim\wd8>\min@xx \min@xx=\wd8 \fi
	\xrightarrow[{\mathmakebox[\min@xx]{\scriptstyle#1}}]
	{\mathmakebox[\min@xx]{\scriptstyle#2}}
	\endgroup}
\makeatother


% Greek Letters
\newcommand\ag        {\alpha}
\newcommand\bg        {\beta}
\newcommand\cg        {\gamma}
\newcommand\dg        {\delta}
\newcommand\eg        {\epsi}
\newcommand\zg        {\zeta}
\newcommand\hg        {\eta}
\newcommand\tg        {\theta}
\newcommand\ig        {\iota}
\newcommand\kg        {\keppa}
\renewcommand\lg      {\lambda}
\newcommand\og        {\omicron}
\newcommand\rg        {\rho}
\newcommand\sg        {\sigma}
\newcommand\yg        {\usilon}
\newcommand\wg        {\omega}

\newcommand\Ag        {\Alpha}
\newcommand\Bg        {\Beta}
\newcommand\Cg        {\Gamma}
\newcommand\Dg        {\Delta}
\newcommand\Eg        {\Epsi}
\newcommand\Zg        {\Zeta}
\newcommand\Hg        {\Eta}
\newcommand\Tg        {\Theta}
\newcommand\Ig        {\Iota}
\newcommand\Kg        {\Keppa}
\newcommand\Lg        {\Lambda}
\newcommand\Og        {\Omicron}
\newcommand\Rg        {\Rho}
\newcommand\Sg        {\Sigma}
\newcommand\Yg        {\Usilon}
\newcommand\Wg        {\Omega}

% Other shortcuts
\newcommand\tl    {\tilde}
\newcommand\op    {^{-1}}

\newcommand\sof[1]    {\left | #1 \right |}
\newcommand\cl [1]    {\left ( #1 \right )}
\newcommand\csb[1]    {\left [ #1 \right ]}
\newcommand\ccb[1]    {\left \{ #1 \right \}}

\newcommand\bs        {\blacksquare}
\newcommand\dequad    {\!\!\!\!\!\!}
\newcommand\dequadd   {\dequad\duquad}

\renewcommand\phi     {\varphi}

\newtheorem{Theorem}{משפט}
\theoremstyle{definition}
\newtheorem{definition}{הגדרה}
\newtheorem{Lemma}{למה}
\newtheorem{Remark}{הערה}
\newtheorem{Notion}{סימון}


\newcommand\theo  [1] {\begin{Theorem}#1\end{Theorem}}
\newcommand\defi  [1] {\begin{definition}#1\end{definition}}
\newcommand\rmark [1] {\begin{Remark}#1\end{Remark}}
\newcommand\lem   [1] {\begin{Lemma}#1\end{Lemma}}
\newcommand\noti  [1] {\begin{Notion}#1\end{Notion}}

% DS
\newcommand\limsi     {\limsup_{n \to \inf}}
\newcommand\limfi     {\liminf_{n \to \inf}}

\DeclareMathOperator\amort   {amort}
\DeclareMathOperator\worst   {worst}
\DeclareMathOperator\type    {type}
\DeclareMathOperator\cost    {cost}
\DeclareMathOperator\tim     {time}

\newcommand\dsList{
	\sFunc{List}
	\sFunc{Retrieve}
	\SetKwFunction{RetrieveFirst}{Retrieve-First}
	\SetKwFunction{RetrieveLast}{Retrieve-Last}
	\sFunc{Delete}
	\SetKwFunction{DeleteFirst}{Delete-First}
	\SetKwFunction{DeleteLast}{Delete-Last}
	\sFunc{Insert}
	\SetKwFunction{InsertFirst}{Insert-First}
	\SetKwFunction{InsertLast}{Insert-Last}
	\sFunc{Shift}
	\sFunc{Length}
	\sFunc{Concat}
	\sFunc{Plant}
	\sFunc{Split}
}
\newcommand\dsQueue{
	\sFunc{Queue}
	\sFunc{Enqueue}
	\sFunc{Head}
	\sFunc{Dequeue}
}
\newcommand\dsStack{
	\sFunc{Stack}
	\sFunc{Push}
	\sFunc{Top}
	\sFunc{Pop}
}
\newcommand\dsVector{
	\sFunc{Vector}
	\sFunc{Get}
	\sFunc{Set}
}
\newcommand\dsGraph{
	\sFunc{Graph}
	\sFunc{Edge}
	\SetKwFunction{AddEdge}{Add-Edge}
	\SetKwFunction{RemoveEdge}{Remove-Edge}
	\sFunc{InDeg} \sFunc{OutDeg}
}
\newcommand\importDs{
	\dsList
	\dsQueue
	\dsStack
	\dsVector
	\dsGraph
	\SetKwProg{Fn}{function}{ is}{end}
	\SetKwData{error}{\color{codered}error}
	\SetKwInOut{Input}{input}
	\SetKwInOut{Output}{output}
	\SetKwRepeat{Do}{do}{while}
	\SetKwData{Null}{\color{codegreen}null}
	\SetKwData{True}{\color{codeblue}true}
	\SetKwData{False}{\color{codeblue}false}
}


% Algorithems
\newcommand\sFunc [1] {\SetKwFunction{#1}{#1}}
\newcommand\sData [1] {\SetKwData{#1}{#1}}
\newcommand\sIO   [1] {\SetKwInOut{#1}{#1}}
\newcommand\ttt   [1] {\sen \texttt{#1} \she\,}
\newcommand\io    [2] {\Input{#1}\Output{#2}\BlankLine}

%! ~~~ Document ~~~

\author{שחר פרץ}
\title{\textit{מבני נתונים $\sim$ תרגיל בית 5 $\sim$ 2025B}}
\begin{document}
	\maketitle
	\textbf{ת.ז.: }334558962\\
	\textbf{שם במודל: }shaharperets
	
	\section{}
	\begin{enumerate}
		\item נניח שבאלגוריתם MedofMed מחלקים את המערך לתשיעיות, ובמקום לבחור את החציון נבחור את האיבר ה־$i$ בגודלו (כאשר $1 \le i \le 9$). נמצא את ה־$i$ המינימלי כך שזמן הריצה לינארי. \begin{proof}[פתרון]
			בהרצאה, צוין ש־$T(n) = cn + T(\ag n) + T(\bg n)$ מקיימת $T(n) = \oc(n)$ אם $\ag + \bg < 1$. נוכיח אמ''מ, כלומר נוכיח גם את הגרירה השנייה. נעשה זאת בקונטרא־פוזיטיב. נניח $\ag + \bg \ge 1$, נפרק למקרים. 
			\begin{itemize}
				\item אם $\ag + \bg = 1$, אזי זוהי הוכחה זהה לזו ש־merge sort מסתיים בזמן $n \logn \neq \oc(n)$. 
				\item אם $\ag + \bg > 1$, אז $\ag > 0.5 \lor \bg > 0.5$, ולכן נוכל להניח בה''כ $\ag > 0.5$. מתקיים: 
				\[ T(n) = cn + T(\ag n) + T(\bg n) \ge cn + T(\ag n) > cn + T(0.5n) \]
				ממשפט האב $T(n) = cn + T(0.5n)$ מקיים $T(n) = \Theta(n\logn)$ ולכן בפרט $T(n) = \Omega(n\logn)$ ומכאן נסיק $T(n) \neq \oc(n)$. 
			\end{itemize}
			
			סה''כ הוכחנו שקילות. נחזור לשאלה עצמה. בדומה להרצאה, סיבוכיות MedofMed ניתנת לחסימה ע''י: 
			\[ T(n) = cn + T\cl{n - \frac{ni}{22.5}} + T\cl{\ceil{\frac{n}{9}}} \]
			כאשר הביטוי חוסם במדויק שכן גודל המלבן: 
			\[ \text{\en{width}} = \ceil{\frac{1}{(9 + 1)/i}\ceil{\frac{i}{9}}} \overset{n \to \inf}{=} \frac{ni}{90}, \ \text{\en{length}} = 5 \implies \text{\en{size}} = \frac{ni}{90} \cdot 5 - 1 - 5 \overset{n \to \inf}{=} \frac{ni}{18} \]
			ועלות הקריאה הרקורסיבית למציאת האיבר ה־$i$ בגודלו היא כמובן $T(\ceil{\frac{n}{9}})$. 
			עתה נדרוש ש־: 
			\[ 1 - \frac{i}{18} + \frac{1}{9} < 1 \iff \frac{i}{18} > \frac{1}{9} \iff i > 2 \]
			ה־$i$ הטבעי המינימלי שייקים זאת הוא $\bm{i = 3}$. 
		\end{proof}
		\item נחזור על הסעיף הקודם בעבור חלוקה לשביעיות. \begin{proof}[פתרון]
			בדומה לסעיף הקודם: 
			\[ \text{\en{width}} = \ceil{\frac{1}{(7 + 1)/i}\ceil{\frac{i}{7}}} \overset{n \to \inf}{=} \frac{ni}{56}, \ \text{\en{length}} = 4 \implies \text{\en{size}} = \frac{ni}{56} \cdot 4 - 1 - 4 \overset{n \to \inf}{=} \frac{ni}{14} \]
			כלומר נקבל: 
			\[ T(n) = cn + T\cl{n\cl{1 - \frac{i}{14}}} + T\cl{\ceil{\frac{n}{7}}} \]
			ונדרוש: 
			\[ 1 - \frac{i}{14} + \frac{1}{7} < 1 \implies \frac{i}{14} > \frac{1}{7} \implies \N \ni i > 2 \implies i = 3 \]
			גם כאן נקבל ש־$i = 3$ המינימלי. עם זאת, התבקשנו למצוא מקסימלי. משום ברמת הקבועים, מציאת האיבר ה־$i$ בגודלו ומציאת האיבר ה־$(7 + 1) - i$ בגודלו מסמטריה, אז ה־$i$ הטבעי המקסימלי הוא $8 - 3 = \bm{5}$. 
		\end{proof}
	\end{enumerate}
	\npage
	
	\section{}
	נתון מערך בגודל $n$, המכיל $n$ איברים שונים זה מזה. נתון ששלכל $i \in [n]$, האיבר ה־$i$ בגודלו נמצא בטווח $[i - k, i + k] \cap \N$ עבור $k \in [n]$ כלשהו. 
	\begin{enumerate}
		\item נוכיח ש־insertion sort ימיין את המערך ב־$\oc(n \log k)$. \begin{proof}
			לשם כך, נוכיח טענת עזר: בעבור $A$ מערך המקיים את התכונות לעיל, $I(A) \le cnk$ עבור $c$ קבוע כלשהו, כאשר $I(A)$ מוגדר להיות $I(A) = \#\{i < j \in [n] \co A_i > A_j\}$. נתבונן באיבר ה־$i$ במערך מסודר. אזי ב־$A$, במקרה הגרוע, יהיה במיקום $i + k$ או במיקום $i - k$. במקרה הראשון, יהיו לכל היותר $2k - 1$ שיכולים להיות גדולים ממנו ב־$A$ אך לא בסדר ממוין, שכן הם בהכרח מצויים בטווח $[i - k, i + k) \cap \N$ ובמקרה השני יכולים להיות לכל היותר $2k - 1$ מספרים שקטנים ממנו, באופן דומה. סה''כ נקבל, אם נסכום על כל $i \in [n]$: 
			\[ I(A) \le n(2k - 1) = 2nk - n \le 2nk \]
			ואכן הראינו את הדרוש בעבור $c = 2$. 
			
			ניגש להוכיח את הטענה עצמה. בתרגיל בית 3, הוכחנו בסעיף האחרון ש־merge sort פועל בסיבוכיות: 
			\[ O\cl{n \log \cl{\frac{I(A)}{n} + 2}} = \cost \overset{\forall n \ge n_0}{=} \gamma \cdot n \log \cl{\frac{I(A)}{n} + 2} \le \gamma {n \log \cl{\frac{2nk}{n} + 2}} = \gamma n \log (2k + 2) \le \gamma n \log k \]
			(כאשר $\gamma$ קבוע ממשי) סה''כ $\cost = \oc(n \log k)$ כדרוש. 
		\end{proof}
		
		\item עבור $n = \sqrt n$, נוכיח $\Omega(n \logn)$ למיון המערך. \begin{proof}
			נניח בשלילה שעבור $n = \sqrt n$ אפשר למיין בסיבוכיות $ \Omega(n \logn)\neq$. 
			
			נתבונן בבעיה הבאה: יהיו $(A^{i})_{i = 1}^{\sqrt n}$ מערכים מגודל $\sqrt n$. בעזרת אלגו' המיון שהנחנו בשלילה את קיומו, נוכל לקחת את איברי כל המערכים, לסמן כל אחד מהם במספר מהערך שמהם הגיעו, ולשרשר אותם יחדיו, ולקבל $A = \bigoplus_{i = 1}^{\sqrt n}(\la i, A^i_j \ra)_{j = 1}^{\sqrt n}$, כלומר: 
			\[ A = \la 1, A^1_1 \ra \cdots \la 1, A^1_{\sqrt n} \ra, \la 2, A^2_1 \ra \cdots \la 2, A^{2}_{\sqrt n} \ra \cdots\cdots \la \sqrt n, A^{\sqrt n}_1 \ra \cdots \la \sqrt n, A^{\sqrt n}_{\sqrt n} \ra \]
			נמיין אותם בסדר הלקסיקוגרפי על $\la i, j \ra$, ובכך נקבל את $\sqrt n$ המערכים שלנו ממויינים כדרוש, בפחות מ־$\Omega(n \logn)$. הפעלת האלגו' שהנחנו בשלילה את קיומו חוקית, כי כל איבר נמצא במרחק $\sqrt n$ לכל היותר ממקומו המקורי, במקרה הגרוע בו המערך ה־$A_i$ הפוך לסדר המסודר שלו (נובע ישירות מהבנייה שלנו ומהגדרת הסדר הלקסיקוגרפי). 
			
			לעומת זאת, ידוע שאת בעיית המיון בעבור $A_i$ ניתן לפתור ב־$\Omega(\sqrt n \log \sqrt n)$. נכפיל זאת ב־$\sqrt n$ הפעלות ונקבל: 
			\[ \sqrt n\Omega(\sqrt n \log \sqrt n) = \Omega((\sqrt n)^2 \log n^{0.5}) = \Omega(0.5 n\logn) = \Omega(n \logn) \]
			כלומר, החסם התחתון לבעיה זו הוא $\Omega(n \log n)$ על אף שתחת הנחת השלילה פתרנו אותה בפחות מכך. סתירה, כדרוש. 
		\end{proof}
	\end{enumerate}
	
	\npage
	\section{}
	בהינתן $A, B$ מערכים בגודל $n$, המכילים איברים מתחום סדור כלשהו, נרצה למצוא מערך $C$ כך ש־$C_i$ הוא מספר האיברים ב־$A$ הקטנים או שוווים ל־$B[i]$. 
	\begin{enumerate}[A.]
		\item נמצא אלגו' לחישוב $C$ \begin{proof}[תשובה.]
			ראשית כל, נמיין את $A$ באמצעות merge sort בסיבוכיות $\oc(n\logn)$. לאחר מכן, נבצע את ההשמה הבאה: 
			\en{\[ \forall i \in [n] \co C\texttt{[$i$]} \gets \texttt{Search($A$, $B\texttt{[$i$]}$)} + 1 \]}\vspace{-1em}
			
			כאשר \texttt{Search} מבצע חיפוש בינארי של $B\texttt{\en{[$i$]}}$ ב־$A$ (משום ש־$A$ מוין, הוא מחזיר כמה איברים ב־$A$ קטנים מ־$B\texttt{\en{[$i$]}}$). עלות החיפוש הבינארי היא $\oc(\logn)$ ואנו מבצעים אותו $n$ פעמים, סה''כ $\oc(n \logn)$ לפעולה זו. סיבוכיות כוללת $\oc(n \logn) + \oc(n\logn) = \oc(n\logn)$. 
		\end{proof}
		\item נוכיח חסם תחתון הבדוק במודל ההשוואות לבעיה. \begin{proof}
			ישנו חסם תחתון $\Omega(n\logn)$ לבעיה: משום ש־$C[i] \in [n - 1]$ (לא יכולים להיות יותר מ־$n$ איברים שקטנים או שווים מאיבר נתון, כאשר ישנם רק $n$ איברים), אזי $C \in [n] \to [n - 1]$ ולכן ישנם $n^{n - 1}$ אפשרויות ל־output של האלגו'. לכן, יש $\ml = n^{n - 1}$ עלים בעץ ההשוואות, ועומקו $\Omega(\log n^{n - 1}) = \Omega((n - 1)\logn)  = \Omega(n\logn)$ מה שגם מהווה חסם תחתון על זמן הריצה של האלגו'. חסם זה הדוק שכן הראינו קיום אלגו' במודל ההשוואות בסעיף א'.
		\end{proof}
	\end{enumerate}
	
	\npage
	\section{}
	\begin{enumerate}[A.]
		\item נממש את ה־ADT הבא: \en{\texttt{Insert($Q, k$)}, \texttt{Delete($Q, X$), \texttt{ApproxMedian(Q)}}} כאשר \texttt{AprroxMedian} מחזיר משפר הגדול מלפחות $\frac{n}{4}$ מאיברי $Q$ וקטן מלפחות $\frac{n}{4}$ מהם. 
		\begin{proof}[פתרון]
			ניעזר ברשימה מקושרת עצלה. הפעולה \texttt{Insert} תכניס אליה את $k$ למבנה $Q$. הפעולה \texttt{Delete} תמחק את $x$ מהמבנה, בצורה עצלה: כלומר, תסמן אותו כאיבר מחוק, אך לא תזיז את הרשימה בפועל. כבר הוכחנו בהרצאה שפעולות אלו פועלות באמורטייז $\oc(1)$ (כאשר \texttt{Insert} מסוגל לתקן את הרשימה במעבר על איברים מחוקים). 
			
			ננקט בשיטה הבאה כדי לתחזק שדה Median: נחזיק counter, שבכל פעם שעובר חזקה של שתיים, הוא מחשב Median באמצעות MedofMed. \textit{הערה לגבי מימוש: יש צורך לדלג על איברים מחוקים. }נעשה זאת בכל הוספה שבא $\floor{\logn}$ גדל ב־$1$, ובכל מחיקה שבא $\floor{\log(0.75n)}$ (הכפל בקבוע נועד כדי למנוע רצף של חיסורים ואז הוספות שבכל אחד מהם מחשבים Median). 
			\begin{itemize}
				\item \textbf{סיבוכיות: }הוספת חישוב ב־$\oc(n)$ לפעולה בכל פעם לאחר $2^{i}$ ריצות, שקול לחלוטין להעתקת מערכים שמכפילים את עצמם, מה שהוכחנו כבר שלוקח $\oc(1)$ אמורטייז. נוכל להוכיח זאת בשיטת הבנק, כאשר מוסיפים מספר קבוע של $2$ מטבעות על כל איבר שמוסיפים, וכן $2$ מטבעות על כל איבר שמוחקים. נשתמש במטבעות שעל המחוקים כדי לעבור עליהם ולהוציא אותם מהרשימה, ומשום שיש לפחות $\frac{n}{2}$ איברים שנוספו מאז הקריאה האחרונה ל־MedofMedians, ועליהם שתי מטבעות, יהיה לנו $n$ מטבעות כדי לחשב את החציון. נשמור אותו בשדה המתאים. 
				\item \textbf{תקינות: }משום שהחציון מחושב כל לכל היותר חזקה של שניים, במקרה הגרוע בהוספה הבאה המערך יצטרך לחשב חציון מחדש, כלומר החציון חושב לפני $\frac{n + 1}{2}$ הוספות. בפרט, אז היה החציון של $\frac{n + 1}{2}$ האיברים, ובמקרה הגרוע שבו כל האיברים שנוספו מאז גדולים ממש או קטנים ממש ממנו, החציון כרגע עם $\frac{n + 1}{4} > \frac{n}{4}$ איברים לפניו/אחריו. 
			\end{itemize}
			
			לסיום, הקריאה ל־\texttt{AprroxMedian} תחזיר את החציון שחושב ונשמר בשדה. 
		\end{proof}
		
		\item נוכיח אי־קיום מימוש ל־ADT עם הפעולות \en{\texttt{Insert($Q, k$), \texttt{Median($Q$)}}}\, כאשר שתיהן פועלות ב־$\oc(1)$ אמורטייז, תחת מודל ההשוואות. \begin{proof}[פתרון.]
			נניח בשלילה שקיים מבנה מתאים. יהיו $a_1 \dots a_n \in \N$. נעשה \texttt{Init} ל־$Q$ ונפעיל \en{\texttt{Insert($Q, a_i$), Insert($Q, -a_i$)}} \, לכל $i \in [n]$ (אומנם $a_i \in \N$, אך יחס הסדר שאנחנו עובדים מעליו הוא $<_\Z$). עתה, החציון עומד עם המינימום. נמצא את המקסימום הנוכחי $m$, ונבצע את הסדרה הבאה: 
			
			\en{\texttt{Insert($Q, m + 1$), $b_1$ $\gets$ Median($Q$), Insert($Q, m + 2$), Insert($Q, m + 3$), $b_2$ $\gets$ Median($Q$), Insert($Q, m + 4$), Insert($Q, m + 5$), $b_3$ $\gets$ Median($Q$), $\dots\dots$ Insert($Q, m + 2n$), Insert($Q, m + 2n + 1$), $b_n$ $\gets$ Median($Q$)}}
			
			בסדרה זו ישנן $3n$ פעולות שאורכות $\oc(1)$ אמורטייז, כלומר סה''כ אורכות $\oc(n)$. נבחין ש־$b_1 \dots b_n$ הם $a_1 \dots a_n$ מסודרים, שכן לבצע שתי הוספות גדולות מהחציון (ובפרט ביצוע הוספות גדולות מהמקסימום) שקולות ללעשות $median \gets \texttt{\en{successor($median$)}}$ (כלומר שהחציון יהפוך להיות העוקב של עצמו). עוד נבחין ש־$b_n = m$ כמצופה. סה''כ מיינו $n$ איברים בסיבוכיות $\oc(n)$ בתוך מודל ההשוואות, בסתירה לחסם תחתון $\Omega(n\logn)$. 
		\end{proof}
	\end{enumerate}
	
	\npage
	\section{}
	\begin{enumerate}[A.]
		\item נתון עץ השוואות המתאים לאלגוריתם הממיין $n$ מספרים. נמצא את העומק המינימלי של עלה בעץ. \begin{proof}
			נגיע לעלה כאשר האלגו' ביצע מספיק השוואות כדי להיות בטוח בתוצאה. נוכיח שקיים ומינימלי עלה בעומק $n - 1$. 
			\begin{itemize}
				\item \textbf{קיום: }עבור מערך מסודר, רצף ההשואות שמשווה את האיבר ה־$i$ ל־$i + 1$, המתחיל ב־$1$ ומסתיים ב־$n - 1$, מוכיח לאלגו' שמערך ממוין ובכך מאפשר לעשות halting. כלומר, מצאנו מקרה ועלה מעומק $n - 1$. 
				\item \textbf{מינימליות: }נניח בשלילה שקיים מסלול קצר יותר. משום שהאלגו' ממיין את הקלט, אז הוא בפרט מוצא את המינימום בעץ. כלומר, מצאנו את המינימום במערך תוך $a \le n - 2$ השוואות. בפרט, לא השוונו את המינימום לכל האיברים באופן ישיר או עקיף (כלומר, קפלסיבית או טרנזטיבית) וייתכן איבר גדול יותר מהמינימום במערך, מה שמהווה סתירה. לכן אורך מסלול הוא לפחות $n - 1$. 
			\end{itemize}
			סה''כ הראינו ש־$n - 1$ העומק המינימלי של עלה בעץ, והראנו שחסם זה התחתון ההדוק ביותר שניתן לתת. 
		\end{proof}
		\item נתון אלגו' השוואות המקבל כקלט שתי רשימות ממויינות באורך $n$ כל אחת, וממזג אותן לרשימה ממוינת אחת. 
		\begin{enumerate}[1.]
			\item נראה שמספר העלים בעץ הוא לפחות $\binom{2n}{n}$. \begin{proof}
				כדי להראות שקיימים $\binom{2n}{n}$ עלים בעץ, באופן שקול נוכיח שבהינתן מערכים $(a_i)_{i = 1}^{n}$ ו־$(b_i)_{i = 1}^{n}$ כלשהם, ישנם $ \binom{2n}{n}$ פלטים אפשריים. נתבונן ב־$(a_i)_{i = 1}^{n}$ – הסדר שלהם מקובע, ובהכרח נצטרך בפלט לשבץ את איברי $(b_i)_{i = 1}^{n}$ בינהם. יש $n + 1$ מקומות בינהם, והבעיה שקולה קומבינטורית לחלוקת $n$ איברים ל־$n + 1$ דליים, מה שמותיר אותנו עם $S(n + 1, n) = \binom{n + 1 + n - 1}{n} = \binom{2n}{n}$ (סימון מקורס בדידה 2) אפשרויות כדרוש. מספר עלי העץ כמספר הפלטים האפשריים מהגדרה. 
				\end{proof}
			\item נסיק חסם תחתון לגבי זמן הריצה. \begin{proof}
				בהרצאה ראינו למה, על־פיה בהינתן $\ml$ עלים, גובה עץ בינארי הוא בהכרח $h \ge \log \ml$. משום שעץ ההשוואות הוא בפרט עץ בינארי, מתקיים שגובהו: 
				\[ h \ge \log \ml \ge \log \binom{2n}{n} = \log \cl{(-1)^{n}4^{n}\binom{-1/2}{n}} = \log\cl{4^{n}\sof{\binom{-1/2}{n}}} > \log 4^n = n \log 4 = \Omega(n)  \]
				סה''כ הוכחנו חסם תחתון לינארי. (את הנוסחה לבינום האמצעי רואים בבדידה 2)
			\end{proof}
		\end{enumerate}
	\end{enumerate}
	
	\npage
	\section{}
	נתון מערך $A$ של $n$ מספרים שלמים. ננסה למצוא אלגו' הבודק האם קיימים $i, j \in [n]$ כך ש־$j - i = \sof{A_j - A_i}$. 
	\begin{enumerate}
		\item בסעיף זה, נניח $\forall i \in [n] \co A_i \in [-n^2, n^2] \cap \Z$. \begin{proof}[פתרון.]
			נספק אלגו' דטרמינסטי בסיבוכיות לינארית. ניצור מילון הממומש באמצעות vector בגודל $n$, כשערכיו $i \mapsto A_i$. 
			נבחין ש־: 
			\[ j -i = \sof{A_j - A_i} \iff \begin{cases}
				j - i = A_j - A_i \\
				j - i = -A_j + A_i
			\end{cases}\dequad\iff \begin{cases}
				j - A_j = i - A_i \\
				A_j + j = A_i + i
			\end{cases} \]
			לכן, נוכל לבדוק באופן שקול שוויון זוגות כמתוארים לעיל. ניצור שני מערכים: 
			\begin{align*}
				\ac_1[i] &= i - A_i \\
				\ac_2[i] &= i + A_i
			\end{align*}
			קיום כפילות ב־$\ac_1$ או ב־$\ac_2$, שקול לקיום $i \neq j \co A_j + j = A_i + i \lor A_i - i = A_j - i$, או בניסוח שקול, $j - i = \sof{A_j - A_i}$, שזה מה שאנו רוצים לבדוק. לשם בדיקת הכפילויות, נמיין את $\ac_1, \ac_2$ באמצעות radix sort על איבריהם בבסיס $n$. משום שהערכים בטווח $-n^3 < -n^2 -n \cdots n^2 + n < n^3$, ומתקיים שכמות הספרות של $n^3$ בבסיס $n$ היא $\log_n n^3 = 3\log_nn = 3$ (קבועה), נצטרך להריץ את count sort הלינארי כמות קבועה של פעמים ולכן ה־radix sort יפעל בסיבוכיות לינארית. לאחר המיון, נותר לבדוק האם קיימים שני איברים עוקבים זהים ב־$\ac_1, \ac_2$, כלומר לעבור עליהם $n - 1 = \oc(n)$ פעמים, ולבדוק האם $\ac_i[j] = \ac_[j + 1]$ (עבור $i \in \{1, 2\}, j \in [n - 1]$).
			
			משום שביצעו פעולות לינאריות בלבד, בסידור, השוואה, ויצירת $\ac_1, \ac_2$, סיבוכיות האלגו' שלנו לינארית דטרמינסטית במקרה הגרוע. 
		\end{proof}
		\item עתה, נוותר על ההנחה, ונפתור בתוחלת. \begin{proof}[פתרון.]
			נעזר בשיטה דומה לזו של הסעיף הקודם, כלומר, נוכל לבנות טבלת hash בגודל $n$ בשיטת ה־chaining, ולשמור בה את הזוגות הסדורים $(i, A_i + i)$ (יש $n$ כאלו ולכן בתוחלת יקח $\oc(n)$ לשמור אותם) כאשר ה־hash מקבל כקלט את $A_i + i$. כדי למצוא איברים מגודל זהה, נעבור ב־$\oc(n)$ על כל המספרים ונבדוק האם בתת־הרשימה המקושרת בתא ה־$h(A_i + i)$ יש עוד איבר מלבדם (בתוחלת יהיה שם $\oc(1)$ איברים, כלומר המעבר יארך $n \cdot \oc(1)$) בעל אותו הערך. באופן דומה נוכל לשמור מערך עם הערכים $(i, i - A_i)$ וב־$\oc(n)$ לבדוק עבור כל $j$ האם קיים אחר מלבדו בתא ה־$h(i - A_i)$ בעל אותו הערך. 
			
			סה''כ סיבוכיות $\oc(n)$ בתוחלת, וטבלת ה־hash מגודל לינארי. 
		\end{proof}
	\end{enumerate}
	
	
	
	
	\ndoc
\end{document}