\documentclass[]{article}

% Math packages
\usepackage[usenames]{color}
\usepackage{forest}
\usepackage{ifxetex,ifluatex,amsmath,amssymb,mathrsfs,amsthm,witharrows}
\WithArrowsOptions{displaystyle}
\renewcommand{\qedsymbol}{$\blacksquare$} % end proofs with \blacksquare. Overwrites the defualts. 
\usepackage{cancel,bm}

% Deisgn
\usepackage[labelfont=bf]{caption}
\usepackage[legalpaper, margin=0.5in]{geometry}
\usepackage[skip=4pt, indent=0pt]{parskip}
\usepackage[normalem]{ulem}
\forestset{default preamble={for tree={circle, draw}}}
\renewcommand\labelitemi{$\bullet$}

% Hebrew initialzing
\usepackage{polyglossia}
\setmainlanguage{hebrew}
\setotherlanguage{english}
\newfontfamily\hebrewfont[Script=Hebrew]{David CLM}

% Math shortcuts

\newcommand\N     {\mathbb{N}}
\newcommand\Z     {\mathbb{Z}}
\newcommand\R     {\mathbb{R}}
\newcommand\Q     {\mathbb{Q}}

\newcommand\ml    {\ell}
\newcommand\mj    {\jmath}
\newcommand\mi    {\imath}

\newcommand\powerset {\mathcal{P}}
\newcommand\ps    {\mathcal{P}}
\newcommand\pc    {\mathcal{P}}
\newcommand\ac    {\mathcal{A}}
\newcommand\bc    {\mathcal{B}}
\newcommand\cc    {\mathcal{C}}
\newcommand\dc    {\mathcal{D}}
\newcommand\ec    {\mathcal{E}}
\newcommand\fc    {\mathcal{F}}
\newcommand\nc    {\mathcal{N}}

\newcommand\siff  {\longleftrightarrow}
\newcommand\ssiff {\leftrightarrow}
\newcommand\so    {\longrightarrow}
\newcommand\sso   {\rightarrow}

\newcommand\epsi  {\epsilon}
\newcommand\vepsi {\varepsilon}
\newcommand\vphi  {\varphi}
\newcommand\Neven {\N_{\mathrm{even}}}
\newcommand\Nodd  {\N_{\mathrm{odd }}}
\newcommand\Zeven {\Z_{\mathrm{even}}}
\newcommand\Zodd  {\Z_{\mathrm{odd }}}
\newcommand\Np    {\N_+}

\newcommand\open  {\big(}
\newcommand\qopen {\quad\big(}
\newcommand\close {\big)}
\newcommand\also  {\text{, }}
\newcommand\defi  {\text{ definition}}
\newcommand\defis {\text{ definitions}}
\newcommand\given {\text{given }}
\newcommand\case  {\text{if }}
\newcommand\syx   {\text{ syntax}}
\newcommand\rle   {\text{ rule}}
\newcommand\other {\text{else}}
\newcommand\set   {\ell et \text{ }}

\newcommand\ra    {\rangle}
\newcommand\la    {\langle}

\newcommand\oto   {\leftarrow}

\newcommand\QED   {\quad\quad\mathscr{Q.E.D.}\;\;\blacksquare}
\newcommand\QEF   {\quad\quad\mathscr{Q.E.F.}}
\newcommand\eQED  {\mathscr{Q.E.D.}\;\;\blacksquare}
\newcommand\eQEF  {\mathscr{Q.E.F.}}
\newcommand\jQED  {\mathscr{Q.E.D.}}

\newcommand\dom   {\text{dom}}
\newcommand\Img   {\text{Im}}
\newcommand\range {\text{range}}

\newcommand\trio  {\triangle}

\newcommand\rc    {\right\rceil}
\newcommand\lc    {\left\lceil}
\newcommand\rf    {\right\rfloor}
\newcommand\lf    {\left\lfloor}

\newcommand\lex   {<_{lex}}

\newcommand\bs    {\blacksquare}

\newcommand\az    {\aleph_0}
\newcommand\taz   {2^{\aleph_0}}
\newcommand\al    {\aleph}

\newcommand\n     {$n$־יה\ }

\newcommand\logn  {\log n}

\newcommand\en[1] {\selectlanguage{english}#1\selectlanguage{hebrew}}
\newcommand\del   {$ \!\! $}

\newcommand\seq   {\overset{!}{=}}
\newcommand\sle   {\overset{!}{\le}}
\newcommand\sge   {\overset{!}{\ge}}
\newcommand\sll   {\overset{!}{<}}
\newcommand\sgg   {\overset{!}{>}}

\newcommand\p     {\text{, }}
\newcommand\ttt[1]{\en{\texttt{#1}}}
\newcommand\tl[1] {\tilde{#1}}

\title{עוצמות 6}
\author{שחר פרץ}
\date{13 למרץ 2024}

\begin{document}
	\maketitle
	\section*{עוד על חשבון עוצמות}
	תזכורות: בעבור $ A, B $ קבוצות הגדרנו: 
	\begin{itemize}
		\item $ |A| + |B| = |A \times \{0\}| \uplus |B \times \{1\}| $ (הערה: במידה ונתונה לנו רק אחת מהקבוצות, מותר לבחור את הקבוצה השניה כך שהן יהיו זרות)
		\item $ |A| \cdot |B| = |A \times B| $
		\item $ |A|^{|B|} = |B \to A| = |A^B| $ (כי $ A \to B := A^B $)
		\item על תמציאו $ -, \tfrac{|A|}{|B|} $ כי זה לא מוגדר. 
	\end{itemize}
	\textbf{משפט (מונוטוניות): }יהיו $ a, b, c, d $ עוצמות כך ש־$ c \le d \land a \le b $, לכן: 
	\begin{enumerate}
		\item $ a + c \le b + d $
		\item $ a \cdot c \le b \cdot d $
		\item $ a^c \le a^d $ (אלא אם $ a = c = 0 \land d \neq 0 $)
		\item $ a^c \le b^c $
	\end{enumerate}
	\begin{proof}
		נוכיח את הטענה השלישית. יהיו $ A, C, D $ קבוצות כך ש־$ |A| = a, \ |C| = c, \ |D| = d $. נתון $ c \le d $ ולכן קימת $ g \colon C \to D $ חח"ע. נרצה להגדיר $ \vphi \colon (C \to A) \to (D \to A) $ חח"ע. נגדיר $ \tl{g} \colon D \to C $ [ללא שימוש ב־AC – בחירה סופית]: יהי $ \tl{c} \in C $ קבוע (נניח $ C \neq \emptyset $): 
	\[ \tl{g} = \lambda d \in D. \begin{cases}
		\iota x \in C. g(x) = d, & d \in \Img g \\
		\tl{c} & d \in D \setminus \Img g
	\end{cases} \]
	$ \tl{g} $מוגדרת היטב מאחר ו־ $ g $ חח"ע. מתקיים: $ \tl{g} \circ g = id_C $. את $ \vphi = \lambda h \in C \to A. h \circ \tl{g} $. נוכיח $ \vphi $ חח"ע: יהיו $ h_1, h_2 \in C \to A $ כך ש־$ \vphi(h_1) = \vphi(h_2) $. לכן $ h_1 \circ \tl{g} = h)2 \circ \tl{g} $. אם נרכיב על $ g $, נקבל $ (h \circ \tl{g}) \circ g = (h_2 \circ \tl{g}) \circ g $ ומאסוציאטיביות $ h_1 \circ ( \tl{g} \circ g) = h_2 \circ(\tl{g} \circ g) $ ומהזהות לעיל $ h_1 \circ id_C = h_2 \circ id_C $ כלומר $ h_1 = h_2 $ כדרוש. \\
	נחזור למקרה בו $ C = \emptyset $. אם לכן $ c = 0 $. נפריד למקרים: 
	\begin{itemize}
		\item אם $ a \neq 0 $: אז $ a^c = a^0 = 1 $ וכן $ a^d $ אם $ d = 0 $ tz $ a^d = a^0 = 1 $ ואז $ a^c = a^d $. אם $ d \neq 0 $ אז $ a^d $. אם $ d \neq 0 $ אז $ a^d = |D \to A| \ge 1 $, כלומר $ ^d \le a^c = 1 $ כדרוש. 
		\item אם $ a = 0 $ אז $ a^c = 0^0 = 1 $, במקרה זה הנחנו $ d = 0 $ ולכן $ a^d = 0^0 = a^c $
	\end{itemize}
	כיסינו את כל המקרים. 
	\end{proof}
	\textbf{משפט: }("סופר־שימושי"): 
	\begin{enumerate}
		\item $ \az \cdot \az = \az + \az  = \az $ (כבר הוכחנו את השוויון לגבי הכפל כשהוכחנו $ |\N\times\N| = \az $)
		\item $ \taz \cdot \taz = \taz + \taz = \taz $ (גם כאן הוכחנו את הכפל בשיעור שעבר)
		\item $ \az \cdot \taz = \az + \taz = \taz $
	\end{enumerate}
	(הטענות האלו נכונות עבור כל עוצמה אינסופית תחת אקסיומת בחירה, אבל אסור לנו להניח את זה ואין משפט כזה בקורס)
	\begin{proof} נוכיח את שלושת הטענות: \\
	(1): עבור כפל כבר הוכחנו, עבור חיבור נשתמש בקש"ב. מהמונוטוניות: $ \az \le \az + 0 \le \az + \az $ ומצד שני $ \az + \az = |\Nodd \uplus \Neven| = |\N| = \az $ כדרוש. \\
	(2): עבור כפל כבר הוכחנו (שיעור שעבר הוכחנו כי $ |\R\times\R| = |\R| $), ובעבור $ \taz + \taz $ נוכיח באמצעות קש"ב, $ \taz + \taz \ge \taz + 0 = \taz $ ומהצד השני $ \taz + \taz = |\underbrace{\R \times \{0\} \uplus \R\times \{1\}}_{\subseteq \R\times\R}| \le |\R\times\R| = \taz$ כדרוש. [אפשר גם להוכיח עם קרנות]. \\
	(3): תזכורת; היה תרגיל בשיעורי הבית להוכיח כי $ |[0, 1) \times \Z| = |\R| $ ולכן $ \taz \cdot \az = \taz $. אפשר גם להוכיח ע"י חשבון עוצמות: $ \az \times \taz \ge 1 \cdot \taz = \taz $, ומהצד השני $ \az \times \taz \ge \taz  \cdot \taz = \taz $, וסה"כ הוכחנו מקש"ב $ \az \times \taz = \taz $ כדרוש. בעבור $ \az + \taz $ ידוע $ \az + \taz \ge 0+ \taz = \taz \land \az + \taz \le \taz + \taz = \taz $ וסה"כ מקש"ב $ \az + \taz = \taz $ כדרוש. 
	\end{proof}
	
	\textbf{משפט (חוקי חזקות): }
	\begin{enumerate}
		\item $ (a^b)^c = a^{b \cdot c} $
		\item $ (a \cdot b)^c = a^c \cdot b^c $
		\item $ a^{b + c} = a^b \cdot a^c $
	\end{enumerate}
	\begin{proof}
		יהיו $ A, B, C, D $ קבוצות ונסמן $ a = |A|, b = |B|, c = |C|, d = |D| $. כדי להוכיח את (1) צ.ל. קיום זיווג $ \vphi \colon ((B \times C) \to A) \to (C \to (B \to A)) $ והרי שפונקציית $ Cu := curry $ (קורי) היא זיווג\dots \ וזהו כאן די נגמרים השימושים של הפונקציה הזו לקורס הזה, סתם חפרו לנו עליה לפני חצי שנה. 
	\end{proof}
	\textbf{טענה: }
	\begin{enumerate}
		\item לכל עוצמה אינסופית $ a $ מתקיים $ a + \az = a $
		\item לכל עוצמה אינסופית $ a $ ולכל $ n \in \N $, מתקיים $ a + n = a $
	\end{enumerate}
	\begin{proof}
		("אה רגע יש לי שתי דקות, אז בו נוכיח"). נוכיח את (1). תהי $ A $ קבוצה כך ש־$ |A| = a $ ומאחר ש־$ A $ אינסופית, אזי קיימת לה ת"ק [=תת קבוצה] בת מנייה $ B \subseteq A $. 
		\begin{align*}
			a + \az = |A| + |B| = |(A \setminus B) \uplus (B)| + |B| = |A \setminus B| + \underbrace{|B|}_{\az} + \underbrace{|B|}_{\az} = |A \setminus B| + \az = |(A \setminus B) \uplus B| = |A| = a
		\end{align*}
	\end{proof}
\end{document}