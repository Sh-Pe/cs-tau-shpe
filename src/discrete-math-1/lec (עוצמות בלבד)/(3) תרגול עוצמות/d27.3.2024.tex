\documentclass[]{article}

% Math packages
\usepackage[usenames]{color}
\usepackage{forest}
\usepackage{ifxetex,ifluatex,amsmath,amssymb,mathrsfs,amsthm,witharrows}
\WithArrowsOptions{displaystyle}
\renewcommand{\qedsymbol}{$\blacksquare$} % end proofs with \blacksquare. Overwrites the defualts. 
\usepackage{cancel,bm}

% Deisgn
\usepackage[labelfont=bf]{caption}
\usepackage[A4, margin=0.5in]{geometry}
\usepackage{multicol}
\usepackage[skip=4pt, indent=0pt]{parskip}
\usepackage[normalem]{ulem}
\forestset{default preamble={for tree={circle, draw}}}
\renewcommand\labelitemi{$\bullet$}

% Hebrew initialzing
\usepackage{polyglossia}
\setmainlanguage{hebrew}
\setotherlanguage{english}
\newfontfamily\hebrewfont[Script=Hebrew, Ligatures=TeX]{David CLM}
\usepackage[shortlabels]{enumitem}
\newlist{hebenum}{enumerate}{1}
\setlist[hebenum,1]{
	labelindent=\parindent,
	label={{\hebrewfont{\protect\hebrewnumeral{\value{hebenumi}}}}.}
}

% Math shortcuts

\newcommand\N     {\mathbb{N}}
\newcommand\Z     {\mathbb{Z}}
\newcommand\R     {\mathbb{R}}
\newcommand\Q     {\mathbb{Q}}

\newcommand\ml    {\ell}
\newcommand\mj    {\jmath}
\newcommand\mi    {\imath}

\newcommand\powerset {\mathcal{P}}
\newcommand\ps    {\mathcal{P}}
\newcommand\pc    {\mathcal{P}}
\newcommand\ac    {\mathcal{A}}
\newcommand\bc    {\mathcal{B}}
\newcommand\cc    {\mathcal{C}}
\newcommand\dc    {\mathcal{D}}
\newcommand\ec    {\mathcal{E}}
\newcommand\fc    {\mathcal{F}}
\newcommand\nc    {\mathcal{N}}
\newcommand\sca   {\mathcal{S}} % \sc is already definded
\newcommand\rca   {\mathcal{R}} % \rc is already definded

\newcommand\siff  {\longleftrightarrow}
\newcommand\ssiff {\leftrightarrow}
\newcommand\so    {\longrightarrow}
\newcommand\sso   {\rightarrow}

\newcommand\epsi  {\epsilon}
\newcommand\vepsi {\varepsilon}
\newcommand\vphi  {\varphi}
\newcommand\Neven {\N_{\mathrm{even}}}
\newcommand\Nodd  {\N_{\mathrm{odd }}}
\newcommand\Zeven {\Z_{\mathrm{even}}}
\newcommand\Zodd  {\Z_{\mathrm{odd }}}
\newcommand\Np    {\N_+}

\newcommand\open  {\big(}
\newcommand\qopen {\quad\big(}
\newcommand\close {\big)}
\newcommand\also  {\text{, }}
\newcommand\defi  {\text{ definition}}
\newcommand\defis {\text{ definitions}}
\newcommand\given {\text{given }}
\newcommand\case  {\text{if }}
\newcommand\syx   {\text{ syntax}}
\newcommand\rle   {\text{ rule}}
\newcommand\other {\text{else}}
\newcommand\set   {\ell et \text{ }}

\newcommand\ra    {\rangle}
\newcommand\la    {\langle}

\newcommand\oto   {\leftarrow}

\newcommand\QED   {\quad\quad\mathscr{Q.E.D.}\;\;\blacksquare}
\newcommand\QEF   {\quad\quad\mathscr{Q.E.F.}}
\newcommand\eQED  {\mathscr{Q.E.D.}\;\;\blacksquare}
\newcommand\eQEF  {\mathscr{Q.E.F.}}
\newcommand\jQED  {\mathscr{Q.E.D.}}

\newcommand\dom   {\text{dom}}
\newcommand\Img   {\text{Im}}
\newcommand\range {\text{range}}

\newcommand\trio  {\triangle}

\newcommand\rc    {\right\rceil}
\newcommand\lc    {\left\lceil}
\newcommand\rf    {\right\rfloor}
\newcommand\lf    {\left\lfloor}

\newcommand\lex   {<_{lex}}

\newcommand\bs    {\blacksquare}

\newcommand\az    {\aleph_0}
\newcommand\uaz   {^{\aleph_0}}
\newcommand\al    {\aleph}
\newcommand\ual   {^\aleph}
\newcommand\taz   {2^{\aleph_0}}
\newcommand\utaz  { ^{\left (2^{\aleph_0} \right )}}
\newcommand\tal   {2^{\aleph}}
\newcommand\utal  { ^{\left (2^{\aleph} \right )}}
\newcommand\ttaz  {2^{\left (2^{\aleph_0}\right )}}

\newcommand\n     {$n$־יה\ }

\newcommand\logn  {\log n}

\newcommand\en[1] {\selectlanguage{english}#1\selectlanguage{hebrew}}
\newcommand\del   {$ \!\! $}

\newcommand\seq   {\overset{!}{=}}
\newcommand\sle   {\overset{!}{\le}}
\newcommand\sge   {\overset{!}{\ge}}
\newcommand\sll   {\overset{!}{<}}
\newcommand\sgg   {\overset{!}{>}}

\newcommand\p     {\text{, }}
\newcommand\ttt[1]{\en{\texttt{#1}}}

\newcommand\tl    {\tilde}
\newcommand\op    {^{-1}}

\newcommand\h     {\hat}
\newcommand\ve    {\vec}
\newcommand\lv    {\overrightarrow}

\title{עוצמות 8}
\author{שחר פרץ}
\date{27 במרץ 2024}

\begin{document}
	\maketitle
	
	\section{דוגמאות מתרגילים מש.ב. 9}
	\subsection{תרגיל 9}
	הגדרה: נאמר ש־$ f, g \colon \N \to \R $ הן "כמעט מסכימות" אם קיים $ i $ כך שלכל $ j \ge i $ מתקיים $ f(j) = g(j) $. [נסיק: $ i \in \N $]. 
	נגדיר: 
	\[ R = \{\la f, g \ra \in (\R^\N)^2 \mid f, g \text{agrees \ almost } \} \]
	\begin{enumerate}[(a)]
		\item אמור להיות קל
		\item מצאו את העוצמה של כל מחלקת שקילות; תהי $ f \in \N \to \R $. נחשב את $ |[f]_R| $. עוזר לאינטואיציה, ולרוב גם להוכחה: מצד אחד, $ [f]_R \subseteq \R^\N $ ולכן $ |[f]_R| \le |\R^\N| = (\taz)\uaz = \taz =\al $. מהכיוון השני: נגדיר $ G \colon \R\to [f]_R $ ע"י: 
		\[ g = \lambda r \in \R. \lambda x \in \N. \begin{cases}
			r, & n = 0 \\
			f(n), & n \ge 1
		\end{cases} \]
		צריך להוכיח שהיא מוגדרת היטב (מגיעה לטווח מתאים: $ \forall r \in \R. g(r) \in [f]_R $) ושהיא חח"ע. כרגע לא נוכיח זאת כאן [אני אנצל את זה בשביל להגיד שיש לי הוכחה פורמלית על הפונקציה הזו בדיוק בענן, שאמורה להיות \textit{יחסית} נכונה]. 
		
		מקיום הפונקציה, $ \al = |\R| \le |[f]_R| $. סה"כ מקש"ב $ |[f]_R| = \al $. 
		\item מצאו את העוצמה של קבוצת המנה; $ |(\N \to \R) / R| = ? $. טעות נפוצה הייתה היא, להתבונן בחסם העליון של ההכלה ולהניח כי הוא החסם העליון. במקרה הזה, $ (\N \to \R) / R \subseteq \ps(\ps(\N \to \R)) $, ומכאן נובע החסם העליון $ 2\utal $ (הערה: מותר גם בלי סוגריים והכוונה היא לזה). החסם העליון הזה לעוצמה הוא לא ההדוק ביותר! ראוי לציון שלכל יחס שקילות $ R $ מעל $ A $ תמקיים $ |A / R| \le |A| $ – אומנם אסור להשתמש בזה כמשפט, אבל הוא מאפשר להקטין את החסם יותר כשמדברים על מחלקות שקילות. אין זה משפט, אבל די לנמקו בקצרה: $ A / R = \{[x]_R \mid x \in A\} $. ניקח $ A' \subseteq A $ מערכת נציגים ליחס $ R $ (AC)\ \del . כעת $ \lambda a \in A'. [a]_R $ היא זיווג $ A' \to A / R $ ולכן $ |A/R| = |A'| \le |A| $ (אין צורך להוכיח מעבר לכך). \textbf{זו הדרך הנוחה למצוא חסם עליון לקבוצות מנה}. 
		
		דיברנו די; נחזור להוכחה. מהטענה לעיל, $ |(\N \to \R) / R| \le |\N \to \R| = (\taz)\uaz = \taz = \al $. בעבור הכיוון השני: נתבונן בכל הפונקציות הקבועות האפשרויות, שברור כי הן אינן מסכימות (נידרש להוכיח זאת בהמשך). נגדיר $ F \colon \R\to (\N \to \R) / R $ ע"י: 
		\[ F = \lambda r \in \R. [\lambda x \in \N. r]_R \]
		אין צורך להוכיח ש־$ F $ מוגדרת היטב כי אנחנו מחזירים מחלקת שקילות של $ R $ וזה די ישיר. יש צורך להוכיח שהיא חח"ע (כלומר $ \forall r_1 \neq r_2 \in \R. \la x \in \N. r_1, \lambda x \in \N. r_2 \ra \not \in R $). 
		[הערה: אם אתם תחפשו את ההוכחה המוגמרת אצלי, תמצאו בעיקר חרא, אז אל תעשו את זה. כל שאר התרגילים חוץ מהסעיף הזה כתובים טוב]
	\end{enumerate}
	\subsection{סעיף 10)}
	נגדיר: 
	\[ S = \{\la A< B \ra \in \ps(\Z) \times \ps(\Z) \colon |A| = |B| = |A \cup B| \} \]
	\begin{enumerate}
		\item $ [\N]_S = \{X \in \ps(\Z) \mid |X| = \az\}, \ [\{2, 3\}]_S = \big \{\{2, 3\}\big \} $. בהכללה, אפשר להכיח אשר $ \forall A \ \text{finite}. [A]_S = \{A\}, \ \forall A \in \ps(\Z) \ \text{infinite}. [A]_S = [\N]_S $. 
		\item מהי קבוצת המנה? מהי עוצמתה? קבוצת המנה: 
		\[ \ps(\Z) / S = \{[A]_S \mid A \in \ps(\Z)\} = \big\{ \{A\} \mid A \in \underbrace{\{X \in \ps(\Z) \mid |X| < \az\}}_{:=\ps_{\textrm{finite}}(\Z)} \big\} \uplus \big\{ \underbrace{\{X \in \ps(\Z) \mid |X| = \az\}}_{[\N]_S}\big\} \]. מכאן: $ |\ps(\Z) / S| = |\ps_{\textrm{finite}}(\Z)| + 1 $. טענה: $ |\ps_{\textrm{finite}}(\Z)| = \az $ (כדאי לזכור את זה). נוכיח זאת. $ \ps_{\textrm{finite}} \subseteq \bigcup_{n \in \N}\underbrace{\ps(\underbrace{\{-n, \dots, n\}}_{2n + 1})}_{2^{2n + 1}\implies \ \text{ finite}} $
		סה"כ איחוד בן מניה של קבוצות סופיות, הוא לכל היותר בן מניה. מצד שני, $ \lambda n \in \N. \{n\} $ היא חח"ע $ \Z\to \ps_{\textrm{finite}}(\Z) $, לכן $ |\ps_{\textrm{finite}}(\Z)| \ge \az $. סה"כ שוויון מקש"ב. סיימנו: $ |\ps(\Z) / S| = \az + 1 = \az $ כדרוש. 
	\end{enumerate}
\end{document}