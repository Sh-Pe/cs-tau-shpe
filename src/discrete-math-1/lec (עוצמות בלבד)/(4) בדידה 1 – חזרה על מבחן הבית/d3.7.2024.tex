%! ~~~ Packages Setup ~~~ 
\documentclass[]{article}


% Math packages
\usepackage[usenames]{color}
\usepackage{forest}
\usepackage{ifxetex,ifluatex,amsmath,amssymb,mathrsfs,amsthm,witharrows,mathtools}
\WithArrowsOptions{displaystyle}
\renewcommand{\qedsymbol}{$\blacksquare$} % end proofs with \blacksquare. Overwrites the defualts. 
\usepackage{cancel,bm}
\usepackage[thinc]{esdiff}


% tikz
\usepackage{tikz}
\newcommand\sqw{1}
\newcommand\squ[4][1]{\fill[#4] (#2*\sqw,#3*\sqw) rectangle +(#1*\sqw,#1*\sqw);}


% code 
\usepackage{listings}
\usepackage{xcolor}

\definecolor{codegreen}{rgb}{0,0.35,0}
\definecolor{codegray}{rgb}{0.5,0.5,0.5}
\definecolor{codenumber}{rgb}{0.1,0.3,0.5}
\definecolor{codeblue}{rgb}{0,0,0.5}
\definecolor{codered}{rgb}{0.5,0.03,0.02}
\definecolor{codegray}{rgb}{0.96,0.96,0.96}

\lstdefinestyle{pythonstylesheet}{
	language=Python,
	emphstyle=\color{deepred},
	backgroundcolor=\color{codegray},
	keywordstyle=\color{deepblue}\bfseries\itshape,
	numberstyle=\scriptsize\color{codenumber},
	basicstyle=\ttfamily\footnotesize,
	commentstyle=\color{codegreen}\itshape,
	breakatwhitespace=false, 
	breaklines=true, 
	captionpos=b, 
	keepspaces=true, 
	numbers=left, 
	numbersep=5pt, 
	showspaces=false,                
	showstringspaces=false,
	showtabs=false, 
	tabsize=4, 
	morekeywords={as,assert,nonlocal,with,yield,self,True,False,None,AssertionError,ValueError,in,else},              % Add keywords here
	keywordstyle=\color{codeblue},
	emph={object,type,isinstance,copy,deepcopy,zip,enumerate,reversed,list,set,len,dict,tuple,print,range,xrange,append,execfile,real,imag,reduce,str,repr,__init__,__add__,__mul__,__div__,__sub__,__call__,__getitem__,__setitem__,__eq__,__ne__,__nonzero__,__rmul__,__radd__,__repr__,__str__,__get__,__truediv__,__pow__,__name__,__future__,__all__,},          % Custom highlighting
	emphstyle=\color{codered},
	stringstyle=\color{codegreen},
	showstringspaces=false,
	abovecaptionskip=0pt,belowcaptionskip =0pt,
	framextopmargin=-\topsep, 
}
\newcommand\pythonstyle{\lstset{pythonstylesheet}}
\newcommand\pyl[1]     {{\lstinline!#1!}}
\lstset{style=pythonstylesheet}

\usepackage[style=1,skipbelow=\topskip,skipabove=\topskip,framemethod=TikZ]{mdframed}
\definecolor{bggray}{rgb}{0.85, 0.85, 0.85}
\mdfsetup{leftmargin=0pt,rightmargin=0pt,innerleftmargin=15pt,backgroundcolor=codegray,middlelinewidth=0.5pt,skipabove=5pt,skipbelow=0pt,middlelinecolor=black,roundcorner=5}
\BeforeBeginEnvironment{lstlisting}{\begin{mdframed}\vspace{-0.4em}}
	\AfterEndEnvironment{lstlisting}{\vspace{-0.8em}\end{mdframed}}


% Deisgn
\usepackage[labelfont=bf]{caption}
\usepackage[margin=0.6in]{geometry}
\usepackage{multicol}
\usepackage[skip=4pt, indent=0pt]{parskip}
\usepackage[normalem]{ulem}
\forestset{default}
\renewcommand\labelitemi{$\bullet$}
\usepackage{titlesec}
\titleformat{\section}[block]
{\fontsize{15}{15}}
{\sen \dotfill (\thesection) \she}
{0em}
{\MakeUppercase}
\usepackage{graphicx}
\graphicspath{ {./} }


% Hebrew initialzing
\usepackage[bidi=basic]{babel}
\PassOptionsToPackage{no-math}{fontspec}
\babelprovide[main, import]{hebrew}
\babelprovide[import]{english}
\babelfont[hebrew]{rm}{David CLM}
\babelfont[hebrew]{sf}{David CLM}
\babelfont[english]{tt}{Monaspace Xenon}
\usepackage[shortlabels]{enumitem}
\newlist{hebenum}{enumerate}{1}

% Language Shortcuts
\newcommand\en[1] {\begin{otherlanguage}{english}#1\end{otherlanguage}}
\newcommand\sen   {\begin{otherlanguage}{english}}
	\newcommand\she   {\end{otherlanguage}}
\newcommand\del   {$ \!\! $}
\newcommand\ttt[1]{\en{\footnotesize\texttt{#1}\normalsize}}

\newcommand\npage {\vfil {\hfil \textbf{\textit{המשך בעמוד הבא}}} \hfil \vfil \pagebreak}
\newcommand\ndoc  {\dotfill \\ \vfil {\begin{center} {\textbf{\textit{שחר פרץ, 2024}} \\ \scriptsize \textit{נוצר באמצעות תוכנה חופשית בלבד}} \end{center}} \vfil	}

\newcommand{\rn}[1]{
	\textup{\uppercase\expandafter{\romannumeral#1}}
}

\makeatletter
\newcommand{\skipitems}[1]{
	\addtocounter{\@enumctr}{#1}
}
\makeatother

%! ~~~ Math shortcuts ~~~

% Letters shortcuts
\newcommand\N     {\mathbb{N}}
\newcommand\Z     {\mathbb{Z}}
\newcommand\R     {\mathbb{R}}
\newcommand\Q     {\mathbb{Q}}
\newcommand\C     {\mathbb{C}}
\renewcommand\P   {\mathbb{P}}

\newcommand\ml    {\ell}
\newcommand\mj    {\jmath}
\newcommand\mi    {\imath}

\newcommand\powerset {\mathcal{P}}
\newcommand\ps    {\mathcal{P}}
\newcommand\pc    {\mathcal{P}}
\newcommand\ac    {\mathcal{A}}
\newcommand\bc    {\mathcal{B}}
\newcommand\cc    {\mathcal{C}}
\newcommand\dc    {\mathcal{D}}
\newcommand\ec    {\mathcal{E}}
\newcommand\fc    {\mathcal{F}}
\newcommand\nc    {\mathcal{N}}
\newcommand\sca   {\mathcal{S}} % \sc is already definded
\newcommand\rca   {\mathcal{R}} % \rc is already definded

\newcommand\Si    {\Sigma}

% Logic & sets shorcuts
\newcommand\siff  {\longleftrightarrow}
\newcommand\ssiff {\leftrightarrow}
\newcommand\so    {\longrightarrow}
\newcommand\sso   {\rightarrow}

\newcommand\epsi  {\epsilon}
\newcommand\vepsi {\varepsilon}
\newcommand\vphi  {\varphi}
\newcommand\Neven {\N_{\mathrm{even}}}
\newcommand\Nodd  {\N_{\mathrm{odd }}}
\newcommand\Zeven {\Z_{\mathrm{even}}}
\newcommand\Zodd  {\Z_{\mathrm{odd }}}
\newcommand\Np    {\N_+}

% Text Shortcuts
\newcommand\open  {\big(}
\newcommand\qopen {\quad\big(}
\newcommand\close {\big)}
\newcommand\also  {\text{, }}
\newcommand\defi  {\text{ definition}}
\newcommand\defis {\text{ definitions}}
\newcommand\given {\text{given }}
\newcommand\case  {\text{if }}
\newcommand\syx   {\text{ syntax}}
\newcommand\rle   {\text{ rule}}
\newcommand\other {\text{else}}
\newcommand\set   {\ell et \text{ }}
\newcommand\ans   {\mathit{Ans.}}

% Set theory shortcuts
\newcommand\ra    {\rangle}
\newcommand\la    {\langle}

\newcommand\oto   {\leftarrow}

\newcommand\QED   {\quad\quad\mathscr{Q.E.D.}\;\;\blacksquare}
\newcommand\QEF   {\quad\quad\mathscr{Q.E.F.}}
\newcommand\eQED  {\mathscr{Q.E.D.}\;\;\blacksquare}
\newcommand\eQEF  {\mathscr{Q.E.F.}}
\newcommand\jQED  {\mathscr{Q.E.D.}}

\newcommand\dom   {\text{dom}}
\newcommand\Img   {\text{Im}}
\newcommand\range {\text{range}}

\newcommand\trio  {\triangle}

\newcommand\rc    {\right\rceil}
\newcommand\lc    {\left\lceil}
\newcommand\rf    {\right\rfloor}
\newcommand\lf    {\left\lfloor}

\newcommand\lex   {<_{lex}}

\newcommand\az    {\aleph_0}
\newcommand\uaz   {^{\aleph_0}}
\newcommand\al    {\aleph}
\newcommand\ual   {^\aleph}
\newcommand\taz   {2^{\aleph_0}}
\newcommand\utaz  { ^{\left (2^{\aleph_0} \right )}}
\newcommand\tal   {2^{\aleph}}
\newcommand\utal  { ^{\left (2^{\aleph} \right )}}
\newcommand\ttaz  {2^{\left (2^{\aleph_0}\right )}}

\newcommand\n     {$n$־יה\ }

% Math A&B shortcuts
\newcommand\logn  {\log n}
\newcommand\cosx  {\cos x}
\newcommand\cost  {\cos \theta}
\newcommand\sinx  {\sin x}
\newcommand\sint  {\sin \theta}
\newcommand\tanx  {\tan x}
\newcommand\tant  {\tan \theta}
\newcommand\sex   {\sec x}
\newcommand\sect  {\sec^2}
\newcommand\cotx  {\cot x}
\newcommand\cscx  {\csc x}
\newcommand\sinhx {\sinh x}
\newcommand\coshx {\cosh x}
\newcommand\tanhx {\tanh x}

\newcommand\seq   {\overset{!}{=}}
\newcommand\sle   {\overset{!}{\le}}
\newcommand\sge   {\overset{!}{\ge}}
\newcommand\sll   {\overset{!}{<}}
\newcommand\sgg   {\overset{!}{>}}

\newcommand\h     {\hat}
\newcommand\ve    {\vec}
\newcommand\lv    {\overrightarrow}
\newcommand\ol    {\overline}

\newcommand\mlcm  {\mathrm{lcm}}

\DeclareMathOperator{\sech}   {sech}
\DeclareMathOperator{\csch}   {csch}
\DeclareMathOperator{\arcsec} {arcsec}
\DeclareMathOperator{\arccot} {arcCot}
\DeclareMathOperator{\arccsc} {arcCsc}
\DeclareMathOperator{\arccosh}{arccosh}
\DeclareMathOperator{\arcsinh}{arcsinh}
\DeclareMathOperator{\arctanh}{arctanh}
\DeclareMathOperator{\arcsech}{arcsech}
\DeclareMathOperator{\arccsch}{arccsch}
\DeclareMathOperator{\arccoth}{arccoth} 

\newcommand\dx    {\,\mathrm{d}x}
\newcommand\dt    {\,\mathrm{d}t}
\newcommand\dtt   {\,\mathrm{d}\theta}
\newcommand\df    {\mathrm{d}f}
\newcommand\dfdx  {\diff{f}{x}}
\newcommand\dit   {\limhz \frac{f(x + h) - f(x)}{h}}

\newcommand\nt[1] {\frac{#1}{#1}}

\newcommand\limz  {\lim_{x \to 0}}
\newcommand\limxz {\lim_{x \to x_0}}
\newcommand\limi  {\lim_{x \to \infty}}
\newcommand\limh  {\lim_{x \to 0}}
\newcommand\limni {\lim_{x \to - \infty}}
\newcommand\limpmi{\lim_{x \to \pm \infty}}

\newcommand\ta    {\theta}
\newcommand\ap    {\alpha}

\renewcommand\inf {\infty}
\newcommand  \ninf{-\inf}

% Combinatorics shortcuts
\newcommand\sumnk     {\sum_{k = 0}^{n}}
\newcommand\sumni     {\sum_{i = 0}^{n}}
\newcommand\sumnko    {\sum_{k = 1}^{n}}
\newcommand\sumnio    {\sum_{i = 1}^{n}}
\newcommand\sumai     {\sum_{i = 1}^{n} A_i}
\newcommand\nsum[2]   {\reflectbox{\displaystyle\sum_{\reflectbox{\scriptsize$#1$}}^{\reflectbox{\scriptsize$#2$}}}}

\newcommand\bink      {\binom{n}{k}}
\newcommand\setn      {\{a_i\}^{2n}_{i = 1}}
\newcommand\setc[1]   {\{a_i\}^{#1}_{i = 1}}

\newcommand\cupain    {\bigcup_{i = 1}^{n} A_i}
\newcommand\cupai[1]  {\bigcup_{i = 1}^{#1} A_i}
\newcommand\cupiiai   {\bigcup_{i \in I} A_i}
\newcommand\capain    {\bigcap_{i = 1}^{n} A_i}
\newcommand\capai[1]  {\bigcap_{i = 1}^{#1} A_i}
\newcommand\capiiai   {\bigcap_{i \in I} A_i}

\newcommand\xot       {x_{1, 2}}
\newcommand\ano       {a_{n - 1}}
\newcommand\ant       {a_{n - 2}}

% Other shortcuts
\newcommand\tl    {\tilde}
\newcommand\op    {^{-1}}

\newcommand\sof[1]    {\left | #1 \right |}
\newcommand\cl [1]    {\left ( #1 \right )}
\newcommand\csb[1]    {\left [ #1 \right ]}

\newcommand\bs    {\blacksquare}

%! ~~~ Document ~~~

\author{שחר פרץ}
\title{בדידה 1 $\sim$ נטלי שלום $\sim$ חזרה על מבחן הבית}
\date{3 ליולי 2024}

\begin{document}
	\maketitle
	\section{\en{Q1}}
	\begin{enumerate}[a.]
		\item \textbf{שאלה: }נניח שהשערת הרצף נכונה. (כלומר, $\taz$ היא העוצמה הקטנה בויותר שגדולה מ־$\az$). הוכיחו/הפריכו: קיימת חלוקה $\Pi$ של $\R$ כך ש־$|\Pi| < \taz \land \forall X \in \Pi. |X| < \taz$. 
		
		\begin{proof}
			נפריך. לפי השערת הרצף, מתקיים $|\Pi| \le \az$, ולכל $X \in \Pi$ מתקיים $|X| \le \az$. אז: 
			\[ |\R| =\sof{\bigcup_{X \in \Pi} X} \le \az \]
			לפי איחוד לכל היותר בן מנייה של קבוצות לכל היותר בנות מנייה הוא לפחות בן מנייה. 
		\end{proof}
		\item\textbf{שאלה: }הוכיחו קיום חלוקה $\Pi$ של $\R$ המקיימת $|\Pi| = \taz$, ולכל $X \in \Pi, |X| = \taz$. 
		
		\begin{proof}
			ידוע $|\R| = |\R \times \R|$. לכן קיים זיווג $f \colon \R \times \R \to \R$. נגדיר לכל $r \in \R$, $A_r = f[\{r\} \times \R]$. החלוקה $\Pi = \{A_r \mid r \in \R\}$ מקיימת את הדרוש. צ.ל. $\Pi$ חלוקה, $|\Pi| = \al, \forall X \in \Pi. |X| = \al$. נוכיח ש־$\Pi$ חלוקה. 
			\begin{itemize}
				\item $\bm{\varnothing \neq \Pi}$\textbf{: }לכל $r \in \R$, נראה $A_r \neq \emptyset$. מתקיים, $\la r, 0 \ra \in \{r\} \times \R$. $f$ פונקציה ובפרט יחס מלא, ולכן קיים $f(\la r, 0 \ra) \in f[\{r\} \times \R] = A_r$. לכן $A_r$ אינו ריק. 
				\item $\bm{\bigcup \Pi = \R}$: ניעזר בהכלה דו־כיוונית. 
				\begin{itemize}
					\item[$\subseteq$] לכל $r \in \R$, $A_r = f[\{r\} \times \R]$ ולכן גם $\bigcup_{r \in \R} A_r \subseteq \R$. 
					\item[\reflectbox{$\subseteq$}] יהי $x \in \R$. צ.ל. קיום $r \in \R$ כך ש־$x \in A_r$. מכיוון ש־$F$ על, אז קיים $\la a, b \r a\in \R \times \R$ כך ש־$f(\la a, b \ra) = x$. ניקח $r := a$, אז $x = f(\la a, b \ra) \in f[\{r\} \times \R]=A_r$. 
				\end{itemize}
				\item \textbf{זרות בזוגות: }יהיו $r_1, r_2 \in \R$ שונים, נרצה להוכיח $A_{r_1} \neq A_{r_2}$. אז ניקח ש־\[x \in A_{r_1} = f[\{r_1\} \times \R] \implies \exists b_1 \in \R. x = f(\la r_1, b_1)\]
				באופן דומה, מכיוון ש־$x \in A_{r_2}$ אז נובע שקיים $b_2 \in \R$ כך ש־$x = f(\la r_2, b_2 \ra)$. כלומר, $f(\la r_1, b_1\ra) = f(\la r_1, b_1 \ra)$. מכיוון ש־$f$ זיווג, נקבל $\la r_1, b_1 \ra = \la r_2, b_r \ra$ ובפרט $r_1 = r_2$ וסתירה לז שהם שונים. 
			\end{itemize}
			עד כה, $\Pi$ חלוקה. נוכיח את הטענות שנותרו. 
			\begin{itemize}
				\item $\bm{|\Pi| = \al}$\textbf{: }הוכחנו שלכל $r_1, r_2 \in \R$ שונים, $A_{r_1}, A_{r_2}$ זרות ולא ריקות ובפרט נובע $A_{r_1} \neq A_{r_2}$ ולכן $\lambda r \in \R. A_r$ זיווג, ולכן $|\Pi| = \al$. 
				\item \textbf{לכל }$\bm{r \in \R, |A_r| = \al}$\textbf{: }לכל $r \in \R$, הפונקציה $\lambda x \in \R. f(\la r, x \ra)$ היא זיווג בין $\R$ ל־$A_r$ (כי $f$ חח"ע ועל. במבחן לפרט עוד) ולכן $|A_r| = \al$. 
			\end{itemize}
		\end{proof}
		\item שאלה ברמה קצת יותר גבוהה ממבחן. \textbf{שאלה: }נגדיר יחס שקילות מעל $\R\to \R$ ע"י: 
		\[ S = \{\la f, g \ra \in (\R \to \R)^{2} \colon \sof{\{x \in \R\mid f(x) \neq g(x)\}} \le \az\} \]
		מצאו את $|\underbrace{(\R\to \R) / S}_{B}|$
		
		\textit{רמז: }סעיף ב'. 
		
		\begin{proof}
			נוכיח שהעוצמה היא $2^{\al}$.
			\begin{itemize}
				\item[$\le$]  לכל מערכת נציגים של $S$, הפונקציה $\lambda a \in A. [a]_S$ היא זיווג. לכן, אם נקבע $a$ כזו, נקבל $|(\R \to \R)/ S| = |A| \le |\R\to \R| = \al^{\al} = (\taz)\ual = 2^{\al \cdot \az} = \tal$. 
				\item[$\ge$] נמצא פונקציה חח"ע $(\R \to \R) \to B$: 
				\[ H = \lambda g \in \R\to \R. [f_g]_S \]
				כמה הערות: (ה"מטרה": להגדיר את $f_g$ באופן כזה שלכל $g_1, g_2 \in \R \to \R$ שונות, יתקיים $|\{x \in \R\mid f_{g_1}(x) \neq f_{g_{2}}(x)| = \al\}$). נסמן $t \in X_t = [t]_{R_\Pi}$ כאשר $\Pi$ החלוקה מסעיף (ב), ו־$R_\Pi$ הוא היחס המושרה של החלוקה (תזכורת: בהינתן $\Pi$ חלוקה של $A$, יחס השיקלות המושרה מ־$\Pi$ הוא $R_\Pi := \{\la x, y \ra \in A \times A \mid \mid \exists X \in \Pi. x, y \in X\} = \bigcup_{C \in \Pi} (C \times C)$). 
				
				מסעיף ב', ידוע שקיימת חלוקה $\Pi$ של $\R$ כך ש־..., ומכיוון ש־$|\Pi| = \al$, קיים $\vphi \colon \Pi \to \R$ זיווג. נבחר: 
				\[ f_g := \lambda t \in \R. g(\vphi([t]_{R_\Pi})) \]
				
				נוכיח כמה טענות שנדרשות מאיתנו: 
				\begin{itemize}
					\item \textbf{$\bm H$ חח"ע: }נניח $g_1 \neq g_2 \colon \R \to \R$ צ.ל. $\al = \sof{\{x \in \R\mid f_{g_1}(x) \neq f_{g_2}(x)\}}$. אז קיים $r \in \R$ כך ש־$g_1(r) \neq g_2(r)$. אז קיים $X \in \Pi$ כך ש־$\vphi(x) = r$ מכיוון ש־$\vphi$ על. אז לכל $t \in X$ מתקיים $f_{g_1}(t) = g(\vphi([t]_{R_\Pi})) = g_1(\vphi(X)) = g_1(r)$ ובאופן דומה $f_{g_2}(t) = g_2(r) \neq g_1(r)$ ולכן $\al = |X| \le \sof{\{x \in \R \mid f_{g_1}(x) \neq f_{g_2}(x)\}}$. 
				\end{itemize}
			\end{itemize}
		\end{proof}
		
		\end{enumerate}
		
		\section{\en{Q4}}
		נגדיר $X = \{f \in \N \to \N \mid \forall n \in \N. f(n^2) = f(n)\}$. הוכיחו ע"י לכסון ש־$|X| \neq \az$. 
		
		\begin{proof}
			נניח בשלילה שיש זיווג $F \colon \P \to X$. כאשר $\P$ קבוצת הראשוניים, ומתקיים $|\P| = \az$. 
			\[ g = \lambda n \in \N. \begin{cases}
				F(p)(n) + 1 & \exists p \in \P. \exists k \in \N_+. p^{k} = n \\
				0 & \other
			\end{cases} \]
			
			צ.ל.: $g \in X$, $\forall p \in \P. g \neq F(p)$. נוכיח. 
			\begin{itemize}
				\item יהי $n \in \N$. נוכיח $g(n^2) = g(n)$. נפלג למקרים. 
				\begin{enumerate}
					\item אם $\exists p \in \P. \exists k \in\N_+. n = p^{k}$ אז: 
					\[ g(n^2) = F(p)(n^2) + 1 = F(p)(n) + 1 = g(n) \]
					כי $F(p) \in X, \ n = p^k, n^2 = p^{2k}$. 
					\item אחרת: גם $n, n^2$ ךא מקיימים את התנאי, ולכן $g(n^2) = 0 = g(n)$
				\end{enumerate}
				\item לא הוכחנו בכיתה. 
			\end{itemize}
			
		\end{proof}
\end{document}