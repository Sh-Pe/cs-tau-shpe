%! ~~~ Packages Setup ~~~ 
\documentclass[]{article}
\usepackage{lipsum}
\usepackage{rotating}


% Math packages
\usepackage[usenames]{color}
\usepackage{forest}
\usepackage{ifxetex,ifluatex,amsmath,amssymb,mathrsfs,amsthm,witharrows,mathtools,mathdots}
\WithArrowsOptions{displaystyle}
\renewcommand{\qedsymbol}{$\blacksquare$} % end proofs with \blacksquare. Overwrites the defualts. 
\usepackage{cancel,bm}
\usepackage[thinc]{esdiff}

% Deisgn
\usepackage[labelfont=bf]{caption}
\usepackage[margin=0.3in]{geometry}
\usepackage{multicol}
\usepackage[skip=4pt, indent=0pt]{parskip}
\usepackage[normalem]{ulem}
\forestset{default}
\renewcommand\labelitemi{$\bullet$}
\usepackage{titlesec}
\titleformat{\section}[block]
	{\fontsize{20}{20}}
	{\Large \dotfill \ (\thesection) \dotfill}
	{0em}
	{\vspace{2pt} \newline \hfil \Large \filleft \filright \MakeUppercase}
\titleformat{\subsection}[block]
	{\bfseries \large}
	{\hfil (\thesubsection) }
	{}
	{}
	{}
\titlespacing*{\subsection}{0pt}{5pt}{2pt}
\usepackage{graphicx}
\graphicspath{ {./} }


% Hebrew initialzing
\usepackage[bidi=basic]{babel}
\PassOptionsToPackage{no-math}{fontspec}
\babelprovide[main, import, Alph=letters]{hebrew}
\babelprovide[import]{english}
\babelfont[hebrew]{rm}{David CLM}
\babelfont[hebrew]{sf}{David CLM}
\babelfont[english]{tt}{Monaspace Xenon}
\usepackage[shortlabels]{enumitem}
\newlist{hebenum}{enumerate}{1}

% Language Shortcuts
\newcommand\en[1] {\begin{otherlanguage}{english}#1\end{otherlanguage}}
\newcommand\sen   {\begin{otherlanguage}{english}}
	\newcommand\she   {\end{otherlanguage}}
\newcommand\del   {$ \!\! $}

\newcommand\npage {\vfil {\hfil \textbf{\textit{המשך בעמוד הבא}}} \hfil \vfil \pagebreak}
\newcommand\ndoc  {\dotfill \\ \vfil {\begin{center} {\textbf{\textit{שחר פרץ, 2024}} \\ \scriptsize \textit{נוצר באמצעות תוכנה חופשית בלבד}} \end{center}} \vfil	}

\newcommand{\rn}[1]{
	\textup{\uppercase\expandafter{\romannumeral#1}}
}

\makeatletter
\newcommand{\skipitems}[1]{
	\addtocounter{\@enumctr}{#1}
}
\makeatother

%! ~~~ Math shortcuts ~~~

% Letters shortcuts
\newcommand\N     {\mathbb{N}}
\newcommand\Z     {\mathbb{Z}}
\newcommand\R     {\mathbb{R}}
\newcommand\Q     {\mathbb{Q}}
\newcommand\C     {\mathbb{C}}

\newcommand\ml    {\ell}
\newcommand\mj    {\jmath}
\newcommand\mi    {\imath}

\newcommand\powerset {\mathcal{P}}
\newcommand\ps    {\mathcal{P}}
\newcommand\pc    {\mathcal{P}}
\newcommand\ac    {\mathcal{A}}
\newcommand\bc    {\mathcal{B}}
\newcommand\cc    {\mathcal{C}}
\newcommand\dc    {\mathcal{D}}
\newcommand\ec    {\mathcal{E}}
\newcommand\fc    {\mathcal{F}}
\newcommand\nc    {\mathcal{N}}
\newcommand\vc    {\mathcal{V}} % Vance
\newcommand\sca   {\mathcal{S}} % \sc is already definded
\newcommand\rca   {\mathcal{R}} % \rc is already definded

\newcommand\prm   {\mathrm{p}}
\newcommand\arm   {\mathrm{a}} % x86
\newcommand\brm   {\mathrm{b}}
\newcommand\crm   {\mathrm{c}}
\newcommand\drm   {\mathrm{d}}
\newcommand\erm   {\mathrm{e}}
\newcommand\frm   {\mathrm{f}}
\newcommand\nrm   {\mathrm{n}}
\newcommand\vrm   {\mathrm{v}}
\newcommand\srm   {\mathrm{s}}
\newcommand\rrm   {\mathrm{r}}

\newcommand\Si    {\Sigma}

% Logic & sets shorcuts
\newcommand\siff  {\longleftrightarrow}
\newcommand\ssiff {\leftrightarrow}
\newcommand\so    {\longrightarrow}
\newcommand\sso   {\rightarrow}

\newcommand\epsi  {\epsilon}
\newcommand\vepsi {\varepsilon}
\newcommand\vphi  {\varphi}
\newcommand\Neven {\N_{\mathrm{even}}}
\newcommand\Nodd  {\N_{\mathrm{odd }}}
\newcommand\Zeven {\Z_{\mathrm{even}}}
\newcommand\Zodd  {\Z_{\mathrm{odd }}}
\newcommand\Np    {\N_+}

% Text Shortcuts
\newcommand\open  {\big(}
\newcommand\qopen {\quad\big(}
\newcommand\close {\big)}
\newcommand\also  {\text{, }}
\newcommand\defis {\text{ definitions}}
\newcommand\given {\text{given }}
\newcommand\case  {\text{if }}
\newcommand\syx   {\text{ syntax}}
\newcommand\rle   {\text{ rule}}
\newcommand\other {\text{else}}
\newcommand\set   {\ell et \text{ }}
\newcommand\ans   {\mathscr{A}\!\mathit{nswer}}

% Set theory shortcuts
\newcommand\ra    {\rangle}
\newcommand\la    {\langle}

\newcommand\oto   {\leftarrow}

\newcommand\QED   {\quad\quad\mathscr{Q.E.D.}\;\;\blacksquare}
\newcommand\QEF   {\quad\quad\mathscr{Q.E.F.}}
\newcommand\eQED  {\mathscr{Q.E.D.}\;\;\blacksquare}
\newcommand\eQEF  {\mathscr{Q.E.F.}}
\newcommand\jQED  {\mathscr{Q.E.D.}}

\DeclareMathOperator\dom   {dom}
\DeclareMathOperator\Img   {Im}
\DeclareMathOperator\range {range}

\newcommand\trio  {\triangle}

\newcommand\rc    {\right\rceil}
\newcommand\lc    {\left\lceil}
\newcommand\rf    {\right\rfloor}
\newcommand\lf    {\left\lfloor}

\newcommand\lex   {<_{lex}}

\newcommand\az    {\aleph_0}
\newcommand\uaz   {^{\aleph_0}}
\newcommand\al    {\aleph}
\newcommand\ual   {^\aleph}
\newcommand\taz   {2^{\aleph_0}}
\newcommand\utaz  { ^{\left (2^{\aleph_0} \right )}}
\newcommand\tal   {2^{\aleph}}
\newcommand\utal  { ^{\left (2^{\aleph} \right )}}
\newcommand\ttaz  {2^{\left (2^{\aleph_0}\right )}}

\newcommand\n     {$n$־יה\ }

% Math A&B shortcuts
\newcommand\logn  {\log n}
\newcommand\logx  {\log x}
\newcommand\lnx   {\ln x}
\newcommand\cosx  {\cos x}
\newcommand\cost  {\cos \theta}
\newcommand\sinx  {\sin x}
\newcommand\sint  {\sin \theta}
\newcommand\tanx  {\tan x}
\newcommand\tant  {\tan \theta}
\newcommand\sex   {\sec x}
\newcommand\sect  {\sec^2}
\newcommand\cotx  {\cot x}
\newcommand\cscx  {\csc x}
\newcommand\sinhx {\sinh x}
\newcommand\coshx {\cosh x}
\newcommand\tanhx {\tanh x}

\newcommand\seq   {\overset{!}{=}}
\newcommand\slh   {\overset{LH}{=}}
\newcommand\sle   {\overset{!}{\le}}
\newcommand\sge   {\overset{!}{\ge}}
\newcommand\sll   {\overset{!}{<}}
\newcommand\sgg   {\overset{!}{>}}

\newcommand\h     {\hat}
\newcommand\ve    {\vec}
\newcommand\lv    {\overrightarrow}
\newcommand\ol    {\overline}

\newcommand\mlcm  {\mathrm{lcm}}

\DeclareMathOperator{\sech}   {sech}
\DeclareMathOperator{\csch}   {csch}
\DeclareMathOperator{\arcsec} {arcsec}
\DeclareMathOperator{\arccot} {arcCot}
\DeclareMathOperator{\arccsc} {arcCsc}
\DeclareMathOperator{\arccosh}{arccosh}
\DeclareMathOperator{\arcsinh}{arcsinh}
\DeclareMathOperator{\arctanh}{arctanh}
\DeclareMathOperator{\arcsech}{arcsech}
\DeclareMathOperator{\arccsch}{arccsch}
\DeclareMathOperator{\arccoth}{arccoth}
\DeclareMathOperator{\atant}  {atan2} 
\DeclareMathOperator{\Sp}     {span} 
\DeclareMathOperator{\sgn}    {sgn} 
\DeclareMathOperator{\row}    {Row} 
\DeclareMathOperator{\adj}    {adj} 
\DeclareMathOperator{\rk}     {rank} 
\DeclareMathOperator{\col}    {Col} 
\DeclareMathOperator{\tr}     {tr}

\newcommand\dx    {\,\mathrm{d}x}
\newcommand\dt    {\,\mathrm{d}t}
\newcommand\dtt   {\,\mathrm{d}\theta}
\newcommand\du    {\,\mathrm{d}u}
\newcommand\dv    {\,\mathrm{d}v}
\newcommand\df    {\mathrm{d}f}
\newcommand\dfdx  {\diff{f}{x}}
\newcommand\dit   {\limhz \frac{f(x + h) - f(x)}{h}}

\newcommand\nt[1] {\frac{#1}{#1}}

\newcommand\limz  {\lim_{x \to 0}}
\newcommand\limxz {\lim_{x \to x_0}}
\newcommand\limi  {\lim_{x \to \infty}}
\newcommand\limh  {\lim_{x \to 0}}
\newcommand\limni {\lim_{x \to - \infty}}
\newcommand\limpmi{\lim_{x \to \pm \infty}}

\newcommand\ta    {\theta}
\newcommand\ap    {\alpha}

\renewcommand\inf {\infty}
\newcommand  \ninf{-\inf}

% Combinatorics shortcuts
\newcommand\sumnk     {\sum_{k = 0}^{n}}
\newcommand\sumni     {\sum_{i = 0}^{n}}
\newcommand\sumnko    {\sum_{k = 1}^{n}}
\newcommand\sumnio    {\sum_{i = 1}^{n}}
\newcommand\sumai     {\sum_{i = 1}^{n} A_i}
\newcommand\nsum[2]   {\reflectbox{\displaystyle\sum_{\reflectbox{\scriptsize$#1$}}^{\reflectbox{\scriptsize$#2$}}}}

\newcommand\bink      {\binom{n}{k}}
\newcommand\setn      {\{a_i\}^{2n}_{i = 1}}
\newcommand\setc[1]   {\{a_i\}^{#1}_{i = 1}}

\newcommand\cupain    {\bigcup_{i = 1}^{n} A_i}
\newcommand\cupai[1]  {\bigcup_{i = 1}^{#1} A_i}
\newcommand\cupiiai   {\bigcup_{i \in I} A_i}
\newcommand\capain    {\bigcap_{i = 1}^{n} A_i}
\newcommand\capai[1]  {\bigcap_{i = 1}^{#1} A_i}
\newcommand\capiiai   {\bigcap_{i \in I} A_i}

\newcommand\xot       {x_{1, 2}}
\newcommand\ano       {a_{n - 1}}
\newcommand\ant       {a_{n - 2}}

% Linear Algebra
\DeclareMathOperator{\chr}    {char}

\newcommand\lra       {\leftrightarrow}
\newcommand\chrf      {\chr(\F)}
\newcommand\F         {\mathbb{F}}
\newcommand\co        {\colon}
\newcommand\tmat[2]   {\cl{\begin{matrix}
			#1
		\end{matrix}\, \middle\vert\, \begin{matrix}
			#2
\end{matrix}}}

\makeatletter
\newcommand\rrr[1]    {\xxrightarrow{1}{#1}}
\newcommand\rrt[2]    {\xxrightarrow{1}[#2]{#1}}
\newcommand\mat[2]    {M_{#1\times#2}}
\newcommand\gmat      {\mat{m}{n}(\F)}
\newcommand\tomat     {\, \dequad \longrightarrow}
\newcommand\pms[1]    {\begin{pmatrix}
		#1
\end{pmatrix}}

% someone's code from the internet: https://tex.stackexchange.com/questions/27545/custom-length-arrows-text-over-and-under
\makeatletter
\newlength\min@xx
\newcommand*\xxrightarrow[1]{\begingroup
	\settowidth\min@xx{$\m@th\scriptstyle#1$}
	\@xxrightarrow}
\newcommand*\@xxrightarrow[2][]{
	\sbox8{$\m@th\scriptstyle#1$}  % subscript
	\ifdim\wd8>\min@xx \min@xx=\wd8 \fi
	\sbox8{$\m@th\scriptstyle#2$} % superscript
	\ifdim\wd8>\min@xx \min@xx=\wd8 \fi
	\xrightarrow[{\mathmakebox[\min@xx]{\scriptstyle#1}}]
	{\mathmakebox[\min@xx]{\scriptstyle#2}}
	\endgroup}
\makeatother


% Greek Letters
\newcommand\ag        {\alpha}
\newcommand\bg        {\beta}
\newcommand\cg        {\gamma}
\newcommand\dg        {\delta}
\newcommand\eg        {\epsi}
\newcommand\zg        {\zeta}
\newcommand\hg        {\eta}
\newcommand\tg        {\theta}
\newcommand\ig        {\iota}
\newcommand\kg        {\keppa}
\renewcommand\lg      {\lambda}
\newcommand\og        {\omicron}
\newcommand\rg        {\rho}
\newcommand\sg        {\sigma}
\newcommand\yg        {\usilon}
\newcommand\wg        {\omega}

\newcommand\Ag        {\Alpha}
\newcommand\Bg        {\Beta}
\newcommand\Cg        {\Gamma}
\newcommand\Dg        {\Delta}
\newcommand\Eg        {\Epsi}
\newcommand\Zg        {\Zeta}
\newcommand\Hg        {\Eta}
\newcommand\Tg        {\Theta}
\newcommand\Ig        {\Iota}
\newcommand\Kg        {\Keppa}
\newcommand\Lg        {\Lambda}
\newcommand\Og        {\Omicron}
\newcommand\Rg        {\Rho}
\newcommand\Sg        {\Sigma}
\newcommand\Yg        {\Usilon}
\newcommand\Wg        {\Omega}

% Other shortcuts
\newcommand\tl    {\tilde}
\newcommand\op    {^{-1}}

\newcommand\sof[1]    {\left | #1 \right |}
\newcommand\cl [1]    {\left ( #1 \right )}
\newcommand\csb[1]    {\left [ #1 \right ]}
\newcommand\ccb[1]    {\left \{ #1 \right \}}

\newcommand\bs        {\blacksquare}
\newcommand\dequad    {\!\!\!\!\!\!}
\newcommand\dequadd   {\dequad\duquad}

\renewcommand\phi     {\varphi}

\newtheorem{Theorem}{משפט}
\theoremstyle{definition}
\newtheorem{definition}{הגדרה}
\newtheorem{Lemma}{למה}
\newtheorem{Remark}{הערה}
\newtheorem{Notion}{סימון}

\newcommand\theo  [1] {\begin{Theorem}#1\end{Theorem}}
\newcommand\defi  [1] {\begin{definition}#1\end{definition}}
\newcommand\rmark [1] {\begin{Remark}#1\end{Remark}}
\newcommand\lem   [1] {\begin{Lemma}#1\end{Lemma}}
\newcommand\noti  [1] {\begin{Notion}#1\end{Notion}}

%! ~~~ Document ~~~

\author{\en{Shahar Perets}}
\title{\en{Shit Cheat Sheet {\normalsize(\textbf{v}\textit{2})} $\sim$ Linear Algebra 1A $\sim$ TAU}}
\date{\en{\textit{14.2.2025} $\to$ \textit{8.8.2025}}}
\begin{document}
	\setlength{\abovedisplayskip}{3pt}
	\setlength{\belowdisplayskip}{3pt}
	
	\begin{multicols}{2}
		\defi{הטבעיים יסומנו ב־$\N$ ויכללו את אפס. }
		\section{שדות}
		\defi{תהי $\F$ קבוצה, ונניח קיום $a \co \F^2 \to \F$ \textit{חיבור} ו־$m \co \F^2 \to \F$ \textit{כפל}. $\F$ יקרה \textit{שדה} אמ"מ: 
		\noti{
		\[ \forall x, y \in \F\co m(x, y) := x \cdot y = xy, \ a(x, y) = x + y \]
		}
		\begin{enumerate}
			\item קיום ניטרלי לחיבור: \hfill $\exists x \in \F \, \forall y \in \F\co x + y = y$
			\noti{\textit{איבר האפס} יסומן ב־$0$ או $0_\F$, הוא $x$.}
			\item אסוציאטיביות חיבור: \hfill $\forall x, y, z \in \F\co (x + y) + z = x + (y + z)$
			\item חילופיות חיבור: \hfill $\forall x, y \in \F\co x + y = y + x$
			\item קיום איבר נגדי: \hfill $\forall x \in \F \, \exists y \in \F \co x + y = y + x = 0_\F$
			\noti{ה\textit{איבר הנגדי} של $x$ הוא $-x$, הוא $y$.}
			\item קיום ניטרלי לכפל: \hfill $\exists x \in \F \, \forall y \in \F\co xy = y$
			\noti{ה\textit{ניטרלי לכפל} יסומן ב־$1_\F$ או ב־$1$}
			\item אסציאטיביות של כפל: \hfill $\forall x, y, z \in \F \co (xy)z = x(yz)$
			\item קיום הופכי: \hfill $\forall 0 \neq x \in \F \, \exists y \in \F\co xy = yx = 1$
			\noti{\textit{ההופכי} של $x$ יהיה $x\op$ או $\frac{1}{x}$}
			\item חילופיות כפל: \hfill $\forall x, y \in \F \co xy = yx$
			\item דיסטריביוטיביות: \hfill $\forall x, y, z \in \F\co x(y + z) = xy + xz$
			\item \hfil $1_\F \neq 0_\F$
		\end{enumerate}
		}
		
		\theo{הרציונליים $\Q$, הממשיים $\R$, והמרוכבים $\C$ הם שדות. }
		\theo{בעבור שדה כלשהו: 
			\begin{enumerate}
				\item ניטרלי לחיבור הוא יחיד. 
				\item \hfil $\forall a \in \F \co 0 \cdot a = 0$
				\item ניטרלי לכפל הוא יחיד. 
				\item \hfil $\forall a \in \F \, (\exists! -a \co -a + a = 0) \land (-a = (-1) \cdot a)$
				\item לכל $0 \neq a \in \F$ הופכי יחיד. 
				\item \hfil $(b = 0 \lor a = 0) \iff ab = 0$
				\item \hfil $b = c \iff a + b = a + c$
				\item \hfil $a \neq 0 \implies b = c \iff ab = ac$
				\item \hfil $\forall a \in \F \co -(-a) = a$
				\item \hfil $\forall a, b \in \F \setminus \{0\} \co (ab)\op = a\op b\op$
			\end{enumerate}
		}
		\theo{$\F'$ הוא תת־שדה של $\F$ אמ''מ $\forall 0 \neq a, b \in \F' \co a + b, ab, -a, a\op \in \F'$. }
		\defi{לכל $\N \ni n \ge 1$ טבעי, נגדיר יחס לכל $x, y \in \Z$ זוגות שלמים: 
			\[ x \equiv y \mod n \iff \exists k \in \N \co x - y = nk \]
		}
		\lem{אם $n \ge 1$, אז $x \equiv y \mod n$ יחס שקילות. }
		\defi{יהיו $x \in \Z, \ 1 \le n \in \Z$. נגדיר:
			\[[x]_n := \{y \in \Z \mid x \equiv y \mod n\}\]
			  להיות \textit{מחלקת השקילות} של $x$.}
		\theo{$[x]_n = \{x + nk \mid k \in \Z\}$}
		\theo{כל שתי מחלקות שקילות שוות או זרות. }
		\theo{בעבור $[x]_n$, יש בדיוק אחד מבין $\{0, \dots, n - 1\}$. }
		\theo{$\Z_p$ שדה אמ"מ $p$ ראשוני}
		\theo{בהינתן שדה סופי $\F$, קיים $p$ ראשוני כך ש־$\exists k \in \N\co p^k = |\F|$.}
		\defi{$\Z/_{nz} = \{[x]_n \mid x \in \Z\}$, כאשר הפעולות על השדה מוגדרות: 
		\[ [x]_n + [y]_n = [x + y]_n, \ [x]_n \cdot [y]_n = [x \cdot y]_n \]
		והם מוגדרים היטב, לא תלויים בנציגים. איבר האפס הוא $[0]$ ואיבר היחידה $[1]$.}
		\defi{יהי $\F$ שדה, 
			ה\textit{מקדם} (char) של השדה יהיה $0$ אם $\forall n > 0 \co n \cdot 1_\F \neq 0$ אחרת: 
			\[ \mathrm{char}(F) = \min\{n \in \N\mid n \cdot 1_\F = 0\} \]
			כאשר $n \cdot 1_\F := 1_\F + \cdots + 1_\F$, $n$ פעמים. 
			}
		\theo{יהי $\F$ שדה, ו־$0 $ מקדם השדה. אז: 
		\begin{enumerate}
			\item $p$ ראשוני הוא $p = 0$
			\item המקדם של שדה סופי הוא חיובי. 
		\end{enumerate}
		}
		\theo{השדה $\Z_p$ הוא תת־שדה של שדה $\F$ המקיים $\chr \F = p$. }
		\section{מערכת משוואות לינארית}
		\defi{\textit{משוואה לינארית} מעל שדה $\F$ ב־$n$ נעלמים $x_1 \dots x_n$ עם מקדמים $a_1 \dots a_n$ היא משוואה מהצורה: 
		\[ ax_1 + \cdots + a_nx_n = b \]
		כאשר זהו ה\textit{ייצוג הסטנדרטי} של המשוואה. 
		}
		\defi{\textit{מערכת של $m$ משוורות ב־$n$ נעלמים מעל שדה $\F$} הוא אוסף של $m$ משוואות ב־$n$ נעלמים, כאשר הייצוג הסטנדרטי: 
		\[ \begin{cases}
			a_{11}x_1 + a_{12}x_2 + \cdots + a_{1n} = b_1 \\
			\vdots \\
			a_{m1}x_{1} + \cdots + a_{mn} = b_n
		\end{cases} \]}
		\defi{תהי $A$ קבוצה לא ריקה, $n \in \N$, ו־$a_1 \dots a_n \in A$, נסמן את \textit{ה־\n שאיבריה לפי הסדר} להיות $(a_1 \dots a_n) \in A^n$. }
		\defi{\textit{פתרון למערכת משוואות} הוא $(x_1 \dots x_n) \in \F^n$ כך שכל המשוואות מתקיימת לאחר הצבה. }
		\defi{שתי מערכות משוואות נקראות \textit{שקולות} אם יש להן את אותה קבוצת הפתרונות. }
		\defi{תהי מערכת משוואות. \textit{פעולה אלמנטרית} היא אחת מבין: 
		\begin{enumerate}
			\item החלפת מיקום של שתי משוואות. 
			\item הכפלה של משוואה אחת בסקלר שונה מ־0. 
			\item הוספה לאחת משוואות משוואה אחרת מוכפלת בסקלר. 
		\end{enumerate}
		}
		\theo{פעולה אלמנטרית על מערכת משוואות מעבירה למערכת שקולה. }
		\defi{\textit{מטריצה} מסדר $m \times n$ הוא אוסף של $mn$ סקלרים. יתקיים: 
		\begin{gather*}
			i \in \{1 \dots m\}, \ j \in \{1 \dots n\} \\ 
			A = (a_{ij}) = \pms{a_{11} & a_{12} & \cdots & a_{1n} \\ \vdots &\vdots & \ddots & \vdots \\ a_{m1} & a_{m2} & \cdots & a_{mn}}
		\end{gather*}}
		\defi{נגדיר את $0_{n \times m}$ או $0$ להיות מטריצה $A \in M_{m \times n}(\F)$ כך ש־$(A)_{ij} = 0_\F$. }
		\defi{\textit{וקטור שורה} הוא $R_i := (a_{i1} \dots a_{in}) \in \F^{n}$. }
		\defi{\textit{וקטור עמודה} הוא $C_j := (a_{1j} \dots a_{mj}) \in \F^{m}$. }
		\defi{$M_{mn}(\F)$ הוא מרחב המטריצות מסדר $m \times n$ מעל השדה $\F$. }
		\defi{$M_n(\F)$ הוא מרחב ה\textit{מטריצות הריבועיות}, הוא מטריצות מסדר $n \times n$ מעל השדה $\F$. }
		\defi{בהינתן מערכת משוואות עם מקדמים $a_{ij}$, ה\textit{מטריצה של מערכת המשוואות} תהיה $(a_{ij})$, כאשר ה\textit{מטריצה המצומצמת} שלה היא מטריצה בלי העמודה ה־$m + 1$. 
		}
		\defi{\textit{פעולות אלמנטריות על מטריצה} הן: 
		\begin{enumerate}
			\item החלפת מיקום שורות, תסומן $R_i \lra R_j$. 
			\item הכפלה של שורה בסקלר שונה מ־$0$, תסומן ב־$R_i \to \lg R_i$. 
			\item הוספה לשורה אחרת מוכפלת בסקלר, לסומן $R_i \to R_i + \lg R_j$ כאשר $0 \neq \lg \in \F$. 
		\end{enumerate}}
		\defi{יהיו $A, B \in M_{n, m}(\F)$ מטריצות. נאמר ש־$A, B$ \textit{שקולות} אם ניתן לקבל מ־$B$ את $A$ ע"י מספר סופי של פעולות אלמנטריות. נסמן $A \sim B$. }
		\theo{$\sim$ יחס שקילות. }
		\defi{\textit{שורה אפסים} שורה בה כל הרכיבים $0$.}
		\defi{\textit{שורה שאיננה אפסים} היא שורה שאיננה אפסים. }
		\defi{\textit{איבר פותח} הוא האיבר הכי שמאלי במטריצה שאינו 0. }
		\defi{\textit{מטריצה מדורגת} אם: 
		\begin{enumerate}
			\item כל שורות האפסים מתחת לשורות שאינן אפסים. 
			\item האיבר הפותח של שורה נמצא מימין לאיבר הפותח של השורה שמעליה. 
		\end{enumerate}}
		\defi{תהי $A$ מטריצה. $A$ \textit{מדורגת קאנונית} אם כל איבר פותח הוא $1$ וגם שאר האיברים בעמודה הם $0$, שאר האיברים בעמודה הם $0$, ו־$A$ מדורגת.}
		\defi{\textit{משתנה קשןר (תלוי)} אם בעמדוה שלו, בצורה מדורגת קאנונית יש איבר פותח. }
		\defi{\textit{משתנה חופשי} הוא משתנה לא תלוי. }
		\theo{על מטריצה שקולת שורות למטריצה מדורגת קאנונית יחידה. }
		\theo{בהינתן מערכת משוואות שבה יותר נעלמים ממשואות, אז אין פתרונות, או שמספר הפתרונות הוא לפחות $|\F|$.  }
		\theo{בהינתן מערכת משוואות, אחד מהמקרים הבאים יתקיים: 
		\begin{enumerate}
			\item אין פתרונות. 
			\item יש בדיוק פתרון אחד. 
			\item יש לפחות $|\F|$ פתרונות. 
		\end{enumerate}}
		\defi{מערכת משוואות שכל מקדמיה החופשיים הם $0$ היא \textit{מערכת הומוגנית}. }
		\defi{הפתרון $x_1 \dots x_n = 0$ הוא הפתרון הטרוויאלי. }
		\theo{\, 
		\begin{enumerate}
			\item למערכת משוואות הומוגנית שבה מספר נעלמים גדול מהמשוואות, יש ממש יותר מ־$|\F|$ פתרונות. 
			\item למערכת משוואות הומוגנית יש רק פתרון טרוויאלי או לפחות $|\F|$ פתרונות. 
			\item המרצה מסמן מערכת משואות הומוגנית בהומו'. 
		\end{enumerate}}
		
		\section{מרחבים וקטוריים}
		\defi{בהינתן $\F$ שדה, \textit{מרחב וקטורי} (לעיתים קרוי גם \textit{מרחב ליניארי}) הוא $\la  V, a \co V^2 \to V, m \co \F \times V \to V\ra$ כאשר $a$ נקרא \textit{חיבור} ו־$m$ \textit{כפל בסקלר}, המקיים תכונות: 
			\noti{\[ \forall v, w \in V, \ \lg \in \F\co \lg v = \lg \cdot v = m(\lg, v), \ v + w = a(v, w) \]}
		\begin{enumerate}
			\item חילופיות לחיבור. 
			\item אסוציאטיביות לחיבור. 
			\item קיום איבר אפס ניטרלי לחיבור. 
			\noti{האיבר הניטרלי לחיבור יסומן ב־$0$ או $0_V$. }
			\item קיום נגדי לחיבור. 
			\noti{לכל $v$, נסמן ב־$-v$ את הנגדי לחיבור. }
			\item דיסטריביוטיביות מסוג ראשון: \hfill $\forall \lg \in \F, \ u, v \in V \co \lg(u + v) = \lg u + \lg v$
			\item דיסטריבטיוביות מסוג שני: \hfill $\forall \lg, \mu \in \F, v \in V \co (\lg + \mu) \cdot v = \lg v + \mu v$
			\item אסוציאיטיביות של כפל: \hfill $\forall \lg, \mu \in \F \co (\lg \mu) v = \lg (\mu v)$
			\item זהות באיבר היחידה: \hfill $\forall v \in V \co 1_\F \cdot v = v$
		\end{enumerate}
		}
		
		\theo{$M_{n \times m}$ ו־$\F$ הם מרחבים וקטוריים. }
		\defi{יהי $V$ מ"ו, \textit{תת־מרחב־וקטורי (תמ"ו) של $V$} הוא $W \subseteq V$ אם: 
		\begin{enumerate}
			\item $W$ סגור לחיבור. 
			\item $W$ סגור לכפל בסקלר. 
		\end{enumerate}}
		\theo{תמ"ו הוא מ"ו. }
		\theo{קבוצת הפתרונות של מערכת משוואות הומוגנית היא תמ"ו ב־$\F^n$. }
		\theo{\,
		\begin{enumerate}
			\item \hfil $ \forall \lg \in \F \co \lg \cdot 0_V = 0_V $
			\item \hfil $\forall v \in V \co 0 \cdot v = 0$
			\item \hfil $\lg v = 0 \implies \lg = 0 \lor v = 0_V$
			\item \hfil $\forall v \in V \co -v = (-1) v$
		\end{enumerate}}
		\theo{יהי $V$ מ"ו מעל שדה $\F$, ויהיו $U, W \subseteq V$ תמ"וים של $V$. אז, $U \cap W$ ו־$U \cup W$ תמ"ו בנפרד, אמ"מ $U \subseteq W \lor W \subseteq U$. }
		\defi{יהי $V$ מעל $\F$. יהיו $V, W \subseteq V$ תמ"וים. נגדיר $U + W = \{u + w \mid u \in V, w \in W\}$}
		\defi{אם $U \cap W  = \{0\}$ תחת הקשירה לעיל, אז נסמן $U + W = U \oplus W$ ונקרא סכום זה \textit{סכום ישר}. }
		\theo{יהי $V$ מעל שדה $\F$, ו־$U, W \subseteq V$ תמ"וים. אז $U + W$ תמ"ו של $V$. }
		\theo{יהי $V$ מעל שדה $\F$, אז $U + W$ סכום ישר אמ"מ כל וקטור בסכום נין להגדיר בצורה חידה ע"י וקטור מ־$U$ או וקטור מ־$W$. }
		\theo{\textbf{\textit{(משפט ההחלפה)}} בהינתן $v_1 \dots v_n \in V$ פורשת ו־$w_1 \dots w_m \in V$ בת''ל כך ש־$m \le n$, אז ניתן להחליף $m$ איברים מתוך $(v_i)_{i = 1}^{n}$ באיברים מ־$(w_i)_{i = 1}^{n}$ כך שמתקבלת סדרה פורשת. }
		
		\section{ממדים}
		\defi{יהי $0 \le s \in \Z$, וקטורים $v_1 \dots v_s \in V$ וסקלרים $\lg_1 \dots \lg_s \in \F$. ה\textit{צירוף הליניארי} שלהם הוא: 
		\[ \sum_{i = 1}^{s} \lg_iv_i = \lg_1v_1 + \cdots \lg_sv_s \]}
		\defi{צירוף ליניארי עבור סקלרים $\lg_i = 0$. }
		\defi{יהי $B = (v_1 \dots v_s) \in V^s$, ו$V$ מ"ו. אז $B$ \textit{בסיס} אם לכל $v \in V$ קיים ויחיד צירוף ליניארי מהוקטורים ב־$B$, כלומר: 
		\[ \forall v \in V \exists! (\lg_i)_{i = 1}^{|B|} \in \F \co v = \sum_{i = 1}^{s}\lg_ix_i \]}
		\defi{$e_i \in \F^n$ מוגדר להיות $e:= (0 \dots 1 \dots 0)$ כאשר $1$ בקודאינאטה ה־$i$. }
		\defi{$\{e_i\}_n$ הוא \textit{הבסיס הסטנדרטי}. }
		\defi{בעבור $V$ מ"ו עם בסיס $B$, $\dim V := |B|$ (מוגדר היטב ממשפט יחידות גודל הבסיס). }
		\defi{יהיו $v_1 \dots v_s \in V$ וקטורים, הם יקראו \textit{סדרה תלויה לינארית} אם קיימים $\lg_1 \dots \lg_s$ כך אחד מהם שונה מ־$0$ וגם $\sum_{i = 1}^s\lg_is_i = 0$. }
		\defi{\textit{סדרה בלתי תלויה לינארית (בת"ל)} היא סדרה לא תלוי לינארית. }
		\theo{הוקטורים $v_1 \dots v_s \in V^s$ בת"ל אמ"מ $\forall (\lg_i)_{i = 1}^s\co \sum\lg_iv_i = 0$. }
		\theo{בהינתן $v_1 \dots v_n \in \F^n$ ו־$A$ מטריצת העמודות שלה, הסדדרה בת"ל אמ"מ בצורה הקאנונית ששקולה ל־$A$ יש בכל שורה איבר פותח.}
		\theo{הבסיס הסטנדטי הוא בסיס. }
		\theo{בהינתן $U \subseteq V$ תמ"ו, ובהינתן $\{u_i\}_{i = 1} \subseteq U$, אז כל צירוף לינארית שלהם ב־$U$.}
		\defi{בהינתן $x = v_1 \dots v_s$ קבוצת וקטורים, אז \[\Sp(X) := \{\sum_{i = 1}^{s}\lg_iv_i \mid \{\lg_i\}^s_{i = 1} \in \F\}\]}
		\theo{יהיו $V$ מ"ו, $X = (v_1 \dots v_s) \subseteq V$. אז $\Sp(X)$ הוא התמ"ו המינימלי (ביחס ההכלה) שמכיל את $X$. }
		\defi{בהינתן $V$ מ"ו, $X \subseteq V$, נאמר ש־$X$ \textit{פורש} את $V$ אמ"מ $V = \Sp(X)$. לעיתים יקרא $X$ \textit{"קבוצת היוצרים"} של $V$. } 
		\defi{בהינתן $V$ מ"ו, נאמר ש־$V$ נוצר סופית אם קיים $X \subseteq V$ סופי כך ש־$X$ פורש את $Y$. }
		\theo{יהי $V$ נוצר סופית, $X \subseteq V$ פורשת סופית. כל סדרה בת"ל ב־$V$ גדולה לכל היותר $|X|$. }
		\lem{יהי $X$ בת"ל ב־$V$ מ"ו. $u \in V \setminus \Sp(X)$ גורר $X \cup \{u\}$ בת"ל. }
		\theo{בהינתן $V$ נוצר סופית, $X$ פורש, $v_1 \dots v_m$ בת"ל, קיימים $v_{m + 1} \dots v_n \in X$ כך ש־$v_1 \dots v_m, v_{m + 1} \dots v_n$ פורשת ובת"ל (כל בת"ל אפשר להשלים לבסיס). }
		\theo{יהי $B = (v_1 \dots v_s) \in V$. אז $B$ בסיס אמ"מ פורש ובת"ל. }
		\theo{בהינתן $V$ מ"ו, $X$ פורש:
		\begin{enumerate}
			\item כל שדה בת"ל ניתן להשלים ע"י וקטורים מ־$X$. 
			\item בעבור $B_1, B_2$ בסיסים של מ"ו $V$, יתקיים $|B_1| = |B_2|$. 
		\end{enumerate}} 
		\defi{יהי $V$ מ"ו, $B$ בסיס. אז $|B| := \dim V$ \textit{("מימדו" של $V$)}. }
		\theo{בהינתן $V$ מ"ו , $v_1 \dots v_s$ פורש, ניתן למצמצמה לבסיס. }
		\theo{יהיו $V$ מ"ו
		\begin{enumerate}
			\item סדרה בת"ל מגודל מקסימלי היא בסיס. 
			\item סדרה פורשת מגודל מינימלי היא בסיס. 
			\item סדרה בת"ל/פורשת עם $\dim V$ איברים, היא בסיס. 
		\end{enumerate}}
		\theo{יהיו $V$ מ"ו ו־$U\subseteq V$ תמ"ו: 
		\begin{enumerate}
			\item $\dim U \le \dim V$
			\item $\dim U = \dim V \iff U = V$
		\end{enumerate}}
		\theo{יהי $V \subseteq \F^n$ מרחב הפתרונות של משוואה הומוגנית. אז $\dim V$ מספר המשתנים החופשיים במטריצה הקאנונית המתאימה. }
		\theo{(משפט הממדים) יהיו $U, W \subseteq V$ תמ"וים נוצרים סופית. אז: 
		\[ \dim (U + W)  = \dim U + \dim W - \dim(U \cap W) \]}
		
		\section{טרנספורמציות ליניאריות}
		\defi{בהינתן $V_1, V_s$ מ"ו מעל שדה $\F$, נניח קיום $\phi \co V_1 \to V_2$. נקרא את $\phi$ \textit{"העתקה לינארית"} (לעיתים יקרא \textit{"טרנספורמציה לינארית"} או בקיצור \textit{'ט"ל'}) אם: 
		\begin{enumerate}
			\item \hfil $\forall u, v \in v_1 \co \phi(u + v) = \phi(u) + \phi(v)$
			\item \hfil $\forall \lg \in \F \co \phi(\lg v) = \lg\phi(v)$
		\end{enumerate}}
		\theo{$\phi$ העתקה לינארית אמ"מ $\forall \lg_1, \lg_s \in \F, v_1, v_2 \in V \co \phi(\lg_1v_1 + \lg_2v_2) = \lg_1(\phi (v_1)) + \lg_2(\phi(v_2))$. }
		\defi{המרחב $L(V, W)$ יסומן כמרחב ההעתקות הלינאריות $T \co V \to W$. }
		\theo{הקבוצה $L(V, W)$ היא מ''ו מממד $\dim V \cdot \dim W$. }
		\defi{פונקציה תיקרא \textit{שיכון} אמ"מ היא חח"ע. }
		\noti{בהינתן $\phi \co V_1 \to V_2$ העתקה לינארית, \textit{תמונה (Image)} תהיה: 
			\[\Img(\phi) := \Img(\phi) := \{\phi(v) \mid v \in V_1\} \subseteq V_2\]}
		\noti{בהינתן $v \co V_1 \to V_2$ העתקה לינארית, \textit{גרעין (קרנל)} יהיה: 
		\[ \ker\phi := \ker(\phi) = \{v \in V_1 \mid \phi(v) = 0\} \]}
		\noti{\textit{הומומורפיזם} יהיה: 
		\[ \hom_\F(V_1, V_2) = \{\phi \co V_1 \to V_2 \mid \text{העתקה לינארית}\ \phi \} \]}
		\noti{\hfil $\hom(V) := \hom(V, V)$}
		\theo{\hfil $\dim \hom_{\F}(V, W) = \dim V \cdot \dim W$}
		\theo{יהי $\phi \co V \to U$, $\F$ שדה: 
		\begin{enumerate}
			\item \hfil $\phi(0_V) = 0_V$
			\item $\Img\phi$ תמ"ו של $U$. 
			\item $\ker\phi$ תמ"ו של $V$. 
			\item $\phi$ על אמ"מ $\Img\phi = U$
			\item $\phi$ חח"ע אמ"מ $\ker\phi = \{0\}$. 
		\end{enumerate}}
		\noti{$\phi$ \textit{העתקת האפס} אמ"מ $\Img\phi = \{0\}$ אמ"מ $\ker\phi = V$}
		\defi{$\phi \co V_1 \to V_2$ יקרא \textit{איזומורפיזם (איזו')} אם קיימת $\psi$ ט"ל כך ש־$\psi \co V_2 \to V_1$ וגם: 
		\[ \psi \circ \phi  = id_{V_1} \land \phi \circ \psi = id_{V_2} \]}
		\noti{בקשירה בהגדרה לעיל, $\psi =: \phi\op$. }
		\lem{תהי $\phi \co V_1 \to V_2$. אז: 
		\begin{enumerate}
			\item $\phi$ איזו אמ"מ $\phi$ חח"ע ועל. 
			\item אם $\phi$ איזו, אז קיימת לה הופכית יחידה. 
		\end{enumerate}}
		\noti{נאמר שקבוצה היא \textit{איזומורפית} לקבוצה אחרת, אם קיים איזומורפיזם בינהם}
		\theo{נתבונן ב־$\hom(V_1, V_2)$ מ"ו מעל $\F$ בעבור הפעולות: 
		\[ (\phi + \psi)(v) := \phi(v) + \psi(v), \ (\lg\phi) := \lg\phi(v) \]}
		\theo{בעבור $\phi \co V_1 \to V_2, \ \psi \co V_2 \to V_3$ העתקות ליניאריות, יתקיים $\psi \circ \phi$ העתקה לינארית. }
		\theo{הרכבת ט"לים, ביחס עם חיבור פונקציות, על $\hom(V_1, V_2)$ מקיים אסוציאטיביות בהרכבה, דיסטרביוטיביות משמאל ומימין, ותאימות עם כפל בסקלר. }
		\theo{יהיו $\phi \co V \to U, \ V_1 \dots V_2 \in V$ ו־$\lg_1 \dots \lg_s \in \F$. אז 
		\[ \phi\cl{\sum\lg_iv_i} = \sum\lg_i\phi(v_i) \]}
		\theo{יהי $V$ מ"ו עם בסיס $(v_1 \dots v_n)$, אז לכל $(u_1 \dots u_n) \subseteq U$ קיימת ויחידה העתקה לינארית $\phi \co V \to U$ כך ש־$\forall i \in [n] \co \phi(v_i) = u_i$. }
		\noti{יהיו $\phi V \to U$ ט"ל ו־$B = (v_1 \dots v_s)$ וקטורים ב־$V$. נסמן $\phi(B) := (\phi(v_1) \dots \phi(v_s))$ להיות \textit{סדרת התמונות}. }
		\theo{בקשירה לעיל, 
		\begin{enumerate}
			\item אם $\phi(B)$ בת"ל, אז $B$ בת"ל. 
			\item אם $B$ פורשת, אז $\phi(B)$ פורשת את $\Img\phi$. 
			\item אם $\ker\phi = \{0\}$, אז ($B$ בת"ל אמ"מ $\phi(B)$ בת"ל). 
			\item אם $\phi$ איזו, ($B$ בת"ל/פורשת/בסיס (בנפרד) גורר $\phi(B)$ בת"ל/פורשת/בסיס). 
		\end{enumerate}}
		\theo{\hfil $\dim V = \dim \ker\phi + \dim\Img\phi$}
		\theo{תהי $\phi \co V \to U$ ט"ל. אם $\dim V$ סופי, אז: 
		\begin{enumerate}
			\item אם $\phi$ שיכון, אז $\dim V \le \dim U$. 
			\item אם $\phi$ על, אז $\dim U \le \dim V$. 
			\item אם $\phi$ איזו', אז $\dim V = \dim U$. 
			\item אם $\phi$ חח"ע ועל, וגם $\dim V = \dim U$, אז $\phi$ איזו'. 
		\end{enumerate}}
		\defi{$f \co V \to V$ יקרא \textit{פעולה אונרית}. }
		\defi{$f \co V \times V \to V$ יקרא \textit{פעולה בינארית}. }
		\noti{נסמן $V \simeq W$ אמ"מ קיים $f \co V \to W$ איזו'. נאמר $V$ \textit{איזומורפי} ל־$W$. }
		
		\section{ט"לים כמטריצות}
		\theo{יהיו $U, V$ מ"ו ממימד $n$, $B = (v_1 \dots v_n)$ בסיס, אז ישנה ט"ל איזו' בין $\phi \co V \to U$ לבין בסיס של $U$. היא תוגדר באמצעות $\phi(B)$ עבור $\phi$ איזו, ועבור $C$ בסיס של $U$ נתאים את $\phi_C \co V \to U$ כך ש־$\forall i \in [n] \co \phi_C(v_i) = u_i$. }
		\noti{
		\[ [v]_B = (\lg_1 \dots \lg_n) \in \F^n, \ v = \sum_{i = 1}^{n}\lg_iv_i \]}
		\theo{יהי $V$ מ"ו עם בסיס $B = (v_1 \dots v_n)$, נסמן $f(B) = \phi_B \co \F^n \to V$ כך ש־$\phi(\lg_1 \dots \lg_n) = \sum \lg_iv_i$. אז $f$ איזו' וההופכית שלה $f\op = \lg v \in V . [v]_B$.}
		\defi{יהי $\phi \co V \to U$, $B = (v_i)^n_{i = 1}$ בסיס של $V$ ו־$C$ בסיס של $U$ מגודל $m$. נסמן: 
		\[ [\phi]^B_C = \pms{\vdots & & \vdots \\ [\phi(v_1)]_C & \cdots & [\phi(v_n)]_C \\ \vdots & & \vdots} \in M_{m \times n} \]}
		ונקראה \textit{המטריצה המייצגת של $\phi$ לבסי בסיס $B$ ו־$C$}. 
%		\theo{יהיו $U, V$ מ"וים מעל שדה $\F$ ממדים $n = \dim V, \ m = \dim U$. יהיו $C = (u_i) \subseteq U, \ B = (v_i) \subseteq V$ בסיסים. אז: 
%		\[ \sum_{\mathclap{i, j \in [m] \times [n]}}x_ja_{ij}u_j = \sum_{i = 1}^{m}u_i \cl{\sum_{j = 1}^{n}x_ja_{ij}} = \sum_{i = 1}^{n}u_ix_i\col_i \]}
		\defi{מטריצת הסיבוב ב־$\ta$ מעלות היא $R_{\ta} \co \R^2 \to \R^2$ תהיה $R_{\tg} := \binom{\cos \tg \, -\sin \tg}{\sin \tg \,\,\, \cos \tg\,}$. }
		\theo{ייצוג העתקה באמצעות מטריצה היא העתקה לינארית $\psi \co L(V, W) \to M_{m \times n}$ והיא איזומורפיזם (כאשר $\dim V = n, \dim W = m$). }
		\theo{$[Tv]_\ac = [T]^\bc_\ac[v]_\bc$ כאשר $v \in V, T \co V \to W$ ו־$\bc, \ac$ בסיסים של $V, W$ בהתאמה. }
		
		
		\section{כפל מטריצות}
		\theo{יהיו $\phi, \psi$ העתקות ליניאריות, מבסיסים $B$ ל־$C$. אז: 
			\[ [\psi + \phi]_C^B = [\phi]_C^B + [\psi]_C^B, \ [\lg\phi]_C^B = \lg[\phi]_C^B \]}
		\theo{יהיו $U, V$ מ"וים, ו־$B, C$ בסיסים ממדים $m, n$, בהתאמה פעמיים. אז $T \co \hom(V, U0 \to M_{m \times n}(\F))$ המוגדרת לפי $T(\phi) = [\phi]_C^B$, היא איזומורפיזם. }
		\defi{יהיו $A = (a_{ij}) \in M_{m \times s}, \ B = (b_{ij}) \in M_{s \times n}$ מטריצות. נגדיר: 
		\[ AB := A \cdot B = \sum_{k = 1}^{s}a_{ik}b_{kj} \in M_{m \times n} \]}
		\theo{יהיו $V \overset{\phi}{\longrightarrow} U \overset{\psi}{\longrightarrow} W$ ט"לים. $B_v, B_u, B_w$ בסיסיהן בהתאמה. אז: 
		\[ [\psi \circ \phi]_{B_w}^{B_v} = [\psi]_{B_w}^{B_u}[\phi]_{B_u}^{B_v} \]}
		\theo{יהיו $A, B, C$ מטריצות, אז: 
		\begin{enumerate}
			\item \hfil $(AB)C = A(BC)$
			\item \hfil $A(B + C) = AB + AC$
		\end{enumerate}}
		\defi{
		\[ I_n = \pms{1 & 0 & 0 & \cdots \\ 0 & 1 & 0 & \cdots \\ 0 & 0 & \ddots & \cdots \\ 0 & \cdots & 0 & 1} \]}
		\theo{עבור $V$ מ"ו, אם $\dim V = n$ אז $[id_V]^B_B = I_n$}
		\theo{תהי $A = (a_{ij}) \in M_{m \times n}(\F)$ מטריצה. יהי $x = (x_i) \in \F^m$ ו־$b = (b_i) \in \F^m$. אז $Ax = b$ אמ"מ $x$ פתרון למערכת המשוואות ש־$(A \mid b)$ מייצגת. }
		\theo{תחת הקשירה של הטענה הקודמת, מרחב הפתרונות של $Ax = 0$ הוא מ"ו. }
		\theo{תחת הקשירה של הטענה הקודמת, לכל $\phi$ ט"ל מ־$V$ ל־$U$ עם בסיסים $B, C$ בהתאמה, כך ש־$[\phi]_C^B = A$, יתקיים שמרחב הפתרונות של $(A \mid 0)$ יהיה $\ker \phi$. }
		
		\section{מטריצות הפיכות ואלמנטריות}
		\defi{בהינתן מטריצה $A \in M_{m \times n}(\F)$, \textit{המטריצה המשוחלפת שלה} תהיה 
		$A^T = (a_{ji}) \in M_{n \times m}(\F)$. }
		\theo{תהי $A$ מטריצה: 
		\begin{enumerate}
			\item \hfil $(A^T)^T = A$
			\item \hfil $(\lg A)^T = \lg A^T$
			\item \hfil $(A + B)^T = A^T + B^T$
			\item \hfil $(AB)^T = B^TA^T$
		\end{enumerate}}
		\theo{יהיו $A \in M_{m \times n}(\F)$ מטריצה, $\phi \co \F^m \to \F^n$ העתקה' tz: 
		\[ \phi_A := (\lg_1 \dots \lg_m) = (\lg_1 \dots \lg_n) \cdot A, \ [\phi_A]^E_E = A^T \] ???}
		
		\defi{$A$ \textit{הפיכה מימין} אם קיימת $\exists B \in M_{n}(\F) \co AB = I_n$. }
		\defi{$A$ \textit{הפיכה משמאל} אם קיימת $\exists B \in M_{n}(\F) \co BA = I_n$. }
		\defi{$A$ הפיכה אם קיימת $\exists B \in M_{n}(\F) \co AB = BA = I_n$. }
		\theo{בהינתן $A \in M_n(\F)$, אז $A$ הפיכה אמ"מ היא מייצגת איזומורפיזם אמ"מ כל ההעתקות שהיא מייצגת הן איזומורפיזם. }
		\defi{ההופכית למטריצה היא יחידה. }
		\noti{בהינתן מטריצה הפיכה $A$, את ההופכית שלה נסמן ב־־$A\op$ (מוגדר היטב מיחידות). }
		\theo{$A$ הפיכה מימין אמ"מ $A$ הפיכה משמאל אמ"מ $A$ הפיכה מימין. }
		\theo{תהי $Ax = b$ מערכת משוואות עם $n$ נעלמים, $A \in M_{n}(\F), \ x = (x_i)^n_{i = 1}$, ווקטור משתנים $b = (b_i)_{i = 1}^n$. אז $A$ הפיכה גורר $A\op b = x$ פתרון יחיד. }
		\theo{יהיו $A, B \in M_{n}(\F)$ הפיכות, אז: 
		\begin{enumerate}
			\item $A\op$ הפיכה (ידוע גם כ''רגולרית'' ו''לא סינגולרית''). 
			\item $(A\op)\op =A$. 
			\item $A^T$ הפיכה. 
			\item $(A^T)\op = (A\op)^T$. 
			\item $AB$ הפיכה, ומתקיים $(AB)\op = B\op A\op$. 
		\end{enumerate}}
		\theo{\hfil $(A_1 \cdots A_s)\op = A_s\op \cdots A_1 \op$}
		\defi{\textit{מטריצה אלמנטרית} היא מטריצה שמתקבלת ממטריצת היחידה ע"י פעולה אלמנטרית אחת. }
		\theo{תהי $\phi$ פעולה אלמנטרית, $E := \phi(I_n)$, אז$\phi(A) = E \cdot A$. }
		\theo{תהי $A$ מטריצה אלמנטרית, אזי $A$ הפיכה וההופכית שלה אלמנטרית. }
		\theo{מכפלה של אלמנטרית היא הפיכה. }
		\theo{יהי $B \in M_{m \times n}$, אז קיימת $A \in M_{m}(\F)$ מכפלת אלמנטריות, ו־$B' \in M_{m \times n}(\F)$ מדורגת קאנונית, כך ש־$B = AB'$. }
		\theo{תהי $B \in M_{n}(\F)$ מדורגת קאנונית, אז $B = I_n$ אמ"מ $B$ הפיכה. }
		\theo{יהיו $A, B, C \in M_n(\F)$, ונניח $A = CB$. אם $C$ הפיכה, אז $B$ הפיכה אמ"מ $A$ הפיכה. }
		\theo{יהיו $A, B  \in M_n$ מטריצות מדורגות קאנונית כך ש־$B = E_s \cdots E_1A$ עבור $E_i$ מטריצה אלמנטרית. אז: 
		\begin{enumerate}
			\item $A$ הפיכה אמ"מ $B = I$
			\item אם $A$ הפיכה, אז $A\op = E_s \cdots E_1$. 
		\end{enumerate}}
		\defi{$A$ תקרא \textit{סימטרית} אם $A^T = A$ (ובפרט $A$ ריבועית). }
		\defi{$A$ אנטי־סימטרית אם $A^T = -A$. }
		\defi{עבור מטריצה $A \in M_{m \times n}(\C)$, נגדיר $A^* \in M_{n \times m}(\C)$ ע"י $(A^*){ij} = \ol{A_{ij}}$ להיות \textit{המטריצה הצמודה של $A$}. }
		\theo{תהי $A \in M_n(\F)$, התנאים הבאים שקולים: 
		\begin{enumerate}
			\item $A$ הפיכה
			\item $\forall v \in \F^n$ למערכת המשוואות $Ax = b$ קיים פתרון יחיד. 
			\item $\forall b \in \F^n$ למערכת המוושואת $Ax = b$ קיים פתרון. 
			\item קיים $b \in \F^n$ כך שלמערכת $Ax = b$ פתרון יחיד. 
			\item למערכת $Ax =0$ פתרון יחיד. 
			\item $A$ שקולת שורות ל־$I$. 
			\item עמודות $A$ בת"ל. 
			\item שורות $A$ בת"ל. 
			\item עמודות $A$ פורשות את $\F^n$. 
			\item שורות $A$ פורשות את $\F^n$. 
		\end{enumerate}}
	
		\section{שינוי בסיס}
		\theo{יהי $B = \{\ta_1 \dots \ta_n\}$ בסיס ל־$V$ וגם $B' = \{u_1 \dots u_n\}$ כך ש־$\forall i \in [n] \co u_i = \sum \ag_{ji} \ta_j$, אז: 
			\[ M := \pms{\ag_{11} & \cdots & \ag_{in} \\ \vdots & \ddots & \vdots \\ \ag_{n1} & \cdots & \ag_{nn}} \]
			היא הפיכה אמ"מ $B'$ בסיס ל־$V$. 
		}
		\defi{יהיו $B, B'$ בסיסים של $V$ מ"ו. אז $M = [id]^{B'}_B$ היא \textit{מטריצת המעבר} מבסיס $B'$ ל־$B$. }
		\theo{יהי $V$ מ"ו ונסמן $\dim V = n$, ו־$B, \ B'$ בסיסים ל־$V$, אז מטריצת המעבר $M$ מ־$B'$ ל־$B$ תקיים $\forall \ta \in V \co [\ta]_B = M[\ta]_{B'}$. }
		\noti{תהי $T \co V \to V$ ט"ל ו־$V$ מ"ו. נסמן $[T]_B^B = [T]_B$. }
		\theo{תהי $T \co V \to W$ איזו', ו־$B, C$ בסיסים של $V$ ו־$W$ בהתאמה. אז $[T\op]^{C}_B = ([T]^B_C)\op$. }
		\theo{יהיו $T \co V \to V$ ט"ל, נסמן $\dim V = n$, ויהיו $B, B'$ בסיסים של $V$, ו־$M$ מטריצת מעבר בסיס מ־$B'$ ל־$B$. אז $[T]_{B'} = M\op[T]_BM$. }
		
		\defi{יהיו $A, B \in M_n(\F)$. נאמר ש־$A$, $B$ \textit{דומות} אם קיימת מטריצה הפיכה $P \in M_n(\F)$ כך ש־$A = P\op B P$. }
		\theo{תהי $T \co V \to V$ ט"ל ויהיו בסיסים $B, C$ של $V$. אז $[T]^B_B, [T]^C_C$ דומות. }
		\theo{יהיו $A, B$ מטריצות דומות. $\rk A = \rk B \land \tr A = \tr B$. }
		
		\defi{יהיו מטריצות $A, B \in M_{m \times n}(\F)$. נאמר כי הן מטריצות \textit{מתאימות} אם קיימות מטריצות הופכיות $P \in M_n(\F), \ Q \in M_m(\F)$ כך ש־$A = Q\op BP$. }
		\theo{יהיו $V, W$ מעל $\F$, ותהי $T \co V \to W$ ט"ל. כמו כן, יהיו $B, B'$ ביסים של $V$ ו־$C, C'$ בסיסים של $W$. אז $[T]^B_C, \ [T]^{B'}_{C'}$ מטריצות מתאימות. }
		\theo{מטריצות $A, B \in \gmat$ מתאימות אמ"מ $\rk A = \rk B$. }
		
		\section{דרגת מטריצה}
		\defi{תהי $A \in M_{m \times n}(\F)$. נגדיר את \textit{דרגת} $A$ להיות הממד של התמ"ו של $\F^n$ הנפרש ע"י שורות $A$. }
		\noti{עבור $v_1 \dots v_m$ שורות של $A$ נסמן $\rk A = \dim \row A$. }
		\lem{בהינתן $m$ שורות מעל $\F^n$, נדע $\rk A \le \min(m, n)$}
		\theo{תהי $A \in \gmat$ ו־$B \in \mat{n}{s}(\F)$. אז \[ \rk AB \le \rk B \] ואם $A$ ריבועית והפיכה, $\rk AB = \rk B$}
		\theo{עבור מטריצה מדורגת, מספר השורות השונות מ־$0$ הוא $\rk A$. }
		\theo{\hfil $\rk A^T = \rk A$}
		\theo{\hfil $\rk A = \dim \row A = \dim \col A$}
		\theo{בעבור $A \in M_n(\F)$ מרחב הפתרונות של מערכת המשוואות $Ax = 0$ הוא $n - \rk A$. }
		\theo{\hfil $\rk (A + B) \le \rk A + \rk B$}
		\theo{\textbf{\textit{(משפט האפסיות של סילבסטר)}} כאשר $U \overset{S}{\to} V \overset{T}{\to} W$:
			 \begin{enumerate}
			 	\item \hfil $\rk(TS) \ge \rk T + \rk S - \dim V$ 
			 	\item \hfil $\dim \ker(TS) \le \dim \ker T + \dim \ker S$
			 \end{enumerate}
		}
		\theo{\!\!\hfil$\rk A \!=\! \min\{k \in \N \mid \exists B \!\in\! M_{n \times k}, C \in M_{k \times n} \co A = BC\}$}
		
		\section{דטרמיננטות}
		\defi{פונ' $\det \co M_n(\F) \to \F$ תקרא \textit{דטרמיננטה} אמ"מ: 
		\begin{itemize}
			\item $\det$ מולטילינארית (לינארית בשורה). 
			\item בעבור $M \in M_n(\F)$ ו־$M'$ מטריצה שהוחלפו בה שתי שורות כלשהן, $\det M = -\det M'$. 
			\item $\det I_n = 1$. 
		\end{itemize}}
		\theo{הדרמיננטה היא פונקציית נפח. }
		\theo{תהי $A \in M_{2 \times 2}(\F)$ ו־$A = \binom{a\, b}{c\, d}$. אז $\det A = ad - bc$. }
		\theo{בהינתן $\phi$ פעולה אלמנטרית ו־$\det$ דטרמיננטה, אז: 
		\begin{itemize}
			\item אם $\phi$ החלפת שורות, $\det \phi (A) = - \det A$. 
			\item אם $\phi$ הכפלה בסקלר $\lg$, אז $\det \phi (A) = \lg \det A$. 
			\item אם $\phi$ הוספת שורה מוכפלת בסקלר לאחרת, אז $\det \phi (A) = \det A$. 
		\end{itemize}}
		\theo{ההדטרמיננטה קיימת ויחידה. }
		\rmark{אם אתם שונאים את עצמכם, תוכיחו את יחידות הדטרמיננטה. }
		\theo{$\det A = \det A^T$}
		\lem{תהי $A \in M_n(\F)$ עם שורת אפסים. אז $\det A = 0$. }
		\noti{\hfil $|A| := \det A$}
		\rmark{\textit{סימון 22} מוגדר היטב לכל $A$ כי הדטרמיננטה קיימת ויחידה. }
		\theo{יהיו $\det \co M_n(\F) \to \F, A, B \in M_n(\F)$ דטרמיננטות. אז $\det AB = \det A \cdot \det B$. }
		\theo{\hfil $\det A\op = (\det A)\op$}
		\theo{תהי $A \in M_n(\F)$. אז $A$ הפיכה אמ"מ $|A| \neq 0$. }
		\defi{תהי $A \in M_n(\F)$ ויהיו $i, j \in [n]$. אז ה\textit{מינור} $A_{ij}$ היא המטריצה המתקבלת מ־$A$ ע"י מחיקת השורה ה־$i$ והעמודה ה־$j$. }
		\begin{Theorem}\textbf{\textit{(פיתוח לפי עמודה)}}
			תהי $(a_{ij}) = A \in M_n(\F)$. אז 
			\[ \forall i \in [n] \co |A| = \sum_{j = 1}^{n}(-1)^{i + k}a_{ij} |A_{ij}| \]
		\end{Theorem}
		\begin{Theorem}\textbf{\textit{(פיתוח לפי שורה)}}
			תהי $(a_{ij}) = A \in M_n(\F)$. אז 
			\[ \forall j \in [n] \co |A| = \sum_{i = 1}^{n}(-1)^{i + k}a_{ij} |A_{ij}| \]
		\end{Theorem}
		
		\defi{\textit{תמורה} היא פרמוטציה}
		\noti{נסמן ב־$S_n$ את קבוצת כל התמורות על $[n]$. }
		\defi{תהי $\sg \in S_n$, נגדיר את $\sgn \sg$ להיות מספר ההחלפות ש־$\sg$ מבצעת ב־$\la n \ra$. }
		\defi{\hfil $\forall \sg \in S_n \co \sg := \pms{1 & \cdots & n \\ \sg(1) & \cdots & \sg(n)}$}
		\defi{\hfil $\forall sg \in \S_n \co P_\sg := (e_{\sg(1)} \cdots e_{\sg(n)})$}
		\theo{\hfil $\sgn(\sg) = \det(P_\sg)$}
		\lem{$\sgn(\sg)\sgn(\tau) = \sgn(\sg \tau)$}
		
		\begin{Theorem}\textit{\textbf{(פיתוח לפי תמורות)}}
			תהי $A \in M_n(\F)$. אז: 
			\[ \det A = \sum_{\sg \in S_n} \cl{\sgn(\sg) \prod_{i = 1}^{n} a_{i,\,\sg(i)}} \]
		\end{Theorem}
%		\columnbreak
		
		
		\section{אחר}
		
		\subsection{מטריצת בלוקים}
		\defi{\textit{מטריצת בלוקים} תהיה כזה בלוקים שיש במטריצה (אין לי כוח להגדיר פורמלית). }
		\theo{כפל מטריצות בלוקים שקול לכפל מטריצות אלכסוניות מעל חוג המטריצות. }
		\theo{תהינה $A \in M_n(\F), B \in \gmat, D \in M_m(\F)$ מטריצות. אז $\det \binom{A \, B}{0\, D} = \det A \det D$ והופכית $\binom{A\op \,\,\, -A\op BD\op}{0\quad\quad\quad\, D\op\,\,\,\,\,}$. }
		\theo{המטריצות $\binom{A\,\, 0}{0\, D} \sim \binom{D\,\, 0}{0\,A}$ דומות. }
		
		\subsection{מטריצה מצורפת}
		\defi{תהי $A \in M_n(\F)$. נגדיר את ה\textit{מטריצה המוצמדת} (עיתים קרויה גם "\textit{מצורפת}") להיות מוגדרת ע"י: 
			\[ (\adj A)_{ij} = (-1)^{i + j}|A_{ji}| \]}
		\theo{תהי מטריצה $A \in M_n(\F)$. אז $A \cdot \adj A = \adj A \cdot A = |A| I$. בפרט, בעבור $A$ הפיכה, $A\op = \frac{1}{|A|} \adj A$}
		\theo{\,
			\begin{enumerate}
				\item \hfil $\adj(AB) = \adj A \adj B$
				\item \hfil $\det(\adj A) = (\det A)^{n - 1}$
				\item \hfil $\adj(A^T) = (\adj A)^T$
				\item \hfil $\adj(cA) = c^{n -1}\adj A$
				\item \hfil $\rk A = n - 1 \implies \rk \adj A = 1$
				\item \hfil $\rk A \le n - 2 \implies \rk \adj A = 0$
		\end{enumerate}}
		
		\subsection{עקבה}
		\defi{תהי $A \in M_n(\F)$. נגדיר את ה\textit{עקבה} של $A$ להיות $\tr A = \sum_{i = 1}^{n} (A)_ii$. }
		\theo{\textbf{\textit{(ציקליות העקבה)}}\hfil $\forall A, B \in M_n(\F) \co \tr(AB) = \tr(BA)$}
		\theo{$\tr \co M_n(\F) \to \F$ היא ט"ל. }
		\theo{העקבה לא תלויה בנציגי יחס הדמיון. }
		
		\subsection{ונדרמונדה וקרמר}
		\begin{Theorem}\textit{\textbf{(כלל קרמר)}}
			תהי $Ax = b$ מערכת משוואות לינארית כאשר $A \in M_n(\F)$\ ו־$b \in \F^n$. אז אם $\det A \neq 0$, הפתרון היחיד של המערכת $Ax = b$ נתון ע"י: 
			\[ x = \cl{\frac{\det A_i}{\det A}}_{i = 1}^{n} \]
			כאשר $A_i$ המטריצה המתקבלת ע"י החלפת עמודה ה־$i$ של $A$ ב־$b$. 
		\end{Theorem}
		\defi{יהיו $(\ag)_{i = 1}^{n - 1}$ סקלרים ב־$\F$, אזי מטריצת ונרדמונדה מוגדרת לפי
			\[ V = \pms{1 & \ag_1 & \ag_1^2 & \cdots & \ag_1^{n - 1} \\ 1 & \ag_2 & \ag_2^2 & \cdots & \ag_2^{n - 1} \\ \vdots & \vdots  & \vdots && \vdots \\ 1 & \ag_n & \ag_n^{2} & \cdots & \ag_n^{n - 1}} \]
		}
		\theo{מטריצת ונדרמונדה ריבועית והדטרמיננטה שלה:
			\[\det V = \, \prod_{\mathclap{1 \le i < j \le n}}\:(\ag_i - \ag_j)\]\
		}	
		\defi{העתקה \textit{אפינית} היא העתקה לינארית עד לכדי חיבור סקלר. }
		
	\end{multicols}
	{\vspace{-30pt}\let\newpage\relax\maketitle\vspace{-100pt}}
	\maketitle
	
\end{document}
