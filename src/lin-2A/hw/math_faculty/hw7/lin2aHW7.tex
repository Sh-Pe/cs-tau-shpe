\documentclass[]{../../../../../tex/classes/homework}
\usepackage{../../../../../tex/packages/hebrewSupport}
\usepackage{../../../../../tex/packages/mathShortcuts}

\renewcommand\mut[2] {\left \la #1, #2 \right \ra}

\author{שחר פרץ}
\title{לינארית 2א $\sim$ \textit{תרגיל בית 7}}
\begin{document}
	\maketitle
	\section{}	
	נוכיח שאם $A, B \in M_n(\R)$ חיוביות, אז $A + B$ חיובית. 
	\begin{proof}
		משום ש־$A, B$ חיוביות נסיק שלכל $0 \neq v \in V$ מתקבל $v^TAv > 0$ וכן $v^TBv > 0$. מכאן ש־: 
		\[ v^{T}(A + B)v = \underbrace{v^TAv}_{>0} + \underbrace{v^TBv}_{> 0} > 0 \]
	\end{proof}
	
	\section{}
	\begin{enumerate}[(A)]
		\item נתבונן במטריצות בסעיפים הבאים. נמצא $M$ הפיכה ו־$B$ אלכסונית כך ש־$B = M^TAM$. 
		נתחיל מהמטריצה הבאה: 
		\begin{gather*}
			\tmat{0 & 2 & 0 \\ 2 & 0 & 0 \\ 0 & 0 & 1}{I} \rrt{R_1 \to R_1 + R_2}{C_1 \to C_1 + C_2} \tmat{4 & 2 & 0 \\ 2 & 0 & 0 \\ 0 & 0 & 1}{1 & 1 & 0 \\ 1 & 1 & 0 \\ 0 & 0 & 1} \rrt{R_2 \to R_2 - \frac{1}{2}R_1}{C_2 \to C_2 - \frac{1}{2}C_2} \tmat{4 & 0 & 0 \\ 0 & -1 & 0 \\ 0 & 0 & 1}{1 & \frac{1}{2} & 0 \\ \frac{1}{2} & 0 & 0 \\ 0 & 0 & 1} \\ 
			M^{T} = \pms{1 & \frac{1}{2}  & 0 \\ \frac{1}{2} & 0 & 0 \\ 0 & 0 & 1} \quad M^{T}AM = \diag(4, -1, 1)
		\end{gather*}
		עתה נחפסן את המטריצה הבאה: 
		\begin{gather*}
			\tmat{2 & -1 & 0 \\ -1 & 2 & -1 \\ 0 & -1 & 2}{1 & 0 & 0 \\ 0 & 1 & 0 \\ 0 & 0 & 1} \rrt{R_2 \to R_2 + \frac{1}{2}R_1}{C_2 \to C_2 + \frac{1}{2}C_1} \tmat{2 & 0 & 0 \\ 0 & 1.5 & -1 \\ 0 & -1 & 2}{1 & 0.5 & 0 \\ 0.5 & 1 & 0 \\ 0 & 0 & 1}
			\rrt{R_3 \to R_3 + \frac{2}{3}R_2}{C_3 \to C_3 + \frac{2}{3}C_2}\tmat{2 & 0 & 0 \\ 0 & 1.5 & 0 \\ 0 & 0 & \frac{4}{3}}{1 & 0.5 & \frac{2}{6} \\ 0.5 & 1 & \frac{2}{3} \\ \frac{2}{6} & \frac{2}{3} & 1}
		\end{gather*}
		סה''כ: 
		\[ M^{T} = \pms{1 & \frac{1}{2} & \frac{2}{6} \\ \frac{1}{2} & 1 & \frac{2}{3} \\ \frac{2}{6} & \frac{2}{3} & 1} \quad M^TAM = \diag\cl{1, \frac{3}{2}, \frac{4}{3}} \]
		\item נניח ש־$A, B \in M_n(\R)$ חופפות ו־$A$ חיובית. נראה ש־$B$ חיובית. \begin{proof}
			יהי $0 \neq \in V$. נבחין מאסוציאטיביות כפל מטריצות: 
			\[ v^{T}Bv = v^{T}(M^{T}AM)v = (v^{T}M^{T})A(Mv) = (Mv)^{T}A(Mv) > 0 \]
			כאשר השוויון האחרון נובע מכך ש־$A$ חיובית כלומר לכל $0 \neq w \in V $ מתקיים $w^TAw > 0$ ובפרט בעבור $w = Mv$ ($M$ הפיכה ולכן $Mv \neq 0$). סה''כ $B$ חיובית כנדרש. 
		\end{proof}
		\item לגבי כל אחת מהמטריצות, אם היא חיובית נוכיח זאת, ואחרת נמצא $v \neq 0$ כך ש־$v^{T}Av \le 0$. בעבור המטריצה הראשונה, נגדיר: 
		\[ v = \pms{1 \\ 0 \\ 0} \quad v^{T}Av = \pms{1 \\ 0 \\ 0}^{T}\pms{0 \\ 2 \\ 0} = 0 \cdot 1 + 2 \cdot 0 + 0 = 0 \le 0 \]
		כלומר המרטיצה הראשונה אינה מוגדרת חיובית. לעומתה, לכל $v \in V$, ניתן להציגו כצירוף לינארי $\sumnio \ag_i v_i$ כאשר $v_i$ עמודות $M^{T}$, ולקבל מאיך ששינוי בסיס עובד (נסמן $q(v) = v^{T}Av$) ש־$q(v) = \sumnio q(\ag_i v_i) = \sumnio \ag^{2}q(v_i)$. משום ש־$q(v_i)$ מספר חיובי (הוא $1, \frac{3}{2}$ או $\frac{4}{3}$) כי $q(v_i) = e_i^{T}De_i$, אז סה''כ הסכום לעיל חיובי (כי גם הריבוע $\ag_i^{2}$ חיובי) כלומר המטריצה כולה מוגדרת חיובית. מכאן שהמטריצה השנייה מוגדרת חיובית. 
	\end{enumerate}
	
	\section{}
	יהיו $p, q \in \F[x]$ פולינומים פורמליים שאינם $0$. נוכיח ש־$\deg(p \cdot q) = \deg p + \deg q$. 
	
	\begin{proof}
		הגדרנו את $\deg p$ להחזיר את המעלה הגבוהה ביותר של פולינום $p$. יהיו $p, q \in \F[x]$. נסמן $p = \sum_{n}^{i = 0}p_i$ וכן $q = \sum_{i = 0}^{m} q_i$. ניעזר בהגדרת מכפלת פולינומים: 
		\[ p \cdot q = {\cl{\sum_{i = 0}^{n}p_i}\cdot\cl{\sum_{i = 0}^{m}q_i}} = \sum_{i = 0}^{n}\cl{\sum_{k = 0}^{m}p_iq_ix^{i + k}} \]
		נבחין שהערך המקסימלי בביטוי הימני של $i + k$ הוא $n + m$ מהגדרת הסכום. מכאן ש־$\deg(p \cdot q) = n + m$. 
	\end{proof}
	
	\section{}
	נחשב להנאתנו את החלוקה $p$ חלקי $q$ עם שארית. 
	\begin{enumerate}[(A)]
		\item נגדיר $p = x^{20}$ ו־$q = x^{10}$. נבחין ש־$x^{10} \cdot x^{10} = x^{20}$ כלומר $p = x^{10}q$. 
		\item נגדיר $p = x^{10}$ ו־$q = x^{20}$. נבחין ש־$\deg x^{20} < \deg x^{10}$ כלומר החלוקה עם השארית איננה מוגדרת. 
		\item נגדיר $p = x^{4} - x^{3} - 18x^{2} + 7x + 1$ ו־$q = x + 4$. 
		\begin{itemize}
			\item נתחיל מהמעלה הראשונה. $x^{3}(x - 4) = x^{4} - 4x^{3}$. נקבל ש־$p = x^{3}q + 3x^{3} - 18x^{2} + 7x + 1$. 
			\item נמשיך על השארית. נחשב את $3x^{2}(x - 4) = 3x^{2} - 12x^{2}$. נקבל ש־$3x^{3} - 18x^{2} - 7x + 1 = 3x^{2}q - 6x^{2} + 7x + 1$. 
			\item נמשיך על השארית. נחשב את $-6x(x - 4) = 24x - 6x^{2}$. נקבל ש־$-6x^{2} + 7x + 1 = -6xq + 1 - 17x$. 
			\item נמשיך על השארית. נחשב את $-17(x - 4) = 68 - 17x$. נקבל ש־$-17x + 1 = -17q - 67$. 
			\item נבחין ש־$67$ אינו ניתן לחלוקה ב־$q = x - 4$ מהגדרת חלוקה. 
		\end{itemize}
		נסכום ונסכם: 
		\[ p = (x^{3} + 3x^{2} - 6x - 17)(x - 4) + 67 \]
		התוצאה מסתדרת עם משפט בזו הקטן, שכן $4$ אינו שורש של $q$, וכן בהכרח היינו אמורים להיוותר עם שארית חלוקה ממעלה קטנה מ־$1$. 
		\item נסמן $p = x^{3} + x^{2} - 5$ ו־$q = 6$. נניח $\chr \F \neq 2, 3$ ע''מ שחלוקה בפולינום תהיה מוגדרת (אי אפשר לחלק בפולינום האפס). נבחין שבאופן כללי, חלוקת פולינום $p$ בפולינום קבוע $q = \ag$ שקולה להכפלת $p$ פי $\frac{1}{\ag}$ (עקרונית צריך להוכיח את זה, אבל זה סעיף חישובי אז אני מניח שאין צורך). מכאן נקבל: 
		\[ p \mid q = \frac{1}{6}\cl{x^{3} + x^{2} - 5} = \frac{1}{6}x^{3} + \frac{1}{6}x^{2} - \frac{5}{6} \]
		נבחין שהתוצאה מוגדרת היטב שכן אנו עובדים בשדה בו $\chr \F \neq 2, 3$. 
	\end{enumerate}
	
	\section{}
	יהי $p \in \R[x]$ ונניח ששארית החלוקה ב־$x + 1$ היא $2$ ושארית החלוקה ב־$x+ 2$ היא $1$. נמצא את שארית החלוקה ב־$(x+ 1)(x +2)$. 
	
	נבחין ש־$p = b(x + 1) + 2$ וכן $p = a(x+2) + 1$ כאשר $a, b \in \F[x]$. מהמשפט הקטן של בזו $-1$ שורש של $p - 2$ וכן $-2$ שורש של $p - 1$. נתבונן בחלוקה $p = c(x + 1)(x + 2) + r$. נבחין ש־: 
	\[ \begin{cases}
		2 = 2 + a(-1 + 1) = p(-1) = \cancel{c(1 - 1)(2 - 1)} + r(-1) \implies r(-1) = 2 \\ 
		1 = 1 + b(-2 + 2) = p(-2) = \cancel{c(1 - 2)(2 - 2)} + r(-2) \implies r(-2) = 1 \\ 
	\end{cases} \]
	אלו למעשה שתי הדרישות היחידות שהצבנו על $r$ (שכן, אם הוא מקיים את שתי המשוואות לעיל, השארית בהכרח נשמרת). נבחין ש־$r$ לא קבוע, כי אז אותו הקבוע הוא גם $2$ וגם $1$. יש לנו מערכת של שתי משוואות, וראינו שלא קיים לה פתרון מסדר $1$, ננסה למצוא פולינום מסדר $2$. 
	\[ r = \ag x + \bg \quad \begin{cases}
		-\ag + \bg = r(-1) = 2 \\
		-2\ag + \bg = r(-2) = 1
	\end{cases} \bg = \ag + 2 \implies -2\ag + (\ag + 2) = 1 \implies -\ag + 2 = 1 \implies \begin{cases}
		\ag = 1 \\
		\bg = 1 + 2 = 3
	\end{cases} \]
	סה''כ $r = x + 3$ השארית. 
	
	\section{}
	נמצא את המחלק המשותף המקסימלי של הפולינומים הבאים, ונציג אותם כצירוף לינארית שלהם עם מקדמים מ־$\R[x]$. 
	\begin{enumerate}[(A)]
		\item נגדיר $p_1 = 2x^{3} + 4$ וכן $p_2 = x^{2} - 2$. ניעזר באלגוריתם אויקלידס. נחשב את $2x(x^2 - 2) = 2x^3 - 4x$ ונקבל $x^{3} + 4 - 2(x^{2} - 4) = 4x + 4$. סה''כ שארית חלוקה $4x + 4$. נחשב את $\gcd(p_2, 4x + 4)$. לשם כך נחלק ב־$p_2$ את $4x + 4$. נחשב את $\frac{1}{4}x(4x + 4) = x^{2} + x$ ונקבל $p_2 - \frac{1}{4}(4x + 4) = -x - 2$. עתה נחשב את $\gcd(4x + 4, -x - 2)$. נבחין שהם מתחלקים אחד בשני, כלומר $4x + 4 = -2(x - 2)$. סה''כ $\gcd(4x + 4, -x - 2) = 4x + 4$. נסכם ש־$\gcd(p_2, p_1) = x + 1$. נציג אותם כצירוף לינארי: 
		\[ \cl{-\frac{x}{2}}(x^{2} - 2) + \frac{1}{4}(2x^{3} + 4) = \frac{1}{4}\cl{\cancel{-2x^{3} + 2x^{3}} + 4x + 4} = x + 1 \]
		כנדרש. 
		\item נגדיר $p_2 = x^{3} - 1$ ו־$p_1 = x^{4} - 1$. די קל לפרק את הפולינומים האלו ולמצוא כך את ה־$\gcd$ ללא אלגוריתם אויקלידס (זאת כי ניכר ש־$1$ שורש של שניהם). 
		\begin{gather*}
			p_1 = x^{4} - 1 = (x^{2} - 1)(x^{2} + 1) = (x - 1)(x + 1)(x^{2} + 1) \\
			p_2 = x^{3} - 1 = (x - 1)(x^{2} + x + 1)
		\end{gather*}
		משום ש־$x^{2} + x + 1$ ו־$x^{2} + 1$ אינם פריקים ב־$\R$ (הדיסקרימיננטה של שניהם שלילית), הפירוק לראשוניים מופיע לעיל. ה־$\gcd$ הוא כפל הראשוניים הזהים בשניהם (כולל ריבוי), וסה''כ $\gcd(p_1, p_2) = x - 1$. ואכן בהתאם למשפט בזו (האחר): 
		\[ (-x)(x^{3} - 1) + (x^{4} - 1) = -x^{4} + x + x^{4} - 1 = x - 1 \]
		כנדרש. 
	\end{enumerate}
	
	
	\section{}
	\begin{enumerate}[(A)]
		\item יהי $p = \sumni \ag_i x_i \in \Z[x]$ פולינום בעל מקדמים שלמים. יהי $\frac{a}{b} \in \Q$ שורש מצומצם של $p$. נוכיח ש־$b \mid \ag_n$ וכן $a \mid \ag_0$. \begin{proof}
			נבחין ש־: 
			\begin{align}
				0 = p\cl{\frac{a}{b}} = \sumni \ag_i \cl{\frac{a}{b}}^{i} = \sumni \ag_i {\frac{a^{i}}{b^{i}}} &= \ag_0 + \sumnio \ag_i{\frac{a^{i}}{b^{i}}} \\
				&= \ag_n \cdot \frac{b^{n}}{a^{n}} + \sum_{i = 0}^{n - 1}\ag_i\frac{a^{i}}{b^{i}}
			\end{align}
			נעביר אגפים, נקבל: 
			\setcounter{equation}{0}
			\[ \begin{WithArrows}[format=rlCrl]
				\ag_0 &= \sum_{i = 1}^{n}\ag_i \frac{a^{i}}{b^{i}} &\ \overset{\cdot b^{n}}{\implies} &b^{n} \cdot \ag_0 &\ \overset{(1)}{=} \sumnio \ag_{i}{a^{i}}b^{n - i} \\
				\frac{\ag_n a^{n}}{b^{n}} &= \sum_{i = 0}^{n - 1}\ag_i \frac{a^{i}}{b^{i}} &\ \overset{\cdot b^{n}}{\implies} & a^{n} \cdot \ag_n &\ \overset{(2)}{=} \sum_{i = 0}^{n -1}\ag_{i}a^{i}b^{n -1 - i}
			\end{WithArrows} \]
			עתה ניעזר בכך ש־$a, b$ זרים מהיות $\frac{a}{b}$ מצומצם. ניעזר במשפט היסודי של האלגברה. 
			\begin{itemize}
				\item בעבור $(1)$, מהמשפט היסודי של האלגברה בהכרח $b^{n}$ ``מביא'' $n$ פעמים את $b$ בפירוק הראשוני של האגף הימני, אך $b$ מופיע יותר פעמים מכך. לכן $b \mid \ag_0$ בהכרח. 
				\item באופן דומה, ב־$(2)$ בהכרח $a^{n}$ ``מביא'' $n$ פעמים את $a$ לפירוק הראשוני, אך $a$ מופיע יותר פעמים לכך. מכאן לכן $a \mid \ag_n$. 
			\end{itemize}
			סה''כ הוכחנו את הנדרש. 
		\end{proof}
		\item נמצא את כל השורשים הממשיים של $p = 2x^{4} - 2x^{3} - 15x^{2} + 24 x - 9$ ונפרק אותו כפולינום ב־$\R[x]$ וכן כפולינום ב־$\Q[x]$. 
		
			נתפלל לאלוהים שלפולינום הזה יש שורש רציונלי. במקרה זה, יהיה ניתן לפרק אותו בעזרת ''ניחוש`` של השורש באמצעות המשפט מהסעיף הקודם. במידה וקיים שורש כזה בצורתו המצומצמת $\frac{a}{b}$, מתקיים $a \mid \ag_0 = 9$ וכן $b \mid \ag_n = 2$. מכאן ש־$b \in \{1, 2\}$ ו־$a \in \{1, 2, 3, 9\}$, עד לכדי חברות (כאשר משמידים פולינומים לא מצומצמים, נקבל $7$ אפשרויות, כפול סימן סה''כ $14$ ניחושים). נתחיל מלנסות את המקרה הפשוט ביותר בו $a = b = 1$. 
			\[ p(1) = 2 \cdot 1^{4} - 2 \cdot 1^{3} - 15 \cdot 1^{2} + 24 \cdot 1 - 9 = 0 \]
			התמזל מזלנו ואכן $1$ שורש. נבצע חלוקת פולינומים (ללא שארית) ונקבל: 
			\[ p = (x - 1)(2x^{3} - 15x + 9) \]
			
			בשלב הזה אפשר להיעזר בנוסחה הסגורה למציאת שורשים של פולינום ממעלה שלישית, שאף אחד לא זוכר, או להמשיך לנחש. מכיוון שאלו אותם המקדמים כמו הפולינום הראשון, לא יעזור להפעיל את המשפט שוב, ופשוט ננסה את כל $13$ המקרים עד שנגלה ש־$a = 3$ ו־$b = -1$ עובד: 
			\[ 2(-3)^{3} - 15(-3) + 9 = 0 \]
			נבצע שוב חלוקת פולינומים של $2x^{3} - 15x + 9$ (ממשפט בזו גם כאן ניוותר ללא שארית) ונקבל: 
			\[ p = (x - 1)(x + 3)(2x^{2} - 6x + 3) \]
			זהו פירוק מעל $\Q[x]$. נמשיך לפרק את הפולינום ב־$\R[x]$ באמצעות נוסחת השורשים, ונקבל ש־$\frac{3\pm \sqrt{3}}{2}$ שורשים. ממשפט בזו: 
			\[ p = (x - 1)(x + 3)\cl{x - \frac{3 + \sqrt{3}}{2}}\cl{x + \frac{3 - \sqrt{3}}{2}} \]
			משום ש־$\frac{3\pm\sqrt{3}}{2}$ אי־רציונלי, נקבל שבהכרח לא קיים פירוק רציונלי קטן יותר מ־$(x - 1)(x + 3)(2x^{2} - 6x + 3)$ (כי אחרת לפולינום יש $4$ שורשים רציונליים, אך יש לו $2$ שורשים רציונליים ב־$\R[x]$, ו־$\Q[x]$ ניתן לשיכון ב־$\R[x]$ וכן הפירוק הראשוני ב־$\R[x]$ חוג אוקלידי יחיד, וממשפט בזו סתירה). 
	\end{enumerate}
	
	\ndoc
	
\end{document}
