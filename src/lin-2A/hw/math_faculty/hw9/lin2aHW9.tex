\documentclass[]{../../../../../tex/classes/homework}
\usepackage{../../../../../tex/packages/hebrewSupport}
\usepackage{../../../../../tex/packages/mathShortcuts}

\renewcommand\mut[2] {\left \la #1, #2 \right \ra}

\author{שחר פרץ}
\title{לינארית 2א $\sim$ \textit{תרגיל בית 9}}
\begin{document}
	\maketitle
	
	\section{}
	נתבונן במטריצה הבאה: 
	\[ A = \pms{1 & c & 4 \\ 0 &2 & b \\ 0 & 0 & a} \]
	עבור $a, b, c \in \Q$ רציונליים כלשהם. נמצא עבור אילו ערכים $A$ לכסינה מעל $\Q$. 
	ראשית, נמצא את הפולינום האופייני של המטריצה: 
	\[ p_A(x) = \det(A - Ix) = (1 - x)(2 - x)(a - x) \]
	אם $a \neq 2$, אזי קיבלנו $3$ ע''עים שונים וסיימנו. אחרת, $a = 2$, ואז נדרוש שהריבוי הגיאומטרי של הע''ע $2$ הוא $2$ (כדי שנוכל לבנות בסיס מלכסן). לשם כך: 
	\[ 2 = \dim \nc\cl{A - 2I} = \dim \nc\pms{-1 & c & 4 \\ 0 & 0 & b \\ 0 & 0 & 0} \iff b = 0 \]
	סה''כ, המטריצה לכסינה אם $a = 0 \lor (a \neq 0 \land b = 0)$. לטענה זו שקילות לוגית לכך ש־: 
	\[ \bm{a = 0 \lor b = 0} \]
	
	\section{}
	תהי $B \in M_n(\F)$ ו־$a_0 \dots a_n \in \F$ סקלרים. נסמן $f = a_nx^{n} + \cdots + a_1x^{0}$ ונניח $f(B) = 0$. 
	\begin{itemize}
		\item נוכיח שאם $\ml$ ע''ע של $B$, אז $f(\ml) = 0$. \begin{proof}
			אם $\ml$ ע''ע של $B$ עבור ו''ע $v$ נבחין ש־$B^{i}v = B^{i - 1}(\ml v) = \ml^{i} v = \cdots \ml^{i}v$ לכל $i \in \N$ (באינדוקציה). לכן: 
			\[ f(\ml)\cdot v = \sumni a_i\ml^{i} = \sumni a_i\cl{B^{i}v} = \cl{\sumni a_iB^{i}}v = 0 \cdot v = 0 \]
			משום שמהגדרת ע''ע $v \neq 0$  אז $f(\ml) = 0$ כדרוש וסיימנו. 
		\end{proof}
		\item תהי $A \in M_n(\F)$ ונניח $A^{2} = 4A - 3I$. נוכיח שאם $\ml \in \R$ אז $\ml = 1 \lor \ml = 3$. 
		\begin{proof}
			מן ההוכחה לעיל כל פולינום $\deg f \le n$ (לא הכרחי $\deg f \le n$ אבל רק מקרה זה הוכחנו) שמאפס את $A$ מחלק את הע''ע של $A$. נבחין שבעבור $f = x^{2} - 4x + 3$ נקבל:
			\[ A^{2} = 4A - 3I \implies A^{2} - 4A + 3I = 0 \implies f(A) = 0 \]
			סה''כ $f$ מאפס את $A$. נניח $\ml$ ע''ע של $A$. מהנימוקים לעיל, תנאי הכרחי לכך הוא ש־$f(\ml) = 0$. נבחין ש־$f = (x - 3)(x - 1)$ וממשפט בזו ומיחידות הפירוק לגורמים אי־פריקים בתחום אוקלידי, בהכרח $\ml = 3 \lor \ml = 1$, כדרוש. 
		\end{proof}
	\end{itemize}
	
	\section{}
	נאמר ש־$\ml$ ע''ע שמאלי של $A \in M_n(\F)$ אם קיים $0 \neq v \in \F^{n}$ כך ש־$v^TA = \ml v^T$, ו־$v$ יקרא ו''ע שמאלי. 
	\begin{enumerate}[(A)]
		\item נוכיח: $\ml$ הוא ע''ע אמ''מ הוא ע''ע שמאלי. \begin{proof}\,
			\begin{itemize}
				\item[$\implies$] נוכיח שאם $v$ ע''ע שמאלי אז הוא ע''ע של $A$. יהי $v$ ע''ע שמאלי של $A^{T}$, מכאן קיים $v$ כך ש־$v^TA^T = \ml v^T$. נפעיל שחלוף על שני הצדדים ונקבל $Av = (v^TA^T)^T = (\ml v^T)^T = \ml v$ כלומר $\ml$ ע''ע של $A$ כדרוש. 
				\item[$\impliedby$] עקרונית הכיוון השני מדואליות שחלוף אבל נוכיח את מפורשות. יהי $v$ ע''ע של $A$. אזי $Av = \ml v$ ומכאן שהוא ע''ע שמאלי של $A^{T}$ כי $v^{T}A^T = \ml v^T$. לכן $p_{A^T}(\ml) = 0$. משום ש־$p_{A^T}(x) = p_{A}(x)$ בפרט $p_{A}(\ml) = 0$ וסה''כ $\ml$ ע''ע של $A$ שכן הוא מאפס את הפולינום האופייני. 
			\end{itemize}
		\end{proof}
		\item נפריך: $v$ ו''ע עצמי אמ''מ הוא ו''ע שמאלי. נתבונן במטריצה הבאה: 
		\[ A = \pms{1 & 1 \\ 0 & 1} \quad v = \pms{1 \\ 0} \quad Av = \pms{1 \\ 0} = 1 \cdot v \]
		עם זאת: 
		\[ v^T\pms{1 & 1 \\ 0 & 1} = \pms{1 & 0}\pms{1 & 1 \\ 0 & 1} = \pms{1 & 1} \neq \lg v \ \forall \lg \in \F \]
	\end{enumerate}
	
	\section{}
	יהי $V$ מ''ו מעל שדה $\F$ ויהיו $T< S \co V \to V$ העתקות לינאריות וכן $\lg_T, \lg_S$ ע''ע של $T, S$ בהתאמה. נוכיח או נפריך את הטענות הבאות. 
	\begin{enumerate}[(A)]
		\item נפריך את הטענה $\lg_T + \lg_S$ ע''ע של $T + S$. \begin{proof}[הפרכה]
			נתבונן בהעתקות הבאות (טכנית, בהעתקות המוגדרות ע''י הכפל במטריצות הבאות): 
			\[ A = \pms{1 & 1 \\ 0 & 1} \quad B = \pms{1 & 0 \\ 1 & 1} \]
			נבחין ש־$1$ ע''ע לשתיהן בעבור $\binom{1}{0}$ ו־$\binom{0}{1}$ בהתאמה. אך: 
			\[ p_{A + B}(x) = \det \cl{\pms{2 & 1 \\ 1 & 2} - Ix} = (x - 2)^{2} - 1 = x^{2} - 4x + 3 \]
			פולינום ש־$1 + 1 = 2$ אינו שורש שלו ולכן לא ע''ע שלו. 
		\end{proof}
		\item נפריך את הטענה $\lg_T\lg_S$ ע''ע של $T \circ S$. \begin{proof}
			נתבונן בהעתקה הבאה: 
			\[ T(v) = \pms{1 & 1 \\ 0 & 1}v \quad S(v) = Iv = \pms{1 & 0 \\ 1 & 1}v \]
			לשניהם ע''ע $1$. אך נבחין ש־: 
			\[ (T \circ S)(v) = \pms{1 & 1 \\ 0 & 1}\pms{1 & 0 \\ 1 & 1}v = \pms{2 & 1 \\ 1 & 1}v  \]
			הפולינום האופייני של העתקה זו הוא $(2 - x)(1- x) - 2$ ונבחין ש־$1$ (כפל הע''עים) לא מאפס את הפולינום. לכן הוא אינו ע''ע שלה כדרוש. 
		\end{proof}
		\item נוכיח את הטענה $\lg_T^2$ ע''ע של $T \circ T$. \begin{proof}
			נתבונן ב־$v$ ו''ע של $\lg_T$. נבחין ש־: 
			\[ (T \circ T)(v) = T(T(v)) = T(\lg_T v) = \lg_T Tv = \lg_T^{2}v \]
			ומכאן ש־$\lg_T^{2}$ ע''ע של $T \circ T$ כדרוש. 
		\end{proof}
		\item נוכיח ש־$\ag \lg_T$ ע''ע של $\ag T$ לכל $a \in \F$. \begin{proof}
			נסמן ב־$v$ ו''ע של $\lg$ ב־$T$. מכאן ש־: 
			\[ (\ag T)v = \ag (Tv) = (\ag \lg) v \]
			כנדרש. 
		\end{proof}
	\end{enumerate}
	
	\section{}
	יהי $V = \C_n$ מרחב הפולינומים ממעלה לכל היותר $n$. נגדיר העתקה $T \co V \to V$ ע''י $(T(p))(x) = p'(x) + p(0)x^{n}$. נוכיח שהיא העתקה לכסינה. 
	\begin{proof}
		נתבונן בבסיס $e_i = x^{i}$ למרחב הפולינומים ($i \in \{0 \dots n\}$). נבחין ש־$T(e_i) = T(x^{i}) = ix^{i - 1} + x^{n}$ (שכן $e_i(0) = 0$) אלא אם $i = 0$ ואז $e_0(0) = 1, e_0'=0$ ונקבל $T(e_0) = 1$. מכאן שהמטריצה המייצגת בבסיס זה היא (זוהי מטריצה $(n + 1) \times (n + 1)$): 
		\[ [T]_\ec = \pms{0 & 0 & \cdots & 0 & 1 \\ n & 0 & 0 & \cdots & 0 \\ 0 & n - 1 & 0 & \ddots & \vdots \\ \vdots & \ddots & \ddots & \ddots & 0 \\ 0 & \cdots & 0 & 1 & 0} \]
		מכאן שהפולינום האופייני שלה הוא (נפתח דטרמיננטה לפי השורה העליונה): 
		\begin{multline*}
			p_T(x) = \det(Ix - [T]_\ec) = \detms{x & 0 & \cdots & 0 & -1 \\ -n & x & 0 & \cdots & 0 \\ 0 & 1 - n & x & \ddots & \vdots \\ \vdots & \ddots & \ddots & \ddots & 0 \\ 0 & \cdots & 0 & -1 & x} = 
			x \detms{x & 0 & 0 & \cdots & 0 \\1 - n & x & \ddots & \ddots & \vdots \\ 0 & 2 - n & \ddots & \ddots & 0 \\ \vdots & \ddots & \ddots & x & 0 \\ 0 & \cdots & 0 & -1 & x}
			+ 1\detms{-n & x& 0 & \cdots & 0 \\ 0 & 1-n & x & \ddots & \vdots \\ \vdots & \ddots & \ddots & \ddots & 0 \\ 0 & \cdots & 0 & -2 & x \\ 0 & \cdots & 0 & 0 & -1} 
			\\ = x \cdot x^{n} + 1 \cdot 1 \cdot \prod_{i = 1}^{n}(-i) = x^{n + 1} + (-1)^{n}n!
		\end{multline*}
		נוכיח שלפולינום $x^{n + 1} + (-1)^{n}n!$ יש $n + 1$ שורשים מעל המרוכבים. נטען שלכל $k \in [n + 1]$ (יש $n + 1$ כאלו), המספר הבא הוא שורש של הפולינום: 
		\[ x_k = \sgn\cl{(-1)^{n}}\!\!\sqrt[n + 1]{n!}\cl{\cos{\frac{2\pi k}{n + 1}} + \sin \frac{2 \pi k}{n + 1}} = \sqrt[n + 1]{n!}e^{\frac{2\pi ik}{n + 1}} \]
		(אינטואציה: שורשי יחידה לא מנורמלים) $\sgn$ מסמן את הסימן של המספר שהוא מקבל, שבמקרה הזה שווה ל־$-1$ אם $n$ אי־זוגי ול־$1$ אחרת. 
		כאשר $\sqrt[n]{n!}$ קיים כי לכל מספר ממשי יש שורש. נוכיח: 
		\begin{multline*}
			x_k^{n + 1} + (-1)^{n}n! = \cl{\sgn\cl{(-1)^{n + 1}}\!\!\sqrt[n + 1]{n!}e^{\frac{2\pi ik}{n + 1}}\!}^{n + 1} \dequad\dequad + (-1)^{n}n! = \underbrace{\cl{\!\!\sqrt[n + 1]{n!}}^{n}}_{n!} \!\!\cdot \overbrace{e^{2\pi \frac{ik}{\cancel{n + 1}} \cdot \cancel {(n + 1)}}}^{\mathclap{e^{2\pi ik} = \cos(2 \pi k) + i\sin(2\pi k) = 1 + 0i = 1}} \cdot \underbrace{\cl{\sgn((-1)^{n})}^{n}}_{(-1)^{n + 1}} - (-1)^{n} \\
			= n! \cdot 1 \cdot (-1)^{n + 1} + (-1)^{n}n! = n! - n! = 0
		\end{multline*}
		עוד נבחין ש־$x_k = x_j \iff j = k$, שכן הזווית היא $\arctan$ (פונקציה חח''ע) של המעריך, שמשתנה בעבור כל $k$ שונה. מכאן של־$p_T(x)$ פולינום ממעלה $n + 1$ יש $n + 1$ שורשים שונים וסה''כ המטריצה לכסינה ממשפט. 
	\end{proof}
	
	(אני לא ממש התעסקתי במרוכבים בחיים שלי אז מקווה שאין בעיות מהותיות בהוכחה)
	
	\section{}
	נגדיר את ההעתקה $T \co M_n(\C) \to M_n(\C)$ הנתונה ע''י $T(A) = A^{T}$. נראה ש־$T$ לכסינה. 
	\begin{proof}
		נבחין שלכל $A\in \sym_n(\F)$ מתקיים $T(A) = A^T = A$ ולכל $A \in \asym_n(\F)$ מתקיים $T(A) = A^T = -A$. ידוע ש־$M_n(\C) = \sym_n(\C) \oplus \asym_n(\C)$. לכן נוכל לקחת בסיס $A_{1} \dots A_n$ כך ש־$A_{1} \dots A_{\binom{n}{2}}$ א־סימטריות ו־$A_{\binom{n}{2} + 1} \dots A_n$ סימטריות. בעבור הסימטריות נקבל ע''ע $1$ ובעבור האנטי־סימטריות נקבל ע''ע $-1$. סה''כ מצאנו בסיס מלכסן, כאשר $\sym_n(\C)$ המ''ו העצמי של $-1$ ו־$\asym(\C)$ המ''ו העצמי של $-2$. 
	\end{proof}
	
	\section{}
	עבור המטריצות הבאות, נקבע אם הן נורמליות, הרמיטיות או אוניטריות. 
	\begin{enumerate}[(A)]
		\item נתבונן במטריצה הבאה: 
		\[ A = \pms{1 & 1 & 1 \\ 1 & 0 & 1 \\ 0 & 1 & 1} \]
		נבדוק אם היא נוראמלית: 
		\[ AA^{*} = \pms{3 & 2 & 2 \\ 2 & 2 & 1 \\ 2 & 1 & 2} \neq \pms{2 & 1 & 2 \\ 1 & 2 & 2 \\ 2 & 2 & 3} = A^{*}A \]
		מכאן שהיא איננה נורמאלית. היא בבירור איננה אוניטרית שכן הנורמה של השורה הראשונה היא $\sqrt{2}$ ולא $1$, ובפרט איננה הרמטית שכן היא איננה נורמאלית. 
		\item נתבונן במטריצה הבאה: 
		\[ A = \pms{3 & 0 & 0 \\ 0 & \frac{1}{\sqrt 2} & -\frac{1}{\sqrt 2} \\ 0 & \frac{1}{\sqrt 2} & \frac{1}{\sqrt 2}} \]
		נבדוק אם היא נורמאלית: 
		\[ AA^{*} = AA^{T} = \pms{9 & 0 & 0 \\ 0 & 1 & 0 \\ 0 & 0 & 1} = A^{T}A = A^{*}A \]
		סה''כ היא נורמאלית ולכן לכסינה אורתוגונלית מעל $\C$. אך היא איננה צמודה לעצמה, שכן $A_{23} \neq A^{T}_{23}$. מכאן שהיא איננה לכסינה אורתוגונלית מעל $\R$. 
		\item נתבונן במטריצה הבאה: 
		\[ A = \pms{2 & 2 & 2 \\ 2 & 2 & 2 \\ 2 & 2 & 2} \]
		נבחין שהיא צמודה לעצמה שכן $A^{T} = A$. מכאן שבפרט היא נורמאלית. היא איננה אוניטרית שכן איננה משמרת את נורמת הוקטור $(1, 0, 0)$. בגלל שהיא צמודה לעצמה ניתן למצוא לה לכסון אורתוגונלי מעל $\R$. לשם כך, נמצא את הפולינום האופייני: 
		\begin{gather*}
			p_A(x) = \det(Ix - A) = \detms{x - 2 & -2 & -2 \\ -2 & x - 2 & -2 \\ -2 & -2 & x - 2} = (x - 2)\detms{x -2 & -2 \\ -2 & x - 2} - 2\detms{-2 & -2 \\ -2 & x - 2} - 2 \detms{-2 & x - 2 \\ -2 & -2} \\
			= (x - 2)((x - 2)^{2} + 4) + \cancel{2(2(x - 2) + 2^{2})} + \cancel{2(2^{2}-2(x - 2))} = x^{3} - 6x^{2} + 16x - 16x = x^{2}(x - 6)
		\end{gather*}
		נמצא בסיס למ''ו העצמי של $0$: 
		\[ \nc(A - 0I) = \nc\pms{2 & 2 & 2 \\ 2 & 2 & 2  \\ 2 & 2 & 2} = \nc\pms{2 & 2 & 2 \\ 0 & 0 & 0 \\ 0 & 0 & 0} = \nc\pms{1 & 1 & 1 \\ 0 & 0 & 0 \\ 0 & 0 & 0} = \ccb{\pms{-x + -y \\ -x \\ -y} \mid x, y \in \R} = \Sp\ccb{\pms{- 1\\ 0 \\ 1}, \pms{-1 \\ 1 \\ 0}} \]
		נעשה גרם־שמידט: 
		\[ \pms{-1 \\ 0 \\ 1} \mapsto \pms{-1 \\ 0 \\ 1} \cdot \frac{1}{\sqrt {1 + (-1)^{2} + 0^{2}}} = \pms{-\frac{1}{\sqrt 2} \\ 0 \\ \frac{1}{\sqrt 2}} =: v_1 \]
		אחרי שנרמלנו: 
		\[ \pms{-1 \\ 1 \\ 0} - \mut{\pms{-1 \\ 1 \\ 0}}{v_1}v_1 = \pms{-1 \\ 1 \\ 0} - \frac{1}{\sqrt 2}\pms{-\frac{1}{\sqrt 2} \\ 0 \\ \frac{1}{\sqrt 2}} = \pms{-\frac{1}{2} \\ 1 \\ -\frac{1}{2}} \]
		ננרמל שוב: 
		\[ \pms{-\frac{1}{2} \\ 1 \\ -\frac{1}{2}} \mapsto \pms{-\frac{1}{2} \\ 1 \\ -\frac{1}{2}} \cdot \frac{1}{\sqrt{0.5^{2} \cdot 2 + 1^{2}}} = \pms{-\frac{1}{\sqrt 6} \\ \frac{2}{\sqrt 6} \\ - \frac{1}{\sqrt 6}} =: v_2 \]
		נותר לנו למצוא את המ''ו העצמי של ע''ע $6$, שממשפט הפירוק הספקטרלי מאונך למרחב העצמי של ע''ע $0$. 
		\[ \nc{(A - 6I)} = \nc\pms{-4 & 2 & 2 \\ 2 & -4 & 2 \\ 2 & 2 & -4} = \nc\pms{-2 & 1 & 1 \\ 0 & -3 & 3 \\ 0 & 3 & -3} = \pms{-2 & 1 & 1 \\ 0 & -1 & 1 \\ 0 & 0 & 0} = \nc\pms{1 & 0 & -1 \\ 0 & 1 & -1 \\ 0 & 0 & 0} = \Sp{\pms{1 \\ 1 \\ 1}} \]
		ננרמל את הוקטור האחרון שקיבלנו: 
		\[ \pms{1 \\ 1 \\ 1}\mapsto \pms{1 \\ 1 \\ 1} \cdot \frac{1}{\sqrt{1^{2} \cdot 3}} = \pms{\frac{1}{\sqrt 3} \\ \frac{1}{\sqrt 3} \\ \frac{1}{\sqrt 3}} =: v_3 \]
		כלומר הבסיס הבא אורתונורמלי מלכסן: 
		\[ v_1, v_2, v_3 = \pms{-\frac{1}{\sqrt 2} \\ 0 \\ \frac{1}{\sqrt 2}}, \pms{-\frac{1}{\sqrt 6} \\ \frac{2}{\sqrt 6} \\ - \frac{1}{\sqrt 6}}, \pms{\frac{1}{\sqrt 3} \\ \frac{1}{\sqrt 3} \\ \frac{1}{\sqrt 3}} \]
		כאשר $v_1, v_2$ משוייכים לע''ע $0$ ו־$v_3$ לע''ע $6$ של המטריצה הנתונה בתחילת השאלה. 
	\end{enumerate}
	
	\section{}
	תהי $A \in M_n(\C)$ מטריצה עם ע''ע $\lg_1 \dots \lg_n$ כך ש־$\sof{\lg_i} = 1$. נוכיח/נפריך ש־$A$ אוניטרית. 
	\begin{proof}[הפרכה]
		נתבונן במטריצה הבאה: 
		\[ B = \pms{1 & 1 \\ 0 & 1} \quad A = B\op \pms{1 & 0 \\ 0 & -1}B = \pms{1 & 2 \\ 0 & -1} \]
		נבחין שמשום ש־$B$ הפיכה אז $A \sim \diag(1, -1)$, ומכאן שיש לה ע''ע $\lg_1 = 1, \lg_2 = -1$. אך הוקטור $\binom{0}{1}$ לא שומר על הנורמה שלו: 
		\[ \norm{\pms{1 & 2 \\ 0 & -1}\pms{0 \\ 1}} = \norm{\pms{2 \\ -1}^{T}} = \sqrt{2^{2} + 1} = \sqrt 5 \neq 1 = \norm{\pms{0 \\ 1}} \]
		דוגמה נגדית מתאימה. 
	\end{proof}
	
	\ndoc
	
\end{document}
