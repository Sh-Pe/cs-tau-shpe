\documentclass[]{../../../../../tex/classes/homework}
\usepackage{../../../../../tex/packages/hebrewSupport}
\usepackage{../../../../../tex/packages/mathShortcuts}

\renewcommand\mut[2] {\left \la #1, #2 \right \ra}

\author{שחר פרץ}
\title{לינארית 2א $\sim$ \textit{תרגיל בית 6}}
\begin{document}
	\maketitle
	\section{}
	יהי $V = \C_{\le 2}[x]$ עם המכפלה האוקלידית: 
	\[ \mut{p}{q} = \int^{1}_0 p(x)\ol{q(x)} \dx \]
	נעשה גרם־שמידט על $(1, x + i, x^{2})$. (לא סיימתי חדו''א 1א עדיין אבל אני מניח שאני אסתדר). ננרמל תוך כדי ביצוע התהליך כדי לחסוך חלוקה בנורמה. 
	\begin{proof}[גרם־שמידט]
		כאשר נחשב את האינטגרלים, נתעלם מהצמוד (כלומר נחשב את $\int^{0}_1 p(x)q(x) \dx$), כי האינטגרל הוא בין $0$ ל־$1$, שניהם מספרים ממשיים, והצמוד משפיע רק על החלק המרוכב – כלומר הצבה של $0$ או $1$ בביטוי האינטגרל לא תשתנה כתוצאה מקיומו של הצמוד. נבחין ש־$\int^{1}_0 1 \cdot \bar 1\dx = 1$ כלומר $1$ וקטור מנורמל. נסמן $\bar v_1 = 1$. 
		\[ \bar v_2 = x + i - \mut{x + i}{v_1}v_1 = (x + i) - \cl{\frac{1}{2} + i} = x - \frac{1}{2} \]
		ננרמל: 
		\[ \norm{\bar v_2}^{2} = \int^{1}_0 \cl{x - \frac{1}{2}}^{2} = \frac{\cl{x - \frac{1}{2}}^{3}}{3}\Big|^{1}_0 = \frac{1}{12} \quad v_2 = \frac{\bar v_2}{\norm{\bar v_2}} = \sqrt{12}\cl{x - \frac{1}{2}} \]
		נחשב את האינטגרלים הבאים בנפרד: 
		\[ \mut{x^{2}}{v_1} = \mut{x^{2}}{1} = \int^{1}_0 x^{2}\dx = \frac{1}{3}\Big|^{1}_0 = \frac{1}{3} \]
		\[ \mut{x^{2}}{v_2} = \mut{x^{2}}{\sqrt{12}(x - 0.5)} = \sqrt{12}\int^{1}_0 x^{2}\cl{x - \frac{1}{2}}\dx = \sqrt{12} \cdot \frac{x^{3}(3x - 2)}{12} \Big|^{1}_0 = \frac{1}{\sqrt{12}} \]
		עתה נחשב את הוקטור שנותר: 
		\[ \bar v_3 = x^{2} - \mut{x^{2}}{v_1}v_1 - \mut{x^{2}}{v_2}v_2 = x^{2} - \frac{1}{3} - \frac{1}{\sqrt{12}}\cl{x - \frac{1}{2}} = x^{2} - \frac{x}{\sqrt {12}} + \cl{\frac{1 - 12 \sqrt 3}{4 \sqrt 3}} \]
		זה כבר הוקטור האחרון, אז נוותר על לנרמל שוב. סה''כ הוקטורים הבאים בסיס אורתוגונלי של $V$: 
		\[ v_1 = 1 \quad \quad v_2 = \sqrt{12}\cl{x - \frac{1}{2}} \quad\quad v_3 = x^{2} - \frac{x}{\sqrt {12}} + \cl{\frac{1 - 12 \sqrt 3}{4 \sqrt 3}} \]
	\end{proof}
	
	\section{}
		יהי $V$ ממפנ''ס מעל $\C$ כך ש־$\dim V = n$. נוכיח שבהינתן $v_1 \dots v_k \subseteq V$ אורתוגונלית, ניתן להשלים אותה לבסיס אורתוגונלי של $V$, כלומר קיימים $v_{k + 1} \dots v_n \in V$ כך ש־$v_1 \dots v_n$ אורתוגונלי. 
		\begin{proof}
			נשלים אותה לבסיס רגיל ע''י הוקטורים $\bar v_{k + 1} \dots \bar v_n$. נבצע עליהם תהליך גרם־שמידט, כלומר נגדיר רקורסיבית: 
			\[ v_{m > k} = \bar v_{m} - \sum_{i = 1}^{m - 1}\frac{\mut{\bar v_m}{v_i}}{\norm{v_i}}v_i \]
			מנכונות תהליך גרם־שמידט, סה''כ קיבלנו $v_k \dots v_n$ אותוגונליים, כך ש־$v_1 \dots v_n$ בסיס אורתוגונלי של $V$. השתמשנו בנוצרותו הסופית של $V$ כאשר הנחנו ש־$n$ מוגדר היטב. 
		\end{proof}
		
	\section{}
	\begin{enumerate}[(A)]
		\item נמצא $v_1, v_2, v_3 \in \R^{2}$ כך ש־$ i \neq j \iff v_i \cdot v_j < 0$. 
		
		נגדיר: 
		\[ v_1 = \pms{1 \\ 1} \quad v_2 = \pms{0 \\ -1} \quad v_3 = \pms{-1 \\ 0.5} \]
		ואכן: 
		\begin{alignat*}{9}
			\mut{v_1}{v_2} &= 1 \cdot 0 + (-1) \cdot 1 = -1 &\,< 0 \\
			\mut{v_2}{v_3} &= 0 \cdot (-1) + (-1) \cdot 0.5 = -0.5 &\,<0 \\
			\mut{v_3}{v_1} &= (-1) \cdot 1 + 1 \cdot 0.5 = -0.5 &\,<0
		\end{alignat*}
		כדרוש. 
		\item יהי $V$ מ''ו אוקלידי מממד $1$. נוכיח שלא קיימים $v_1, v_2, v_3 \in V$ כך ש־$\mut{v_i}{v_j} < 0 \iff i \neq j$. \begin{proof}
			יהיו $v_1, v_2, v_3 \in V$. נניח $\dim V = 1$, לכן קיימים $\lg_1 \dots \lg_3 \in \F$ כך ש־$\lg_1v_1 = \lg_2v_2 = \lg_3 v_3$ (תלות לינארית). אם $v_i = 0$ אזי $\mut{v_i}{v_1} = 0$ וסיימנו. אחרת $\norm{v_i} \neq 0$, כלומר: 
			\begin{alignat*}{9}
				0 &< \mut{v_1}{v_1} &= \mut{\lg_2 v_2}{\lg_3 v_3} &= \lg_2\lg_3 \mut{v_2}{v_3} \\
				0 &< \mut{v_2}{v_2} &= \mut{\lg_1 v_1}{\lg_3 v_3} &= \lg_1\lg_3 \mut{v_1}{v_3} \\
				0 &< \mut{v_3}{v_3} &= \mut{\lg_1 v_1}{\lg_2 v_2} &= \lg_1\lg_2 \mut{v_1}{v_2}
			\end{alignat*}
			נניח בשלילה $\mut{v_i}{v_i} < 0$. נסיק: 
			\[ \lg_1 \lg_2 <0 \land \lg_2 \lg_3 < 0 \land \lg_3 \lg_1 < 0 \]
			נפרק למקרים. 
			\begin{itemize}
				\item אם $\lg_1 < 0$, אזי $\lg_1\lg_3 < 0 \so v_3 > 0$, ואז $\lg_2 \lg_3 < 0 \so v_2 < 0$, וסה''כ $\lg_1\lg_2 < 0 \so \lg_1 > 0$ – סתירה. 
				\item אם $\lg_1 > 0$, אזי $\lg_1\lg_3 < 0 \so v_3 < 0$, ואז $\lg_2 \lg_3 < 0 \so v_2 > 0$, וסה''כ $\lg_1\lg_2 < 0 \so \lg_1 < 0$ – סתירה. 
			\end{itemize}
			בשני המקרים סתירה, וסיימנו. 
		\end{proof}
		\item רשות (לא היה לי זמן השבוע, אעשה בחנוכה)
		\item רשות (לא היה לי זמן השבוע, אעשה בחנוכה)
	\end{enumerate}
	
	\section{}
	\begin{enumerate}[(A)]
		\item נוכיח שהמטריצות הבאות אינן חופפות: 
		\[ A = \pms{1 &0 \\ 0 & 1} \quad B = \pms{1 & 0 \\ 0 & -1} \]
		\begin{proof}
			נתחיל בלחשב את הדטרמיננטות: 
			\[ \det A = 1 \quad \det B = -1 \]
			נניח בשלילה שהן חופפות. אזי קיימת $M \in M_n(\R)$ הפיכה כך ש־$A = M^TBM$ (הראינו שיחס החפיפה הוא יחס שקילות, ומכאן שאין צורך לטפל במקרה ההפוך). נסמן $c = \det M$. ידוע $\det M = \det M^{T}$. לכן: 
			\[ 1 = \det A = \det (M^TBM) = \det M^T \det B \det M = c^{2} \det B = -c^{2} \]
			משום שאנו מעל הממשיים, $c \in \R$, ומכאן ש־$c^{2} > 0$. סה''כ קיבלנו ש־$1$ הוא השלילי של מספר חיובי, כלומר $1$ מספר שלילי, וסתירה. 
		\end{proof}
		\item יהי $\F_3$. נמצא $A, B \in M_2(\F_3)$ הפיכות שאינן חופפות מעל $\F_3$. \begin{proof}
			נתבונן במטריצות ההפיכות הבאות: 
			\[ A = \pms{1 & 0 \\ 0 & 1} \quad \det A = 1 \quad B = \pms{1 & 0 \\ 0 & 2} \quad \det B = 2 \]
			משום ש־$\det A, \det B \not\equiv_3 0$ אזי שתיהן הפיכות. בדומה לסעיף הקודם, נניח בשלילה שקיימת $M \in M_n(\F)$ הפיכה כך ש־$A = M^TBM$. מכאן ש־$\det A = c^{2}\det B$, בדיוק כמו קודם. נבחין ש־: 
			\begin{itemize}
				\item לא ייתכן $c = 0$ כי $M$ הפיכה ולכן $c = \det M \neq 0$. 
				\item אם $c = 1$, אז $c^{2} = 1$. 
				\item אם $c = 2$, אז $c^{2} = 4 \equiv 1$. 
			\end{itemize}
			כלומר בכל מקרה $c^{2} \equiv_3 1$. מכאן ש־$\det A \equiv_3 \det B$, כלומר $1 \equiv_3 2$, וזו סתירה. 
		\end{proof}
	\end{enumerate}
	
	\section{}
	\begin{enumerate}[(A)]
		\item נמצא את כל המטריצות $A \in M_2(\R)$ כך שהתבנית הריבועית המתאימה להן היא: 
		\[ q\pms{x_1 \\ x_2} = x_1^{2} + 3x_1x_2 - 7x_2^{2} \]
		תהי $A \in M_2(\R)$ מטריצה. נדרוש: 
		\[ x_1^{2} + 3x_1x_2 - 3x_2^{2} = \pms{x_1 \\ x_2}^{T}\pms{a & b \\ c & d}\pms{x_1 \\ x_2} = \pms{x_1 \\ x_2}^{T}\pms{x_1a + x_2b \\ x_1c + x_2d} = ax_1^{2} + (b + c)x_1x_2 + dx_2^{2} \]
		הפולינום הוא אובייקט פורמלי, ולכן: 
		\[ d = -3 \land a = 1 \land b + c = 3 \]
		כלומר, $A$ היא חלק מהמרחב האפיני: 
		\[ A \in \ccb{\pms{1 & 3 - x \\ x & -7} \mid x \in \R} \]
		\item נמצא את כל המטריצות הסימטריות שמתאימות לתבנית מהסעיף הקודם. (ממשפט, קיימת רק אחת). ידוע קיום $x \in \R$ כך ש־$A = \binom{1\,\,3 - x}{\,x\,\,\,\,-7\,\,}$. מהיותה סימטרית $3 - x = x$, כלומר $x = 1.5$. סה''כ קיבלנו: 
		\[ A = \pms{1 & 1.5 \\ 1.5 & -7} \]
		\item עתה נמצא את כל המטריצות $A \in M_2(\R)$ כך ש־: 
		\[ \pms{x_1 \\ x_2} \cdot_A \pms{y_1 \\ y_2} = x_1y_1 + 3x_1y_2 - 7x_2y_2 \]
		תהא $A \in M_2(\R)$. נדרוש: 
		\[ x_1y_1 + 3x_1y_2 - 7x_2y_2 = \pms{x_1 \\ x_2}^{T}\pms{a & b \\ c& d}\pms{y_1 \\ y_2} = \pms{x_1 \\ x_2}^{T}\pms{ay_1 + by_2 \\ cy_1 + dy_2} = ax_1y_1 + by_2x_1 + cy_1x_2 + dy_2x_2 \]
		מעל הממשיים, נקבל: 
		\[ a = 1 \land b = 3 \land c = 0 \land d = -7 \]
		כלומר: 
		\[ A = \pms{1 & 3 \\ 0 & -7} \]
		וסיימנו. 
	\end{enumerate}
	
	\section{}
	\begin{enumerate}[(A)]
		\item נוכיח שחפיפת מטריצות היא יחס סימטרי. \begin{proof}
			נסמן ב־$\sim$ את יחס החפיפה. יהיו $A, B$ חופפות, כלומר קיימת $M \in M_n(\F)$ הפיכה כך ש־$A = M^TBM$. נבחין ש־$(M^T)\op = (M\op)^{T}$. נכפול ב־$(M^T)\op$ מצד שמאל, נקבל $(M^T)^{-1}A = BM$. נכפול ב־$M\op$ מצד ימין, נקבל $(M^T)\op BM = A$. סה''כ $B \sim A$ מהגדרה (עבור $M\op$ מטריצה הפיכה) כלומר $\sim$ יחס סימטרי. 
			\end{proof}
		\item נוכיח שאם $A \in M_n(\F)$ חופפת למטריצה אלכסונית, אז $A$ סימטרית. \begin{proof}
			מההנחה קיימת $\Lg$ כך ש־$\Lg = \diag(\lg_1 \dots \lg_n)$ עבור $\lg_i \in \F$ סקלר כלשהו, וכן $M \in M_n(\F)$ הפיכה, כך ש־$A = M^T\Lg M$ (נעזרתי בסימטריות יחס החפיפה מסעיף קודם). נראה ש־$A$ סימטרית. נבחין ש־$\Lg_{\ml k}$ מקיים $\Lg_{\ml k} = \lg_k$ אם $k = \ml$, ו־$\Lg_{\ml k} = 0$ אחרת, לכל $\ml, k \in [n]$. לכן מהגדרת כפל מטריצות: 
			\[ A_{ij} = \sum_{k = 1}^{n} M^{T}_{ik}(\Lg M)_{kj} = \sum_{k = 1}^{n}M_{ki}\sum_{\ml = 1}^{n} \Lg_{k\ml }M_{\ml j} = \sum_{k = 1}^{n} \lg_k M_{ki}M_{kj} \]
			\[ A_{ji} = \sum_{k = 1}^{n} M^{T}_{jk}(\Lg M)_{ki} = \sum_{k = 1}^{n}M_{kj}\sum_{\ml = 1}^{n} \Lg_{k\ml }M_{\ml i} = \sum_{k = 1}^{n} \lg_k M_{kj}M_{ki} \]
			מקומטטיביות כפל ב־$\F$ ומטרזטיביות שוויון, $A_{ij} = A_{ji}$ כלומר $A$ סימטרית כנדרש. 
			\end{proof}
	\end{enumerate}
	
	\section{}
	נגדיר: 
	\[ A = \pms{4 & 1 \\ 1 & -3} \in M_2(\R) \]
	\begin{enumerate}[(A)]
		\item נמצא את התבנית הריבועית שמתאימה ל־$A$: 
		\[ q{\pms{x \\ y}} = \pms{x \\ y}^T\!\!\!A\pms{x \\ y} = \pms{x \\ y}^{T}\!\! \pms{4x + y \\ x - 3y} = 4x^{2} + xy + xy - 3y^{2} = 3x^{2} +2xy - 3y^{2} \]
		\item נמצא חילוף משתנים שמלכסן את התבנית: 
		\[ 3x^{2} + 2xy - 3y^{2} = \cl{3x^{2} + 2xy +4y^{2}} - 4y^{2} - 3y^{2} = \cl{\sqrt 3x + {\frac{2}{\sqrt3}}y}^{2} - 7y^{2} = z^{2} - 7w^{2} =: p(z, w) \]
		ע''י הצבת $z = \sqrt 3 x + \frac{2}{\sqrt 3}y$ וכן $w = y$, קיבלנו ש־: 
		\[ M = \pms{\sqrt 3 & \frac{2}{\sqrt 3} \\ 0& 1} \quad M^{T}\diag(1, -7)M = A \]
		\item נמצא $B$ סימטרית כך ש־$p$ התבנית הריבועית המתאימה לה. נבחין שהמטריצה לעיל, $B := \diag(1, -7)$, מקיימת: 
		\[ \pms{x \\ y}^{T}\pms{1 & 0 \\ 0 & -7}\pms{x \\ y} = x^{2} - 7y^{2} = p(x, y) \]
		כלומר $\diag(1, -7)$ היא המטריצה המתאימה לתבנית הריבועית $p$. 
%		\item נוודא שאכן $A = M^{T}BM$: 
	\end{enumerate}
	
\end{document}