\documentclass[]{../../../../../tex/classes/homework}
\usepackage{../../../../../tex/packages/hebrewSupport}
\usepackage{../../../../../tex/packages/mathShortcuts}

\renewcommand\mut[2] {\left \la #1, #2 \right \ra}

\author{שחר פרץ}
\title{לינארית 2א $\sim$ \textit{תרגיל בית 8}}
\begin{document}
	\maketitle
	
	\section{}
	ניעזר במשפט היסודי של האלגברה כדי להראות שכל פולינום אי־פריק $f \in \C[x]$ הוא ממעלה $1$. 
	\begin{proof}
		נעבוד מעל $\F[x]$ סגור אלגברית כללי. יהי $p \in \F[x]$, ונניח שהוא אי פריק. משום ש־$\F$ סגור אלגברית (כך הנחנו), קיים ל־$p$ שורש $\ag$ כלשהו, כלומר $(x - \ag) \mid p$ ממשפט בזו. מהגדרת חלוקה, קיים $f$ כך ש־$(x - \ag)f = p$. משום ש־$p$ אי־פריק, $f$ בהכרח איבר הפיך בחוג, כלומר $f \sim (x - a)$ כאשר $\sim$ יחס החברות. האיברים ההפיכים בחוג הפולינומים פולינומים קבועים לכן $\deg f = 1$ כנדרש. בפרט בעבור $\C = \F$ משום ש־$\C$ סגור אלגברית מהמשפט היסודי של האלגברה. 
	\end{proof}
	
	\section{}
	\begin{enumerate}[(A)]
		\item יהי $p \in \F[x]$ פולינום ללא שורשים ממעלה $2$ או $i$. נראה שהוא אי־פריק. \begin{proof}
			נניח בשלילה ש־$p$ פריק. אזי קיימים $f, g \in \F[x]$ כך ש־$p = fg$, וכן $f, g$ אינם איברים הפיכים. מכאן ש־$\deg f, g \ge 1$ והוכחנו בתרגיל בית קודם $3 = \deg p = \deg fg = \deg f + \deg g$ כאשר $i \in \{2, 3\}$. אם $i =2$ בהכרח $\deg f = \deg g = 1$, אחרת $\deg f = 1 \land \deg g = 2 \land \deg f = 2 \land \deg g = 1$. בה''כ $\deg f = 1$ בשני המקרים. נקבל שקיים $\ag$ כך ש־$f = (x - \ag)$ (עד לכדי חברות) ומכאן ש־$(x - \ag) \mid p$ וממשפט בזו $\ag$ שורש של $p$ וסתירה. 
		\end{proof}
		\item נוכיח ש־$x^{2} + x + 1$ אי־פריק מעל $\Z_2$. \begin{proof}
			נסמן $p = x^{2} + x + 1$. נניח בשלילה של־$p$ יש שורשים מעל $\Z_2$ (יש לציין את השדה שכן הוא אובייקט פורמלי ולא פונקציה). נוכיח שאין לו שורשים ב־$\Z_2$ ע''י הצבה כל מגוון המספרים השונים בשדה. 
			\begin{gather*}
				p(0) = 0^{2} + 0 + 1 = 1 \neq 0 \\
				p(1) = 1^{1} + 1 + 1 = 3 \equiv 1 \neq 0
			\end{gather*}
			סה''כ ל־$p$ אין שורשים ב־$\Z_2$. 
		\end{proof}
	\end{enumerate}
	
	\section{}
	\begin{enumerate}[(A)]
		\item נמצא מטריצה הפיכה ולכסינה. נתבונן במטריצה המרתקת $I$, הזהות היא $\diag(1, 1)$. ידוע שהזהות הפיכה, והיא דומה לעצמה, מטריצה אלכסונית. (מעל $\R^{2}$, לדוגמה). 
		\item נמצא מטריצה הפיכה שאינה לכסינה. נתבונן במטריצה $A = \binom{1\,0}{1\,1}$ מעל $\R^{2}$. היא הפיכה כי $\det A = 1 \neq 0$, אך היא איננה לכסינה משאלה 8 בתרגיל בית זה. 
		\item נמצא מטריצה לכסינה שאינה הפיכה. נתבונן במטריצה $A = \diag(1, 0)$ מעל $\R^{2}$. היא איננה הפיכה כי יש בה שורת אפסים ומכאן ש־$\det A = 0 $. היא לכסינה כי היא זהה ובפרט דומה לעצמה, מטריצה אלכסונית. 
	\end{enumerate}
	
	
	\section{}
	יהי $V$ מ''ו נוצר סופית מעל $\F$ ו־$T \co V \to V$ לינארית, כאשר $\bc$ בסיס של $V$. נוכיח ש־$v$ ו''ע של $T$ שמתאים לע''ע $\lg$ אמ''מ $[v]_\bc$ ו''ע של $[T]_B^{\bc}$ שמתאים לע''ע $\lg$.
	\begin{proof}
		\begin{itemize}
			\item[$\implies$] אם $[v]_\bc$ ע''ע של המטריצה $[T]_\bc$ אז $[Tv]_\bc = [T]^{\bc}_\bc[v]_\bc = \lg [v]_\bc = [\lg v]_\bc$. משום ש־$[\cdot]_\bc \co V \to \R^{n}$ איזו', אז היא הפיכה ואז בהכרח $Tv = \lg v$ וסיימנו. 
			\item[$\impliedby$] אם $v$ ע''ע של העתקה $T$, אז $[T]^{\bc}_\bc[v]_\bc = [Tv]_\bc = [\lg v]_\bc = \lg [v]_\bc$ כלומר $[v]_\bc$ ע''ע של $[T]_\bc^{\bc}$ כדרוש. 
		\end{itemize}
	\end{proof}
	
	\section{}
	נניח ש־$0$ ע''ע של $A \in M_n(\F)$ אמ''מ $A$ איננה הפיכה. 
	\begin{proof}
		\begin{itemize}
			\item[$\implies$]נניח $A$ איננה הפיכה. מכאן שקיים $v \neq 0$ כך ש־$v \in \nc (A)$. דהיינו $Av = 0 = 0 \cdot v$ ומהגדרה $v$ ו''ע של $A$ (בעבור הע''ע $0$) כלומר $0$ ע''ע כדרוש. 
			\item[$\impliedby$]נניח $0$ ע''ע של $A$. מכאן ש־$\exists 0 \neq v \in V \co Av = 0$ מהגדרה. נסיק ש־$v \in \nc(A)$ כלומר $A$ איננה הפיכה (אופרטור לינארי עם קרנל לא ריק לא הפיך). 
		\end{itemize}
	\end{proof}
	
	\section{}
	תהי $A \in M_n(\F)$ משולשית. נמצא את הע''ע של $A$. 
	\begin{proof}
		נסמן ב־$\lg_1 \dots \lg_n$ את האיברים על האלכסון של $A$. אזי: 
		\[ p_A(x) = \det(A - Ix) = \prod_{i = 1}^{n}(\lg_i - x) \]
		נבחין ש־$x$ ע''ע של $A$ אמ''מ $p_A(x) = 0$, שכן: 
		\[ p_A(x) = 0 \iff \det(A - Ix) = 0 \iff \nc(A - Ix) \neq \varnothing \iff \exists 0 \neq v \in V \co (A - Ix)v = 0 \iff \exists 0 \neq v \in V \co Av = xv \]
		הטענה הימנית ביותר שקולה לכך ש־$x$ ע''ע של $A$. ממשפט בזו וכל מיני דברים כאלו, $\lg_1 \dots \lg_n$ השורשים היחידים של $p_A(x)$ מעל $\F$, ומכאן שהם הע''ע של $A$. 
	\end{proof}
	
	\section{}
	יהי $V \subseteq M_2(\F)$ תמ''ו המטריצות האלכסוניות. נגדיר $T \co V \to V$ על־ידי: 
	\[ T\pms{x & 0 \\ 0 & y} = \pms{x + y & 0 \\ 0 & x - y} \]
	\begin{enumerate}[(A)]
		\item נבדוק האם $T$ לכסינה ב־$\F = \Q$ וב־$\F = \R$. קל לראות ש־$e_1 = \binom{1\,0}{0\,0}$ ו־$e_2 = \binom{0\,0}{0\,1}$ בסיס, נסמנו $\ec$. אז: 
		\[ [T]_\ec = \pms{[Te_1]_\ec & [Te_2]_\ec} = \pms{\csb{\pms{1 & 0\\ 0 & 1}}_\ec & \csb{\pms{1  & 0 \\ 0 & -1}}_\ec} = \pms{1 & 1 \\ 1 & -1} \]
		נמצא פ''א של $T$ (שלא תלוי בנציג/בסיס)
		\[ p_{[T]_\ec}(x) = \det\cl{[T]_\ec - Ix} = \detms{1 - x & 1 \\ 1 & -1  - x} = (1 - x)(-1 - x) - 1 = x^{2} - x + x - 1 - 1 = x^{2} - 2 = (x - \sqrt 2)(x + \sqrt 2) \]
		נבחין שהשוויון האחרון קיים מעל $\R$ בלבד, ולצערנו מעל $\Q$ הפולינום $x^{2} - 2$ אי פריק. מכאן שאין לו שורשים רציונליים (בזו), ולכן אין שום ע''עים למטריצה ובפרט ההעתקה כולה אינה לכסינה. מעל $\R$, נקבל שני ע''עים שונים – $\sqrt 2$ ו־$-\sqrt 2$, ולכן היא לכסינה. 
		\item בסעיף זה החליטו לסמן $\bc := \ec$. נלכסן את $[T]_\bc$ ב־$\R$ (כלומר נמצא בסיס ו''עים). 
		
		נבחין שהמרחב העצמי: 
		\begin{gather*}
			\nc\pms{1 + \sqrt 2 & 1 \\ 1 & -1 + \sqrt 2} = \nc\pms{1 & -1 + \sqrt 2 \\ 0& 0} = \Sp{\pms{1 - \sqrt 2 \\ 1}} \\ 
			\nc\pms{1 - \sqrt 2 & 1 \\ 1 & -1 - \sqrt 2} = \nc\pms{1 & -1 - \sqrt 2 \\ 0& 0} = \Sp{\pms{1 + \sqrt 2 \\ 1}}
		\end{gather*}
		נעביר חזרה ב־$[\cdot]_\ec\op$ ונקבל: 
		\[ v_1 = \pms{1 - \sqrt 2 & 0 \\ 0 & 1} \quad v_2 = \pms{1 + \sqrt 2 & 0 \\ 0 & 1} \]
		אז $v_1$ ע''ע בעבור $\sqrt 2$ ו־$v_2$ ע''ע בעבור $-\sqrt 2$. המרחב מממד $2$ ולכן סה''כ לכסנו את הההעתקה $T$. 
	\end{enumerate}
	\section{}
	יהיו $a, b \in \F$. נראה שהמטריצה $A := \binom{a\,1}{0\,\,b} \in M_2(\F)$ לכסינה אמ''מ $a \neq b$, ונמצא בסיס של ו''ע. 
	\begin{proof}
		נמצא את הפ''א של המטריצה: 
		\[ \det(A - Ix) = \det\pms{a - x & 1 \\ 0 & b - x} = (a - x)(b - x) \]
		נבחין ש־$a, b$ ע''עים. נפרק למקרים. 
		\begin{itemize}
			\item אם $a \neq b$, אז ממשפט המטריצה לכסינה. נמצא ו''עים. נתבונן במרחבים העצמיים: 
			\[ \nc{{\pms{0 & x \\ 0 & b - a}}} = \Sp\pms{1 \\ 0} \quad\quad \nc{{\pms{a - b & x \\ 0 & 0}}} = \Sp\pms{1 \\ a - b} \]
			בסיס למרחבים שנוצרים ע''י וקטור יחיד ניתן ע''י אותו הוקטור. סה''כ $\binom{1}{a - b}$ ו־$\binom{1}{0}$ בסיס של ו''עים למטריצה. 
			\item אם $a = b$, נבחין שהמרחב העצמי: 
			\[ \nc\pms{0 & 1 \\ 0 & 0} = \Sp\pms{1 \\ 0} \implies \dim V_{a} = 1 \]
			סה''כ סכום הממדים של המרחבים העצמיים (יש רק אחד כזה) הוא $1 \neq 2$, ומכאן שהמטריצה אינה לכסינה. 
		\end{itemize}\envendproof
	\end{proof}
	
	\section{}
	תהא $A \in M_n(\F)$ ונניח ש־$\lg^{2}$ הוא ע''ע של $A^{2}$. נוכיח של־$A$ יש ע''ע $\pm \lg$. 
	\begin{proof}
		משום ש־$\lg^{2}$ ע''ע של $A^{2}$, אזי הוא שורש של $p_{A^{2}}$. ממשפט בזו $(x - \lg^{2}) \mid p_{A^{2}}$. נסיק: 
		\[ \det(A - Ix)\det\cl{A + Ix} = \det\cl{A^{2} - Ix} = p_{A^{2}} = f \cdot (x - \lg^{2}) = f(x - \lg)(x + \lg) \]
		בגלל שבחוג הפולינומים הוא תחום ראשי, הפירוק לראשוניים (גורמים לינאריים) קיים ויחיד, בהכרח $(x + \lg)$ ו־$(x - \lg)$, כל אחד בנפרד, מחלקים את $\det(A - Ix)$ או את $\det(A + Ix)$. נפרק למקרים. 
		\begin{itemize}
			\item אם $(x - \lg)$ מחלק את $\det(A + Ix)$ אז $\det(A + Ix)(\lg) = 0$ כלומר $\det(A + \lg I) = 0$ ומכאן ש־$\dim V_{-\lg} > 0$ כלומר $-\lg$ ע''ע. 
			\item אם $(x - \lg)$ מחלק את $\det(A - Ix)$ אז $\det(A - Ix)(\lg) = 0$ כלומר $\det(A - \lg I) = 0$ ומכאן ש־$\dim V_{\lg} > 0$ כלומר $\lg$ ע''ע. 
%			\item אם $(x + \lg)$ מחלק את $\det(A + Ix)$ אז $\det(A + Ix)(-\lg) = 0$ כלומר $\det(A - \lg I) = 0$ ומכאן ש־$\dim V_{\lg} > 0$ כלומר $\lg$ ע''ע. 
%			\item אם $(x + \lg)$ מחלק את $\det(A - Ix)$ אז $\det(A - Ix)(-\lg) = 0$ כלומר $\det(A + \lg I) = 0$ ומכאן ש־$\dim V_{-\lg} > 0$ כלומר $-\lg$ ע''ע. 
		\end{itemize}
		(אפשר להגיד אותם הדברים על $(x + \lg)$, אבל אין צורך). מכאן ש־$-\lg$ ע''ע או ש־$\lg$ ע''ע. 
	\end{proof}
	
	\ndoc
	
\end{document}
