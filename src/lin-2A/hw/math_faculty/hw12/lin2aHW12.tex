\documentclass[]{../../../../../tex/classes/homework}
\usepackage{../../../../../tex/packages/hebrewSupport}
\usepackage{../../../../../tex/packages/mathShortcuts}

\renewcommand\mut[2] {\left \la #1, #2 \right \ra}
\DeclareMathOperator{\Id}{Id}

\author{שחר פרץ}
\title{לינארית 2א $\sim$ \textit{תרגיל בית 12}}
\begin{document}
	\maketitle
	\section{}
	\begin{enumerate}[(A)]
		\item יהי $V$ מ''ו נוצר סופית, $T \co V \to V$ העתקה לינארית ו־$S = T - \Id$. נוכיח ש־$\deg m_T = \deg m_S$. 
		\begin{proof}
			ניעזר במשפט ג'ורדן. נניח בה''כ שאנחנו בשדה סגור אלגברית $\F$, ואם לא, נרחיב לשדה כזה $\F\subseteq \K$ ו־$T$ מעליו תשאר לינארית בעלת אותו הפולינום המינימלי (שכן $\K$ משכן את $\F$). לכן קיים בסיס $B$ מג'רדן כך ש־$[T]_B$ בצורת ג'ורדן. נסמן $m_T = \prod_{i = 1}^{k}(T - \lg_i)^{d_i}$ (אפשר לבע פירוק כזה כי אנו בשדה סגור אלגברית). ממשפט $\lg_i$ ע''ע (ב־$\K$) של $T$ ו־$d_i$ הבלוק הגדול ביותר בצורת הג'ורדן. נבחין ש־: 
			\[ [S]_B = [T - \Id]_B = [T]_B - [\Id]_B = [T]_B - I \]
			נתבונן בבלוקי הג'ורדן $J_{d_i}(\lg_i)$ שבהכרח ב־$[T]_B$. לאחר חיסור $I$ מהבלוק, נקבל בלוק $J_{d_i}(\lg_i - 1)$. סה''כ $[S]_B$ כוללת כוללת לכל $i$ בלוק ג'ורדן מקסימלי בעבור הע''ע $\lg_i - 1$ מגודל $d_i$ סה''כ ממשפט נקבל ש־$m_{[S]_B} = \prod_{i = 1}^{k} (T - (\lg_i - 1))^{d_i}$. ממשפט לגבי היות הפולינום המינימלי נשמר תחת דימיון וייצוג בבסיס, נקבל $m_S = \prod_{i = 1}^{k}(T - (\lg_i - 1))^{d_i}$. ידוע ש־$m_S$ ב־$\K$ זהה ל־$m_S$ ב־$\F$ כי הרחבת שדה לא משנה את הפולינום המינימלי, ומכאן ש־$m_S$ הפולינום המינימלי של $S$. נבחין ש־$\deg m_T = \sum_{i = 1}^{k}d_i = \deg m_S$ כנדרש. 
		\end{proof}
		\item יהי $V = M_n(\C)$ ונגדיר $T \co V \to V$ ע''י $T(A) = A^{T}$. נמצא את הפולינום המינימלי של $T$. \begin{proof}
			בתרגיל בית קודם הראינו ש־$1$ ע''ע מריבוי $\frac{n(n + 1)}{2}$ ו־$-1$ ע''ע מריבוי $\frac{n(n - 1)}{2}$, כי $T$ היא $T$־איווריאנטית על $\sym_n(\C)$ וכן על $\asym_n(\C)$. לכן $T$ לכסינה ועל אלכסונה יופיע $1$ בדיוק $\frac{n^{2}+ n}{2}$ פעמים ו־$-1$ בדיוק $\frac{n^{2} - n}{2}$ פעמים. בגלל שצורתה הלכסינה היא בפרט צורת הג'ורדן שלה (עם בלוקים $J_1(\pm 1)$), ממשפט ידוע ש־$m_T = (x - 1)^{d_+}(x + 1)^{d_-}$ כאשר $d_-$ בלוק הג'ורדן המקסימלי של ע''ע $-1$ ו־$d_+$ בלוק הג'ורדן המקסימלי של ע''ע $+1$. בגלל שהיא לכסינה בלוקי הג'ורדן שלה הם מגודל $1$, וסה''כ: 
			\[ m_T = (x - 1)(x + 1) = x^{2} - 1 \]
			כנדרש. 
		\end{proof}
		\item יהי $V = M_n(\C)$ ונגדיר $S \co V \to V$ ע''י $S(A) = A^{T} - A$. נמצא את הפולינום המינימלי של $S$. \begin{proof}
			נתחיל מלחקור כמעה את $S$. תהי $A \in \sym_n(\C)$. נבחין ש־: 
			\[ S(A) = A^{T} - A = A - A = 0 \]
			מכאן ש־$0$ ע''ע לכל $A \in \sym_n(\C)$. 	ובעבור $A \in \asym_n(\C)$ נקבל ($A^T = -A$): 
			\[ S(A) = A^{T} - A = -A - A = -2A \]
			מכאן ש־$-2$ ע''ע לכל $A \in \asym_n(\C)$. משום ש־$\sym_n(\C) \oplus \asym_n(\C) = V$ סכום ישר, סה''כ נוכל לקחת ולשרשר בסיס מ־$\sym_n(\C)$ ובסיס מ־$\asym_n(\C)$ לקבלת בסיס מלכסן, ובעבורו נקבל שהצורתה הלכסינה ובפרט צורת הג'ורדן שלה היא $0$ בדיוק $\frac{n^{2} + n}{2}$ פעמים ו־$-2$ בדיוק $\frac{n^{2} - n}{2}$ פעמים. מנימוקים דומים לאלו של סעיף קודם ריבוי הע''עים בפולינום המינימלי הוא $1$ ולכן: 
			\[ m_S = (x - (-2))^{1}(x - 0)^{1} = x^{2} + 2x \]
		\end{proof}
		\item \begin{proof}
			משום ש־$\sym_n(\C) \oplus \asym_n(\C)$ פירוק למ''וים עבורם $T$ לכסינה בהם בעבור אותו הע''ע ובפרט $T$־איווריאנטית, ומיחידות הפירוק למ''וים עצמיים מורחבים (הפירוק הפרימרי בהקשר של הקורס הזה), בהכרח הפולינום האופייני נתון ע''י כמות הוקטורים העצמיים המוכללים, אם כי במקרה זה ההעתקה לכסינה ואין צורך שיהיו מוכללים, ונוכל פשוט להתבונן בכמות הוקטורים בבסיס המלכסן שבחרנו בסעיף קודם לכל ע''ע. נבחין שיש לנו $\frac{n^{2} + n}{2}$ ו''עים ל־$0$ ו־$\frac{n^{2} - n}{2}$ ו''עים ל־$-2$. סה''כ הפולינום האופייני: 
			\[ p_S(x) = (x + 2)^{\frac{n^{2} - n}{2}}x^{\frac{n^{2} - n}{2}} \]
		\end{proof}
	\end{enumerate}
	
	\section{}
	\begin{enumerate}[(A)]
		\item נתבונן במטריצה: 
		\[ A = \pms{n & n - 1 & \cdots & 2 & 1 \\ 0 & n & \ddots & 3 & 2  \\ \vdots & \ddots  & \ddots & \vdots & \vdots \\ 0 & \cdots & 0 & n & n - 1 \\ 0 & 0 & \cdots & 0 & n} \]
		נבחין שמשום שזו מטריצה משולשית, אז $n$ הוא הע''ע הקיים והיחיד של $A$ (שכן פ''א $(x - n)^{n}$). קל לראות ששורתה הימנית של המטריצה $A - nI$ היא שורת אפסים, וכל שאר השורות בת''ל. מכאן ש־$(A - nI)^{j}$ כוללות $j$ שורות אפסים – לכן דרגת הנילפוטנטיות של $A - nI$ תהיה כאשר $n = j$ כלומר היא $n$. סה''כ עבור $0 \neq v \in V \setminus \ker(A - nI)^{n - 1}$ נקבל בסיס $v, (A - nI)v \dots (A - nI)^{n - 1}v$ שהוא שרשרת ולכן מג'רדן, כלומר $J_n(0)$ צורת הג'ורדן של $A - nI$, דהיינו:
		\[ A \dim J_n(n) \]
		היא צורת הג'ורדן של $A$. 
		\item נתבונן במטריצה: 
		\[ B = \pms{1 & 0 & 0 & \cdots & 0 \\ 1 & 2 & 0 & \cdots & 0 \\ 1 & 2 & 3 & \cdots & 0 \\ \vdots & \vdots & \vdots & \ddots & \vdots \\ 1 & 2 & 3 & \cdots & n} \]
		נבחין שמשום שמטריצה זו משולשית הפ''א נתון ע''י מכפלת איברי האלכסון כלומר $p_B(x) = \prod_{i = 1}^{n}(x - i)$. זוהי מטריצה לכסינה כי יש לה $n$ ע''עים שונים והיא מגודל $n \times n$. סה''כ צורת הג'ורדן שלה מתלכדת עם הצורה הלכסינה והיא: 
		\[ B \sim \diag(1 \dots n) \]
		\item נתבונן במטריצה $C$ המתוארת בתרגיל הבית. משום שהיא משולשית אז $\ag$ הע''ע היחיד שלה. נבחין שעבור $C - \ag I$ יש שתי שורות אפסים, וכל כפל שלה בעצמה יחשוף שתי שורות אפסים נוספות, בעוד שאר השורות הת''ל. לכן באלגו' למציאת צורת ג'ורדן נקבל $\dim \ker A^{0} = n, \dim \ker A^{1} = n - 2 \dots \dim \ker A^{i} = n - 2i$. סה''כ נקבל מהאלגו' $d_0 = 2, d_1 = 2 \dots d_{\frac{n}{2}} = 2$. כלומר מספיק לקחת $v_1, v_2 \in V \setminus \ker (C - \ag I)^{\frac{n}{2}}$ כדי לקבל שתי שרשראות באורך $\frac{n}{2}$. נקבל שצורת הג'ורדן של $T - \ag I$ היא $\diag(J_{\frac{n}{2}}(0, J_{\frac{n}{2}}(0)))$ ולכן צורת הג'ורדן של $C$ היא: 
		\[ C \sim \diag(J_{\frac{n}{2}}(\ag), J_{\frac{n}{2}}(\ag)) \] 
	\end{enumerate}
	
	\section{}
	ניעזר בפולינום האופייני והמינימלי של $A$ ע''מ למצוא את כל האפשרויות לצורת ג'ורדן. 
	\begin{enumerate}[(A)]
		\item נתון $p_A(x) = (x + 10)^{7}$ וכן $m_A(x) = (x + 10)^{3}$. 
		
		מכאן שבהכרח בלוק הג'ורדן המקסימלי בגודלו הוא מגודל $3$, ויש $7$ ו''עים מוכללים השייכים לע''ע $10$, היחיד. מכאן שנדרוש מטריצה $7 \times 7$ בה $J_3(10)$ בלוק ג'ורדן וכל שאר בלוקי הג'ורדן קטנים או שווים ל־$3$. נמצא את כל האופציות הקיימות: 
		\begin{align*}
			\diag(J_3(7), J_3(7), J_1(7)) && \diag(J_3(7), J_2(7), J_2(7)) && \diag(J_3(7), J_1(7), J_1(7), J_1(7), J_1(7))
		\end{align*}
		עד לכדי סדר בלוקים. 
		\item נתון $p_A(x) = (x - 3)^{4}(x - 5)^{4}$ וכן $m_A(x) = (x - 3)^{2}(x - 5)^{3}$. 
		
		מכאן שבהכרח ישנם שני ע''עים $3, 5$ עבורם $4, 4$ ו''עים עצמאיים מוכללים בהתאמה (מ־$p_A(x)$). משום ש־$4 + 4 = 8$ נדרוש מטריצה $8 \times 8$. בלוק הג'ורדן המקסימלי של $3$ הוא $2$, והמקסימלי של $5$ הוא $3$. מכאן ש־$J_3(5), J_2(3)$ בלוקים. נכתוב את כל האופציות למה שנשאר: 
		\begin{align*}
			\diag(J_3(5), J_1(5), J_2(3), J_2(3)) && \diag(J_3(5), J_1(5), J_2(3), J_1(3), J_1(3))
		\end{align*}
		עד לכדי שינוי סדר בלוקים. 
	\end{enumerate}
	
	\section{}
	בכל סעיף נמצא מטריצות המקיימות את הנדרש. 
	\begin{enumerate}[(A)]
		\item ל־$J_1, J_2$ יש אותו הפולינום האופייני אך פולינום מינימלי שונה. 
		
		נבחר את מטריצת האפס והמטריצה $\binom{0\,0}{1\,0}$. שתיהן ניל' ולכן הע''ע היחיד הוא $0$. מכאן שהפולינום האופייני של שתיהן הוא $x^{2}$. נבחין שהפולינום המינימלי של מטריצת האפס הוא $x$ שכן היא שווה ל־$0$, אך דרגת הניל' של $\binom{0\,0}{1\,0}$ היא $2$ כלומר הפולינום המינימלי שלה הוא $x^{2}$. סה''כ שתי מטריצות עם אותו הפולינום האופייני אך פולינום מינימלי שונה. 
		\item ל־$J_1, J_2$ יש אותו הפולינום המינימלי אך פולינום אופייני שונה. 
		
		נבחר: 
		\[ A = \pms{} \]
		הערה לעצמי: חייב להיות $3 \times 3$. 
		\item ל־$J_1, J_2$ יש אותו הפולינום האופייני והמינימלי, אבל עם ריבוי גיאומטרי שונה ביחס לכל אחת. 
		
		נתבונן במטריצות הגו'רדן האפשרויות לפולינום האופייני והמינימלי המצויים בשאלה 3(ב). שתיהן מטריצות ג'ורדן שונות ביחס לאותו הפולינום המינימלי. נתבונן בע''ע $3$ – באחת יש $3$ בלוקי ג'ורדן המשוייכים אליו ומכאן שיש לה ריבוי גיאומטרי $3$, בעוד לשנייה יש שני בלוקי ג'ורדן המשוייכים אליו ומכאן ריבוי גיאומטרי $2$. סה''כ ריבוי גיאומטרי שונה בין שתיהן בעבור ע''ע כלשהו. 
		\item ל־$J_1, J_2$ יש אותו הפולינום האופייני והמינימלי, ולא ניתן לקבל אחת מהשנייה ע''י שינוי סדר בלוקים בצורת הג'ורדן. 
		
		נתבונן במטריצות הבאות: 
		\begin{align*}
			J_1 = \diag(J_1(0), J_3(0), J_3(0)) && J_2 = \diag(J_2(0), J_2(0), J_3(0))
		\end{align*}
		נבחין שלשתיהן ע''ע יחיד $0$ כלומר הפ''א הוא $x^{7}$ בעבור שתיהן. נוסף על כך יש להן את אותה הכמות של בלוקי הג'ורדן, כלומר יש להן את אותו הריבוי הגיאומטרי. נוסף על כך בלוק הג'ורדן הגדול ביותר זהה בין השתיים, והוא $3$, כלומר פולינום מינימלי $x^{3}$. עם זאת, לא ניתן לקבל את $J_1$ ע''י שינוי סדר בלוקים ב־$J_2$, שכן בלוקי הג'ורדן הם שונים בין המטריצות. 
	\end{enumerate}
	
	\section{}
	יהי $\F$ שדה. 
	\begin{enumerate}[(A)]
		\item תהא $J_n(\lg) \in M_n(\F)$ בלוק ג'ורדן. נמצא את תוצאת החיסור הבא לכל $k \in \N_0$: 
		\[ d_k = \dim \zc_r((J_n(\lg) - \lg I)^{k + 1}) - \dim \zc_r((J_n(\lg) - \lg I)^{k}) \]
		\begin{proof}
			קל לראות שבאינדוקציה, לכל $k \in [n]$ מתקיים: 
			\[ J_n(0)^{k} = \pms{\vert & \cdots & \vert & \vert & \cdots & \vert \\ e_{n - k} & \cdots & e_1 & 0 & \cdots &0 \\ \vert & \cdots & \vert & \vert & \cdots & \vert} \]
			כאשר $e_1 = (1, 0 \dots)$ ו־$e_n = (0 \dots 0, 1)$. שורות האפסים לא קיימות במקרה הבסיס $k= 0$ (נקבל פשוט את הזהות) ויש בדיוק $k$ שורות אפסים ב־$J_n(0)^{k}$, ושאר השורות הן בת''ל, כלומר הדרגה היא $n - k$ ומרחב האפסות הוא $k$. נבחין ש־$J_n(\lg) - \lg I = J_n(0)$. מכאן ש־: 
			\begin{align*}
				d_k &= \dim \zc_r((J_n(\lg) - \lg I)^{k + 1}) - \dim \zc_r((J_n(\lg) - \lg I)^{k}) \\
				&= \dim \zc_r(J_n(0)^{k + 1}) - \dim \zc_r(J_n(0)^{k}) \\
				&= k + 1 - k = 1
			\end{align*}
			עבור $k \in 0 \dots n - 1$. עם זאת, עבור $k \ge n$ המטריצה בהכרח מתאפסת ($J_0(0)$ ניל') כלומר ניוותר עם ממדים מגודל $0$, ו־$d_{k \ge n} = 0- 0 = 0$. סה''כ: 
			\[ d_k = \begin{cases}
				1 & k \in [n - 1] \\
				0 & \other
			\end{cases} \]
		\end{proof}
		\item יהי $\mu \in \F$ כאשר $\mu \neq \lg$. לכל $k \in \N$ נוכיח ש־$(J_n(\lg)- \mu I)^{k}$ הפיכה ונחשב את: 
		\[ \dim \zc_r((J_n(\lg) - \mu I)^{k + 1}) - \dim \zc_r((J_n(\lg) - \mu I)^{k}) \]
		\begin{proof}
			נבחין ש־$A := J_n(\lg) - \mu I$ מטריצה אלכסונית משולשית עליונה עם אלכסון ללא אפסים. מאלגו' גאוס היא הפיכה. באינדוקציה $A^{k}$ הפיכה לכל $k \ge 1$ שכן כפל הפיכות הוא הפיך. עבור $k= 0$ נקבל $A^{0}  = I$ הפיכה גם היא. סה''כ $\rk A = n$ ולכן $\dim \zc_r(A) = 0$. מכאן ש־: 
			\[ \dim \zc_r((J_n(\lg) - \mu I)^{k + 1}) - \dim \zc_r((J_n(\lg) - \mu I)^{k}) = \dim \zc_r A^{k + 1} - \dim \zc_r A^{k} = 0 - 0 = 0 \]
			לכל $k \in \N$ כנדרש. 
		\end{proof}
		\item תהי $J \in M_n(\F)$ מטריצת ג'ורדן ויהי $\lg$ ע''ע שלה. לכל $k \in N_0$ נמצא את: 
		\[ d_k:= \dim \zc_r((J - \lg I)^{k + 1}) - \dim \zc_r((J - \lg I)^{k}) \]
		\begin{proof}
			משום ש־$J$ מטריצת ג'ורדן ניתן לכותבה בצורה $\diag(J_{\ml_1}(\lg) \dots J_{\ml_p}(\lg), J_1)$ כאשר $J_{\ml_1} \dots J_{\ml_p}$ בלוקי הג'ורדן שלה. נתבונן בכמה מוגדל הקרנל של כל בלוק ג'ורדן כאשר מעלים בחזקה, תוך שימוש בסעיפים א' וב' – מסעיף ב' $J_1 - \lg I$ הפיכה ולכן $\ker(J - \lg)^{k} = 0$ לכל $k \in \N$. לפי סעיף א', עבור $k = 0$ הקרנל $\ker(J - \ml I)^{k}$ טרוויאלי והקרנל $\ker(J- \lg I)^{1}$ הוא כמות בלוקי הג'ורדן. עם זאת עבור $k = 1$ כל בלוקי הג'ורדן מגודל אחת כבר התאפסו, ולכן הקרנל $\ker(J - \lg I)^{2}$ יהיה כמות בלוקי הג'ורדן מגודל לכל הפחות $2$ + כמות בלוקי הג'ורדן מגודל לכל היותר $1$. באופן כללי נקבל ש־$d_k$ הוא כמות בלוקי הג'ורדן מגודל לכל היותר $k + 1$ (כאשר עבור $k \ge n + 1$ נקבל שבהכרח שאין כאלו בכלל) ישירות מסעיף א' (עם אינדוקציה). (הערה: $d_0$ מתלכד עם הריבוי הגיאומטרי)
		\end{proof}
		
		\item נניח ש־$A \in M_8(\C)$ מקיימת: 
		\begin{align*}
			\dim \zc_r(A + 6I) = 3 && \dim \zc_r((A + 6I)^{2}) = 6 \\
			\dim \zc_r((A + 6I)^{3}) = 7 && \dim \zc_r((A + 6I)^{4}) = 8
		\end{align*}
		נמצא את צורת הג'ורדן שלה. 
		\begin{proof}
			נבחין שהחיסורים $d_0 = 3, d_1 = 6 - 3 = 3, d_2 = 7 - 6 = 1, d_3 = 8 - 7 = 1$. נעבוד לפי האלגוריתם למציאת צורת ג'ורדן ונבחין שראשית כל יש יש לנו שרשרת באורך $4$ ש־$d_3$ פורס. יש לנו כבר איבר אחד ב־$d_2$ ולכן הוא לא ייתן שום דבר. לאחריה נקבל עוד שתי שרשראות באורך $2$ ש־$d_1$ פורש, הכל בעבור ע''ע $-6$. קיבלנו $8$ ו''עים מוכללים ונעצור. סה''כ: 
			\[ \diag(J_2(-6), J_2(-6), J_4(-6)) \]
			כנדרש. 
		\end{proof}
	\end{enumerate}
	
	\section{}
	תהי $A \in M_n(\C)$ מטריצה עם פולינום מינימלי $x(x - 1)^{k}$ עבור $k \ge 0$ כלשהו. נראה ש־$A$ דומה ל־$A^{2}$. \begin{proof}
		נבחין שבצורת הג'ורדן של $A$ ישנם $J_{d_1}(1) \dots J_{d_p}(1)$ כאשר $p$ הריבוי הגיאומטרי של $1$, בלוקי ג'ורדן ($d_i \le k$). נוסף על כך כמות לא ידועה של אפסים על האלכסון כי $1$ הריבוי של $x$ בפולינום המינימלי. נפרק למקרים. 
		
		אם $k = 0$ ואין אף $1$ על האלכסון, אזי קיבלנו את מטריצת האפס (כי $x$ מאפס אותה) ו־$0^2 = 0$. אחרת, נראה ש־$J_{d_i}(1)^{2} = J_{d_i}(1)$. זה טרוויאלי מכפל מטריצות. קיימת מטריצה $P$ מעבר בסיס כך ש־: 
		\[ A^{2} = P^{2}J^{2}(P^{-1})^{2} = P^{2}\diag(0^{2}, \dots ,0^{2}, J_{d_1}(1)^{2}, \dots, J_{d_p}(1)^{2}) (P^{2})\op = P^{2}\diag(0, \dots, 0, J_{d_1}(1), \dots, J_{d_p}(1))(P^{2})\op = P^{2}J(P^{2})\op \]
		סה''כ $J$ צורת הג'ורדן של $A^{2}$ ולכן הן דומות. 
	\end{proof}
	
	
	\ndoc
	
\end{document}
