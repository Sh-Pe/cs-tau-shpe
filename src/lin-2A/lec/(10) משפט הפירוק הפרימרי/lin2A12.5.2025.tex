%! ~~~ Packages Setup ~~~ 
\documentclass[]{article}
\usepackage{lipsum}
\usepackage{rotating}


% Math packages
\usepackage[usenames]{color}
\usepackage{forest}
\usepackage{ifxetex,ifluatex,amssymb,amsmath,mathrsfs,amsthm,witharrows,mathtools,mathdots}
\usepackage{amsmath}
\WithArrowsOptions{displaystyle}
\renewcommand{\qedsymbol}{$\blacksquare$} % end proofs with \blacksquare. Overwrites the defualts. 
\usepackage{cancel,bm}
\usepackage[thinc]{esdiff}


% tikz
\usepackage{tikz}
\usetikzlibrary{graphs}
\newcommand\sqw{1}
\newcommand\squ[4][1]{\fill[#4] (#2*\sqw,#3*\sqw) rectangle +(#1*\sqw,#1*\sqw);}


% code 
\usepackage{algorithm2e}
\usepackage{listings}
\usepackage{xcolor}

\definecolor{codegreen}{rgb}{0,0.35,0}
\definecolor{codegray}{rgb}{0.5,0.5,0.5}
\definecolor{codenumber}{rgb}{0.1,0.3,0.5}
\definecolor{codeblue}{rgb}{0,0,0.5}
\definecolor{codered}{rgb}{0.5,0.03,0.02}
\definecolor{codegray}{rgb}{0.96,0.96,0.96}

\lstdefinestyle{pythonstylesheet}{
    language=Java,
    emphstyle=\color{deepred},
    backgroundcolor=\color{codegray},
    keywordstyle=\color{deepblue}\bfseries\itshape,
    numberstyle=\scriptsize\color{codenumber},
    basicstyle=\ttfamily\footnotesize,
    commentstyle=\color{codegreen}\itshape,
    breakatwhitespace=false, 
    breaklines=true, 
    captionpos=b, 
    keepspaces=true, 
    numbers=left, 
    numbersep=5pt, 
    showspaces=false,                
    showstringspaces=false,
    showtabs=false, 
    tabsize=4, 
    morekeywords={as,assert,nonlocal,with,yield,self,True,False,None,AssertionError,ValueError,in,else},              % Add keywords here
    keywordstyle=\color{codeblue},
    emph={var, List, Iterable, Iterator},          % Custom highlighting
    emphstyle=\color{codered},
    stringstyle=\color{codegreen},
    showstringspaces=false,
    abovecaptionskip=0pt,belowcaptionskip =0pt,
    framextopmargin=-\topsep, 
}
\newcommand\pythonstyle{\lstset{pythonstylesheet}}
\newcommand\pyl[1]     {{\lstinline!#1!}}
\lstset{style=pythonstylesheet}

\usepackage[style=1,skipbelow=\topskip,skipabove=\topskip,framemethod=TikZ]{mdframed}
\definecolor{bggray}{rgb}{0.85, 0.85, 0.85}
\mdfsetup{leftmargin=0pt,rightmargin=0pt,innerleftmargin=15pt,backgroundcolor=codegray,middlelinewidth=0.5pt,skipabove=5pt,skipbelow=0pt,middlelinecolor=black,roundcorner=5}
\BeforeBeginEnvironment{lstlisting}{\begin{mdframed}\vspace{-0.4em}}
    \AfterEndEnvironment{lstlisting}{\vspace{-0.8em}\end{mdframed}}


% Deisgn
\usepackage[labelfont=bf]{caption}
\usepackage[margin=0.6in]{geometry}
\usepackage{multicol}
\usepackage[skip=4pt, indent=0pt]{parskip}
\usepackage[normalem]{ulem}
\forestset{default}
\renewcommand\labelitemi{$\bullet$}
\usepackage{titlesec}
\titleformat{\section}[block]
{\fontsize{15}{15}}
{\sen \dotfill (\thesection)\dotfill\she}
{0em}
{\MakeUppercase}
\usepackage{graphicx}
\graphicspath{ {./} }

\usepackage[colorlinks]{hyperref}
\definecolor{mgreen}{RGB}{25, 160, 50}
\definecolor{mblue}{RGB}{30, 60, 200}
\usepackage{hyperref}
\hypersetup{
    colorlinks=true,
    citecolor=mgreen,
    linkcolor=black,
    urlcolor=mblue,
    pdftitle={Document by Shahar Perets},
    %	pdfpagemode=FullScreen,
}


% Hebrew initialzing
\usepackage[bidi=basic]{babel}
\PassOptionsToPackage{no-math}{fontspec}
\babelprovide[main, import, Alph=letters]{hebrew}
\babelprovide[import]{english}
\babelfont[hebrew]{rm}{David CLM}
\babelfont[hebrew]{sf}{David CLM}
%\babelfont[english]{tt}{Monaspace Xenon}
\usepackage[shortlabels]{enumitem}
\newlist{hebenum}{enumerate}{1}

% Language Shortcuts
\newcommand\en[1] {\begin{otherlanguage}{english}#1\end{otherlanguage}}
\newcommand\he[1] {\she#1\sen}
\newcommand\sen   {\begin{otherlanguage}{english}}
    \newcommand\she   {\end{otherlanguage}}
\newcommand\del   {$ \!\! $}

\newcommand\npage {\vfil {\hfil \textbf{\textit{המשך בעמוד הבא}}} \hfil \vfil \pagebreak}
\newcommand\ndoc  {\dotfill \\ \vfil {\begin{center}
            {\textbf{\textit{שחר פרץ, 2025}} \\
                \scriptsize \textit{קומפל ב־}\en{\LaTeX}\,\textit{ ונוצר באמצעות תוכנה חופשית בלבד}}
    \end{center}} \vfil	}

\newcommand{\rn}[1]{
    \textup{\uppercase\expandafter{\romannumeral#1}}
}

\makeatletter
\newcommand{\skipitems}[1]{
    \addtocounter{\@enumctr}{#1}
}
\makeatother

%! ~~~ Math shortcuts ~~~

% Letters shortcuts
\newcommand\N     {\mathbb{N}}
\newcommand\Z     {\mathbb{Z}}
\newcommand\R     {\mathbb{R}}
\newcommand\Q     {\mathbb{Q}}
\newcommand\C     {\mathbb{C}}
\newcommand\One   {\mathit{1}}

\newcommand\ml    {\ell}
\newcommand\mj    {\jmath}
\newcommand\mi    {\imath}

\newcommand\powerset {\mathcal{P}}
\newcommand\ps    {\mathcal{P}}
\newcommand\pc    {\mathcal{P}}
\newcommand\ac    {\mathcal{A}}
\newcommand\bc    {\mathcal{B}}
\newcommand\cc    {\mathcal{C}}
\newcommand\dc    {\mathcal{D}}
\newcommand\ec    {\mathcal{E}}
\newcommand\fc    {\mathcal{F}}
\newcommand\nc    {\mathcal{N}}
\newcommand\vc    {\mathcal{V}} % Vance
\newcommand\sca   {\mathcal{S}} % \sc is already definded
\newcommand\rca   {\mathcal{R}} % \rc is already definded

\newcommand\prm   {\mathrm{p}}
\newcommand\arm   {\mathrm{a}} % x86
\newcommand\brm   {\mathrm{b}}
\newcommand\crm   {\mathrm{c}}
\newcommand\drm   {\mathrm{d}}
\newcommand\erm   {\mathrm{e}}
\newcommand\frm   {\mathrm{f}}
\newcommand\nrm   {\mathrm{n}}
\newcommand\vrm   {\mathrm{v}}
\newcommand\srm   {\mathrm{s}}
\newcommand\rrm   {\mathrm{r}}

\newcommand\Si    {\Sigma}

% Logic & sets shorcuts
\newcommand\siff  {\longleftrightarrow}
\newcommand\ssiff {\leftrightarrow}
\newcommand\so    {\longrightarrow}
\newcommand\sso   {\rightarrow}

\newcommand\epsi  {\epsilon}
\newcommand\vepsi {\varepsilon}
\newcommand\vphi  {\varphi}
\newcommand\Neven {\N_{\mathrm{even}}}
\newcommand\Nodd  {\N_{\mathrm{odd }}}
\newcommand\Zeven {\Z_{\mathrm{even}}}
\newcommand\Zodd  {\Z_{\mathrm{odd }}}
\newcommand\Np    {\N_+}

% Text Shortcuts
\newcommand\open  {\big(}
\newcommand\qopen {\quad\big(}
\newcommand\close {\big)}
\newcommand\also  {\mathrm{, }}
\newcommand\defis {\mathrm{ definitions}}
\newcommand\given {\mathrm{given }}
\newcommand\case  {\mathrm{if }}
\newcommand\syx   {\mathrm{ syntax}}
\newcommand\rle   {\mathrm{ rule}}
\newcommand\other {\mathrm{else}}
\newcommand\set   {\ell et \text{ }}
\newcommand\ans   {\mathscr{A}\!\mathit{nswer}}

% Set theory shortcuts
\newcommand\ra    {\rangle}
\newcommand\la    {\langle}

\newcommand\oto   {\leftarrow}

\newcommand\QED   {\quad\quad\mathscr{Q.E.D.}\;\;\blacksquare}
\newcommand\QEF   {\quad\quad\mathscr{Q.E.F.}}
\newcommand\eQED  {\mathscr{Q.E.D.}\;\;\blacksquare}
\newcommand\eQEF  {\mathscr{Q.E.F.}}
\newcommand\jQED  {\mathscr{Q.E.D.}}

\DeclareMathOperator\dom   {dom}
\DeclareMathOperator\Img   {Im}
\DeclareMathOperator\range {range}

\newcommand\trio  {\triangle}

\newcommand\rc    {\right\rceil}
\newcommand\lc    {\left\lceil}
\newcommand\rf    {\right\rfloor}
\newcommand\lf    {\left\lfloor}
\newcommand\ceil  [1] {\lc #1 \rc}
\newcommand\floor [1] {\lf #1 \rf}

\newcommand\lex   {<_{lex}}

\newcommand\az    {\aleph_0}
\newcommand\uaz   {^{\aleph_0}}
\newcommand\al    {\aleph}
\newcommand\ual   {^\aleph}
\newcommand\taz   {2^{\aleph_0}}
\newcommand\utaz  { ^{\left (2^{\aleph_0} \right )}}
\newcommand\tal   {2^{\aleph}}
\newcommand\utal  { ^{\left (2^{\aleph} \right )}}
\newcommand\ttaz  {2^{\left (2^{\aleph_0}\right )}}

\newcommand\n     {$n$־יה\ }

% Math A&B shortcuts
\newcommand\logn  {\log n}
\newcommand\logx  {\log x}
\newcommand\lnx   {\ln x}
\newcommand\cosx  {\cos x}
\newcommand\sinx  {\sin x}
\newcommand\sint  {\sin \theta}
\newcommand\tanx  {\tan x}
\newcommand\tant  {\tan \theta}
\newcommand\sex   {\sec x}
\newcommand\sect  {\sec^2}
\newcommand\cotx  {\cot x}
\newcommand\cscx  {\csc x}
\newcommand\sinhx {\sinh x}
\newcommand\coshx {\cosh x}
\newcommand\tanhx {\tanh x}

\newcommand\seq   {\overset{!}{=}}
\newcommand\slh   {\overset{LH}{=}}
\newcommand\sle   {\overset{!}{\le}}
\newcommand\sge   {\overset{!}{\ge}}
\newcommand\sll   {\overset{!}{<}}
\newcommand\sgg   {\overset{!}{>}}

\newcommand\h     {\hat}
\newcommand\ve    {\vec}
\newcommand\lv    {\overrightarrow}
\newcommand\ol    {\overline}

\newcommand\mlcm  {\mathrm{lcm}}

\DeclareMathOperator{\Sp}     {span} 
\DeclareMathOperator{\sgn}    {sgn} 
\DeclareMathOperator{\row}    {Row} 
\DeclareMathOperator{\adj}    {adj} 
\DeclareMathOperator{\rk}     {rank} 
\DeclareMathOperator{\col}    {Col} 
\DeclareMathOperator{\tr}     {tr}

% Linear Algebra
\DeclareMathOperator{\chr}     {char}
\DeclareMathOperator{\diag}    {diag}
\DeclareMathOperator{\Hom}     {Hom}
\DeclareMathOperator{\Sym}     {Sym}
\DeclareMathOperator{\Asym}    {ASym}
\newcommand\lcm                {\ell\mathrm{cm}}

\newcommand\lra       {\leftrightarrow}
\newcommand\chrf      {\chr(\F)}
\newcommand\F         {\mathbb{F}}
\newcommand\K         {\mathbb{K}}
\newcommand\co        {\colon}
\newcommand\pms[1]    {\begin{pmatrix}
        #1
\end{pmatrix}}


% Greek Letters
\newcommand\ag        {\alpha}
\newcommand\bg        {\beta}
\newcommand\cg        {\gamma}
\newcommand\dg        {\delta}
\newcommand\eg        {\epsi}
\newcommand\zg        {\zeta}
\newcommand\hg        {\eta}
\newcommand\tg        {\theta}
\newcommand\ig        {\iota}
\newcommand\kg        {\keppa}
\renewcommand\lg      {\lambda}
\newcommand\og        {\omicron}
\newcommand\rg        {\rho}
\newcommand\sg        {\sigma}
\newcommand\yg        {\usilon}
\newcommand\wg        {\omega}

% Other shortcuts
\newcommand\tl    {\tilde}
\newcommand\op    {^{-1}}

\newcommand\sof[1]    {\left | #1 \right |}
\newcommand\cl [1]    {\left ( #1 \right )}
\newcommand\csb[1]    {\left [ #1 \right ]}
\newcommand\ccb[1]    {\left \{ #1 \right \}}

\newcommand\bs        {\blacksquare}
\newcommand\dequad    {\!\!\!\!\!\!}
\newcommand\dequadd   {\dequad\duquad}

\renewcommand\phi     {\varphi}

\newtheorem{Theorem}{משפט}
\theoremstyle{definition}
\newtheorem{definition}{הגדרה}
\newtheorem{Lemma}{למה}
\newtheorem{Remark}{הערה}
\newtheorem{Notion}{סימון}
\newtheorem{Hence}{מסקנה}

\newcommand\theo  [1] {\begin{Theorem}#1\end{Theorem}}
\newcommand\defi  [1] {\begin{definition}#1\end{definition}}
\newcommand\rmark [1] {\begin{Remark}#1\end{Remark}}
\newcommand\lem   [1] {\begin{Lemma}#1\end{Lemma}}
\newcommand\noti  [1] {\begin{Notion}#1\end{Notion}}


% Algorithems
\newcommand\sFunc [1] {\SetKwFunction{#1}{#1}}
\newcommand\sData [1] {\SetKwData{#1}{#1}}
\newcommand\sIO   [1] {\SetKwInOut{#1}{#1}}
\newcommand\ttt   [1] {\sen \texttt{#1} \she\,}
\newcommand\io    [2] {\Input{#1}\Output{#2}\BlankLine}



%! ~~~ Document ~~~

\author{שחר פרץ}
\title{\textit{לינארית 2א 10}}
\date{12 במאי 2025}
\begin{document}
    \maketitle
    \textbf{מרצה: }בן בסקין
    תזכורת: שיעור שעבר עצרנו בהוכחת המשפט הבא. 
    
    \theo{תהי $A$ מטריצת בלוקים ריבועיים על האלכסון $A = \diag(A_1 \dots A_n)$, אז מתקיים $m_A(x) = \lcm(m_{A_1} \dots m_{A_k})$}
    
    \theo{$V$ מ''ו מעל $\F$, ו־$T \co V \to V$ ט''ל, יהי $f(x) \in \F[x]$ כך ש־$f(T) = 0$. נניח ש־$f = g \cdot h$, כאשר $\gcd(g, h) = 1$, אז: 
    \begin{enumerate}
        \item \hfil $V = \ker g(T) \oplus \ker h(T)$
        \item אם $f(x) = m_T(x)$ אז בפירוק לעיל $g, h$ הם הפולינום המינימליים של צמצום $T$ לתתי־המרחבים (הקרנלים) בהתאמה. 
    \end{enumerate}
    }
    
    נחזור על מה שהתחלנו להוכיח ונסיים את אשר נותר: 
    \begin{proof}
        ידוע $h= g \cdot h$ ולכן $\exists a(x), b(x) \in \F[x]$ כך ש־$a(x)g(x) + b(x)h(x) = 1$, כך ש־: 
        \[ \underbrace{(a(T) \circ g(T))(v)}_{\in \ker h(T)} + \underbrace{(b(T) \circ h(T))(v)}_{\in \ker g(T)} = V \]
        ולכן $V = \ker h(T) + \ker g(T)$. 
        וכן הראינו שזהו סכום ישר. עתה, נסמן: 
        \begin{align*}
            W_2 = \ker h(T) & W_1 = \ker g(T) \\
            T_2 =  T_{|_{W_2}} & T_1 = T_{|_{W_1}}
        \end{align*}
        וכן $B_1$ בסיס ל־$W_1$, $B_2$ ל־$W_2$. לכן $B = B_1 \uplus B_2$ בסיס ל־$V$. משום שהראינו ש־$W_1, W_2$ הם $T$־אינוו' בשיעור קודם: 
        \[ [T]_B = \pms{[T_1] & 0 \\ 0 & [T_2]} \]
        מהמשפט שראינו, $m_T = \lcm(m_{T_1}, m_{T_2})$. ברור ש־$m_{T_1} \mid g$ וגם $m_{T_2} \mid h$. אז: 
        \[ \deg m_{T_1} = \deg g + \deg h \ge \deg m_{T_1} + \deg m_{T_2} = \deg(m_{T_1} \cdot m_{T_2}) \ge \deg(\lcm(m_{T_1}, m_{T_2})) = \deg m_T \]
        ולכן כל ה''אשים לעיל הדוקים ושוויון בכל מקום. 
        \[ \deg m_{T_1} \le \deg g \land \deg m_{T_2} \le \deg h \]
        אם אחד מהשווינות לא הדוקים, אז: 
        \[ \deg m_{T_1} + \deg m_{T_2} < \deg g + \deg h \]
        וסתירה למה שהראינו. לכן: 
        \[ m_{T_1} \mid g \land \deg m_{T_1} = \deg g \implies m_{T_1} \sim g \]
        אבל שניהם מתוקנים ולכן שווים. כנ''ל עבור $m_{T_2} = h$. 
    \end{proof}
    
    
    \textbf{דוגמה. }נסמן $f(x) = x^{2}(x - 1)^{3}$, $f(T) = 0$. החלק הראשון של המשפט אומר $V = \ker T^2 \oplus \ker (T - I)^3$. החלק השני אומר שאם $f = m_T$ אז $x^2$ הוא הפולינום המינימלי של $T_{|_{\ker T^2}}$ וכן $(x - 1)^{3}$ המינילי של $T_{|_{T - I}^{3}}$. 
    
    \theo{(משפט הפירוק הפרימרי): 
    יהיו $T \co V \to V$, $m_T$ הפולינום המינימלי של $T$, ונניח ש־: 
    \[ m_T = g_1 \cdots g_s \quad \forall i \neq j \co \gcd(g_i, g_j) = 1 \]
    אז: 
    \[ V = \bigoplus_{i = 1}^{s} \ker g_i(T) \]
    ובנוסף $g_i$ הוא הפולינום המינימלי של $T_{|_{\ker g_i(T)}}$. זה פשוט אינדוקציה על המשפט הקודם. 
    } המרצה גם מוכיח את זה על הלוח אבל לא מתחשק לי לכתוב את זה. טוב, אני אכתוב את זה. 
    ``יש לו שם מפוצץ אז הוא כנראה חשוב''
    \begin{proof}באינדוקציה על $s$
        \begin{itemize}
            \item[בסיס: ]עבור $s = 2$ המשפט שהוכחנו. 
            \item[צעד: ]נסמן: 
            \[ h(x) = g_s(x), \ g(x) = \prod_{i = 1}^{s - 1} g_i(x)\]
            ואז: 
            \[ \forall i \neq j \co \gcd(g_i, g_j) = 1 \implies \gcd(h, h) = 1 \]
            מהמשפט שקיבלנו: 
            \[ V = \ker g(T) \oplus \ker h(T) \,\overset{\mathclap{\text{אינדוקציה}}}{\implies}\,\, \bigoplus_{i = 1}^{s}\ker g_i(T) \]
            והמשך דומה עבור $m_{T|_{\ker g_i}} = g_i$: 
%            \[ \tl T \co T_{|_{\ker g_i}} \]
            \[ \ker h(T) = \bigoplus_{i = 1}^{s} \ker g_i (T_{|_{\ker h(T)}}) = \bigoplus_{i = 1}^{s} \ker g_i(T) \]
        \end{itemize}
    \end{proof}
    
    \begin{Hence}
        (מקרה פרטי חשוב): נניח כי $m_T$ מפרק לגורמים לינאריים שונים זה מזה. כלומר: 
        \[ m_T = \prod_{i = 1}^{s}(x - \lg _i) \]
        ן־$\lg_i \neq \lg_j$ לכל $i \neq j$, אז $T$ לכסינה. 
    \end{Hence}
    \begin{proof}
        לפי המשפט: 
        \[ V = \bigoplus_{i = 1}^{s} \ker (T - \lg_i I) \]
        $V$ סכום ישר של מ''ע $\lg_1 \dots \lg_2$ הם כולם ע''ע שונים, אז יש ל־$V$ בסיס של ו''א ולכן $T$ לכסינה. 
    \end{proof}
    
    \textbf{בעיה. }נתונה $A \in M_5(\Z)$. יש לקבוע אם היא לכסינה מעל $\C$. 
    \begin{itemize}
        \item נחשב עת $f_A$
        \item נמצא שורשים ל־$f_A$, אלו הם הע''ע של $A$. 
        \item לכל ע''ע $\lg$ נחשב את $V_\lg$ – המרחב העצמי ואת הממד שלו. 
        \item אם סכום הממדים שווה ל־$5$ אז $A$ לכסינה אחרת לא. 
    \end{itemize}
    \textbf{הבעיה: }לא קיימת נוסחאת שורשים לפולינום כללי מעל $5$. 
    
    \noti{את $f(x) = \prod_{\lg_i}(x - \lg_i)^{r_i}$  בהנחה ש־$\lg_i \neq \lg_j$ אז $f^{\mathrm{red}} (x) = \prod_{i} (x - \lg_i)$}
    \theo{\hfil $f^{{\mathrm{red}}} = \frac{f}{\gcd(f, f')}$}
    (כאשר $f'$ הנגזרת של $f$)
    \theo{$A$ לכסינה אמ''מ $f_A^{\mathrm{red}}(A) = 0$}
    
    \ndoc
\end{document}