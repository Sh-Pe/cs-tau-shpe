%! ~~~ Packages Setup ~~~ 
\documentclass[]{article}
\usepackage{lipsum}
\usepackage{rotating}


% Math packages
\usepackage[usenames]{color}
\usepackage{forest}
\usepackage{ifxetex,ifluatex,amssymb,amsmath,mathrsfs,amsthm,witharrows,mathtools,mathdots}
\usepackage{amsmath}
\WithArrowsOptions{displaystyle}
\renewcommand{\qedsymbol}{$\blacksquare$} % end proofs with \blacksquare. Overwrites the defualts. 
\usepackage{cancel,bm}
\usepackage[thinc]{esdiff}


% tikz
\usepackage{tikz}
\usetikzlibrary{graphs}
\newcommand\sqw{1}
\newcommand\squ[4][1]{\fill[#4] (#2*\sqw,#3*\sqw) rectangle +(#1*\sqw,#1*\sqw);}


% code 
\usepackage{algorithm2e}
\usepackage{listings}
\usepackage{xcolor}

\definecolor{codegreen}{rgb}{0,0.35,0}
\definecolor{codegray}{rgb}{0.5,0.5,0.5}
\definecolor{codenumber}{rgb}{0.1,0.3,0.5}
\definecolor{codeblue}{rgb}{0,0,0.5}
\definecolor{codered}{rgb}{0.5,0.03,0.02}
\definecolor{codegray}{rgb}{0.96,0.96,0.96}

\lstdefinestyle{pythonstylesheet}{
    language=Java,
    emphstyle=\color{deepred},
    backgroundcolor=\color{codegray},
    keywordstyle=\color{deepblue}\bfseries\itshape,
    numberstyle=\scriptsize\color{codenumber},
    basicstyle=\ttfamily\footnotesize,
    commentstyle=\color{codegreen}\itshape,
    breakatwhitespace=false, 
    breaklines=true, 
    captionpos=b, 
    keepspaces=true, 
    numbers=left, 
    numbersep=5pt, 
    showspaces=false,                
    showstringspaces=false,
    showtabs=false, 
    tabsize=4, 
    morekeywords={as,assert,nonlocal,with,yield,self,True,False,None,AssertionError,ValueError,in,else},              % Add keywords here
    keywordstyle=\color{codeblue},
    emph={var, List, Iterable, Iterator},          % Custom highlighting
    emphstyle=\color{codered},
    stringstyle=\color{codegreen},
    showstringspaces=false,
    abovecaptionskip=0pt,belowcaptionskip =0pt,
    framextopmargin=-\topsep, 
}
\newcommand\pythonstyle{\lstset{pythonstylesheet}}
\newcommand\pyl[1]     {{\lstinline!#1!}}
\lstset{style=pythonstylesheet}

\usepackage[style=1,skipbelow=\topskip,skipabove=\topskip,framemethod=TikZ]{mdframed}
\definecolor{bggray}{rgb}{0.85, 0.85, 0.85}
\mdfsetup{leftmargin=0pt,rightmargin=0pt,innerleftmargin=15pt,backgroundcolor=codegray,middlelinewidth=0.5pt,skipabove=5pt,skipbelow=0pt,middlelinecolor=black,roundcorner=5}
\BeforeBeginEnvironment{lstlisting}{\begin{mdframed}\vspace{-0.4em}}
    \AfterEndEnvironment{lstlisting}{\vspace{-0.8em}\end{mdframed}}


% Design
\usepackage[labelfont=bf]{caption}
\usepackage[margin=0.6in]{geometry}
\usepackage{multicol}
\usepackage[skip=4pt, indent=0pt]{parskip}
\usepackage[normalem]{ulem}
\forestset{default}
\renewcommand\labelitemi{$\bullet$}
\usepackage{titlesec}
\titleformat{\section}[block]
{\fontsize{15}{15}}
{\sen \dotfill (\thesection)\she}
{0em}
{\MakeUppercase}
\usepackage{graphicx}
\graphicspath{ {./} }

\usepackage[colorlinks]{hyperref}
\definecolor{mgreen}{RGB}{25, 160, 50}
\definecolor{mblue}{RGB}{30, 60, 200}
\usepackage{hyperref}
\hypersetup{
    colorlinks=true,
    citecolor=mgreen,
    linkcolor=black,
    urlcolor=mblue,
    pdftitle={Document by Shahar Perets},
    %	pdfpagemode=FullScreen,
}
\usepackage{yfonts}
\def\gothstart#1{\noindent\smash{\lower3ex\hbox{\llap{\Huge\gothfamily#1}}}
    \parshape=3 3.1em \dimexpr\hsize-3.4em 3.4em \dimexpr\hsize-3.4em 0pt \hsize}
\def\frakstart#1{\noindent\smash{\lower3ex\hbox{\llap{\Huge\frakfamily#1}}}
    \parshape=3 1.5em \dimexpr\hsize-1.5em 2em \dimexpr\hsize-2em 0pt \hsize}



% Hebrew initialzing
\usepackage[bidi=basic]{babel}
\PassOptionsToPackage{no-math}{fontspec}
\babelprovide[main, import, Alph=letters]{hebrew}
\babelprovide[import]{english}
\babelfont[hebrew]{rm}{David CLM}
\babelfont[hebrew]{sf}{David CLM}
%\babelfont[english]{tt}{Monaspace Xenon}
\usepackage[shortlabels]{enumitem}
\newlist{hebenum}{enumerate}{1}

% Language Shortcuts
\newcommand\en[1] {\begin{otherlanguage}{english}#1\end{otherlanguage}}
\newcommand\he[1] {\she#1\sen}
\newcommand\sen   {\begin{otherlanguage}{english}}
    \newcommand\she   {\end{otherlanguage}}
\newcommand\del   {$ \!\! $}

\newcommand\npage {\vfil {\hfil \textbf{\textit{המשך בעמוד הבא}}} \hfil \vfil \pagebreak}
\newcommand\ndoc  {\dotfill \\ \vfil {\begin{center}
            {\textbf{\textit{שחר פרץ, 2025}} \\
                \scriptsize \textit{קומפל ב־}\en{\LaTeX}\,\textit{ ונוצר באמצעות תוכנה חופשית בלבד}}
    \end{center}} \vfil	}

\newcommand{\rn}[1]{
    \textup{\uppercase\expandafter{\romannumeral#1}}
}

\makeatletter
\newcommand{\skipitems}[1]{
    \addtocounter{\@enumctr}{#1}
}
\makeatother

%! ~~~ Math shortcuts ~~~

% Letters shortcuts
\newcommand\N     {\mathbb{N}}
\newcommand\Z     {\mathbb{Z}}
\newcommand\R     {\mathbb{R}}
\newcommand\Q     {\mathbb{Q}}
\newcommand\C     {\mathbb{C}}
\newcommand\One   {\mathit{1}}

\newcommand\ml    {\ell}
\newcommand\mj    {\jmath}
\newcommand\mi    {\imath}

\newcommand\powerset {\mathcal{P}}
\newcommand\ps    {\mathcal{P}}
\newcommand\pc    {\mathcal{P}}
\newcommand\ac    {\mathcal{A}}
\newcommand\bc    {\mathcal{B}}
\newcommand\cc    {\mathcal{C}}
\newcommand\dc    {\mathcal{D}}
\newcommand\ec    {\mathcal{E}}
\newcommand\fc    {\mathcal{F}}
\newcommand\nc    {\mathcal{N}}
\newcommand\vc    {\mathcal{V}} % Vance
\newcommand\sca   {\mathcal{S}} % \sc is already definded
\newcommand\rca   {\mathcal{R}} % \rc is already definded

\newcommand\prm   {\mathrm{p}}
\newcommand\arm   {\mathrm{a}} % x86
\newcommand\brm   {\mathrm{b}}
\newcommand\crm   {\mathrm{c}}
\newcommand\drm   {\mathrm{d}}
\newcommand\erm   {\mathrm{e}}
\newcommand\frm   {\mathrm{f}}
\newcommand\nrm   {\mathrm{n}}
\newcommand\vrm   {\mathrm{v}}
\newcommand\srm   {\mathrm{s}}
\newcommand\rrm   {\mathrm{r}}

\newcommand\Si    {\Sigma}

% Logic & sets shorcuts
\newcommand\siff  {\longleftrightarrow}
\newcommand\ssiff {\leftrightarrow}
\newcommand\so    {\longrightarrow}
\newcommand\sso   {\rightarrow}

\newcommand\epsi  {\epsilon}
\newcommand\vepsi {\varepsilon}
\newcommand\vphi  {\varphi}
\newcommand\Neven {\N_{\mathrm{even}}}
\newcommand\Nodd  {\N_{\mathrm{odd }}}
\newcommand\Zeven {\Z_{\mathrm{even}}}
\newcommand\Zodd  {\Z_{\mathrm{odd }}}
\newcommand\Np    {\N_+}

% Text Shortcuts
\newcommand\open  {\big(}
\newcommand\qopen {\quad\big(}
\newcommand\close {\big)}
\newcommand\also  {\mathrm{, }}
\newcommand\defis {\mathrm{ definitions}}
\newcommand\given {\mathrm{given }}
\newcommand\case  {\mathrm{if }}
\newcommand\syx   {\mathrm{ syntax}}
\newcommand\rle   {\mathrm{ rule}}
\newcommand\other {\mathrm{else}}
\newcommand\set   {\ell et \text{ }}
\newcommand\ans   {\mathscr{A}\!\mathit{nswer}}

% Set theory shortcuts
\newcommand\ra    {\rangle}
\newcommand\la    {\langle}

\newcommand\oto   {\leftarrow}

\newcommand\QED   {\quad\quad\mathscr{Q.E.D.}\;\;\blacksquare}
\newcommand\QEF   {\quad\quad\mathscr{Q.E.F.}}
\newcommand\eQED  {\mathscr{Q.E.D.}\;\;\blacksquare}
\newcommand\eQEF  {\mathscr{Q.E.F.}}
\newcommand\jQED  {\mathscr{Q.E.D.}}

\DeclareMathOperator\dom   {dom}
\DeclareMathOperator\Img   {Im}
\DeclareMathOperator\range {range}

\newcommand\trio  {\triangle}

\newcommand\rc    {\right\rceil}
\newcommand\lc    {\left\lceil}
\newcommand\rf    {\right\rfloor}
\newcommand\lf    {\left\lfloor}
\newcommand\ceil  [1] {\lc #1 \rc}
\newcommand\floor [1] {\lf #1 \rf}

\newcommand\lex   {<_{lex}}

\newcommand\az    {\aleph_0}
\newcommand\uaz   {^{\aleph_0}}
\newcommand\al    {\aleph}
\newcommand\ual   {^\aleph}
\newcommand\taz   {2^{\aleph_0}}
\newcommand\utaz  { ^{\left (2^{\aleph_0} \right )}}
\newcommand\tal   {2^{\aleph}}
\newcommand\utal  { ^{\left (2^{\aleph} \right )}}
\newcommand\ttaz  {2^{\left (2^{\aleph_0}\right )}}

\newcommand\n     {$n$־יה\ }

% Math A&B shortcuts
\newcommand\logn  {\log n}
\newcommand\logx  {\log x}
\newcommand\lnx   {\ln x}
\newcommand\cosx  {\cos x}
\newcommand\sinx  {\sin x}
\newcommand\sint  {\sin \theta}
\newcommand\tanx  {\tan x}
\newcommand\tant  {\tan \theta}
\newcommand\sex   {\sec x}
\newcommand\sect  {\sec^2}
\newcommand\cotx  {\cot x}
\newcommand\cscx  {\csc x}
\newcommand\sinhx {\sinh x}
\newcommand\coshx {\cosh x}
\newcommand\tanhx {\tanh x}

\newcommand\seq   {\overset{!}{=}}
\newcommand\slh   {\overset{LH}{=}}
\newcommand\sle   {\overset{!}{\le}}
\newcommand\sge   {\overset{!}{\ge}}
\newcommand\sll   {\overset{!}{<}}
\newcommand\sgg   {\overset{!}{>}}

\newcommand\h     {\hat}
\newcommand\ve    {\vec}
\newcommand\lv    {\overrightarrow}
\newcommand\ol    {\overline}

\newcommand\mlcm  {\mathrm{lcm}}

\DeclareMathOperator{\sech}   {sech}
\DeclareMathOperator{\csch}   {csch}
\DeclareMathOperator{\arcsec} {arcsec}
\DeclareMathOperator{\arccot} {arcCot}
\DeclareMathOperator{\arccsc} {arcCsc}
\DeclareMathOperator{\arccosh}{arccosh}
\DeclareMathOperator{\arcsinh}{arcsinh}
\DeclareMathOperator{\arctanh}{arctanh}
\DeclareMathOperator{\arcsech}{arcsech}
\DeclareMathOperator{\arccsch}{arccsch}
\DeclareMathOperator{\arccoth}{arccoth}
\DeclareMathOperator{\atant}  {atan2} 
\DeclareMathOperator{\Sp}     {span} 
\DeclareMathOperator{\sgn}    {sgn} 
\DeclareMathOperator{\row}    {Row} 
\DeclareMathOperator{\adj}    {adj} 
\DeclareMathOperator{\rk}     {rank} 
\DeclareMathOperator{\col}    {Col} 
\DeclareMathOperator{\tr}     {tr}

\newcommand\dx    {\,\mathrm{d}x}
\newcommand\dt    {\,\mathrm{d}t}
\newcommand\dtt   {\,\mathrm{d}\theta}
\newcommand\du    {\,\mathrm{d}u}
\newcommand\dv    {\,\mathrm{d}v}
\newcommand\df    {\mathrm{d}f}
\newcommand\dfdx  {\diff{f}{x}}
\newcommand\dit   {\limhz \frac{f(x + h) - f(x)}{h}}

\newcommand\nt[1] {\frac{#1}{#1}}

\newcommand\limz  {\lim_{x \to 0}}
\newcommand\limxz {\lim_{x \to x_0}}
\newcommand\limi  {\lim_{x \to \infty}}
\newcommand\limh  {\lim_{x \to 0}}
\newcommand\limni {\lim_{x \to - \infty}}
\newcommand\limpmi{\lim_{x \to \pm \infty}}

\newcommand\ta    {\theta}
\newcommand\ap    {\alpha}

\renewcommand\inf {\infty}
\newcommand  \ninf{-\inf}

% Combinatorics shortcuts
\newcommand\sumnk     {\sum_{k = 0}^{n}}
\newcommand\sumni     {\sum_{i = 0}^{n}}
\newcommand\sumnko    {\sum_{k = 1}^{n}}
\newcommand\sumnio    {\sum_{i = 1}^{n}}
\newcommand\sumai     {\sum_{i = 1}^{n} A_i}
\newcommand\nsum[2]   {\reflectbox{\displaystyle\sum_{\reflectbox{\scriptsize$#1$}}^{\reflectbox{\scriptsize$#2$}}}}

\newcommand\bink      {\binom{n}{k}}
\newcommand\setn      {\{a_i\}^{2n}_{i = 1}}
\newcommand\setc[1]   {\{a_i\}^{#1}_{i = 1}}

\newcommand\cupain    {\bigcup_{i = 1}^{n} A_i}
\newcommand\cupai[1]  {\bigcup_{i = 1}^{#1} A_i}
\newcommand\cupiiai   {\bigcup_{i \in I} A_i}
\newcommand\capain    {\bigcap_{i = 1}^{n} A_i}
\newcommand\capai[1]  {\bigcap_{i = 1}^{#1} A_i}
\newcommand\capiiai   {\bigcap_{i \in I} A_i}

\newcommand\xot       {x_{1, 2}}
\newcommand\ano       {a_{n - 1}}
\newcommand\ant       {a_{n - 2}}

% Linear Algebra
\DeclareMathOperator{\chr}     {char}
\DeclareMathOperator{\diag}    {diag}
\DeclareMathOperator{\Hom}     {Hom}
\DeclareMathOperator{\Sym}     {Sym}
\DeclareMathOperator{\Asym}    {ASym}

\newcommand\lra       {\leftrightarrow}
\newcommand\chrf      {\chr(\F)}
\newcommand\F         {\mathbb{F}}
\newcommand\co        {\colon}
\newcommand\tmat[2]   {\cl{\begin{matrix}
            #1
        \end{matrix}\, \middle\vert\, \begin{matrix}
            #2
\end{matrix}}}

\makeatletter
\newcommand\rrr[1]    {\xxrightarrow{1}{#1}}
\newcommand\rrt[2]    {\xxrightarrow{1}[#2]{#1}}
\newcommand\mat[2]    {M_{#1\times#2}}
\newcommand\gmat      {\mat{m}{n}(\F)}
\newcommand\tomat     {\, \dequad \longrightarrow}
\newcommand\pms[1]    {\begin{pmatrix}
        #1
\end{pmatrix}}

% someone's code from the internet: https://tex.stackexchange.com/questions/27545/custom-length-arrows-text-over-and-under
\makeatletter
\newlength\min@xx
\newcommand*\xxrightarrow[1]{\begingroup
    \settowidth\min@xx{$\m@th\scriptstyle#1$}
    \@xxrightarrow}
\newcommand*\@xxrightarrow[2][]{
    \sbox8{$\m@th\scriptstyle#1$}  % subscript
    \ifdim\wd8>\min@xx \min@xx=\wd8 \fi
    \sbox8{$\m@th\scriptstyle#2$} % superscript
    \ifdim\wd8>\min@xx \min@xx=\wd8 \fi
    \xrightarrow[{\mathmakebox[\min@xx]{\scriptstyle#1}}]
    {\mathmakebox[\min@xx]{\scriptstyle#2}}
    \endgroup}
\makeatother


% Greek Letters
\newcommand\ag        {\alpha}
\newcommand\bg        {\beta}
\newcommand\cg        {\gamma}
\newcommand\dg        {\delta}
\newcommand\eg        {\epsi}
\newcommand\zg        {\zeta}
\newcommand\hg        {\eta}
\newcommand\tg        {\theta}
\newcommand\ig        {\iota}
\newcommand\kg        {\keppa}
\renewcommand\lg      {\lambda}
\newcommand\og        {\omicron}
\newcommand\rg        {\rho}
\newcommand\sg        {\sigma}
\newcommand\yg        {\usilon}
\newcommand\wg        {\omega}

\newcommand\Ag        {\Alpha}
\newcommand\Bg        {\Beta}
\newcommand\Cg        {\Gamma}
\newcommand\Dg        {\Delta}
\newcommand\Eg        {\Epsi}
\newcommand\Zg        {\Zeta}
\newcommand\Hg        {\Eta}
\newcommand\Tg        {\Theta}
\newcommand\Ig        {\Iota}
\newcommand\Kg        {\Keppa}
\newcommand\Lg        {\Lambda}
\newcommand\Og        {\Omicron}
\newcommand\Rg        {\Rho}
\newcommand\Sg        {\Sigma}
\newcommand\Yg        {\Usilon}
\newcommand\Wg        {\Omega}

% Other shortcuts
\newcommand\tl    {\tilde}
\newcommand\op    {^{-1}}

\newcommand\sof[1]    {\left | #1 \right |}
\newcommand\cl [1]    {\left ( #1 \right )}
\newcommand\csb[1]    {\left [ #1 \right ]}
\newcommand\ccb[1]    {\left \{ #1 \right \}}

\newcommand\bs        {\blacksquare}
\newcommand\dequad    {\!\!\!\!\!\!}
\newcommand\dequadd   {\dequad\duquad}

\renewcommand\phi     {\varphi}

\newtheorem{Theorem}{משפט}
\theoremstyle{definition}
\newtheorem{definition}{הגדרה}
\newtheorem{Lemma}{למה}
\newtheorem{Remark}{הערה}
\newtheorem{Notion}{סימון}


\newcommand\theo  [1] {\begin{Theorem}#1\end{Theorem}}
\newcommand\defi  [1] {\begin{definition}#1\end{definition}}
\newcommand\rmark [1] {\begin{Remark}#1\end{Remark}}
\newcommand\lem   [1] {\begin{Lemma}#1\end{Lemma}}
\newcommand\noti  [1] {\begin{Notion}#1\end{Notion}}

% DS
\newcommand\limsi     {\limsup_{n \to \inf}}
\newcommand\limfi     {\liminf_{n \to \inf}}

\DeclareMathOperator\amort   {amort}
\DeclareMathOperator\worst   {worst}
\DeclareMathOperator\type    {type}
\DeclareMathOperator\cost    {cost}
\DeclareMathOperator\tim     {time}

\newcommand\dsList{
    \sFunc{List}
    \sFunc{Retrieve}
    \SetKwFunction{RetrieveFirst}{Retrieve-First}
    \SetKwFunction{RetrieveLast}{Retrieve-Last}
    \sFunc{Delete}
    \SetKwFunction{DeleteFirst}{Delete-First}
    \SetKwFunction{DeleteLast}{Delete-Last}
    \sFunc{Insert}
    \SetKwFunction{InsertFirst}{Insert-First}
    \SetKwFunction{InsertLast}{Insert-Last}
    \sFunc{Shift}
    \sFunc{Length}
    \sFunc{Concat}
    \sFunc{Plant}
    \sFunc{Split}
}
\newcommand\dsQueue{
    \sFunc{Queue}
    \sFunc{Enqueue}
    \sFunc{Head}
    \sFunc{Dequeue}
}
\newcommand\dsStack{
    \sFunc{Stack}
    \sFunc{Push}
    \sFunc{Top}
    \sFunc{Pop}
}
\newcommand\dsVector{
    \sFunc{Vector}
    \sFunc{Get}
    \sFunc{Set}
}
\newcommand\dsGraph{
    \sFunc{Graph}
    \sFunc{Edge}
    \SetKwFunction{AddEdge}{Add-Edge}
    \SetKwFunction{RemoveEdge}{Remove-Edge}
    \sFunc{InDeg} \sFunc{OutDeg}
}
\newcommand\importDs{
    \dsList
    \dsQueue
    \dsStack
    \dsVector
    \dsGraph
    \SetKwData{error}{\color{codered}error}
    \SetKwInOut{Input}{input}
    \SetKwInOut{Output}{output}
    \SetKwRepeat{Do}{do}{while}
    \SetKwData{Null}{\color{codeblue}null}
}


% Algorithems
\newcommand\sFunc [1] {\SetKwFunction{#1}{#1}}
\newcommand\sData [1] {\SetKwData{#1}{#1}}
\newcommand\sIO   [1] {\SetKwInOut{#1}{#1}}
\newcommand\ttt   [1] {\sen \texttt{#1} \she\,}
\newcommand\io    [2] {\Input{#1}\Output{#2}\BlankLine}


\newcommand\norm[1]   {\left \vert \left \vert #1 \right \vert \right \vert}
\newcommand\snorm     {\left \vert \left \vert \cdot \right \vert \right \vert}
\newcommand\smut      {\left \la \cdot \mid \cdot \right \ra}
\newcommand\mut[2]    {\left \la #1 \,\middle\vert\, #2 \right \ra}
\newcommand\zc        {\mathcal{Z}}

%! ~~~ Document ~~~

\author{שחר פרץ}
\title{\textit{לינארית 2א 16}}
\begin{document}
    \maketitle
    לגבי תרגלי הבית: 
    \begin{itemize}
        \item מי שלא יעשה תרגיל בית לא יוכל להכנס להרצאות. 
        \item מי שלא עושה שיעורי בית לא יוכל לגשת למבחן. 
    \end{itemize}
    
    תזכורת: הגדרת מכפלה פנימית. 
    וכן: 
    \[ \snorm \co V \to \R_{\ge 0} \land \norm{v} = \sqrt{\mut{v}{v}} \]
    ראינו: 
    \[ \forall t \in \F \co \norm{t \cdot v} = |t| \cdot \norm{v} \]
    \defi{(\textit{נרמול של וקטור}): וקטור בממ''ס $V$ יקרא וקטור יחידה אם $\norm{v} = 1$.  $0 \neq v \in V$. }
    \theo{\[ v'' := \frac{v}{\norm{v}} \] הוא וקטור יחידה}
    \begin{proof}
        \[ |v''| = \norm{\frac{v}{\norm{v}}} = \frac{1}{|\norm{v}|} \cdot \norm{v} \]
        וכן: 
        \[ \norm{v} > 0 \implies |\norm{v}| = \norm{v} \implies \norm{v''} = \frac{\norm{v}}{\norm{v}} = 1 \]
    \end{proof}
    \defi{יהי $V$ מ''ו מעל $\F$, ו־$\snorm \co V \to \R_{\ge 0}$, אז $(V, \snorm)$ יקרא \textit{מרחב נורמי}. }
    \theo{(\textit{''נוסחאת הפולריזציה'')}בהינתן $(V, \snorm)$ מרחב נורמי;
    \textbf{גרסה מעל $\R$: }
    \[ \forall v, u \in V \co \mut{v}{u} = \frac{1}{4}(\norm{u + v}^2 + \norm{u - v}^2) \]
    \textbf{גרסה מעל $\C$: }
    \[ \mut{u}{v} = \frac{1}{4}\Big(\norm{u + v}^2 - \norm{u - v}^2 + i\norm{u + iv} - i\norm{u + iv}\Big) \]
    }
   \begin{proof}[הוכחה (ל־$\C$)]
        \begin{align*}
            \mut{u + v}{u + v} =\,& \norm{u}^2 + \mut{u}{v} + \mut{v}{u} + \norm{v}^2 \\
            =\,& \norm{u}^2 + \norm{v}^2 + 2\Re(\mut{v}{u}) \\
            \mut{v - u}{v - u} = \, & \norm{u}^2 + \norm{v}^2 - 2\Re(\mut{v}{u}) \\
            \mut{u + iv}{u + iv} =\,& \norm{u}^2 + \norm{v}^2 + \mut{u}{iv} + \mut{iv}{u}  \\
            =\,& \norm{u}^2 + \norm{v}^2 - i\mut{u}{v} + i\ol{\mut{u}{v}} \\
            =\,& \norm{u}^2 + \norm{v}^2 - i(2 \Im\mut{u}{v}) \\
            \norm{u - iv} =\,& \norm{u} + \norm{v} - \mut{u}{iv} - \mut{iv}{u} \\
            =\,& \norm{u} + \norm{v} - 2\Im(\mut{v}{u})
        \end{align*}
        וסה''כ אם נציב בנוסחה, אחרי שחישבנו את כל אבירה, הכל יצטמצם וש־$\mut{u}{v}$ אכן שווה לדרוש. 
    \end{proof}
    
    מנוסחאת הפולריזציה, נוכל לשחזר באמצעות נורמה את המכפלה הפנימית. 
    
    \defi{בהינתן $(v, \smut)$ ממ''פ, לכל $v \in V$ נאמר ש־$u$ \textit{מאונך ל־$v$} ונסמן $u\perp v$ אם $\mut{u}{v} = 0$. }
    
    \textit{הערה. }אם $u \perp v$ אז $v \perp u$. (כי צמוד של $0$ הוא $0$). 
    
    \theo{(משפט פיתגורס) (מאוד מועיל) יהי $V$ ממ''פ כך ש-$v \in V$ אז $\norm{v + u}^2 = \norm{v}^2 + \norm{u}^2$}\begin{proof} משום שהם מאונכים מתקיים $\mut{v}{u} = 0$. נפתח אלגברה: 
        \[ \norm{v + u}^2 = \mut{v + u}{v + u} = \norm{v}^2 + \cancel{\mut{v}{u}} + \cancel{\mut{u}{v}} + \norm{u^2} = \norm{v}^2 + \norm{u}^2 \quad \top \]
    \end{proof}
    
    \textit{הערה: }בעבור $v = \R^n$ מ''פ סטנדרטית אז $\norm{v}$ מזדהה עם מושג הגודל של וקטור בגיואמטריה רגילה. 
    
    \textit{הערה מועילה. }בתוך $\R^n$ הוקטורים הסטנדרטיים מאונכים אחד לשני (במכפלה הפנימית הסטנדרטית) ולכן $\mut{e_i}{e_j} = \delta_{ij}$ כאשר $\delta_{ij}$ הדלתא של כרוניקר. באינדוקציה על משפט פיתגורס נקבל ש־: 
    \[ v = \sum_{i = 1}^{n}\ag_i e^i \implies \norm{v} = \sum_{i = 1}^{n}\ag_i^2 \]
    שזה בדיוק מושג הגודל בגיאומטריה אוקלידית. 
    
   \textit{הערה. }מעל $\R$ מקבלים אמ''מ למשפט פיתגורש, מעל $\C$ לאו דווקא. מאונכים – בעברית. בלעז, \textit{אורתוגונלים}. ואכן וקטורים יקראו אורתוגונליים אם הם מאונכים אחד לשני. 
   
   זה מטוס? זה ציפור? לא, זה מתמטיקה B!
   \theo{(אי שוויון קושי־שוורץ)
   \[ \forall v, u \in V \co |\mut{u}{v}| \le \norm{u}\cdot\norm{v} \]
   ושוויון אמ''מ $u, v$ ת''ל. }
   \textit{הערה. }זה בפרט נכון בכיאומטריה סטנדרטית ממשפט הקוסינוסים. 
   
   \begin{proof}
       אם $v$ או $u$ הם $0$, אז מתקקבל שוויון. 
       \textit{טענת עזר: }קיים איזשהו $\ag \in \F$ כך ש־$u - \ag v \perp v$. נסמן $v_u = \ag v$ כאשר נמצא אותו. 
       \textit{הוכחת טענת העזר. }נחפש כזה: 
       \[ \mut{u - \ag v}{v} = 0 \impliedby \mut{v}{u} - \ag \norm{v}^2 = 0 \impliedby \ag = \frac{\mut{v}{u}}{\norm{v}^2} \]
       כדרוש. (מותר לחלק בנורמה כי הם לא $0$). ניעזר בפיתגורס: 
       \begin{align*}
           &\begin{cases}
               u - \ag v \perp v \\ u - \ag v \perp v
           \end{cases}\dequad \implies \norm{u}^2 = \norm{(u - \ag v + \ag v)}^2 = \underbrace{\norm{u - \ag v}^2}_{\ge 0} + |\ag|^2\norm{v}^2 \ge |\ag|\cdot\norm{v}^2 = \frac{|\mut{v}{u}|^2}{(\norm{v}^2)^2} = \norm{v}^2 = \frac{|\mut{u}{v}|^2}{\norm{v}^2} \\ &\implies |\mut{v}{u}|^2 \le \norm{v} \cdot \norm{u}
       \end{align*}
       בפרט $\norm{u - \ag v}^2 = 0$ אמ''מ הם תלויים לינארית ומכאן הכיוון השני של המשפט. 
   \end{proof}
   
   \textbf{דוגמאות. }ממכפלה פנימית סטנדרטית: 
   \begin{enumerate}
       \item \[ \sof{\sum_{i = 1}^{n}\ag_i b_i}^2 \le \cl{\sum_{i = 1}^{n}a_i^2} \cl{\sum_{i = 1}^{n}b_i} \]
       \item  נניח $f, g [0, 1] \to \R$ רציפות אז: 
       \[ \sof{\int^1_0 f(t)g(t) \dt}^2 \le \int_{0}^{1}f^2(t) \dt \cdot\int^1_0 g^2(t) \dt \]
       כאשר $f^2 = f \cdot f$ (לא הרכבה). 
       \item אי־שוויון המשולש: 
       \[ \forall u, v \in V \co \norm{u + v} \le \norm{u} + \norm{v} \]
       ושוויון אמ''מ אחד מהם הוא $0$ או אם הם כפולה חיובית אחד של השני (לא שקול לתלויים לינארית – יכולה להיות כפולה שלילית). 
   \end{enumerate}
   \begin{proof}
       (לאי שוויון המשולש). 
       \textit{תזכורת עבור $\zc \in \C$ מתקיים $\sof{\zc}^2 = (\Re \zc)^2 + (\Im \zc)^2$}
       \[ \norm{u + v}^2 = \norm{u}^2 + \norm{v}^2 + 2\Re(\mut{u}{v}) \le \norm{u}^2 + \norm{v}^2 + 2\sof{\mut{u}{v}}  \]
       ושוויון אמ''מ $u$ הוא אפס או כפולה חיובית של $v$. מקושי־שוורץ: 
       \[ \le \norm{u}^2 + 2\norm{u}\norm{v} + \norm{v}^2 = \cl{\norm{u} + \norm{v}}^2 \]
   \end{proof}
   
   \section{\en{Orthogonality}}
   \defi{יהי $(V, \smut)$ ממ''פ. יהיו $S, T \subseteq V$. נכתוב: 
   \begin{enumerate}[A.]
       \item \hfil $u \in V \co u \perp S \iff (\forall v \in S\co u \perp v)$
       \item \hfil $S \perp T \iff \forall v \in \S\, \forall u \in T \co v \perp u$ 
       \item \hfil $S^{\perp} := \{v \in V \mid v \perp S\}$
   \end{enumerate}}
    \lem{תהי $S \subseteq V$. אז: 
    \begin{enumerate}[A.]
        \item $v \perp S$ אמ''מ $v \perp \Sp(S)$
        \item $S^{\perp} \subseteq V$ תמ''ו
        \item אם $\subseteq T$ אז $T^\perp \subseteq S^\perp$
    \end{enumerate}}
    \begin{proof}[הוכחה (לג'). ]
        \[ \forall v \perp T \co c \perp S \implies v \in S^{\perp} \]
    \end{proof}
    הערה: שוויון בג' מתקיים אמ''מ $\Sp S = \Sp T$. 
    
    \defi{משפחה של וקטורים $A \subseteq V$ נקראת \textit{אורתוגונלית} אם $\forall u \neq v \in V\co u \perp v$}
    \textit{הערה. }אם $A$ משפחה אורתוגונלית וגם $0 \notin A$ אז ניתן לייצור ממנה משפחה של וקטורים אורתוגונלים שהם גם וקטורי יחידה, ע''י נרמול. \\
    
    \defi{משפחה של וקטורים $A \subseteq V$, אם היא אורתוגונלית ובנוסף כל הוקטורים הם וקטורי יחידה. }
    
    \defi{יהי $U \subseteq V$ תמ''ו. יהא $v \in V$. אז \textit{ההטלה האורתוגונלית} של $V$ על $U$ היא $p_U(v)$ הוא וקטור המקיים: 
    \begin{itemize}
        \item \hfil $p_U(v) \in U$ 
        \item \hfil $v - p_U(v) \in U^\perp$
    \end{itemize}}
    \theo{בסימונים לעיל, $\forall u \in U \co \norm{v - u} \ge \norm{v - p_U(v)}$ ושוויון אמ''מ $u = p_U(v)$. }
    
    \begin{proof}
        יהי $u \in U$. ידוע $p_U(v) \in U$. אזי $u - P_u(v) \in U$. כמו כן ידוע $p_U(v) - v \perp u$. אזי בפרט $\mut{u - p_U(v)}{p_U(v) - v}$. נתבונן ב־: 
        \[ \norm{u - v}^2 = \norm{(u - p_U(v)) + (p_U(v) - v)}^2 \overset{\text{פית'}}{=} \norm{u - p_U(v)}^2 + \norm{v - p_U(v)}^2 \]
        וסה''כ $\norm{v - u}^2 \ge \norm{v - p_U(v)}^2$. ושוויון אמ''מ $\norm{u- p_U(v)} = 0$ אמ''מ $u  = p_U(v)$. 
    \end{proof}
    \theo{ההטלה הניצבת (אם קיימת), היא יחידה. }\begin{proof}
        יהיו $p_U(v)$ וכן $p'_U(v)$ הטלות של $v$ על $U$. מהטענה: 
        \[ \norm{v - p_U(v)} \le \norm{v - p'_U(v)} \]
        אבל בהחלפת תפקידים מקבלים את אי־השוויון ההפוך. לכן יש שוויון נורמות. מהמשפט לעיל $p_U(v) = p'_U(v)$. 
    \end{proof}
    
    \theo{תהי $A \subseteq V$ משפחה אורתוגונלית ללא $0$. אז היא בת''ל. }\begin{proof}
        יהיו $v_1 \dots v_n \in A$ וכן $\ag_1 \dots \ag_n \in \F$, כך ש־$\sumni \ag_i v_i = 0$. יהי $i \in [n]$. אז: 
        \[ 0 = \mut{0}{v_j} = \mut{\sumni \ag_i v_i}{v_j} = \sumni \ag_i \mut{v_i}{v_j} = \ag _j \underbrace{\norm{v_j}^2}_{\neq 0} \implies \ag_j = 0 \]
        כאשר השוויון האחרון מהיות הקבוצה אורוגונלית. 
    \end{proof}
    \theo{נניח ש־$u \subseteq V$ תמ''ו. נניח $U$ נ''ס וכן $B = (e_1 \dots e_n)$ בסיס אורתונורמלי של $U$ (כלשהם, לא בהכרח סטנדרטיים כי גם לא בהכרח $\F^n$). אז 
    \[ \forall v \in V \co p_U(v) = \sum_{i = 1}^{m}\mut{v}{e_i}e_i \]}
    \begin{proof}
        צ.ל. $p_U(v) \in U$ וגם $\forall u \in U \co \mut{v - p_U(v)}{u} = 0$ אך לגבי התנאי האחרון די להוכיח $\forall j \in [n] \co \mut{v_i p_U(v)}{e_j}  = 0$. החלק הראשון ברור, נותר להוכיח: 
        \[ \mut{v - p_U(v)}{e_j} = \mut{v}{e_j} - \mut{p_u(v)}{e_j} =: * \]
        ידוע: 
        \[ \mut{p_U(v)}{e_j} = \mut{\sum_{i = 1}^{m}\mut{v_i}{e_i} e_i}{e_j} = \sum_{i = 1}^{m}\mut{v_i}{e_i} \cdot \mut{e_i}{e_j} = \sum_{i = 1}^{m}\mut{v}{e_i}\delta_{ij} = \mut{v}{e_j} \]
        נחזור לשוויון לעיל: 
        \[ *= \mut{v}{e_j} - \mut{v}{e_j} = 0 \]
        כדרוש. 
    \end{proof}
    (בכך הוכחנו את קיום $p_U(v)$ לכל מ''ו נ''ס, אם נשלב את זה עם המשפט הבא)
    
    \theo{(אלגרויתם גרהם־שמידט) תהי $(b_1 \dots b_k)$ קבוצה סדורה בת''ל של וקטורים בממ''ס $V$. אז בכל משפחה א''נ $(u_1 \dots u_k)$ כך ש־$\Sp(b_1 \dots b_k) = \Sp(u_1 \dots u_k)$. }
    \textbf{מסקנות מהמשפט. }לכל ממ''ס נ''ס קיים בסיס א''נ (=אורתונורמלי). יתרה מזאת, בהינתן בסיס $B = (b_1 \dots b_n)$ ניתן להופכו לבסיס א''נ $(u_1 \dots u_n)$ המקיים $\forall k \in [n] \co \Sp (b_1 \dots b_k) = \Sp(u_1 \dots u_k)$. 
    
    \begin{proof}
        בנייה באינדוקציה. נגדיר עבור $k = 1$ את $u_1 = b_1''$. מתקיים $\Sp u_1 = \Sp b_1$ וכן $\{u_1\}$ קבוצה א''נ. נניח שבנינו את $k$ האיברים הראשונים, נבנה את האיבר ה־$k + 1$ (כלומר את $u_{k + 1}$). במילים אחרות, הנחנו $u_1 \dots u_k$ אורתונורמלית וגם $\Sp (u_1 \dots u_k) = \Sp(b_1 \dots b_k) = U$. 
        
        מהסעיף הקודם $p_U(b_{k + 1})$ קיים, וגם $b_{k + 1} - p_U(b_{k + 1}) \neq 0$ מהבנייה. נגדיר $u_{k + 1} = (b_{k + 1} - p_U(b_{k + 1}))$. בצורה מפורשת: 
        \[ u_{k + 1} = \frac{b_{k + 1} - \sum_{i = 1}^{k}\mut{b_{k + 1}}{u_i}u_i}{\norm{b_{k + 1} - \sum_{i = 1}^{k}\mut{b_{k + 1}}{u_i}u_i}} \]
        מהגדרת $p_U(b_{k + 1})$, מתקיים $b_{k + 1} - p_U(b_{k + 1}) \in U^{\perp}$ ולכן גם $u_{k + 1} \in U^{\perp}$ ולכן $(u_1 \dots u_{k + 1}$ משפחה א''נ. 
        \[ b_1 \dots b_k = \overbrace{\Sp(u_1 \dots u_{k + 1})}^{\text{בת''ל}} \]
        נשאר להוכיח ש־$b_{k + 1} \in \Sp(u_1 \dots u_{k + 1})$. זה מספיק משום שאז נקבל $\Sp(b_1 \dots b_{k + 1}) \subseteq \Sp(u_1 \dots u_{k + 1})$. אבל הם שווי ממד ולכן שווים. סה''כ: 
        \[ b_{k + 1} = \norm{b_{k + 1} - p_U(b_{b + 1})} \cdot u_{k + 1} + \sum_{i = 1}^{k}\mut{b_{k + 1}}{u_i}u_i \implies b_{k + 1} \in \Sp(u_1 \dots u_{k + 1}) \]
        מש''ל. 
    \end{proof}
    
    \theo{יהי $V$ מ''ו $U \subseteq V$. נניח שלכל $v \in V$ מוגדר $p_U(v)$ (בפרט כל מ''ו נ''ס). אז $p_U \co V \to V$ המוגדרת לפי $v \mapsto p_U(v)$ העתקה לינארית. }\begin{proof}
        יהיו $v, v' \in V, \ag \in \F$. ידוע $v - p_U(v), v' - p_U(v') \in U^{\perp}$ ועל כן: 
        \[ (v - p_U(v)) + \ag(v' - p_u(v')) \in U^T \implies (v + \ag v') - (p_U(v) + \ag p_U(v')) \in U^T \implies (v + \ag v') - p_U(v + \ag v') \in U^{T} \]
        מה מקיים היטל וקטור? ראשית ההיטל ב־$U$, ושנית $v$ פחות ההיטל מאונך. הוכחנו שבהינתן היטל, הוא יחיד. והראינו ש־$(v + \ag v') - p_U(v + \ag v')$ מקיים את זה, ולכן אם יש וקטור אחד אז הוא יחיד, וסה''כ שווים וליניארית. 
    \end{proof}
    
    
    
    
    \ndoc
\end{document}