%! ~~~ Packages Setup ~~~
\documentclass[]{../../../tex/classes/styledArticle}
\usepackage{../../../tex/packages/hebrewSupport}
\usepackage{../../../tex/packages/mathShortcuts}
\usepackage{../../../tex/packages/theoremsSupport}


%! ~~~ Document ~~~

\author{שחר פרץ}
\title{\textit{חדו''א 1א}}
\date{26 לאוקטובר 2025}
\begin{document}
	\maketitle

	\textbf{שם מרצה: }ליאור קמה

	\textbf{אימייל: }liorkammma@tauex.tau.ac.il

	ניוטון פיתח לראשונה את החדו''א ככלי לנתח ככובי לכת וקליעים. תוך כדי כך לייבניץ פיתח את החדו''א. ההגדרות לא היו פורמליות בכלל. זה השתנה לאחר פרדוקס ראסל, ולאחרי שזרם הפורמליזם של הילברט בגטינגן השתלט על הכל.

	\section{מבוא}
	\subsection{שדות סדורים שלם}
	דיברנו על מערכת המספרים הממשיים בלינארית. נדבר על הקבוצה $\R$. הקבוצה היחידה שניתנת לנו מהמשיים היא $\N$ (מהאקסיומות של תקבצ). מהטבעיים בונים את הקבוצות האחרות, כמו השלמים והרציונליים. לבנות את הממשיים זה יותר בלגן, זה לא קשה, בעיקר לוקח זמן.

	אופציה אחרת, היא במקום לבנות את $\R$, ניגש בקבוצה האקסיומטית, כמו שראינו בתורת החוגים. נניח כל מני דברים על הקבוצה הזו, נקווה שהיא קיימת, ונוכיח כל מני טענות על גבי זה.

	אינטואיטיבית נחשוב על זה כעל כל מספר שיכול להתבטא באורך של קטע.

	יש לנו שתי פעולות, $+ \co \R \times \R \to \R$ ו־$\cdot \,\co \R \times \R \to \R$. עקרונית $+(3, 5)$ כיתוב חוקי, אבל כתיב פולני של $3 + 5$ מקובל מספיק. הקבוצה $\R$ היא חבורה בחיבור, חבורה בכפל, ודיסטרבוטיבית. כלומר לכל $x, y, z$ מתקיים:
	\begin{enumerate}
		\item קומטטיביות: \hfill $\forall x, y \in \R \co x + y = y + x$
		\item אסוציאטיביות: \hfill $\forall x, y, z \in \R\co x + (y + z) = (x + y) + z$
		\item קיום איבר 0 (יחידת חיבור): \hfill $\exists 0 \in \R\co \forall x \in \R x + 0 = x$
		\item קיום נגדי (הופכי לחיבור): \hfill $\forall x \in \R \co \exists y \in \R\co x + y = 0$
	\end{enumerate}
	כבר בעזרת ההנחות האלו אפשר לעשות דברים.
	\theo{לכל $x, y, z \in \R \co (x + y = z + y) \implies x = z$}\begin{proof}
		יהיו $x, y, z \in \R$. נניח $x+ y = z + y$. מ־4 קיים $t \in \R$ כך ש־$y + t = 0$. נרכיב את $+$ עם $t$ על שני האגפים ונקבל $(x + y) + t = (z + y) + t$ מח''ע הפונקציה. מ־2 נקבל $x + (y + t) = z + (y + t)$ כלומר $x + 0 = t + 0$ ולכן מ־3 $x = z$ כדרוש.
	\end{proof}
	\cola{לכל $x \in \R$ קיים $y \in \R$ \textit{יחיד} כך ש־$x+  y = 0$. }
	\noti{יהי $x \in \R$. את \textit{ה}מספר $y$ המקיים $x+ y = 0$ נכנה \textit{הנגדי} של $x$ ונסמן $-x$. }

	נמשיך עתה עם אקסיומות כפל.

	\begin{enumerate}
		\skipitems{4}
		\item קומוטטיביות: \hfill $\forall x, y \in \R\co x\cdot y = y \cdot x$
		\item אסוציאטיביות: \hfill $\forall x, y, z, \in \R\co (xy)z = x(yz)$
		\item קיום ניטרלי לחיבור (קיום יחידה בכפל): \hfill $x \cdot 1 = x$
		\item קיום הופכי בכפל: \hfill $\forall x \in \R\setminus\{0\}\co \exists y \in \R\co xy = 1$
	\end{enumerate}
	\theo{לכל $x, y, z \in \R$, אם $xy = zy \land y \neq 0$ אז $x = z$. }
	שימו לב לדרישה $y \neq 0$.

	\begin{proof}
	 תרגיל לבית
	\end{proof}
	\cola{לכל $x \in \R\setminus\{0\}$ קיים $y \in \R \setminus \{0\}$ יחיד, כך ש־$xy = 1$. }
	\noti{יהי $x \in \R\setminus \{0\}$, את המספר המקיים $y \neq 0 \land xy = 1$ נכנה \textit{ההופכי} של $x$ ונסמן $x\op$. }

	עתה נוסיף את התכונה האחרונה שנדרשה מאיתנו:
	\begin{enumerate}
		\skipitems{8}
		\item דיסטרבוטיביות: $\forall x, y, z \in \R\co x(y + z) = xy + xz$
	\end{enumerate}

	תשעת האקסיומות הללו מגדירות על $(\R, +, \cdot)$ מבנה הקרוי \textit{שדה}. הוא למעשה חוג עם הופכי בכפל, ומקיים כל מני תכונות נחמדות שראינו באלגברה לינארית 1א.

	\theo{לכל $x \in \R$ מתקיים $x \cdot 0 = 0$. }\begin{proof}
		יהי $x \in \R$. לפי 3 $0 + 0 = 0$ כלומר $x \cdot (0 + 0) = x \cdot 0$ מהיות כפל פונקציה ולכן חד־ערכי. לפי 9 $x \cdot 0 + x \cdot 0 = x \cdot 0$. לפי 3 $x \cdot 0 + x \cdot 0 = x \cdot 0 + 0$. מהטענה הראשונה שהוכחנו $x \cdot 0 = 0$.
	\end{proof}
	\theo{$\forall x \in \R\co (-1) \cdot x = -x$. }\begin{proof}
		יהי $x$. מטענה קודמת, $x \cdot 0 = 0$. מההגדרה, $1 + (-1) = 0$. לכן $x(1 + (-1)) = 0$. לפי 9, $x \cdot 1 + x \cdot (-1) = 0$. לפי 7 ו־5 $x + (-1)x = 0$. הוכחנו את יחידות הנגדי ולכן $(-1) \cdot x = -x$.
	\end{proof}

	עתה, נגדיר \textit{יחס סדר} (כמ ושעשינו בבדידה 1). קבוצה $R$ קרויה \textit{יחס} אם $R \subseteq A \times A$ עבור $A$ כלשהו. ואכן, טוענים $< \subseteq \R\times \R$. במקום לכתוב $(2, 3) \in <$ נכתוב $2> 3$.

	\begin{enumerate}
		\skipitems{9}
		\item אנטי־סימטריות חזקה: \hfill $\forall x, y \in \R \co x < y \implies x \not< y$
		\item טרנזטיביות: \hfill $\forall x, y, z \in \R \co (x < y \land y < z) \implies x < z$
		\item מליאות: \hfill $\forall x, y \in \R \co x < y \lor x = y \lor y < x$
		\item אדטיביות: \hfill $\forall x, y, z \in \R \co x < y \implies x + z < y+ z$
		\item ססקווי־כפליות: \hfill $\forall x, y, z \in \R\co (x < y \land 0 < z) \implies xz < yz$
	\end{enumerate}

	הקבוצה $(\R, +, \cdot, <)$ נקראת שדה סגור.
	\theo{יהיו $x, y \in \R$. אם $x < y$ אז $-y < -x$. }\begin{proof}
		נניח $x < y$. לפי 13 $x + (-y) < y + (-y)$, כלומר $x + (-y) < 0$. לפי 1, 13 מתקיים $-x + (x + (-y)) < -x + 0$. לפי $2, 3$ מתקיים $(-x + x) + (-y) < -x$ וסה''כ $0 + (-y) < -x$ ומ־$3$ נקבל $-y < -x$ כדרוש.
	\end{proof}

	\theo{לכל $x, y, z, w \in \R$, אם $x < y \land z < w$ אז $x + z < y +w$. }
	שימו לב שזה לא עובד בכפל, אלא אם מניחים שהכל חיובי (לבית).

	יש לציין שגם $(\Q, +, \cdot, <)$ הוא יחס סדר סדור.

	אז מה מיוחד ב־$\R$? תמתינו, אבל הרעיון הוא שהוא יותר ''רציף``. המהות של החשבון הדיפרנציאלי הוא הרצף הזה. את ה''נעילה`` הזו של האקסיומות כך שרק $\R$ יקימן (עד לכדי איזו') יתבצע ע''י הוספת אקסיומת השלמות.

	\subsection{קבוצות חסומות וחסמים}
	\defi{תהא $A \subset \R$. יהי $\ag \in \R$. נאמר ש־$\ag$ \textit{חסם מלעיל} של $A$ אם לכל $a \in A$ מתקיים $a \le \ag$. }
	\defi{תהא $A \subseteq \R$. יהי $\ag \in \R$. נאמר ש־$\ag$ \textit{חסם מלרע} של $A$ אם לכל $a \in A$ מתקיים $\ag \le a$. }
	\defi{$A$ תקרא \textit{חסומה מלעיל} כאשר קיים לה חסם מלעיל. }
	\defi{$A$ תקרא \textit{חסומה מלרע} אם קיים לה חסם מלרע. }
	\defi{$A$ תקרא \textit{חסומה} אם היא חסומה מלעיל ומלרע. }
	\defi{$\ag$ ייקרא \textit{חסם עליון} (סופרמום) כאשר:
	\begin{enumerate}
		\item $\ag$ חסם מלעיל, כלומר $\forall a \in A \co a \le \ag$
		\item החסימה הדוקה, כלומר $\forall \eg > 0\,\exists a \in A \co a > \ag - \epsi$
	\end{enumerate}}
	נבחין ש־2 \textit{לא} שקול ל''קיים $a\in A$ כך ש־$\ag = a$``. לדוגמה, $A = \{x \in \R\co x < 1\}$. מטרנזטיביות כל $\ag > 1$ הוא חסם עליון, אך רק אחד הוא סופרמום, על אף ש־$1 \notin A$. עם זאת, הכיוון השני עובד: אם $\ag \in A$ חסם עליון של $A$ (קוראים למספר כזה מקסימום), אז $\ag$ סופרמום. כלומר, מקסימום הוא סופרמום, אבל סופרמום לא בהכרח מקסימום.

	האינטואציה ל־2 – לא משנה כמה מעט נוריד (כמה ה־$\eg$ קטן), ברגע שנוריד משהו מ־$\ag$, נקבל משהו שהוא כבר לא חסם מלעיל. כלומר, החסם העליון הוא ''החסם המלעיל הקטן ביותר``. כמו שנראה בהמשך, האינטואציה הזו אולי עוזרת להבין את ההגדרה, אבל היא אינטואציה מטעה מאוד.

	\lem{1 חסם עליון של הקבוצה לעיל}\begin{proof}
		יהי $\ag \in A$. אז $a < 1$ ולכן $a \le 1$ ומכאן הוא חסם מלעיל. נותר להוכיח שהחסימה הדוקה. יהי $\eg > 0$. אז $0 < \frac{\vepsi}{2} < \eg$. לכן $1 > 1 - \frac{\eg}{2} > 1 - \eg$. לכן $1 - \frac{\eg}{2} \in A$, וגם $1 - \frac{\eg}{2} > 1 - \eg$. לכן $1$ חסם עליון.
	\end{proof}
	\theo{תהא $A \subseteq \R$. אם יש ל־$A$ חסם עליון, יש לה חסם עליון יחיד. }\begin{proof}
		נניח $\ag$ חסם עליון של $A$ וגם $\bg$ חסם עליון של $A$. נניח בשלילה $\ag < \bg$. נסמן $\eg = \bg - \ag$ ומההנחה $\epsi > 0$. נקבל קיום $a \in A$ כך ש־$a > \bg - (\bg - \ag)$ ולכן $a > \ag$, בסתירה לכך ש־$\ag$ חסם מלעיל של $A$ חסם מלעיל של $A$.
	\end{proof}

	\noti{תהר $A \subseteq \R$ קבוצה חסומה מלעיל. נסמן את החסם העליון של $A$ ב־$\sup A$. }

	לבית – תגדירו באופן דומה חסם תחתון.
	% TODO
	\noti{חסם תחתון (שהגדרתם בבית) יקרא \textit{אינפימום} ויסומן ב־$\inf A$. }

	עתה, נוכל להגדיר את האקסיומה ה־15 של הממשיים.
	\begin{enumerate}
		\skipitems{14}
		\item \textit{אקסיומת השלמות} (או \textit{אקסיומת החסם העליון}): לכל $A \subseteq \R$. אם $A \neq \varnothing$ וגם $A$ חסומה מלעיל, אז ל־$A$ קיים חסם עליון.
	\end{enumerate}
	\lem{לכל $x \in \Q$, $x^2 \neq 2$. }(כלומר, $\sqrt2$ מספר אי־רציונלי) \begin{proof}
		יהי $x \in \Q$. נניח בשלילה $x^2 = 2$. קיימים $m, n \in \Z$ כך ש־$n \neq 0$ וגם $x = \frac{m}{n}$. ללא הגבלת הכלליות, $m$ אי־זוגי או $n$ אי־זוגי (לבית: לסגור את הפינה הזו באינדוקציה). לכן $\cl{\frac{m}{n}}^2 = 2$, כלומר $\frac{m^2}{n^2} = 2$. מכאן $m^2 = 2n^2$. לכן $m^2$ זוגי ולכן $m$ זוגי (כי ריבוע לא משנה גורמים ראשוניים). סה''כ קיים $k$ כך ש־$m = 2k$ כלומר $4k^2 = 2n^2$ ומכאן $n^2 = 2k^2$ ואז $n$ זוגי וסתירה. לכן $x^2 \neq 2$.
	\end{proof}
	\lem{יהיו $x, y \in \R$, אם $x > 0 \land y > 0 \land x^2 < y^2$ אז $x < y$. }

	\theo{$(\Q, \cdot, +, <)$ אינה מקיימת את אקסיומת השלמות. }\begin{proof}
		נתבונן בקבוצה $A = \{\Q \ni x > 0 \co x^2 < 2\}$. נתבונן ב־1. $1 \in \Q$ וכמו כן $1 > 0$ וגם $1^2 < 2$ כלומר $1 \in A$ ו־$A \neq \varnothing$. נתבונן ב־2. נראה ש־2 חסם מלעיל. יהי $a \in A$. ידוע $a^2 < 2$. נפצל למקרים. מקרה $1$, נניח $a \ge 1$ ואז $a \le a^2 < 2$. מקרה 2, נניח $a<1$. אז $a<2$ וסיימנו. לכן $2$ חסם מלעיל של $A$ כלומר $A$ חסומה מלעיל.

		נותר להוכיח שאין ל־$A$ חסם עליון. יהי $\ag \in \Q$. נראה ש־$\ag$ לא חסם עליון. ידוע ממשפט קודם $\ag^2 \neq 2$. לכן, $\ag^2 < 2 \lor \ag^2 > 2$.
		\begin{itemize}
			\item אם $\ag^2 < 2$.
			[טיוטה: היינו רוצים לקחת ממוצע חשבוני, עם $\sqrt 2$. אבל $\sqrt 2$ לא מוגדר. כלומר היינו רוצים למצוא $\dg$ כך ש־$(\ag + \dg)^2 >2$. זה יוצא $(\ag + \dg)^2 = \ag^2 + 2\ag \dg + \dg^2 < 2$. מכאן $2\ag\dg + \dg^2 < 2 - \ag^2$. בגלל ש־$2 - \ag^2$ קבוע חיובי יש לנו תקווה שזה אפשרי. נקווה $\dg < 1$ ואז $2\ag\dg + \dg^2 \le 2\ag\dg + \dg < 2 - \ag^2$ ואז $\dg < \frac{2 - \ag^2}{2\ag + 1}$, וברגע שנדע שזה לא $0$ בהכרח קיים $\dg > 0$ מתאים. ניקח את המינימום בין זה לבין $1$ ונגמור עניין – סוף טיוטה].
			\begin{itemize}
				\item אם $\ag < 0$ אז $\ag$ אינו חסם עליון, אחרת נסמן $\dg = \frac{1}{2} \min \csb{1, \frac{2 - \ag^2}{2\ag + 1}}$. אז $\dg \in \Q$ ו־$\dg > 0$, שכן מההנחה $2 - \ag^2 > 0$ ולכן $\dg \neq 0$. לכן $\ag + \dg \in \Q$ וגם $\ag + \dg > \ag > 0$. כמו כן, $(\ag + \dg)^2 = \ag^2 + 2\ag\dg + \dg^2$. ידוע $\dg \le \frac{1}{2}$ דהיינו $\dg^2 < \dg$. לכן:
				\[ (\ag + \dg)^2 < \ag^2 + 2\ag\dg + \dg = \ag^2 + \dg(2\ag + 1) < \ag^2 + \frac{2 - \ag^2}{2\ag + 1}(2\ag + 1) = \ag^2 + 2 - \ag^2 = 2 \]
				לכן $\ag + \dg \in A$ כלומר $\ag$ אינו חסם מלעיל של $A$ ולכן איננו חסם עליון.
				\item בדומה למקרה הקודם, אם $\ag \le 0$ אז $\ag$ אינו חסם עליון של $A$. נניח $\ag > 0$. [טיוטה: הפעם נעשה הפוך, נרצה למצוא $\dg > 0$ כך ש־$(\ag - \dg)^2 = \ag^2 - 2\ag\dg + \dg^2 > 2$ ומכאן $2\ag\dg - \dg^2 < \ag^2 - 2$ צ.ל. חייבים להניח $\dg < \ag$, בלי קשר $2 \ag \dg - \dg^2 < 2 \ag \dg < \ag^2 - 2$ וסה''כ $\dg < \frac{\ag^2 - 2}{2\ag}$.  – סוף טיוטה]

				נפנה לאשכרה הוכחה. נבחר $\dg = \frac{1}{2}\min \csb{\ag, \frac{\ag^2 - 2}{2\ag}}$. נראה ש־$\ag - \dg$ גם חסם מלעיל. אז $\dg < \ag$ ולכן $\ag - \dg >0$. כמו כן:
				\[ (\ag - \dg)^2 = \ag ^2 - 2\ag \dg + \dg^2 > \ag^2 - 2 \ag \dg = \cdots \]
				ידוע $\dg < \frac{\ag^2 - 2}{2\ag}$ כלומר $2\ag\dg < \ag^2 - 2$ וגם $-2\ag\dg > 2 - \ag^2$.
				\[ \cdots > \ag^2 + (2 - \ag^2) = 2 \]
				נותר להראות ש־$\ag - \dg$ אשכרה חסם עליון. יהי $a \in A$. אז $a^2 < 2 < (\ag - \dg)^2$. מהיות $\ag - \dg > 0$ כי בחרנו את $\dg$ כך ש־$\ag > \dg$, ידוע $\ag < \ag - \dg$ (מהלמה השנייה שהוכחנו). לכן $\ag - \dg$ חסם מלעיל של $A$, ו־$\ag$ אינו חסם עליון של $A$.
			\end{itemize}
		\end{itemize}
	\end{proof}

	לסיכום – אקסיומת השלמות היא ההבדל המשמעותי בין $\R$ ל־$\Q$. לבינתיים, נניח ש־$\R$ שדה סדור מלא שמקיים את אקסיומת החסם העליון, ואפשר להראות קיום, ואף להראות שכל השדות המתאימים איזומורפים אחד לשני.

	\theo{לכל $x \in \R$, אם $x > 0$ אז קיים $y \in \R$ \textit{יחיד} כך ש־$y > 0$ וגם $y^2 = x$. }
	\begin{proof}
		לא נוכיח במדויק, נוכיח רק בערך. נגדיר את $A = \{A \co a^2 < x\}$. ממש כמו שהוכחנו קודם, אפשר להראות ש־$A$ חסומה מלעיל, וב־$\R$ יש לה חסם עליון. צ.ל. שריבוע החסם העליון הזה, הוא $x$.
	\end{proof}
	יש הכללה למשפט הזה:
	\theo{לכל $x \in \R$, ולכל $n \in \N_+$, אם $x > 0$ אז קיים $y \in \R$ יחיד כך ש־$y > 0$ וגם $y^n = x$. }
	ההכלה הזו יותר מסובכת, וצריך בשביל זה את הבינום של ניוטון. זה הרבה עבודה ידנית.
	\noti{נסמן את ה־$y$ היחיד שמקיים את המשפט לעיל ב־$\sqrt[n]{x}$. }
	כמה מילים לגבי חזקות. חזקות שלמות אפשר להגדיר רקורסיבית. חזקות רציונליות אפשר להגדיר בפחות או יותר באופן הבא:
	\[ a^{\frac{m}{n}} = \sqrt[n]{a}^{m} \]
	שקיים מהמשפטים שלנו. בשביל ההגדרה הזו, צריך להראות שזה לא תלוי בייצוג של הרציונלי – לא איכפת לנו בעבור אילו $n, m$ אנו מגדירים את זה, כלומר $\forall m, n, k, \ml \co \frac{m}{n}  = \frac{k}{\ml} \implies \sqrt[n]{a^{m}} = \sqrt[\ml]{a^{k}}$.


	\subsection{מסקנות על מספרים טבעיים בתוך הממשיים}
	בפעם שעברה דיברנו על אפיון אקסיומתי של $\R$, ובמיוחד אקסיומת השלמות שמייחדת את $\R$ באופן ספציפי. מה שניתן מהשמיים זה $\N$, השאר נבנים ידנית או אקסיומתית.

	באופן כללי, אקסיומות שמבטיחות קיום לא קונסטרוקטיבי לכל מיני דברים, כמו אקסיומת המקבילים, אקסיומת הבחירה, וגם אקסיומת השלמות – במקרים רבים ''לא באמת נדרשות``, וההנחה שלהן מאפשרת קיום מבנים ספציפיים.

	הנושא הבא הוא סדרות. לכן לפני כן נדבר על כמה תכונות של המספרים הממשים כת''ק בתוך $\R$.
	\begin{itemize}
		\item \textbf{הארכימדיאניות של הטבעיים בממשיים: }\hfill $\forall x, y \in \R \co x > 0 \implies (\exists n \in \N \co nx > y)$\\
		 למרות שזה נשמע אינטואיטיבי, צריך את אקסיומת השלמות בשביל זה.
		\begin{proof}
			נניח בשלילה כי לכל $n \in \N$ מתקיים $nx \le y$. נסמן $A := \{nx \co n \in \N\}$. מהנחת השלילה $y$ חסם מלעיל של $A$, בפרט $x \in A$ ולכן $A \neq \varnothing$. מאקסיומת השלמות קיים חסם עליון $\ag$ ל־$A$. [טיוטה: (\rn{1}) $\forall a \in A \co a \le \ag$ וגם (\rn{2}) לכל $\eg > 0$ קיים $a \in A$ כך ש־$a \ge \ag - \eg$, אין לנו יותר מדי משתנים לעבוד איתם, אז ננסה להתעסק עם $x$]. נתבונן ב־$x$. יהי $a \in A$, אז קיים $n \in \N$ כך ש־$a = nx$. נבחין ש־$(n + 1)x \in A$ ולכן $(n + 1)x \le \ag$ כלומר $nx \le \ag - x$ ועבור $\eg = x$ מצאנו $a - \eg$ שהוא חסם עליון שקטן מממש מהחסם מלעיל $\ag$ כלומר סתירה להיות $\ag$ חסם מלעיל. לכן $A$ אינה חסומה מלעיל, כלומר קיים $n \in \N$ עבורו $nx > y$.
		\end{proof}

		אז, למה צריך את אקסיומת השלמות למרות שזה מתקיים גם ברציונליים? כי ברציונליים הקיום קונסטקרטיבי, והם קשורים הדוקות לטבעיים. בניגוד לקבוצה סגורה מלא כללית.
		\item \textbf{הסדר הטוב של הטבעיים: }לכל $A \subseteq \N$ אם קיים $A \neq \varnothing$ אז קיים איבר מינימלי ב־$A$.

		\cola{לכל קבוצה $A \subseteq \Z$ אם $A \neq \varnothing$ וחסומה מלרע, אז קיים איבר מינימלי ב־$A$. }
		\cola{לכל קבוצה $A \subseteq \Z$ אם $A \neq \varnothing$ וחסומה מלעיל, אז קיים איבר מקסימלי ב־$A$. }
	\end{itemize}

	\theo{\hfil $\forall x \in \R \, \exists! k \in \Z \co k \le x < k + 1$. }
	\begin{proof}
		יהי $x \in \R$. נסמן $A = \{m \in \Z \co m > x\}$. ברור ש־$A \subseteq \Z$, נרצה להראות $A \neq \varnothing$. מארגימדיאניות קיים $n \in \N$ כך ש־$n > x$ ולכן $n \in A \implies A \neq \varnothing$. $A$ חסומה מלרע ע''י $x$. לכן קיים איבר מינימלי $t$ כלשהו ב־$A$. נסמן $k = t - 1$. נתבונן ב־$k$. ידוע $k < t$ ו־$k$ המינימום של $A$, כלומר $k \notin A$. מכאן $k \le x$. כמו כן $k + 1 = t \in A$ לכן $x < k + 1$. הראינו קיום, עכשיו יש להראות יחידות.

		יהי $\ml \in \Z$. נניח $\ml \neq k$, אז $\ml < k \lor k < \ml$.
		\begin{itemize}
			\item אם $\ml < k$ אז $\ml + 1 \le k$ ולכן $\ml + 1 \le x$. בפרט $x \not< \ml + 1$.
			\item אם $k < \ml$ אז $k + 1 \le \ml$ ולכן $x < \ml$ בפרט $\ml \not\le x$.
		\end{itemize}
		סה''כ כל $\ml \neq k$ לא מקיים את הדרוש ולכן $\ml$ יחיד.
	\end{proof}
	\noti{יהי $x \in \R$. אז \textit{ה}שלם \textit{ה}יחיד $k$ המקיים $k \le x < k + 1$ יסומן ב־$\floor{x}$ והוא יקרא \textit{ערך שלם תחתון}. }
	באותו האופן ניתן להגדיר ערך שלם עליון, $\ceil{x}$.

	\begin{Theorem}[צפיפות הממשיים]
		יהיו $x, y \in \R$. אם $x, y$ אז קיים $z \in \R$ כך ש־$x < z < y$.
	\end{Theorem}
	\begin{proof}
		נניח $x < y$. נתבונן ב־$\frac{x + y}{2}$. נסמן $z = \frac{x + y}{2}$ ומתקיים:
		\[ x = \frac{2x}{2} = \frac{x + x}{2} < \frac{x + y}{2} < \frac{y + y}{2} = \frac{2y}{2} = y \]
%		\envendproof
	\end{proof}
	\begin{Theorem}[צפיפות הרציונליים בממשיים]
		נניח $x < y$. אז $y - x > 0$ ולכן מהארגימדיאניות קיים $n \in \N$ כך ש־$n(y - x) > 1$. במקרה הזה $ny > nx + 1$ ולכן זה לא מפתיע שקיים טבעי באמצע, ואכן נוכל לסמן $m = \ceil{yn} - 1$ (שימו לב שבמקרה של $yn$ טבעי, זה לא הערך השלם התחתון). אז:
		\[ x < y - \frac{1}{n} = \frac{ny - 1}{n} \ge \frac{\floor{ny} - 1}{n} < \frac{ny + 1 - 1}{n} = y \]
		כמו כן $\frac{\floor{ny} - 1}{n} \in \Q$.

		%		\envendproof
	\end{Theorem}


	בתרגול נוכיח את נכונות המשפט עבור $z \in \Q \setminus \{0\}$.

	\section{סדרות}
	אחת ההגדרות האינטואטיביות לסדרה היא $n$־יה סדורה, אבל זו יכולה להיות רק סופית.

	לכן, נגדיר סדרה ממשית להיות פונקציה שתחומה $\N$ וטווחה $\R$. סדרות נסמן לרוב באותיות $a, b, c$ במקום $f, g, h$. במקום לסמן $a(n)$ בסימון פונקציות, נסמן $a_n$.

	\defi{\textit{סדרה ממשית} היא פונקציה $a(n) \co \N \to \R$}
	\defi{לעיתים רבות תבחינו שמסמנים סדרות באמצעות $(a_n)_{n = 1}^{\inft}$, או $\{a_n\}_{n = 1}^{\inf}$, או אפילו סתם $a_n$. }
	\defi{בהינתן סדרה, $a_n := a(n)$}

	\defi{נאמר ש־$a_n$ חסומה/חסומה מלעיל/חסומה מלרע כאשר הקבוצה $\ang$ חסומה/חסומה מלעיל/חסומה מלרע. }
	\defi{אם $a_n$ חסומה מלעיל, נסמן $\sup a_n := \sup_{n \in \N} a_n := \sup \ang$}
	\defi{אם $a_n$ חסומה מלרע, נסמן $\inf a_n := \inf_{n \in \N} a_n := \inf \ang$}
	\noti{ה\textit{סופרמום} הוא $\sup A$ והוא חסם עליון, וה\textit{אימפימום} $\inf A$ הוא החסם התחתון. }
	\defi{סדרה $a_n$ תקרא \textit{מונוטונית עולה} (או \textit{מונוטונית עולה חלש}) כאשר לכל $n, m \in \N$ מתקיים $n < m \implies a_n \le a_m$}
	\defi{סדרה $a_n$ תקרא \textit{מונוטונית עולה ממש} (או \textit{מונוטונית עולה חזק}) כאשר לכל $n, m \in \N$ מתקיים $n < m \implies a_n < a_m$}
	\defi{סדרה $a_n$ תקרא \textit{מונוטונית יורדת} (או \textit{מונוטונית יורדת חלש}) כאשר לכל $n, m \in \N$ מתקיים $n < m \implies a_n \ge a_m$}
	\defi{סדרה $a_n$ תקרא \textit{מונוטונית יורדת ממש} (או \textit{מונוטונית יורדת חזק}) כאשר לכל $n, m \in \N$ מתקיים $n < m \implies a_n > a_m$}
	\defi{סדרה תקרא \textit{מונוטונית} כאשר היא מונוטונית עולה או מונוטונית יורדת. }

	''אני לא מאמין שעשיתי את זה. מחקתי LIFO. היה לי מרצה שהגדי ללעשות והיה מוחק עם המרפק מה שהוא כתב הרגע``

	\subsection{גבולות של סדרות}
	\defi{תהא $\an$ סדרה. יהי $\ml \in \R$. נאמר כי $\ml$ הוא גבול של $\an$ כאשר \hfill $\forall \eg > 0.\, \exists N \in \N.\, \forall n \ge N \co \sof{a_n - \ml} < \eg$. }
	\lem{\hfil $\forall x \in \R .\, (\forall \eg > 0 \co \sof{x} < \eg) \implies x = 0$}
	\lem{מאי שוויון המשולש נקבל באופן מיידי:
	\[ \sof{x - y} \le \sof{x - z} + \sof{y - z} \]
	(זה גם ממש כמו המשפט בגיאומטריה לפיו אורך צלע קטנה מסכום האורכי הצלעות במשולש)}
	\theo{תהא $a_n$ סדרה. יהי $\ml \in \R$. אם $\ml$ גבול של $\an$ אז $\ml$ גבול \textit{יחיד} של $a_n$. }
	\begin{proof}
		נניח $\an$ מתכנסת ל־$\ml$. יהי $m \in \R$. נניח ש־$m$ גבול של $\an$. יהי $\eg > 0$. אז $\frac{\eg}{2} > 0$, ולכן קיים איזשהו $N_1 \in \N$ כך שלכל $n \ge N_1$, מתקיים $\sof{a_n - \ml} < \frac{\eg}{2}$. באופן דומה קיים $N_2 \in \N$ כך שלכל $n \ge N_2$, מתקיים $\sof{a_n - m} < \frac{\eg}{2}$. נסמן $N = \max{N_1, N_2}$. אז $N \ge N_1 \land N \ge N_2$, ומאי שוויון המשולש:
		\[ \sof{m - \ml} \le \sof{a_n - \ml} + \sof{a_n - m} < \frac{\eg}{2} + \frac{\eg}{2} = \eg \]
		לכן, לפי התרגיל, $m - \ml = 0$ כלומר $m = \ml$.
	\end{proof}
	\defi{נאמר כי סדרה $\an$ \textit{מתכנסת} כאשר קיים לה גבול $\ml \in \R$}
	\defi{אם $\an$ מתכנסת וגבולה (היחיד) הוא $\ml$, נסמן $\lim_{n \to \inft} a_n = \ml$. }
	''אבל בפיזיקה עשינו את זה עד עכשיו וזה עבד``

	\lem{קבוצה חסומה אמ''מ $\exists M > 0 \co \forall a \in A \co \sof{a} \le M$. }
	\theo{תהא $\an$ סדרה. אם $\an$ מתכנסת, אז $\an$ חסומה. }
	\begin{proof}
		מההנחה, קיים $\ml$ כך ש־$\lim_{n \to \inft} a_n = \ml$. מהגדרת הגבול קיים $N \in \N$ כך שלכל $n \ge N$ מתקיים $\sof{a_n - \ml} < 1$. נסמן $M = \max\{\sof{a_1}, \sof{a_2}, \dots, \sof{a_{N - 1}}, 1 + \sof{\ml}\}$. זו קבוצה סופית ולכן יש לה מקסימום. יהי $n \in \N$.
		\begin{itemize}
			\item \textbf{מקרה 1: }נניח $n < N$. אז $\sof{a_n} \le M$ פחות או יותר מהגדרת מקסימום.
			\item \textbf{מקרה 2: }נניח $n \ge N$. אז $\sof{a_n - \ml} < 1$ ולכן $-1 < a_n - \ml < 1$. נקבל $-\sof{\ml} - 1 < a_n < \ml + 1 \le \sof{\ml} + 1$ וסה''כ נקבל $\sof{a_n} \le \sof{\ml} + 1 \le M$.
		\end{itemize}
		סה''כ $\forall n \in \N \co \sof{a_n} \le M$ ולכן $a_n$ חסומה.
	\end{proof}

	\textbf{תרגיל: }הראו כי:
	\begin{itemize}
		\item \hfil $\displaystyle\limsi \frac{1}{n} = 0$
		\begin{proof}
			צ.ל. שלכל $\eg > 0$ ניתן למצוא $N \in \N$ כך ש־$\forall n \ge N \co \sof{\frac{1}{n} - 0} < \eg$. אז יהי $\eg > 0$. נבחר $N = \ceil{\frac{1}{\epsi}} + 1$. יהי $n \ge N$. אז:
			\[ \sof{\frac{1}{n}  - 0} = \frac{1}{n} \le \frac{1}{N} = \frac{1}{\ceil{\frac{1}{\eg}} + 1} < \frac{1}{\eg\op} = \eg \]
		\end{proof}
		\item נגדיר $\an = (-1)^{n}$. נוכיח ש־$\an$ איננה מתכנסת. \begin{proof}
			יהי $\ml \in \R$ כלשהו. נתבונן ב־$\eg = 1$. יהי $N \in \N$. נפרק למקרים על $\ml$.
			\begin{itemize}
				\item אם $\ml \ge 0$, נתבונן ב־$n = 2N + 1$. אז $n \ge N$ וגם $\sof{a_n - \ml} = \sof{(-1)^{2N + 1} - \ml} = \sof{-1 - \ml}= \ml + 1 \ge 1$
				\item אם $\ml < 0$, נתבונן ב־$n = 2N$. אז $n \ge N$ וגם $\sof{a_n - \ml} = \sof{(-1)^{2N} - \ml} = \sof{1 - \ml} =1 - \ml \ge 1$.
			\end{itemize}
			לכן $\an$ אינה מתכנסת ל־$\ml$ ולכן אינה מתכנסת.
		\end{proof}
	\end{itemize}

	מתבלבלים עם שלילה של הגדרת הגבול? נוכל להשתמש בחוקי השלילה של כמתים:
	\[ \lnot(\forall \eg > 0.\, \exists N \in \N,\ \forall n \ge N \co \sof{a_n - \ml} < \eg) \iff (\exists \eg > 0.\,\forall N \in \N.\,\exists n \in \N \co \sof{a_n - \ml} \ge \eg) \]

	''אין לי שום דבר נגד הוכחות בשלילה. אני תמיד נמנע מהן``. ''למה את 1?`` -- ''כי למה לא`` -- ''כי למה לא 1 זה נכון``. ''וזו ההתייחסות הנכונה להוכחות. אנחנו כותבים שירה``. ''לאחד חלקי איש יש $\frac{1}{10}$ אצבעות``.

	\theo{תהא $\an$ סדרה. יהי $\ml \in \R$. נניח כי $\ml \neq 0$ וגם $\limasi = \ml$. אז קיים $N \in \N$ כך שלכל $n \ge N$ מתקיים $\sof{a_n} \ge \frac{\sof{\ml}}{2}$. }
	במילים אחרות – $a_n$ הוא bounded away from zero. באופן כללי אפשר גם להוכיח את זה עם $\frac{\sof{\ml}}{\pi}$ או כל מספר אחר במכנה. אבל הרעיון העקרי הוא, ש־$\an$ לא יכול להתקרב ל־$0$ החל מנקודה כלשהי, אם הסדרה שואפת לנקודה שאיננה אפס.

	\begin{proof}
		ידוע $\ml \neq 0$ ולכן $\sof{\ml} > 0$. דהיינו $\frac{\sof{\ml}}{2} > 0$. אז עבור $\eg = \frac{\sof{\ml}}{2}$ קיים $N \in \N$ כך שלכל $n \ge N$ מתקיים $\sof{a_n - \ml} < \frac{\sof{\ml}}{2}$. נתבונן ב־$n$. יהי $n \ge N$, אז:
		\[ \sof{a_n - \ml} < \frac{\sof{\ml}}{2} \implies -\frac{\sof{\ml}}{2} < a_n - \ml < \frac{\sof{\ml}}{2} \]
		אפשר גם להשתמש בא''ש המשולש, אבל זה פחות אינטואיטיבי. נפרק למקרים.
		\begin{itemize}
			\item נניח $\ml > 0$. אז $a_n > \ml - \frac{\sof{\ml}}{2} = \frac{\sof{\ml}}{2}$. לכן $\sof{a_n} \ge \frac{\sof{\ml}}{2}$ וסיימנו.
			\item נניח $\ml < 0$. אז $a_n < \ml + \frac{\sof{\ml}}{2} = - \frac{\sof{\ml}}{2} < 0$. ולכן $\sof{a_n} > \frac{\sof{\ml}}{2}$
		\end{itemize}
	\end{proof}

איך מוכיחים זאת עם א''ש המשולש? באמצעות הטריק הבא:
	\[ \sof \ml - \sof{a_n - \ml} \overset{(1)}{=} \sof{\sof \ml - \sof{a_n - \ml}} \overset{(2)}{\le} \sof{\ml - (a_n - \ml)} = \sof{a_n} < \frac{\sof \ml}{2} \]
	כאשר $(1)$ נכון כי החל מנקודה כלשהי $a_n - \ml < \ml$ (עבור $\epsi = \ml$) ו־$(2)$ נכון מא''ש המשולש ההפוך.

	''אל תגידו א''ש המשולש. תגידי לי פד''ח ואני מנשל אותך מהירושה. אנחנו לא אומרים את זה יותר בחדר הזה`` $\sim$ המרצה.

	\subsection{אריתמטיקה של גבולות}
	זה הקטע שבו אנחנו רואים שגבול הוא לינארי.
	\theo{תהאנה $\an, \bc$ סדרות. יהיו $\ml, m \in \R$ ממשיים. נניח כי $\limasi = \ml \land \limbsi = m$. אז:
	\begin{enumerate}
		\item \hfil $\displaystyle \limsi (a_n + b_n) = \ml + m$
		\item \hfil $\displaystyle \forall \ag \in \R \co \limsi (\ag a_n) = a \ml$
		\item \hfil $\displaystyle \limsi a_n b_n = \ml \cdot m$
		\item \hfil $\displaystyle m \neq 0 \implies (\exists N \in \N.\, \forall n \ge N\co b_n \neq 0) \land \cl{\limsi \frac{a_n}{b_n} = \frac{\ml}{m}}$
	\end{enumerate}}

	\rmark{כדי להגדיר את הגבול $\limsi \frac{a_n}{b_n}$, דבר ראשון הראינו שמנקודה מסויימת $N$ מתקיים $b_n \neq 0$. אבל מה קורה לפני $N$? זה לא כזה משנה, נוכל לצורך הנקודה לקבע את הסדרה:
	\[ \frac{a_n}{b_n} := \begin{cases}
		0 & n < N \\
		\frac{a_n}{b_n} & n \ge N
	\end{cases} \]
	בכל מקרה חדו''א מתעסקת במה שקורא החל מנקודה מסויימת, ולא איכפת לנו מה קורה ב־$N$ האיברים הסופיים הראשונים. }

	\begin{enumerate}
		\item \begin{proof}[הוכחה שלי]
			נוכיח אדטיביות. יהיו $\an, \bn$ סדרות עם גבול $\limasi = \ml, \limbsi = m$. נראה ש־$\limsi a_n + b_n = \ml + m$. יהי $\eg > 0$. מהגדרת הגבול ידוע שקיימים $N_1, N_2$ טבעיים שהחל מהם $\forall n \ge N_1 \co a_n - \ml < \frac{\epsi}{2}$ וכן $\forall n \ge N_2 \co b_n - m < \frac{\epsi}{2}$. בפרט עבור $N = \max\{N_1, N_2\}$ מתקיים:
			\[ \forall n \ge N \co (a_n + b_n) - (\ml + m) = \underbrace{(a_n - \ml)}_{< \frac{\eg}{2}} + \underbrace{(b_n - m)}_{< \frac{\eg}{2}} < 2 \cdot \frac{\eg}{2} = \eg \]
			סה''כ מצאנו $N$ שהחל ממנו $(a_n + b_n) - (\ml + m) < \epsi$, ומהגדרת הגבול $\limsi a_n + b_n = \ml + m$ כדרוש.
		\end{proof}
		\item \begin{proof}[הוכחה שלי]
			תהי $\an$ סדרה עם גבול $\limasi = \ml$. נוכיח $\limsi \ag a_n = \ag \ml$. יהי $\eg > 0$. מהגדרת הגבול ומהנתון, קיים $N\in \N$ כך ש־$\forall n \ge N \co a_n - \ml < \frac{\eg}{\ag}$.
			\[ \ag a_n - \ag\ml = \ag(\underbrace{a_n - \ml}_{< \epsi}) < \ag \cdot \frac{\eg}{\ag} = \eg \]
			סה''כ מהגדרת הגבול $\limasi = \ag \ml$ כדרוש.
		\end{proof}
		\item \begin{proof}\,\![טיוטה: $\sof{a_n b_n - \ml m} = \sof{a_n b_n - a_n m + a_n m - \ml m} \le \sof{a_n b_n - a_n m} + \sof{a_n m - \ml m} = \sof{a_n}\sof{b_n - m} + \sof{m}\sof{a_n - \ml}$ ואז נקבל $\sof{a_n - \ml} < \frac{\eg}{2(\sof{m} + 1)}$, ולגבול השני נבחר חסם בהתאם לגבול]

			יהי $\eg > 0$. אז $\an$ מתכנסת ולכן חסומה, כלומר קיים $k > 0$ כך שלכל $n \in \N$, מתקיים $\sof{a_n} \le k$.

			$\an$ מתכנסת ל־$\ml$ ולכן עבור $0 < \frac{\eg}{2(\sof{m} + 1)}$ קיים $N_1 \in \N$ כל שלכל $n \ge N_1$, $\sof{a_n - \ml} < \frac{\eg}{2(\sof{m} + 1)}$

			נבחין ש־$\bn$  מתכנסת ל־$m$ לכן עבור $\frac{\eg}{2k} > 0$ קיים $N_2 \in \N$ כך שלכל $n \ge N_2$, $\sof{b_n - m} < \frac{\eg}{2k}$.

			עתה נתבונן ב־$N = \max\{N_1, N_2\}$. יהי $n \ge N$. אז:
			\[ \sof{a_n b_n - \ml m} = \sof{a_n b_n - a_n m + a_n m - \ml m}  \le \sof{a_n b_n - a_n m} + \sof{a_n m - \ml m} = \sof{a_n}\sof{b_n - m} + \sof{m} \sof{a_n - \ml} \]
			כיוון ש־$n \ge N_1$, $\sof{a_n - \ml} < \frac{\eg}{2(\sof{m} + 1)}$, ולכן $\sof{m}\sof{a_n - \ml} < \frac{\eg}{2}$. כיוון ש־$n \ge N_2$, $\sof{b_n - m} < \frac{\eg}{2k}$ ולכן $\sof{a_n}\sof{b_n - m} \le k\sof{b_n - m} < \frac{\eg}{2}$. מכאן $\sof{a_n b_n - \ml m} < \eg$ ולכן $\limsi a_n b_n = \ml m$.
		\end{proof}
		\item \begin{proof}[הוכחה שלי]
			יהיו $\an, \bn$ סדרות. נניח $\limasi = \ml, \limbsi = m$, וכן $m \neq 0$. נוכיח שהחל מאיזושהי נקודה $N_0$ מתקיים $\forall n \ge N \co b_n \neq 0$, וגם ש־$\limsi \frac{a_n}{b_n} = \frac{\ml}{m}$.

			ראשית כל, נוכיח שקיים $N_0$ שממנו $\forall n \ge N_0 \co b_n \neq 0$. נניח בשלילה שלא כך, ונוכיח שבעבור $\eg = \frac{\sof m}{2}$ מתקיים שלכל $N \in \N$ קיים $n \in \N$ כך ש־$\sof{b_n - m}  \not < \eg$. למעשה, נוכל להראות זאת כמעט במיידי: מהנחת השלילה, קיים $n \in \N$ כך ש־$b_n = 0$, ושם אכן:
			\[ \sof{b_n - m} = \sof{0 - m} = \sof m > \eg = \frac{\sof m}{2} \quad \bot \]
			וסתירה להגדרת הגבול ולכך ש־$\limbsi = m$.

			עתה, נוכיח $\limsi \frac{1}{b_n} = \frac{1}{m}$. יהי $\eg > 0$. נבחין שהסדרה $\frac{1}{b_n}$ מוגדרת רק לאחר ה־$N_0$ שהוכחנו את קיומו קודם לכן, ולכן נקבע את $b_{n < N_0} = 0$ (אסימפטוטית זה לא משנה בכל מקרה). בגלל ש־$\limbsi = m$, בהכרח החל מנקודה $N_1$ כלשהי מתקיים ממשפט שהראינו ש־$b_n > \frac{m}{2}$. נוסף על כך, החל מ־$N_2$ כלשהי $\sof{b_n - m} < \frac{2\eg}{m^2}$. בפרט, עבור $N = \max\{N_0, N_1, N_2\}$:

			אכן מתקיים לכל $n \ge N$: (נבחין שהביטוי מוגדר לכל $n \ge N_0$ ובפרט לכל $n \ge N$):
			\[ \sof{\frac{1}{b_n} - \frac{1}{m}} = \frac{\sof{b_n - m}}{\sof{b_nm}} \overset{n \ge N_1}{<} \frac{\sof{b_n - m}}{0.5m^2} \overset{n \ge N_2}{<} \frac{2\eg \cdot \frac{1}{m^2}}{0.5m^2} = \eg \]
			כדרוש. עתה, מ־3, שהוכח ללא תלות בסעיף זה, נקבל ישירות ש־$\limsi \frac{a_n}{b_n} = \frac{\ml}{m}$, כנדרש, וסיימנו.
		\end{proof}
	\end{enumerate}

	\defi{תהא $\an$ סדרה. נאמר כי $\an$ שואפת ל־$+\inft$ כאשר: \hfill $\forall M > 0.\, \exists N \in \N.\, \forall n \i \ge N \co a_n > M$}
	\defi{תהא $\an$ סדרה. נאמר כי $\an$ שואפת ל־$-\inft$ כאשר: \hfill $\forall M > 0.\, \exists N \in \N.\, \forall n \ge N \co a_n < -M$}

	\theo{תהיינה $\an, \bn$ סדרות. נניח $\limasi = +\inft \land \limbsi b_n = + \inft$. אז $\limsi a_n + b_n = + \inft$}
	\begin{proof}
		יהי $M > 0$. קיים $N_1, N_2 \in \N$ כך ש־$\forall n \ge N_1 \co a_n > M$ וכן $\forall n \ge N_2\co b_n > M$. נתבונן ב־$N = \max\{N_1, N_2\}$. אז $a_n + b_n > M + M = 2M > M$ וסיימנו.
	\end{proof}
	לבית: תעשו אותו הדבר עם כפל. לגבי חיסור וחילוק, אין תוצאה מוגדרת.
	in{document}
	\maketitle
	תהאנה $\an, \bn$ סדרות. הוכיחו או הפריכו:
	\begin{enumerate}[A.]
		\item אם $\an$ אינה מתכנסת, וגם $\bn$ אינה מתנכסת, אז $\an + \bn$ אינה מתכנסת.

		\textbf{תשובה: }לא נכון. אפשר להראות שזה לא עובד, לדוגמה עבור $a_n = (-1)^{n}$ ו־$b_n = (-1)^{n + 1}$, הראינו ש־$\an$ אינה מתנכסת ובאופן דומה $\bn$ אינה מתכנסת. אבל, $\forall n \in \N \co a_n + b_n = 0$ ולכן $\limsi a_n + b_n = 0$

		\item אם $\an$ מתכנסת וגם $\bn$ אינה מתכנסת, אז $(a_n + b_n)_{i = 1}^{\inft}$ אינה מתכנסת. \textbf{תשובה: }

		\begin{proof}
			נניח ש־$\an$ מתכנסת וגם $\bn$ אינה מתכנסת. נניח בשלילה ש־$a_n + b_n$ מתכנסת. מכאן שיש גבול $\ml$ לסדרה, ומאריתמטיקה של גבול $\limsi a_n + b_n - \limsi a_n = \limsi (\cancel{a_n - a_n} + b_n) = \limsi b_n$ אך $\limsi b_n$ אינו מוגדר שכן $\bn$ לא מתכנסת.
		\end{proof}
		\item אם $\an$ מתכנסת וגם $\bn$ אינה מתכנסת, אז $a_n \cdot b_n$ מתכנסת.

		\textbf{תשובה: }בניגוד להוכחה הקודמת, צריך בשביל להוכיח לטעון ש־$\limsi a_n \neq 0$ כדי שנוכל לחלק. לא מפתיע אם כן שעבור $b_n = (-1)^{n}, a = 0$ נקבל סתירה שכן $\bn$ לא מתכנסת, אך $a_n \cdot b_n = 0$ הסדרה הקבוע.

		\item לבית: כמו הסעיף הקודם אבל $\limsi b_n \neq 0$.
	\end{enumerate}

	\theo{תהא $\an$ סדרה, יהי $\ml \in \R$. אם $\limasi = \ml$ אז $\limsi \sof{a_n} = \sof \ml$. }\begin{proof}
		נניח $\limasi = \ml$. יהי $\eg > 0$. אז קיים $N \in \N$ כך ש־$\forall n \le N \co \sof {a_n - \ml} < \eg$. נתבונן ב־$N$. יהי $n \ge N$. מא''ש המשולש ההפוך:
		\[ \sof{\sof {a_n} - \sof \ml} \le \sof{a_n - \ml} < \eg \]
		מההגדרה $\limsi \sof{a_n} = \sof \ml$ כדרוש.
	\end{proof}
	\rmark{הבעייתיות בכיוון השני היא אם $\an$ מחליפה סימן אינסוף פעמים. הוא נכון אם $\an$ שומרת סימן מגבול מסויים (מה ששקול לכך שיש לה גבול, ממשפט נחמד שהראינו בעבר). }

	\begin{Theorem}
		תהאנה $\an, \bn, \cn$ סדרות. נניח כי־:
		\begin{enumerate}
			\item \hfil $\exists N \in \N .\,\forall n \ge N \co a_n \le c_n \le b_n$
			\item \hfil $\displaystyle \limasi = \ml  = \limbsi$
		\end{enumerate}
		אז $\limcsi = \ml$
	\end{Theorem}
	\begin{proof}
		יהי $\eg > 0$. קיים $N_2 \in \N$ כך ש־$\forall n \ge N_2 \co \sof{b_n - \ml} < \eg$, ובאופן דומה קיים $N_3 \in \N$ כך ש־$\forall n \in N_3\co \sof{a_n - \ml} < \eg$. נסמן $N = N_1$ (מהנתון). נתבונן ב־$N = \max\{N_1, N_2, N_3\}$. יהי $n \ge N$. אז נסיק $\ml - \eg < b_n < \ml + \eg$ וכן $\ml - \eg < a_n < \ml + \eg$ ואז:
		\[ \ml - \eg < c_N \le c_n \le b_n < \ml + \eg \]
		כלומר $\sof{c_n - \ml} < \eg$ כדרוש.
	\end{proof}

	\subsection{גבולות ושוויונות}
	\theo{תהנא $\an, \bn$ סדרות. יהיו $\ml, m \in \R$. נניח כי:
	\begin{enumerate}[(1)]
		\item לכל $n \in \N$, מתקיים $\an < \bn$. (\textit{הערה: }מספיק גם אם החל מ־$N$ כלשהו התנאי הזה מתקיים)
		\item מתקיים $\limasi = \ml$
		\item מתקיים $\limbsi = m$
	\end{enumerate}
אז $\ml \le m$}
	\begin{proof}
		נניח בשלילה $m < \ml$. נסמן $\eg = \frac{\ml - m}{2}$, בהכרח $\eg > 0$. לכן $\exists N_1 \in \N.\, \forall n \ge N_1 \co \sof{a_n - \ml} < \eg$. כמו כן $\exists N_2 \in \N.\, \forall n \ge N_2 \co \sof{b_n - m} < \eg$. נתבונן ב־$N = \max\{N_1, N_2\}$. שם, מתקיים:
		\[ b_N < m + \eg = \frac{\ml + m}{2} = \ml - \eg < a_N \]
		בסתירה ל־$(1)$. וסיימנו.
	\end{proof}
	למה היינו צריכים להניח בשלילה? כי עקרונית נרצה לקחת את $\frac{\sof{\ml - m}}{2}$ ולעבוד עם זה, ולהפעיל על זה את הגדרת הגבול, אבל זה יכול להיות $0$. לכן נרצה להניח בשלילה ש־$\ml > m$, כי כאן יש א''ש חזק ממש.

	לועמת זאת, אם נניח ב־$(1)$ במקום זאת $\forall n \in N \co a_n < b_n$, עדיין נוכל לדעת $\ml \le m$ בלבד, למרות שהא''ש לכאורה חזק. לדוגמה, בעבור $a_n = \frac{1}{n}$ ו־$b_n = 0$ מתקיים שוויון חלש ולא חזק בגבול, על אף ש־$b_n < a_n$.


	\theo{תהאנה $\an, \bn$ סדרות, ויהיו $\ml, m \in \R$. נניח ש־$\limasi = \ml, \limbsi = m$. נניח גם $\ml < m$. נוכיח שהחל ממקום מסויים $\exists N \in \N.\, \forall n \ge N \co a_n < b_n$. }
	הוכחה לבית.

	\begin{Theorem}[משפט ויירשטראס הראשון]
		תהא $\an$ סדרה. אם $\an$ מונוטונית וחסומה, אז $\an$ מתכנסת.
	\end{Theorem}
	\begin{proof}
		בה''כ נניח ש־$\an$ מונוטונית עולה (אחרת ההוכחה בדומה). ידוע ש־$\an$ חסומה, ובהכרח מלמעלה, ולכן לפי אקסיומת השלמות חיים לה חסם עליון, נסמנו $\ml = \sup a_n$. יהי $\eg > 0$. אז קיים $N \in \N$, כך ש־$a_n > \ml - \eg$. נתבונן ב־$N$. יהי $n \ge N$. אז:
		\[ \ml - \eg < a_N \le a_n \le \ml < \ml + \eg \]
		כלומר $\sof{a_n - \ml} < \eg$. לכן $\an$ שואפת ל־$\ml$, כדרוש.
	\end{proof}
	\defi{סדרה $\an$ תקרא \textit{בעלת גבול במובן הרחב} אם $(\exists \ml \in \R\co \limasi = \ml) \lor \limasi = \pm \infty$. }
	\theo{בהינתן סדרה מונוטונית לא חסומה, היא שואפת ל־$\pm \inft$. }
	\cola{תהי $\an$ מונוטונית. אז ל־$\an$ יש גבול במובן הרחב. }

	\subsection[$e$]{\hfil $e$}
	\theo{נגדיר $a_n = \cl{a + \frac{1}{n}}^{n}$ לכל $n \in \N$, ו־$b_n = \sumnk \frac{1}{k!}$ לכל $n \in \N$. אז:
	\begin{enumerate}
		\item $\an$ חסומה, מונוטונית עולה וחסומה ב־$3$.
		\item $\bn$ חסומה, ומונוטונית עולה.
		\item $\forall n \in \N \co a_n \le b_n$
		\item $\forall n \in \N.\, \exists k > n \co b_n \le a_{n + k}$
	\end{enumerate}}
	המסקנה מ־1, 2 הוא שיש להן גבול (ממשפט וויראשטראס). ממשפט אחר שהראינו, 3 ו־4 גוררים ש־$\an, \bn$ מתכנסות לאותו הגבול. נסמנו ב־$e$.
	\defi{נסמן:
		\[e := \limsi \cl{1 + \frac{1}{n}}^{n} \dequad\ = \limsi \sumnk \frac{1}{k!}\]}

	\section*{תת־סדרות וגבולות חלקיים}
	\defi{תהי פונקציה $n_k \co \N \to \N$ סדרה עולה ממש של טבעיים, ותהא $\an$ סדרה. אז הסדרה $a_{(n_k)}$ נקראת \textit{תת־סדרה של} $\an$. פורמלית, זוהי הרכבה $a_n \circ n_k$. }
	\defi{$\ml$ יקרא \textit{גבול חלקי} של $\ml$ כאשר קיימת ת''ס של $\an$ המתכנסת ל־$\ml$. }
	\defi{$\pm\infty$ יקרא גבול חלקי של $\an$, כאשר קיימת ת''ס השואפת ל־$\pm\infty$. }

	לדוגמה, עבור $a_n = (-1)^{n}$, ו־$a_{2k}$ ת''ס של $\an$, אז $\limsi a_{2k} = 1$. לכן $1$ גבול חלקי של $\an$. באופן דומה $-1$ גבול חלקי של $\an$ ואפשר גם להוכיח יחידות.

	\rmark{לעיתים, לגבולות חלקיים קוראים \textit{נקודות גבול}. }

	להלן משפט שקצת יוצא מתחומי החדו''א.
	\begin{Theorem}[משפט הרקורסיה]
		תהא $f \in \N \times \R \to \R$. יהי איזשהו $a \in \R$. אז קיימת סדרה יחידה $\an$ המקיימת:
		\[ \begin{cases}
			a_0 = a \\
			\forall n \in \N \co a_{n + 1} = f(n, a_{n})
		\end{cases} \]
	\end{Theorem}
	למה אנחנו צריכים את המשפט הזה? כי אם כותבים משהו כמו $a_0 = 2, \, a_{n + 1} = 2^{n}a_n + 1$ (בהקשר הזה $f(x, y) = 2^{x} y + 1$), למה שבכלל תהיה $\an$ שתקיים את תנאי הנסיגה הזה? המשפט הזה דואג לכך שנוסחאות נסיגה יהיו מוגדרות היטב (קיימות ויחידות בהינתן כלל נסיגה עם תנאי בסיס). אפשר להכליל באינדוקציה לפונקציות נסיגה מדרגה $k$־ית.

	השבוע יעלה למודל תרגיל מודרך העוסק בזה.

	\begin{Theorem}[משפט בוצלנו־וייראסטראס]
		לכל סדרה חסומה, יש ת''ס מתכנסת.
	\end{Theorem}
	\lem{תהא $\an$ סדרה. נניח של־$\an$ אין איבר מסקימלי. אז יש לה תת סדרה מונוטונית עולה ממש. }
	\begin{proof}
		יהי $n \in \N$. נסמן $A_n = \{m \in \N \co m> n \land a_m > a_n\}$. מהיות $\an$ ללא איבר מקסימלי, לכן קיים $m \in \N$ כך ש־$a_m > \max\{a_1 \dots a_n\}$. בפרט $a_m > a_n$ ולכן $m \in A$. מכאן שבהכרח $A_n$ לא ריקה.

		נגדיר $n_k \co \N\to \N$ ברקורסיה:
		\[ \begin{cases}
			n_1 = 1 \\
			n_{k + 1} = \min A_{(n_k)}
		\end{cases} \]
		המינימום בהכרח מוגדר היטב מהיות $A_{(n_k)}$ לא ריקה. ממשפט הרקורסיה $(n_k)_{i = 1}^{\infty}$ מוגדרת היטב. כדי להראות שהיא ת''ס, יש להראות שהיא מונוטונית עולה חזק. ואכן מהגרדה $n_{k + 1} > n_k$. יתרה מכך, היא גם מונוטונית עולה חזק על $\an$ שכן מהגדרה $a_{n_{k + 1}} > a_{n_k}$. סה''כ $(a_{n_k})_{k = 1}^{\infty}$ ת''ס מונוטונית עולה ממש וסיימנו.
	\end{proof}
	\rmark{מה שהבטיח לנו את קיום המינימום, פרט לכך שהקבוצה לא ריקה, הוא שהסדר על הטבעיים \textbf{סדר טוב}. }

	\lem{תהא $\an$ סדרה שבה אינסוף איברים שונים. אם ל־$\an$ אין ת''ס מונוטונית עולה ממש, אז יש לה ת''ס מונוטונית יורדת ממש. }
	\rmark{ת''ס של ת''ס היא ת''ס}
	\begin{proof}[הוכחת משפט בולצאנו־וייראשטראס]
		תהא $\an$ סדרה. נפריד למקרים.
		\begin{itemize}
			\item נניח ש־$\{a_n\}_{n = 1}^{\infty}$ (הטווח של $\an$) סופית. אז קיים $\ml \in \R$ כך ש־$\sof{\{n \in \N \mid a_n = \ml\}} = \az$. נבנה ברקורסיה ת''ס $\{a_{n_k}\}$ עבורה $a_{n_k} = \ml$ לכל $k \in \N$.
			\item אם $\{a_n\}_{n = 1}^{\infty}$ אין־סופי, אז מהלמה הקודמת קיימת ל־$\an$ ת''ס סדרה מונוטונית (ממש) $a_{n_k}$. בגלל ש־$\an$ חסומה אז בפרט $a_{n_k}$ חסומה. לפי משפט קודם כל סדרה מונוטונית חסומה היא מתכנסת, וסיימנו.
		\end{itemize}
		סה''כ בשני המקרים מצאנו ת''ס מתכנסת.
	\end{proof}
	משפט בולצאנו־וייראשטראס השתמש במשפט וייראשטראס (הראשון), שתלוי באקסיומת השלמות. משפט בוצאלנו־וייראשטראס תלוי באקסיומת השלמות!

	להלן הוכחה נוספת, קונסטקרטיבית אפילו פחות (לא שפונקציית בחירה זה קונסטקרטיבי במיוחד):
	\begin{proof}[הוכחה נוספת לבולצאנו־וייראשטראס]
		נסמן ב־$A$ את התמונה של $\an$. אם $A$ סופית – כמו קודם. אחרת $A$ אינסופית. נגדיר את הקבוצה:
		\[ B = \{x \in \R \co \sof{\{y \in A \mid y \le x\}} \ge \az\} \]
		$B$ חסומה מלמטה (למשל ע''י $\inf a_n$). היא לא ריקה, כי לדוגמה $\sup a_n \in B$. לכן ל־$B$ קיים חסם תחתון (אקסיומת השלמות). נסמן $\ag = \inf B$. יהי $\eg > 0$. אז קיים $b \in B$ כך ש־$b < \ag + \eg$. מתקיים $\ag - \eg < \ag \implies \ag - \eg \notin B$  לכן $\{y \in A \mid y \le b \le \ag + \eg\}$ אין־סופית, אבל $\{y \in A \mid y \le \ag - \eg\}$ סופית. לכן, $A_{\eg} = \{y \in A \co \sof{y - \ag} < \eg\}$ אינסופית!

		נסכם: לכל $\eg > 0$, ולכל $N \in \N$ קיים $n \ge \N$ כך ש־$\sof{a_n - \ag} < \eg$, בגלל ש־$A_\eg$ אינסופי.  וזו כבר הגדרה שקולה לקיום גבול חלקי, כמו שנראה בקרוב.
	\end{proof}
	\theo{$\forall \eg > 0 .\, \forall N \in \N.\, \exists n \ge N \co \sof{a_n - \ag} < \eg$ אמ''מ לקבוצה יש גבול חלקי ב־$\ag$. } \begin{proof}
		\begin{itemize}
			\item[$\impliedby$]נניח את הטענה שנראית מפחיד. נגדיר:
			\[ \begin{cases}
				n_1 = \min\{n \in \N \co \sof{a_n - \ag} < 1\} \\
				n_{k + 1} = \min \{n \in \N \co n < n_k \land \sof{a_n - \ag} < \frac{1}{k + 1}\}
			\end{cases} \]
			אז $a_{n_{k}}$ ת''ס של $\an$. יהי $\eg > 0$. קיים $K \in \N$ כך ש־$\frac{1}{K + 1} < \eg$. יהי $k \ge K$ אז:
			\[ \sof{a_{n_k} - \ag} < \frac{1}{k} \le \eg \]
			וסה''כ $\limsi a_{n_k} = \ag$ וסיימנו.
			\item[$\implies$]לבית
		\end{itemize}
	\end{proof}
	\rmark{המשפט לעיל הוא לא באמת משפט בקורס. צריך להוכיח אותו כל פעם מחדש. }

	''בשפה של בני אדם``, הטענה השקולה הזו אומרת שבכל קטע פתוח שמכיל את $\ag$ יש אינסוף איברים מהסדרה, [וההגדרה של גבול לא חלקי דורשת ש־] מחוץ אליו, יש מספר סופי של איברים.

	\cola{לכל סדרה יש גבול חלקי במובן הרחב. }

	\theo{סדרה מתכנסת אמ''מ יש לה גבול חלקי יחיד. }\begin{proof}
		\begin{itemize}
			\item[$\impliedby$] בכיוון הראשון, נוכיח: תהא $\an$ סדרה. יהי $\ml \in \R$. אם $\limasi = \ml$ אז כל ת''ס של $\an$ מתכנסת ל־$\ml$.

			כיוון זה לבית. שימו לב שצריך להפריד למקרים גבולות מתבדרים וכאלו שאינם.

			\item[$\implies$] עתה, תהא $\an$ סדרה, ויהי $\ml \in \R$. אם כל ת''ס של $\an$ מתכנסת ל־$\ml$, אז $\limasi a_n = \ml$.

			לבית גם כן. (זה טרוויאלי. $\an$ ת''ס של עצמה וסיימנו)
		\end{itemize}
	\end{proof}

	\theo{תהא $\an$ סדרה חסומה ויהי $\ml \in \R$. נניח כי כל ת''ס \textit{מתכנסת} של $a_n$ מתכנסת ל־$\ml$. אז $\limasi = \ml$. }
	\rmark{מה לא טרוויאלי כאן? אי אפשר פשוט לבחור את $\an$, שכן היא לא מתכנסת בהכרח (צריך להוכיח את זה). }
	\begin{proof}
		יהי $\eg > 0$. נסמן $A = \{n \in \N \co \sof{a_n - \ml} \ge \eg\}$. נניח בשלילה ש־$A$ אינסופית. נסמן $A_{+} := \{n \in \N\co a_n \ge \ml + \eg\}$ ו־$A_- = \{ n \in \N \co a_n \le \ml - \eg\}$. משום ש־$A = A_+ \cup A_-$, ללא הגבלת הכלליות, $A_+$ אינסופית. לכן קיימת ת''ס $a_{n_k}$ כך שלכל $k \in \N$, $n_{k} \in A_+$. $a_{n_k}$ חסומה ולכן יש לה ת''ס $a_{n_{k_j}}$ מתכנסת. נסמן את גבולה $m$. לכל $j \in \N$, מתקיים $\ml + \eg \le a_{n_{k_j}}$. לכן $m \ge \ml + \eg> \ml$, כלומר $m \neq \ml$ גבול חלקי של $\an$ בסתירה.
	\end{proof}
	המטרה בלהכניח ש־$\an$ חסומה, היא לחסוך את הפיצול למקרה האין־סופי.
		יש לנו שתי תכונות עבור תכונות של סדרות. תהא $\an$ סדרה, אז:
	\begin{itemize}
		\item הרעיון: לכל אינדקס שנבחר, יש עוד אינסוף מעליו שמקיימים את התכונה.
		\defi{נאמר כי תכונה היא \textit{שכיחה} בסדרה כאשר אינסוף מאיברי הסדרה מקיימים את התכונה. (באנגלית: infinitely often). }
		\item הרעיון: החל מנקודה מסויימת, כל איברי הסדרה מקיימים את הדרוש.
		\defi{נאמר שתכונה קוראת \textit{כמעט תמיד} כאשר כל איברי הסדרה, פרט למספר סופי, מקיימים את התכונה. (באנגלית: almost everywhere)}. \envendproof
	\end{itemize}
	ואז, נקבל כמו בשבוע שעבר את שתי הטענות הבאות:
	\begin{itemize}
		\item יהי $\ml \in \R$ הוא גבול של $\an$ אמ''מ לכל $\eg > 0$, מתקיים $\sof{a_n - \ml} < \eg$ כמעט תמיד.
		\item יהי $\ml \in \R$ הוא גבול חלקי של $\an$ אמ''מ לכל $\eg > 0$ מתקיים $\sof{a_n - \ml} < \eg$ שכיחה.
	\end{itemize}

	השקילויות האלו נכונות רק בגלל שהסדרות שלנו בדידות. נזכור שראינו בשבוע שעבר ש־$\ml$ גבול חלקי של $\an$ אם ורק אם לכל $\eg > 0$ ולכל $N \in \N$ קיים $n \ge N$ כך ש־$\sof{a_n - \ml} < \eg$.

	\noti{תהא $\an$ סדרה. את אוסף הגבולות החלקיים של $\an$ נסמן $\hat P(\an)$. }
	\noti{תהא $\an$ סדרה. את אוסף הגבולות החלקיים הסופיים (כלומר לא $\pm \inft$) של $\an$ נסמן $P(\an)$. }
	יש כאן קצת abuse of notation כאשר אנו מתייחסים ל־$\pm \inft$ כאובייקטים.

	בעזרת הסימונים הללו נקבל ניסוח שקול של משפט בולצאנו־ווייראשטראס (לכל סדרה חסומה יש ת''ס מתכנסת):
	\cola{לכל $\an$ סדרה, $\hat P(\an) \neq \varnothing$. }

	\theo{תהא $\an$ סדרה, חסומה. תהא $\bn$ סדרה, המקיימת:
		\begin{enumerate}
			\item $\forall n \in \N$ ש־$b_n \in P(\an)$
			\item $b_n$ מתכנסת ל־$\ml$
	\end{enumerate}
	אז $\ml \in P(\an)$. }

	\begin{proof}
		יהי $\eg > 0$. יהי $N \in \N$. ידוע $\limbsi = \ml$ לכן קיים $N_1 \in \N$ החל ממנו $\forall n \ge N_1 \co \sof{b_n - \ml} < \frac{\eg}{2}$. אז $b_{N_1} \in P$, לכן קיים $n \ge N$ כך ש־$\sof{a_n - b_{N_1}} < \frac{\eg}{2}$. מא''ש המשולש:
		\[ \sof{a_n - \ml} \le \sof{a_n - b_{N_1}} + \sof{b_{N_1} - \ml} < \frac{\eg}{2} \cdot 2 = \eg \]
	\end{proof}

	\theo{תהא $\an$ חסומה. אז ל־$P$ יש מקסימום ומינימום. }
	\rmark{הסופרמום של $\an$ הוא לא הסופרמום של $P$. לדוגמה עבור $\an = \frac{1}{n}$ אז $1 = \sup \an$ למרות ש־$P(\an) = \{0\}$. }\begin{proof}
		ראינו ש־$\an$ חסומה לכן $P$ חסומה. מבולצאנו־וויראשטראס, $P \neq \varnothing$. לכן ל־$P$ יש סופרמום ואינפימום. נסמן $\ag = \sup P, \bg = \inf P$. יהי $\eg > 0$. ידוע שקיים $\ml \in P$ כך ש־$\ml > \ag - \frac{\eg}{2}$. כמו כן $\ml \le \ag < \ag + \frac{\eg}{2}$. סה''כ $\sof{\ml - \ag} < \frac{\eg}{2}$. יהי $N \in \N$. אז קיים $n \ge N$ כך ש־$\sof{a_n - \ml} < \frac{\eg}{2}$. מא''ש המשולש:
		\[ \sof{a_n - \ml} < \sof{a_n - \ml} + \sof{\ml - \ag} < \frac{\eg}{2} \cdot 2 = \eg \]
		מכאן ש־$\ag$ גבול חלקי של $\an$ ולכן $\ag \in P$, כלומר $\ag = \max P$.

		באופן דומה (תרגיל לבית) אפשר להראות ש־$\bg = \min P$.
	\end{proof}

	מכאן, אפשר להראות את הטענה הבאה (זהו \textit{אינו} משפט בקורס):
	\theo{תהא $\varnothing \neq A \subseteq \R$. אם $A$ חסומה מלעיל, אז קיימת סדרה $\an\co \N \to A$ כך ש־$\limasi = \sup A$. }\begin{proof}
		נסמן $\ag = \sup A$. ידוע שלכל $n \in \N$ קיים $a_n \in A$ כך ש־:
		\[ \ag - \frac{1}{n} < a_n \le \ag < \ag + \frac{1}{n} \]
		(מהגדרת סופרמום). נקבל ש־$\limasi = \ag$. (הערה: $\an$ למעשה פונקציית בחירה, וצריך כאן את אקסיומת הבחירה הרציפה).
	\end{proof}

	\noti{תהי $\an$ סדרה. נסמן ב־$\limssi a_n$ את הגבול החלקי הגדול ביותר של $\an$. בעברית, הוא יקרא \textit{גבול עליון. }}
	\noti{תהי $\an$ סדרה. נסמן ב־$\limisi a_n$ את הגבול החלקי הקטן ביותר של $\an$. בעברית, הוא יקרא \textit{גבו לתחתון}. }
	\rmark{אם $\an$ אינה חסומה מלעיל, $\limssi a_n = \infty$ ואם $\an$ אינה חסומה מלרע אז $\limisi a_n = -\infty$. בשביל להראות את זה צריך עוד קצת טענות. }

	\theo{תהא $\an$ חסומה מלעיל. בהינתן $\ml \in \R$ הגבול העליון של $\an$ אמ''מ לכל $\eg > 0$ מתקיים:
	\begin{enumerate}
		\item $a_n < \ml + \eg$ כמעט תמיד.
		\item $a_n > \ml - \eg$ שכיח.
	\end{enumerate}}

	\begin{proof}\,
		\begin{itemize}
			\item[$\impliedby$]נניח ש־$\ml$ הגבול העליון של $\an$. יהי $\eg > 0$. נניח בשלילה כי לכל $N \in \N$ קיים $n \ge N$ כך ש־$a_n \ge \ml + \eg$. נבנה ת''ס באופן הבא:
			\[ \begin{cases}
				n_1 = \min\{n \in \N \mid a_n \ge \ml + \eg\} \\
				n_{k + 1} = \min \smash{\underbrace{\{n > n_k \mid a_n \ge \ml + \eg\}}_{\neq \varnothing}}
			\end{cases} \]

			הסדרה לעיל אכן איננה ריקה בגלל הנתון. אז $a_{n_k}$ ת''ס של $\an$ שכל איבריה בקטע $[\ml + \eg, + \infty)$ כת''ס של $\an$ היא חסומה, ולכן יש לה ת''ס מתכנסת לגבול של $m \in \R$. $m$ גבול חלקי של $\an$ עצמה (ת''ס של ת''ס היא ת''ס) ומקיים $m \ge \ml + \eg > \ml$ (כי $a_{n_k}$ חסומה ב־$\ml + \eg$) בסתירה לכך ש־$\ml$ גבול עליון.

			לכן קיים $N \in \N$ כך שלכל $n \ge N$, מתקיים $a_n < \ml + \eg$. מכאן ש־$\an < \ml + \eg$ כמעט תמיד.

			עתה נראה ש־$a_n > \ml - \eg$ שכיחה. יהי $N \in \N$. ידוע $\ml$ גבול חלקי של $\an$ לכן קיים $n \ge N$ כך ש־$\sof{a_n - \ml} < \eg$. לכן $a_n > \ml - \eg$ (עם קצת מנניפולציות אלגבריות).
			\item[$\implies$]נניח (1) $a_n < \ml + \eg$ (2)כמעט תמיד ו־ $\an > \ml - \eg$ שכיחה. יהי $\eg > 0$. מ־(1) קיים $N_1 \in \N$ כך ש־$\forall n \ge N_1 \co a_n < \ml + \eg$. יהי $N \in \N$. מ־(2) קיים $n \ge \max{N, N_1}$ כך ש־$a_n > \ml - \eg$. אז $n \ge N_1$ לכן $a_n < \ml + \eg$ ומכאן $\sof{a_n - \ml} < \eg$ לכן $\ml \in P$.

		נראה שהוא העליון. יהי $m \in P$. נניח בשלילה $m > \ml$. נסמן $\eg = \frac{m - \ml}{2}$. מכיוון ש־$m \in P$ לכן $\sof{a_n - m} < \eg$ שכיח, כלומר אינסוף מאיברי הסדרה גדולים מ־$\ml - \eg$. לכן $a_n < \ml + \eg$ לא כמעט תמיד, בסתירה. מכאן ש־$m \le \ml$ ולכן $\ml = \limsup a_n$.
		\end{itemize}\envendproof
	\end{proof}
	\rmark{אפשר לבצע הוכחה סימטרית עם $\liminf$. }
	\rmark{אם גם (1) וגם (2) מתקיימות כמעט תמיד, נקבל מיד את הגדרת הגבול. }

	\theo{תהא $\an$ סדרה חסומה. אז לכל $\eg > 0$ כמעט תמיד:
	\[ \liminf a_n - \eg < a_n < \limsup a_n + \eg \]}
	בעצם, יש את הקטע הפתוח:
	\[ (\liminf a_n - \eg, \limsup a_n + \eg) \]
	וכל איברי הסדרה פרט לכמות סופית של מספרים נמצאים בו.

	\noti{בהינתן $F \co \N \to \R \cup \{\pm \infty\}$ כלשהי:
		\[ \inf_n F(n) = \inf \{F(n) \mid n \in \N\} \quad \sup_n F(n) = \sup \{F(n) \mid n \in \N\} \]}

	תרגיל: תהא $\an$ סדרה חסומה. אז:
	\[ \limssi a_n = \inf_n \sup_{k \ge n}a_k \]
	\begin{proof}
		נגדיר $S_n = \sup_{k \ge n} a_k = \sup\{a_k \mid k \ge n\}$.

		יהיו $n, m$ ונניח $n > m$. אז:
		\[ \{a_k \mid k \ge n\} \subseteq \{a_k \mid k \ge m\} \]
		לכן (תרגיל):
		\[ S_n =\sup\{a_k \mid k \ge n\} \le \sup \{a_k \mid k \ge m\} = S_m \]
		לכן $S_n$ מונוטונית יורדת ולכן מתכנסת ל־$\inf S_n$. נסמן $S = \inf S_n$. יהי $\ml$ גבול חלקי של $\an$. אז קיימת ת''ס $a_{m_k}$ של $\an$ המקיימת $\lim_{k \to \inft} a_{n_k} = \ml$. לכל $k \in \N$ מתקיים $a_{n_k} \le S_{n_k}$ לפי הגדרת חסם עליון. כמו כן $\lim_{k \to \inft} S_{n_k} = S$ מקיים $\ml \le S$. לכן $\lg := \limsup {n \to \infty} a_n \le S$.

		יהי $\eg > 0$. אז $a_n < \lg + \eg$ כמעט תמיד. קיים $N \in \N$ כך ש־$\forall n \ge N \co a_n < \lg + \eg$. לכן $\forall n \ge N \co S_n \le \lg + \eg$. לכן $S \le \lg + \eg$ כלומר $S \le \lg$ (יש כאן למה: $(\forall \eg > 0 \co \ag \le \bg + \eg) \implies \ag \le \bg$). מכאן $S = \lg$.
	\end{proof}

	''טרוויאלי זה היבריס``.

	\section*{סדרות קושי}
	''הוא היה כומר, ואת כל הטענות שלו הוא גנב מתלמידים שלו. המון תלמידים מיוחסים לו``.

	\defi{תהא $\an$ סדרה. נאמר ש־$\an$ סדרת קושי, כאשר:
		\[ \forall \eg > 0 .\, \exists N \in \N .\, \forall n, m \ge N \co \sof{a_n - a_m} <\eg  \]}
	הטענה המרכזית שנראה על סדרות קושי, היא שסדרה מתכנסת אמ''מ היא סדרת קושי.

	יש כאן נקודה נחמדה. אנחנו לא באמת צריכים לעבוד ערך מוחלט. יש לנו רק שלוש תכונות שמעניינות אותנו:
	\begin{enumerate}
		\item \textbf{אי־שליליות ולא מנוונת: }לכל $x, y \in \R \co \sof{x - y} \ge 0$ ו־$\sof{x - y} = 0$ אמ''מ $x = y$.
		\item \textbf{סימטריות: }$\forall x, y \in \R \co \sof{x - y} = \sof{y - x}$
		\item \textbf{א''ש המשולש: }$\forall x, y \in \R \co \sof{x - z} \le \sof{x - y} + \sof{y - z}$
	\end{enumerate}
	פונקציה $d \co A^2 \to \R$ נקראת \textit{מטריקה} אם היא מקיימת את שלושת התכונות לעיל. מרחב מטרי נקרא \textit{שלם} אם כל סדרת קושי מתכנסת בו. באיזשהו הבט, צריך משהו בסגנון $\R$ (אקסיומת השלמות) או דברים דומים לו כדי שהמרחב המטרי יהיה שלם. ההגדרה של סדרת קושי מאוד תועיל לנו (בקורסים אחרים) כאשר לא בהכרח ברור מזה המושג של גבול.

	\theo{תהא $\an$ סדרה. אז $\an$ מתכנסת אמ''מ $\an$ סדרת קושי. }\begin{proof}\,
		\begin{itemize}
			\item[$\implies$]נניח ש־$\an$ מתכנסת. אז קיים $\ml \in \R\co \limasi = \ml$. יהי $\eg > 0$. קיים $N \in \N$ כך ש־$\forall n \ge N \co \sof{a_n - \ml} < \frac{\eg}{2}$. נתבונן ב־$N$. יהי $n, m \ge N$:
			\[ \sof{a_n - a_m} \le \sof{a_n - \ml} + \sof{\ml - a_m} < \frac{\eg}{2} + \frac{\eg}{2} = \eg \]
			(הכיוון הזה נכון בכל מרחב מטרי. היינו צריכים את תכונות המטריקה בלבד, ולא היינו צריכים את אקסיומת השלמות)
			\item[$\impliedby$]נניח $\an$ סדרת קושי. קיים $N \in \N$ כך ש־$\forall n, m \ge N \co \sof{a_n - a_m} < 1$. נתבונן ב־$M = \max\{\sof{a_1}, \sof{a_2}, \dots \sof{a_{N - 1}}, \sof{a_N} + 1\}$. יהי $n \in \N$. אם $n \ge N$ אז $\sof{a_n} \le M$. אחרת $\sof{a_n - a_N} < 1$ ולכן $\sof{a_n} < \sof{a_N} + 1 < M$. מכאן ש־$\an$ חסומה. לפי בולצאנו־ווייראשטראס (פוף! הנחנו את אקסיומת השלמות) ל־$\anc$ יש ת''ס $a_{n_k}$ המתכנסת לגבול $\ml \in \R$.

			עתה, אקסיומת השלמות הפילה לנו גבול $\ml$ מהשמיים, ומכאן נוכל להמשיך לעבוד לפי הגדרה. יהי $\eg > 0$. אז קיים $K_1$ ש־$\forall k \ge K_1 \co \sof{a_{n_k}} < \frac{\eg}{2}$ וכן קיים $N_1$ כך ש־$\forall n, m \ge N_1 \co \sof{a_n - a_m} < \frac{\eg}{2}$. קיים $K_2$ כך שלכל $k \ge K_2$, $n_k > N_1$ (כי $n_k$ סדרת טבעיים מונוטונית עולה ממש). נתבונן ב־$N = \max\{n_{K_1}, n_{K_2}\}$. יהי $n \ge N$. קיים $k \in \N$ כך ש־$n_k > n$. ואז:
			\[ \sof{a_n - \ml} \le \sof{a_n - a_{n_k}} + \sof{a_{n_k} - \ml} < \frac{\eg}{2} + \frac{\eg}{2} = \eg \]
		\end{itemize}\envendproof
	\end{proof}

	\subsection*{חזקות ממשיות}
	\theo{תהא $a_n$ סדרת רציונליים המתכנסת ל־$0$. אז $\forall x > 0 \co \limsi x^{a_n} = 1$. }\begin{proof}\,
		\begin{itemize}
			\item נוכיח למקרה $x > 1$. ראינו בתרגול ש־$\limsi x^{\pm n\op} = \limsi \sqrt[n]{x} = 1$. יהי $\eg > 0$. קיים $P \in \N$ כך ש־$1 - \eg < x^{-\frac{1}{P}} < 1 < x^{\frac{1}{P}} < 1 + \eg$. קיים $N \in \N$ כך שלכל $n \ge N$, $\sof{a_n} \le \frac{1}{P}$. אזי $-\frac{1}{P} < a_n < \frac{1}{P}$. ממונוטוניות החזקה (שלא הוכחנו אבל ניחא) $1 - \eg < x^{-P\op} < x^{a_n} < x^{P\op} < 1 + \eg$.
			\item אם $x = 1$ זה טרוויאלי ואם $x < 1$ אז מאריתמטיקה של גבולות סיימנו.
		\end{itemize}
	\end{proof}

	\theo{תהא $\an$ סדרת רציונלים מתכנסת. אז לכל $x \ge 0$ הסדרה $x^{\an}$ מתכנסת. }\begin{proof}
		יהי $\eg > 0$. $\an$ מתכנסת ולכן היא חסומה. מכאן ש־$x^{\an}$ חסומה. כלומר קיים $M > 0$ כך שלכל $n \in \N$ מתקיים $\sof{x^{a_n}} \le M$. קיים $p \in \N$ כך ש־:
		\[ 1 - \frac{\eg}{M + 1} < x^{- \frac{1}{O}} < 1 < x^{\frac{1}{P}} < 1 + \frac{\eg}{M + 1} \]
		$\an$ מתכנסת ולכן סדרת קושי. קיים $N \in \N$ כך ש־$\forall n, m \ge N \co \sof{a_n - a_m} < \frac{1}{P}$ ומחוקי חזקות $\sof{x^{a_n} - x^{a_m}} = \sof{x^{a_m}}\sof{x^{a_n - a_m} - 1} \le M \cdot \frac{\eg}{M + 1} < \eg$.
	\end{proof}

	\theo{בהינתן $\an, \bn$ סדרות רציונליים שתיהן מתכנסות לאותו הגבול, אז $\limsi x^{a_n} = \limsi x^{b_n}$. }
	ההוכחה לבית. מהמשפט האחרון יש לנו אי־תלות בבחירת נציג. אפשר גם להראות שזהו אכן יחס שקילות (בפרט קיימת סדרת רציונליים השואפת ל־$\ag$, לכל $\ag \in \R$). לכן נוכל להגדיר:
	\defi{יהי $\ag \in \R$ ו־$x> 0$. נגדיר $x^{\ag} := \limsi x^{a_n}$ כאשר $\an$ סדרת רציונליים המתכנסת ל־$\ag$. }

	\theo{תהא $\an$ סדרה (לא בהכרח סדרת רציונליים) ויהי $x > 0$. יהי $\ag \in \R$. אז $\limsi a_n = a$ אמ''מ $\limsi x^{a_n} = x^{a}$. }

	\theo{חזקות ממשיות מקיימות חוקי חזקות. }

	\subsection*{עקרון הרווחים המקוננים של קנטור}
	(ידוע בעיקר כ''משפט החיתוך של קנטור``)
	תהאנה $\an, \bn$ סדרות. נניח כי:
	\begin{enumerate}
		\item \hfil $\forall n \in \co a_n < a_{n + 1} < b_{n + 1} < b_n$
		\item \hfil $\limsi b_n - a_n = 0$
	\end{enumerate}
	אז:
	\[ \exists c \in \R \co \bigcap_{n = 1}^{\infty}[a_n, b_n] = \{c\} \]
	\begin{proof}
		ידוע $\an$ מונוטונית עולה וחסומה מלעיל (ע''י $b_1$). לכן $\an$ מתכנסת. נסמן את גבולה $c$. מאריתמטיקה של גבולות:
		\[ \limsi a_n + (b_n - a_n) = c \]
		לכן $\limbsi = c$. ידוע $\an$ עולה ו־$\bn$ יורדת ולכן לכל $n \in \N$, מתקיים $a_n \le c \le b_n$, כלומר $c \in [a_n, b_n]$ ומכאן $c \in \bigcup_{n = 1}^{\infty} [a_n, b_n]$. יהי $d \in \R$. נניח $d \in \bigcap_{n =1}^{\inft} [a_nm b_n]$. יהי $\eg > 0$. קיים $n \in \N$ כך ש־$b_n - a_n < \eg$. $c, d \in [a_n, b_n]$ אזי $\sof{c - d} < b_n - a_n < \eg$ לכן $c = d$.
	\end{proof}
	גם כאן – ההוכחה נראית תמימה, אבל איפשהו באמצע מתחבא משפט וויראשטראס הראשון, שאומר שכל סדרה מונוטונית חסומה היא בעלת גבול. למעשה, עקרון הרווחים המקוננים של קנטור שקול לאקסיומת השלמות! בבית, מאוד מומלץ להוכיח את הכיוון ההפוך. תרגיל מעניין אחר הוא להוכיח את בולצאנו־וייראשטראס באמצעות עקרון הרווחים המקוננים במקום אקסיומת השלמות.


	\subsection*{לגוריתמים}
	\theo{לכל $a, b > 0$, אם $a \neq 1$ אז קיים ויחיד $x \in \R$ כך ש־$a^{x} = b$. }\begin{proof}
		נוכיח למקרה $a >1$. הוכחנו בבית ש־$\{a^{k} \mid k \in \N\}$ אינה חסומה. לכן קיים $k \in \N$ כך ש־$a^{k} > b$. מעקרון הסדר הטוב בטבעיים, קיים $k \in \N$ כך ש־$a^{k - 1} \le b < a^{k}$. נגדיר $x_1 = k - 1, y_1 = k$. נסמן $c = \frac{x_n + y_n}{2}$. אם $a^{x_n} \le b < a^{c}$ נגדיר $x_{n + 1} = x_n, y_{n + 1} = c$. אחרת נגדיר $x_{n + 1}= c, y_{n + 1} = y_n$ (המרצה מבצע חיפוש בינארי). ממשפט הרקורסיה $x_n$ קיימת. בשלב ה־$n + 1$ נקבל ש־$b \in [a^{x_n}, a^{y_n}]$ וגם $y_n - x_n = \frac{1}{2^{n - 1}}$. אז לכל $n \in \N$ מתקיים $x_n \le x_{n + 1} \le y_{n + 1} \le y_n$ וכן $\limsi y_n - x_n = 0$. לכן קיים $x \in \R$ כך ש־$\bigcup^{\infty}_{n = 1} [x_n, y_n] = \{x\}$. אז $\limsi y_n = \limsi x_n = x$ לכן $\limsi a^{x_n} = \limsi a^{y_n} = a^{x}$. כמו כן $\bigcup_{n = 1}^{\infty} [x^{a_n}, a^{y_n}] = \{b\}$ (לבית). לכן $b$ כזה קיים.

		היחידות נובעת ממונוטוניות החזקה.
	\end{proof}

	כל סדרה ניתנת לייצוג כטור. זו דרך אחרת להציג סדרות.

	\defi{תהא $\an$ סדרה. נגדיר את סדרת הסכומים החלקיים של $\an$ להיות:
	\[ \forall n \in \N \co S_n = \sumnko a_k \]}

	\textbf{הבחנה: }כל סדרה היא סדרת סכומים חלקיים של איזושהי סדרה. \begin{proof}
		תהי $\an$ סדרה, נגדיר את:
		\[ \begin{cases}
			b_1 = a_1 \\
			b_{n + 1} = a_{n + 1} - a_n
		\end{cases} \]
		נקבל שהסכום הטלסקופי:
		\[ \sumnko b_k = a_n \]
	\end{proof}
	אז למעשה אין שום דבר חשוב בסכום עצמו. מה שחשוב זה הקשר בין הסדרה עצמה לבין סדרת הסכומים החלקיים שלה.

	\noti{תהא $\an$ סדרה. תהי ב־$S_n$ את סדרת הסכומים החלקיים של $\an$. אז אם $S_n$ מתכנסת לגבול $\ml \in \R$ נאמר כי הטור $\sum_{k = 1}^{\inft} a_k$ מתכנס, ונסמן:
	\[ \sum_{k = 1}^{\inft} a_k = \ml \]}
	הבחנה חשובה: הסימון הזה של $\sumninf$ משמש אותנו להגיד שהטור לא מתכנס, כלומר נאמר ''$\sumninf$ לא מתכנס`` גם אם $\sumninf$ לא קיים. זאת בניגוד לגבולות, שם אנחנו לא ממש יכולים לכתוב ''$\limsi$ לא קיים`` (שכן $\limsi$ לא ביטוי מוגדר).

	\begin{itemize}
		\item \textbf{דוגמה: }יהי $1 \neq q \in \R$ ונגדיר $a_n = q^{n - 1}$ לכל $n\in \N^{+}$. נסמן ב־$S_n$ את סדרת הסכומים החלקיים. אז:
		\begin{itemize}
			\item \hfil $\displaystyle \forall n \in \N^{+} \co S_n = \frac{1 - q^{n}}{1 - q}$

			לכן הטור מתכנס אמ''מ $\sof{q} < 1$ (הוכחנו את זה בתרגיל הבית) ואז:
			\item \hfil $\displaystyle \sumninf q^{n - 1} = \frac{1}{q - 1}$
		\end{itemize}
		\item \textbf{דוגמה 2: }נגדיר $\forall n \in \N \co a_n := \frac{1}{n(n + 1)}$, ונסמן ב־$S_n$ את סדרת הסכומים החלקיים המתאימה ל־$\an$.

		נבחין שממכנה משותף:
		\[ \forall n \in \N \co \frac{1}{n(n + 1) = \frac{1}{n}} - \frac{1}{n + 1} \]
		ואז (סכום טלסקופי):
		\[ S_n = \sumkinf \cl{\frac{1}{k} - \frac{1}{k + 1}} = 1 - \frac{1}{n + 1} \]
		לכן $\sumkinf a_k$ מתכנסת וכן $\sumkinf a_k = 1$.

		מכאן אפשר להוכיח ש־$\sumninf \frac{1}{n^2}$ מתכנס (עשינו את זה גם בתרגול).
	\end{itemize}
	אלו פחות או יותר הדוגמאות היחידות (גיאומטרי וטלסקופי) שנראה בקורס הזה לגבי משהו שאשכרה מתכנס. בד''כ נרצה לדעת האם טור מסויים הוא מתכנס או לא. כשיהיו לנו אינטגרלים (בחדו''א 2א) יהיה לנו קצת יותר כוח להוכיח טורים. אבל כמו הרבה דברים בחדו''א, גם זה לא תמיד יספיק.

	\subsubsection*{קריטריון קושי להתכנסות טורים}
	תהא $\an$ סדרה. אז הטור $\sumninf a_n$ מתכנס אמ''מ:
	\[ \forall \eg > 0.\, \exists n \in \N.\, \forall N \le n \le m \co \underbrace{\sof{\sum_{k = m}^{n} a_k}}_{\sof{S_n -S_{m + 1}}} < \eg \]
	זה לא מעניין בכלל. זה פשוט קריטריון קושי לסדרות, אבל על סדרת הסכומים החלקיים.

	\cola{תהא $\an$ סדרה. אז אם $\sumninf a_n$ מתכנס, אז $\limsi a_n = 0$. }
	\rmark{הצד השני לא מתקיים, לדוגמה עבור $a_n = \frac{1}{n}$ מתקיים ש־$\sumninf \frac{1}{n} \approx \ln n \to \inft$ למרות ש־$n\op \to 0$. }

	בדומה לגבולות של סדרות, שינוי של מספר סופי של איברים (בסדרה המקורית) אולי ישנה את הגבול (כי סוכמים אותם), אבל לא עומד לשנות את ההתכנסות.

	\theo{הטור הוא לינארי, כלומר יהיו $\sumninf a_n, \sumninf b_n$ טורים מתכנסים. אז:
	\[ \sumninf \cl{a_n \pm b_n} = \sumninf a_n + \sumninf b_n \quad \quad \sumninf \ag a_n = \ag \sumninf a_n \]מתכנסים. }
	זה נובע ישירות מאריתמטיקה של גבולות, על סדרת הסוכמים החלקיים.

	\subsubsection*{התכנסות בהחלט}
	כאן יש אשכרה הגדרה חדשה.
	\defi{תהא $\an$ סדרה. נאמר כי הטור $\sumninf a_n$ \textit{מתכנס בהחלט} כאשר $\sumninf \sof{a_n}$ מתכנס. }
	\theo{אם טור מתכנס בהחלט, אז הוא בפרט מתכנס. }
	אין לנו שום דבר חכם להגיד על הקשר בין הגבולות של שניהם. עם ננסה להוכיח עם סנדוויץ' (תנסו), נכשל במהרה. יש כאן צורך בקסם, שיפילו לנו גבול מהשמיים, וזה בדיוק מה שאקסיומת השלמות מספקת לנו. ספציפית, נשתמש בקריטריון קושי שתלוי בה.
	\begin{proof}
		תהא $\an$ סדרה, ונניח ש־$\sumninf a_n$ מתכנס בהחלט. מקריטריון קושי, קיים $N \in \N$ כך ש־:
		\[ \forall n \ge m \ge N \co \sof{\sum_{k = m}^{n}\sof{a_k}}<\eg \]
		נתבונן ב־$N$. יהי $n \ge m \ge N$. מא''ש המשולש המוכלל:
		\[ \sof{\sum_{k = m}^{n} a_k} \le \sum_{k = m}^{n} \sof{a_k} = \sof{\sum_{k = m}^{n} \sof{a_k}} < \eg \]
		סה''כ מקריטריון קושי לטורים גם $\sumninf a_n$ מתכנס.
	\end{proof}

	\subsubsection*{טורים אי־שליליים}
	יש פרק שלם בטורים שעוסק בטורים שומרי סימן (איבריהם גדולים ממש מאפס או קטנים ממש מאפס. לצורך הנוחות מתעסק במקרה הראשון). יש לזה שתי סיבות:
	\begin{itemize}
		\item בגלל הנושא של התכנסות בהחלט.
		\item זה מקרה נפוץ שקורה הרבה בעולם האמיתי.
		\item יש משפטים מועילים על זה.
	\end{itemize}
	בהרבה מהמקרים נדרוש אי־שליליות בכל $\N$ גם אם זה נכון רק החל מ־$\N$ מסויים.

	''אפס הוא חיובי יחסית``

	\theo{תהא $\an$ סדרה, ונניח ש־$\forall n \in \N \co a_n \ge 0$. אז $\sumninf a_n$ אמ''מ סדרת הסכומים החלקיים חסומה. }
	(זה דורש את אקסיומת השלמות) אין כאן אשכרה הוכחה. אם $a_n \ge 0$ אז סדרת הסכומים החלקיים מונוטונית עולה, וממשפט (וויראשטראס 1) כל הסיפור הזה מתכנס אמ''מ סדרת הסכומים החלקיים חסומה.

	נתעסק קצת בקריטריוני השוואה.
	\begin{enumerate}
		\item תהיינה $a_n, b_n$ סדרות אי־שליליות. נניח כי $\forall n \in \N \co a_n \le b_n$ (למעשה, לא צריך לכל $\N$, מספיק כמעט תמיד. ההוכחה קצת שונה אבל כמעט תמיד יותר חזק). אז אם $\sumninf b_n$ מתכנס אז $\sumninf a_n$ מתכנס.
		\begin{proof}
			נניח ש־$\sumninf b_n$ מתכנס ל־$\ml$. ידוע $b_n$ מונוטונית ולכן מתכנסת לסופרמום שלה, ונסיק:
			\[ \sumnko a_k < \sumnko b_k \le \ml \]
			לכן $\sumninf a_n$ מונוטונית עולה וחסומה ולכן מתכנסת (יש כאן שימוש באקסיומת השלמות).
		\end{proof}
		\item נניח $\forall n \in \N\co b_n > 0$ (חיובית ממש!) ונניח $\limsi \frac{a_n}{b_n} \to \ml$ וכמו כן $\ml > 0$. אז $\sumninf a_n$ מתכנס אמ''מ $\sumninf b_n$ מתכנס. \begin{proof}
			נוכיח רק כיוון אחד, והכיוון השני יגרר מאריתמטיקה של גבולות (נהפוך את $\frac{a_n}{b_n}$ וזה חוקי כי $a_n$ ממקום מסויים לא נוגע ב־$0$ כי $\ml \neq 0$). קיים $N \in \N$ כך שלכל $n \ge N$ מתקיים:
			\[ \frac{a_n}{b_n} < \frac{3\ml}{2} \]
			(הראינו שזה נכון באופן כללי לכל מספר שגדול מ-$\ml$ + הנחנו אי־שליליות). כלומר לכל $n \ge \N$, מתקיים $a_n < \frac{3\ml}{2}b_n$. מקריטריון ההשוואה הראשון $\sumninf b_n$ מתכנס, ומאריתמטיקה $\sumninf \frac{3\ml}{2}b_n$ מתכנס.
		\end{proof}
		\item \textbf{מבחן השורש: }תהא $\an$ סדרה אי־שלילית. נניח כי קיים $q \in (0, 1)$ כך ש־$\forall n \in \N \co \sqrt[n]{a_n} \le q$. אז $\sumninf a_n$ מתכנס.
		\begin{proof}
			לכל $n \in \N$ נבחין ש־$a_n \le q^{n}$, וממבחן השוואה עם הטור הגיאומטרי (שמתכנס) סיימנו.
		\end{proof}
		\item \textbf{מבחן השורש הגבולי: }תהא $\an$ סדרה אי־שלילית. נניח ש־$\exists q \in [0, 1) \co \limsup_{n \to \inft} \sqrt[n]{q_n} < q$ אז $\sumninf a_n$ מתכנס.
		\rmark{זה משפט קצת יותר חזק מהקודם. }
		\rmark{לשני מבחני השורש כיוון אחר – אם $q < 1$ אז הטור מתבדר. }
		\begin{proof}ידוע $q < 1$ אז $\sqrt[n]{a_n} < \limsup_{n \to \inft} \sqrt[n]{a_n} + \frac{1 - q}{2}$ כמעט תמיד. לכן $\sqrt[n]{a_n} < \frac{1 + q}{2}$ כמעט תמיד. אזי $a_n < \cl{\frac{1 + q}{2}}^{n}$ כמעט תמיד, ידוע $\frac{1 + q}{2} < 1$ (כי $q < 1$, וכזה ממוצע משהו) כלומר $\sumninf \cl{\frac{1 + q}{n}}^{n}$ מתכנס ומהקריטריון הראשון $\sumninf a_n$ מתכנס.
		\end{proof}
		\item \textbf{מבחן המנה: }נניח $a_n > 0$ (כמעט תמיד) ויהי $q \in (0, 1)$, ונניח $\frac{a_{n + 1}}{a_n} \le q$ (כמעט תמיד) אז $\sumninf a_n$ מתכנס.
		\begin{proof}
			השורה התחתונה של ההוכחה היא:
			\[ a_{n + 1} = \frac{a_{n + 1}}{a_n} \cdot \frac{a_n}{a_{n - 1}} \cdots \frac{a_2}{a_1}a_1 \le q^{n} \cdot a_1 \]
			ואז מבחן ההשוואה.
		\end{proof}
		\item \textbf{מבחן המנה הגבולי: }יהי $\an > 0$. נסמן $\ml = \limsup_{n \to \inft} \frac{a_{n + 1}}{a_n}$ ו־$m = \liminf{n \to \inft} \frac{a_{n + 1}}{a_n}$ אז אם $\ml < 1$ אז $\sumninf a_n$ מתכנס, ואם $m > 1$ אז $\sumninf a_n$ מתבדר. \begin{proof}
			לבית.
		\end{proof}
	\end{enumerate}

	\textbf{דוגמה: }האם הטור $\sumkinf \frac{k^{\frac{k}{2}}}{k!}$ מתכנס? נסמן ב־$a_n = \frac{n^{\frac{n}{2}}}{n!}$. אז:
	\[ \frac{a_{n + 1}}{a_n} = \frac{\frac{(n + 1)^{\frac{n + 1}{2}}}{(n + 1)!}}{\frac{n^{\frac{n}{2}}}{n!}} = \frac{(n + 1)^{\frac{n + 1}{2}}n!}{n^{\frac{n}{2}}(n + 1)!} = \frac{(n + 1)^{\frac{n - 1}{2}}}{n^{\frac{n}{2}}} = \sqrt{\frac{(n + 1)^{n - 1}}{n^{n}}} = \sqrt{\cl{\frac{n + 1}{n}}^{n} \cdot \frac{1}{n + 1}} \to \sqrt{e \cdot 0} = 0 \]
	לכן $\limsup \frac{a_{n + 1}}{a_n} = 0 < 1$. ממשפט המנה הגבולי נקבל שזה מתכנס.

	\subsubsection*{קירוב סטרלינג}
	קירוב סטרלינג אומר ש־:
	\[ \limsi \frac{n!}{\cl{\frac{n}{e}}^{n}\sqrt{e \pi n}} = 1 \]
	אינטואיטבית, זה אומר ש־$n$ עצרת בגבול מתנהג כמו החזקה $\cl{\frac{n}{e}}^{n} \cdot \sqrt{2\pi n}$. לא נוכיח אותו – המרצה לא מודע לאף הוכחה שמשתמשת בכלים שלמדנו.

	עתה נפתור את התתרגיל ממקודם, של $\sumkinf \frac{k^{\frac{k}{2}}}{k!}$, באמצעות קירוב סטירלינג ומבחן השורש הגבולי. נגדיר $b_n = \frac{n^{\frac{n}{2}}}{\cl{\frac{n}{e}}^{n}\sqrt{2\pi n}}$. נקבל:
	\[ \sqrt[n]{b_n} = \frac{\sqrt n}{\frac{n}{e} \cdot \underbrace{(2 \pi n)^{\frac{1}{2n}}}_{1}} \to 0 \]
	כמו כן לפי סטרלינג:
	\[ \limsi \frac{\frac{n^{n/2}}{n!}}{\frac{n^{n/2}}{\cl{\frac{n}{e}}^{n}\sqrt{2 \pi n}}} = \limsi \frac{\cl{\frac{n}{e}}^{n}\sqrt{2 \pi n}}{n!} = 1 \]
	לכן לפי משפט ההשוואה הגבולי $\sumninf \frac{n^{n/2}}{n!}$ מתכנס.

	למעשה יש עוד מבחן לטורים אי־שליליים שלא הזכרנו.
	\begin{enumerate}
		\skipitems{8}
		\item תהא $\an$ סדרה מונוטונית יורדת ואי־שלילית אז $\sumninf a_n$ מתכנסת אמ''מ $\sumninf 2^{n}a_{2n}$ מתכנסת.
		\begin{proof}
			\begin{itemize}
				\item[$\implies$]נניח שהטור מתכנס. יהי $n \in \N$.
				\[ \sumnko 2^{k}a_{2k} = 2 \cdot \sumnko 2^{k - 1}a_{2k} = 2 \cdot \sumnko \sum_{\ml = 1}^{2^{k} - 1}a_{2k} = 2 \cdot \sumnko \sum_{\ml = 1}^{2^{k - 1}}a_{2^{k - 1} + \ml} = 2 \sum_{k = 2}^{2^{n}} a_k \le 2 \sumninf a_n \]
				וכזה מהמבחן הראשון סיימנו שוב.
				\item[$\impliedby$]נניח ש־$\sumninf 2^{n}a_{2n}$ מתכנס. נוכיח ש־$\sumninf a_n$ מתכנס.
				\[ \sumnko \le \sum_{k = 1}^{2^{n}}a_k = \sum_{k = 1}^{2^{n}} \sum_{\ml = 0}^{2^{k -1} -1}a_{2^{k} + \ml} \le \sumnko 2^{k - 1}a_{2^{k - 1}} \le \sum_{k = 0}^{\inft} 2^{k}a_{2k} \]
				כאן, נחליף באיבר הראשון.
			\end{itemize}
		\end{proof}
	\end{enumerate}

	אז למה אנחנו צריכים את מבחן העיבוי?
	\begin{itemize}
		\item עבור $\ag \le 1$ הראינו ש־$\frac{1}{n^{\ag}} \ge \frac{1}{n}$ ולכן $\sum \frac{1}{n^{\ag}}$ מתבדר.
		\item עבור $\ag \ge 2$ הראינו ש־$\frac{1}{n^{\ag}} \le \frac{1}{n^{2}}$ ולכן $\sum \frac{1}{n^{\ag}}$ מתכנס.
	\end{itemize}
	מה לגבי כל מה שבין 1 ל־2?
	\theo{הטור $\sumninf \frac{1}{n^{\ag}}$ מתכנס אמ''מ $\ag > 1$. }\begin{proof}
		יהי $\ag > 0$. אז $\frac{1}{n^{\ag}}$ מונוטונית יורדת וחיובית. נסמן $a_n = \frac{1}{n^{\ag}}$. אז:
		\[ b_n = 2^{n}a_{2n} = \frac{2^{n}}{n^{\ag n}} = 2^{n(1 - \ag)} = (2^{1 - \ag})^{n} \]
		נבחין ש־$b_n$ גיאומטרי. הוא מתכנס אמ''מ $2^{1 - \ag} \in (0, 1)$, שמתקיים אמ''מ $\ag > 1$. עוד ידוע ש־$b_n$ מתכנס אמ''מ $a_n$ מתכנס ממבחן העיבוי, וסה''כ $a_n$ מתכנס אמ''מ $\ag > 1$.
	\end{proof}

	נעשה עוד תרגיל ,אולי קצת פחות מועיל.

	\exe{האם $\sumninf \frac{1}{n\ln n}$ מתכנס? (בסיס הלוגוריתם לא משנה)}\begin{proof}
		נגדיר $a_n = \frac{1}{n \ln n}$. ניעזר במבחן העיבוי:
		\[ 2^{n}a_{2n} = 2^{n} \cdot \frac{1}{2^{n}\ln (2^{n})} = (n \ln 2)\op \to \inf \]
		לכן גם $\sum 2^{n}a_{2n}$ מתבדר לכן $\sum a_n$ מתבדר.
	\end{proof}

	\subsubsection*{טורים משני סימן}
	כל מה שאמרנו על שומרי סימן נכון על מי ששומר סימן כמעט תמיד. כלומר אלו שלא נופלים לקטגוריה הזו, הטורים משני הסימן, מחליפים סימן באופן שכיח. הטורים הראשונים שנדבר עליהם הם כאלו שלא רק משנים סימן באופן שכיח, אלא ממש כל מעבר.

	\begin{Theorem}[משפט לייבניץ]
		תהא $\an$ סדרה חיובית ומונוטונית יורדת שגבולה $0$. אז:
		\[ \sumninf (-1)^{n}a_n \]
		מתכנס.
	\end{Theorem}
	התובנה החשובה בהוכחה היא שאפשר לדעת את המרחק מהגבול, בכל נקודה בסכום, גם אם קשה לחשב אותו. ''אבל המבחנים לא עובדים לי. [תשובה: ]אויויוי``.

	אנחנו לא יודעים מה הגבול, אנחנו לא רוצים לדעת מה הגבול, המבחנים עובדים רק לדברים משמרי סימן... נשארנו עם כושי.
	\begin{proof}
		יהי $\eg > 0$. קיים $N \in \N$ כך שלכל $n \ge N$ מתקיים $\sof{a_n} < \eg$. נתבונן ב־$N$. יהיו $n \ge m \ge N$. השטיק יהיה שהזוגות $(a_{k} - a_{k - 1}) < 0$.
		\begin{itemize}
			\item אם $n- m$ זוגי נקבל:
			\begin{align*}
				&a_m - a_{m + 1} + a_{m + 2} - a_{m + 3} + \cdots + a_n \\ = &a_m + (a_{m + 2} - a_{m + 1}) + (a_{m + 1} - a_{m + 3}) + \cdots + (a_n - a_{n - 1}) < a_m < \eg
			\end{align*}
			מצד שני:
			\begin{align*}
				&a_m - a_{m + 1} + a_{m + 2} - a_{m + 3} + \cdots + a_n \\
				= &(a_m - a_{m + 1}) + (a_{m + 2} - a_{m + 3}) + \cdots + (a_{n - 2} - a_{n - 1}) + a_1 \ge a_n > 0 > - \eg
			\end{align*}
			לכן:
			\[ \sof{\sum_{k = m}^{n}(-1)^{k}a_k} = \sof{\sum_{k = m}^{n}(-1)^{k - m}a_k} < \eg \]
			\item אם $n - m$ אי־זוגי נקבל הוכחה דומה.
		\end{itemize}
	\end{proof}
	מההוכחה, ניתן להסיק:
	\[ \ml := \sumninf (-1)^{n}a_n \quad \quad \implies \forall n \in \N \co \sof{\ml - \sumnko (-1)^{n}a_n} \le a_{n + 1} \]
	מכאן, ש־$\sumninf \frac{(-1)^{n}}{n}$ מתכנס, ולא בהחלט. על טור כזה, אומרים שהוא \textit{מתכנס בתנאי}.

	למזלנו, ליאור הכין למעננו עוד קריטריונים מרתקים לשיעור, והם הכללה של קריטריון לייבניץ. האחד קריטריון דיריכלה והשני אבל.

	\subsubsection*{קריטריון אבל}
	תהאנה $\an, \bn$ סדרות. נניח כי:
	\begin{enumerate}
		\item $\bn$ מונוטונית (יורדת) (אבל לא בהכרח גבול $0$).
		\item נניח $\sumninf a_n$ מתכנס.
	\end{enumerate}
	אז $\sumninf a_n b_n$ מתכנס.

	לבית – יש להוכיח שקריטריון אבל נובע מקריטריון דיריכלה.

	\subsubsection{קירטריון דיריכלה}
	\begin{enumerate}
		\item $\bn$ מונוטונית (יורדת) וגבולה $0$.
		\item סדרת הסכומים החלקיים המתאימה ל־$\an$ חסומה (אבל לא בהכרח מתכנסת).
	\end{enumerate}
	תהאנה $\an, \bn$ סדרות. נניח כי:

אז $\sumninf a_n b_n$ מתכנס.

	\begin{proof}
		לכל $n \in \N$ נסמן $A_n = \sumnko a_k$. ידוע $\exists M > 0 .\, \forall n \in \N \co \sof{A_n} \le M$ (כי היא חסומה). בגלל ש־$\bn$ שואפת ל־0 אז $\exists N \in \N.\, \forall n \ge N \co \sof{b_n} < \frac{\eg}{2M}$. יהיו $n \ge m \ge N$. אז:
		\begin{multline*}
			\sum_{n = m}^{k} a_k b_k = \sum_{k = m}^{n}(A_k - A_{k - 1})b_k = \sum_{k = m}^{n}A_k b_k - \sum_{k =m}^{n}A_{k - 1}b_k = \sum_{k = m}^{n}A_k b_k - \sum_{k = m - 1}^{n - 1}A_k b_{k + 1} \\
			= A_n b_n + \sum_{k = m}^{n - 1}A_k b_k - \sum_{k = m}^{n - 1}A_k b_{k + 1} - A_{m - 1}b_m = \sum_{k = m}^{n - 1}\cl{A_k(b_k - b_{k + 1})} + A_nb_n - A_{m - 1}b_m
		\end{multline*}
		לכן (ניעזר בזה ש־$\bn$ מונוטונית יורדת):
		\begin{multline*}
			\sof{\sum_{k = m}^{n}a_kb_k} \le \sof{\sum_{k = m}^{n - 1}A_k(b_k - b_{k + 1})} + \sof{A_nb_n} + \sof{A_{m - 1}b_m} \le \sum_{k = m}^{n} \cl{\sof {A_k}(b_k - b_{k + 1})} + \sof{A_nb_n} + \sof{A_{m - 1}b_m} \\
			\le M(b_m - b_{n + 1}) + Mb_n + Mb_m < 2 \cdot M \frac{\eg}{2M} = \eg
		\end{multline*}
		לסכום הזה קוראים סכום אבל. אולי קוראים לזה דיריכלה אבל אבל הראשון שעשה את זה.
	\end{proof}

	\exe{האם הטור $\sumninf \frac{\sin n}{n}$ מתכנס בהחלט, בתנאי, או מתבדר? }
	רמז שאפשר להוכיח באינדוקציה:
	\[ \sumnko \sin(\ag + \bg k) = \frac{\sin \cl{\frac{b \bg}{2} \sin \cl{\ag + \frac{(n - 1)}{2}\bg}}}{\sin \cl{\frac{\bg}{2}}} \]
	נסמן $a_n = \sin n$ ו־$b_n = \frac{1}{מ}$. אפשר לדעת ש־$b_n$ מונוטונית יורדת שגבולה 0, ו־:
	\[ \sof{\sumnko a_k} \le \text{משהו שהמרצה לא בטוח לגביו} \le \frac{1}{\sin \frac{1}{2}} \]
	''יש כאן איזו טעות בנוסחה. בתרגיל בית תקבלו את זה כמו שצריך``.
	ואז זה זה מתכנס לפי דיריכלה. יש כאן שאלה, האם זה מתכנס בהחלט?
	\[ \sof{\frac{\sin n}{n}} = \frac{\sof{\sin n}}{n} \ge \frac{\sin  ^2 n}{n} = \frac{1 - \cos 2n}{2} = \underbrace{\frac{1}{2n}}_{\text{מתבדר (בסכום)}} - \underbrace{\frac{\cos 2n}{2n}}_{\text{מתכנס מדיריכלה}} \]
	מאריתמטיקה של גבולות, סיימנו.



	בהרצאה הזו נדבר עוד על טורים.

	\exe{בדקו את התכנסות הטור $\sumninf \sin(\pi \sqrt{n^2 + 1})$}\begin{proof}[תשובה]
		הטריק הוא להבין ש־$\sin(\pi \sqrt{n^2 + 1}) = (-1)^{n}\sin(\pi(\sqrt{n^2 + 1} - \pi n))$, לכל $n\in \N$. נוסף על כך מכפל בצמוד, $\frac{1}{\sqrt{n^2 + 1} + n} = \sqrt{n^2 + 1} - n$. כלומר:
		\[ \sumninf \sin(\pi \sqrt{n^2 + 1}) = \sumninf (-1)^{n}\sin\cl{\frac{\pi}{\sqrt{n^2 + 1} + n}} \]
		עוד נבחין ש־$\forall n \in \N \co 0 \le \frac{\pi}{\sqrt{n^2 + 1 } + n} \le \frac{\pi}{2}$ וגם מונוטוני יורד, כלומר $\sin \frac{\pi}{\sqrt{n^2 + 1} + n}$ מונוטוני יורד. מכך ש־$\forall x \ge 0\co \sin x \le x$ נובע ש־$0 \le \sin \cl{\frac{\pi}{\sqrt{n^2 + 1} + n}} \le \frac{\pi}{\sqrt{n^2 + 1} + n} \to 0$ ולפי סונדווייץ' $\sin \frac{\pi}{\sqrt{n^2 + 1} + n} \to 0$. לכן לפי קריטריון לייבניץ $\sum (-1)^{n}\sin(\pi(\sqrt{n^2 + 1} + n))$.
	\end{proof}

	\exe{תהא $\an$ סדרה חיובית. נניח כי $\sumninf a_n$ מתכנס. נסמן ב־$S_n$ את סדרת הסכומים החלקיים של $\an$. נראה כי $\sumninf (-1)^{n} \frac{S_n}{n}$ מתכנס. }\begin{proof}
		הבעיה היא שלייבניץ לא עובד כאן, כי $\frac{S_n}{n}$ לא בהכרח מונוטונית. לכל $n \in \N$ נגדיר $T_n = \sumnko (-1)^{k}S_k$. נטען שלכל $n \in \N$ קיימת קבוצה $I \subseteq [n]$ כך ש־$T_n = (-1)^{n}\sum_{i \in I} a_i$ (דרך לחסוך פירוק למקרים של זוגי/אי־זוגי). נוכיח את הטענה באינדוקציה.
	\begin{itemize}
		\item עבור $n = 1$ ניקח $I = \{1\}$ ונקבל $T_1 = -S_1 = -a_1 = (-1)^{1}\sum_{i \in I}a_i$.
		\item יהי $n \in \N$. נניח קיום $I \in \ps([n])$ כך ש־$T_n = (-1)^{n}\sum_{i \in I}a_i$. נגדיר $\hat I = [n] \setminus I$. נקבל:
		\[ T_{n + 1} = T_n + (-1)^{n + 1}S_{n + 1} = (-1)^{n}\sum_{i \in I}a_i + (-1)^{n + 1}\sum_{i = 1}^{n + 1}a_i = (-1)^{n}\sum_{i \in I}(\cancel{a_i - a_i}) + (-1)^{n + 1}\sum_{i \in \hat I} a_i = (-1)^{n + 1}\sum_{i \in \hat I}a_i \]
		וסיימנו את האינדוקציה.
	\end{itemize}
	מכאן שלכל $n \in \N$ נקבל $\sof{T_n} \le \sumnio a_i = S_n$. ידוע $S_n$ מתכנסת ולכן חסומה. $\frac{1}{n}$ מונוטונית יורדת ח־$0$ ולכן לפי קיטריון דיריכלה $\sumninf (-1)^{n}\frac{S_n}{n}$ מתכנס.
	\end{proof}

	שאלה: ומה קורה אם $\an$ לא בהכרח חיובית? נגדיר לכל $2 \le n \in \N$ ש־:
	\[ S_n = \frac{(-1)^{n}}{\ln n} \]
	לכל $n \in \N$ נגדיר $a_n = S_n - S_{n + 1}$. אז $S_n$ סדרת הסכומים החלקיים של $\an$. אז $S_n \to 0$ ולכן $\sumninf a_n$ מתכנס. אבל, $\sumninf (-1)^{n} \frac{S_n}{n} = \sumninf \frac{1}{n \ln n}$ מתבדר (כפי שהוכחנו בעבר).
	\subsection*{אסוציאטיביות}
	לעשות אסוציאטיביות של סכום זה כמו לבחור תת־סדרה של סדרת הסכומים החלקיים, ואז לסכום אותה (תחשבו על זה קצת).
	בניסוח של המרצה, תהא $a_n$ סדרה, ונסמן ב־$S_n$ את סדרת הסכומים החלקיים שלה. אז קיבוץ איברים בסכום פירושו הסתכלות על ת''ס של $S_n$. כלומר, נגדיר סדרה עולה של טבעיים $n_1 < n_2 < \cdots$ כך ש־$S_{n_j} = \sum_{\ml = 1}^{j}\sum_{k = n_{\ml - 1}}^{n_\ml} n_k$ והיינו רוצים ש־$S_{n_j}$.

	טענה: תהא $\an$ סדרה, נניח כי הטור $\sumninf$ מתכנס, אז לכל השמה של סוגריים על הסכום, הטור החדש מתכנס. \begin{proof}
		נסמן ב־$S_n$ את סדרת הסכומים החלקיים של $\an$. לכל השמה של סוגריים, סדרת הסכומים החלקיים המתאימה היא ת''ס של $S_n$ ולכן מתכנסת, לאותו הגבול של $S_n$.
	\end{proof}

	הכיוון השני לא נכון – זה שהצלנו לפצל לסוגריים ושדברים יתכנסו, לא אומר שאנחנו מתכנס בעצמנו (יידרש מאיתנו להתכנס מתחתחילה). לדוגמה עבור $a_n = (-1)^{n}$ יש לנו:
	\[ (-1 + 1) + (-1 + 1) + \cdots = -1 + 1 -1 + 1 \cdots = -1 + (1 - 1) + (1 - 1) + (1 - 1) + \cdots = -1 \]
	עם זאת, לכל $\an$ סדרה, ונניח כי קיימת השמה של סוגריים שבה:
	\begin{itemize}
		\item הטור המתאים מתכנס
		\item בתוך כל סוגריים, כל האיברים בעלי אותו הסימן
	\end{itemize}

	\begin{proof}
		השמת הסוגריים מגדירה ת''ס של סדרת הסכומים החלקיים $S_n$. קיים $\ml \in \R$ כך ש־$\lim_{k \to \infty} S_{n_k} = \ml$. יהי $\eg > 0$. קיים $K \in \N$ כך שלכל $k \ge K$ מתקיים $\sof{S_{n_k} - \ml} < \eg$. נתבונן ב־$N = n_K$. יהי $n \ge N$. ידוע $\lim_{t \to \infty} n_T = \infty$ ולכן קיים $t \in \N$ כך ש־$n_t \le n < n_{t + 1}$. ידוע $n \ge n_k$ לכן $t \ge K$. מכאן $\sof{S_{n_t} - \ml} < \eg$ וגם $\sof{S_{n_{t + 1}} - \ml} < \eg$. בהכרח $S_{n_{t + 1}}  - S_{n_t} = \sum_{j = n_t}^{n_t + 1}a_j$, ומההנחה זה סכום של איברים שווי סימן. בה''כ נניח שכולם חיוביים. אז:
		\[ \ml - \eg < S_{n_t} \le S_{n_t} + a_{n_t} + \cdots a_n \le S_{n_{t}} + a_{n_t + 1} + \cdots + a){n_{t + 1}} = S_{n_{t + 1}} < \ml + \eg \]
		סה''כ נקבל $\sof{S_n - \ml} < \eg$. לכן $S_n \to \ml$.
	\end{proof}

	\subsection*{קומטטיביות}
	אז איך ננסח במקרה של טור אינסופי קומטטיביות? באמצעות זיווגים/תמורות. תהא $\an$ סדרה ותהא $\sg \co \N_+ \to \N_+$ תמורה. אז $a_{\sg(n)}$ תקרא \textit{תמורה של $\an$}.

	\theo{תהא $\an$ סדרה מתכנסת. אז לכל $\sg \co \N_+ \to \N_+$, אז $\hat \ps(a_{\sg(n)}) = \hat\ps(a_n)$. }
	\rmark{סדרות זה סקאם. הסדר הוא סתם שטיק איטואיטיבי שלא באמת צריך. ההוכחה פשוטה, כי יש להן את אותה התמונה. }

	ומה לגבי טורים (כלומר תמורות של איברי הטור)? האם הטור של $\an$ ו־$a_{\sg(n)}$ מתכנסים לאותו הגבול? התשובה היא לא. ננסה להגדיר דוגמה קונקרטית. נגדיר $a_n = \frac{(-1)^{n}}{n}$, ונסמן $S_n = \sumninf a_n$. נגדיר:
	\[ \sg \co \N_+ \to \N_+ \quad \sg(n) = \begin{cases}
		4 \cdot \frac{n}{3} &  n \equiv 0 \\
		2 \cdot \frac{n + 2}{3} - 1 & n \equiv 1 \\
		4 \cdot \frac{n + 1}{3} - 2 & n \equiv 2
	\end{cases}\dequad \mod 3 \]
	לדוגמה:
	\begin{gather*}
		1 \mapsto 1 \quad 2 \mapsto 3 \quad 7 \mapsto 5 \quad 10 \mapsto 7 \\
		2 \mapsto 2 \quad 5 \mapsto 6 \quad 8 \mapsto 10 \quad 11 \mapsto 14 \\
		3 \mapsto 4 \quad 6 \mapsto 8 \quad 12 \mapsto 12 \quad 15 \mapsto 16
	\end{gather*}
	\lem{$\sg \co \N_+ \to \N_+$ תמורה}\begin{proof}
		לבית
	\end{proof}
	נסמן $\hat S_n = \sumnko a_{\sg(k)}$. נקבל:
	\begin{align*}
		S_{3n} &= \sum_{k = 1}^{3n}a_{\sg(k)} \\
		&= \sum_{\ml = 1}^{n}a_{3\ml - 2} + a_{3\ml - 1} + a_{3\ml - 1} \\
		&= \sum_{i = 1}^{n}(-1)^{2\ml - 1} \cdot \frac{1}{2\ml - 1} + (-1)^{4\ml - 2} \cdot \frac{1}{4\ml - 2} + (-1)^{4\ml} \cdot \frac{1}{4\ml} \\
		&= \sum_{\ml = 1}^{n} \frac{-1}{2\ml - 1} + \frac{1}{4\ml - 2} + \frac{1}{4\ml} \\
		&= \frac{1}{2}\sum_{\ml =1}^{n}\frac{-1}{2\ml - 1} + \frac{1}{2\ml} = \frac{1}{2}S_{2n} \to \frac{1}{2}S
	\end{align*}
	משום ש־$S \neq 0$ כבר הקיום של גבול חלקי שהולך ל־$\frac{1}{2}S$ מספיק לנו כדי לדעת ששתי הסדרות מתכנסות למקומות שונים. יתרה מכך, אפשר להראות שהוא מתכנס ל־$\frac{1}{2}S$ כי $\hat S_{3n + 1} = \hat S_{3n} + a_{\sg(3n + 1)}$ וכנ''ל עבור $\hat S_{3n + 2}$, ומאריתמטיקה של גבולות ובגלל ש־$a_n \to 0$ (וכן הגבולות החלקיים) וממשפט הכיסוי $\hat S$ מתכנסת ל־$\frac{1}{2}S$. ממש מצאנו סדרה שהתמורה שלה מתכנסת למקום אחר.

	(הסיבה ש־$S$ לא מתכנס ל־$0$, כי הוא תמיד מתחת ל־$0$, ולכן הוא bound away מ־$0$. עם זאת הוא בהכרח מתכנס מלייבניץ)

	טוב, אז קומטטיביות לא עובד. ננסה למצוא תנאים שבהם זה עובד.
	\theo{תהא $\an$ חיובית. נניח ש־$\sumninf a_n$ מתכנס. אז כל תמורה של הגבול מתכנסת לאותו הגבול. }\begin{proof}
		תהא $\sg \co \N_+ \to \N_+$ תמורה. נסמן $\sumninf a_n = \ml$. יהי $n \in \N$. נסמן $N = \max \Img \sg$. נקבל:
		\[ \sumnko a_{\sg(k)} \le \sum_{k = 1}^{N} a_k \le \ml \]
		מכאן ש־$\sumninf a_{\sg(n)}$ מתכנס, וכמו כן $\sumninf a_n \le \ml$ (כי סדרת הסכומים החלקיים של $a_{\sg(n)}$ מונוטונית עולה וחסומה ב־$\ml$). עכשיו אפשר לדבר על ערך ההתכנסות של התמורה ולסמן $\sumninf a_{\sg(n)} = m$. מכיוון ש־$\sg\op$ תמורה, נובע (אותו הטיעון כמו קודם, אבל הפוך):
		\[ \ml \le \sumninf a_n \le m \]
		לכן $m \le \ml$ וגם $\ml \le m$ ומכאן $\ml = m$.
	\end{proof}

	\theo{תהא $\an$ סדרה. נניח כי $\sumninf a_n$ מתכנס בהחלט. אז לכל תמורה $\sg$ של $a_n$, הטור המתאים מתכנס לאותו הסכום. }
	זה תרגיל לבית.
	\begin{Theorem}[משפט רימן]
		תהא $\an$ סדרה. נניח כי הטור $\sumninf a_n$ מתכנס בתנאי. אז לכל $-\infty \le \ag \le \bg \le + \infty$ (במובן הרחב) קיימת תמורה $\sg \co \N_+ \to \N_+$ כך ש־$S_n$ סדרת הסכומים החלקיים של $a_{\sg(n)}$, מקיימת:
		\[ \liminf S_n = \ag \quad \limsup S_n = \bg \]

		צימרמן למה יש לך swastika במחברת.
	\end{Theorem}
	\begin{proof}
		תהא $\an$ סדרה. 	נגדיר שתי סדרות:
		\[ p_n = \begin{cases}
			a_n & a_n \ge 0 \\
			0 & \other
		\end{cases} \quad q_n = \begin{cases}
			-a_n & a_n < 0 \\
			- & \other
		\end{cases} \]
		הם נקראים החלק החיובי והשלילי של $\an$. לכל $n \in \N$ מתקיים $a_n = p_n - q_n$ ו־$\sof{a_n} = p_n + q_n$. די קל להראות ש־$\sumninf a_n$ מתכנס בהחלט אמ''מ $\sumninf p_n$ ו־$\sumninf q_n$  מתכנסות, כאשר צד אחד טרוויאלי מאריתמטיקה. מהצד השני, אם $\sumninf \sof{a_n}$ מתכנס, אז $\sumninf p_n + q_n$ מתכנס, וממשפט $\sumninf a_n$ מתכנס ולכן $\sumninf p_n - q_n$ מתכנס, ואז $\sumninf p_n, q_n$ שניהם מתכנסים מאריתמטיקה.

		עתה, תהא $\an$ סדרה. נניח ש־$\sumninf a_n$ מתכנס בתנאי. אז $\sumninf p_n = +\infty$ וכן $\sumninf q_n = +\infty$ (מאי־התכנסות בהחלט) וגם $\limsi p_n = \limsi q_n = 0$ (מהתכנסות $a_n$).

		נראה את קווי ההוכחה למשפט רימן. לא נוכיח אותו עד הסוף. במקרה ש־$\ag \le \bg$ מספרים (ולא במובן הרחב), אז קיים $n_1$ כך ש־$\sum_{i = 1}^{n_1} p_n > \bg$, ו־$n_1$ מינימלי כזה (מהסדר הטוב בטבעיים). את האיברים $p_1, \dots p_{n_1}$ נכניס לתחילת הסדרה. באופן דומה הסכום של $\sumninf q_n =   \infty$, ולכן קיים $n_2 \in \N$ כך ש־$\sum_{i = 1}^{n_1} - \sum_{n = 1}^{m_1}q_1 < \ag$. נמשיך את התמורה ע''י $q_1 \dots q_{m_1}$. ``בשלב הרקורסיה'' יש לנו רישא של $a_{\sg(1)} \dots a_{\sg(n_1)}, a_{\sg(n_1 + 1)} \dots a_{\sg(n_1 + m_1)}, \dots a_{\sg(n_{k + 1})} \dots a_{\sg(n_k + 1) + 1} \dots a_{\sg(n_{k + 1} + m_{k + 1})}$. כמו בבסיס, $\sum_{n - n_{k + 1}+1}^{\infty}p_n = \infty$ לכן קיים $n_{k + 2}$ מינימלי כך ש־:
		\[ \sum_{n = 1}^{\mathclap{n_{k + 1} + m_{k + 1}}} a_{\sg(n_{k + 1} + m_{k + 1})} + \sum_{\mathclap{n = n_{k + 1} + 1}}^{n_{k + 2}} p_n > \bg \]
		ובאופן דומה קיים $m_{k + 2}$ מינימלי כך שכל הסיפור מלמעלה פחות $\sum_{n = m_{k + 1} + 1}^{m_{ k + 2}} q_n$ קטן מ־$\ag$.

		התמורה שתתקבל תעבוד.
	\end{proof}

	\section*{טורי חזקות}
	טור חֲזַקוֹת הוא הטור הפורמלי $\sum_{n = 0}^{\infty} a_nx^{n}$. השאלה היא איזה $x$־ים אני יכול להציב כך שהחרא יתכנס. זה טור חזקות סביב $0$, באופן כללי טור חזקות סביב $a \in \R$ הוא הסכום הפורמלי $\sum_{n = 0}^{\infty} a_n(x - r)^{n}$.

	''אי אפשר שווא נח על הח''ת ולכן יש חטף פתח. מי הביא את הסגול?``.

	''בנפרד זה חַזקות, בסומך זה חֵזקות. אבל פה זה סומך, לא נסמך``

	(פורמלי = מה שמגדיר אותו זה המקדמים, לא הפונקציה. כמו בלינארית)

	\theo{תהא $\an$ סדרה. יהי $x_0 \in \R$, ונניח כי $\sum_{n =0}^{\infty}a_n(x_0 -a)^{n}$ מתכנס. אז לכל $x \in \R$ אם $\sof{x - a} < \sof{x_0 - a}$ אז $\sum_{n = 0}^{\infty}a_n(x - a)^{n}$ מתכנס. }

	\begin{proof}
		יהי $x \in \R$. ניח $\sof{x - a} < \sof{x_0 - a}$. ואז $\sumnoinf a_n(x_0 - a)^{n}$ מתכנס ולכן $\limsi a_0(x_0 - a)^{n} = 0$. בפרט היא חסומה ע''י $M$. נקבל:
		\[ \sumnoinf \sof{a_n(x - a)^{n}} = \sumnoinf \sof{a_n(x_0 - a)}\sof{\frac{x - a}{x_0 - a}} \le M \sumnoinf \sof{\frac{x - a}{x_0 - a}}^{n} \]
		הטור מימין הוא טור גיאומטרי עם מנה קטנה מ־$1$ ולכן מתכנס. לכן $\sumnoinf a_n(x - a)^{n}$ מתכנס בהחלט ובפרט מתכנס.
	\end{proof}

	\begin{Theorem}[משפט אבל]
		תהא $\an$ סדרה ויהי $a \in \R$. קיים מספר יחיד $R \ge 0$ כך ש־
		\begin{enumerate}
			\item \hfil $\displaystyle \forall x \in (a - R, a + R) \co \sumnoinf a_n(x - a)^{n} \ \text{\en{converges}}$
			\item \hfil $\displaystyle x \notin [a - R, a + R] \co \sumnoinf a_n(x - a)^{n}  \ \text{מתבדר}$
		\end{enumerate}
		החלק הזה נקרא \textit{רדיוס ההתכנסות} של הטור, והתחום נקרא \textit{תחום ההתכנסות}.
	\end{Theorem}


	\textbf{תזכורת: משפט אבל}

	תהא $\an$ סדרה ויהי $a \in \R$. אז קיים ויחיד $R \in [0, +\inft]$ כך שלכל $x \in \R$:
	\begin{enumerate}
		\item אם $\sof{x - a} < R$ אז $\sumninf a_n(x - a)^{n}$ מתכנס בהחלט.
		\item אם $\sof{x - a} > R$ אז $\sumninf a_n(x - a)^{n}$ מתבדר.
	\end{enumerate}

	\textbf{תזכורת: קריטריון אבל}

	יהיו $\an, \bn$ סדרות. נניח $\sumninf a_n$ מתכנס ו־$b_n$ מונוטונית יורדת ומתכנסת. אז $\sumninf a_nb_n$ מתכנס.

	אבל לא נותן דרך למצוא את ה־$R$ הזה. בשביל זה יש את המשפט הבא, שהוא יותר קונסטרקטיבי.

	\subsubsection*{משפט קושי־הדמרד}
	\theo{תהא $\an$ סדרה ויהי $a \in \R$. נסמן $\og = \limsup \sqrt[n]{\sof{a_n}}$. אז:
		\begin{itemize}
			\item אם $\og = 0$, אז $R = + \infty$.
			\item אם $\og = + \infty$, אז $R = 0$.
			\item אחרת $R = \frac{1}{\og}$.
	\end{itemize}}
	(זה ה־$R$ היחיד מאבל)

	\begin{proof}
		\begin{itemize}
			\item נניח ש־$\og = 0$. יהי $x \in \R$. אז:
			\[ \limsup \sqrt[n]{\sof{a_n(x - a)^{n}}} = \limsup \sqrt[n]{a_n}\sof{x - a} = 0\sof{x - a} = 0 \]
			לפי מבחן השורש, הטור $\sumninf a_n(x - a)^{n}$ מתכנס בהחלט ובפרט מתכנס.
			\item נניח ש־$\og = +\infty$. יהי $x \in \R$, ונניח $x \neq a$. אז באופן דומה:
			\[ \limsup \sqrt[n]{\sof{a_n}\sof{x - a}^{n}} = + \infty \]
			ולכן הטור מתבדר. [ידוע רק שטור הערכים המוחלטים מתבדר, כלומר הטור לכאורה יכול להתכנס אבל לא בהחלט. בטורי חזקות נובע שגם הטור הרגיל מתבדר. צ.ל. בבית שבטורי חזקות התכנסות גוררת התכנסות בהחלט]
			\item נניח $\og \in (0, + \infty)$. יהי $x \in \R$. נניח $\sof{x - a} < \frac{1}{\og}$. אז מנימוקים דומים:
			\[ \limsup \sqrt[n]{\sof{a_n}\sof{x - a}^{n}} < 1 \]
			ולכן הטור $\sumninf a_n(x - a)^{n}$ מתכנס. אם $\sof{x - a} > R$ הטור מתבדר.
		\end{itemize}
	\end{proof}

	[למי שעשה בדידה 2] כל הנושא של פונקציות יוצרות – זה בדיוק טורי חזקות. הרי $f \co \R \to \R$ יוצרת את $\an$ כאשר:
	\[ \exists \dg > 0.\, \forall x \in \R \co \sof{x} < \dg \implies \sumninf a_nx^{n} = f(x) \]


	\section*{לנטוש את הבדידות}
	עתה נתחיל לדבר על פונקציות במשתנה רציף. קודם לכן – נעסוק קצת בטופולוגיה.
	\subsection*{קצת טופולוגיה}
	\defi{יהי $x \in \R$. לכל $\eg > 0$, הקטע $(x - \eg, x + \eg)$, יקרה \textit{סביבת $\eg$ של $x$}. }

	\rmark{נבחין ש־$(x - \eg, x + \eg) = \{y \in \R \mid \sof{x - y} < \eg\}$. קוראים לזה כדור פתוח. זה פשוט המקרה החד־ממדי של כדורים. }

	\defi{יהי $x \in \R$ ותהא $U \subseteq \R$, ויהי $x \in U$. אז $U$ תקרא \textit{סיבה של $x$} אם קיים $\eg > 0$ עבורו $U$ מכילה סביבת $\eg$ של $x$. }

	\defi{קבוצה $U$ תקרא \textit{פתוחה} כאשר היא סביבה של כל אחת מהנקודות שלה. }

	לדוגמה, $(0, 1)$ הוא קבוצה פתוחה.
	\begin{proof}
		יהי $x \in (0, 1)$. נסמן $\eg = \min\{x, 1 - x\}$. נתבונן ב־$\eg$, ידוע $x > 0 \land x < 1$ כלומר $\eg > 0$. עוד נבחין:
		\[ x + \eg \le x + 1 - x  = 1 \]
		לכן $(x - \eg, x + \eg) \subseteq (0, 1)$.
	\end{proof}
	דוגמה אחרת היא ש־$[0, 1)$ קבוצה לא פתוחה.
	\begin{proof}
		נתבונן ב־$0$. יהי $\eg > 0$. נתבונן ב־$-\frac{\eg}{2}$. אז:
		\[ (-\eg, \eg) \ni - \frac{\eg}{2} \notin [0, 1) \]
		סתירה לפתיחות.
	\end{proof}

	למעשה, קבוצת כל הסביבות (\textit{הטופולוגיה של $\R$}) נוצרת ע''י איחוד וחיתוך של כדורים פתוחים (\textit{הבסיס לטופולוגיה}). זו קבוצה סגורה לאיחוד וחיתוך.

	\defi{$A \subseteq \R$ תקרא \textit{סגורה} כאשר $\bar A$ פתוחה (עולם דיון $\R$). }

	\theo{$A$ סגורה אם היא סגורה סדרתית. }
	\begin{proof}
		\begin{itemize}
			\item[$\implies$]נניח $A$ קבוצה. תהא $\an$ סדרה מתכנסת. נניח $\forall n \in \N \co a_n \in A$. נסמן $\limsi a_n = \ml$. נניח בשלילה $\ml \in \bar A$. אז קיים $\eg > 0$ כך ש־$(\ml - \eg, \ml + \eg) \subseteq \bar A$ (כי $\bar A$ פתוחה). קיים $n \in \N$ כך שלכל $n \ge N$ מתקיים $\sof{a_n - \ml} < \eg$. בפרט $a_N \in (\ml - \eg, \ml + \eg) \subseteq \bar A$ בסתירה לכך ש־$a_N \in A$ ולכן $\ml \in A$.
			\item[$\impliedby$]נניח ש־$A$ סגורה סדרתית. יהי $x \in \bar A$. נניח בשלילה שלכל $\eg > 0$, מתקיים $(x - \eg, x + \eg) \nsubseteq \bar A$. לכל $n \in \N$, קיים $(x - \frac{1}{n}, x + \frac{1}{n}) \cap \bar A$ (זה מתקיים בפרט עבור $\eg = \frac{1}{n}$). לכל $n \in \N$, $a_n \in A$ וכן $\limsi a_n = x$. $A$ סגורה סדרתית, לכן $x \in A$ בסתירה. [למי שלא שם לב, בשביל הטיעון הזה צריך גם ארכימדיאניות שתלויה באקסיומת השלמות וגם את אקסיומת הבחירה].
		\end{itemize}\envendproof
	\end{proof}

	\defi{תהא $A \subseteq \R$. אז $x \in \R$ תקרא \textit{נקודת־סגור} של $A$, כאשר $\forall \eg > 0 \co (x- \eg, x + \eg) \cap A \neq \varnothing$ (כלומר כל סביבה של $x$ מכילה איבר מ־$A$)}
	לדוגמה, $1$ נקודת סגור של $[0, 1)$.
	\begin{proof}
		יהי $\eg > 0$. נסמן $r = \min \{\eg, 1\}$. נתבונן ב־$1 - \frac{r}{2}$. אז $1 \le 1$ לכן $1 - \frac{r}{2} \ge \frac{1}{2}$. כמו כן $r \ge 0$ לכן $1 - \frac{r}{2} < 1$. מכאן ש־$1 - \frac{r}{2} \in [0, 1)$. כמו כן:
		\[ 1 - \eg < 1 - \frac{r}{2} < 1 < 1 + \eg \implies 1 - \frac{r}{2} \in [0, 1) \cap (1 - \eg, 1 + \eg) \]
		כנדרש.
	\end{proof}

	\theo{$A$ סגורה אמ''מ כל נקודת סגור של $A$ נמצאת ב־$A$. }
	\begin{proof}
		\begin{itemize}
			\item[$\impliedby$]נניח $A$ סגורה. תהא $x$ נקודת סגור של $A$. נניח בשלילה ש־$x \in \bar A$. $\bar A$ פתותחה לכן $\exists \eg > 0 \co (x - \eg, x + \eg) \subseteq \bar A$, כלומר $(x - \eg, x +\eg) \cap A = \varnothing$ בסתירה. לכן $x \in A$.
			\item[$\implies$]תהא $x \in \bar A$. מההנחה, $x$ אינה נקודת סגור של $A$. אז קיים $\eg > 0$ כך ש־$(x - \eg, x + \eg) \cap A = \varnothing$, דהיינו $(x - \eg, x +\eg) \subseteq \bar A$. לכן $\bar A$ פתוחה, כלומר $A$ סגורה.
		\end{itemize}\envendproof
	\end{proof}
	\defi{$A \subseteq \R$ תקרא \textit{קומפקטית} כאשר $A$ סגורה וחסומה. }
	\theo{$A \subseteq \R$ קומפקטית אמ''מ לכל סדרה $\an$, אם לכל $n \in \N$, ל־$a_n$ יש ת''ס מתכנסת שגבולה ב־$\an$. }

	\defi{יהי $x \in \R$ ותהא $U$ סביבה של $x$. אז $U \setminus \{x\}$ נקראת \textit{סביבה נקובה} של $x$. }

	\defi{תהא $U \subseteq \R$. $x \in \R$ תקרא \textit{נקודת הצטברות} של $A$ כאשר לכל סביבה \textbf{נקובה} $U$ של $x$, מתקיים $U \cap A \neq \varnothing$. }
	אינטואיטיבית, אפשר להתקרב בסביבות נקובות כמה שבא לנו ל־$x$, אבל אסור לנו לגעת בו.

	בקורס שאנו למדנו, כמעט אך ורק נעבוד עם קטעים, ולא עם קבוצות פתוחות כלליות. זה לא בחומר של הקורס.

	אם נגדיר $U \subseteq \R$ סביבה של $+\infty$, כאשר קיים $a > 0$ כך ש־$[a, +\infty)$, ו־$U$ סביבה של $-\infty$ כאשר קיים $a > 0$ כך ש־$(-\inft, -a] \subseteq U$, אז לכל $\ml \in \R \cup \{\pm\inft\}$, נקבל שסדרה $\an$ \textit{שואפת ל־$\ml$} כאשר לכל סביבה $U$ של $\ml$, קיים $N \in \N$ כך ש־$a_n \in U$.

	חומר קריאה: Stephen Willard $\sim$ General Topology


	\subsubsection{מבוא – פונקציות של משתנה ממשי}
	\noti{בכל קונטקסט בפרק זה, $f \co A \to \R$ עבור $A \subseteq \R$ כלשהו. }
	\defi{\textit{התמונה של $f$} היא $\Img f := \{x \in \R\mid \exists a \in A \co f(a) = x\}$}
	\defi{\textit{התחום של $f$} הוא $\dom f = A$. }
	ניתן להגדיר מנה, כפל, מכפלה, חיבור, חיסור, כפל בקבוע של פונקציות, וכו'.
	\defi{$f$ תקרא \textit{חסומה} כאשר $\Img f$ חסומה. }
	\defi{$f$ תקרא \textit{מונוטונית עולה} כאשר $\forall x \le y \in A \co f(x) \le f(y)$}
	בדומה לסדרות, נגדיר \textit{עולה ממש}, \textit{יורדת} ו\textit{יורדת ממש}.

	\exe{תהא $A \subseteq \R$ ותהאנה $f, g \co A \to \R$ חסומות. אז $f + g$ חסומה ומתקיים:
	\[ \inf f + \inf g \le \inf (f + g) \le \sup (f + g) \le \sup f + \sup g  \]}

	\begin{proof}
		לכל $x \in A$, מתקיים $f(x) \le \sup f \land g(x) \le \sup g$. לכן $f(n) + g(n) \le \sup f + \sup g$ ומכאן $\sup f + \sup g$ חסם מלעיל של $f + g$, ובפרט $\sup(f + g) \le \sup f + \sup g$ (כי הסופרמום הוא חסם מלעיל מינימלי). האינפימום בדומה, והשוויון האמצעי ידוע על קבוצות. והשטיק של החסימה זו בדיחה שהמרצה לא טרח להוכיח.
	\end{proof}

	השוויונות לא הדוקים. לדוגמה $f(x) = \sinx , g(x) = -\sinx$, אז $\sup f + \sup g = 2$ בזמן ש־$\sup (f + g) = 0$.

	\subsubsection*{גבולות של פונקציות}
	\defi{תהא $f \co A \subseteq \R \to \R$, ותהא $x_0 \in \R$ נקודת הצטברות של $A$, ויהי $\ml \in \R$. נאמר כי $\ml$ הוא גבול של $f$ ב־$x_0$ כאשר:
	\[ \forall \eg > 0 .\, \exists \dg > 0 .\, \forall x\in A \co 0 < \sof{x - x_0} < \dg \implies \sof{f(x) - \ml} < \eg \]}

	זה לא עובד במובן הרחב. למעשה נצטרך לקחת כל קומבינציה של $\ml, x_0$ כאשר אחד באינסוף, אחד ממשי, והאחד במינוס אינסוף, וזה יגרור אותנו ל־9 הגדרות.

	לקבוצת הטבעיים של נקודת הצטברות אחת, היא $+\infty$. למעשה סדרות זה מקרה פרטי כאשר $A = \N$.

	למה דווקא נקודות הצטברות? כי ככה אנחנו לא מגדירים דברים עבור ``קפיצות'' ודברים מוזרים כאלו. נגיד עבור $\{2\} \cup [0, 1]$, לא נתעסק עם $2$, למרות שהיא נקודת סגור.

	\textbf{דוגמה. }נגדיר $f \co \R\to \R$ ע''י:
	\[ f(x) = \begin{cases}
		x^{2} & x \neq 2 \\
		8 & x = 2
	\end{cases} \]
	הוכיחו כי $4$ הוא גבול של $f$ ב־$2$.
	\begin{proof}
		יהי $\eg > 0$. נחפש $\dg$ בטיוטה. [טיוטה: בסוף נרצה ש־$\sof{x^2 - 4} < \eg$. נגרר $\sof{x - 2}\sof{x + 2} < \eg$. נרצה $\sof{x - 2}\sof{x + 2} < \dg < \eg$. נבחר $\dg < 1$ (כלומר ניקח מינימום בסוף), אז ידוע $2 - \dg < x < 2 + \dg$ כלומר $4 - \dg < x + 2 < 4 + \dg$. ואז $\sof{x + 2} < 5$. נסכם] נתבונן ב־$\dg - \min \{1, \frac{\eg}{5}\}$. יהי $x \in \R$. נניח $0 < \sof{x - 2} < \dg$. אז $x \neq 2$ ולכן $f(x) - x^{2}$. נקבל:
		\[ \sof{f(x) - 4} = \sof{x^2 - 4} = \sof{x - 2}\sof{x + 2} < \sof{x + 2} \dg \]
		ידוע $1 \le 2 - \dg < x < 2 + \dg \le 3$ לכן $0 < x + 2 \le 5$. מכאן $\sof{x + 2} \dg \le 5 \cdot \frac{\eg}{5} = \eg$ לכן $\sof{f(x) - 4}$.
	\end{proof}

	\theo{תהא $f \co A \subseteq \R\to \R$, ותהא $x_0 \in \R$ נקודות הצטברות של $A$. יהיו $\ml, m \in \R$. אם $\ml$ גבול של $f$ ב־$x_0$ וגם $m$ גבול של $f$ ב־$x_0$ אז $\ml = m$. }


	לבית: להשלים 8 הגדרות נוספות.

	\textbf{דוגמה: }פונקציית דיריכלה. חשובה בעיקר בגלל שהיא דוגמה נגדית ממש כיפית.
	\[ D \co \R\to \R \quad D(x) = \begin{cases}
		1 & x \in \Q \\
		0 & \other
	\end{cases} \]
	[למי שעשה בדידה] זה האינדיקטור של $\Q$ ב־$\R$.

	\theo{לכל $x_0 \in \R$, אין ל־$D$ גבול ב־$x_0$. }
	\begin{proof}
		נתבונן ב־$\eg = \frac{1}{2}$. יהי $\dg > 0$. בקטע $(x_0, x_0 + \dg)$, יש מספר רציונלי $x$ ומספר אי־רציונלי $y$. אז:
		\[ 1 = \sof{D(x) - D(y)} \le \sof{D(x) - \ml} + \sof{D(y) - \ml} \le 0.5 \]
		לכן $\sof{D(x) - \ml} \ge 0.5$ או ש־$\sof{D(y) - \ml} \ge 0.5$, כלומר $\ml$ אינו גבול של $D$ ב־$x_0$.
	\end{proof}

	\exe{נגדיר $f \co \R\to \R$ ע''י $f(x) = xD$ לכל $x \in \R$. כאשר $D$ פונקציית דיאיכלה. הראו כי ל־$f$ יש גבול ב־$x_0$ אמ''מ $x_0 = 0$}
	\begin{proof}\,
		\begin{itemize}
			\item[$\implies$]נניח $x_0 = 0$. יהי $\eg > 0$. נתבונן ב־$\dg = \eg$. יהי $x \in \R$. נניח $0 < \sof{x - 0} < \dg$. ידוע $\sof{D(x)} \le 1$ לכן $\sof{f(x) - 0} = \sof{xD(x)} = \sof{x}\sof{D(x)} < \dg \cdot 1 = \eg$. לכן $\lim_{x \to 0} f(x) = 0$.
			\item[$\impliedby$]נניח $x_0 \neq 0$. יהי $\ml \in \R$. נתבונן ב־$\eg = \frac{\sof{x_0}}{2}$. $x_0 \neq 0$ לכן $\eg > 0$. יהי $\dg > 0$. ב־$(x_0, x_0 + \dg)$ יש $x$ רציונלי ו־$y$ אי־רציונלי.
			\[ \sof{x} = \sof{f(y) - f(x)} \le \sof{f(y) - \ml} + \sof{f(x) - \ml} \]
			בקטע $(x_0 - \dg, x_0)$ יש $a$ רציונלי ו־$b$ אי־רציונלי. אז:
			\[ \sof{a} = \sof{f(b) - f(a)} \le \sof{f(b) - \ml} + \sof{f(a) - \ml} \]
			מתקיים ש־$\max\{\sof a, \sof x\} \ge \sof {x_0}$ ולכן:
			\[ \max\{\sof(f(a) - \ml), \sof{f(b) - \ml}, \sof{f(a) - \ml}, \sof{f(y) - \ml}\} \ge \frac{\sof {x_0}}{2} \]
			ולכן $\ml$ אינו גבול של $f$ ב־$x_0$.
		\end{itemize}\envendproof
	\end{proof}

	אם צריך דוגמה נגדית יותר עדינה מדיריכלה הדי כיאוטית, הכירו את פונקציית רימן.
	\defi{פונקציית רימן $R \co \R\to \R$ מוגדרת ע''י:
	\[ R(x) = \begin{cases}
		\frac{1}{n_x} & x \in \Q \\
		0 & \other
	\end{cases} \]
	כאשר $m_x, n_x$ הפירוק היחיד של $x \in \Q $ כך ש־$x = \frac{m}{n}$ וגם $\gcd(m, n) = 1$.
	}
	\theo{לכל $x_0 \in \R$, מתקיים $\lim_{x \to x_0} R(x) = 0$. }\begin{proof}
		יהי $x_0 \in \R$. ללא הגבלת הכלליות $x_0 \in [0, 1]$ (בשאר התחומים היא מתנהגת אותו הדבר). יהי $\eg > 0$. אז קיים $N \in\N$ כך ש־$\frac{1}{N} < \eg$. נבחין ש־:
		\[ \ccb{x \in [0, 1] \setminus \{x_0\} \mid R(x) \ge \frac{1}{n}} \subseteq \underbrace{\ccb{\frac{m}{n} \mid n \in \N^{+}, m \in \Z, m \le n \le N}}_{A} \]
		(בקבוצה מימין לא דרשנו שהשברים יהיו מצומצמים). הקבוצה $A$ סופית! כן נוכל לסמן $\dg = \min\{\sof{x_0 - x} \co  \in A\}$, והמינימום אכן יהיה קיים. אז $\dg > 0$. נתבונן ב־$\dg$. יהי $x \in [0, 1]$, נניח $0 < \sof{x - x_0} < \dg$. אז $x \notin A$ ולכן $\sof{R(x) - 0} < \frac{1}{N} = \eg$. לכן $\lim_{x \to x_0} R(x) = 0$.
	\end{proof}

	\theo{תהא $f \co A \subseteq \R\to \R$, ותהא $x_0 \in \R$ נקודת הצטברות של $A$. נניח כי עבור כל סדרה $\an$ המקיימת:
	\begin{enumerate}
		\item $\Img \an \subseteq A$
		\item $\forall n \in \N \co a_n \neq x_0$
		\item $\limsi a_n = x_0$
	\end{enumerate}
	את $f(\an)$ מתכנסת, אז קיים $\ml \in \R$ כך שלכל סדרה $\an$ המקיימת את 1-3, $\limsi f(a_n) = \ml$. }

	''קרני משהו מטריד אותך\en{?}`` ''בעיקר תרגיל בית 5. אבל כבר ביקשתי הארכה ל־3 ו־4 אז לא נעים לי``.

	כלומר – אם כל הסדרות שמקיימות את 1-3 מתכנסות לאנשהו, אז כולן מתכנסות לאותו הגבול.

	''נקבובית כזו``. ''סדרה נקובה\en{!}``.

	\begin{proof}
		תהאנה $\an, \bn$ סדרות המקיימות את 1-3. מההנחה $f(\an)$ מתכנסת, ונסמן את גבולה ב־$\ml$. באופן דומה $f(\bn) =: m$. נגדיר סדרה:
		\[ c_n = \begin{cases}
			a_{\frac{n}{2}} & n \in \Neven \\
			b_{\frac{n - 1}{2}} & n \in \Nodd
		\end{cases} \]
		נבחין ש־$c_n$ מקיימת את 1-3. לכן $f(c_n)$ מתכנסת, ממשפט הכיסוי $\ps(c_n) = \{\ml, m\}$, והיא מתכנסת, כלומר $m = \ml$.
	\end{proof}

	\subsubsection*{קיטריון היינה}
	$\ra$ ערימה של אנשים הוגים ''היינה`` במבטא גרמני מזויף כבד $\la$

	\theo{תהא $f \co A \subseteq \R \to \R$. תהא $x_0 \in \R$ נקודת הצטברות של $A$. ל־$f$ יש גבול ב־$x_0$ אמ''מ לכל סדרה $\an$, אם $\an$ מקיימת את 1-3 מהטענה הקודמת, $f(\an)$ מתכנסת. }

	מה זה אומר? גם עבור פונקציות במשתנה רציף, הסדרות מגלמות בתוכן את מה שאנחנו צריכים כדי להגדיר ולעבוד עם גבולות. המשפט הראשון אומר לנו שכל הסיפור הזה לא תלוי בנציג, מה שמאפשר לנו לטעון שהגבול הזה יחיד.
	\begin{proof}
		\begin{itemize}
			\item[$\impliedby$]נניח של־$f$ יש גבול ב־$x_0$, ונסמנו $\ml$. תהא $\an$ סדרה. נניח כי: (1) $\forall n \in\N \co a_n \in A$ (2) $\forall n \in \N \co a_n \neq x_0$ (3) $\limsi a_n = x_0$. יהי $\eg > 0$. קיים $\dg > 0$ כך ש־$\forall x \in A \co \sof{x - x_0} \in (0, \dg) \implies \sof{f(x) - \ml} < \eg$. לכן קיים $N \in \N$, כך שלכל $n \ge N$, $\sof{a_n- x_0} < \dg$. נתבונן ב־$N$ הזה. יהי $n \ge N$. אז $a_m \in A$ וגם $a_m \neq x_0$. ידוע $0 < \sof{a_n - x_0} < \dg$ לכן $\sof{f(a_n) - \ml} < \eg$ סיימנו.
			\item[$\implies$]נניח כי לכל סדרה $\an$ המקיימת 1-3, אז $f(\an)$ מתכנסת. מהטענה הקודמת, קיים $\ml \in \R$ כך שכל הסדרות המצחיקות האלו מקיימות $f(\an) \to \ml$. נניח בשלילה שהגבול $\lim_{x \to x_0}f(x_0) \neq \ml$. אז קיים $\eg > 0$ כך שלכל $\dg > 0$ קיים $x \in A$ כך ש־$0 < \sof{x -x_0} < \dg$, וגם $\sof{f(n) - \ml}\ge \eg$. לכל $n \in \N$ קיים $a_n \in A$ כך ש־$\sof{a_n - x_0} \in (0, \frac{1}{n})$ וגם $\sof{f(a_n) - \ml} \ge \eg$. לכל $n \in \N$, מתקיים $a_n \in A \land a_n \neq a_0$. כמו כן $\limsi a_n = x_0$. אבל $\forall n \in \N\co \sof{f(a_n) - \ml}\ge \eg$. לכן $\limsi f(\an) \neq \ml$ וסתירה.
		\end{itemize}\envendproof
	\end{proof}

	\rmark{זה עובד גם במובן הרחב. המרצה לא טרח להוכיח. }

	זו דרך נוחה להראות שלפונקציה \textit{אין} גבול בנקודה.
	\exe{נגדיר $f \co \R\setminus \{0\} \to \R$ ע''י $f(x) = \frac{1}{x}$. אז ל־$f$ אין גבול ב־$0$. }\begin{proof}
		נגדיר:
		\[ a_n = \frac{(-1)^{n}}{n} \]
		אז לכל $n \in \N$, $a_n \in \R\setminus \{0\}$. עוד נבחין ש־$a_n \neq 0$ ו־$\limsi a_n= 0$. אבל, $f(a_n) = (-1)^{n} \cdot n$, וזו סדרה חסרת גבולות, אפילו במובן הרחב.
	\end{proof}

	\exe{נגדיר $f \co \R\setminus \{0\}\to \R$ ע''י $f(x) = \sin \frac{1}{x}$. ל־$f$ אין גבול ב־$0$}\begin{proof}
		נגדיר $a_n = \frac{1}{\pi n}$. נגדיר $b_n = \frac{1}{\pi n + \frac{\pi}{2}}$. הן מקיימות את 1-3. נבחין ש־$f(a_n) =0 \land f(b_n) = (-1)^{n}$ וזה סתירה.
	\end{proof}
	\subsection*{אריתמטיקה של גבולות}
	\theo{תהאנה $f, g \co \R\to \R$ ותהא $X_0 \in \R$ נקודת הצטברות של $A$. (בסיכומים של שירי ואסף, הם יגידו ש־$f, g \co I \setminus \{x_0\} \to \R$. זה מקרה מאוד פרטי – הם עוסקים בקטעים בלבד, במקום בקבוצות פתוחות). נניח כי יהיו $\ml, m \in \R$, ונניח כי $\limxo f (x) = \ml \land \limxo g(x) = m$.
		\begin{enumerate}
			\item \hfil $\disty\forall \ag, \bg \in \R \co \limxo f(\ag g(x) + \bg g(x)) = \ag \ml + \bg m$
			\item \hfil $\disty\limxo f(x)g(x) = \ml m$
			\item \hfil $\disty m \neq 0 \implies (\exists \dg > 0 .\, \forall x \in A \co 0 < \sof{x - x_0} < \dg \implies g(x) \neq 0) \land \cl{\limxo \frac{f(x)}{g(x)} = \frac{\ml}{m}}$
		\end{enumerate}
	}

	\begin{proof}[הוכחת 3]
		נניח $m \neq 0$. ידוע $\limxo g(x) = m$. לכן קיים $\dg > 0$ כך שלכל $x \in A$, אם $0 < \sof{x - x_0} < \dg$ אז $\sof{g(x) - m} < \frac{m}{2}$. נתבונן ב־$\dg$. יהי $x \in A$. נניח $0 < \sof{x - x_0} < \dg$. אז $\sof{g(x) - m} < \frac{m}{2}$. לכן:
		\[ \sof{g(x)} \ge \sof{m} - \sof{g(x) - m} > \sof m - \frac{\sof{m}}{2} = \frac{\sof m}{2} \]
		סיימנו את החלק הראשון של המשפט. עתה נותר להוכיח ש־$\limxo \frac{f(x)}{g(x)} = \frac{\ml}{m}$. תהא $\an$ סדרה המקיימת $\Img \an \subseteq A \setminus \{x_0\}$ וכן $\limsi a_n = x_0$. לפני היינה $\limsi f(a_n) = \ml$, וכן $\limsi g(a_n) = m$. ידוע $m \neq 0$, ולכן לפי אריתמטיקת גבולות של סדרות, נקבל:
		\[ \limsi \frac{f(a_n)}{g(a_n)} = \frac{\ml}{m} \]
		לפי היינה (מהכיוון השני) סה''כ $\limxo \frac{f(x)}{g(x)} = \frac{\ml}{m}$.
	\end{proof}

	אפשר להכליל את החלק הראשון של שלוש (זו אותה ההוכחה) ולקבל את המשפט הבא:
	\theo{תהא $f \co A \subseteq \R \to \R$ ותהא $x_0 \in \R$ נקודת הצטברות של $A$. אם קיים ל־$f$ גבול סופי ב־$x_0$, קיימת סביבה נקובה של $x_0$ שבה $f$ חסומה. }

	''להיות חכם זה לדעת שעבניה זה פרי, ולהיות אינטליגנט זה לדעת לא להכניס אותו לסלט פירות``

	\rmark{המרצה רימה. לא בהכרח קיימת סביבה נקובה שמוכלת כולה ב־$A$. מההקשר, אפשר להבין שהכוונה ב''שבה`` היא כל נקודה שבתחום ההגדרה מוכלת בסביבה הזו. }


	\theo{תהאנה $f, g \in A \subseteq \R\to \R$ ונניח כי $A$ אינה חסומה מלעיל [כלומר אינסוף הוא נקודת הצטברות]. נניח כי $g$ חסומה וכי הגבול $\lim_{x \to \inft} f(x) = -\infty$. אז $\lim_{x \to \infty} f(x) + g(x) = -\infty$. }\begin{proof}
		$g$ חסומה לכן קיים $M > 0$ חסם שלה כך ש־$\forall x \in A \co \sof{g(x)} \le M$. [מה צ.ל.? שלכל $K > 0$ קיים $N > 0$ כך ש־$\forall x \in A$ אם $x > N$ אז $f(x) + g(x) < -K$] יהי $K > 0$. ידוע $\lim_{x \to \infty} f(x) = -\infty$ ולכן קיים $N > 0$ כך שלכל $ A \ni x > N$ מתרחש $f(x) < -K -M $. נניח $x > N$. אז $f(x) + g(x)< -K - M + M = -K$. לכן $\lim_{x \to \infty} f(x) + g(x) = -\infty$
	\end{proof}

	\theo{תהאנה $f, g \co A \subseteq \R\to \R$ ותהא $x_0 \in \R$ נקודת הצטברות של $A$. נניח כי קיימת סביבה נקובה של $x_0$ שבה לכל $x$, $f(x) \le g(x)$. נניח כי $\limxo f(x) = \infty$, אז $\limxo g(x) = +\infty$. }\begin{proof}
		מהנתון קיים $\dg > 0$ כך שלכל $x \in A$, אם $0  < \sof{x - x_0} < \dg$, אז $f(x) \le g(x)$. תהא $\an$ סדרה המקיימת $\Img \an \subseteq A \setminus \{x_0\}$ וכן $\limsi a_n = x_0$. קיים $N_1$ כך ש־$\forall n \ge N_1$ מתקיים $0 < \sof{a_n - x_0} < \dg$ (השתמשנו בהגדרת הגבול, כאשר ''ה־$\eg$ שלנו`` הוא $\dg$). ידוע $\limxo f(x) = \inft$ לכן לפי היינה $\limsi f(a_n) = \inft$. יהי $K>0$. קיים $N_2 \in \N$ כך ש־$\forall n \ge N_2 \co f(a_n) > K$. נתבונן ב־$N = \max\{N_1, N_2\}$. יהי $n \ge N$. אז $g(a_n) \ge f(a_n) > K$. לכן $\limsi f(a_n) = \infty$ ולכן מהיינה $\limxo g(x) = + \infty$.
	\end{proof}

	\theo{תהנה $f, g, h \co A \subseteq \R\to \R$, ותהא $x_0 \in \R$ נקודת הצטברות של $A$. נניח כי קיימת סביבה נקובה של $x_0$ שבה לכל $x$ $h(x) \le f(x) \le g(x)$. יהי $\ml \in \R$. נניח $\limxo g(x) = \limxo h(x) = \ml$. אז $\limxo f(x) = \ml$. }
	\begin{proof}
		לבית – להוכיח עם הגדרת קושי, ועם הגדרת היינה.
	\end{proof}

	\theo{תהאנה $f, g \co A \subseteq \R \to \R$ ותהא $x_0 \in \R$ נקודת הצטברות של $A$. יהיו $\ml, m \in \R$. נניח $\limxo f(x) = \ml \land \limxo g(x) = m$.
	\begin{enumerate}
		\item אם קיימת סביבה של $x_0$, כל שלכל $x$ בה $f(x) \le g(x)$ אז $\ml \le m$.
		\item אם $\ml < m$, אז קיימת סביבה נקובה של $x_0$ שבה לכל $x$ בה $f(x) < g(x)$.
	\end{enumerate}}
	\begin{proof}[הוכחה ל־$2$]
		נניח $\ml < m$. קיים $\dg_1 > 0$ כך ש־$\forall x \in A$, אם $0 < \sof{x - x_0} < \dg_1$ אז $\sof{f(x) - \ml} < \frac{m - \ml}{2}$. באותו האופן קיים $\dg_2 > 0$ כך ש־$\forall x \in A$, אם $0 < \sof{x - x_0} < \dg_2$ אז $\sof{g(x) - m} < \frac{m - \ml}{2}$. נתבונן ב־$\dg = \min\{\dg_1, \dg_2\}$. יהי $x \in A$. נניח $0 < \sof{x - x_0} < \dg$. אז:
		\[ f(x) < \ml + \frac{m - \ml}{2} = \frac{m + \ml}{2} = m - \frac{m - \ml }{2} < g(x) \]\envendproof
	\end{proof}
	ההוכחה של 1 מאוד דומה.

	להלן משפט העונה לשם ''משפט על גבולות והרכבה``.
	\theo{תהאנה $f \co A \subseteq \R\to B \subseteq \R, \ g \co B \to \R$. תהא \lxo. יהיו $y_0, \ml \in \R$. 	נניח כי:
	\begin{enumerate}
		\item $\limxo f(x) = y_0$
		\item קיימת סביבה נקובה של $x_0$ שבה לכל $f(x) \neq y_0$.
		\item $\lim_{x \to y_0} g(x) = \ml$
	\end{enumerate}
	אז $\limxo g \circ g(x) = \ml$.
	}
	\rmark{גם כאן המרצה עשה עברה – יש כאן הנחה ש־$y_0$ נקודת הצטברות של $B$. זה בסדר, כי באמצעות 1 ו־2 אפשר להראות ש־$y_0$ נקודת הצטברות של $B$ בכל מקרה. }

	יש גם ניסוח עם קטעים, פחות בעייתי:
	תהאנה $f \co I \setminus \{x_0\} \to J \setminus \{x_0\}$ וכן $g \co J \to \R$ (מקובל ש־$I, J$ מסמנים קטעים) ואז ממשיכים את שאר המשפט. אבל הניסוח הזה מקרה פרטי למדי. חשוב לדעת להתבטא כך כי ככה מלמדים בקורס ברגיל.

	\begin{proof}
		ראשית כל, נצטרך לוודא שכל החרא שלנו מוגדר היטב. לשם כך נראה שמ־1 ו־2 אכן נובע ש־$y_0$ נקודת הצטברות של $B$. יהי $\eg > 0$. קיים $\dg_1 > 0$ כך שלכל $x \in A$ אם $0 < \sof{x - x_0} < \dg_1$ ואז $\sof{f(x) - y_0} < \eg$. קיים $\dg_2 > 0$ כך שלכל $x \in A$, אם $0 < \sof{x - x_0} < \dg$ אז $f(x) \neq y_0$. נסמן $\dg  = \min \{\dg_1, \dg_2\}$. $x_0$ נקודת הצטברות של $A$, לכן קיים $x\in A$ כך ש־$0 < \sof{x - x_0} < \dg$. נתבונן ב־$f(x)$. אז $f(x) \in B$ וכן $0 < \sof{f(x) - y_0} < \dg$ ולכן $y_0$ נקודת הצטברות של $B$ וסיימנו.

		תהא $\an$ כך ש־$\Img \an \subseteq A \setminus \{x_0\}$ וגם $\limsi a_n = x_0$. אז לפי היינה $\limsi f(a_n) = y_0$. אז:
		\begin{enumerate}
			\item מתקיים $\Img f(a_n) \subseteq B$.
			\item כמעט תמיד $f(a_n) \neq y_0$ (זה מספיק להיינה. את זה גם צריך להוכיח, לבית).
			\item בהכרח $\limsi a_n = y_0$.
		\end{enumerate}
		לכן לפי היינה $\limsi g(f(a_n)) = \ml$, כלומר $\limsi (g \circ f)(a_n) = \ml$. לפי היינה $\limxo (f \circ g)(x_0) = \ml$.
	\end{proof}

	\subsection*{גבולות חד־צדיים}
	\defi{תהא $f \co A \to B$ פונקציה. תהי ת''ק $C \subseteq A$. נגדיר $g \co C \to B$ על־ידי $g(x) = f(x)$ לכל $x \in B$. $g$ נקראת \textit{הצמצום של $f$ ל־$C$} ומסמנים $g = f|_C$. }

	ניתן היה אפשר להגדיר תת־סדרה של $\an$ (בדידה) כצמצום של הסדרה לקבוצה אינסופית של טבעיים. הטרמינולוגיה הזו לא צריכה שהסדר על התחום יהיה סדר טוב. לכן נוכל להכליל אותה ל־$\R$.

	\theo{
	\begin{enumerate}
		\item תהא $A \subseteq \R$ ותהא $B \subseteq A$ ויהי $x_0 \in \R$. אם $x_0$ נקודת הצטברות של $B$ אז \lxo.
		\item תהא $A \subseteq \R$ ותהאנה $B, C \subseteq A \setminus \{x_0\}$ כך ש־$B \cup C = A$. אם \lxo אז $x_0$ נקודת הצטברות של $B$ או ש־$x_0$ נקודת הצטברות של $C$ (ה''או`` לא בהכרח xor).
	\end{enumerate}}
	\begin{proof}
		לבית
	\end{proof}

	מה שנעשה עכשיו על ת''קים ספציפיים, היה אפשר לעשות על כל תת־קבוצה.

	נגדיר את הסימון הבא לסיכום הזה בלבד (הוא לא מקובל). תהא $A \subseteq \R$ ותהא $x_0 \in \R$ נקודת הצטברות של $A$. נסמן $A_{x_0^{+}} := \{x \in A \mid x > x_0\} = A \cap (x_0, +\infty)$. נגדיר את $A_{x_0^{-}} := \{x \in A \mid x< x_0\} = A \cap (-\infty, x_0)$.

	מהמשפט הקודם, אם $x_0$ נקודת הצטברות של $A$, אז $x_0$ נקודת הצטברות של $A_{x_0^{+}}$ וכן של $A_{x_0^{-}}$.

	\defi{תהא $f \co A \subseteq \R \to \R$ ותהא $x_0$ נקודת הצטברות של $A$. אם $x_0$ נקודת הצטברות של $A_{x_0^{+}}$ וגם קיים הגבול של $f|_{A_{x_0}^{+}}$ ב־$x_0$, אז נאמר של־$f$ יש גבול מימין ב־$x_0$ ונסמנו $\lim_{x \to x_0^{+}} f(x)$. }
	\defi{תהא $f \co A \subseteq \R \to \R$ ותהא $x_0$ נקודת הצטברות של $A$. אם $x_0$ נקודת הצטברות של $A_{x_0^{-}}$ וגם קיים הגבול של $f|_{A_{x_0}^{-}}$ ב־$x_0$, אז נאמר של־$f$ יש גבול מימין ב־$x_0$ ונסמנו $\lim_{x \to x_0^{-}} f(x)$. }


	הכל כמובן במובן הרחב.

	\textbf{דוגמה. }נוכיח ש־$\lim_{x \to 0^{+}} \frac{1}{x} = +\infty$. \begin{proof}
		יהי $K > 0$. נתבונן ב־$\dg = \frac{1}{K}$. יהי $x > 0$ בסביבת הדלתא של $0$ (כלומר $x < \dg$), אז $x \in (0, \dg)$ ולכן $\frac{1}{k} > \frac{1}{\dg} = K$ מכאן $\lim_{x \to x_0^{+}} = + \infty$.
	\end{proof}

	\theo{תהא $f \co A \subseteq \R \to \R$ ותהא \lxo. יהי $\ml \in \R$ ונניח $\lim_{x \to x_0^{+}} f(x) = \ml$. אז אם $x_0$ נקודת הצטברות של $A_{x_0^{-}}$, אז $\lim_{x \to x_0^{-}}f(x) = \ml$. אם $x_0$ נקודת הצטברות של $A_{x_0^{+}}$, אז $\lim_{x \to x_0^{+}}f(x) = \ml$. }
	\rmark{אין באמת סיבה להסתכל על $A_{x_0^{+}}$ ו־$A_{x_0^{-}}$. אפשר היה להגדיר ''גבול חלקי`` על קבוצה כללית ולטעון את המשפט הזה. היינו מקבלים משפט הומורפי לכך שכל הגבולות החלקיים של פונקציה בדידה מתכנסים לגבול יחיד כאשר היא מתכנסת. עוד הערה: בד''כ לא יכתבו ''$x_0$ נקודת הצטברות של $A \cap (x_0, \infty)$`` אלא ''אם יש משמעות לגבול משמאל ס־$x_0$``. }

	\theo{תהא \gf ותהא \lxo. יהי $\ml \in \R$.
	\begin{enumerate}
		\item אם $x_0$ נקודת הצטברות של $A_{x_0^{+}}$ וכן נקודת הצטברות של $A_{x_0^{-}}$, אז $\lim_{x \to x_0^{-}} f(x) = \lim_{x \to x_0^{+}}f(x) = \ml$ גורר ש־$\limxo f(x) = \ml$.

		אחרת [כלומר $x_0$ אינה נקודת הצטברות של אחת מהקבוצות]:
		\item אם $x_0$ נקודת הצטברות של $A_{x_0^{-}}$ אז $\lim_{x \to x_0^{-}} = \ml$ גורר $\limxo f(x) = \ml$. [כלומר, אם אני יכול להגיע ל־$x_0$ רק מהצד השלילי – זה יקבע את הגבול]
		\item אם $x_0$ נקודת הצטברות של $A_{x_0^{+}}$ אז $\lim_{x \to x_0^{+}} = \ml$ גורר $\limxo f(x) = \ml$. [כלומר, אם אני יכול להגיע ל־$x_0$ רק מהצד החיובי – זה יקבע את הגבול]
	\end{enumerate}}

	''הוא ריחם על היאור, על החול במדבר... אבל לסלע הוא נתן זאפטה``

	נתחיל מלהוכיח את המשפט הקודם. \begin{proof}
		נניח ש־$x_0$ נקודת הצטברות של $A_{x_0^{-}}$. יהי $\eg > 0$. ידוע $\limxo f(x) = \ml$ לכן קיים לנו $\dg > 0$ כך שלכל $x \in A$ אם $0 < \sof{x - x_0} < \dg$ אז $\sof{f(x) - \ml} < \eg$. נתבונן ב־$\dg$. יהי $x \in A$ ונניח $x < x_0$. אז בפרט $0 < \sof{x - x_0} < \dg$ כלומר $\sof{f(x) - \ml} < \eg$ לכן $\lim_{x \to x_0^{-}} = \ml$.

		החלק השני (החיובי) – בדומה. ובכך סיימנו.
	\end{proof}

	עכשיו נחזור להוכיח את המשפט האחרון.
	\begin{proof}[הוכחת 1]
		נניח $x_0$ נקודת הצטברות של $A_{x_0^{-}}$ וגם $x_0$ נקודת הצטברות של $A_{x_0^{+}}$. נניח שהגבול משמאל ומימין שניהם $\ml$. יהי $\eg >0$. קיים $\dg_1 > 0$ כך שלכל $x \in A$ אם $x_0 < x < x_0 + \dg$ אז $\sof{f(x) - \ml} < \eg$. קיים $\dg_2 > 0$ כך שלכל $x \in A$ אם $x_0  - \dg < x < x_0$ אז $\sof{f(x) - \ml} < \eg$. נתבונן ב־$\dg := \min \{\dg_1, \dg_2\}$. יהי $x \in A$. נניח $0 < \sof{x - x_0} < \dg$. אז $x_0 < x < x_0 + \dg$ או $x_0 - \dg < x < x_0$. לכן $\sof{f(x) - \ml} < \eg$.
	\end{proof}
	\begin{proof}[הוכחת 2]
		נניח $x_0$ נקודת הצטברות של $A_{x_0^{-}}$ וגם $x_0$ אינה נקודת הצטברות של $A_{x_0^{+}}$. נניח $\lim_{x \to x_0^{-}} f(x) = \ml$. קיים $\dg_1 > 0$ כך ש־$A \cap (x_0, x + \dg_1) = \varnothing$. ידוע $\lim_{x \to x_0^{-}} f(x) = \ml$ לכן קיים $\dg_2 > 0$ כך שלכל $x \in A$, אם $x_0 - \dg_2 < x< x_0$ אז $\sof{f(x) -\ml} < \eg$. מכאן ממשיכים כמו ההוכחה הקודמת.
	\end{proof}

	\subsection*{קריטריון קושי לקיום גבול של פונקציה}
	\theo{תהא \gf ותהא \lxo. ל־$f$ יש גבול סופי ב־$x_0$ אמ''מ לכל $\eg > 0$, קיים $\dg > 0$, כך שלכל $x, y \in A$ אם $0 < \sof{x - x_0} < \dg$ וגם $0 < \sof{y - x_0} < \dg$ אז $\sof{f(x) - f(y)} < \eg$. }
	ההוכחה זה פחות או יותר היינה עם קושי.

	\subsection*{רציפות}
	רציפות וגזירות אלו שני המושגים שהחלו את החדו''א. בימים של לגראנג', ניוטון ולייבניץ הגדירו באמצעות זה שהפונקציה סימפטית מספיק ואפשר לצייר אותה על דף. ההגדרה הפורמלית היא \textbf{תכונה לוקאלית} – היא מוגדרת בעבור נקודה, לא בעבור כל הפונקציה. ישנן גם תכונות גלובליות, כמו ''בכל נקודה לפונקציה יש גבול`` או ''הפונקציה רציפה בכל התחום``. ההגדרה האינטואיטיבית של רציפות היא תכונה גלובלית.

	\defi{תהא \gf ותהא $x_0 \in A$. נאמר ש־$f$ רציפה ב־$x_0$ אם:
	\[ \forall \eg > 0.\, \exists \dg > 0.\ \forall x \in A \co \cl{\sof{x - x_0} < \dg} \implies \sof{f(x) - f(x_0)} < \eg \]}

	\rmark{כדי לדבר על רציפות בנקודה, חייבים לדבר על נקודה בתחום ההגדרה של הפונקציה. לא מספיקה נקודת התכנסות. מכאן גם, שאם יש חור בתחום ההגדרה, זה לא אומר שהפונקציה לא רציפה. לדוגמה, סדרות רציפות בכל נקודה. }

	''לקחתי את העפרון ודחפתי נקודות קצת על הגרף, וזהו! הכל רציף!!``` $\sim$ פיזיקאי כועס

	\theo{תהא \gf ותהא $x_0 \in A$. אם \lxo, אז $f$ רציפה ב־$x_0$ אמ''מ $\limxo f(x) = f(x_0)$. }

	כשמדברים על קטעים, המשפט הזה פשוט מספק הגדרה שקולה. זה לא עובד יותר כשיש נקודות מבודדות.

	''הוא נחנק, אבל הוא בסדר?``
	\begin{proof}
		נניח ש־\lxo.
		\begin{itemize}
			\item[$\impliedby$]נניח $f$ רציפה ב־$x_0$. יהי $\eg > 0$. מהגדרת הרציפות קיים $\dg > 0$ כך שלכל $x \in A$ אם $\sof{x - x_0} < \dg$ אז $\sof{f(x) - f(x_0)} < \eg$. נתבונן ב־$\dg$. יהי $x \in A$. נניח $0 < \sof{x - x_0} < \dg$. בפרט $\sof{x - x_0} < \dg$ ולכן $\sof{f(x) - f(x_0)} < \eg$ ומכאן $\limxo f(x) = f(x_0)$.
			\item[$\implies$]נניח $\limxo f(x) = x_0$. נוכיח שהיא רציפה ב־$x_0$. יהי $\eg > 0$. מהגדרת הגבול קיים $\dg > 0$ כך שלכל $x \in A$ אם $0 < \sof(x - x_0) < \dg$ אז $\sof{f(x) - f(x_0)} < \eg$. נתבונן ב־$\dg$. יהי $x \in A$. נניח $\sof{x - x_0} < \dg$. אם $x = x_0$ אז $0 = \sof{f(x) - f(x_0)} < \eg$. אחרת $0 < \sof{x - x_0} < \eg$ ולכן $\sof{f(x) - f(x_0)} < \eg$ וסיימנו.
		\end{itemize}\envendproof
	\end{proof}


	\textbf{הבחנה: }תהא \gf ונניח $x_0 \in A$ נקודת הצטברות של $A_{x_0^{+}}$ וכן של $A_{x_0^{-}}$. אז $f$ רציפה ב־$x_0$ אמ''מ מתקיימים שלושת התנאים הבאים:
	\begin{enumerate}
		\item קיים ל־$f$ גבול סופי ב־$x_0$ משמאל, וקיים ל־$f$ גבול סופי ב־$x_0$ מימין
		\item שני הגבולות להלן שווים
		\item שני הגבולות להלן שווים ל־$f(x_0)$
	\end{enumerate}

	(זה בדיוק כמו להגיד את מה שכתוב במשפט למעלה)

	למה זה מנוסח כזה פרגמטי (עם פ' רפה)? כי לפעמים יעניין אותנו ''עד כמה $f$ רציפה בנקודה``.

	\subsubsection*{מיון נקודות רציפות}
	\defi{תהא \gf ותהא $x_0 \in \R$. נניח ש־$f$ אינה רציפה בה. [מכאן, שבהכרח היא נקודת הצטברות – כי נקודה שאיננה נקודת הצטברות, היא רציפה. לכן אפשר לדבר על הגבול]. אז [הדוגמאות ל־$x_0 = 0$]:
		\begin{itemize}
			\item אם 1-2 מתקיים (מהמיון לעיל) אז $x_0$ תקרא \textit{אי־רציפות סליקה}. \textit{לדוגמה: }
			\[ f(x) = \begin{cases}
				x^{2} & x \neq 0 \\
				67 & x = 0
			\end{cases} \]
			\item אחרת, אם רק 1 מתקיים, $x_0$ תקרא \textit{אי־רציפות מסוג ראשון}. \textit{לדוגמה: }
			\[ f(x) = \begin{cases}
				x^{2} & x > 0 \\
				x^{2} + 67 & x \le 0
			\end{cases} \]
			\item אחרת, רק 2 מתקיים, ו־$x_0$ תקרא \textit{אי־רציפות מסוג שני}. \textit{לדוגמה: }פונקציית דיריכלה, $\frac{1}{x}$.
	\end{itemize}}
	מה המשמעות של אי־רציפות סליקה? שהפונקציה פחות או יותר רציפה בנקודה הזו, אבל ספציפית הנקודה הזו קופצת.

	\theo{תהא $f \co I \to \R$ מונוטונית עולה. אז לכל $x_0 \in I$, יש ל־$f$ גבול סופי משמאל ב־$x_0$ וגם גבול סופי מימין. }
	זה למעשה משפט וויראשטראס בעבור סדרות.
	\begin{proof}
		נסמן $A = \sup\{f(x) \mid x < x_0\}$. לכל $a \in A$, מתקיים $a \le f(x_0)$ ומהמונוטוניות של $f$. לכן $A$ חסומה. $A \neq \varnothing$ כי $x_0$ בתוך הקטע. לכן קיים ל־$A$ חסם עליון. נסמן $\ml = \sup A$. יהי $\eg > 0$. אז קיים $x < x_0$ כך ש־$f(x) > \ml - \eg$. נתבונן ב־$\dg  = x_0 - x$. יהי $x = x_0 - \dg < y < x_0$. אז $\ml - \eg < f(x) \le f(y) \le \ml < \ml + \eg$. לכן $\sof{f(y) - \ml} < \eg$. מכאן $\lim_{x \to x_0^{-}} f(x) = \ml$. בדומה יש ל־$f$ גבול מימין ב־$x_0$.
	\end{proof}
	לפונקציה מונוטונית יש רק נקודות רציפות מסוג ראשון. מכאן שיש רק כמות בת־מנייה של נקודות רציפות.
		\subsection*{המשך רציפות}
	\textbf{אריתמטיקה של רציפות}
	מייבא אוטומאטית הכל מאריתמטיקה של גבולות פונקציות.
	\theo{תהאנא $f, g \co A \subseteq \R \to \R$ ותהא $x_0 \in \R$. נניח כי $f$ רציפה ב־$x_0$ וכן $g$ רציפה ב־$x_0$. אז:
	\begin{itemize}
		\item $f\pm g$ רציפה ב־$x_0$
		\item $f \cdot g$ רציפה ב־$x_0$.
		\item אם $g(x_0) \neq 0$ אז $\frac{f}{g}$ רציפה ב־$x_0‏$.
	\end{itemize}}

	\theo{תהאנה $f \co A \to B$ ו־$g \co B \to \R$, ותהא $x_0 \in A$. נניח כי $f$ רציפה ב־$x_0$ ו־$g$ רציפה ב־$f(x_0)$. אז $g \circ f$ רציפה ב־$x_0$. }

	\textbf{דוגמאות לפונקציות רציפות: }
	\begin{itemize}
		\item פולינומים (מראים שהזהות והקבועה רציפות, ואז מאריתמטיקה סיימנו).
		\item הפונקציות הטריגונומטריות רציפות בכל נקודה בה הן מוגדרות.
		\item הפונקציות הטריגונומטריות ההפוכות רציפות בכל נקודה בה הן מוגדרות.
		\item הפונקציות המעריכיות רציפות ב־$\R$ (מהיינה וממשפט קודם שהגדיר היטב חזקה).
		\item לכל $1 \neq a > 0$ הפונקציה $\log_a x$ רציפה ב־$(0, \infty)$.
		\item הפונקציה $\sof x$ רציפה בכל $\R$.
	\end{itemize}

	\textbf{גבולות חשובים}
	\theo{\hfil $\disty \lim_{x \to 0}\frac{\sinx}{x} = 1$}
	\begin{proof}[הוכחה אבל חצי כח]
		לא באמת אני יכול להעתיק כי יש כאן מעגל היחידה ודברים שאין לי כח להעתיק. ההוכחה לא פורמלית בכל מקרה. זו הוכחה מאוד סטנדרטית שיצא לי לראות בעבר ואני משוכנע שתוכלו למצוא הוכחות באינטרנט. שימו לב שלופיטל זה טיעון מעגלי. עקרונית מראים על מעגל היחידה באמצעות טיעונים גיאומטריים לא מוגדרים היטב על משולש עם זווית $x_{\mathrm{rad}}$ על המעגל, ש־$\sinx \le x \le \tanx$ ומכאן $1 \le \frac{x}{\sinx} \le \frac{1}{\cosx}$ וידוע מרציפות $\limsi \frac{1}{\cosx} = 1$ ומסנדוויץ' סיימנו.
	\end{proof}

	\theo{\hfil $\disty \limz \frac{\ln(1 + x)}{x} = 1$}
	\begin{proof}
		די בקלות. לכל $x \in (-1, \infty) \setminus \{0\}$ נקבל:
		\[ \frac{\ln(1 + x)}{x} = \ln\cl{\cl{1 + x}^\frac{1}{x}} \]
		מסדרות + היינה:
		\[ \limz \cl{(1 + x)^{\frac{1}{x}}} = e \]
		הסלנג הוא ''להכניס את הגבול פנימה``, אבל זה רציפות והרכבה:
		\[ \limz \ln\cl{(1 + x)^{\frac{1}{x}}} = \ln e = 1 \]
	\end{proof}
	\theo{\hfil $\disty \frac{e^{x} - 1}{x} = 1$}
	\begin{proof}
		נעשה מעברים אלגברים:
		\[ \frac{e^{x} - 1}{x} = \frac{e^{x}}{\ln(e^{x} - 1 + 1)} \]
		נציב $t = e^{x} - 1$ (בפועל, משמעו הרכבה שחוקית רק מרציפות $\ln$):
		\[ = \frac{\ln(1 + t)}{t} = 1 \]
		תוך שימוש בסעיף הקודם.
	\end{proof}

	לבית חשבו את הגבול $\limsi \frac{(1 + x)^{\ag} - 1}{x}$.

	\subsection*{תכונות גלובליות של פונקציות רציפות}
	\defi{פונקציה $f$ היא \textit{רציפה} אם היא רציפה בכל נקודה. }
	\theo{תהא  $f \co A \to \R$. אז $f$ רציפה אמ''מ לכל קבוצה פתוחה $V \subseteq \R$ קיימת קבוצה פתוחה $U \subseteq \R$ כך ש־$f\op(V) = U \cap A$. }
	\begin{proof}
		\begin{itemize}
			\item[$\implies$]תהא $V \subseteq \R$. תהא $x \in f\op(V)$. אחרת $f(x) \in V$ לכן לא קיים $\eg > 0$כך ש־$f(x)- \eg, f(x) + eg$. ידוע $f$ רציפה ב־ $x$ (מהנתון). לכן קיים $\dg_x > 0$ כך שלכל $y \in (x - \dg_x), d + \dg_x$ מתקיים $f(y) \in (x - \dg_x, x + \dg_x)$ לכן $(x - \dg_x, x + \dg_x) \cap A \subseteq f\op(V)$.

			נגדיר:
			\[ U = \bigcup_{\mathclap{x \in f\op(V)}}(x - \dg_x, x + \dg_x) \]
			נבחין ש־$U$ פתוחה שכן היא איחוד של קבוצות מבסיס הטופולוגיה. כמו כן לכל $x \in f\op(V)$ מתקיים $(x - \dg_x, x + \dg_x)$ לכן $U \cap A \subseteq f\op(V)$. בנוסף מהגדרת האיחוד, $f\op(V) \subseteq UA \cap A$. לכן $U \cap A = f\op(V)$.
			\item[$\impliedby$]נניח שלכל $V$ פתוחה קיימת $U \subseteq \R$ פתוחה כך ש־$f\op(A)(V) = U \cap A$. יהי $x \in A$. יהי $\eg > 0$. $(x - \eg, x + \eg) \subseteq \R$ פתוחה ולכן קיימת $U \subseteq \R$ פתוחה כך ש־$U \cap A = f\op((f(x) - \eg, f(x) + \eg))$. יהי $x \in U \cap A$ לכן $x \in U$ ולכן לא קיים $\dg > 0$ כך ש־$(x - \dg, x + \dg) \subseteq U$. לכל $y \in A$ אם $\sof{y - x} < \dg$ אז $\sof{\sof{y} - \sof{x}} < \eg$. לכן $f$ רציפה וסיימנו.
		\end{itemize}
	\end{proof}

	\defi{תהא $f \co I \to \R$ כאשר $I$ קטע. נאמר כי $f$ מקיימת \textit{תכונת דרבו} כאשר לכל $a, b \in R$ כך ש־$a < b$, לכל $\lg \in \R$ בין $f(a) \le \lg \le f(b)$. קיים $c \in [a, b]$ כך ש־$f(x) = \lg$. }
	\begin{Theorem}[משפט ערך הביניים]
		פונקציה רציפה מקיימת את תכונת דרבו.
	\end{Theorem}
	\begin{proof}
		יהיו $a, b \in I$ ונניח ש־$a< b$. יהי $\lg \in \R$ כך ש־$f(a) \le \lg \le f(b)$. נבנה סדרת קטעים ברקורסיה: $a_1 =a, b_1 = b$ ואז צעד:
		\[ \begin{cases}
			a_{n + 1} = \frac{a_n + b_n}{2}, b_{n + 1} = b_n & f\cl{\frac{a_n  + b_n}{2}} \le \lg \\
			a_{n  + 1} = a_n \land b_n = \frac{a_n + b_n}{2} & f\cl{\frac{a_n + b_n}{22}} > \lg
		\end{cases} \]
		\item
		לכל $n \in \N$, נקבל $\sof{b_n - a_n} = \frac{b - a}{2^{n - 1}}$. לכל $n \in \N$ נקבל $f(a_n) \le \lg \le f(b_n)$ (אינדוקציה). ידוע $\limsi \frac{b - a}{2^{n - 1}} = 0$. לפי קנטור קיימת $ c\ni \R$כך ש־$\bigcup_{n = 1}^{\infty}[a_n, b_n] = \{c\}$. לכן $\limsi a_n = c$. מרציפות הגבול $\limsi a_n = c$ ומרציפות $\limsi I\lim f(a))$ לכל $n \in \N$, $f(a) \le \lg$ ולכן $f(x) \le \lg$. באופן דומה $f(c) \ge \lg$ ולכן $f(x) = \lg$.
	\end{proof}
	\begin{proof}[הוכחה נוספת]
		אחרי יהיו יהי תהיינה, אם $f(a) = \lg$ סיימנו. אחרת $f(a) < \lg$. נגדיר $A = \{x \in [a, b] \co f(x) < \lg\}$. אז $A$ לא ריקה כי $f(a) < \lg$ כלומר $\sup A$ קיים מאקסיומת השלמות. נניח בשלילה ש־$f(\ag) < \lg$. מרציפות $f$ קיים $\dg > 0$ כך שלכל $x \n (\ag - \dg, \ag + \dg)$, מתקיים $\sof{f(x) - f(\ag)} < \frac{num}{\lg f(x2)}$ נובע $f(x) < \lg$ בסתירה למינימליות הסופרמום. מהצד השני נוכל להפעיל ותו הטיעון ההפוך. לכן $f(\ag) = \lg$.
	\end{proof}
	\rmark{זה לא אמ''מ. להלן דוגמאות לפונקציות לא רציפות שמקיימות את תכונת ערך הביניים: }
	\begin{itemize}
		\item \textbf{פונקציית צימרמן: }בהינתן $r$, נגדיר שהיא תחזיר את הגבול של הממוצע החשבוני של הספרות במידה והוא קיים, אחרת $0$.
		\item \textbf{פונקציה סימפטית מספיק: } $\sin \frac{1}{x}$ (שמחזירה $0$ ב־$0$ בשביל נוחות). היא מקיימת דרבו אך אינה רציפה כי אין לה גבול ב־$0$.
	\end{itemize}

	\begin{Theorem}[משפט ווירשטראס (עוד אחד)]
		תהא $f \co A \to \R$ רציפה. אם $A$ קומפקטית (סגורה וחסומה) אז $f$ חסומה ומשיגה את חסמיה (יש לה מינימום ומקסימום).
	\end{Theorem}
	\begin{proof}[חלק ראשון]
		נניח בשלילה ש־$f$ אינה חסומה. אז לכל $n \in \N$ קיים $x_n \in A$ כך ש־$\sof{f(x_n)} > n$. $A$ [הערה: $x_n$ מוגדרת היטב כי קיים יחס סדר טוב על הטבעיים] חסומה ולכן $x_n$ חסומה. יש לה ת''ס $x_{n_k}$ מתכנסת. נסמן את גבולה $x_0$. $A$ סגורה ולכן $x_0 \in A$ (סגירות סדרתית). לכן $\lim_{k \to \inft} x_{n_k} = x_0 \in A$ ומרציפות $\lim_{k \to \inft} f(x_{n_k}) = f(x_0) \in \R$ בסתירה לכך ש־$\limsi \sof{f(x_n)} = \inft$. לכן $f$ חסומה.
	\end{proof}
	\begin{proof}[חלק שני]
		ידוע $f$ חסומה ולכן ניתן לסמן $f = \sup f(A)$. לכל $n \in \N$ קיים $y_n \in f(A)$ כך ש־$M - \frac{1}{n} \le y_n \le M$. לכל $n \in \N$ קיים $x_n \in A$ כך ש־$f(x_n) = y_n$. $A$ חסומה ולכן $x_n$ חסומה ומכאן שקיימת $x_{n_k}$ מ־BW שמתכנסת. נסמן גבולה $x_0$. $A$ סגורה ולכן $x_0 \in A$. מכאן ש־$M = \lim_{k \to \infty} y_{n_k} = \lim_{k \to \inft} f(x_{n_k}) = f(x_0)$. בדומה בעבור $\inf (f(A))$.
	\end{proof}
	\rmark{בד''כ יציינו את זה על קטע סגור, שזה מקרה פרטי של קבוצה קומפקטית. צריך רק קומפקטיות – השתמשנו גם בכל התכונות, הסגירות והחסימות. }

	\theo{תהא $f \co I \to \R$ המקיימת תכונת דרבו. אז ל־$f$ אין נקודות אי־רציפות סליקות או מסוג ראשון. }\begin{proof}
		תהא $x_0 \in I$. נניח שקיים $\lim_{x \to x_0^{-}} f(x)$ וסופי. נסמנו $\ml$. נוכיח ש־$\ml = f(x_0)$. נניח בשלילה ש־$\ml < f(x)$ (כנ''ל לגבי גדול, בה''כ). קיים $\dg > 0$ כך שלכל $x \in I$ אם $x_0 - \ml < x < x_0$ אז $\sof{f(x) - \ml} < \frac{f(x_0) - \ml}{2}$. בקטע $[x_0 - \frac{\dg}{2}, x_0]$, מתקיים:
		\[ f\cl{x_0 - \frac{\dg}{2}} < \frac{\ml + f(x_0)}{2} < f(x_0) \]
		מתכונת דרבו קיים $x_0 - \frac{\dg}{2} < y < x_0$ כך ש־$f(y) = \frac{\ml + f(x_0)}{2}$. כלומר $\sof{f(y) - \ml} \ge \frac{f(x_0) - \ml}{2}$ בסתירה. לכן $f(x_0) \le \ml$. באופן דומה $f(x_0) \ge \ml$. לכן $f(x_) = \ml$. באופן דומה, אם קיים וסופי הגבול $\lim_{x \to x_0^{+}} f(x) =: m$ אז $f(x) = m$.

		מכאן שלא קיימות נקודות אי־רציפות סליקות ומסוג ראשון.
	\end{proof}

	\cola{תהא $f \co I \to \R$. אם $f$ מקיימת תכונת דרבו ומונוטונית, היא בהכרח רציפה. }\begin{proof}
		תהא $f$ מונוטונית המקיימת את תכונת דרבו, מהמשפט הקודם אין לא נקודות אי־רציפות סליקות או מסוג ראשון. משום ש־$f$ מונוטונית, אין לה נקודות אי־רציפות מסוג שני (משפט קודם). מכאן של־$f$ אין נקודות אי־רציפות ולכן היא רציפה.
	\end{proof}

	\rmark{עקרונית אפשר להגדיר את תכונת דרבו בעבור $A$ פתוחה ולהגדירה כך שכל קטע פתוח $I \subseteq A$ מקיים את דרבו כפי שהגדרנו אותה. }

	אם ננסה להוכיח את הרציפות של $\frac{1}{x}$, נצטרך לבחור $\dg = \min\{\frac{x_0}{2}, \frac{\eg x_0^{2}}{2}\}$
	\[ \sof{\frac{1}{x} - \frac{1}{x_0}} = \frac{\sof{x - x_0}}{x_0x} < \frac{\dg}{xx_0} < \frac{\dg}{x_0(x_0 - \dg)} < \frac{2\dg}{x_0^{2}} = \eg \]

	מאוד ברור שה־$\dg$ תלוי באיזה $x_0$ אנחנו בוחרים. זה גם ניכר מההגדרה של רציפות: ''לכל $x \in A$, ולכל $\eg > 0$, קיים $\dg > 0$ כך שלכל $y \in A$ אם לכל $\sof{y - x} < \dg$ אז $\sof{f(x) - f(y)} < \eg$``.

	כאשר אנו אומרים ''במידה שווה``, הכוונה היא שה־$\dg$ לא תלוי בנקודה. דהיינו:
	\defi{$f$ \textit{רציפה במידה שווה} אם לכל $\eg > 0$ קיים $\dg > 0$ כך שלכל $x, y \in A$ אם $\sof{x - y} < \dg$ אז $\sof{f(x) - f(y)} < \eg$. }

	''$\frac{1}{x}$ היא לא סימפטית`` – המרצה (לא פיזיקאי מוסמך).

	\theo{אם $f$ רציפה במידה שווה ב־$A$ אז $f$ רציפה ב־$A$. }\begin{proof}
		כאילו דה
	\end{proof}
	אינטואציה: נדבר על זה בהמשך, אבל נגזרת חסומה אומר שהפונקציה רציפה במידה שווה.

	לדוגמה, נראה ש־$f(x) = x^{2}$ אינה רציפה במידה שווה ב־$\R$, אך רציפה במידה שווה לכל קטע חסום ב־$\R$.
	\begin{proof}
		יהי $M > 0$. נגדיר $f \co [-M, M] \to \R$ ע''י $f(x) = x^{2}$ לכל $x \in [-M, M]$. יהי $\eg > 0$. נבחר $\dg = \frac{\eg}{2M}$. יהיו $x, y \in [-M, M]$ ונניח $\sof{x - y} < \dg$. אז:

		\[ \sof{x^{2} - y^{2}} = \sof{x - y}\sof{x + y} \le 2M\sof{x - y} < 2M\dg = \eg \]
	\end{proof}
	לא סתם בחרנו $\dg$ להיות $\eg$ כפול נקודת המקסימום של הנגזרת, אבל לא מדברים על זה.
	\begin{proof}
		עתה נראה ש־$x^{2}$ אינה רציפה במידה שווה ב־$\R$. נבחר $\eg = 1$ ויהי $\dg > 0$. נבחר $y = x + \frac{\dg}{2}$. נבחר $x = \frac{1}{\dg}$. מכאן $\sof{x - y} < \dg$.
		\[ \sof{x^{2} - y^{2}} = \sof{x - y}\sof{x + y} = \frac{\dg}{2}\cl{2x + \frac{\dg}{2}} > \frac{\dg}{2} \cdot 2x = 1 = \eg \]
	\end{proof}
	תרגיל טוב הוא להוכיח ש־$\sin x^{2}$ אינה רציפה במידה שווה ב־$\R$.

	\theo{תהאנה $f, g \co A \to \R$. נניח כי $f$ רציפה במידה שווה ב־$A$ וגם $g$ רציפה במידה שווה ב־$A$. אז:
	\begin{itemize}
		\item $f\pm g$ רציהפ במידה שווה ב־$A$.
		\item אם $f$ ו־$g$ חסומות ב־$A$, אז $fg$ רציפה במידה שווה.
	\end{itemize}}
	\begin{Theorem}[משפט קנטור (עוד אחד)]
		תהא $f \co A \to \R$. אם $f$ רציפה ב־$A$ וגם $A$ קומפקטית, אז $f$ רציפה במידה שווה ב־$A$.
	\end{Theorem}\begin{proof}
		נניח בשלילה ש־$f$ אינה רציפה במידה שווה ב־$A$. אז קיים $\eg_0 > 0$ כך שלכל $n \in \N$ קיימים $x_n, y_n \in A$ כך ש־$\sof{x_n - y_n} < \frac{1}{n}$ וגם $\sof{f(x_n) - f(y_n)} \ge \eg_0$. אז $\limsi x_n - y_n = 0$. $A$ חסומה ולכן $x_n$ חסומה. לכן מ־BW קיימת לה ת''ס מתכנסת $x_{n_k}$. נסמן גבולה $x_0$. $A$ סגורה ולכן $x_0 \in A$. ידוע ש־$\lim_{k \to \inft} x_{n_k} - y_{n_k} = 0$ שכן כל ת''ס של סדרה מתכנסת מתכנסת לאותו הגבול. לכן מאריתמטיקה $\lim_{k \to \inft} y_{n_k} = x_0$. מהרציפות $\lim_{k \to \inft} f(x_{n_k}) = \lim_{k \to \inft} f(y_{n_k}) = f(x_0)$ בסתירה לכך ש־$\sof{f(x_{n_k}) - f(y_{n_k})} \ge \eg_0$ לכל $k \in \N$. לכן $f$ רציפה במידה שווה ב־$A$.
	\end{proof}

	\theo{יהיו $a, b \in \R\cup\{\pm\inft\}$. נניח $a < b$. יהי $a < c < b$. תהא $f\co (a, b) \to \R$ ונניח $f$ רציפה במידה שווה ב־$(a, c)$ וכן $f$ רציפה במידה שווה ב־$(c, b]$, אז $f$ רציפה במידה שווה ב־$(a, b)$. }\begin{proof}
		יהי $\eg > 0$. ידוע ש־$f$ רב''ש ב־$(a, c]$ ולכן קיים $\dg_1 > 0$ כך ש־$\forall x, y \in (a, c]$ אם $\sof{x - y} < \dg_1$ אז $\sof{f(x) - f(y)} < \frac{\eg}{2}$. $f$ רציפה במ''ש ב־$[c, b)$ לכן קיים $\dg_2 > 0$ כך שלכל $x, y \in [c, b)$, אם $\sof{x - y}$ אז $\sof{f(x) - f(y)} < \frac{\eg}{2}$. נתבונן ב־$\dg = \min\{\dg_1, \dg_2\}$. יהיו $x, y \in (a, b)$. נניח $\sof{x - y} < \dg$. נפרק למקרים.
		\begin{itemize}
			\item אם $x, y \ge c$ אז מכיוון ש־$\sof{x - y} < \dg \le \dg_2$ נובע ב־$\sof{f(x) - f(y)} < \frac{\eg}{2} < \eg$.
			\item אם $x, y \le c$ אז מכיוון ש־$\sof{x - y} < \dg \le \dg_1$ נובע ב־$\sof{f(x) - f(y)} < \frac{\eg}{2} < \eg$.
			\item אם $x \le c \le y$ אז $\sof{c - x} < \sof{y - x} < \dg_1$ ולכן $\sof{f(c) - f(x)} < \frac{\eg}{2}$. אז $\sof{y - c} < \sof{y - x} < \dg_2$ ולכן $\sof{f(c) - f(y)} < \frac{\eg}{2}$. ניעזר בא''ש במשולש:
			\[ \sof{f(x) - f(y)} \le \sof{f(x) - f(c)} + \sof{f(c) - f(y)} = \frac{\eg}{2} + \frac{\eg}{2} < \eg \]
			\item נניח $y \le c \le x$. בדומה.
		\end{itemize}\envendproof
	\end{proof}

	''אתה לא רוצה לשדר זלזול. מקרה 4 בדומה``.

	\exe{נניח ש־$f$ רציפה במידה שווה ב־$(a, c)$ ו־$(b, c)$, ורציפה ב־$c$ (כאשר $a < c < b$). נוכיח ש־$f$ רציפה במידה שווה ב־$(a, b)$. }

	\theo{הפונקציה $\sqrt x$ רציפה במ''ש בקטע $[0, \infty)$. }\begin{proof}
		יהי $\eg > 0$.
		\begin{itemize}
			\item יהיו $x, y \in [1, \infty)$. נניח $\sof{x - y} < \dg$ עבור $\dg = \eg$ ונקבל:
			\[ \sof{\sqrt x - \sqrt y} = \frac{\sof{x - y}}{\sqrt x + \sqrt y} < \frac{\dg}{\sqrt x + \sqrt y} \le \dg = \eg \]
			\item בקטע $[0, 1]$ נקבל ש־$\sqrt x$ רציפה ומשום שהקטע חסום היא רציפה במידה שווה לפי קנטור.
		\end{itemize}
		משום ש־$\sqrt x$ רציפה במ''ש ב־$[0, 1]$ ו־$(1, \infty)$ סה''כ מהמשפט הקודם היא רציפה במ''ש.
	\end{proof}
	\theo{תהא $f \co [a, \infty) \to \R$. נניח $f$ רציפה וגם קיים וסופי $\lim_{x \to \inft}f(x)$. הראו כי $f$ רציפה במ''ש ב־$[a, \infty)$. }\begin{proof}
		ויהי $\eg > 0$. אז קיים $M > 0$ כך שלכל $x, y > M$ מתקיים $\sof{f(x) - f(y)} < \frac{\eg}{2}$ (קושי). הקטע $[a, M]$ הוא קטע קומפקטי, ומשום ש־$f$ רציפה בו ולפי קנטור $f$ רציפה בו במידה שווה. לכן קיים $\dg > 0$ כך שלכל $x, y \in [a, M]$ אם $\sof{x - y} < \dg$ אז $\sof{f(x) - f()} < \frac{\eg}{2}$. נתבונן ב־$\dg$. יהיו $x, y \in [a, \infty)$. נניח בה''כ $x \le y$. נפרק למקרים.
		\begin{itemize}
			\item נניח $x \le y \le M$, מכיוון ש־$\sof{x - y} < \dg$ נובע ש־$\sof(f(x) - f(y)) < \frac{\eg}{2} < \eg$.
			\item נניח $M \le x \le y$, נובע ש־$\sof(f(x) - f(y)) < \frac{\eg}{2} < \eg$.
			\item אם $x \le M \le y$ מא''ש המשולש:
			\[ \sof{f(x) - f(y)} = \sof{f(x) - f(M)} + \sof{f(M) - f(y)} < \frac{\eg}{2} + \frac{\eg}{2} = \eg \]
			לכן $f$ רציפה במידה שווה ב־$[a, \infty)$.
		\end{itemize}
	\end{proof}
	\rmark{לא היה עובד להשתמש במשפט של האיחוד קטעים כאן – כי $M$ תלוי ב־$\eg$. }
	\rmark{זה לא אמ''מ. לדוגמה $\sqrt x$ או $x$. }

	\theo{יהי $a, b \in \R$ ונניח $a < b$. תהא $f \co (a, b) \to \R$ רציפה. אז $f$ רציפה במידה שווה ב־$(a, b)$ אמ''מ קיימים ל־$f$ הגבולות ב־$a$ וב־$b$ והסם סופיים. }\begin{proof}
		\begin{itemize}
			\item[$\implies$]נסמן $\ml = \lim_{x \to a^{+}} f(x)$ ו־$m = \lim_{x \to b^{-}}$. נגדיר:
			\[ F \co [a, b] \to \R \quad F(x) = \begin{cases}
				\ml & x = a \\
				f(x) & x \in (a, b) \\
				m & x = b
			\end{cases} \]
			נבחין ש־$F$ רציפה ב־$[a, b]$ ולפי קנטור, $F$ רציפה במידה שווה ב־$[a, b]$. לכן $f = F|_{(a, b)}$ רציפה במידה שווה ב־$(a, b)$.
			\item[$\impliedby$]רוצים להוכיח שקיים גבול סופי ואין לנו מושג מה הוא. כלומר זה כנראה קושי. נניח כי $f$ רציפה במ''ש ב־$(a, b)$. יהי $\eg > 0$ וידוע קיו ם$\dg > 0$ כך שלכל $x, y \in (a, b)$ אם $\sof{x - y} < \dg$ אז $\sof{f(x) - f(y)} < \eg$. נתבונן ב$\dg$. יהיו $x, y \in (a, a + \dg)$. אז $\sof{x - y} < \dg$ ולכן $\sof{f(x) - f(y)} < \eg$. לפי קריטריון קושי יש ל־$f$ גבול סופי ב־$a$ מימין. באופן דומה יש ל־$f$ גבול סופי משמאל ב־$b$.
		\end{itemize}\envendproof
	\end{proof}




	שנה שעברה עסקנו בתכונות גלובליות של פונקציות רציפות. עתה נתחיל לדבר על הנושא המכעט אחרון, גזירות.

	\section*{גזירות}
	''למי אתה מאמין? לניוטון או לייבניץ?``

	''אני לא זוכר איך קוראים לך, כי אתה אף פעם לא מדבר איתי אלא רק עם האנשים הקרובים אליך``

	אז מכאן התחיל החדו''א. האינטואציה הגיאומטרית הוא מציאת ה־slope של המשיק בנקודה מסוימת.

	\defi{בהינתן $f \co I \to \R$, וכן $x_0 \in I$ בפנים הקטע (איננה נקודת קצה). נאמר ש־$f$ \textit{גזירה ב־$x_0$} כאשר קיים וסופי הגבול $\limxz \frac{f(x) - f(x_0)}{x - x_0}$. }
	\rmark{גזירות היא תכונה נקודית, לוקאלית. }
	\noti{בהנחה שהגבול ב־$x_0$ של הפונקציה $f$ קיים, נסמן $\frac{\dd f}{\dx}(x_0)$ או $f'(x_0)$. }
	\theo{$f$ גזירה ב־$x_0$ אמ''מ קיים וסופי:
	\[ \lim_{h \to 0}\frac{f(x_0 + h) - f(x_0)}{h} \]}
	''עוד הגדרה שקשורה לזה שבמממד אחד היא לא ממש makes sense``
	\defi{תהי $f \co I \to \R$ וכן $x_0 \in I$ בפנים הקטע. $f$ תקרא \textit{דיפרנציאבילית} ב־$x_0$ כאשר קיימת העתקה לינארית $T \co \R\to \R$ המקיימת שהגבול $\limxz \frac{f(x) - f(x_0) - T(x - x_0)}{x - x_0} = 0$. }
	במממד אחד זה לא ממש מעניין. מה זה אומר? נתחיל מהגבול של $\lim_{x \to x_0} f(x) - g(x) = 0$, שאומר שבגבול הן הולכות לאותו המקום. זה אומר שאפשר לעשות ''משפט השוואה``, אפשר להציב באחת ולקבל קירוב של השנייה. גם בהגדרה של דיפרנציאביליות יש לנו שתי פונקציות. מה המשמעות של כך ש־:
	\[ \lim_{ x\to x_0} \frac{g(x) - f(x_)}{x - x_0} \]
	אומרת? זה אומר שלא רק ש־$g$ קירוב טוב ש־$f$, אלא גם שכאשר מחלקים ב־$x - x_0$ ששואף ל־$0$ הקירוב נשאר טוב. הקירוב הזה הולך לאפס יותר מהר מזה ש־$x - x_0$ הולך לאפס. ההגדרה של דיפרנביאביליות אומרת שאפשר לקרב את $f$ בנקודה ע''י פונקציה לינארית, והקירוב הזה יותר מהיר מ־$x - x_0$.

	במשתנה אחד, $f$ גזירה ב־$x_0$ אמ''מ $f$ דיפרנציאבילית. ההעתקה הלינארית $T$ הזו נקראת \textit{הדיפרנציאל} של $f$ ב־$x_0$.

	\theo{תהי $f \co I \to \R$ ותהי $x_0 \in I$ בפנים הקטע. אם $f$ גזירה ב־$x_0$ אז $f$ רציפה ב־$x_0$. }
	\begin{proof}
		נניח ש־$f$ גזירה ב־$x_0$ ונגדיר $h \co I \to \R$ ע''י:
		\[ h(x) = \begin{cases}
			\frac{f(x) - f(x_0)}{x - x_0} & x \neq x_0 \\
			f'(x_0) & x = x_0
		\end{cases} \]
		לכל $x \in I$. נבחין ש־$f$ גיזרה ב־$x_0$ ולכן:
		\[ \lim_{\mathclap{x \to x_0}} h(x) = \lim_{\mathclap{x \to x_0}} \frac{f(x) - f(x_0)}{x - x_0} = f'(x_0) = h(x) \]
		לכן $h$ רציפה ב־$x_0$. מאריתמטיקת גבולות נקבל:
		\[ \lim_{\mathclap{x \to x_0}} f(x) - f(x_0) = \lim_{\mathclap{x \to x_0}} h(x) (x - x_0) \]
		ומכאן ש־$f$ רציפה ב־$f_0$.
	\end{proof}

	\textbf{דוגמאות. }
	\begin{itemize}
		\item נתבונן בפונקציה $f \co \R \to \R$ המוגדרת ע''י $f(x) = x^{n}$. נבחין שלכל $x_0 \in \R$ מתקיים ש־$f$ גיזרה בו ומתקיים $f'(x_0) = nx_0^{n - 1}$.
		\begin{proof}
			לול ראיתי את ההוכחה הזו במכינה של אודיסאה בכיתה ח'. יהי $x_0 \in \R$. ניעזר בבינום של ניוטון, אריתמטיקה ורציפות פולינומים.
			\[ \lim_{\mathclap{x \to x_0}} \frac{x^{n} - x_0^{n}}{x - x_0} = \lim_{\mathclap{x \to x_0}} \sum_{j = 0}^{n - 1}x^{j}x_0^{n - 1 - j} = \sum_{j = 0}^{n - 1}\lim_{\mathclap{x \to x_0}}x^{j}x_0^{n - 1 - j} = \sum_{j = 0}^{n - 1}x^{n - 1} = nx_0^{n - 1} \]
		\end{proof}
		\item נבחין ש־M$\sin$ גזירה בכל $\R$ ונגזרתה $\cos$. \begin{proof}
			יהי $x_0$. לכל $x \neq x_0$:
			\[ \tfl \frac{\sinx - \sin x_0}{x - x_0} = \tfl \frac{2\sin\cl{\frac{x - x_0}{2}}\cos\cl{\frac{x + x_0}{2}}}{x - x_0} = \tfl \frac{\sin\cl{\frac{x - x_0}{2}}}{\frac{x - x_0}{2}} \cdot \cos{\frac{x + x_0}{2}} \]
			מהרציפות של $\cos$ ב־$x_0$ ומהגבולות וההרכבה $\lim_{x \to x_0} \cos\cl{\frac{x + x_0}{2}} = \cos x_0$ ומהרציפות של $\frac{\sinx}{x}$ ב־$0$ ומגבולות וההרכבה $\lim_{x \to x_0}\frac{\sin\cl{x - x_0}}{\frac{x - x_0}{2}} = 1$ ומאריתמטייקה גבולות, נקבל:
			\[ \tfl \frac{\sinx - \sin x_0}{x - x_0} = \cos x_0 \]
		\end{proof}

		\item \textbf{דוגמה 3. }יהי $x_0 \in \R$. נמצא את הנגזרת של $e^{x}$ ב־$x_0$.
		\begin{proof}
			האמת את ההוכחה הזו ראיתי בכיתה ט' במתמטיקה ב'. יש לי אותה מוקלדת עם יותר פירוט בסיכום של $e$ במתמטיקה B. יהי $x_0 \in \R$. נבחין ש־:
			\[ \frac{e^{x} - e^{x_0}}{x - x_0} = e^{x_0} \cdot \frac{e^{x - x_0} - 1}{x - x_0} \]
			ובפרט:
			\[ \lim_{h \to 0} \frac{e^{h} - 1}{h} = 1 \]
			(כי ראינו את שיעור שעבר) ומגבולות והרכבה $\lim_{x \to x_0} \frac{e^{x - x_0} - 1}{x - x_0} = 1$. מאריתמטיקה סיימנו.
		\end{proof}
		\item נגדיר $f \co \R\to \R$ ע''י $f(x) = xD(x)$. נוכיח ש־$f$ אינה גזירה באף נקודה ב־$\R$. \begin{proof}
			לכל $x_0 \neq 0$, הראינו ש־$f$ אינה רציפה ב־$x_0$. לכן היא אינה גזירה ב־$x_0$. נטפל עתה ב־$0$ (נראה שהיא אומנם רציפה אך לא גזירה בו). נתבונן ב־$\eg_0 = \frac{1}{2}$ ויהי $\dg > 0$. בקטע $(-\dg, \dg) \setminus \{0\}$  יש רציונלי $x$ ורציונלי $y$ כך ש־:
			\[ \sof{\frac{f(x) - f(0)}{x - 0} - \frac{f(y) - f(0)}{y - 0}} = \sof{D(x) - D(y)} = 1 \ge \eg_0 \]
			מקריטריון קושי לא קיים $\lim_{x \to 0} \frac{f(x) - f(0)}{x - 0}$.
		\end{proof}
		\item עבור $x^{2}D(x)$, היא אומנם עדיין לא רציפה ב־$x_0 \neq 0$, אבל ב־$0$ יקרו דברים קצת פחות מנוונים:
		\begin{proof}
			הראינו ש־$\limxz xD(x) = 0$ והראינו ש־$\limxz \frac{f(x) - f(0)}{x- 0} = \limxz \frac{x^{2}D(x)}{x} = \lim_{x \to x_0} xD(x) = 0$.
		\end{proof}
	\end{itemize}

	\subsection*{נגזרות חד־צדדיות}
	\defi{תהא $f \co  I \to \R$ ותהא $x_0 \in I$ המקיימת $\exists \dg > 0 \co (x_0 - \dg, x_0) \subseteq I$. אז נאמר שנאמר ש־$f$ \textit{גזירה משמאל ב־$x_0$} כאשר קיים וסופי הגבול $\lim_{x \to x_0^{-}} \frac{f(x) - f(x_0)}{x - x_0}$. }
	\defi{\textit{נגזרת מימין} מוגדרת באופן דומה}
	\noti{נסמן את הגזירה משמאל ב־$f_-'(x_0)$ ומימין $f_+'(x_0)$. }


	\subsubsection*{אריתמטיקה של גזירות}
	\theo{יהיו $f, g \co I \to \R$ ותהי $x_0 \in I$ בפנים הקטע. נניח ש־$f, g$ גזירות ב־$x_0$. אז:
	\begin{itemize}
		\item לכל $\ag, \bg \in \R$ מתקיי ם$\ag f + \bg g$ גזירה ב־$x_0$ וכן $(\ag f + \bg g)' = \ag f'(x_0) + \bg g'(x_0)$ (הנגזרת לינארית)
		\item מתקיים ש־$fg$ גזירה ב־$x_0$ ומתקיים ש־$(fg)'(x_0) = f'(x_0)g(x) + f(x_0)g'(x_0)$.
		\item אם $g(x_0) \neq 0$ אז $\frac{f}{g}$ גזירה ב־$x_0$ ומתקיים:
		\[ \cl{\frac{f}{g}}'\!\!(x_0) = \frac{f'(x_0)g(x_0) - f(x_0)g'(x_0)}{(g(x_0))^{2}} \]
	\end{itemize}}
	נוכיח את (2) ואת השאר לבית.
	\begin{proof}
		לכל $x \neq x_0$ מתקיים ש־:
		\begin{multline*}
			\tfl \frac{f(x)g(x) - f(x_0)g(x_0)}{x - x_0} = \tfl \frac{f(x) g(x) - f(x)g(x_0) + f(x)g(x_0) - f(x_0)g(x)}{x - x_0} \\
			= \tfl f(x) \frac{g(x) - g(x_0)}{x - x_0} + g(x_0) \frac{f(x) - f(x_0)}{x - x_0}
		\end{multline*}
		ידוע ש־$f$ גזירה ב־$x_0$ ולכן רציפה ב־$x_0$. לכן $\lim_{x \to x_0} f(x) = f(x_0)$. $g$ גזירה ב־$x_0$ ולכן $\lim_{x \to x_0} \frac{g(x) - g(x_0)}{x - x_0} = g'(x_0)$. מאריתמטיקה קיבלנו:
		\[ \tfl f(x) \frac{g(x) - g(x_0)}{x - x_0} f(x_0)g(x_0) \quad\quad \tfl g(x_0) \frac{f(x) - f(x_0)}{x - x_0} = g(x_0)f'(x_0) \]
		מאריתמטיקה סיימנו:
		\[ \tfl \frac{f(x)g(x) - f(x)g(x)}{x - x)} = f(x_0)g'(x_0) + g'(x_0)g(x_0) \]

	\end{proof}
	''זהו אתה פורש?``

	\subsection*{כלל השרשרת}
	\theo{תהא $f \co I \to J$ ותהא $g \co J \to \R$. נניח $x_0 \in I$ בפנים הקטע. נניח ש־$f$ גזירה ב־$x_0$ וגם $g$ גזירה ב־$x_0$. אז $g \circ f$ גיזרה ב־$x_0$ וכן $(g \circ f)'(x_0) = g'(f(x_0))f'(x_0)$.}
	\begin{proof}[הוכחה שגויה, ידועה בכינוייה הוכחהחהחה]
		\[ \tfl \frac{(g \circ f)(x) - (g \circ f)(x_0)}{x - x_0} = \tfl \frac{g(f(x)) - g(f(x_0))}{f(x) - f(x_0)} \cdot \frac{f(x) - f(x_0)}{x - x_0} = g'(f(x_0)) \cdot f'(x_0) \]
		מה הבעיה בהוכחהחהחה? שלא מובטח ש־$f(x) - f(x_0) \neq 0$. מותר להניח $x \neq x_0$ (כי אנחנו בגבול), אבל הטענה השנייה לא עובדת. לדוגמה עבור פונקציה קבועה ההוכחהחהחה לא עובדת. יש כאן עוד בעיה. במשפט של הרכבה, דרשנו שהגבול של הפונקציה הפנימית מקיימת כל מני דברים. לכן נצטרך לעשות חלוקה למקרים.
	\end{proof}
	\rmark{שיגאות מעין אילו הרבה פעמים חומקות מתחת לרדאר. אבל גם הוכחה שגויה אפשר לתקן – אם נפצל למספיק מקרים נוכל לטפל בבעיה. }
	\begin{proof}[הוכחה נכונה]
		\begin{itemize}
			\item במקרה הראשון, $f'(x_0) \neq 0$. ואז הגבול $\lim_{x \to x_0} \frac{f(x) - f(x_0)}{x - x_0} \neq 0$ ולכן קיים $\dg > 0$ כך שלכל $x \in (x_0 - \dg, x_0 + \dg)$ ומכאן $\frac{f(x) -f(x_0)}{x - x_0} \neq 0$  ובפרט $f(x) \neq f(x_0)$. בקטע הזה אפשר לבצע את ההוכחהחהחה – מתקיימים תנאי המשפט על גבולות והרבה, ולכן הגבול:
			\[ \tfl \frac{g(f(x)) - g(f(x_0))}{f(x) - f(x_0)} = \lim_{\mathclap{y \to x_0}} \frac{g(y) - g(f(x_0))}{y - f(x_0)} = g'(f(x_0)) \]
			$f$ גזירה ב־$f_0$ ולכן $\lim_{x \to x_0} \frac{f(x) - f(x_0)}{x - x_0} = f'(x_0)$. מאריתמטיקה קיבלנו $\lim_{x \to x_0} \frac{(g \circ f)(x) - (g \circ f)(x_0)}{x - x_0} = g'(f(x_0))f'(x_0)$.
			\item אם $f'(x) = 0$. במקרה זה, לביטוי:
			\[ \lim_{y \to f(x_0)} \frac{g(y) -g(f(x_0))}{y - f(x_0)} \]
			קיים וסופי (הגדרת הנגזרת). לכן קיים $M > 0$ כך שקיים $\dg > 0$ כך שלכל $y \in (f(x_0) - \dg, f(x_0) + \dg)$, מתקיים ש־$\sof{\frac{g(y) - g(f(x_0))}{y - f(x_0)}} < M$ (ברה יכולת פשוט להגיד שהדבר הזה חסום ע''י $M$ בסביבת $\dg$ נקובה ולגמור עניין). יהי $\eg > 0$. אז קיים $\dg_2 > 0$ כך שללכ $x \in (x_0 - \dg, x_0 + \dg)$, מתקיים $\sof{\frac{f(x) - f(x_0)}{x - x_0}} < \frac{\eg}{M}$ (הגדרת הגבול). $f$ רציפה ב־$x_0$ ולכן קיים $\dg_3 > 0$ כך שלכל $x \in (x_0 - \dg, x_0 + \dg)$ מתקיים $\sof{f(x) - f(x_0)} < \dg$. נסמן ב־$\eta = \min\{\dg_2, \dg_3\}$ ויהי $x \in (x_0 - \eta, x_0 + \eta)$. נחלק למקרים.
			\begin{itemize}
				\item אם $f(x) = f(x_0)$, אז $\sof{\frac{g(f(x)) - g(f(x_0))}{x - x_0}} = 0 < \eg$ וסיימנו.
				\item אחרת:
				\[ \frac{(g \circ f)(x) - (g \circ f)(x_0)}{\sof{x - x_0}} = \sof{\frac{(g \circ f)(x) - (g \circ f)(x_0)}{f(x) - f(x_0)}}\sof{\frac{f(x) - f(x_0)}{x - x_0}} < M \cdot \frac{\eg}{M} = \eg \]
			\end{itemize}
			סה''כ:
			\[ \tfl \frac{(g \circ f)(x) - (g \circ f)(x_0)}{x - x_0} = 0 = g'(f(x_0))f'(x) \]\envendproof
		\end{itemize}
	\end{proof}

	\subsection*{מסקנות נוספות}
	\theo{תהא $f \co I \to J$ פונקציה חח''ע ועל, כאשר $I, J$ קטעים פתוחים (אך לא בהכרח, סתם למרצה לא בא להתעסק עם הקצוות). אז $f\op$ גזירה בכל נקודה ב־$J$ ומתקיים $\forall y \in J \co (f\op(y))(y) = \frac{1}{f'(f\op(y))}$. }\begin{proof}[הוכחהחהחה]
		ידוע שלכל $x \in J$ מתקיים $(f \circ f\op)(x) = x$. מכלל השרשרת נקבל $f'(f\op(x))(f\op)'(x) = 1$. נחלק ונקבל את הדרוש.

		מה הבעיה בהוכחהחהחה? כלל השרשרת דרש שהפונקציה גזירה בנקודה. לא הראינו את זה.
	\end{proof}

	\begin{proof}[הוכחה נכונה]
		יהי $y \in J$. נניח $f\op(f'(y_0)) \neq 0$. מסמן $x_0 = f\op(f'(y_0))$. ידוע $f'(x_0) \neq 0$, לכן קיימת סביבה מנוקבת $U$ של $x_0$ כך שבה לכל $x \in U$ מתקיים $\frac{f(x) - f(x_0)}{x - x_0} \neq 0$. נגדיר $g \co U \to \R$. לכל $x \in U$ מתקיים:
		\[ g(x) = \begin{cases}
			\frac{1}{\frac{f(x) - f(x_0)}{x - x_0}} & x \neq x_0 \\
			\frac{1}{f\op(x_0)} & x = x_0
		\end{cases} \]
		ניתן להבחין ש־$g$ רציפה, ובפרט רציפה ב־$x_0$. $f\op$ רציפה ב־$y_0$. לכן $g \circ f\op$ רציפה ב־$y_0$. כלומר:
		\[ \lim_{y \to y_0} (g \circ f\op)(y) = (g \circ f\op)(y_0) = g(x_0) = \frac{1}{f'(x_0)} = \frac{1}{f'(f\op(y_0))} \]
		מצד שני,
		\[ \lim_{y \to y_0} (g \circ f\op)(y) = \lim_{ y \to y_0} \frac{f'(y) - f\op(y_0)}{y - y_0} \]
		לכן $f\op$ גזירה ב־$y_0$ ומתקיים
		\[ (f\op)'(y_0) = \frac{1}{f'(f\op(y_0))} \]
		וסיימנו.
	\end{proof}
	\textbf{דוגמה. }יהי $n \in \N^{+}$, ונגדיר $f(x) = \sqrt[n]{x}$ לכל $x \in (0, \inft)$. נשים לב ש־$\mathrm{Range} f = (0, \infty)$. נגדיר $g \co (0, \inft) \to (0, \inft)$ על ידי $g(x) = x^{n}$ לכל $x \in (0, \inft)$. אז $f = g\op$. לכן $f$ גזירה בכל נקודה $(0, \inft)$ ומתקיים לכל $y \in (0, \inft)$ ש־$f'(y) = \frac{1}{g'(f(y))}$. סה''כ:
	\[ f'(y) = \frac{1}{g'(f(y))} = \frac{1}{n \cdot (f(y))^{n - 1}} = \frac{1}{n}y^{\frac{1 - n}{n}} = \frac{1}{n}y^{\frac{1}{n} - 1} \]

	\textbf{דוגמה. }יהיו $m, n \in \N^{+}$. נגדיר $f \co (0, \inft) \to (0, \inft)$ ע''י $f(x) = x^{\frac{m}{n}}$. אז אפשר לנסח $f(x) = \cl{x^{\frac{1}{n}}}^{m}$ ומכלל השרשרת נקבל ש־$f(x) = m\cl{x^{\frac{1}{n}}}^{m - 1} \cdot \frac{1}{n}x^{\frac{1}{n} - 1} = \frac{m}{n}x^{\frac{m}{n} - \frac{1}{n} + \frac{1}{n} + 1} = \frac{m}{n}x^{\frac{m}{n} - 1}$.
	סה''כ באופן כללי $(x^{q})' = qx^{q - 1}$ לכל $q \in \Q_+$.

	לבית, להוכיח ל־$\Q_-$ וכן ל־$\R$, ש־$(x^{r})' =  rx^{r - 1}$ לכל $r \in \R$.

	\textbf{דוגמה. }יהי $a > 0$ לכל $x \in \R$. נקבל $f(x) = e^{x\ln a}$ ומכלל השרשרת קיבלנו (הבהרה, לא גזרנו $\ln$, גזרנו קבוע) $f'(x) = e^{x\ln a} \ln a = a^{x}\ln a$.

	\textbf{דוגמה. }נגדיר $f(x) = \ln x$ לכל $x \in (0, \inft)$. נסמן $g(x) = e^{x}$ לכל $x \in \R$. נבחין ש־$g\op = f$ וכן $g'$ אינה מתאפסת באף נקודה. מכאן ש־:
	\[ f'(x) - \frac{1}{g'(f(x))} = \frac{1}{e^{f(n)}} = \frac{1}{e^{\ln x}} = \frac{1}{x} \]

	\textbf{דוגמה. }נגדיר $f(x) = \arctan x$ לכל $x \in \R$. נבחין $g(x) = \tanx$ לכל $x \in (-\frac{\pi}{2}, \frac{\pi}{2})$. אז $g\op = f$ ולכל $x \in \cl{-\frac{\pi}{2}, \frac{\pi}{2}}$ מתקיים $g'(x) = \frac{1}{\cos^{2}x}\neq 0$. לכן $f$ גזירה בכל $\R$ ומתקיים:
	\[ f'(x) = \frac{1}{g'(f(x))} = \frac{1}{\frac{1}{\cos^{2}\arctan(x)}} = \frac{1}{1 + x^{2}} \]

	\subsection*{תכונות גלובליות של פונקציות גזירות}

	\begin{Theorem}[המשפט הלא אחרון של פרמה]
		תהא $f \co I \to \R$ ותהי $x_0 \in I$ בפנים הקטע. נניח $f$ גזירה ב־$x_0$ ונניח של־$f$ יש קיצון מקומי ב־$x_0$. אז $f'(x_0) = 0$.
	\end{Theorem}
	\rmark{המשפט הזה הוא חד־כיווני. לא כל נקודה סטרציונרית היא נקודת קיצון. }
	\defi{ל־$f$ יש מקסימום מקומי ב־$x_0$ כאשר קיים $\dg > 0$ כך שלכל $x \in (x_0 - \dg, x_0 + \dg)$ מתקיים $f(x) \le f(x_0)$. }
	\defi{מינימום מקומי בדומה. }
	\begin{proof}[הוכחה למשפט פרמה הלא אחרון]
		נניח $x_0$ מקסימום מקומי (ההוכחה עבור מינימום בדומה). $x_0$ פנימית בקטע ולכן מוגדרות (הנחת גזירות) ושוות הנגזרות החד־צדדיות בנקודה. קיים $\dg > 0$ כך שלכל $x \in (x_0 - \dg, x_0 + \dg)$ מתקיים $f(x) \le f(x_0)$. לכן לכל $x \in (x - \dg, x + \dg)$ נקבל:
		\[ \frac{f(x) - f(x_0)}{x - x_0} \ge 0 \]
		(משפט לפיו גבול משמר א''ש חלש). לכן $f'(x_0) = \lim_{x \to x_0^{-}} \frac{f(x) - f(x_0)}{x - x_0} \ge 0$. באופן דומה מימין נקבל $f'(x_0)$. לכן $f'(x_0)$.
	\end{proof}

	\subsubsection*{המשפט היסודיים של החשבון הדיפרנציאלי (להבדיל מהמשפט היסודי של החד''א)}
	\begin{Theorem}[משפט רול]
		תהא $f \co [a, b] \to \R$. נניח ש־$f$ רציפה בקטע ב־$[a, b]$ וכן גזירה ב־$(a, b)$, $f(a) = f(b)$. אז קיימת $c \in (a, b)$ שבה $f'(c) = 0$.
	\end{Theorem}
	\begin{proof}
		\begin{enumerate}
			\item מקרה ראשון: נניח $f$ קבועה ב־$[a, b]$. נתבונן ב־$c = \frac{a + b}{2}$ ובסביבת  $c$ היא קבועה כלומר $f'(c) =0$ וסיימנו.
			\item במקרה השני, קיים $x \in (a, b)$ כך שלכל $f(x) \ne f(a)$, בה''כ $f(x) > f(a)$, $f$ רציפה ב־$[a, b]$ ולכן לפי וויראשטראס יש לה מקסימום בקטע, כלומר קיימת $c \in [a, b]$ כך שלכל $x \in (a, b)$ מתקיים $f(x) \le f(c)$. ידוע $f(c) \ge f(x_0) > f(a)$ לכן $c \neq a$ וגם $c \in (a, b)$ ולכן לפי פרמה מכיוון ש־$f$ גזירה ב־$c$ נובע ש־$'f(c) = 0$.
		\end{enumerate}
	\end{proof}

	\begin{Theorem}[משפט ערך הביניים של לגראנג']
		תהי $f \co [a, b] \to \R$ ונניח $f$ רציפה ב־$[a, b]$ וכן גזירה ב־$(a, b)$. אז קיימת $c \in (a, b)$ כך ש־$f'(c) = \frac{f(b) - f(a)}{b - a}$.
	\end{Theorem}
	\rmark{מקרה פרטי של רול, עבור $a = b$. }
	\begin{proof}
		נגדיר: (נחסר את המיתר כדי להשתמש ברול)
		\[ h(x) = f(x) - f(a) - \frac{f(b) - f(a)}{b - a}(x - a) \]
		מאריתמטיקה $h$ רציפה  ב־$[a, b]$ וכן גזירה ב־$(a, b)$. כמו כן $h(a) = h(b) = 0$. לכן $h$ מקיימת את תנאי משפט רול ב־$[a, b]$ כלומר קיימת $c \in (a, b)$ כך ש־$h(c) = 0$, ומתקיים:
		\[ 0 = h'(c) = f'(c) - \frac{f(b) - f(a)}{b - a} \implies f'(c) = \frac{f(b) - f(a)}{b - a} \]
		כנדרש.
	\end{proof}


	\theo{תהא $f \co I \to \R$ ונניח כי $f$ גזירה בכל $I$ וכי לכל $x \in I$ מתקבל $f'(x) = 0$. הראו כי $f$ קבועה. }\begin{proof}
		יהיו $x, y \in I$ ונניח $x < y$. בקטע $[x, y]$ $f$ רציפה. בקטע $(x, y)$ $f$ גזירה. לכן קיימת $c \in (a, b)$ כך ש־:
		\[ f'(c) = \frac{f(x) - f(y)}{x - y} \]
		ו־$f'(c) = 0$ כלומר $f(x) = f(y)$.
	\end{proof}
	\rmark{הוכחה דומה מראה שאם $f'$ שלילית אז $f$ יורדת ואם $f'$ חיובית אז $f$ עולה. }

	\theo{תהא $f \co I \to \R$ ונניח כי $f$ גזירה בכל $I$. הראו ש־$f$ עולה ב־$I$ אמ''מ $\forall x \in I \co f'(x) \ge 0$. }
	\begin{proof}
		\begin{itemize}
			\item[$\impliedby$]אותה הפריקינג הוכחה. יהיו $x, y \in I$ ונניח $x < y$. $f$ רציפה ב־$[x, y]$ וכן גיזרה ב־$(x, y)$. מכאן ש־$f$ מקיימת את תנאי משפט לגראנג' וקיימת $c \in (x, y)$ כך ש־:
			\[ 0 \le f'(x) = \frac{f(x) - f(y)}{x - y} \]
			כלומר $f(x) \le f(y)$ וסיימנו.
			\item[$\implies$]יהי $x_0 \in I$. קיים $\dg > 0$ כך ש־$(x_0 - \dg, x_0 + \dg) \subseteq I$. מהגדרה $\lim_{x \to x_0} \frac{f(x) - f(x_0)}{x - x_0} = f'(x_0)$. ידוע $f(x) \le f(x_0)$ כי הנחנו שהיא עולה, ואז לכל $x \in (x_0 - \dg, x_0)$ קיבלנו $\frac{f(x) - f(x_0)}{x - x_0} \ge 0$. מכאן $f'(x_0) \ge 0$. ז
		\end{itemize}
	\end{proof}


	\subsubsection*{המשך נגזרות ולופיטל}

	\begin{Theorem}[משפט דרבו]
		תהא $f \co (a, b) \to \R$ גזירה ב־$(a, b)$. אז $f' \co (a, b) \to \R$ מקיימת את תכונת דרבו.
	\end{Theorem}\begin{proof}
		יהיו $a < x_0 < y_0 < b \in \R$ ויהי $\lg \in (f'(y_0), f'(x_0)) \uplus (f'(x_0), f'(y_0))$. נוכיח תחילה בעבור $\lg = 0$. בה''כ $f'(x_0) < 0 < f'(y_0)$. ידוע $f$ רציפה ב־$[x_0, y_0]$ (ממשפט) ולכן לפי וויראשטראס מקבלת מינימום בקטע, דהיינו $\exists c \in [x_0, y_0] \co \forall x \in [x_0, y_0] \co f(c) \le f(x)$. מהגדרת נגזרת (המטרה היא להראות באופן טרחני ש־$c$ לא מינימום קיצון, אלא מינימום מקומי):
		\[ 0 > f'(x_0) = \limxz \frac{f(x) - f(x_0)}{x - x_0} = \lim_{x \to x_0^{+}} \frac{f(x) - f(x_)}{x - x_0} \]
		כלומר קיים $\dg > 0$, כל שלכל $x_0 < x < x_0 + \dg$ מתקיים:
		\[ \sof{\frac{f(x) - f(x_0)}{x - x_0} - f'(x_0)} < \sof{f'(x_0)} \]
		בפרט:
		\[ \frac{f\cl{x_0 + \frac{\dg}{2}} - f(x_0)}{x + \frac{\dg}{2} - x_0} < f'(x_0) + \sof{f'(x_0)} = 0 \]
		(חוקי כי $0 < \frac{\dg}{2} < \dg$) כלומר נובע $f(x_0 + \frac{\dg}{2}) < f(x_0)$ ולכן $c \neq x_0$. באופן דומה $c \neq y_0$. מכאן $c \in (x_0, y_0)$ וממשפט פרמה $f'(c) = 0$.

		המקרה בו $\lg \neq 0$ נובע באופן טרוויאלי ע''י הזזה אנכית של הפונקציה וחזרה למקרה בו $\lg = 0$. המרצה עושה את זה פורמלית אבל אין לי הרבה זמן עד ההרצאה ואני צריך להשלים הכל.
	\end{proof}

	\begin{Theorem}[משפט קושי]
		יאי עוד משפט קושי. תהאנה $f, g \co [a, b] \to \R$ ונניח כי שתיהן רציפות ב־$[a, b]$, שתיהן גזירות ב־$(a, b)$, ולכל $x \in (a, b)$, מתקיים $g'(x) \neq 0$. אז $g(b) \neq g(a)$ וגם קיימת $c \in (a, b)$ כך ש־$\frac{f(b) - f(a)}{g(b) - g(a)} = \frac{f'(c)}{g'(c)}$. \begin{proof}
			לפי רול מכיוון ש־$g$ רציפה ב־$[a, b]$ וגזירה ב־$(a, b)$, וגם $g'(x) \neq 0$ לכל $x \in (a, b)$, נובע ש־$g(b) \neq g(a)$.

			עתה נגדיר:
			\[ g \co [a, b] \to \R\quad h(x) = f(x)- f(a) - \frac{f(b) - f(a)}{g(b) - g(a)}(g(x)- g(a)) \]
			אז $h$ רציפה ב־$[a, b]$ וגזירה ב־$(a, b)$ שכן היא צירוף לינארי של פונקציות גזירות ורציפות. נבחין ש־$h(a) = 0 = h(b)$. לכן ממשפט רול קיימת $c \in (a, b)$ כך ש־$h'(c) = 0$. לפי כללי גזירה:
			\[ h'(c) = f'(c) - \frac{f(b) - f(a)}{g(b) - g(c)} \cdot g'(c) \]
			ומכאן
			\[ \frac{f(b) - f(a)}{g(b) - g(a)} = \frac{f'(c)}{g'(c)} \]
			כנדרש. כמובן שממש לא עשינו אינטגרל וסתם הפלצנו את $h$ משום מקום.
		\end{proof}
	\end{Theorem}

	\exe{יהיו $a < b \in \R$ ותהא $f \co (a, b) \to \R$ גזירה. נניח ש־$\lim_{x \to a^{+}}f(x) = \lim_{x \to b^{-}} f(x) = +\infty$. הראו כי $f'$ על $\R$. }\begin{proof}
		יהי $\lg \in \R$. מתקיים $\lim_{x \to b^{-}} f(x) = \infty$ ולכן קיים $\dg > 0$ כך ש־:
		\[ \forall x \in (b - \dg, b) \co f(x) > \sof{f\cl{\frac{a + b}{2}}} + \sof{\lg}(b - a) \]
		אינטואיטיבית $\lg$ צריך ליפול בין $b - \frac{\dg}{2}$ לבין $\frac{a + b}{2}$, ולכן הדרישה לעיל. ממנה נסיק בפרט:
		\[ f\cl{b - \frac{\dg}{2}} - f\cl{\frac{a + b}{2}} > (b - a) \lg \]
		נחלק אגפים ונקבל:
		\[ \frac{f\cl{b -\frac{\dg}{2}} - f\cl{\frac{a + b}{2}}}{b - \frac{\dg}{2} - \frac{a + b}{2}} > \frac{f\cl{b -\frac{\dg}{2}} - f\cl{\frac{a + b}{2}}}{b- a} > \lg \]
		בקטע $[\frac{a + b}{2}, b - \frac{\dg}{2}]$ $f$ מקיימת את תנאי משפט לגראנג' ולכן קיימת $c$ בקטע המדובר כך ש־:
		\[ f'(c) = \frac{f\cl{b - \frac{\dg}{2}} - f\cl{\frac{a +b}{2}}}{b - \frac{\dg}{2} - \frac{a + b}{2}} > \lg \]
		באופן דומה קיימת $d \in (a, b)$ כך ש־$f'(d) < \lg$ (אותו הדבר הפוך), ואז ממשפט דרבו ישנה $\ag \in (a, b)$ כך ש־$f'(\ag) = \lg$. לכן $f'$ על $\R$.
	\end{proof}

	\begin{Theorem}[משפט לופיטל 1]
		תהאנה $f, g \co T \setminus \{a\} \to \R$. נניח ש־$a$ נקודת הצטברות של $I \setminus \{a\}$. עוד נניח ש־$f, g$ רציפות ב־$I \setminus \{a\}$ וכן $f, g$ גזירות ב־$I \setminus \{a\}$. נניח ש־$\lim_{x \to a} f(x) = \lim_{x \to a} g(x) = 0$ (במקרים האחרים אפשר פשוט להשתמש בכללי גבולות כרגיל), וכן קיים הגבול $\lim_{x \to a} \frac{f'(x)}{g'(x)}  = \ml$. תחת כל התנאים הללו $\lim_{x \to a}\frac{f(x)}{g(x)} = \ml$ (כאשר $a$ ו־$\ml$ מוגדרים במובן הרחב).
	\end{Theorem}

	\rmark{לופיטל גנב את המשפט ממישהו אחר בלה בלה בלה}

	\begin{proof}
		בהרצאה נוכיח רק את המקרה בו $a \in I$ ו־$\ml \in \R$ (באופן כללי, צריך לפצל ל־4 מקרים, בהתאם להיותם של $a, \ml$ מוגדרים במובן הרחב או לאו). בה''כ $f, g$ מוגדרות ב־$a$ ומתקיים $f(a) = g(a) = 0$ (פורמלית, נגדיר $\tl f, \tl g$ חדשות שמוגדרות ב־$a$). נוכיח $\lim_{x \to a^{-}} \frac{f(x)}{g(x)} = \ml$, והגבול מימין באופן דומה. יהי $\eg > 0$. קיים $\dg > 0$ כך שלכל $a - \dg < x < a$ מתקיים $\sof{\frac{f'(x)}{g'(x)} - \ml} < \eg$. נתבונן ב־$\dg$. יהי $a - \dg < x < a$. בקטע $[x, a]$ מתקיימים תנאי משפט קושי: $f, g$ רציפות, גזירות, ו־$\forall \ag \in (x, a) \co g'(\ag) \neq 0$. מכאן קיימת $a - \dg < x < c < a$ כך ש־$frac{f(x)}{g(x)} = \frac{f(a) - f(x)}{g(a) - g(x)} = \frac{f'(c)}{g'(c)} $, וסה''כ:
		\[ \sof{\frac{f(x)}{g(x)} - \ml} = \sof{\frac{f'(c)}{g'(x)} - \ml} < \eg \]
		וסיימנו את הוכחת הגבול לפי הגדרה.
	\end{proof}

	\begin{Lemma}[הלמה של שטולץ]
		תהאנה $\an, \bn$ סדרות ונניח ש־$\bn$ מונוטונית ממש ו־$b_n \to +\infty$. אם קיים וסופי $\limsi \frac{a_{n + 1} - a_n}{b_{n + 1} - b_n}$ אז קיים וסופי $\limsi \frac{a_n}{b_n}$ וגבולותיהם שווים (לופיטל 2 בדיד).
	\end{Lemma}
	\begin{Theorem}[משפט לופיטל 2]
		תהאנה $f, g \co I \setminus \{a\} \to \R$ כאשר $I$ קטע ו־$a$ נקודת הצטברות. נניח ש־$f, g$ גזירות ב־$I \setminus \{a\}$ ו־$\forall x \in I \setminus \{a\} \co g'(x) \neq 0$. עוד נניח $\lim_{x \to a}\sof{g(x)} \to \infty$ (המקרה היחיד שבאמת מעניין אותנו זה כשגם $f$ שואף לאינסוף בנקודה) וקיים $\lim_{x \to a} \frac{f'(x)}{g'(x)} = \ml$. אז $\lim_{x \to a}\frac{f(x)}{g(x)}$ קיים וערכו $\ml$.
	\end{Theorem}
	\begin{proof}
		נוכיח ש־$\lim_{x \to a^{+}} \frac{f(x)}{g(x)} = \lim_{x \to a} \frac{f'(x)}{g'(x)}$, והכיוון השני באופן דומה. גם כאן, נתעסק רק במקרה בו $a$ סופי והגבול סופי (של חלוקת הנגזרות), ועקרונית צריך לפרק למקרים. תהא $x_n$ סדרה המקיימת $x_n < x_{n + 1} \in I \setminus \{a\}$ וכן גבולה $\limsi x_n = a$. לכל $x \in I \setminus \{a\}$, מתקיים $g'(x) \neq 0$ ולכן מדרבו $g'$ דומת סימן בקטע. ללא הגבלת הכלליות, $g'(x) > 0$ (במובן הרחב) ולכן היא מונוטונית עולה ומתקיים $\lim_{x \to a}g(x) = \infty$. נסיק ש־$g(x_n)$ סדרה מונוטונית עולה ממש וגבולה $+\infty$. יהי $n \in \N$. בקטע $[x_n, x_{n + 1}]$ הפונקציות $f, g$ מקיימות את תנאי משפט קושי. לכן קיים $z_n \in (x_n, x_{n + 1})$ כך ש־:
		\[ \frac{f(x_{n + 1} - f(x_n))}{g(x_{n + 1} - g(x_n))} = \frac{g'(z_n)}{g'(z_n)} \]
		ומסנדוויץ', בהכרח $\limsi z_n = a$. כמו כן לכל $n \in \N$ בהכרח $z_n \neq a$. לפי היינה
		$\limsi \frac{f(x_{n + 1}) - f(x_n)}{g(x_{n + 1} - g(x_n))}  = \limsi \frac{f'(z_n)}{g'(z_n)} = \ml$
		לפי הלמה של שטולץ קיבלנו $\limsi \frac{f(x_n)}{g(x_n)} = \ml$ ומהיינה קיבלנו את הדרוש.
	\end{proof}

	\exe{נמצא את הגבול הבא:
	\[ \limz \frac{\cosx - 1}{x^{2}} \]

	נגדיר $f, g \co \R \setminus\{0\} \to \R$ ע''י $f(x) = \cosx - 1$ ו־$g(x) = x^{2}$. שתיהן רציפות וגזירות ב־$\R\setminus\{0\}$ וב־$0$ גבולן $0$. מלופיטל:
	\[ \limz \frac{\cosx - 1}{x^{2}} \slh \limz \frac{f'(x)}{g'(x)} = \limz \frac{-\sinx}{2x} = -\frac{1}{2} \]
	סימנו את השוויון שנובע מלופיטל ב־$\slh$ כדי להבהיר שהוא נכון בתנאי שהגבול מימין אכן מוגדר (אחרת – אי אפשר להגיד שום דבר על הגבול לפי לופיטל!).
	}
	\exe{נמצא את הגבול הבא:
	\[ \limz (\cosx)^{\frac{1}{\tan^2x}} \]} \begin{proof}[פתרון]
		נגדיר $f(x) = e^{\frac{\ln \cosx }{\tan^2 x}}$ לכל $x \in (-1, 1) \setminus \{0\}$. נבחין ש־$\limz \ln \cosx = 0 $ וכן $\limz \tan^2 x = 0$ מרציפות וכן שתיהן גזירות, וערכן $(\ln\cosx )' = \frac{-\sinx}{\cosx}$ וכן $(\tan^2x) = \frac{2 \tanx}{\cosx} \neq 0$. הגבול של החלוקה קיים וערכו:
		\[ \limz \frac{(\ln\cosx)'}{(\tan^2)x'} = \limz \frac{-\cancel{\tanx}}{2\cancel{\tanx} \cdot \frac{1}{\cos^2 x}} = \frac{-1}{2} \]
		מהרציפות. סה''כ מלופיטל $\limz \frac{\ln\cosx}{\tan^2x} = -\frac{1}{2}$ ולכן $\limz f(x) = e^{-0.5}$ וסיימנו.
	\end{proof}

	\exe{נתבונן בגבול הבא:
	\[ \limz \frac{\sinx - x}{x^{3}} \]
	מלופיטל:
	\[ \limz \frac{\sinx - x}{x^{3}} = \csb{\frac{0}{0}} \slh \limz \frac{\cosx - 1}{3x^{2}} = -\frac{1}{6} \]
	מהגבול הקודם (כלומר תיאורטית היינו צריכים להפעיל לופיטל פעמיים)
	}

	\section{נגזרות מסדר גבוה ופולינום טיילור}
	\defi{תהא $f \co I \to \R$ ויהי $x_0 \in \R$. ניתן להגדיר רקורסיבית את $f^{(n + 1)}(x_0) := (f^{(n)}(x_0))'$ כאשר $f^{(0)} = f$ בסיס. נבחין שלשם כך נדרוש ש־$f^{(n)}$ מוגדרת בסביבה של $x_0$. }
	\noti{לעיתים $f^{(n)}$ תסומן גם ב־$\frac{\dd^{n} f}{\dd x^{n}}(x_0)$. }
	\textbf{דוגמה: }נבחין שהפונקציה $f(x) = x^{m}$ עבור $m \in \N^{+}$ קבוע מתקיים:
	\[ f^{(n)}(x) = \begin{cases}
		\frac{m!}{(m - n)!}x^{m - n} & n \le m \\
		0 & \other
	\end{cases} \]
	באופן דומה:
	\begin{align*}
		f(x) &= \sinx &\implies& &f^{(n)}(x) &= \sin\cl{x + \frac{\pi n}{2}} \\
		f(x) &= \cosx &\implies& &f^{(n)}(x) &= \cos\cl{x + \frac{\pi n}{2}} \\
		f(x) &= e^{x} &\implies& &f^{(n)}(x) &= e^{x}
	\end{align*}

	\defi{תהא $f \co I \to \R$ וכן $x_0 \in I$. יהי $n \in \N$. נניח ש־$f$ גזירה $n$ פעמים ב־$x_0$. נגדיר את \textit{פולינום הטיילור של $f$ מסדר $n$ סביב $x_0$} ע''י:
	\[ T_n(x) := \sum_{i = 0}^{n}\frac{f^{(i)}(x_0)}{i!}(x - x_0)^{i} \]
	ואת השארית להיות:
	\[ R_n(x) := f(x) - T_n(x) \]
	}
	\lem{
	\begin{enumerate}
		\item $T_n$ גזירה מכל סדר
		\item $R_n$ גזירה $n$ פעמים ב־$x_0$
		\item לכל $i \in [n] \cup \{0\}$ בהכרח $R_n(x_0) = 0$ וכן $T_n(x_0) = f^{(n)}(x_0)$
	\end{enumerate}
	}

	\theo{מתקיים:
	\[ \limxz \frac{R_n(x)}{(x - x_0)^{n}} = 0 \]}

	\cola{תהא $f \co I \to \R$ ותהא $x_0 \in I$. יהי $n \in \N$ ונניח ש־$f$ גזירה $n$ פעמים ב־$x_0$. אז קיימת $\wg \co I \to \R$ המקיימת $\wg(x_0) = 0$ ו־$\wg$ רציפה בנקודה $x_0$, וגם:
	\[ R_n(x) = \wg(x)(x - x_0)^{n} \]}\begin{proof}
		ההוכחה בעיקרה נשארה לבית, אבל $\wg$ מוגדרת ע''י:
		\[ \wg(x) = \begin{cases}
			\frac{R_n(x)}{(x - x_0)^{n}} & x \neq x_0 \\
			0 & \other
		\end{cases} \]
		ואנחנו אמורים והמשיך מכאן.
	\end{proof}

	\lem{בהינתן $f, g \co A \to \R$ וכן $x_0$ נקודת הצטברות של $A$, אם $\limxz f(x) = \ml > 0$ וגם $\limxz g(x) = m$ אז מתקיים $\limxz (f(x))^{g(x)} = \ml^{m}$. }

	\exe{נגדיר $f(x) = \ln(1 + x)$ בתחום $(-1, \infty)$. אז לכל $n \in \N$ מתקיים:
	\[ f^{(n)}(x) = (-1)^{n}\frac{(n - 1)!}{(1 + x)^{n}} \]}
	\exe{נחשב את הגבול שראינו בתחילת ההרצאה, הוא $\frac{\ln\cosx}{\tan^{2}x}$ סביב $0$, לא באמצעות לופיטל אלא באמצעות טיילור. }\begin{proof}[פתרון]
		\[ \frac{\ln\cosx}{\tan^{2}x} = \underbrace{\cos^{2}x}_{\to 1} \cdot \underbrace{\frac{x^{2}}{\sin^{2}x}}_{\to 1} + \underbrace{\frac{\cosx - 1}{x^{2}}}_{\to-\frac{1}{2}} \cdot \underbrace{\frac{\ln(1 + (\cosx - 1))}{\cosx - 1}}_{\to 1} + \cdots \]
		כנראה מה שמחברים בסוף זניח כי משהו משהו $\wg$ משהו משהו $R_n$ ואני מקווה שירחיבו יותר בתרגול.
	\end{proof}

	\exe{נחשב את הגבול שראינו בתחילת ההרצאה, הוא $\frac{\ln\cosx}{\tan^{2}x}$ סביב $0$, לא באמצעות לופיטל ולא באמצעות טיילור אלא באמצעות כלים אלמטריים שכבר ראינו לפני ההרצאה. }\begin{proof}[פתרון]
		\[ (\cosx)^{\tan^{-2}x} = \underbrace{\cl{\cl{1 + \cosx - 1}^{\frac{1}{\cosx - 1}}}^{\overbrace{\frac{\cosx - 1}{\tan^{2}x}}^{\to -\frac{1}{2}}}\dequad\dequad\dequad\dequad}_{\to e} \quad\quad\quad\quad\!\!= e^{-0.5} \]
	\end{proof}

		\theo{תהאנה $f, g \co A \to \R$. תהא $x_0$ נקודת הצטברות של $A$. נניח כי $\limxz f(x) = \ml \in (0, \infty)$ וכן $\limxz g(n) = m \in \R$. אזי $\limxz f(x)^{g(x)} = \ml^{m}$. }\begin{proof}
		מרציפות $\ln$ וממשפט על רציפות והרכבה, נקבל $\limxz \ln(f(x)) = \ln(\ml)$. מאריתמטיקת גבולות $\limxz g(x) \cdot \ln f(x) = m \ln \ml$. מרציפות $e^{x}$ וממשפט על רציפות והרכבה, נקבל:
		\[ \climxz f(x)^{g(x)} = \climxz e^{g(x)\ln f(x)} = e^{m \ln l} = \ml^{m} \]
	\end{proof}
	\rmark{השתמשנו חזק בזה ש־$\ml$חיובי, כי $\ln$ רציפה רק בעבור מספרים חיוביים. }
	לא צריך להסביר את זה. אפשר להפעיל את זה במיידית. אם $\ml = 0$ צריך לעבוד יותר קשה. יש צורך גם לדבר בקצרה על זה שהגבול מוגדר, משום ש־$f(x)$ היא bounded away from zero.

	\theo{תהא $f(x) \co I \to \R$ גזירה פעמיים ב־$x_0 \in I$ (בפנים הקטע). נניח כי $f'(x_0) = 0$. אם $f''(x_0) > 0$ אז $x_0$ מינימום. אם $f''(x_0) < 0$ אז $x_0$ מקסימום. }
	ה''בעיה`` בהוכחה של הטענה הזו, היא שאנחנו יודעים ש־$f$ גזירה \textbf{רק ב־$x_0$}. לכן אי אפשר לעשות לגראנג'. זכרו שהגדרנו מינימום מקומי כמצב בו קיימת סביבה שבה כל הנקודות גדולות או שוות לנקודה. \begin{proof}
		ידוע $f$ גזירה פעמיים ב־$x_0$ ולכן קיימת סביבה של $x_0$ שבה $f$ גזירה (אחרת הנגזרת השנייה בנקודה אינה מוגדרת). ללא הגבלת הכליות $f$ גזירה בכל $I$ (למה בה''כ כי אפשר פשוט לצמצם ולהגדיר $\tl f = f|_I$, או להגדיר דלתא ולעשות מינימומים). מהנתון:
		\[ \lim_{x \to x_0^{-}} \frac{f'(x) - f'(x_0)}{x - x_0} = f''(x_0) > 0 \]
		ולכן קיים $\dg > 0$ כך שלכל $x_0 - \dg  < x < x_0$ מתקיים
		\[ \frac{f'(x) - f'(x_0)}{x - x_0} > \frac{f''(x_0)}{2} > 0 \]
		(כלומר הוא bounded away). לכן לכל $x_0 - \dg < x < x_0$ מתקיים $f'(x) < f'(x_0) = 0$. מנימוקים דומים קיים $\dg_2> 0$ כך שלכל $x_0 < x < x_0 + \dg_2$ מתקיים $\frac{f'(x) - f'(x_0)}{x - x_0} > \frac{f''(x_0)}{2} > 0$.
		לכן לכל $x_0 < x < x_0 + \dg_2$  מתקיים $f'(x) > f'(x_0) = 0$. נתבונן ב־$\dg = \min(\dg_1, \dg_2)$. יהי $x$. נניח $x$ בסביבת $\dg$ של $x_0$.
		\begin{itemize}
			\item אם $x = x_0$ אז $f(x) = f(x_0) \ge f(x_0)$ וסיימנו.
			\item נניח $x_0 - \dg < x < x_0$ אז בקטע $[x, x_0]$, $f$ מקיימת את תנאי משפט לגראנג' ומכאן קיימת $x \in (x, x_0)$ כך ש־$\frac{f(x_0) - f(x)}{x_0 - x} = f'(c) < 0$. לכן, ומכיוון ש־$x_0 > x$, בהכרח $f(x_0) < f(x)$.
			\item נניח $x_0 < x < x_0 + \dg$. בדומה.
		\end{itemize}
	\end{proof}
	עיקרי ההוכחה: להראות שמכיוון שיש גבול, מכן אחד מהצדדים כל הנגזרת היא bounded away from zero, ומכיוון שהיא גזירה ורציפה באיזושהי סביבה, אפשר להפעיל לגראנג' ולקבל את מה שצריכים.

	נוכל לתת הוכחה שקצת פחות נוגעת בגבולות ובדלתאות. ''פולינום טיילור, לא טור! לילי חזרה תשובה?``

	\begin{proof}[הוכחה נוספת]
		$f$ גזירה פעמיים ב־$x_0$, לכן לפי משפט קיימת $\wg \co I \to \R$ המקיימת כי $\wg$ רציפה ב־$x_0$, $\wg(x_0) = 0$ וגם לכל $x \in I$ מתקיים:
		\[ f(x) = f(x_0) + f'(x_0)(x - x_0) + f''(x_0)(x - x_0)^{2} + \wg(x)(x - x_0)^{2} \]
		ידוע $\limxz \wg(x) = 0$. מהרציפות של $\wg$ ב־$x_0$ קיים $\dg > 0$ כך שלכל $x_0 - \dg < x < x_0 + \dg$ כך ש־$\wg(x) > -\frac{f''(x_0)}{4}$ (רציפות, כי אנחנו רוצים סביבה שאיננה נקובה). יהי $x$ בסביבת ה־$\dg$ של $x$. אז (מהמשוואה הקודמת):
		\[ f(x) = f(x_0) + \underbrace{(x - x_0)^{2}}_{\ge 0} \cdot \underbrace{\cl{\frac{f''(x_0)}{2} + \wg(x)}}_{> \frac{f''(x_0)}{4} >  0} \ge f(x_0) \]
		וסיימנו.
	\end{proof}
	ישנה גרסה מוכללת באינדוקציה לטענה זו. (או לא באינדוקציה אם עושים עם טיילור)

	\theo{יהי $n \in \Nodd$. תהא $f \co I \to \R$ גזירה $n + 1$ פעמים ב־$x_0$. נניח $f^{(i)}(x_0) = 0$ וגם $f^{(n + 1)}(x_0) \neq 0$. אז אם $f^{(n + 1)}(x_0) > 0$ אז יש ל־$f$ מינימום ב־$x_0$. אם $f^{(n + 1)(x_0) < 0}$ אז יש ל־$f$ מקסימום ב־$x_0$. באותם התנאים, אם $n \in \Neven$ אז אין קיצון, יש פיתול. }

	\exe{חשבו את
	\[ \limz \frac{x^{2}\sinx}{x^{2} - \sin^{2}x} \]
	או הוכיחו שאינו קיים. }
	לופיטל יעבוד פה מתישהו. אבל good luck בלגזור את המונה. של עומד להתפשט ולגדול כמו סרטן בגוף של סבא שלי. אופציה אחת, ללהפוך את הלופיטל לנחמד, היא לבוא ולהגיד $x^{2}\sin^2x = x^{4} \cdot \frac{\sin^{2}x}{x^{2}}$. החלק הימני שואף ל־$1$, ולשאר אפשר לעשות לופיטל גם 4 פעמים וזה יהיה בסדר (מה איכפת לי לגזור מונום). זה חוקי מהטענה הבאה: נניח של־$f$ אין גבול ב־$x_0$, ונניח שהגבול $\limxz g(x) = \ml \neq 0$, אז $f \cdot g$ חסרת גבול ב־$x_0$. אם היינו מוציאים גבול שהוא איננו $0$, זה לא היה עובד, כלומר אם $\frac{\sin^{2}x}{x^{2}}$ היה שואף למקום אחר. במקום זה, נעשה טיילור. ולהלן הפתרון עם טיילור.

	\begin{proof}[פתרון]
		נפתח את $\sin^{2}x$ לפולינום מסדר $4$ עם שארית פאנו. נגדיר $f(x) = \sin^{2}x$. אז:
		\[ f(x) = \sin^{2}x \quad f'(x) = 2\sinx\cosx = \sin(2x) \quad f''(x) = 2 \cos(dx) \quad f'''(x) = -4\sin(2x) \quad f^{(4)}(x) = -8\cosx \]
		לכן קיימת $\wg \co \R\to \R$ כך ש־$\wg(0) = 0$, $\wg$ רציפה ב־$0$ וגם לכל $x$ מתקיים:
		\[ \sin^{2}x = 0 + 0x + \frac{2}{2}x^{2} + 0x^{3} + \frac{-8}{4!}x^{4} + \wg(x)x^{4} = x^{2} - \frac{1}{3}x^{4} + \wg(x)x^{4} \]
		לכן לכל $x$, מתקיים:
		\[ \frac{x^{2}\sin^{2}x}{x^{2} - \sin^{2}x} = \frac{x^{4} - \frac{1}{3}x^{6} + \wg(x)x^{6}}{\frac{1}{3}x^{4} - \wg(x)x^{4}} = \frac{1 - \frac{1}{3}x^{2} + \wg(x)}{\frac{1}{3} - \wg(x)} \rrr{x \to 0} \frac{1 - 0}{\frac{1}{3} - 0} = 3 \]
	\end{proof}
	למה היינו צריכים פולינום טיילור ממעלה 4? כי אחרת היינו מקבלים:
	\[ \sin^{2}x = x^{2} + \wg(x)x^{2} \implies \frac{\cdots}{\cancel{x^{2} - x^{2}} + \wg(x)x^{2}} \]
	כלומר המכנה הוא $0$ ואנחנו עדיין בבעיה.

	אגב, הנה הפתרון עם לופיטל.
	\begin{proof}[לופיתרון]
		\[ \frac{x^{2} - \sin^{2}x}{x^{4}} = \frac{x^{4}\overbrace{\frac{\sin^{2}x}{x^{2}}}^{\to 1}}{x^{2} - \sin^2x} \slh \frac{2x - \sin(2x)}{x^{3}} \slh \frac{2 - 2\cos 2c}{12x^{2}} = \frac{1}{6} \cdot \frac{1 - \cosx 2x}{x^{2}} \]
		בשלב הזה אפשר גם לעשות זהויות טריגו ולעשות את זה סבבה. אם הייתם עושים שוב לופיטל ועושים $\frac{2\sin(2x)}{2x}$ זה כבר מוגזם ו''אני הייתי מוריד נקודות`` (המרצה). ואיך עושים בלי לופי?
		\[ \frac{1 - \cos2x}{x^{2}} = \frac{2\sin^{2}x}{x^{2}} = 2 \cdot 1 \]
	\end{proof}

	\exe{חשבו את הגבול:
	\[ \limz (1 + \arctan x - x)^{\frac{1}{x^{3}}} \]
	או הוכיחו שאינו קיים. }

	\begin{proof}[פתרון]
		לכל $x \neq 0$, מתקיים:
		\[ (1 + \arctan x - x)^{\frac{1}{x^{3}}} = \cl{\cl{1 + \arctan x - x}^{\frac{1}{\arctan x - x}}}^{\frac{\arctan x - x}{x^{3}}} = e^{\frac{\arctan x - x}{x^{3}}} = \cdots \]
		כי כל הבפנוכו שואף ל־$e$ (ממשפט שהוכחנו בתרגילי הבית + היינה במבטא גרמני). אפשר לעשות פולינום טיילור. אפשר גם לעשות לופיטל. ידוע $\limz \arctan x - x = 0$ וגם $\limz x^{3} = 0$. הן גזירות ורציפות ואנחנו יודעים את הנגזרת ובלה בלה ולכן:
		\[ \frac{\arctan x - x}{x^{3}} \slh \frac{\frac{-x^{2}}{1 + x^{2}}}{3x^{2}} = -\frac{1}{3} \]
		אפשר גם לעשות טיילור ל־$\arctan$. נחזור למעלה:
		\[ \cdots = e^{-\frac{1}{3}} \]
	\end{proof}

	\rmark{אל תציבו חצי גבול. הדבר הזה:
	\[ \limz (2 + x)^{\frac{1}{x}} = \limz 2^{\frac{1}{x}} \]
	זה כמו הדבר הזה:
	\[ \frac{\can 64}{1\can 6} = 4 \]
	במקרה של $\frac{1}{x^{2}}$ ולא $\frac{1}{x}$ זה היה עובד, כי זה ב־$0$ שואף ל־$+\inft$, אבל צריך לנמק את כל זה. כי אפשר להגיד:
	\[ \limz (2 + x)^{\frac{1}{x^{2}}} = \limz e^{\frac{1}{x^{2}}\ln(2 +x)} \]
	וכל הדבר למעלה רציף ונחמד, ואפשר לעבוד איתו.
	}

	\rmark{טור טיילור סביב $0$ קרוי טור מק'לורן. כנ''ל על טורים. }
	עתה נוכיח משפט משיעור שעבר.
	\theo{תהא $f \co I \to \R$ ותהא $x_0 \in I$ נקודת םפנים. נניח $f$ גזירה $n$ פעמים ב־$x_0$. נסמן ב־$T_n$ את פולינום הטיילור של $f$ מסדר $n$ סביב $x_0$. נסמן ב־$R_n$ את השארית המתאימה. אז:
	\[ \limxz \frac{R_n(x)}{(x - x_0)^{n}} = 0 \]}
	\begin{proof}
		הבחנה: $T_n'$ הוא פולינום טיילור של $f'$ מסדר $n - 1$ סביב $x_n$, ויתר על כן, $R_n'$ היא השארית המתאימה. הסיבה:
		\[ \cl{\frac{f^{(k)(x_0)}}{k!}(x - x_k)^{k}}' = \frac{f^{(k)}(x_0)}{(k -1)!}(x -x_0)^{k} = \frac{(f')^{(k -1)(x_0)}}{(k - 1)!}(x -x_0)^{k} \]
		ההוכחה באינדוקציה על $n$.
		\begin{itemize}
			\item בסיס: עבור $n = 1$ נקבל:
			\[ \limxz \frac{R_1(x)}{x - x_0} = \limxz \frac{R_1(x) - R_1(x_0)}{x - x_0} = R_1'(x_0) = 0 \]
			\item נניח נכונות בעבור $n$ (\textbf{לא הנחנו ל־$\bm f$ ספציפית}). אז $T_{n + 1}'$ פולינום טיילור מסדר $n$ של $f'$ ו־$R_{n + 1}'$ השארית המתאימה (כי הנגזרת לינארית והכל). לכן:
			\[ \limxz \frac{R'_{n + 1}(x)}{(x - x_0)^{n}} = 0 \]
			אפשר לעשות לופיטל. אפשר גם לעשות לגראנג'. יהי $\eg > 0$. מכאן שקיים $\dg > 0$ כך שלכל $x \in I$, אם $0< \sof{x - x_0} < \dg$, אז:
			\[ \sof{\frac{R'_{n + 1}(x)}{(x - x_0)^{n}}} < \eg \]
			יהי $x \in I$. נניח $0 < \sof{x - x_0} < \dg$. בקטע שבין $x$ ל־$x_0$, $R_{n + 1}$ מקיימת את תנאי משפט לגארנג'. לכן קיים $c$ בין $x$ ל־$x_0$ כך ש־:
			\[ \frac{R_{n + 1}(x) - R_{n + 1}(x_0)}{x - x_0} = R_{n + 1}'(c) \]
			סה''כ (ניעזר בכך ש־$R_{n + 1}(x_0) = 0$ ושהנגזרת בנקודה הזו היא $0$):
			\[ \sof{\frac{R_{n + 1}(x)}{(x - x_0)^{n + 1}}} = \sof{\frac{\frac{R_{n + 1}(x) - R_{n + 1}(x_0)}{x - x_0}}{(x - x_0)^{n}}} = \sof{\frac{R'_{n + 1}(c)}{(x -x_0)^{n}}} < \sof{\frac{R'_{n + 1}(c)}{(c - x_0)^{n}}} < \eg \]
			ואז כנראה סיימנו.
		\end{itemize}
	\end{proof}
	''לופיטל גורם לריפיון שכל. הוא גורם לסטונדטים לעשות דברים מטופשים``. ואז המרצה מסביר איך לופיטל זה כמו לחצות את הכביש לא במעבר חציה.

	\subsection*{פולינום טיילור עם שארית לגראנג'}
	עד עכשיו עבדנו עם שארית פאנו. אנחנו רוצים יותר כי אנחנו גרידי. לא מספיק למרצה שהשארית שואפת ל־$0$ יותר מהר מ־$x^{n}$, הוא רוצה יותר מזה.

	\noti{נגדיר את $C^{(n + 1)}(A)$ את קבוצת הפונקציות הגזירות ברציפות ב־$I$. }
	\theo{תהא $f \co I \to \R$ ותהא $x_0 \in \R$ בפנים הקטע. נניח כי $f$ גזירה $n + 1$ פעמים בכל $I$ ונגזרותיה רציפות (כלומר $f \in C^{(n + 1)}$). לכל $x \in I$ קיים $c$  בין $x_0$ ל־$x$ כך ש־:
	\[ R_{n}(x) = \frac{f^{(n + 1)}(c)}{(n + 1)!}(x - x_0)^{n + 1} \]}
	ההוכחה נעשית ע''י משפט קושי.

	איך נשתמש במשפט הזה?
	\begin{enumerate}
		\item הערכת הקירוב.

		\textbf{דוגמה: }נחשב את $\sin 1$ עם שגיאה של לכל היותר $\frac{1}{1000}$. נגדיר $f(x) = \sinx$ ונפתח את $f$ לפולינום מק'לורן מסדר $7$.
		\[ \sinx = x - \frac{x^{3}}{3!} + \frac{x^{5}}{5!} - \frac{x^{7}}{7!} + R_7(x) \]
		כלומר:
		\[ \sin1 = 1 - \frac{1}{6} + \frac{1}{120} - \frac{1}{7!} + R_7 \]
		תחשבו את זה להנאתכם בלי מחשבון. לפי פיתוח שארית לגראנג', קיים $c \in (0, 1)$ כך ש־:
		\[ R_7(1) = \frac{f^{(8)(c)}}{8!}(1 - 0)^{8} \]
		מכיוון ש־$c \in (0, 1)$ בהכרח $\sof{f^{(8)}(c)} \le 1$. מכאן:
		\[ \sof{R_7(1)} \le \frac{1}{8!} < \frac{1}{1000} \]
		אגב, השארית האמיתית היא משהו בסביבת $0.000002730839643$.
		\item הוכחת התכנסות של {\bf \Large \textit{טור הטיילור}} לפונקציה עצמה.

		\textbf{דוגמה: }
		נגדיר $f(x) = \sinx$. יהי $x \in \R$. יהי $\eg > 0$. קיים $n \in \N$ כך ש־$\sof{\frac{x^{n}}{n!}} < \eg$ (סטירלינג, וגם הוכחנו בלי. שימו לב שה־$n$ תלוי ב־$\eg$ וב־$x$). לפי פיתוח מק'לורן של סינוס עם שארית לגראנג':
		\[ \sof{\sinx - \sumnk (-1)^{k}\frac{x^{2k + 1}}{(2k + 1)!}} = \sof{R_{2n + 1}(x)} \le \sof{\frac{1}{(2n + 2)!}x^{2n + 2}} < \eg \]
		קבענו את $x$. לכן אנחנו בטור חמודי:
		\[ \sinx = \limsi \sumnk (-1)^{k}\frac{x^{2k + 1}}{(2k + 1)!} = \sum_{k = 0}^{\infty} (-1)^{k} \frac{x^{2k + 1}}{(2k + 1)!} \]
	\end{enumerate}

	\href{https://eincyclopedia.org/wiki/\%D7\%98\%D7\%95\%D7\%A8\_\%D7\%98\%D7\%99\%D7\%99\%D7\%9C\%D7\%95\%D7\%A8}{מבולבלים מטיילור? אני ממש ממליץ על הסיכום הבא (זה קישור לחיץ)}

	יש לטורי טיילור יותר כוח ממה שאנחנו רואים כאן. אגב, בהקשר ל־$\sinx$ שטור הטיילור שלה מתכנס, זה לא נכון לכל פונקציה. לדגוהמ אם מגדיר $f \co \R\to \R$.
	\[ f(x) = \begin{cases}
		e^{-\frac{1}{x^{2}}} & x \neq 0 \\
		0 & \other
	\end{cases} \]
	אפשר להוכיח באינדוקציה של $f$, שיש לה נגזרת מסדר $n$ ב־$0$, ושלכל $n \in \N^{+}$ מתקיים $f^{(n)}(0) =0$. הוא לא מתכנס לפונקציה. נ.ב. זו פונקציה מוכרת עם שימושים בסטטיסטיקה או משהו כזה.

	יש פונקציות שעבורן הטור לא מתכנס בכלל. לדוגמה $\frac{1}{1 - x}$ שטורה $\sum_{n = 0}^{\inft}x^{n}$, לא מתכנס בתחומים מסויימים.


	טור הטיילור לא יכול להיות כיאוטי יותר מדי – הראינו שקיים רדיוס התכנסות, ואומנם לא ברור מה קורה בקצוות שלו, אבל בכלים של חדו''א 2א אפשר להראות שטור חזקות רציף בתחום הזה.

	האמריקאים מזיזים את נושאת המטוסים שלהם מסין לאיראן. שיעור הבא תלוי במצב הרוח של טראמפ.

	\begin{proof}
		באינדוקציה על $n \in \N^{+}$.
		\begin{itemize}
			\item \textbf{בסיס: }ב־$n = 0$ נקבל שפולינום הטיילור קבוע, וערכו $T_n = f(x_0)$. מכאן $R_0(x) = f(x)- f(x_0)$. נבחין ש־$f$ מקיימת את תנאי משפט לגראנג' בקטע שבין $x$ ל־$x_0$. לכן קיים $c$ בין $x$ ל־$x_0$ כך ש־:
			\[ R_0(x) = f(x) - f(x_0) = f'(c)(x -x_0) \]
			\item \textbf{צעד: }נניח באינדוקציה על $n$ ונוכיח ל־$n + 1$. יהי $x_0 \neq x \in I$. $R_{n + 1}$ ו־$(x - x_0)^{n + 2}$ גזירות בין $x$ ל־$x_0$ והנגזרת של $(x - x_0)^{n + 2}$ אינה מתאפסת בקטע הפתוח שבין $x$ ל־$x_0$. לכן קיים $c$ בין $x$ ל־$x_0$, כך ש־:
			\[ \frac{R_{n + 1}(x)}{(x - x_0)^{n + 2}} = \frac{R_{n + 1}(x) - R_{n + 1}(x_0)}{(x - x_0)^{n + 2} - 0^{n + 2}} = \frac{R_{n + 1}'(c)}{(n + 2)(c - x_0)^{n + 1}} \]
			זאת ממשפט קושי. $R_{n + 1}'$ היא השארית בפיתוח של $f'$ מסדר $n$ סביב $x_0$ (הוכחנו את זה כחלק מהוכחה הקודמת). מה.א. קיים $d$ בין $c$ ל־$x_0$ כך ש־:
			\[ R'_{n + 1}(c) = \frac{(f')^{n + 1}(d)}{(n + 1)!}(c - x_0)^{n + 1} \]
			לכן:
			\[ \frac{R_{n + 1}'(c)}{(c - x_0)^{n +1}} = \frac{f^{(n + 2)}(d)}{(n + 1)!} \]
			מציבים כל הדרך למעלה, מקבלים:
			\[ \frac{R_{n + 1}(x)}{(x - x_0)^{n +1}} = \frac{f^{(n + 2)}(d)}{(n+2)!} \implies R_{n + 1} = \frac{f^{(n + 2)(d)}}{(n + 2)!}(x - x_0)^{n + 2} \]
		\end{itemize}\envendproof
	\end{proof}
	\defi{מסמנים ב־$C^{\inft}(A)$ את קבוצת הפונקציות הגזירות (ובפרט רציפות) מכל סדר ב־$A$. }
	\theo{תהא $f \in C^{\inft}(A)$. אם קיים $M > 0$ כך ש־$\forall n \in \N \forall x \in I \co \sof{f^{(n)}(x)}\le M$ (''הנגזרות חסומות באופן אחיד``), אז טור טיילור של $f$ מתכנס ל־$f$ בכל $I$. }

	\theo{טור הטיילור של $e^{x}$ מתכנס ל־$e^{x}$ בכל נקודה, כלומר $\forall x \in \R\co e^{x} = \sum_{n = 0}^{\inft}\frac{x^{n}}{n!}$. }
	לא נוכל להשתמש במשפט הקודם, כי הנגזרות לא חסומות. כן נוכל להוכיח התכנסות.
	\begin{proof}
		יהי $\hat x \in \R$. נסמן $I = (\sof{\hat x} - 1, \sof{\hat x} + 1)$. בקטע $I$ כל הנגזרות של $e^{x}$, חסומות ע''י $e^{\sof{\hat x}} + 1$ (הסיבה שצריך ערך מוחלט: כי צריך קטע שכולל גם את $0$ וגם את $\hat x$, כי זו הנקודה סביבה הטור מפותח). לכן, מהמשפט, לכל $x \in I$ מתקיים $e^{x} = \sum_{i = 0}^{\inft}\frac{x^{n}}{n!}$. בפרט $e^{\hat x} = \sum_{i = 0}^{\inft}\frac{x^{n}}{n!}$.
	\end{proof}
	במילים אחרות, מה שצריך בפועל זה שהנגזרות יהיו חסומות בכל קטע קומפקטי. רק שאת זה לא הפכו למשפט בקורס.


		נחזור על שארית לגראנג':

	נתבונן בפונקציה $f \co I \to \R$ ותהא $x_0 \in I$ פנימית. יהי $n \in \N^{+}$ ונניח כי $f$ גזירה $n + 1$ פעמים ב־$I$. (שימו לב: לא ב־$x_0$ אלא בכל הקטע). אז לכל $x \in I$ קיים $c$ בין $x$ ל־$x_0$ כך ש־$R_n(x) = \frac{f^{(n + 1)}(c)}{(n + 1)!}(x - x_0)^{n + 1}$.

	נחזור על שארית פאנו:

	תהי $\wg \in I \to \R$ כך ש־$\wg(x_0) = 0$. $\wg$ רציפה ב־$x_0$ וכן $R_n(x) = \wg(x)(x - x_0)^{n}$ לכל $x \in I$.

	נתבונן בהכללה הבאה של שארית לגראנג':
	\theo{יהי $ \le p \le n + 1$ ונתבונן בפונקציה $f \co I \to \R$ ותהא $x_0 \in I$ פנימית. יהי $n \in \N^{+}$ ונניח כי $f$ גזירה $n + 1$ פעמים ב־$I$. אז לכל $x \in I$ קיים $c$ בין $x_0$ ל־$x$ כך ש־$R_n(x) = \frac{f^{(n + 1)}(c)}{פ p \cdot n!}(x - x_0)^{p}(c - x_0)^{n + 1 - p}$}

	כאשר $p = n + 1$ נקבל בדיוק את שארית לגראנג'. כאשר $p = 1$, השארית נקראת שארית קושי.

	\begin{proof}
		יהי $x \in I$. נגדיר $\phi, \psi \co I \to \R$ באופן הבא:
		\begin{align*}
			\phi(t) := f)x) - f(t) - \frac{f'(t)}{1!}(x - t(x)) - \cdots - \frac{f^{(n)}(t)}{n!}(x - t)^{n} && \psi(t) = (x - t)^{p}
		\end{align*}

		נבחין בכמה דברים: ראשית כל $\phi(x) = \psi(x) = 0$. עוד נבחין $\phi(x_0) = R_n(x)$ ו־$\psi(x_0) = (x - x_0)^{p}$. לכל $t \in I$ נבחין שההנגזרת של $\phi$ היא כמו טור טלסקופי:
		\[ \phi(t) = -f'(t) + f'(t) - f''(t)(x - t) + f'''(x - t) -\frac{f'''(t)}{2!}(x - t)^{2} + \cdots  = -\frac{f^{(n + 1)}(t)}{n!} \]
		קל יותר למצוא את $\psi'$ ולסכם ש־$\psi'(x) = -p(x - t)^{p -1}$. נשים לב שבקטע בין $x$ ל־$x_0$ שתי ה]פונקציות קציפות בקטע הסוגר, רציפות בקטע הפתוח, ו־$\psi$ אינה מתאפסת בקטע הפתוח. מקושי קיים $c$ בין $x$ ל־$x_0$ כך ש־:
		\[ \frac{R_n(x)}{(x - x_0)^{2}} = \frac{\phi(x) - \phi(x)}{\psi(x) - \psi(y)} = \frac{\phi'(c)}{\psi'(c)} = \frac{\frac{f^{(n + 1)}(c)}{n!}(x - c)^{n}}{p(x - c)^{o - 1}} = \frac{f^{(n + 1)}(c)}{p \cdot n!}(x - c)^{n - p + 1} \]

	\end{proof}


	\exe{נגדיר $I = (-1, 1, f \co)$ גזירה מכל סדרה ומתקיים $\forall x \in (-1, \infty)$ לכל $n \in \N^{+}$ מתקיים $f^{(n)}(x) = (-1)^{n - 1}\frac{(n - 1)!}{(1 + x)^{n}}$}. אז עבור $x \in (-1, \inft)$ נקבל כי שארית לגראנג' של פולינום מק'לורן מסדר $n$ היא $R_n(x) = (-1)^{n}\frac{n!}{(n + 1)!(1 + c)^{n + 1}}c^{n + 1} = (-1)^n\cdot \frac{1}{n + 1} \cdot \frac{x^{n}}{(1 + c)^{n + 1}}$ עבור $c$ בין $x$ ל־$0$ כלשהו.
	כאשר $0 \le x \le 1$ נגיד למשל ש־:
	\[ \sof{R_n(x)} \le \frac{1}{n + 1} \cdot \frac{\sof{x}^{n + 1}}{(a + x)^{n + 1}} \le \frac{1}{n + 1} \toinf 0 \]
	מכאן $\sof{x^{n + 1}} \le 1$ וכן $(1 + c)^{n + 1} \ge 1$.

	כאשר $-\frac{1}{2} \le x \le 0$ אז $\sof{R_n(x)} = \frac{1}{n + 1} \cdot \frac{\sof{x}^{n + 1}}{(1 + c)^{n + 1}}$ נבחין ש־$-\frac{1}{2} \le x < c < 0$	 ומכאן $\sof{x} \le \frac{1}{2} \le 1 + 2x < 1 + c < 1$ ןאז $\frac{\sof{x}}{1 + c} \le 1$.

	סה''כ לכל $x \in \csb{-\frac{1}{2}, 1}$ מתקיים $R_n(x) \toinf 0$.


	עבור $-1 > x < -\frac{1}{2}$ לא נוכל לחסום את $\frac{\sof x}{c + 1}$ על ידי $1$. ואז שארית לגראנג' תפסיק לעבוד. למקרה זה נשתמש בשארית קושי. לכל $x \in (-1, -0.5)$ ולכל $n \in \N$, קיים $x < c < 0$ כך ש־:
	\[ \sof{R_n(x)} = \sof{\frac{\frac{(-1)^{n}n!}{(1 + c)^{n + 1}}}{n!}(x - c)^{n}(x - 0)} = \frac{\sof x}{1 + c} \cdot {\sof{\frac{x - c}{1 + c}}^{n}} = \cdots \]

	שימו לב! $c$ תלוי ב־$x$ בכל הדברים לעיל. יש כאן טריק קטן שפותר את זה: (ניעזר בכך ש־$c, x$ שליליים)
	\[ \sof{x - c} = \sof x - \sof c < \sof x - \sof x \sof c = \sof x (1 - c) = \sof x (1 + c) \]
	נחזור למעלה:
	\[ \cdots < \frac{\sof{x}^{n + 1}}{1 + c} \toinf 0 \]
	נבחין ש־$1 + c$ חסום בין $0$ ל־$x$ (למרות שהוא עדיין תלוי ב־$n$!) ואנחנו מחלקים משהו מעריכי בקבוע, ולכן כל הסיפור לעיל הולך ל־$0$.


	\exe{הוכיחו/הפריכו:
	\begin{enumerate}
		\item \hfil $\limxz f'(x) = f'(0)$
		\item קיימת סדרה $\{x_n\}$ השואפת ל־$0$ עבורה $\limsi f'(x_n) = f'(0)$, וכן $x_n \neq 0$ לכל $n$.
		\end{enumerate}}
	\begin{enumerate}
		\item נראה דוגמה נגדית ל־$1$. נגדיר:
		\[ f(x) = \begin{cases}
			x^{2}\sin\cl{\frac{1}{x}} & x \neq 0 \\
			0 & x = 0
		\end{cases} \]
		אזי לכל $x \neq 0$ $f$ גזירה כמהפלה והרכבה של גזירות. ב־$0$:
		\[ \limh \frac{f(0 + h) - f(0)}{h} = \limh h \sin \cl{\frac{1}{h}} = 0 \]
		לכן:
		\[ f'(x) = \begin{cases}
			2x\sin \frac{1}{x} - \cos \frac{1}{x} & x \neq 0 \\
			0 & x = 0
		\end{cases} \]
		הגבול $\limxz 2\sin\cl{\frac{1}{x}} = 0$ וכמו כן ל־$\cos\frac{1}{x}$ ללא גבול ב־$0$, ולכן $f'$ ללא גבול ב־$0$ מאריתמטיקת גבולות.
		\item זו הוכחה. נציג שני פתרונות. \begin{proof}[פתרון 1]
			נגדיר $x_1 =1$.
			עבור $n \in \N$ נסמן $a = \frac{x_n}{2}$. ממשפט דרבו $f'$ מקיימת את תכונת דרבו (תכונת ערך הביניים). נסמן $\lg = \min\{f'(0) + \frac{1}{n}, f'(a)\}$ (בה''כ $f'(a) > f'(0)$, אחרת נחסר). קיים $x_{n + 1} \in [0, a)$ כך ש־$f'(x_{n + 1}) = \lg$. לכל $n \in \N$ נקבל $\sof{x_n} \le \frac{1}{2^{n}}$ וגם $\sof{f'(0) - f'(x_n)} \le \frac{1}{n}$.
		\end{proof}
		עתה נציג פתרון נוסף.
		\begin{proof}[פתרון 2]
			מהיינה $\limsi \frac{f(\frac{1}{n}) - f(0)}{\frac{1}{n}} = f'(0)$ (בחרנו ספציפית מבין כל הסדרות השואפות ל־$0$). לכל $n \in \N$, $f$ מקיימת את תנאי משפט לגראנג' ב־$[0, \frac{1}{n}]$ ולכן קיים $0 < x_n < \frac{1}{n}$ כך ש־:
			\[ \frac{f\cl{\frac{1}{n}} - f(0)}{\frac{1}{n}} = f'(x_n) \]
			ו־$0 \neq x_n \to 0$ וכמובן $f'(x_n) \to f'(0)$.
		\end{proof}
	\end{enumerate}

	עתה נתבונן בעוד שאלה שהייתה בתרגיל הבית: תהי $f \co \R \to \R$ פונקציה גזירה פעמיים ב־$x_0 \in \R$. הראו כי:
	\[ \limh \frac{f(x_0 + h) - f(x_0 - h) - 2f(x_0)}{h^{2}} = f''(x_0) \]
	בבית עשינו כולנו עם לופיטל. שימו לב לנמק הכל ובשום פנים ובאופן לא לעשות לופיטל פעמיים (נתון שהיא גזירה רק פעמיים).
	\begin{proof}[הוכחה באמצעות טיילור]
		נפתח את פולינום טיילור מסדר $2$ סביב $x_0$ משארית פאנו. קיימת $\wg \co \R \to \R$ כך ש־$\wg(x_0) = 0$, $\wg$ רציפה ב־$x_0$, וכן $\forall x \in \R\co f(x) = f(x_0) + f'(x_0)(x - x_0) + \frac{f''(x_0)}{2}(x - x_0)^{2} + \wg(x)(x - x_0)^{2}$. יהי $h \in \R\setminus \{0\}$. נבחין ש־:
		\begin{gather*}
			f(x_0 + h) = f(x_0) + f'(x_0)h + \frac{f''(x_0)h^{2}}{2} + \wg(x - h)h^{2} \\
			f(x_0 - h) = f(x_0) + f'(x_0)(-h) + \frac{f''(x_0)h^{2}}{2} + \wg(x - h)h^{2}
		\end{gather*}
		נחזור לביטוי למעלה, ונקבל:
		\[ \frac{f(x_0 + h) + f(x_0  - h) - 2f(x_0)}{h^{2}} = f''(x_0) + \wg(x_0 + h) + \wg(x_0 - h) \]
		ובגבול:
		\[ \limh \frac{f(x_0 + h) + f(x_0  - h) - 2f(x_0)}{h^{2}} = f''(x_0) \]
	\end{proof}

	כך נראה הפתרון עם לופי:
	\begin{proof}[הוכחה באמצעות לופי]
		נגדיר $g(h) = f(x_0 + h) + f(x_0  - h) - 2f(x_0)$ וכן $t(h) = h^{2}$. ידוע ש־$f$ גזירה ולכן רציפה ב־$x_0$, כלומר $\limh g(h) = 0$ וכמו כן $\limh t(h) = 0$. ידוע ש־$g, t$ גזירות בסביבת $0$, כיוון ש־$g$ גזירה פעמיים ב־$x_0$. $t'$ לא מתאפסת למעט ב־$0$, כי $t'(h) = 2h, g'(h) = f'(x_0 + h) - f'(x_0 -h)$. עכשיו נשתמש בכלל לופיטל:
		\begin{gather*}
			\limh \frac{g'(h)}{t'(h)} = \limh \frac{f'(x_0 + h) - f'(x_0 - h)}{2h} = \frac{1}{2}\cl{\limh \frac{f'(x_0 + h) - f'(x_0)}{h} + \frac{f'(x_0) - f'(x_0 - h)}{h}}
		\end{gather*}
		נחשב את הגבולות בנפרד:
		\begin{align*}
			\limh \frac{f'(x_0 + h) - f'(x_0)}{h} = f''(x_0) && \limh \frac{f'(x_0) - f'(x_0 - h)}{h} \overset{\ag = -h}{=} \lim_{\ag \to 0}\frac{f'(x_0) - f'(x_0 + \ag)}{-\ag} = f''(x_0)
		\end{align*}
		ואפילו כיצד לציין שהחלפת המשתנה נובעת ממשפט על גבולות והרכבה ולציין את תנאיו.
	\end{proof}
	שימו לב – לעשות כאן לופיטל פעמיים זה פטאלי! זה לא נכון ולא נתון שהפונקציה מקיימת את התנאים של כלל לופיטל.

	\exe{תהא $f\co \R \to \R$. נניח $f$ גזירה, $f'$ רציפה במ''ש ואינה חסומה. הראו כי $f$ אינה רציפה במ''ש. }\begin{proof}
		נניח בה''כ ש־$f'$ חסומה מלעיל (אחרת נעבוד עם $-f$).

		אין לנו שום דרך כמעט להראות שפונקציה איננה רציפה במ''ש, אלא לפי הגדרה. לכן נתבונן ב־$\eg = 1$ כלשהו (מקסימום נתקן אותו אחכ), ותהא $\dg > 0$. $f'$ רציפה במ''ש ולכן קיים $\dg_1$ כך ש־$\forall x, y \in \R \co \sof{x - y} < \dg_1 \implies \sof{f'(x)- f'(y)} < \dg$. נסמן $r = \frac{1}{3}\min\{\dg_1, \dg, \frac{1}{\dg}\}$. קיים $t \in \R$ כך ש־$f'(t) > \frac{1}{r}$. לכל $x \in [t - r, t + r]$, מתקיים ש־$\sof{f'(x) - f'(t)} < \dg$. מלגראנג', קיים $c \in (t - r, t + r)$ כך ש־:
		\[ \frac{f(t + r) - f(t - r)}{2r} = f'(c) > \frac{1}{r} - \dg \implies \sof{f(t + r) - f(t - r)} > 2 - 2 r \dg \ge 1 \]
		כמו כן $\sof{t + r - (t - r)}  = 2r < \dg$. סה''כ סתירה.
	\end{proof}

	''גבול אחרון והביתה``

	\exe{נחשב את הגבול הבא:
	\[ \climi x\cl{\cl{1 + \frac{1}{x}}^x - e} \]
	אפשר להתייחס ל־$x$ כאל $\frac{1}{1/x}$. ואז נעשה לופיטל ונקבל במונה:
	\begin{gather*}
		\cl{1 + \frac{1}{x}}^{x} = e^{x \ln\cl{1 + \frac{1}{x}}} = e^{x\ln(1 + \frac{1}{x})}\cl{\ln\cl{1 + \frac{1}{x}} + x \cdot \frac{1}{1 + \frac{1}{x}} \cdot \cl{-\frac{1}{x^{2}}}} =
	x^{2}\cl{1 + \frac{1}{x}}^{x}\cl{\ln\cl{1 + \frac{1}{x}} - \frac{1}{1 + x}}
	\end{gather*}
	וזה ממש לא עוזר. זה יעזור מתישהו, אבל זה יהיה כואב. ונצטרך הרבה לופי.

	נבצע החלפת משתנים, כי כרגע הגבול הוא לאינסוף וזה לא מאפשר (בכלים הנוכחיים שלנו) להשתמש בטיילור. אי אפשר לעשות טיילור סביב אינסוף, אפשר להשתמש במשפט טיילור עבור פולינום טיילור סביב נקודה כלשהי.

	לכן נבצע החלפת משתנים – $t = \frac{1}{x}$. ואז נקבל:
	\[ = \lim_{t \to 0} \frac{1}{t}\cl{\cl{1 + t}^{\frac{1}{t}} - e} \]
	זו לא פונקציה שמוגדרת בכלל באפס. נגדיר למען הנוחות:
	\[ f(t) = \begin{cases}
		(1 + t)^{\frac{1}{t}} & t \neq 0 \\
		e & \other
	\end{cases} \]
	נבחין שהיא רציפה. ואז נקבל את הגבול:
	\[ \cdots = \lim_{t \to 0} \frac{f(t) - e}{t} \]
	זו ליטרלי הגדרת הנגזרת של $f$! אך הנגזרת של $f$ אינה מוגדרת ב־$0$. לכן טיילור (סביב $0$) לא עובד, באופן ישיר, כי טור הטיילור צריך ממש להחזיק את הנגזרות ביד כדי שנוכל לפתח אותו.

	אבל, $f(t)$ רציפה וגזירה בכל $t \neq 0$. שם נקבל:
	\[ f'(t) = \cl{e^{\frac{1}{t}\ln(1 + t)}}' = e^{\frac{1}{t}\ln(1 + t)}\cl{-\frac{1}{t^{2}}\ln(1 + t) + \frac{1}{t} \cdot \frac{1}{1 + t}} = \cl{1 + t}^{\frac{1}{t}}\cl{\frac{-(t + 1)\ln(1 + t) + t}{t^{2}(t + 1)}} = \xi \]
	ה־$t + 1$ בחילוק למטה באפס לא מפריע לנו, וגם לא ה־$(1 + t)^{\frac{1}{t}}$ ששואף ל־$e$ ואם כל השאר יעבוד אז נוכל פשוט להשתמש באריתמטיקה. נוציא אותם החוצה:
	\[ \xi = (1 + t)^{\frac{1}{t}} \cdot \frac{1}{t + 1}\cl{\frac{t - t \ln (t + 1) - \ln (t + 1)}{t^{2}}} = (1 + t)^{\frac{1}{t}} \cdot \frac{1}{t + 1}\Bigg(\underbrace{\frac{t - \ln (1 + t)}{t^{2}}}_{\mathclap{לופיטל בצד}} - \underbrace{\frac{\ln(1 + t)}{t}}_{1}\Bigg ) = \xi \]

	הלופיטל בצד הפך להיות טיילור בצד:
	\[ \frac{t - \ln(1 + t)}{t^{2}} = \frac{t - \frac{t^{2}}{2} + \wg(t)t^{2}}{t^{2}} = -\frac{1}{2} \]
	סה''כ:
	\[ \lim_{t \to 0} f'(t) = \xi = -\frac{e}{2} \]
	זו הפואנטה, עכשיו צריך לתפור הכל. כמו שאמרנו הגבול הזה צריך לצאת הגבול המקורי כי הוא ליטרלי נגזרת לפי הגדרה.
	ניזכר בטענה שראינו: תהא $f \co I \to \R$ גזירה. תהא $x_0$ בקטע ונניח שקיים וסופי הגבול $\limxz f'(x)$, אז הגבול שווה ל־$f'(x_0)$. או במילים אחרות, אם קיים גבול לנגזרת, היא רציפה (כי מדרבו אין נקודות אי־רציפות סליקות ואין נקודות אי־רציפות מסוג ראשון. כל האי רציפות \textit{רע}. זה נכון לגבי כל פונקציה שמקיימת את תכונת חרבו דרבו).

	הנגזרת לא מוגדרת ב־$0$, ועשינו גבול של הנגזרת. אך הנגזרת לא מוגדרת ב־$t = 0$. מהמשפט, נבין שאם הגבול המקורי אכן קיים, אז הוא אכן שווה ל־$-\frac{e}{2}$. אז עכשיו רק נותר להראות שהגבול המקורי שווה ל־$-\frac{e}{2}$. ואפשר לדעת שהגבול קיים! היא גזירה בסביבה נקובה של $0$, כנ''ל לגבי $t$, ולכן אפשר לעשות לופיטל. למעשה לא היינו צריכים לעשות את כל הבלגן עם הנגזרת כי אחרי החלפת משתנים זה פותר. המרצה סתם רצה לדבר על המשפט לעיל.
	}

	ההבדל בין לופיטל בטיילור – הגזירות בנקודה עצמה, בטיילור צריך רק אותו, בלופיטל לא צריך אותו בכלל.


\end{document}
