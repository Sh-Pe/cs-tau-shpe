\documentclass[]{../../../../tex/classes/homework}
\usepackage{../../../../tex/packages/hebrewSupport}
\usepackage{../../../../tex/packages/mathShortcuts}
\usepackage{../../../../tex/packages/theoremsSupport}

\author{שחר פרץ}
\title{\textit{חדו''א 1א} $\sim$ תרגיל בית 2}
\begin{document}
	\maketitle
	\section{}
	תהי $A \subseteq \R$ קבוצה לא ריקה ויהי $s \in \R$. נוכיח $s$ החסם העליון של $A$ אמ''מ $s$ חסם מלעיל מינימלי. 
	\begin{proof}
		תהי $A \subseteq \R$ ו־$\ag \in \R$. נוכיח שקילות באמצעות הוכחת גרירה דו־כיוונית. 
		\begin{itemize}
			\item[$\implies$]נניח $\ag$ חסם עליון של $A$. נוכיח שהוא חסם מלעיל מנימלי. מהיותו חסם עליון, ידוע שהוא חסם מלעיל. נוכיח שהוא מינימלי. יהי $\bg \in \R$ חסם מלעיל של $A$. נניח בשלילה $\bg < \ag$, אז בעבור $\eg = \ag - \bg$ קיים $a \in A$ כך ש־$a > \ag - \eg = \ag - (\ag  - \bg) = \bg$, ומכאן ש־$\bg$ אינו חסם מלעיל של $A$ וסתירה. 
			\item[$\impliedby$]נניח $\ag$ חסם מלעיל מינימלי, נוכיח שהוא חסם עליון. יהי $\eg > 0$. אז נניח בשלילה שלא קיים $a \in A$ כך ש־$a > \ag - \eg$, ואז $\forall a \in A \co a < \ag - \eg$ כלומר $\ag - \eg$ חסם מלעיל של $A$ מהגדרה, אך $\ag - \eg < \ag$ וזו סתירה למינימליות של $\ag$ מבין החסמים מלעיל. סה''כ בהכרח קיים $a$ המתאים לתנאי וסיימנו. 
		\end{itemize}
	\end{proof}
	
	\section{}
	תהיינה $A, B \subseteq \R$ קבוצות לא ריקות. נוכיח או נפריך את את הטענות הבאות. 
	\begin{enumerate}[(A)]
		\item נוכיח שאם ל־$A$ אין איבר מקסימלי אז $A$ אינסופית. \begin{proof}
			תהי $A$ קבוצה ללא איבר מקסימלי. נניח בשלילה שהיא סופית. אזי $\max A$ מוגדר (ממשפט הרקורסיה: ידוע קיום זיווג $f \co [n] \to A$ ואז הפונקציה $m_1 = f(1)$, ותנאי נסיגה $m_{n + 1}= \max\{m_k, f(k + 1)\}$ מגדירה את $\max A := m_n$)וחוסם את הסדרה מלמעלה, וסיימנו. 
		\end{proof}
		\item נפריך את הטענה שאם $A$ אינסופית ללא איבר מינימלי אז $A$ אינה חסומה מלרע. \begin{proof}[הפרכה]
			בעבור הקבוצה $A = \{\frac{1}{n} \co n \in \N\}$ מתקיים תמיד $\frac{1}{n} > 0$ כלומר $0$ חסם מלרע של $A$. מנגד כן הזיווג $f \co A \to \N$ המוגדר לפי $f(a) = a\op$ מראה ש־$\sof A = \az$ כלומר היא אינסופית. סה''כ סתירה לטענה. 
		\end{proof}
		\item נפריך את כך שאם $A, B$ חסומות ו־$\sup A = \inf B$ אז $A \cap B$ מכיל בדיוק איבר אחד. \begin{proof}[הפרכה]
			נתבונן בשתי הסדרות הקבוצות: 
			\[ B = \ccb{\frac{1}{n} \mid n \in \N_+} \quad A = \ccb{-\frac{1}{n} \mid n \in n\N_+} \]
			בהרצאה הוכחנו ש־$\inf B = 0$. באותו האופן $\sup A = 0$. עם זאת, בהינתן $a \in A \cap B$ מתקיים קיום $n \in \N$ כך ש־$\frac{1}{n} = a$ וכן $m \in \N$ כך ש־$-\frac{1}{m} =a$ ואז $\frac{1}{n} = - \frac{1}{m}$. נכפיל אגפים ונקבל $m = -n$, ומשום ש־$m, n > 0$ סתירה (כי בהכרח אחד מהם שלילי). 
		\end{proof}
		\item נפריך את הטענה שאם $A, B$ קבוצות חסומות מלעיל וזרות, אז $\sup A \neq \sup B$. \begin{proof}
			נניח בשלילה את הטענה ונראה דוגמה נגדית. אכן, בעבור: 
			\[ A = \ccb{-\frac{1}{2n} \mid n \in \R_+} \quad B = \ccb{0} \]
			
			נוכיח ש־$\sup A = \sup B$ וכן $A \cap B = \varnothing$. 
			\begin{itemize}
				\item זרות: נניח בשלילה קיום $a \in A \cap B$, אז $a = 0$ וכן $a = (-2n)\op$ סה''כ קיים הופכי לאפס וסתירה. 
				\item נוכיח $\sup A = \sup B$. הסופרמום של סינגילטון הוא $0$ ואכן $\sup B = 0$. נראה ש־$\sup A = 0$. 
				
				ניכר ש־$0$ חוסם את $A$ ולכן חסם מלעיל שלה (שכן הופכי לחיובי הוא חיובי, והכפלתו ב־$(-1)$ תביא למספר שלילי). יהי $\eg > 0$. אכן, בעבור 
				
				\[ A \ni -\frac{1}{2n} < 0 - \eg \impliedby 1 < 2n\eg \impliedby \frac{1}{2\eg} < n \impliedby n = \frac{1}{4\eg} \]
				סה''כ $0$ סופרמום כדרוש. אז $\sup A = \sup B$ וסתירה וטענה שרצינו להפריך. 
			\end{itemize}
		\end{proof}
	\end{enumerate}
	
	\section{}
	תהאנה $\an, \bn$ סדרות כך ש־$\forall n \in \N \co a_n < b_n$ וכן $b_n \le b_{n + 1} \land a_{n + 1} \le a_n$ (כלומר $b_n$ מונוטונית עולה ו־$\an$ מונוטונית יורדת). נגדיר $I_n = [a_n, b_n]$ לכל $n \in \N$. נניח כי תמונת $\{b_n\}_{n = 1}^{\infty}$ חסומה מלעיל ותמונת $\{a_n\}_{n}^{\inft}$ חסומה מלרע. מאקסיומת השלמות קיים $\bg = \sup b_n$ וכן $\ag = \inf a_n$. לכל $n \in \N$ מוגדר הסימון $I_n = [a_n, b_n]$. נוכיח ש־$(\ag, \bg) = \bigcup_{n \in \N} I_n$. 
	
	\begin{proof}באינדוקציה ידוע $a_m < a_0 < b_0 < b_n$ כלומר $\forall m, n \in \N \co a_m < b_n$. נפנה להוכיח את הדרוש הכלה דו כיוונית. 
		\begin{itemize}
			\item[$\subseteq$]יהי $x \in (\ag, \bg)$ כלומר $\ag < x < \bg$. 	ממשפט וויירשטראס הראשון, $\an, \bn$ בעלות גבול. יתרה מכך, $\limasi = \ag, \limbsi = \bg$. יהי $x \in (\ag, \bg)$. 
			נתבונן בקטע $I = (\ag + 1, x)$. מתקיים $x < \ag < \ag + 1$ כלומר $\ag \in I$. מההגדרה השקולה לגבול שראינו, יש כמות סופית של $\an$־ים מחוץ ל־$I$, ומכאן ש־$\sof{\{a \in A \mid a \in I\}} \ge \az$. סה''כ קיים בהכרח $n \in \N$ כך ש־$a \in I$ כלומר $x < a_n < \ag +1$. ידוע $a_n < \ag$ כלומר $a \in (\ag, x)$. באופן זהה ניתן למצוא $b_m$ כך ש־$b_m \in (x, \bg)$. בעבור $k = \max\{m, n\}$ מתקיים: 
			\[ \ag < a_k \le a_n < x < b_m \le b_k < \bg \implies x \in (a_k, b_k) \subseteq [a_k, b_k] = I_k \]
			ומהגדרת איחוד מוכלל $x \in \bigcup_{t \in \N}I_t$ כדרוש. 
			
			\item[$\supseteq$]מצד שני, יהי $x \in \bigcup_{n \in \N} I_n$. נוכיח $x \in (\ag, \bg)$. אכן ידוע קיום $n\in \N$ כך ש־$x \in I_n = [a_n, b_n]$ כלומר $\ag < a_n \le x \le b_n < \bg$ ולכן $\ag < x < \bg $ דהיינו $x \in (\ag, \bg)$ וסיימנו. 
		\end{itemize}
	\end{proof}
	
	\section{}
	נמצא אינפימום, סופרמום, מינימום ומקסימום לקבוצות הבאות: 
	\begin{enumerate}[(A)]
		\item \hfil $\displaystyle A = \ccb{x + \frac{1}{x} \co x > 0}$
		
		נוכיח שיש לקבוצה מינימום הוא $2$. יהי $k \in A$. מכאן $k = x + \frac{1}{x}$ עבור $x > 0$. יהי $k \neq 2$, נוכיח $k > 2$. נניח בשלילה $k< 2$. אז $x + \frac{1}{x} < 2$, ובשקילות נקבל $(x - 1)^2 = x^2 - 2x + 1 < 0$. אך ריבוע מספר ממשי גדול מ־$0$ וזו סתירה. לכן בהכרח $k = 2$ מינימלי וקיים. ידוע שהמינימום אם קיים הוא אינפימום, כלומר $\inf A = \min A = 2$. 
		
		עתה נראה שהקבוצה לא חסימה מלעיל. זאת כי לכל $M \in \R$ בשלילה חסם מלעיל מתקיים שאם $M < 1$ אז סתירה כי $2 \in A$, אחרת $M \ge 1$ ואז: 
		\[ A \ni M + \frac{1}{M} > M \]
		בסתירה להיות $M$ חסם מלעיל. מהיותה לא חסומה מלעיל, אין לה סופרמום (כי סופרמום הוא חסם מלעיל), ואין לא מקסימום (כי אחרת המקסימום היה סופרמום שלא קיים). 
		\item \hfil $\displaystyle B = \{x^2 + x + 1 \co x \in \R\}$
		
		נוכיח שהמינימום הוא $\frac{3}{4}$. עבור $-\frac{1}{2} = x$ אכן מתקיים $x^2 + x + 1 = 0.75$ ומכאן ש־$0.75 \in B$. עתה נוכיח שהוא מינימלי. נניח בשלילה ש־: 
		\[ x^2 + x + 1 < 0.75 \implies 0 < \cl{x + \frac{1}{2}}^2 = x^2 + x + \frac{1}{4} < 0 \]
		וסתירה. מכאן ש־$\frac{3}{4} = \min B = \inf B$. באופן דומה ל־$A$ היא איננה חסומה: יהי $M$ חסם עליון. משום ש־$0.75$ מינימום, $M > 0.75 > 0$. משום שלכל $M \in \R_+$ מתקיים $M + \eg > M$ עבור $\eg > 0$, וכן $M^2 > 0$ וגם $M^2 + 1 > 0$, מתקיים: 
		\[ M < M + \eg = M^2 + M + 1 \in A \]
		וסתירה וסיימנו. מהיותה לא חסומה מלעיל, אין לה סופרמום (כי סופרמום הוא חסם מלעיל), ואין לא מקסימום (כי אחרת המקסימום היה סופרמום שלא קיים). 
		\item \hfil $\displaystyle C = \ccb{\frac{m}{n} \co m, n \in \N \land m < n}$
		
		נוכיח ש־$C$ חסרת מינימום ומקסימום, וכן $\inf C = 0, \sup C = 1$.  
		\begin{itemize}
			\item ראשית כל, נוכיח שהסופרמום הוא $1$. ידוע שלכל זוג $m < n$ אכן $\frac{m}{n} < 1$ ולכן הוא חסם מלעיל. יהי $\eg > 0$. נראה קיום $q \in C, 1 - \eg < q < 1$. למעשה, מצפיפות הרציונלים בממשיים שקיים רציונלי $q \in \Q$ בטווח הזה, ולכל $q < 1$ מתקיים ש־$q = \frac{m}{n}$ כלשהם ועבורם $\frac{m}{n} < 1 \implies m < n$ כלומר $q \in C$. באופן דומה $\inf C = 0$. 
			\item עתה נוכיח שאין לקבוצה מקסימום. יהי $M \in C$, ונניח ש־$M$ מקסימום. אז $M \in \R$ וכן $M < 1$ (אחרת לכל $m, n$ טבעיים כך ש־$M = \frac{m}{n}$ מתקיים $m = n$ וסתירה) ומצפיפות הרציונליים במשיים קיים $q \in \Q$ (וכבר הראינו שעבור $\Q_+ \ni q < 1$ מתקיים $q \in C$) כך ש־$M < q < 1$ וזו סתירה. באופן דומה אפשר להוכיח שאין מינימום. 
		\end{itemize}
		
		\item \hfil $D = \ccb{\frac{1}{n} + (-1)^{n} \co n \in \N}$ 
		
		נראה ש־$\max D = \sup D = 1.5$, $\inf D = -1$ ו־$\min D$ איננו מוגדר. 
		
		נוכיח ש־$1.5$ מקסימום. עבור $n = 2$ אכן $1.5 \in D$. יהי $x \in D$. אז קיים $n \in \N$ כך ש־$x = \frac{1}{n} + (-1)^{n}$. נוכיח $x \le 1.5$. נפרק למקרים. לכל $n \ge 3$
		נבחין ש־: 
		\[ (-1)^{n} \le 1 \land \frac{1}{n} \le \frac{1}{3} \implies (-1)^{n} + 1 \le 1 + \frac{1}{3} = 1\frac{1}{3} \]
		עבור $n = 2$ ברור ש־$x = 1.5$ ועבור $n = 1$ מתקיים $x = 0$. סה''כ בהכרח $x \le 1.5$ וסיימנו. מהיות $1.5$ מקסימום הוא גם סופרמום. לכן $\sup D = \max D = 1.5$. 
		
		עתה נראה שאין מינימום. יהי $M \in D$ בשלילה מינימום. אז קיים $n$ טבעי כך ש־$x = \frac{1}{n} + (-1)^{n}$. עם זאת, עבור $m = 2n$: 
		\[ D \ni \frac{1}{m} + (-1)^{m} = \underbrace{\frac{1}{2n}}_{> n\op} + \underbrace{(-1)^{2n}}_{\ge (-1)^{n}} > \frac{1}{n} + (-1)^{n} = M \]
		וסתירה. עכשיו נראה ש־$-1$ אינפימום. בבירור $-1$ חסם מלרע שכן לכל $x \in D$ קיים $n \in \N$ כך ש־$n\op + (-1)^{n} = x$ ואז: 
		\[ (-1)^{n} \ge -1 \land n\op > 0 \implies (-1)^{n} + n\op > -1 \]
		יהי $\eg > 0$. מארכימדיאניות הטבעיים קיים $n \in \N$ כך ש־$1 \le n \eg$. אז $n\op \le \eg$. מכאן: 
		\[ \frac{1}{2n} < \frac{1}{n} \le \eg \implies \frac{1}{2n} + \underbrace{(-1)^{2n}}_{-1} < -1 + \eg \]
		כדרוש. 
	\end{enumerate}
	
	\section{}
	נגדיר את הקבוצה: 
	\[ A = \ccb{\ceil{\sqrt n} - \sqrt n \co n \in \N} \]
	כאשר $\ceil x := \min\{n \in \Z \mid x \le n\}$. 
	נוכיח ש־$\sup A = 1, \inf A = 0$. \begin{proof}לכל $x \in \R$ מתקיים: 
		\[ 0 \le x - x \le \ceil x - x \le x + 1 - x = 1 \]
		וזאת כי בין $x$ לבין $x + 1$ בהכרח קיים מספר טבעי (הוכח בכיתה). נסמן ב־$\tl x$ את $\ceil x - x$. 
		\begin{itemize}
			\item \textbf{אינפימום: }מהא''ש לעיל בהכרח $0$ חסם תחתון. יהי $\eg > 0$. נראה ש־$\exists x \in A \co x \le 0 + \eg$. אז עבור המספר 
			\[  \]
		\end{itemize}
	\end{proof}
	
	\section{}
	נוכיח שלכל קבוצה סופית קיים מקסימום ומינימום. 
	\begin{proof}
		תהי $A$ קבוצה סופית. אזי $\sof A= n$ עבור $n$ טבעי כלשהו. נוכיח באינדוקציה על $n$ את הטענה. צעד עבור $n = 1$ אז $A$ סינגילטון ואז $A = \{a\}$ כלשהו, ו־$\min A = \max A = a$ וסיימנו. אחרת, $\sof A > 1$ כלומר קיים $a \in A$ וכן $\sof {A \setminus \{a\}}  = n - 1$. מה.א. ל־$A \setminus \{a\}$ קיים מינימום ומקסימום, נסמנם $M_+, M_-$ בהתאמה. אז עבור: 
		\[ \max A =: \begin{cases}
			a & a > M_+ \\
			M_+ & \other
		\end{cases} \quad \min A =: \begin{cases}
		a & a < M_- \\
		M_- & \other
		\end{cases} \]
		מתקיים ש־$\forall b \in A \setminus \{a\} \co \min A \le M_- \le b$ ומהגדרת $\min$ גם $\forall b \in A \co b \ge \min A$ כדרוש (כי $A = (A \setminus \{a\}) \uplus \{a\}$) ובאופן דומה לגבי $\max$ וסיימנו. 
	\end{proof}
	
	\section{}
	קבוצה $A \subseteq \R$ תקרא \textit{דיסקרטית} אם $\forall x \in A .\, \exists \eg > 0 \co (x - \eg, x + \eg) \cap A = \{x\}$. נגדיר את הקבוע: 
	\[ d(A) = \inf \underbrace{\ccb{\sof{x - y} \co x, y \in A \land x \neq y}}_{D(A)} \]
	בעבור קבוצה $A$ כלשהי. 
	
	\begin{enumerate}[(A)]
		\item נוכיח שאם $d(A) > 0$ אז $A$ דיסקרטית. \begin{proof}
			תהי קבוצה $A$ כך ש־$d(A) > 0$. נוכיח שהיא דיסקרטית. יהי $x \in A$. אז עבור $d(A) > 0$ נוכיח ש־$(x - \eg, x + \eg) \cap A = \{x\}$. נניח בשלילה אחרת, אזי קיים $x \neq y \in (x - \eg, x + \eg)$. מהגדרת הימצאות בתחום: 
			\[ -\eg < x - y < \eg \implies \sof{x - y} < \eg = d(A) \]
			מהגדרה $\sof{x - y} \in D(A)$ ומשום ש־$d(A) = \inf D(A)$ אז $\sof{x - y} \ge d(A)$ וזו סתירה לזה שהוכחנו ש־$\sof{x - y} < d(A)$. סה''כ הראינו את הדרוש ו־$A$ דיסקרטית. 
		\end{proof}
		\item נוכיח את הטענה הבאה: אם $A$ חסומה מלעיל ו־$d(A) > 0$ אז יש בה מקסימום. \begin{proof}
			תהי $A$ קבוצה חסומה מלעיל ו־$d(A) > 0$ בעבורה. נוכיח שיש בה מקסימום. מהיותה חסומה מלעיל, ידוע שקיים $\sup A$. נתבונן בסביבה נקובה סביב $\sup A$, מהגדרת הדיסקרטיות בהכרח $(\sup A - \eg, \sup A + \eg) \cap A  =: C = \{\sup A\}$ עבור $\eg > 0$ כלשהו. עם זאת, מהגדרת הסופרמום, קיים $a \in A$ כך ש־$\sup A - \eg < a \le \sup A$. מכאן שבהכרח $a \in C$ ו־$C = \{\sup A\}$ כלומר $a = \sup A$ וסה''כ $\sup A \in A$ כלומר יש מקסימום לקבוצה וסיימנו. 
		\end{proof}
		\textit{הערה: }משום מה ביקשתם להוכיח רק אחת משלושת הטענות בסעיף ב'. בחרתי את $(i)$. 
		\item נוכיח כי $\Z$ דיסקרטית בעבור $d(A) = 1$ \begin{proof}
			לכל $x \neq y$ כאשר $x, y \in \Z$ בהכרח קיים $n\in \N$ כך ש־$x + n = y$ וגם $n \neq 0$. אז: 
			\[ \sof{x - y} = \sof{-n} = n > 0 \]
			מספר טבעי גדול ממש מ־$0$ הוא גדול מ־$1$ כלומר $\sof{x - y} \ge 1$. מכאן ש־$1$ חוסם מלמטה את $D(\Z)$. נבחין ש־$1 \in D(\Z)$ בגלל שעבור $0, 1 \in \Z$ מתקיים $\sof{1 - 0} = 1$. סה''כ $1$ הוא המינימום של $D(\Z)$ ובפרט האינפימום, וסיימנו. 
		\end{proof}
	\end{enumerate}
	
	\section{}
	נוכיח שלכל $x, y \in \R$, אם $x > 1$ אז קיים $n \in \N$ כך ש־$x^{n} > y$. מכאן נוכיח ש־$\limsi x^{n} = +\inft$. נראה גם שלכל $x < -1$ הסדרה חסרת גבולות. 
	\subsection{קיום שורש $n$־י}
	\begin{proof}
		יהי $x \in \R_{\ge 0}$ ממשי אי־שלילי. נתבונן בקבוצה $A = \{a \in \R \co a^n < x\}$. נוכיח שהיא חסומה מלעיל: לכל $a \in A$, נפרק למקרים: 
		\begin{itemize}
			\item אם $a > 1$ אז $a < a^{n} = x$ וסה''כ $\max\{x, 1\}$ חסם מלעיל. 
			\item אם $a < 1$ אז $\max\{x, 1\}$ עדיין חסם מלעיל. 
		\end{itemize}
		אז הדבר הזה באמת חסום מלמעלה. לכן קיים סופרמום, הוא $\sup A$. נראה ש־$(\sup A)^n = x$. נפריד למקרים. 
		\begin{itemize}
			\item אם $\sup A < x$ אז $\sup A \in A$ מהגדרה, ואז $A$ בעלת מקסימום. אבל אוקי להוכיח שאין לזה מקסימום עומד להיות קשה
		\end{itemize}
	\end{proof}
	
	\ndoc
\end{document}