\documentclass[]{../../../../tex/classes/homework}
\usepackage{../../../../tex/packages/hebrewSupport}
\usepackage{../../../../tex/packages/mathShortcuts}
\usepackage{../../../../tex/packages/theoremsSupport}

\author{שחר פרץ}
\title{\textit{חדו''א 1א} $\sim$ תרגיל בית 2}
\date{18 בנובמבר 2025}
\begin{document}
	\maketitle
	\section{}
	תהי $A \subseteq \R$ קבוצה לא ריקה ויהי $s \in \R$. נוכיח $s$ החסם העליון של $A$ אמ''מ $s$ חסם מלעיל מינימלי. 
	\begin{proof}
		תהי $A \subseteq \R$ ו־$\ag \in \R$. נוכיח שקילות באמצעות הוכחת גרירה דו־כיוונית. 
		\begin{itemize}
			\item[$\implies$]נניח $\ag$ חסם עליון של $A$. נוכיח שהוא חסם מלעיל מנימלי. מהיותו חסם עליון, ידוע שהוא חסם מלעיל. נוכיח שהוא מינימלי. יהי $\bg \in \R$ חסם מלעיל של $A$. נניח בשלילה $\bg < \ag$, אז בעבור $\eg = \ag - \bg$ קיים $a \in A$ כך ש־$a > \ag - \eg = \ag - (\ag  - \bg) = \bg$, ומכאן ש־$\bg$ אינו חסם מלעיל של $A$ וסתירה. 
			\item[$\impliedby$]נניח $\ag$ חסם מלעיל מינימלי, נוכיח שהוא חסם עליון. יהי $\eg > 0$. אז נניח בשלילה שלא קיים $a \in A$ כך ש־$a > \ag - \eg$, ואז $\forall a \in A \co a < \ag - \eg$ כלומר $\ag - \eg$ חסם מלעיל של $A$ מהגדרה, אך $\ag - \eg < \ag$ וזו סתירה למינימליות של $\ag$ מבין החסמים מלעיל. סה''כ בהכרח קיים $a$ המתאים לתנאי וסיימנו. 
		\end{itemize}
	\end{proof}
	
	\section{}
	תהיינה $A, B \subseteq \R$ קבוצות לא ריקות. נוכיח או נפריך את את הטענות הבאות. 
	\begin{enumerate}[(A)]
		\item נוכיח שאם ל־$A$ אין איבר מקסימלי אז $A$ אינסופית. \begin{proof}
			תהי $A$ קבוצה ללא איבר מקסימלי. נניח בשלילה שהיא סופית. אזי $\max A$ מוגדר (ממשפט הרקורסיה: ידוע קיום זיווג $f \co [n] \to A$ ואז הפונקציה $m_1 = f(1)$, ותנאי נסיגה $m_{n + 1}= \max\{m_k, f(k + 1)\}$ מגדירה את $\max A := m_n$)וחוסם את הסדרה מלמעלה, וסיימנו. 
		\end{proof}
		\item נפריך את הטענה שאם $A$ אינסופית ללא איבר מינימלי אז $A$ אינה חסומה מלרע. \begin{proof}[הפרכה]
			בעבור הקבוצה $A = \{\frac{1}{n} \co n \in \N\}$ מתקיים תמיד $\frac{1}{n} > 0$ כלומר $0$ חסם מלרע של $A$. מנגד כן הזיווג $f \co A \to \N$ המוגדר לפי $f(a) = a\op$ מראה ש־$\sof A = \az$ כלומר היא אינסופית. סה''כ סתירה לטענה. 
		\end{proof}
		\item נפריך את כך שאם $A, B$ חסומות ו־$\sup A = \inf B$ אז $A \cap B$ מכיל בדיוק איבר אחד. \begin{proof}[הפרכה]
			נתבונן בשתי הסדרות הקבוצות: 
			\[ B = \ccb{\frac{1}{n} \mid n \in \N_+} \quad A = \ccb{-\frac{1}{n} \mid n \in n\N_+} \]
			בהרצאה הוכחנו ש־$\inf B = 0$. באותו האופן $\sup A = 0$. עם זאת, בהינתן $a \in A \cap B$ מתקיים קיום $n \in \N$ כך ש־$\frac{1}{n} = a$ וכן $m \in \N$ כך ש־$-\frac{1}{m} =a$ ואז $\frac{1}{n} = - \frac{1}{m}$. נכפיל אגפים ונקבל $m = -n$, ומשום ש־$m, n > 0$ סתירה (כי בהכרח אחד מהם שלילי). 
		\end{proof}
		\item נפריך את הטענה שאם $A, B$ קבוצות חסומות מלעיל וזרות, אז $\sup A \neq \sup B$. \begin{proof}
			נניח בשלילה את הטענה ונראה דוגמה נגדית. אכן, בעבור: 
			\[ A = \ccb{-\frac{1}{2n} \mid n \in \R_+} \quad B = \ccb{0} \]
			
			נוכיח ש־$\sup A = \sup B$ וכן $A \cap B = \varnothing$. 
			\begin{itemize}
				\item זרות: נניח בשלילה קיום $a \in A \cap B$, אז $a = 0$ וכן $a = (-2n)\op$ סה''כ קיים הופכי לאפס וסתירה. 
				\item נוכיח $\sup A = \sup B$. הסופרמום של סינגילטון הוא $0$ ואכן $\sup B = 0$. נראה ש־$\sup A = 0$. 
				
				ניכר ש־$0$ חוסם את $A$ ולכן חסם מלעיל שלה (שכן הופכי לחיובי הוא חיובי, והכפלתו ב־$(-1)$ תביא למספר שלילי). יהי $\eg > 0$. אכן, בעבור 
				
				\[ A \ni -\frac{1}{2n} < 0 - \eg \impliedby 1 < 2n\eg \impliedby \frac{1}{2\eg} < n \impliedby n = \frac{1}{4\eg} \]
				סה''כ $0$ סופרמום כדרוש. אז $\sup A = \sup B$ וסתירה וטענה שרצינו להפריך. 
			\end{itemize}
		\end{proof}
	\end{enumerate}
	
	\section{}
	תהאנה $\an, \bn$ סדרות כך ש־$\forall n \in \N \co a_n < b_n$ וכן $b_n \le b_{n + 1} \land a_{n + 1} \le a_n$ (כלומר $b_n$ מונוטונית עולה ו־$\an$ מונוטונית יורדת). נגדיר $I_n = [a_n, b_n]$ לכל $n \in \N$. נניח כי תמונת $\{b_n\}_{n = 1}^{\infty}$ חסומה מלעיל ותמונת $\{a_n\}_{n}^{\inft}$ חסומה מלרע. מאקסיומת השלמות קיים $\bg = \sup b_n$ וכן $\ag = \inf a_n$. לכל $n \in \N$ מוגדר הסימון $I_n = [a_n, b_n]$. נוכיח ש־$(\ag, \bg) = \bigcup_{n \in \N} I_n$. 
	
	\begin{proof}באינדוקציה ידוע $a_m < a_0 < b_0 < b_n$ כלומר $\forall m, n \in \N \co a_m < b_n$. נפנה להוכיח את הדרוש הכלה דו כיוונית. 
		\begin{itemize}
			\item[$\subseteq$]יהי $x \in (\ag, \bg)$ כלומר $\ag < x < \bg$. 	ממשפט וויירשטראס הראשון, $\an, \bn$ בעלות גבול. יתרה מכך, $\limasi = \ag, \limbsi = \bg$. יהי $x \in (\ag, \bg)$. 
			נתבונן בקטע $I = (\ag + 1, x)$. מתקיים $x < \ag < \ag + 1$ כלומר $\ag \in I$. מההגדרה השקולה לגבול שראינו, יש כמות סופית של $\an$־ים מחוץ ל־$I$, ומכאן ש־$\sof{\{a \in A \mid a \in I\}} \ge \az$. סה''כ קיים בהכרח $n \in \N$ כך ש־$a \in I$ כלומר $x < a_n < \ag +1$. ידוע $a_n < \ag$ כלומר $a \in (\ag, x)$. באופן זהה ניתן למצוא $b_m$ כך ש־$b_m \in (x, \bg)$. בעבור $k = \max\{m, n\}$ מתקיים: 
			\[ \ag < a_k \le a_n < x < b_m \le b_k < \bg \implies x \in (a_k, b_k) \subseteq [a_k, b_k] = I_k \]
			ומהגדרת איחוד מוכלל $x \in \bigcup_{t \in \N}I_t$ כדרוש. 
			
			\item[$\supseteq$]מצד שני, יהי $x \in \bigcup_{n \in \N} I_n$. נוכיח $x \in (\ag, \bg)$. אכן ידוע קיום $n\in \N$ כך ש־$x \in I_n = [a_n, b_n]$ כלומר $\ag < a_n \le x \le b_n < \bg$ ולכן $\ag < x < \bg $ דהיינו $x \in (\ag, \bg)$ וסיימנו. 
		\end{itemize}
	\end{proof}
	
	\section{}
	נמצא אינפימום, סופרמום, מינימום ומקסימום לקבוצות הבאות: 
	\begin{enumerate}[(A)]
		\item \hfil $\displaystyle A = \ccb{x + \frac{1}{x} \co x > 0}$
		
		נוכיח שיש לקבוצה מינימום הוא $2$. יהי $k \in A$. מכאן $k = x + \frac{1}{x}$ עבור $x > 0$. יהי $k \neq 2$, נוכיח $k > 2$. נניח בשלילה $k< 2$. אז $x + \frac{1}{x} < 2$, ובשקילות נקבל $(x - 1)^2 = x^2 - 2x + 1 < 0$. אך ריבוע מספר ממשי גדול מ־$0$ וזו סתירה. לכן בהכרח $k = 2$ מינימלי וקיים. ידוע שהמינימום אם קיים הוא אינפימום, כלומר $\inf A = \min A = 2$. 
		
		עתה נראה שהקבוצה לא חסימה מלעיל. זאת כי לכל $M \in \R$ בשלילה חסם מלעיל מתקיים שאם $M < 1$ אז סתירה כי $2 \in A$, אחרת $M \ge 1$ ואז: 
		\[ A \ni M + \frac{1}{M} > M \]
		בסתירה להיות $M$ חסם מלעיל. מהיותה לא חסומה מלעיל, אין לה סופרמום (כי סופרמום הוא חסם מלעיל), ואין לא מקסימום (כי אחרת המקסימום היה סופרמום שלא קיים). 
		\item \hfil $\displaystyle B = \{x^2 + x + 1 \co x \in \R\}$
		
		נוכיח שהמינימום הוא $\frac{3}{4}$. עבור $-\frac{1}{2} = x$ אכן מתקיים $x^2 + x + 1 = 0.75$ ומכאן ש־$0.75 \in B$. עתה נוכיח שהוא מינימלי. נניח בשלילה ש־: 
		\[ x^2 + x + 1 < 0.75 \implies 0 < \cl{x + \frac{1}{2}}^2 = x^2 + x + \frac{1}{4} < 0 \]
		וסתירה. מכאן ש־$\frac{3}{4} = \min B = \inf B$. באופן דומה ל־$A$ היא איננה חסומה: יהי $M$ חסם עליון. משום ש־$0.75$ מינימום, $M > 0.75 > 0$. משום שלכל $M \in \R_+$ מתקיים $M + \eg > M$ עבור $\eg > 0$, וכן $M^2 > 0$ וגם $M^2 + 1 > 0$, מתקיים: 
		\[ M < M + \eg = M^2 + M + 1 \in A \]
		וסתירה וסיימנו. מהיותה לא חסומה מלעיל, אין לה סופרמום (כי סופרמום הוא חסם מלעיל), ואין לא מקסימום (כי אחרת המקסימום היה סופרמום שלא קיים). 
		\item \hfil $\displaystyle C = \ccb{\frac{m}{n} \co m, n \in \N \land m < n}$
		
		נוכיח ש־$C$ חסרת מינימום ומקסימום, וכן $\inf C = 0, \sup C = 1$.  
		\begin{itemize}
			\item ראשית כל, נוכיח שהסופרמום הוא $1$. ידוע שלכל זוג $m < n$ אכן $\frac{m}{n} < 1$ ולכן הוא חסם מלעיל. יהי $\eg > 0$. נראה קיום $q \in C, 1 - \eg < q < 1$. למעשה, מצפיפות הרציונלים בממשיים שקיים רציונלי $q \in \Q$ בטווח הזה, ולכל $q < 1$ מתקיים ש־$q = \frac{m}{n}$ כלשהם ועבורם $\frac{m}{n} < 1 \implies m < n$ כלומר $q \in C$. באופן דומה $\inf C = 0$. 
			\item עתה נוכיח שאין לקבוצה מקסימום. יהי $M \in C$, ונניח ש־$M$ מקסימום. אז $M \in \R$ וכן $M < 1$ (אחרת לכל $m, n$ טבעיים כך ש־$M = \frac{m}{n}$ מתקיים $m = n$ וסתירה) ומצפיפות הרציונליים במשיים קיים $q \in \Q$ (וכבר הראינו שעבור $\Q_+ \ni q < 1$ מתקיים $q \in C$) כך ש־$M < q < 1$ וזו סתירה. באופן דומה אפשר להוכיח שאין מינימום. 
		\end{itemize}
		
		\item \hfil $D = \ccb{\frac{1}{n} + (-1)^{n} \co n \in \N}$ 
		
		נראה ש־$\max D = \sup D = 1.5$, $\inf D = -1$ ו־$\min D$ איננו מוגדר. 
		
		נוכיח ש־$1.5$ מקסימום. עבור $n = 2$ אכן $1.5 \in D$. יהי $x \in D$. אז קיים $n \in \N$ כך ש־$x = \frac{1}{n} + (-1)^{n}$. נוכיח $x \le 1.5$. נפרק למקרים. לכל $n \ge 3$
		נבחין ש־: 
		\[ (-1)^{n} \le 1 \land \frac{1}{n} \le \frac{1}{3} \implies (-1)^{n} + 1 \le 1 + \frac{1}{3} = 1\frac{1}{3} \]
		עבור $n = 2$ ברור ש־$x = 1.5$ ועבור $n = 1$ מתקיים $x = 0$. סה''כ בהכרח $x \le 1.5$ וסיימנו. מהיות $1.5$ מקסימום הוא גם סופרמום. לכן $\sup D = \max D = 1.5$. 
		
		עתה נראה שאין מינימום. יהי $M \in D$ בשלילה מינימום. אז קיים $n$ טבעי כך ש־$x = \frac{1}{n} + (-1)^{n}$. עם זאת, עבור $m = 2n$: 
		\[ D \ni \frac{1}{m} + (-1)^{m} = \underbrace{\frac{1}{2n}}_{> n\op} + \underbrace{(-1)^{2n}}_{\ge (-1)^{n}} > \frac{1}{n} + (-1)^{n} = M \]
		וסתירה. עכשיו נראה ש־$-1$ אינפימום. בבירור $-1$ חסם מלרע שכן לכל $x \in D$ קיים $n \in \N$ כך ש־$n\op + (-1)^{n} = x$ ואז: 
		\[ (-1)^{n} \ge -1 \land n\op > 0 \implies (-1)^{n} + n\op > -1 \]
		יהי $\eg > 0$. מארכימדיאניות הטבעיים קיים $n \in \N$ כך ש־$1 \le n \eg$. אז $n\op \le \eg$. מכאן: 
		\[ \frac{1}{2n} < \frac{1}{n} \le \eg \implies \frac{1}{2n} + \underbrace{(-1)^{2n}}_{-1} < -1 + \eg \]
		כדרוש. 
	\end{enumerate}
	
	\section{}
	נגדיר את הקבוצה: 
	\[ A = \ccb{\ceil{\sqrt n} - \sqrt n \co n \in \N} \]
	כאשר $\ceil x := \min\{n \in \Z \mid x \le n\}$. 
	נוכיח ש־$\sup A = 1, \inf A = 0$. 
	\begin{proof}
		נראה ש־$\sup A = 1, \min A = \inf A = 0$, אך $\max A$ אינו מוגדר. 
		לכל $x \in \R$ מתקיים: 
		\[ 0 \le x - x \le \ceil x - x \le x + 1 - x = 1 \]
		וזאת כי בין $x$ לבין $x + 1$ בהכרח קיים מספר טבעי (הוכח בכיתה). נסמן ב־$\tl x$ את $\ceil x - x$. 
		\begin{itemize}
			\item \textbf{אינפימום ומינימום: }מהא''ש לעיל בהכרח $0$ חסם תחתון. יתרה מכך, $0 \in A$ שכן עבור $n = 0$ מתקבל $\ceil{\sqrt 0} - \sqrt 0 = 0$. מהגדרה $0 = \min A$ כלומר $\sup A = \min A = 0$. 
			\item \textbf{סופרמום: }נוכיח ש־$1$ סופרמום. הוא חסם עליון מהא''ש לעיל. נראה שהוא הדוק. יהי $\eg > 0$. נוכיח קיום $a \in A$ כך ש־$a > 1 - \eg$. 
			
			\begin{itemize}
				\item אם $\eg \le 1$, אז נפעל עלי מהשמיים שאם נבחר $k > \max\ccb{2 \cdot \ceil{\frac{2 + 2 \eg - \eg^2}{2\eg - 2}}, 1}$ ואז $n = k^2 - 1$ (שבהכרח טבעי, כי $k$ טבעי) אז $a = \ceil{\sqrt{n}} - \sqrt n$ יעבוד. 
				
				לכל $n$ מהצורה $n = k^{2} - 1$ עבור $k \in \N$ מתקיים $\ceil{\sqrt n} = k = \sqrt{n + 1}$. נדרוש 
				\[ 0 \le \ceil{\sqrt n} - \sqrt n = \ceil{\sqrt{k^2 - 1}} - \sqrt{k^2 - 1} = k - \sqrt{k^2 - 1} \overset{!}{>} 1 - \eg \iff k + \eg - 1 > \sqrt{k^2 + 1} \]
				בגלל ההנחה $\eg < 1$ (במקרים אחרים נטפל בנפרד), נשמור על שקילות לביטוי לעיל אם נעלה בריבוע את שני האגפים, כי הם גדולים מ־$0$. מכאן שהא''ש לעיל שקול לרצף האי־שוויונות הבאים: 
				\[ \begin{WithArrows}
					k^2 + \eg^2 - 1 + 2 k \eg - 2k - 2 \eg &> k^2 + 1 \Arrow[jump=2, ll]{נוסיף \sen $-k^2 - \eg^2+ 2 \eg + 1$ \she לשני האגפים} \\
					\eg^2 - 2 \eg - 1 + k(2 \eg - 2) &> 1 \\
					k(2 \eg - 2) &> 2 + 2 \eg - \eg^2 \\
					k &> \frac{2 + 2 \eg - \eg^2}{2 \eg - 2}
				\end{WithArrows} \]
				החלוקה בסוף חוקית כי $2\eg - 2 > 0$ מההנחה $\eg < 1$. למעשה מבחירת $k$, כאן הראינו שקילות לא''ש לעיל, וסיימנו. 
				\item אם $\eg > 1$, אז $1 - \eg < 0$ ואז כל $a \in A$ יעבוד כי בהכרח $a > 0$, ו־$A$ לא ריקה (לדוגמה בעבור המינימום שהראינו את קיומו). 
			\end{itemize}
			
			
			
			סה''כ מצאנו $a$ מתאים. כלומר $1$ אכן סופרמום. 
			\item \textbf{מקסימום: }נניח בשלילה קיום מקסימום. בגלל ש־$1 = \sup A$, אז $\max A = \sup A = 1$ ואז $1 \in A$. מכאן ש־: 
			\[ \exists n \in \N\co \ceil{\sqrt n} - \sqrt n = 1 \implies \ceil {\sqrt n} = \sqrt n + 1 \]
			אבל לכל $x \in \R$ קיים $k \in \N$ כך ש־$x < k < x + 1$. בפרט עבור $x = \sqrt n$, ואז $\sqrt n < k < \sqrt n + 1 = \ceil{\sqrt n}$ וזו סתירה להגדרת $\ceil{\cdot}$. 
		\end{itemize}\envendproof
	\end{proof}
	
	\section{}
	נוכיח שלכל קבוצה סופית קיים מקסימום ומינימום. 
	\begin{proof}
		תהי $A$ קבוצה סופית. אזי $\sof A= n$ עבור $n$ טבעי כלשהו. נוכיח באינדוקציה על $n$ את הטענה. צעד עבור $n = 1$ אז $A$ סינגילטון ואז $A = \{a\}$ כלשהו, ו־$\min A = \max A = a$ וסיימנו. אחרת, $\sof A > 1$ כלומר קיים $a \in A$ וכן $\sof {A \setminus \{a\}}  = n - 1$. מה.א. ל־$A \setminus \{a\}$ קיים מינימום ומקסימום, נסמנם $M_+, M_-$ בהתאמה. אז עבור: 
		\[ \max A =: \begin{cases}
			a & a > M_+ \\
			M_+ & \other
		\end{cases} \quad \min A =: \begin{cases}
		a & a < M_- \\
		M_- & \other
		\end{cases} \]
		מתקיים ש־$\forall b \in A \setminus \{a\} \co \min A \le M_- \le b$ ומהגדרת $\min$ גם $\forall b \in A \co b \ge \min A$ כדרוש (כי $A = (A \setminus \{a\}) \uplus \{a\}$) ובאופן דומה לגבי $\max$ וסיימנו. 
	\end{proof}
	
	\section{}
	קבוצה $A \subseteq \R$ תקרא \textit{דיסקרטית} אם $\forall x \in A .\, \exists \eg > 0 \co (x - \eg, x + \eg) \cap A = \{x\}$. נגדיר את הקבוע: 
	\[ d(A) = \inf \underbrace{\ccb{\sof{x - y} \co x, y \in A \land x \neq y}}_{D(A)} \]
	בעבור קבוצה $A$ כלשהי. 
	
	\begin{enumerate}[(A)]
		\item נוכיח שאם $d(A) > 0$ אז $A$ דיסקרטית. \begin{proof}
			תהי קבוצה $A$ כך ש־$d(A) > 0$. נוכיח שהיא דיסקרטית. יהי $x \in A$. אז עבור $d(A) > 0$ נוכיח ש־$(x - \eg, x + \eg) \cap A = \{x\}$. נניח בשלילה אחרת, אזי קיים $x \neq y \in (x - \eg, x + \eg)$. מהגדרת הימצאות בתחום: 
			\[ -\eg < x - y < \eg \implies \sof{x - y} < \eg = d(A) \]
			מהגדרה $\sof{x - y} \in D(A)$ ומשום ש־$d(A) = \inf D(A)$ אז $\sof{x - y} \ge d(A)$ וזו סתירה לזה שהוכחנו ש־$\sof{x - y} < d(A)$. סה''כ הראינו את הדרוש ו־$A$ דיסקרטית. 
		\end{proof}
		\item נוכיח את הטענה הבאה: אם $A$ חסומה מלעיל ו־$d(A) > 0$ אז יש בה מקסימום. \begin{proof}
			תהי $A$ קבוצה חסומה מלעיל ו־$d(A) > 0$ בעבורה. נוכיח שיש בה מקסימום. מהיותה חסומה מלעיל, ידוע שקיים $\sup A$. נתבונן בסביבה נקובה סביב $\sup A$, מהגדרת הדיסקרטיות בהכרח $(\sup A - \eg, \sup A + \eg) \cap A  =: C = \{\sup A\}$ עבור $\eg > 0$ כלשהו. עם זאת, מהגדרת הסופרמום, קיים $a \in A$ כך ש־$\sup A - \eg < a \le \sup A$. מכאן שבהכרח $a \in C$ ו־$C = \{\sup A\}$ כלומר $a = \sup A$ וסה''כ $\sup A \in A$ כלומר יש מקסימום לקבוצה וסיימנו. 
		\end{proof}
		\textit{הערה: }משום מה ביקשתם להוכיח רק אחת משלושת הטענות בסעיף ב'. בחרתי את $(i)$. 
		\item נוכיח כי $\Z$ דיסקרטית בעבור $d(A) = 1$ \begin{proof}
			לכל $x \neq y$ כאשר $x, y \in \Z$ בהכרח קיים $n\in \N$ כך ש־$x + n = y$ וגם $n \neq 0$. אז: 
			\[ \sof{x - y} = \sof{-n} = n > 0 \]
			מספר טבעי גדול ממש מ־$0$ הוא גדול מ־$1$ כלומר $\sof{x - y} \ge 1$. מכאן ש־$1$ חוסם מלמטה את $D(\Z)$. נבחין ש־$1 \in D(\Z)$ בגלל שעבור $0, 1 \in \Z$ מתקיים $\sof{1 - 0} = 1$. סה''כ $1$ הוא המינימום של $D(\Z)$ ובפרט האינפימום, וסיימנו. 
		\end{proof}
	\end{enumerate}
	
	\section{}
	נוכיח שלכל $x, y \in \R$, אם $x > 1$ אז קיים $n \in \N$ כך ש־$x^{n} = y$. מכאן נוכיח ש־$\limsi x^{n} = +\inft$. נראה גם שלכל $x < -1$ הסדרה חסרת גבולות. 
	\subsection{קיום שורש $n$־י}
	\begin{proof}
		נוכיח באינדוקציה מלאה על $n$. עבור $n = 1$ מתקיים בפשוטות ש־$x^{1} = x$ לכל $x \in \R$, וזהו הבסיס. עתה נפנה להוכיח את הצעד: יהי $x \in \R_{\ge 0}$ ממשי אי־שלילי, ונראה קיום שורש $n$־י (כאשר מהנחת האינדוקציה קיים שורש של $n- k$ לכל $k \in [n]$), כלומר יהי $n \in \N_{\ge 2}$ ונראה ש־$\sqrt[n]{x}$ אכן קיים. נתבונן בקבוצה $A = \{a \in \R \co a^n < x\}$. נוכיח שהיא חסומה מלעיל: לכל $a \in A$, נפרק למקרים: 
		\begin{itemize}
			\item אם $a > 1$ אז $a < a^{n} = x$ וסה''כ $\max\{x, 1\}$ חסם מלעיל. 
			\item אם $a < 1$ אז $\max\{x, 1\}$ עדיין חסם מלעיל. 
		\end{itemize}
		אז הדבר הזה באמת חסום מלמעלה. לכן קיים סופרמום, הוא $\sup A$. 
		
		נראה ש־$(\sup A)^n = x$. נפריד למקרים. 
		\begin{itemize}
			\item אם בשלילה $(\sup A)^{n} < x$. נסמן $m = \sup A$. נבחין ש־$m \in A$ מהגדרה. בעבור: 
			\[ \dg = \frac{1}{2} \min \Bigg\{1, \underbrace{\frac{x - m^n}{\sum_{i = 1}^{n - 1}\binom{n}{i}m^i}}_{S}\Bigg\} \]
			מתקיים ש־: 
			\[ (m + \dg)^{n} = \sumni \binom{n}{i}m^{n - i}\dg^{i} \overset{\dg < 1}{<} m^{n} + \dg \cdot \sum_{i = 1}^{n}\binom{n}{i}m^{i} \overset{\dg < S}{<} \frac{x - m^n}{\cancel{\sum_{i = 1}^{n - 1}\binom{n}{i}m^i}} \cdot \cancel{\sum_{i = 1}^{n - 1}\binom{n}{i}m^i} + m^{n} = x \,\cancel{-\, m^{n} + m^{n}} = x \]
			סה''כ $(x + \dg)^{n} < x$. בגלל ש־$x > m^{n}$ אז $x - m^{n} > 0$ ומכאן ש־$S$ חיובי, כלומר $\dg > 0$. סה''כ $m + \dg > m$. בגלל ש־$(x + \dg)^{n} < x$ אז $m + \dg \in A$, וסה''כ בהכרח $m$ אינו מקסימום, למרות היותו סופרמום בקבוצה, וסתירה. 
			\item אם בשלילה $(\sup A)^{n} > x$. נסמן $m = \sup A$. לכן $m^{n} - x > 0$ ולכן: 
			\[ \dg = \frac{1}{2}\min\cl{\{m\} \cup \ccb{\sqrt[n - k]{\frac{m^{n} - x}{\binom{n}{k}m^{k}n}}\co k \in [n]}} \]
			מקיים $\dg > 0$. נבחין שהשורש $\sqrt[n - k]{\dots}$ קיים מהנחת האינדוקציה. מתקיים ש־: 
			\[ (m - \dg)^{n} = \sumnk m^{k}(-\dg)^{n - k} > m^{n} - \sum_{\mathclap{k \in \Nodd \setminus \{0\}}}^{n}\quad\binom{n}{k}m^{k}\dg^{n - k} \overset{(1)}{>} m^{n} - c\cl{\frac{m^{n} - x}{n}} \overset{(2)}{>} \cancel{m^{n} - m^{n}} + x = x \]
			בעבור הנימוקים הבאים: 
			\begin{itemize}
				\item א''ש $(1)$ נכון בעבור $c$ הוא מספר הטבעיים האי־זוגיים בין $1$ ל־$n$ (כמות האיברים בסכום), שכן לכל איבר בסכום מתקיים: 
				\[ - \binom{n}{k}m^{k}\dg^{n - k} > -\binom{n}{k}m^{k}\cl{\sqrt[n - k]{\frac{m^{n} - x}{\binom{n}{k}m^{k}n}}}^{n- k} \dequad \dequad \ = - \cancel{\binom{m}{k} \cdot m^{k} \cdot \frac{1}{\binom{m}{k}m^{k}}} \cdot \frac{m^{n} - x}{n} = \frac{m^{n} - x}{n} \]
				\item א''ש $(2)$ נכון כי מספר האיברים האי־זוגיים בין $1$ ל־$n$ קטן ממש מ־$n$. 
			\end{itemize}
			משום ש־$\dg > m$ אז $(\dg - m) > 0$. סה''כ מצאנו $\dg$ כך ש־$(m - \dg)^{n} > x$, כלומר $\sup A - \dg \notin A$. מהגדרת הסופרמום, קיים $\ag \in A$ כך ש־$\ag > m - \dg$, אבל אז קיים $\eg > 0$ כך ש־$\ag = m - \dg + \eg$ ו־: 
			\[ \ag^{n} = (m - \dg + \eg)^{n} > (m - \dg)^{n} > x \]
			כלומר $\ag^{n} \notin A$ וזו סתירה. 
		\end{itemize}
		סה''כ $(\sup A)^{n} \not < x \land (\sup A)^{n} \not > x$. בהכרח $(\sup A)^{n} = x$ וסיימנו. 
	\end{proof}
	\subsection{כאשר הטור הגיאומטריה מתבדר}
	נוכיח שלכל $x > 1$ מתקיים $\limsi x^{n}  = +\infty$. \begin{proof}
		יהי $M > 0$. נוכיח קיום $N \in \N$ כך ש־$\forall n > N \co x^{n} > M$. נתבונן ב־$N = \ceil{\frac{M - 1}{x - 1}}$. אז: 
		\[ N > \frac{M - 1}{x - 1} \implies M < N(x - 1) + 1 \overset{(1)}{<} (1 + x - 1)^{N} < x^{N} \]
		הא''ש $(1)$ נכון מא''ש ברנולי (וכי $x > 1$ כלומר $x - 1 > 0$). הסדרה $x^{n}$ מונוטונית עולה כי $x > 0$ כלומר $x^{n} < x^{n} \cdot x = x^{n + 1}$. לכן, לכל $n \ge N$ מתקיים $x^{N} < x^{n}$. מטרנזטיביות, $M < x^{n}$ לכל $n \ge N$ וסיימנו. 
	\end{proof}
	
	\subsection{כאשר טור גיאומטרי חסר גבול}
	נוכיח שלכל $x < -1$ מתקיים ש־$\limsi x^{n}$ לא מוגדר. \begin{proof} לשם כך, נצטרך להוכיח שני דברים: שהטור לא מתבדר, והטור לא מתכנס. 
		בעבור $a_n = x^{n}$, ידוע שתת־הסדרה $a_{2n}$ מקיימת $a_{2n} = x^{2n} = (-\sof x)^{2n} = {\sof x}^{2n}$ ומ־$8.2$ היא מתבדרת ל־$+\infty$. באופן דומה הסדרה $-a_{2n + 1}$ מתבדרת ל־$+\infty$, ולכן $a_{2n + 1}$ מתבדרת ל־$-\infty$. מהגדרת גבול חלקי מתבדר, לכל $M \in \R$, ולכל $N \in \N$, קיים $n \ge \N$ כך ש־$x^{n} = a_n > M$ וקיים $m \in \N$ כך ש־$x^{n} = a_n < M$ (כי גבול חלקי אחד הולך ל־$+\inft$ ושני ל־$-\inft$). (הערה: הנתון $x < -1$ בא לידי ביטוי כאן, כאשר אנחנו משתמשים בשוויון $x = -\sof x$ שנכון רק אם $x < 0$ וכן בהתבדרות $\sof x^{2n}$ שנכונה בהנחה ש־$\sof x > 1$). 
		\begin{itemize}
			\item ראשית נוכיח ש־$\limsi x^{n} \neq \pm\infty$. נניח בשלילה ש־$\an$ מתבדר. יהי $M > 0$, אז קיים $N$ כך שלכל $n \ge N$ מתקיים $a_n > \pm M$, אך הראינו קיום $n \ge N$ כך ש־$a_n < \pm M$, וזו סתירה. מכאן ש־$\limasi = \limsi x^{n} \neq \pm\inft$. 
			\item עתה נראה ש־$\forall \ml \in \R \co \limsi x^{n} \neq \ml$, כלומר ש־$\an$ מתכנסת ל־$\ml$. נניח בשלילה קיום $\ml \in \R$ כך ש־$\limasi = \ml$. מכאן שלכל $\eg > 0$ ובפרט עבור $\eg = 1$ מתקיים שהחל מ־$N \in \N$ כלשהו, $\forall n \ge N \co \sof{x^{n} - \ml} < 1$. נפריד למקרים. 
			\begin{itemize}
				\item אם $x^{n}  - 1> 0$ אז ידוע שבעבור אותו ה־$N$ קיים $n > N$ כך ש־$x^{n} > 1 + \ml$ (עבור $M = 1 + \ml$) כי $a_{2n}$ ת''ס מתבדרת ל־$+\infty$. 
				\item אם $x^{n}  - 1 < 0$ אז ידוע שבעבור אותו ה־$N$ קיים $n > N$ כך ש־$x^{n} < 1 + \ml$ (עבור $M = 1 + \ml$) כי $a_{2n + 1}$ ת''ס מתבדרת ל־$-\infty$. 
			\end{itemize}
			בהתאם להגדרת ערך מוחלט, בעבור אותו ה־$n$ מתקיים $\sof{a_n - \ml} > 1$ וזו סתירה. 
		\end{itemize}
		מכאן שהסדרה לא מתכנסת לשום איבר, ולא מתבדרת ל־$\pm\inft$. סה''כ $\limsi x^{n}$ אינו מוגדר בעבור $x < -1$. 
	\end{proof}
	
	\section{}
	טענה: $\{\sin n \mid n \in \N\}$ צפוף ב־$[0, 1]$. 
	
	\textbf{הנחות: }
	\begin{itemize}
		\item $\pi \in \R \setminus \Q$, שנתון שאלה. 
		\item ש־$\sin x$ מונוטוני עולה ב־$[-\pi, \pi]$. 
		\item $\sin x$ מחוזרי במחזור של $2\pi$. 
	\end{itemize}
	יש צורך להניח את שתי ההנחות השניות, כי לא הגדרנו את $\sin x$. בהינתן $r \in \R\setminus \Q$, נגדיר את הקבוצות הבאות: 
	\[ A_r = \ccb{n \bmod r \mid n \in \N} \quad B_r = \ccb{\ccb{\frac{n}{r}} \mid n \in \N} \quad \cl{\ccb{x} := x - \floor x, \ n \bmod r = n - \floor{\frac{n}{r}}\cdot r} \]
	\textbf{למה 1. }$B_{r}$ צפופה ב־$[0, 1]$. \begin{proof}
		מצפיפות הרציונליים ב־$\R$ בהכרח קיים $\Q \ni q \in [0, \eg)$. בה''כ $q = \frac{n}{m}$ עבור $n, m \in \N$ כלשהם ואז $q > \frac{n}{m} > \frac{1}{m} = s$ (כלומר, $s$ שבר מצרי). 
		
		נתבונן בקבוצה הבאה: 
		\[ \ccb{\ccb{\frac{0}{r}}, \ccb{\frac{1}{r}}, \ccb{\frac{2}{r}} \dots \ccb{\frac{m + 1}{r}}} = C_{/r} \]
		\textit{הערה: }$\ccb{x}$ הוא למעשה החלק השברי, או $x\bmod 1$. 
		שכוללת $m + 1$ איברים בקטע $[0, 1]$. האיברים שונים, שכן אם לא כן, אז נקבל $\ccb{\frac{a}{r}} = \ccb{\frac{b}{r}}$ ואז: 
		\[ 0 = \ccb{\frac{a - b}{r}} \implies \frac{a - b}{r} = n \in \N \implies r = \frac{a - b}{n} \]
		אך $a, b \in \N$ וכן $n \in \N$ ומכאן ש־$r$ רציונלי וזו סתירה. נבחין שיש $m$ קטעים מהצורה $[\frac{k - 1}{m}, \frac{k}{m}) \mapsto k \in [m]$ (עד לכדי הקטע האחרון שיהיה סינגלטון $1$). מעקרון שובך היונים, יש לנו $m + 1$ קטעים ו־$m + 2$ מספרים שונים ב־$C_{/r}$, אזי קיימים שני מספרים $\ag, \bg \in C_{/r}$ שנמצאים באותו הקטע. קיימים $a, b \in \N$ כך ש־$\ag =: \ccb{\frac{a}{r}}, \bg =: \ccb{\frac{b}{r}}$ מהגדרת $C_{/r}$, בה''כ $a > b$, ואז: 
		\[ \left [\frac{k - 1}{m}, \frac{k}{m}\right) \ni \ag - \bg = \underbrace{\ccb{\frac{a - b}{r}}}_{\cg} < \sof{\frac{k - 1}{m} - \frac{k}{m}} = \frac{1}{m} \]
		מכאן שלכל $m \in \N$ יש $\cg \in B_r$ כך ש־$\cg < \frac{1}{m}$ קטן ככל רצוננו. 
		
		נפנה להוכיח את הצפיפות. יהיו $x < y \in [0, 1]$. נסמן $\Dg = x - y$. מהגבול $\limsi \frac{1}{n} = 0$ נוכל לבחור $\frac{1}{m}$ קטן כרצוננו ובפרט נוכל לבחור $\frac{1}{m} < \Dg$. ואז קיים $\cg$ כך ש־$\cg < \frac{1}{m} < \Dg$ ו־$\cg \in B_r$. מארכימדיאניות קיים $k \in \N$ כך ש־$\cg k > x$, והמינימלי מבינהם יסומן $k_0$ (קיים מינימום כי הסדר על המממשיים סדר טוב). בהכרח $\cg k_0 < y$ כי אחרת $k' = k_0 - 1$ מקיים: 
		\[ (x - y)k' < \cg k' < x < \cg k_0 \implies (x - y)k_0 - x < (x - y)k_0 - (x - y) = \Dg k' < \cg k' \implies \cg k_0 > \cg k' > x \]
		וזו סתירה למינימליות של $k_0$. סה''כ קיבלנו: 
		\[ x < k_0 \cg < y \]
		כאן נצטרך את העובדה ש־$x, y \in [0, 1]$: מסיבה זו, $k_0 \cg \in [0, 1]$, ומכאן ש־$k_0 \cg \in B_r$ והוכחנו את הצפיפות (לכל $p \in B_r$ ו־$z \in \Z$, אם $pz \in [0, 1]$ אז $pz \in B_r$). 
	\end{proof}
	
	\textbf{למה 2. }$A_r$ צפופה ב־$[0, r]$. \begin{proof}
		מלמה $1$, $B_{r}$ צפופה ב־$[0, 1]$ מלמה 1. מהגדרה $B_{r\op} = \{\frac{n}{r} \in \N\}$. יהיו $x, y \in [0, r]$. אז $\frac{x}{r}, \frac{y}{r} \in [0, 1]$ ואז קיים $c \in B_{r}$ כך ש־$x < c < y$. מתקיים: 
		\[ \exists n \in \N \co c = \frac{n}{r} \implies \frac{x}{r} < \ccb{\frac{n}{r}} < \frac{y}{r} \implies x < \underbrace{\ccb{\frac{n}{r}} \cdot r}_{B} < y \]
		מתקיים ש־$B =\frac{n}{r}r - \floor{\frac{n}{r}}r = n - \floor{\frac{n}{r}} r$ וסה''כ מהגדרה $B = n \bmod r$ כלומר $B \in A_r$ וסיימנו. 
	\end{proof}
	
	נפנה להוכחת המשפט. 
	\begin{proof}
		יהיו $a < b \in [-1, 1]$. 
		המונוטוניות החזקה ב־$[-\pi, \pi]$ קיימת $\arcsin$ הופכית לצמצום $\sin$ על $[-\pi, \pi]$. ואז נדרוש $a < \sin n < b$ ששקול לכך ש־:
		\[ \sin(\arcsin a) < \sin n < \sin (\arcsin b) \overset{\text{מונוטונית עולה ומחזוריות}}{\iff} \arcsin a + 2\pi k_1 < n + 2\pi k_2 < \arcsin b + 2 \pi k_3 \]
		נתבונן ב־$A_{2\pi}$ שצפופה ב־$[0, 2\pi]$ מלמה 2. בה''כ $a, b \in [0, 2\pi]$ (עד לכדי הוספת $2\pi$). מהצפיפות קיים $c \in A_{2\pi}$ כך ש־$a < c < b$ ומהגדרתה $c = n - \ceil{\frac{n}{2\pi}} \cdot 2\pi$. עבור $k_2 = -\ceil{\frac{n}{2\pi}}$, קיבלנו את הנדרש. 
	\end{proof}
	
	\ndoc
\end{document}