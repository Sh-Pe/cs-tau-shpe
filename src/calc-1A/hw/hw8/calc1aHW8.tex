\documentclass[]{../../../../tex/classes/homework}
\usepackage{../../../../tex/packages/hebrewSupport}
\usepackage{../../../../tex/packages/mathShortcuts}

\usepackage[colorlinks]{hyperref}
\definecolor{mgreen}{RGB}{25, 160, 50}
\definecolor{mblue}{RGB}{30, 60, 200}
\RequirePackage{hyperref}
\hypersetup{
	colorlinks=true,
	citecolor=mgreen,
	linkcolor=black,
	urlcolor=mblue,
	pdftitle={Document by Shahar Perets},
	%	pdfpagemode=FullScreen,
}


\newcommand\anpowers  {\sum_{i = 1}^{\inft} a_n(x - x_0)}
\newcommand\bnpowers  {\sum_{i = 1}^{\inft} a_n(x - x_0)}
\newcommand\limzp{\limin{0^{+}\!}\ }
\newcommand\iitem{\skipitems{1}\item}

\author{שחר פרץ}
\title{חדו''א 1א $\sim$ תרגיל בית 8}
\date{19 בינואר 2026}
\begin{document}
	\maketitle
	\section{}
	נחשב את הגבולות הבאים: 
	\begin{enumerate}[(A)]
		\iitem 
		\begin{multline*}
			\limz \frac{(1+x)^{1983} - (1 + 1983x)}{x^{2} + x^{1983}} = \limz\frac{-1 + 1983x + \sum_{i = 0}^{1983}\binom{i}{1983}x^{i}}{x^{1983} + x^{2}} \\
			= \limz\frac{1 - 1^{-1982} + 1983x^{-1981} + \sum_{i = 0}^{1982}\binom{i}{1983}x^{i - 1983}}{1 + x^{-1981}} = \frac{1}{1} = 1
		\end{multline*}
		\iitem
		\begin{multline*}
			\limin{1}(1 - x)\tan\cl{\frac{\pi x}{2}} = \limin{1} \frac{(x - 1)\sin\cl{\frac{\pi}{2}x}}{\cos\cl{\frac{\pi}{2}x}} \slh \limin 1 \frac{(x - 1)\frac{\pi}{2}\cos\cl{\frac{\pi}{2}x} + \sin\cl{\frac{\pi}{2}x}}{\frac{\pi}{2}\sin\cl{\frac{\pi}{2}x}} \\
			= \limin 1 \cl{(x - 1)\cot\cl{\frac{\pi}{2}x}} + \limin 1\cl{\frac{2}{\pi}\cdot\cancel{\frac{\sin\cl{\frac{\pi}{2}x}}{\sin\cl{\frac{\pi}{2}x}}}} = 0 \cdot 0 + \frac{2}{\pi} \cdot 1 = \frac{2}{\pi}
		\end{multline*}
		\iitem
		\[ 0 = \climz x = \climz x\sqrt{-1 + 2} < \climz x \sqrt{\sin\cl{\frac{1}{x}} + 2} < \climz x\sqrt{1 + 2} = \climz \sqrt 3 x = 0 \]
		סה''כ מסנדוויץ' הגבול הוא $0$. 
		\iitem
		\[ 1 = \frac{1}{1} = \frac{\cancel{\frac{1}{x}}}{1 - \cancel{\frac{1}{x}}} = \dequad \ \ \! \limpmi \frac{x - 1}{x + 1} < \dequad\ \ \!\limpmi\frac{x + \sinx}{x - \sinx} < \dequad\ \ \!\limpmi \frac{x + 1}{x - 1} = \dequad\ \ \!\limpmi \frac{1 + \cancel{\frac{1}{x}}}{1 - \cancel{\frac{1}{x}}} = \frac{1}{1} = 1 \]
		סה''כ מסנדוויץ' הגבול הוא $1$. 
		\item יהי $a > 0$. נמצא את הגבול: 
		\[ \climz \frac{a^{x} - 1}{x} \]
		ראה סעיף י''ח
		\item יהיו $a, b > 0$. נתבונן בגבול:
		\[ \frac{b}{a} = \limz \frac{b\cancel x}{a\cancel x} - \cancel {\limz x} = \limz \frac{bx - 0.5x^{2}}{ax} = \limz \frac{x}{a} \cdot \cl{\frac{b}{x} - \frac{1}{2}} < \limz \frac{x}{a} \cdot \cl{\frac{b}{x} + \frac{1}{2}} = \limz \frac{bx + 0.5x^{2}}{ax} = \limz \frac{b\cancel x}{a\cancel x} + \cancel {\limz x} = \frac{b}{a} \]
		סה''כ מסנדוויץ' סיימנו. 
		\iitem
		\[ \limzp(\sinx)^{\frac{1}{\logx}} = \limzp e^{\log\cl{(\sinx)^{\frac{1}{\logx}}}} = \limzp e^{\frac{\log\sinx}{\logx}} = e^{\limzp \frac{\log\sinx}{\logx}} = \cdots \]
		נפנה לחשב את הגבול למעלה בנפרד: 
		\[ \limzp \frac{\logx\sinx}{\logx} \slh \limzp \frac{\frac{\sinx}{x} + \log\cosx}{1/x} = \limzp \!\sinx + \limzp \!\log\cosx = 0 + \log(1) = 0 \]
		סה''כ נקבל שהגבול כולו שווה ל־: 
		\[ \cdots = e^{0} = 1 \]
		\iitem
		\[ \climi \frac{\sqrt{x + \sqrt{x + \sqrt x}}}{\sqrt{x + 1}} = \sqrt{\ \!\climi\frac{x + \sqrt{x + \sqrt x}}{x + 1}} = \sqrt{\ \!\climi \frac{1 + \sqrt{\frac{1}{x^{2}} + \sqrt{\frac{1}{x^{4}}}}}{1 + \frac{1}{x}}} = \sqrt{\frac{1}{1}}= 1 \]
		\iitem ניעזר בזהות הטריגונומטרית $\sin2x = 2\sinx\cosx$
		\begin{multline*}
			\limz \frac{1 - \cos^{3}x}{\sinx \cdot \sin2x} = \limz \frac{1 - \cos^{3}x}{2\sin^{2}x \cdot \cosx} \slh \limz\frac{3{\cos^2x}\cdot \sinx}{4\cos^2x \cdot \sinx - 2\sin^3x} = \frac{3}{2}\limz \cl{\frac{2\cos^2x \sinx}{\cos^2x\sinx} - \frac{\sin^3x}{\cos^2x \cdot \sinx}}\op \\
			=\frac{3}{2}\cl{\cl{\limz 2\frac{\cancel{\cos^2x\sinx}}{\cancel{\cos^2x\sinx}}}
			 + \cl{\limz \tan^2x \cdot \frac{\cancel{\sinx}}{\cancel{\sinx}}}}\op = \frac{3}{2} + \cl{2 + 0 \cdot 1}\op = \frac{3}{2} \cdot \frac{1}{2} = \frac{3}{4}
		\end{multline*}
		\iitem יהי $a > 0$: 
		\[ \climz \frac{a^{x} - 1}{x} \slh \climz \frac{\ln a \cdot a^{x} - 0}{1} = \ln a \]
	\end{enumerate}
	
	\section{}
	תהיינה $x_0 \neq x_1$ שתי נקודות. נמצא פונקציה $f \co \R \to \R$ הרציפה בדיוק ב־$x_0, x_1$. 
	\begin{proof}
		יהיו $x_0, x_1$ כלשהם. משום שנתון $x_0 \neq x_1$, בהכרח קיימות סביבות $\dg_0$ ו־$\dg_1$ ל־$x_0$ ו־$x_1$ בהתאמה בינהן החיתוך זר (כלומר $(x_0 - \dg_0, x_0 + \dg_0) \cap (x_1 - \dg_1, x_1 + \dg_1) = \varnothing$). נגדיר את הפונקציה $f$ הבאה: 
		\[ f = \begin{cases}
			D(x) (x - x_0) & x \in (x_0 - \dg_0, x + \dg_0) \\
			D(x) (x - x_1) & x \in (x_1 - \dg_1, x + \dg_1) \\
			D(x) & \other
		\end{cases} \]
		נוכיח שהיא מקיימת את הדרוש. 
		\begin{itemize}
			\item \textbf{רציפה ב־$\bm{x_0}$: }הוכחנו ש־$D(x) \cdot x$ רציפה ב־$0$. מהזזת פונקציות $D(x)(x - x_0)$ רציפה ב־$x_0$. כלומר היא רציפה בסביבת $\dg$ קטן ככל רצוננו ובפרט קטן מ־$\dg_0$ סביב $x_0$, וסה''כ גם $f$ רציפה ב־$x_0$ כדרוש. 
			\item \textbf{רציפה ב־$\bm{x_1}$: }כבר הוכחנו. 
			\item \textbf{לא רציפה ב־$\bm{x \notin \{x_0, x_1\}}$: }עבור $x$ שנמצא בסביבות $\dg_1, \dg_0$ של $x_1, x_0$ בהתאמה, הוכחנו זאת בכיתה כאשר דיברנו על $D(x)x$. כנ''ל בעבור $x$ מחוץ לסביבות ש־$f$ מוגדרת בסביבתו כ־$D(x)$. נצטרך להתעסק ספציפית עם $x = x_0 - \dg_0, x_1 + \dg_1$, וכו' (קצוות הקטעים) משום ש־$f$ אינה מוגדרת להיות פונקציה שאנו מכירים בסביבת $x$. נבחין שהגבול מימין ל־$x$ לא קיים במקרה זה, שכן $D(x)$ חסרת גבולות חד־צדדיים בשני צידיה, ו־$f$ מוגדרת להיות $D(x)$ בסביבה חד־כיוונית כלשהי של $x$. 
		\end{itemize}
		סה''כ הראינו שהפונקציה מקיימת את הדרוש. 
	\end{proof}
	
	\section{}
	נוכיח ונפריך את הטענות הבאות: 
	\begin{enumerate}[(A)]
		\item נפריך את שתי הטענות הבאות: 
		\begin{itemize}
			\item אם $f, g$ לא רציפות ב־$x_0$, אז $f + g$ אינה רציפה ב־$x_0$. \begin{proof}[הפרכה]
				נתבונן בפונקציות הבאות: 
				\[ f = D(x) \quad g = -D(x) \]
				בשיעור הראינו ש־$f, g$ אינן רציפות באף נקודה ובפרט ב־$x_0$. אך $f + g = 0$ פונקציה קבועה שרציפה בכל נקודה ובפרט ב־$x_0$. סה''כ סתירה למשפט. 
			\end{proof}
			\item אם $f, g$ לא רציפות ב־$x_0$, אז $f \cdot g$ אינה רציפה ב־$x_0$. \begin{proof}[הפרכה]
				נסתכל על הפונקציה הבאה: 
				\[ f = D(x) = I_{\Q} \quad g = I_{\R \setminus \Q} \]
				כאשר $I_X$ האינדיקטור של הקבוצה $X$ ב־$\R$. נבחין שבכל $x$ מתקיים $f(x) = 0 \iff g(x) \neq 0$, כלומר $f(x) = 0 \lor g(x) = 0$, ומכאן ש־$(f \cdot g)(x) = 0$ כלומר $f \cdot g$ פונקציה קבועה ב־$0$. ידוע מההרצאה ש־$f$ לא רציפה בשום נקודה, וההוכחה על $g$ זהה. סה''כ סתירה למשפט משום שהפונקציה הקבועה רציפה בכל נקודה ובפרט ב־$x_0$. 
			\end{proof}
		\end{itemize}
		\item אם $f$ רציפה בנק' $x_0$ ו־$g$ אינה רציפה ב־$x_0$, אז $f + g, f \cdot g$ אינן רציפות ב־$x_0$. \begin{proof}
			בנקודות מבודדות הטענה מתקיימת באופן ריק. בנקודה $x_0$ שאינה מבודדת, כלומר $f$ מוגדרת בסביבתה, נפריך את הרציפות. משום ש־$g$ אינה רציפה ב־$x_0$, זוהי נקודת אי־רציפות סליקה, או מסוג כלשהו. בהינתן $+$ בה''כ פעולה מחבורת החיבור או מחבורת הכפל ב־$\R$:
			\begin{itemize}
				\item אם זוהי נקודת אי־רציפות סליקה, נקבל בקלות, אז $\limxz g(x) = z \neq g(x_0)$ כלשהו, ואז			$\lim_{x \to x_0}(f + g)(x_0) = \lim_{x \to x_0} f(x) + \lim_{x \to x_0}g(x) = x_0 + z \neq x_0 + x_1$ וסיימנו. 
				\item אם זו נקודת אי־רציפות מסוג ראשון או שני, הגבול $\limxz g(x)$ אינו קיים. משום שמהנתון הגבול $\limxz f(x)$ קיים ושווה ל־$f(x_0)$, מאריתמטיקת גבולות בהכרח $g(x)$ קיים, שכן אחרת:
				\[ \limxz g(x) = \limxz(f(x) + g(x) - f(x)) \seq \limxz (f(x) + g(x)) - \limxz f(x) = f(x) + g(x) - f(x) = g(x) \]
				כאשר השוויון $\seq$ נכון משום ששני הגבולות שמימנו מוגדרים בהתאם לנתון / הנחה בשלילה. מכאן $\limxz g(x)$ גבול שקיים וסתירה. 
			\end{itemize}
		\end{proof}
		\item אם $f \co \R \to \R$ רציפה וחסומה אז היא משיגה ערך מקסימלי/מינימלי ב־$\R$. \begin{proof}[הפרכה]
			נתבונן בפונקציה $f(x) = \arctan x \sinx$. ידוע $\arctan$ פונקציה מונוטונית עולה ממש וחסומה (ב־$\pm\frac{\pi}{2}$) ב־$\R$. מכאן שאין לה מקסימום, כי אם $x$ מקסימום אז $\arctan(x+1) > \arctan(x)$ ומכאן $x + 1$ מקסימום – סתירה, בעבור מינימום $x - 1$ באופן דומה (ממונוטוניות עולה). עוד ידוע ש־$\tan$ רציפה ומכאן ש־$\arctan$ רציפה (הופכית רציפה היא רציפה) וסה''כ $\arctan$ חסומה ורציפה, אך ללא מקסימום או מינימום. 
		\end{proof}
	\end{enumerate}
	
	\section{}
	תהי $f$ פונקציה רציפה ב־$[0, 1]$ המקיימות $f(x) > x$. נוכיח קיום $h > 0$ כך ש־$f(x) > x + h$ לכל $x$ בתחום ההגדרה. \begin{proof}
		נגדיר את הפונקציה $g(x) = f(x) - x$. ידוע ש־$g(x)$ רציפה ומוגדרת בקטע סגור, ולכן ממשפט וויראשטראס היא חסומה ומקבלת את חסמיה, ואת המינימום נסמן $m$. עוד ידוע $g(x) = f(x) - x > 0$ ומכאן ש־$m > 0$ (כי קיים $x$ כך ש־$f(x) = m$ כי $f$ מקבלת את חסמיה, ומכאן $m= f(x) > 0$ וסיימנו). מהיותו מינימום, $g(x) \ge m$. נגדיר $h = \frac{m}{2}$. 
		\[ f(x) - x = g(x) \ge m > h \implies f(x) > x + h \]
		משום ש־$m > 0$ גם $h = \frac{m}{2} > 0$. סה''כ מצאנו $h$ מתאים. 
	\end{proof}
	
	\section{}
	\begin{enumerate}[(A)]
		\item נבנה פונקציה $f \co \R \to \R$ המקבלת כל ערך ב־$\R$ שלוש פעמים. \begin{proof}[בנייה]
			נגדיר את הפונקציה הבאה: 
			\[ f \co \R\to \R \quad f = \begin{cases}
				x-2\floor{\frac{x+1}{3}} & x \bmod 3 \in [2, 3] \uplus {[0, 1)} \\
				2 - x + 4 \floor{\frac{x + 1}{3}} & x \bmod 3 \in [1, 2)
			\end{cases} \]
			נוכיח שהיא מקיימת את הדרוש. רציפות לכל $x \not\equiv 1, 2$ טרוויאלית כי $x$ ו־$\floor{x}$ רציפות, ולכן גם כפלן והרכבתן. בקצוות בין חיבור הקטעים ניאלץ להוכיח שהפונקציה אכן רציפה. 
			\begin{itemize}
				\item עבור $x \equiv 1$, נראה שהיא רציפה: משמאל, רציפה בגלל ש־$2 - x + 4\floor{\frac{x + 1}{3}}$ רציפה, ומימין נצטרך להראות ידנית. נבחין ש־$\floor{\frac{x + 1}{3}} = \frac{x-1}{3}$ (בדיוק בגלל ש־$x \equiv_3 1$).  
				\[ \lim_{z \to x^{-}}f(z) = f(x) = 2 - x + 4\floor{\frac{x + 1}{3}} = 2 - x + 4 \cdot \frac{x - 1}{3} = \frac{6 - 3x + 4x - 4}{3} = \frac{x + 2}{3} \]
				\[ \lim_{z \to x^{+}}f(z) = x - 2 \floor{\frac{x + 1}{3}} = x - 2 \cdot \frac{x - 1}{3} = \frac{3x - 2x + 2}{3} = \frac{x + 2}{3} \]
				מטרנזטיביות הראינו את הדרוש. 
				\item אם $x \equiv 2$, נקבל הוכחה דומה מהכיוון השני: נבחין ש־$\floor{\frac{x + 1}{3}} = \frac{x + 1}{3}$ (כי $x \equiv 2$). ואז נקבל: 
				\[ \lim_{z \to x^{+}} f(z) = f(x) = x - 2 \floor{\frac{x + 1}{3}} = x - 2\frac{x + 1}{3} = \frac{3x - 2x -2}{3} = \frac{x - 2}{3} \]
				והגבול מהצד השני: 
				\[ \lim_{z \to x^{-}} f(z) = 2 - x + 4\floor{\frac{x + 1}{3}} = 2 - x + 4 \cdot \frac{x - 2}{3} = \frac{6 - 3x + 4x - 4}{3} = \frac{x - 2}{3} = f(x) \]
			\end{itemize}
			סה''כ משני הכיוונים הגבול הוא ערך הפונקציה, כלומר, ממשפט, גבולה הוא ערך הפונקציה, ולכן היא רציפה. 
			קל לראות שהיא מקבלת שלוש פעמים: משום שהיא פונקציה רציפה עם גבולות $-\infty + ,\infty$, מערך הביניים איבר $r$ יתקבל לפחות פעם אחת ע''י $x$. נבחין שבפעם הראשונה ש־$r$, $f(x) = f(x + 2) = f(x + 4)$ (כי בפעם הראשונה נקבל $x \bmod 3 \in [0, 1)$). 
		\end{proof}
		\item נוכיח אי־קיום פונקציה רציפה המקבלת כל ערך ב־$\R$ בדיוק פעמיים. \begin{proof}
			נניח בשלילה קיום $f \in \R^{\R}$ רציפה כך ש־$\forall r \in \R .\,\exists!(x_0, x_1) \in \R^{2} \co f(x_1) = f(x_0) = r$. נתחיל מלהוכיח את הלמה הבאה: פונקציה רציפה $f$ כלשהי לא יכולה לשכן קרן בתוך קטע פתוח, כלומר בהינתן $[a, b]$ כלשהם וקרן $[z, \infty)$ או $(\inft, z]$ כלשהי, בהכרח קיים $x$ בקרן כך ש־$x \notin f([a, b])$. ההוכחה פשוטה: הפונקציה $f$ רציפה ב־$(a, b)$ ובעלת גבולות סופיים בקצוות (מרציפות גם־כן), ולכן ממשפט וויראשטראס $f$ חסומה ב־$[a, b]$, דהיינו $f([a, b])$ הינו קטע סגור, ובפרט בהכרח אינו שווה לקרן, כלומר אכן קיים $x$ בקרן. 
			
			נתבונן ב־$r = 0$. נניח ש־$x_0, x_1$ המתאימים לו ובה''כ $x_0 < x_1$. נסמן $x_3 = \frac{x_0 + x_1}{2}$, ובה''כ $f(x_3) < 0$ (אחרת ההוכחה זהה אך הפוכה באי־השוויונות). מהלמה, נתבונן ב־$x_2 := x_1 + a$, ובה''כ $f(x_2) < 0$ (זאת משום שאם לא קיים $a$ מתאים כזה, נוכל לבחור $x_2 = x_0 - a$ עבור $a$ אחר, ולפי הלמה בהכרח הקרן $[0, \infty)$ מכילה איבר מחוץ ל־$(x_0, x_1)$ כלומר אכן קיים $a$ מתאים. מקרה זה בו $x_2 < x_0$ לא שובר את ההוכחה). סה''כ יש לנו מספרים $x_0 < x_3 < x_1 < x_2$. 
			
			משום ש־$f(x_0) = f(x_1)$, בהכרח $(f(x_3), f(x_1)) = (f(x_3), f(x_1))$ ובגלל ש־$f(x_2) < f(x_1) \land f(x_3) < f(x_1)$, בהכרח $(f(x_3), f(x_1) \cap (f(x_2), f(x_1)) \neq \varnothing$. סה''כ קיים $y \in (f(x_3), f(x_1)) \cap (f(x_2), f(x_1)) \cap (f(x_3), f(x_0))$. 
			
			ממשפט ערך הביניים $f$ מקיימת את תכונת דרבו. 
			\begin{itemize}
				\item \textbf{על $\bm{(f(x_3), f(x_1))}$: }בהכרח קיים $z_1 \in (x_3, x_1)$ כך ש־$f(z) = y$. 
				\item \textbf{על $\bm{(f(x_3), f(x_0))}$: }בהכרח קיים $z_2 \in (x_3, x_0)$ כך ש־$f(z) = y$. 
				\item \textbf{על $\bm{(f(x_2), f(x_1))}$: }בהכרח קיים $z_3 \in (x_2, x_1)$ כך ש־$f(z) = y$. 
			\end{itemize}
			משום ש־$(x_3, x_1)$, $(x_3, x_0)$ ו־$(x_2, x_1)$ זרים בזוגות, סה''כ $z_1, z_2, z_3$ שונים בזוגות. כלומר מצאנו שלושה מספרים שונים עבורם $f$ מחזירה את $y$, בסתירה לכך שקיימים בדיוק שניים. 
		\end{proof}
	\end{enumerate}
	
	\section{}
	\begin{enumerate}[(A)]
		\item תהי $f \co [0, 1] \to \R$ פונקציה רציפה עם $f(0) = f(1)$. נוכיח שמשוואה $f(x) = f\cl{x + \frac{1}{2}}$ יש פתרון $x \in \csb{0, \frac{1}{2}}$. \begin{proof}
			נסמן $g(x) = f(x) - f\cl{x + \frac{1}{2}}$. נבחין: 
			\begin{align*}
				g(0) = f(0) - f(0.5) = f(1) - f(0.5) && g(0.5) = f(0.5) - f(1) && \implies g(0) = -g(0.5)
			\end{align*}
			נפצל למקרים. 
			\begin{itemize}
				\item אם $g(0) = 0$ אז $0 = g(0) = f(0) - f(0.5)$ כלומר $f(0) = f(0.5)$ וסיימנו (כי $x \in [0, 0.5]$)
				\item אחרת $g(0) \neq 0$ ובה''כ $g(0) > 0$ ואז $g(0.5) = -g(0) < 0$, כלומר ממשפט ערך הביניים קיים $x \in (0, 0.5)$ כך ש־$g(x) = 0$ (כי $0 \in (g(0), g(0.5)) \supsetneq \{0\}$). נבחין כי אז $f(x) - f(x + 0.5) = 0$ דהיינו $f(x) = f(x + 0.5)$ וסיימנו. 
			\end{itemize}
			כנדרש בכל המקרים. 
		\end{proof}
		\item יהי $a_1, a_2, a_3 > 0$ ו־$\lg_1 < \lg_2 < \lg_3$ מספרים כלשהם. נראה שלמשוואה הבאה בדיוק שני פתרונות: 
		\[ \frac{a_1}{x -\lg_1} + \frac{a_2}{x - \lg_2} + \frac{a_3}{x - \lg_3} = 0 \]
		\begin{proof}
			נבחין שבהכרח $x \neq \lg_i$, ולכן נוכל להכפיל את הסיפור ולקבל: 
			\[ f(x) := (x - \lg_2)(x - \lg_3)a_1 + (x - \lg_3)(x - \lg_1)a_2 + (x - \lg_2)(x - \lg_1)a_3 = 0 \]
			לאחר צמצום: 
			\[ \overbrace{(a_1 + a_2 + a_3)}^{\smash{\ag}}x^{2} - \overbrace{(a_1(\lg_2 + \lg_3) + a_2(\lg_3 + \lg_1) + a_3(\lg_1 + \lg_2))}^{\smash{\bg}}x + \overbrace{(\lg_2\lg_3a_1 + \lg_3\lg_1a_2 + \lg_2\lg_1a_3)}^{\smash{\cg}}1 = 0 \]
			זוהי משוואה מהצורה $\ag x^{2} - \bg x + \cg = 0$, ולפולינום ממעלה שנייה יש לכל היותר שני שורשים. 
			
			עתה נוכיח קיום שורשים כלשהם. נתבונן בגבולות הבאים: יהי $i \in [3]$
			\[ \limin{\lg_i^{\pm}} f(x) \seq \limin{\lg_i^{\pm}} \frac{a_1}{x - \lg_1} + \limin{\lg_i^{\pm}} \frac{a_2}{\lg_2} + \limin{\lg_i^{\pm}} \frac{a_3}{\lg_3} = a + b + \limin{\lg_i^{\pm}} \frac{a_i}{x - \lg_i} = a + b + \pm\inft = \pm\inft \]
			כאשר $a, b$ ממשיים כלשהם תוצאות שארית החלוקה (כי $\lg_i \neq \lg_j$ ל־$i \neq j$ שונים). מכאן שלכל $i \in [3]$, קיימת סביבת $\dg_i$ נקובה של $\lg_i$ בה מימין $f(x)$ גדול ככל רצוננו, ומשמאל $f(x)$ קטן ככל רצוננו. כלומר נוכל לבחור: 
			\[ x_1^{+} \in (\lg_1, \lg_1 + \dg_1) \quad x_1^{-} \in (\lg_2 - \dg_2, \lg_2) \quad x_2^{+} \in (\lg_2, \lg_2 + \dg_2) \quad x_2^{-} \in (\lg_3 - \dg_3, \lg_3) \]
			כך ש־$f(x_1^{-}), f(x_2^{-}) < 0 \land f(x_2^{-}), f(x_2^{+}) > 0$. ממשפט ערך הביניים קיימים $c_1 \in (x_1^{-}, x_1^{+})$ ו־$c_2 \in (x_2^{-}, x_2^{+})$, עבורם $f(c_1), f(c_2) = 0$. סה''כ יש שני שורשים (שונים) ל־$0$, והוכחנו שישנם לכל היותר שני שורשים, כלומר יש בדיוק שני פתרונות ל־$f(x) = 0$ כנדרש. 
			
		\end{proof}
	\end{enumerate}
	
	\section{}
	נתונה $f \co (a, b) \to \R$ רציפה. נוכיח שלכל $n \in \N$ ו־$x_1 \dots x_n \in (a, b)$ קיימת $x \in (a, b)$ כך ש־: 
	\[ f(x) = \frac{1}{n}\cl{\sumnio f(x_i)} = \AM(x_i) \]
	\begin{proof}
		נסמן $x_{\max} = \max (x_i)_{i = 1}^{n}$ ו־$x_{\min} = \min(x_i)_{i = 1}^{n}$. אם $x_{\min} = x_{\max}$ אז $x_i$ קבוע ואז $\AM(x_i) = x_1$ ו־$x_1 \in (a, b)$ יקיים את הדרוש. אחרת $x_{\min} \neq x_{\max}$. ידוע שהממוצע החשבוני של $f(x_i)$ מקיים $\AM(x_i) \in (f(x_{\min}), f(x_{\max}))$ כי ממוצע בין מספרים נמצא בין המקסימום למינימום, ו־$x_{\min} \neq x_{\max}$. סה''כ ממשפט ערך הביניים קיים $x$ כך ש־$f(x) = \AM(x_i)$ כנדרש וסיימנו. 
	\end{proof}
	
	\section{}
	נוכיח שלמשוואות הבאות יש לפחות פתרון אחד בתחום הנתון. 
	\begin{enumerate}[(A)]
		\item נתבונן במשוואה $(1 - x)\cosx = \sinx$. נוכיח שיש לה לפחות פתרון אחד ב־$(0, 1)$. \begin{proof}
			נתבונן בפונקציה $f(x) = (1 - x)\cosx - \sinx$, ב־$[0, 1]$. 
			\[ f(1) = (1 - 1) - \sin(1) = 0 - \sin 1 < 0 - \sin\cl{\frac{\pi}{4}} = - \frac{\sqrt 2}{2} \]
			\[ f(0) = (1 - 0)\cos 0 - \sin 0 = 1 \cdot 1 - 0 = 1 \]
			ממשפט ערך הביניים קיים $c \in [0, 1]$ כך ש־$f(c) = 0$. הראינו ש־$f(1), f(0) \neq 0$ ולכן $c \in (0, 1)$. סה''כ: 
			\[ f(c) = 0 \implies (1 - c)\cos c - \sin c = 0 \implies (1 - c)\cos c = \sin c \]
			כלומר $c$ הוא הפתרון שחיפשנו למשוואה, כדרוש.
		\end{proof}
		\item נתבונן במשוואה $\cot x = \ag x$ בקטע $\cl{0, \frac{\pi}{2}}$. \begin{proof}
			נגדיר את הפונקציה $f(x) = \cotx - \ag x$. נבחין ש־: 
			\[ \climz f(x) = \limz \frac{\cosx }{\sinx} - \ag x = \infty - 0 = +\inft \]
			\[ \limin{\frac{\pi}{4}^{-}} f(x) = \limin{\frac{\pi}{4}^{-}} \frac{\cosx}{\sinx} - \ag x = -\infty - \frac{\ag \pi}{4} = -\inft \]
			לכן קיימת סביבה נקובה $\dg_1$ שבה כל $x \in (0, \dg_1)$ גדול ככל רצוננו ובפרט גדול מ־$0$. נבחר $x = \frac{\dg_1}{2} \in (0, \dg_1)$ ומכאן $f(x) > 0$. מנגד קיימת סביבה נקובה ימנית $\dg_2$ של $\frac{\pi}{4}$ שבה כל $x \in (\frac{\pi}{4} - \dg_2, \dg_2)$ קטן ככל רצוננו ובפרט קטן מ־$0$. נבחר $y = \frac{\pi}{4} - \frac{\dg_2}{2}$ ונקבל $f(y) < 0$. נבחין ש־$x, y \in \cl{0, \frac{\pi}{4}}$ ולכן ממשפט ערך הביניים קיים $c \in \cl{0, \frac{\pi}{4}}$ כך ש־$f(c) = 0$. נקבל: 
			\[ f(c) = 0 \implies \cot c - \ag c = 0 \implies \cot c = \ag c \]
			וסה''כ $c$ הוא הפתרון שביקשנו כנדרש. 
			
		\end{proof}
	\end{enumerate}
	
	\section{}
	יהי $P(x)$ פולינום שאינו פולינום האפס. נוכיח שלמשוואה $\sof{P(x)} = e^{x}$יש לפחות פתרון ממשי אחד. \begin{proof}
		יהי $P(x)$ פולינום ממעלה $n$ עם מקדמים $a_0 \dots a_n$. נגדיר $f(x) = \sof{P(x)} - e^{x}$. ידוע $e^{x}$ מונוטונית עולה. ידוע $e^{0} = 1$. עוד נבחין ש־$\sof{P(x)} > 1$ המ''מ, שכן הגבול הבא: 
		\[ \climpmi \sof{P(x)} - 1 = \!\limpmi \sof{\sum_{i = 0}^{n}a_ix^{n}} - 1 = \!\limpmi {\sof{(a_i - 1)x^{-n}\sum_{i = 1}^{n}a_ix^{i - n}}}\cdot{x^{n}} = \climpmi\sof{a_n}\cdot{x^{n}} = \sof{a_n} \cdot \pm\inft = \infty \]
		כלומר עבור $x < 0$ מתקיים $\sof{P(x)} - 1 > 0$. ממונוטוניות $e^{x}$ ב־$(-\infty, 0]$ בהכרח $\sof{P(x)} - e^{x} > \sof{P(x)} = e^{0} > 0$ עבור אותו ה־$x < 0$. סה''כ מצאנו $x$ כך ש־$f(x) > 0$. נסמנו $x_1$. 
		
		הראינו בעבר לסדרות (ואפשר מהיינה + מונוטוניות להראות גם לפונקציות) ש־$\lim_{x \to \inft} \frac{\sof{P(x)}}{e^{x}} = 0$. מכאן (מחסמים אסימפטוטיים) שלכל $c > 0$ החל מ־$x_0$ כלשהו $c\sof{P(x)} < e^{x}$ ובפרט עבור $c = 1$ נקבל $\sof{P(x)} - e^{x} < 0$. סה''כ מצאנו $x_0$ כך ש־$f(x_0) < 0$ ו־$x_1$ כך ש־$f(x_1) > 0$. משום ש־$f$ רציפה מערך הביניים נקבל שקיים $\tl x$ כך ש־$f(\tl x) = 0$ (כי $0 \in (f(x_0), f(x_1))$) וסה''כ $\tl x$ מוכיח את הדרוש. 
	\end{proof}
	
	\section{}
	נוכיח שפונקציה מחזורית ורציפה ב־$\R$ מקבלת מינימום ומקסימום. 
	\begin{proof}
		מהיותה מחזורית קיים $r$ ו־$x_0 \in (0, r)$ כך ש־$f(x_0 + rk) = f(x_0)$ לכל $k \in \Z$. נסמן $A_k = [x_0 + rk, x_0 + r(k + 1))$ לכל $k \in \Z$. נבחין ש־$f$ רציפה ו־$A_0$ קומפקטית כלומר ממשפט וויראשטראס $f$ מקבלת מינימום ומקסימום ב־$A_0$, את המינימום נסמן ב־$x^{-} \in A_0$ ואת המקסימום ב־$x^{+} \in A_0$. ממחזוריותה $f\cl{A_k} = f\cl{A_0}$ כלומר: 
		\[ \Img f = f(\R) = f\Big(\biguplus_{k \in \Z}A_k\Big) = \bigcup_{\mathclap{k \in \Z}}f(A_k) = f(A_0) \]
		סה''כ $\Img f$ בעלת מקסימום ומינימום $x^{+}$ ו־$x^{-}$, כלומר $x^{+}$ ו־$x^{-}$ המקסימום והמינימום של כל $f$, וסה''כ $f$ מקבלת מקסימום ומינימום מהגדרת תמונה. 
	\end{proof}
	
	
	
	\ndoc
	
\end{document}
