\documentclass[]{../../../../tex/classes/homework}
\usepackage{../../../../tex/packages/hebrewSupport}
\usepackage{../../../../tex/packages/mathShortcuts}

\usepackage[colorlinks]{hyperref}
\definecolor{mgreen}{RGB}{25, 160, 50}
\definecolor{mblue}{RGB}{30, 60, 200}
\RequirePackage{hyperref}
\hypersetup{
	colorlinks=true,
	citecolor=mgreen,
	linkcolor=black,
	urlcolor=mblue,
	pdftitle={Document by Shahar Perets},
	%	pdfpagemode=FullScreen,
}


\newcommand\anpowers  {\sum_{i = 1}^{\inft} a_n(x - x_0)}
\newcommand\bnpowers  {\sum_{i = 1}^{\inft} a_n(x - x_0)}
\newcommand\limzp{\limin{0^{+}\!}\ }
\newcommand\iitem{\skipitems{1}\item}

\author{שחר פרץ}
\title{חדו''א 1א $\sim$ תרגיל בית 10}
\begin{document}
	\maketitle
	
	\section{}
	נניח $f \co [a, b] \to \R$ פונקציה רציפה וגזירה ב־$(a, b)$. נניח כי $f^{2}(b) - f^{2}(a) = b^{2} - a^{2}$. נוכיח שקיים $c$ כך ש־$f'(c)f(c) =c$. 
	\begin{proof}
		נגדיר את הפונקציה $g(x) = x^{2}$. נבחין $g'(x) = 2x$. נגדיר $h(x) = f^{2}(x)$. נבחין $h'(x) = f(x)f'(x) + f'(x)f(x) = 2f(x)f'(x)$ מכלל המכפלה. 
		\begin{itemize}
			\item אם $a^{2} \neq b^{2}$ משום שידוע $a^{2} - b^{2} = f^{2}(a) - f^{2}(b)$ נוכל לחלק אגפים ולקבל $\frac{f^{2}(a) - f^{2}(b)}{a^{2} - b^{2}} = 1$. ממשפט קושי, קיים $c \in (a, b)$ כך ש־:
			\[ 1 = \frac{f^{2}(a) - f^{2}(b)}{b^{2} - a^{2}} = \frac{h(b) - h(a)}{g(b) - g(a)} = \frac{h'(c)}{g'(c)} = \frac{2f(c)f'(c)}{2c} = \frac{f(c)f'(c)}{c} \]
			נכפול ונקבל $f'(c)f(c) = c$ כנדרש. 
			\item אם $a^{2} = b^{2}$ אז $f^{2}(b) - f^{2}(a)$. המקרה הזה לא נכון באופן כללי ואני מניח שנפלה כאן טעות: לדוגמה עבור $0 = a = b$ והפונקציה $f(x) = 0$, מתקיים שהקטע $[a, b] = \{a\}$ כלומר $f$ רציפה כי הנקודה היחידה בה היא מוגדרת מבודדת, וגזירה באופן ריק ב־$(a, b)$, אך לא קיימת שום $c \in (a, b) = \varnothing$ המקיימת תנאים כאלו ואחרים כי הקבוצה הריקה ריקה. עם זאת $f^{2}(a) - f^{2}(b) = 0 - 0 = a^{2} - b^{2}$. 
		\end{itemize}
%		\envendproof
	\end{proof}
	
	
	\section{}
	נבדוק אילו מהפונקציות רציפות במ''ש. 
	\begin{enumerate}[(A)]
		\item $f(x) = \lnx$ ב־$[1, \infty)$. \begin{proof}[רציפה במ''ש]
			נגזור ונקבל $\lnx'= \frac{1}{x}$. בתחום $(1, \inft)$ הנגזרת חסומה: חסם עליון $\frac{1}{1} = 0$, וחסם מלמטה $\frac{1}{x} > 0$. סה''כ $\lnx$ רציפה במ''ש בקטע הנתון משום שנגזרתה חסומה. 
		\end{proof}
		\item $f(x) = \lnx$ ב־$(1, 0)$. \begin{proof}[אינה רציפה במ''ש]
			נוכיח שהיא איננה רציפה במ''ש. נבחר $\eg = \frac{1}{2}$ ויהי $\dg > 0$. 
			נפרק למקרים. אם $\dg > 1$ אז נוכל לבחור כל $x, y \in (0, 1)$ ובפרט $y = 0.5$ ובגלל ש־$\lim_{x \to 0^{+}}\lnx = -\infty$ נוכל לבחור $x \in (0, 1)$ כך ש־$f(x)$ קטן ככל רצוננו ובפרט קטן מ־$1 - f(y)$ ואז נקבל: (כי $f(x) <  f(y) < 0$)
			\[ f(x) < f(y) - 1 \implies 1 < f(y) - f(x) = \sof{f(y) - f(x)} \]
			סתירה. אחרת $\dg < 1$, ואז נבחר $y = \frac{\dg}{2}$ וגם $x = \frac{4\dg}{4}$. ראשית כל נבחין ש־$\sof{x - y} = \frac{\dg}{2} < \dg$. נוסף על כך $x, y \in (0, 1)$ כי $0 < \dg < 1$. נבחין ש־: (כי $x > y$ כלומר $0 > f(x) > f(y)$)
			\[ \sof{\lnx - \ln y} = \lnx - \ln y = \ln\cl{\frac{x}{y}} = \ln\cl{\frac{\frac{4}{4}\can \dg}{\frac{1}{2} \can \dg}} = \ln\cl{\frac{8}{4}} = \ln\cl{2} > 0.5 = \eg \]
			סתירה. 
		\end{proof}
		\item $f(x) = e^{x}$ בתחום $(0, 1)$: \begin{proof}[רציפה במ''ש]
			ידוע $f(x)$ רציפה ו־$(0, 1)$ קומפקטי. ממשפט מההרצאה $f$ רציפה במ''ש. 
		\end{proof}
		\item $f(x) = e^{x}$ ב־$(0, \inft)$. \begin{proof}[אינה רציפה במ''ש]
			נבחר $\eg = \frac{1}{2}$ ויהי $\dg > 0$. נבחר $x = 1 - \ln \dg$ וכן $y = x + \frac{\dg}{2}$. מלגראנג', קיים $c_0 \in (x_0, x_0 + \frac{\dg}{2})$
			\[ \frac{e^{x + \frac{\dg}{2}} - e^{x}}{x + \frac{\dg}{2} - x} = \frac{e^{y} - e^{x}}{y - x} = f'(c_0) = e^{c_0} \]
			נכפיל אגפים ונקבל ממונוטוניות ורציפות $e^{x}$ (ממנה נסיק $e^{y} > e^{x}$ וגם $e^{c_0} > e^{x_0}$): 
			\[ \sof{f{x} - f{y}} =e^{y} - e^{x} = \frac{\dg}{2}e^{c_0} > \frac{\dg}{2}e^{1 - \ln \dg} = \frac{\dg}{2}e^{\ln\cl{\frac{1}{\dg}}} = \frac{\can \dg}{2} \cdot \frac{1}{\can \dg} = \frac{1}{2} \]
			סה''כ מצאנו $x, y$ כך ש־$\sof{f(x) - f(y)} > \frac{1}{2}$ בסתירה לרציפות במ''ש. 
		\end{proof}
	\end{enumerate}
	
	\section{}
	תהי $f \co [0, \inft) \to \R$ רציפה וגזירה ב־$(0, \inft)$. נניח ש־$\limi f(x) + f'(x) =5$. נראה ש־$c$ רציפה במ''ש. 
	\begin{proof}
		נסמן $c = 5$. נוכיח $f(x)$ חסומה. ידוע $f(x) + f'(x)$ חסומה ע''י $M$ כלשהו שכן המ''מ $x_0$ כלשהו היא חסומה בסביבת $(c - \eg, c + \eg)$ (נגיד עבור $\eg = 1$), וקודם לכן ממשפט וויראשטראס היא רציפה בקטע סגור $(0, x_0)$ ולכן מקבלת את חסמיה. ניקח מקסימום בין $c + \eg$ לבין החסם ב־$(0, x_0)$ ממשפט וויראשטראס, ונקבל חסם עליון ל־$f(x) + f'(x)$. נסמנו ב־$M$. 
		יהי $x \in \R$, נוכיח ש־$f(x)$ חסומה. נתחיל מלחסום מלמעלה. 
		\begin{itemize}
			\item אם $f'(x) > 0$, אז $f(x) < f(x) + f'(x) < M$ וסיימנו. 
			\item אם $f'(x) < 0$, נפריד למקרים. אם $f'(x)$ הנקודה המקסימלית ב־$f(x)$, אז $f(x)$ חסם עליון לפונקציה $f$ וסיימנו (אם כי $M$ אינו החסם במקרה זה). אחרת ישנה נקודה $y > x$ כלשהי בה $f(y) > f(x)$, ומשום שהפונקציה עלתה בתווך זה, נוכל למצוא $y' \le y$ בה $f'(y') > 0 \land f(y') > f(x)$ (מרול). במקרה זה: 
			\[ f(x) < f(y') < f(y') + f'(y) < M \]
			וסיימנו. 
		\end{itemize}
		נוכל למצוא חסם תחתון באופן דומה. סה''כ $f(x)$ חסומה, נסמן את חסמה $P$. נסיק שאם $f(x)$ חסומה ע''י $P$ וכן $f'(x) + f(x)$ חסומה ע''י $M$, אז: 
		\[ \sof{f'(x)} = \sof{f'(x) + f(x) - f(x)} \le \sof{f'(x) + f(x)} + \sof{-f(x)} < M + P \]
		כלומר $f'(x)$ חסומה. משום שהנגזרת חסומה, ממשפט שהראינו בכיתה $f$ רציפה במ''ש, כדרוש. 
		
	\end{proof}
	
	\section{}
	תהי $f \co (a, b) \to \R$ גזירה עם נגזרת חיובית, ו־$f'$ מתאפסת רק בנקודה אחת. נוכיח ש־$f$ עולה ממש. 
	\begin{proof}
		נסמן את נקודת ההתאפסות של $f$ ב־$c$. נבחין ש־$f'$ חיובית ב־$(b, c)$ וחיובית ב־$(a, c)$ ולכן $f$ עולה ממש בתחומים אלו (כי צמצומה עולה ממש, ואז אפשר להשתמש במשפט הידוע). 
		נניח בשלילה ש־$f$ איננה עולה ממש. מכאן שקיימות $x < y \in (a, b)$ עבורן $f(x) \ge f(y)$. נפריד למקרים. 
		\begin{itemize}
			\item אם $x, y$ נמצאים שניהם ב־$(a, c)$ או נמצאים שניהם ב־$(b, c)$, סתירה למונוטוניות בקטע. 
			\item אם $x, y$ נמצאים בתחומים שונים (או שווים ל־$c$), אז $x \in (a, c]$ ו־$y \in [c, b)$. בה''כ $y \neq c$ (במקום $y'$ נגדיר $x'$, הוכחה סימטרית). נתבונן ב־$y' = \frac{y' + c}{2} \in (a, c)$, בגלל ש־$y' < y$ נבחין $f(y') < f(y)$ (כי $f$ עולה ממש ב־$(a, c)$) ולכן $f(x) \ge f(y) > f(y')$ כלומר $f(x) - f(y) > 0$. נבחין שמשום שהם בקטעים שונים $y' > x$ כלומר $x - y' < 0$ וסה''כ מלגראנג': 
			\[ 0 > \frac{f(x) - f(y')}{x - y'} = f'(d) \]
			עבור $d \in (x, y')$ כלשהו. כלומר מצאנו נקודה בה הנגזרת שלילית, סתירה. 
		\end{itemize}
		סה''כ בשני המקרים הגענו לסתירה דהיינו $f(x) < f(y)$ והפונקציה מונוטונית עולה ממש כנדרש. 
	\end{proof}
	
	\section{}
	נוכיח באמצעות קושי את האי־שוויונות הבאים: 
	\begin{enumerate}[(A)]
		\item 
		\[ x - \frac{x^{2}}{2} < \log(1 + x) < x \]
		\begin{proof}
			נתחיל מ־$<$. נגדיר $f(x) = x - \log(1 + x)$ ו־$g(x) = \frac{x^{2}}{2}$. נפעיל קושי: קיים $d \in (0, x)$ כך ש־: 
			\[ \frac{x - \log(1 + x) - 0}{\frac{x^{2}}{2} - 0} = \frac{f(x) - f(0)}{g(x) - g(0)} = \frac{f'(d)}{g'(d)} = \frac{1 - \frac{1}{1 + d}}{d} = \frac{\frac{d}{1 + d}}{d} = \frac{1}{1 + d} < 1 \]
			(האי־שוויון כי $d > 0$) נכפיל אגפים: 
			\[ x - \log(1 + x) < \frac{x^{2}}{2} \implies x - \frac{x^{2}}{2} < \log(1 + x) \]
			באופן דומה בעבור הצד השני, נגדיר $f(x) = \log(1 + x)$ ו־$g(x) = x$. נפעיל קושי ונקבל קיום $d \in (0, x)$ כך ש־: 
			\[ \frac{\log(1 + x) - 0}{x - 0} = \frac{f(x) - f(0)}{g(x) - g(0)} = \frac{f'(d)}{g'(d)} = \frac{\frac{1}{1 + d}}{d} = \frac{1}{d + d^{2}} < 1 \]
			(האי־שוויון כי $d > 0$) נכפיל אגפים ונקבל: 
			\[ \log(1 + x) < x \]
			כנדרש. סה''כ הראינו את אי־השוויון משני הצדדים. 
		\end{proof}
		\item 
		\[ 1 + \frac{x}{2} - \frac{x^{2}}{8} < \sqrt{1 + x} < 1 + \frac{x}{2} \]
		\begin{proof}
			בשביל הספורט נעשה את זה עם שארית לגראנג'. נגזור את $f(x) := \sqrt{1 + x}$ לקבלת $3$ האיברים הראשונים בטור הטיילור. 
			\begin{align*}
				f'(x) = \frac{1}{2 {(1 + x)^{\frac{1}{2}}}} && f''(x) = -\frac{1}{4(1 + x)^{\frac{3}{2}}}
				&& f'''(x) = \frac{3}{8(1 + x)^{\frac{5}{2}}}
			\end{align*}
			(מסתבר שאפשר להשתמש כאן במשפט הבינום המוכלל שזה מגניב אבל overkill) מכאן: 
			\begin{align*}
				f(x) &= f(0) + \frac{f'(0)}{1!}x + \frac{f''(0)}{2!}x^{2} + R_2(x) \\
				&= 1 + \frac{1}{2}x - \frac{1}{8}x + R_2(x)
			\end{align*}
			משארית לגראנג' קיים $c_+ \in (0, x)$ כך ש־
			\[ R_2(c_+) = \frac{f^{(3)}(c_+)}{3!}c_+^{3} > 0 \]
			משום ש־$f^{(3)}$ פונקציה חיובית בכל תחומה, ו־$c_+>0$ כלומר גם $c_+^{3} > 0$. באופן דומה, בעבור פיתוח טיילור נמוך יותר: 
			\begin{align*}
				f(x) &= f(0) + \frac{f'(0)}{1!}x + R_1(x) = 1 + \frac{1}{2}x + R_1(x)
			\end{align*}
			נקבל משארית לגראנג' קיום $c_- \in (0, x)$ כך ש־
			\[ R_1(c_-) = \frac{f^{(2)}(c_-)}{2!}c_-^{(2)} < 0 \]
			כי בבירור $f^{(2)}(x) < 0$ בכל תחומה ו־$c_-^{2} > 0$. סה''כ:
			\begin{gather*}
				1 + \frac{x}{2} + \frac{x^{3}}{8} < 1 + \frac{x}{2} - \frac{x^{3}}{8} + R_2(x) = \sqrt{1 + x} \\
				1 + \frac{x}{2} > 1 + \frac{x}{2} + R_1(x) = \sqrt{1 + x}
			\end{gather*}
			כנדרש. 
		\end{proof}
	\end{enumerate}
	
	\section{}
	נניח $f$ גזירה בסביבת $x$, וגזירה פעמיים בנקודה $x$. נוכיח ש־: 
	\[ \lim_{x \to 0} \frac{f(x + h) - 2f(x) + f(x - h)}{h^{2}} = f''(x) \]
	\begin{proof}
		ניעזר בכלל לופיטל: 
		\begin{multline*}
			\climh \frac{f(x + h) - 2f(x) + f(x - h)}{h^{2}} \slh \climh \frac{f'(x + h) - f'(x - h)}{2h} = \climh \frac{f'(x + h) - f(x) + f(x) - f'(x - h)}{2h}
			\\
			= \frac{1}{2} \climh \cl{\frac{f'(x + h) - f'(x)}{x + h - x}} + \frac{1}{2}\climh\cl{\frac{f'(x) - f'(x - h)}{x - (x - h)}} = \frac{1}{2}f''(x) + \frac{1}{2}f''(x) = f''(x)
		\end{multline*}
		נצטרך להצדיק את השימוש בלופיטל: תוצאת הגבול אכן קיימת שכן מהנתון לגזירות פעמיים ב־$x$ מתקיים בהכרח ש־$f''(x)$ קיים. נוסף על כך: 
		\begin{align*}
			\limh h^{2} = 0 && \limh f(x + h) - 2f(x) + f(x - h) = f(x) - 2f(x) + f(x) = 0
		\end{align*}
		כלומר הגבול שלנו אכן מהצורה $\frac{0}{0}$ וחוקי להפעיל את כלל לופיטל. סה''כ הראינו את הדרוש. 
%		ניעזר בטיילור. נבצע פיתוח טיילור נסדר $4$ ל־$f(x - h)$ ו־$f(x + h)$: 
%		\[ \begin{WithArrows}
%			&\limh \frac{f(x + h) - 2f(x) + f(x - h)}{h^{2}} \\
%			&= \climh \frac{\cl{f(x + h)+ hf'(x + h) + \frac{h^{2}}{2}f''(x + h) + R_3(x + h)} - 2f(x) + \cl{f(x - h)+ hf'(x - h) + \frac{h^{2}}{2}f''(x - h) + R_3(x - h)}}{h^{2}} \\
%			&= \!\limh \frac{h^{2}}{h^{2}}f''(x) + \cancel{\frac{R_3(x + h) + R_3(x - h)}{h^{2}}} + \underbrace{\frac{hf'(x + h) - h(f'(x - h))}{h^{2}}}_{(1)} + \underbrace{\frac{f(x - h) + f(x + h)}{h^{2}} - \frac{2f(x)}{h^{2}}}_{(2)}
%		\end{WithArrows} \]
%		כאשר $\frac{R_3(x \pm h)}{h^{2}}$ שואף ל־$0$ ממשפט טיילור. נחקור את $(1)$ ו־$(2)$: 
%		\[ (1) =  \]
	\end{proof}
	
	\section{}
	נוכיח ש־$2x \arctan x \ge \log(1 + x^{2})$ לכל $x \in \R$. \begin{proof}
		נגדיר את הפונקציה $f(x) = 2x\arctan x - \log(1 + x^{2})$. נבחין ש־: 
		\[ f'(x) = \overbrace{2\arctan x + \frac{2x}{x^{2} + 1}}^{(2x\arctan x)'} - \overbrace{\frac{1}{1 + x^{2}} \cdot 2x}^{\log(1 + x^{2})'} = 2\arctan x \]
		נבחין ש־$f' = 2\arctan x$ היא פונקציה שמקיימת $f'(x) > 0$ עבור $x \in \R_{\ge 0}$ ו־$f'(x) < 0$ עבור $x \in \R_{x \le 0}$ (כי $\arctan$ מקיימת את התנאים הללו). מכאן שהיא יורדת ב־$\R_{\le 0}$ ועולה ב־$\R_{\ge 0}$. עוד ידוע $f(0) = 2 \cdot \arctan 0 - \log(1 + 0^{2}) = 0 - 0 = 0$. סה''כ, ב־$\R_{\ge 0}$ הפונקציה עולה לאחר שהיא פוגשת את $0$ כלומר $f(x) \ge 0$, וב־$\R_{\le 0}$ הפונקציה יורדת עד שהיא מגיעה ל־$0$, כלומר $f(x) \ge 0$ גם־כן. סה''כ בכל התחום $f(x) \ge 0$, נציב ונקבל: 
		\[ 0 \le f(x) = 2x\arctan x - \log(1 + x^{2}) \implies 2x\arctan x \ge \log(1 + x^{2}) \]
		לכל $x \in \R$, כנדרש. 
	\end{proof}
	
	\section{}
	תהי $f \co [0, 1] \to \R$ גזירה כך ש־$f(0) = 0$. נניח בנוסף שלכל $x$ בתחום $\sof{f'(x)} \le \sof{f(x)}$. נוכיח $f(x)$ קבועה ב־$0$. \begin{proof}
		נניח בשלילה ש־$f(x)$ אינה קבועה ב־$0$. מכאן קיים $x_0 \in [0, 1]$ כך ש־$f(x_0) \neq 0$. מלגראנג': 
		\[ \frac{f(x_0)}{x_0} = \frac{f(x_0) - f(0)}{x_0 - 0} = f'(c_0) \implies \sof{f(x_0)} = \sof{f'(c_0)x_0} \le \sof{f(c_0)x_0} \]
		עבור $c_0 \in (0, x_0)$ כלשהו. באינדוקציה נגדיר את $c_n$ להיות מספר $c_n \in (0, c_{n - 1})$ כך ש־$\sof{f(c_n)} \ge \sof{x_0f(c_{n - 1})}$. קיים $c_n$ כזה משום שבאינדוקציה מלגראנג' על $c_{n - 1}$: 
		\[ \frac{f(c_{n - 1})}{c_{n - 1}} = \frac{f(c_{n - 1}) - f(0)}{c_{n - 1} - 0} = f'(c_n) \implies \sof{f(c_{n - 1})} = \sof{f'(c_n)c_{n - 1}} \le \sof{f(c_n)c_{n - 1}} \le \sof{f(c_n)x_0} \]
		כאשר $c_{n - 1} \in (0, c_{n - 2})$ ובאינדוקציה $c_{n - 1} \in (0, x_0)$ ולכן $c_n < x_0$ ומכאן הא''ש האחרון. באינדוקציה נקבל: 
		\[ \sof{f(c_n)} \ge \sof{x_0f(c_{n - 1})} = \cdots = \sof{x_0^{n}f(c_0)} \]
		נסמן $\sof{f(c_0)} = k \neq 0$. נקבל $\sof{f(c_n)} \ge x_0^{n}k$. משום ש־$x^{n} \to \inft$ אז $\sof{f(c_n)} \to \inft$ ממשפט ההשוואה. משום ש־$c_n \in (0, x_0) \subseteq (0, 1)$ נוכל תמיד למצוא $c_n$ כך ש־$f(c_n)$ גדול רצוננו, שזו השלילה לכך ש־$f(x)$ חסומה. סה''כ $f(x)$ פונקציה רציפה שאיננה חסומה בקטע סגור $[0, 1]$ וסתירה למשפט וויראשטראס כדרוש. 
	\end{proof}
	
	\section{}
	נתונה הסדרה $a_1 = \frac{\pi}{4}$ בסיס ו־$a_n = \cos(a_{n - 1})$ צעד. נוכיח ש־$\limsi a_n = \ag$ כאשר $\ag$ הוא פתרון המשוואה $\cosx = x$. 
	\begin{proof}
		נבחין ש־$\Img \cosx = (-1, 1)$ ובתחום זה $\Img \cosx|_{(-1, 1)} \subseteq (0.5, 1]$, כלומר $a_{n \ge 2} \in (0.5, 1]$. 
		
		ראשית כל, נוכיח שגבולה הוא $\ag$ במידה והסדרה אכן מתכנסת. במקרה זה, $\limsi a_n = \limsi a_{n + 1} =: \ml$. אזי: 
		\[ \ml = \climsi a_{n + 1} = \climsi \cos\cl{\limsi a_n} = \limsi\cos(\ml) = \cos \ml \]
		ה־$\ml$ היחיד המקיים זאת הוא $\ag$. עתה נותר להראות שהיא מתכנסת. 
		
		כדי להראות התכנסות, נוכיח שהגבול החלקי $a_{2n}$ מתכנס. מנימוקים דומים, אם הוא קיים ערכו $\ag$. הוא אכן קיים: זוהי סדרה חסומה (כבר טענו ש־$a_{n\ge 2}$ חסומה ב־$(0.5, 1]$, ועתה נשאר לקחת מקסימום בין זה לבין $a_1$). נותר להוכיח שהיא מונוטונית יורדת. למה: לכל $x > \ag$ מתקיים $\cos(\cos(x)) < x$. ההוכחה פשוטה: נגדיר $f(x) = \cos(\cos(x)) -x$, ואז $f(\ag) = 0$ מהגדרה. הנגזרת $f'(x) = \sinx \sin(\cosx) - 1$ היא כפולה של שני סינוסים (עד לכדי הרכבה) שתמונתם $[-1, 1]$, וחיסור של אחד, לכן $f'(x) \le 0$. משום ששורשיה בעלי סביבה עבורם הם אינם שורשים, סה''כ $f(x)$ מונוטונית יורדת. משום ש־$f(\ag)$ שורש לכל $x > \ag$ מתקיים $f(x) < 0$ כלומר $\cos \cosx < x$. 
		
		נוכיח באינדוקציה מלאה ש־$a_{2n}$ קטן האיברים שלפניו וגדול מ־$\ag$. 
		\begin{itemize}
			\item עבור $n = 1$, $a_{2n} = a_2 = 0.77 > \ag$ (האי־שוויון חושב נומרית)
			\item עבור $n$ כלשהו, באינדוקציה מלאה ידוע $a_{2n} > a_{2k}$. מה.א. $a_{2n} < \ag$ ולכן מהלמה $a_{2(n + 1)} = \cos(\cos a_{2n}) < a_{2n}$. סה''כ $a_{2(n + 1)} < a_{2k}$ לכל $k \in [n]$. הטענה ש־$a_{2(n +1)}  > \ag$ נכונה כי בסביבת $\ag$ הפונקציה קטנה, ו־$0.73 < \ag <,  a_{2n} < a_2 < 0.77$, ומכאן ש־$\ag$ נמצא בסביבה של $\cosx$ בה היא מונוטונית יורדת (כנ''ל על $\ag$), דהיינו בגלל ש־$a_{2n} \ge \ag$ אז $\cos a_{2n} < \cos \ag$. משום ש־$\cos_{a_{2n}} < \cos \ag$ אך עדיין נמצא בסיבה בה $\cosx$ יורדת (כי $a_{2n} \in (0.5, 1]$) נקבל $\cos (\cos(a_{2n})) > \cos(\cos(\ag)) = \cos(\ag) = \ag$ כנדרש. 
		\end{itemize}
		סה''כ באינדוקציה הראינו שהפונקציה מונוטונית יורדת. היא מונוטונית יורדת וחסומה ולכן הגבול החלקי $a_{2n}$ מתכנס ל־$\ag$. באופן דומה $a_{2n + 1}$ מונוטונית עולה וחסומה, ולכן מתכנסת ל־$\ag$. ממשפט הכיסוי $a_n$ מתכנסת ל־$\ag$ כנדרש. 
	\end{proof}
	
	\ndoc
	
\end{document}
