\documentclass[]{../../../../tex/classes/homework}
\usepackage{../../../../tex/packages/hebrewSupport}
\usepackage{../../../../tex/packages/mathShortcuts}

\usepackage[colorlinks]{hyperref}
\definecolor{mgreen}{RGB}{25, 160, 50}
\definecolor{mblue}{RGB}{30, 60, 200}
\RequirePackage{hyperref}
\hypersetup{
	colorlinks=true,
	citecolor=mgreen,
	linkcolor=black,
	urlcolor=mblue,
	pdftitle={Document by Shahar Perets},
	%	pdfpagemode=FullScreen,
}


\newcommand\anpowers  {\sum_{i = 1}^{\inft} a_n(x - x_0)}
\newcommand\bnpowers  {\sum_{i = 1}^{\inft} a_n(x - x_0)}
\newcommand\limzp{\limin{0^{+}\!}\ }
\newcommand\iitem{\skipitems{1}\item}

\author{שחר פרץ}
\title{חדו''א 1א $\sim$ תרגיל בית 10}
\begin{document}
	\maketitle
	
	\section{}
	נניח $f \co [a, b] \to \R$ פונקציה רציפה וגזירה ב־$(a, b)$. נניח כי $f^{2}(b) - f^{2}(a) = b^{2} - a^{2}$. נוכיח שקיים $c$ כך ש־$f'(c)f(c) =c$. 
	\[ f'(c) = \frac{c -0 }{f(c) - 0} \]
	
	\section{}
	נבדוק אילו מהפונקציות רציפות במ''ש. 
	\begin{enumerate}[(A)]
		\item $f(x) = \lnx$ ב־$[1, \infty)$. \begin{proof}[רציפה במ''ש]
			נגזור ונקבל $\lnx'= \frac{1}{x}$. בתחום $(1, \inft)$ הנגזרת חסומה: חסם עליון $\frac{1}{1} = 0$, וחסם מלמטה $\frac{1}{x} > 0$. סה''כ $\lnx$ רציפה במ''ש בקטע הנתון משום שנגזרתה חסומה. 
		\end{proof}
		\item $f(x) = \lnx$ ב־$(1, 0)$. \begin{proof}[אינה רציפה במ''ש]
			נוכיח שהיא איננה רציפה במ''ש. נבחר $\eg = \frac{1}{2}$ ויהי $\dg > 0$. 
			נפרק למקרים. אם $\dg > 1$ אז נוכל לבחור כל $x, y \in (0, 1)$ ובפרט $y = 0.5$ ובגלל ש־$\lim_{x \to 0^{+}}\lnx = -\infty$ נוכל לבחור $x \in (0, 1)$ כך ש־$f(x)$ קטן ככל רצוננו ובפרט קטן מ־$1 - f(y)$ ואז נקבל: (כי $f(x) <  f(y) < 0$)
			\[ f(x) < f(y) - 1 \implies 1 < f(y) - f(x) = \sof{f(y) - f(x)} \]
			סתירה. אחרת $\dg < 1$, ואז נבחר $y = \frac{\dg}{2}$ וגם $x = \frac{4\dg}{4}$. ראשית כל נבחין ש־$\sof{x - y} = \frac{\dg}{2} < \dg$. נוסף על כך $x, y \in (0, 1)$ כי $0 < \dg < 1$. נבחין ש־: (כי $x > y$ כלומר $0 > f(x) > f(y)$)
			\[ \sof{\lnx - \ln y} = \lnx - \ln y = \ln\cl{\frac{x}{y}} = \ln\cl{\frac{\frac{4}{4}\can \dg}{\frac{1}{2} \can \dg}} = \ln\cl{\frac{8}{4}} = \ln\cl{2} > 0.5 = \eg \]
			סתירה. 
		\end{proof}
		\item $f(x) = e^{x}$ בתחום $(0, 1)$: \begin{proof}[רציפה במ''ש]
			ידוע $f(x)$ רציפה ו־$(0, 1)$ קומפקטי. ממשפט מההרצאה $f$ רציפה במ''ש. 
		\end{proof}
		\item $f(x) = e^{x}$ ב־$(0, \inft)$. \begin{proof}[אינה רציפה במ''ש]
%			נבחר $\eg = 1$ ויהי $\dg > 0$. 
%			\begin{align*}
%				x - y < \dg && f(x) - f(y) = e^{x} - e^{y} = e^{\ln(e^{x} - e^{y})}
%			\end{align*}
		\end{proof}
	\end{enumerate}
	
	\section{}
	תהי $f \co [0, \inft) \to \R$ רציפה וגזירה ב־$(0, \inft)$. נניח ש־$\limi f(x) + f'(x) =5$. נראה ש־$c$ רציפה במ''ש. 
	\begin{proof}
		נסמן $5 = c$. נוכיח לפי הגדרה. יהי $\eg > 0$, עם $\dg = \frac{\eg}{c}$, ויהיו $x, y \in [0, \infty)$ כך ש־$\sof{x - y} < \dg$. נוכיח $\sof{f(x) - f(y)} < \eg$. בה''כ $f(x) > f(y)$. מלגראנג' מתקיים: 
		\[ \frac{f(x) - f(y)}{x - y} = f'(a) \implies f(x) - f(y) = f'(a)(x - y) \]
		ידוע שהחל מ־$x_0$ כלשהו מתקיים $-c < \sof{f'(x) + f(x)} < c$. בפרט עבור $x = a$ נקבל $\sof{f'(a) + f(a)}$ חסום ב־$c$. עתה נקבל: 
		\[ \sof{f(x) - f(y)} = f(x) - f(y) = f'(a)(x - y) \]
		
		לא סגור איך לעשות את זה
		
	\end{proof}
	
	\section{}
	תהי $f \co (a, b) \to \R$ גזירה עם נגזרת חיובית, ו־$f'$ מתאפסת רק בנקודה אחת. נוכיח ש־$f$ עולה ממש. 
	\begin{proof}
		נסמן את נקודת ההתאפסות של $f$ ב־$c$. נבחין ש־$f'$ חיובית ב־$(b, c)$ וחיובית ב־$(a, c)$ ולכן $f$ עולה ממש בתחומים אלו (כי צמצומה עולה ממש, ואז אפשר להשתמש במשפט הידוע). 
		נניח בשלילה ש־$f$ איננה עולה ממש. מכאן שקיימות $x < y \in (a, b)$ עבורן $f(x) \ge f(y)$. נפריד למקרים. 
		\begin{itemize}
			\item אם $x, y$ נמצאים שניהם ב־$(a, c)$ או נמצאים שניהם ב־$(b, c)$, סתירה למונוטוניות בקטע. 
			\item אם $x, y$ נמצאים בתחומים שונים (או שווים ל־$c$), אז $x \in (a, c]$ ו־$y \in [c, b)$. בה''כ $y \neq c$ (במקום $y'$ נגדיר $x'$, הוכחה סימטרית). נתבונן ב־$y' = \frac{y' + c}{2} \in (a, c)$, בגלל ש־$y' < y$ נבחין $f(y') < f(y)$ (כי $f$ עולה ממש ב־$(a, c)$) ולכן $f(x) \ge f(y) > f(y')$ כלומר $f(x) - f(y) > 0$. נבחין שמשום שהם בקטעים שונים $y' > x$ כלומר $x - y' < 0$ וסה''כ מלגראנג': 
			\[ 0 > \frac{f(x) - f(y')}{x - y'} = f'(d) \]
			עבור $d \in (x, y')$ כלשהו. כלומר מצאנו נקודה בה הנגזרת שלילית, סתירה. 
		\end{itemize}
		סה''כ בשני המקרים הגענו לסתירה דהיינו $f(x) < f(y)$ והפונקציה מונוטונית עולה ממש כנדרש. 
	\end{proof}
	
	\section{}
	נוכיח באמצעות קושי את האי־שוויונות הבאים: 
	\begin{enumerate}[(A)]
		\item 
	\end{enumerate}
	
	\section{}
	נניח $f$ גזירה בסביבת $x$, וגזירה פעמיים בנקודה $x$. נוכיח ש־: 
	\[ \lim_{x \to 0} \frac{f(x + h) - 2f(x) + f(x - h)}{h^{2}} = f''(x) \]
	\begin{proof}
		ידוע: 
		\[ f'(x) = \lim_{h \to 0}\frac{f(x + h) - f(x)}{h} \]
		לכן: 
		\[ f''(x) = \lim_{h \to 0}\frac{f'(x + h) - f'(x)}{h} 
		= \limh \frac{\lim_{z \to 0} \frac{f(x + z + h) - f(x + z)}{z} - \lim_{z \to 0}\frac{f(x + z) - f(x)}{z}}{h} \]
	\end{proof}
	
	\section{}
	נוכיח ש־$2x \arctan x \ge \log(1 + x^{2})$ לכל $x \in \R$. \begin{proof}
		נגדיר את הפונקציה $f(x) = 2x\arctan x - \log(1 + x^{2})$. נבחין ש־: 
		\[ f'(x) = \overbrace{2\arctan x + \frac{2x}{x^{2} + 1}}^{(2x\arctan x)'} - \overbrace{\frac{1}{1 + x^{2}} \cdot 2x}^{\log(1 + x^{2})'} = 2\arctan x \]
		נבחין ש־$f' = 2\arctan x$ היא פונקציה שמקיימת $f'(x) > 0$ עבור $x \in \R_{\ge 0}$ ו־$f'(x) < 0$ עבור $x \in \R_{x \le 0}$ (כי $\arctan$ מקיימת את התנאים הללו). מכאן שהיא יורדת ב־$\R_{\le 0}$ ועולה ב־$\R_{\ge 0}$. עוד ידוע $f(0) = 2 \cdot \arctan 0 - \log(1 + 0^{2}) = 0 - 0 = 0$. סה''כ, ב־$\R_{\ge 0}$ הפונקציה עולה לאחר שהיא פוגשת את $0$ כלומר $f(x) \ge 0$, וב־$\R_{\le 0}$ הפונקציה יורדת עד שהיא מגיעה ל־$0$, כלומר $f(x) \ge 0$ גם־כן. סה''כ בכל התחום $f(x) \ge 0$, נציב ונקבל: 
		\[ 0 \le f(x) = 2x\arctan x - \log(1 + x^{2}) \implies 2x\arctan x \ge \log(1 + x^{2}) \]
		לכל $x \in \R$, כנדרש. 
	\end{proof}
	
	\section{}
	תהי $f \co [0, 1] \to \R$ גזירה כך ש־$f(0) = 0$. נניח בנוסף שלכל $x$ בתחום $\sof{f'(x)} \le \sof{f(x)}$. נוכיח $f(x)$ קבועה ב־$0$. \begin{proof}
		אם $f(x) \neq 0$ עבור $x \in [0, 1]$ כלשהו, נסמן $c = f(x)$, אז ממשפט לגראנג': 
		\[ \frac{f(x)}{x} = \frac{f(x) - f(0)}{x - 0} = f'(c) \]
		עבור $c \in (0, x)$ כלשהו (בהכרח לא ריק כי $\lnot f(x) = 0$). נעביר אגפים ונקבל $f(x) = xf'(c)$
	\end{proof}
	
	\section{}
	נתונה הסדרה $a_1 = \frac{\pi}{4}$ בסיס ו־$a_n = \cos(a_{n - 1})$ צעד. נוכיח ש־$\limsi a_n = \ag$ כאשר $\ag$ הוא פתרון המשוואה $\cosx = x$. 
	\begin{proof}
		נבחין ש־$\Img \cosx = (-1, 1)$ ובתחום זה $\Img \cosx|_{(-1, 1)} \subseteq (0.5, 1]$, כלומר $a_{n \ge 2} \in (0.5, 1]$. 
		
		ראשית כל, נוכיח שגבולה הוא $\ag$ במידה והסדרה אכן מתכנסת. במקרה זה, $\limsi a_n = \limsi a_{n + 1} =: \ml$. אזי: 
		\[ \ml = \climsi a_{n + 1} = \climsi \cos\cl{\limsi a_n} = \limsi\cos(\ml) = \cos \ml \]
		ה־$\ml$ היחיד המקיים זאת הוא $\ag$. עתה נותר להראות שהיא מתכנסת. 
		
		כדי להראות התכנסות, נוכיח שהגבול החלקי $a_{2n}$ מתכנס. מנימוקים דומים, אם הוא קיים ערכו $\ag$. הוא אכן קיים: זוהי סדרה חסומה (כבר טענו ש־$a_{n\ge 2}$ חסומה ב־$(0.5, 1]$, ועתה נשאר לקחת מקסימום בין זה לבין $a_1$). נותר להוכיח שהיא מונוטונית יורדת. למה: לכל $x > \ag$ מתקיים $\cos(\cos(x)) < x$. ההוכחה פשוטה: נגדיר $f(x) = \cos(\cos(x)) -x$, ואז $f(\ag) = 0$ מהגדרה. הנגזרת $f'(x) = \sinx \sin(\cosx) - 1$ היא כפולה של שני סינוסים (עד לכדי הרכבה) שתמונתם $[-1, 1]$, וחיסור של אחד, לכן $f'(x) \le 0$. משום ששורשיה בעלי סביבה עבורם הם אינם שורשים, סה''כ $f(x)$ מונוטונית יורדת. משום ש־$f(\ag)$ שורש לכל $x > \ag$ מתקיים $f(x) < 0$ כלומר $\cos \cosx < x$. 
		
		נוכיח באינדוקציה מלאה ש־$a_{2n}$ קטן האיברים שלפניו וגדול מ־$\ag$. 
		\begin{itemize}
			\item עבור $n = 1$, $a_{2n} = a_2 = 0.77 > \ag$ (האי־שוויון חושב נומרית)
			\item עבור $n$ כלשהו, באינדוקציה מלאה ידוע $a_{2n} > a_{2k}$. מה.א. $a_{2n} < \ag$ ולכן מהלמה $a_{2(n + 1)} = \cos(\cos a_{2n}) < a_{2n}$. סה''כ $a_{2(n + 1)} < a_{2k}$ לכל $k \in [n]$. הטענה ש־$a_{2(n +1)}  > \ag$ נכונה כי בסביבת $\ag$ הפונקציה קטנה, ו־$0.73 < \ag <,  a_{2n} < a_2 < 0.77$, ומכאן ש־$\ag$ נמצא בסביבה של $\cosx$ בה היא מונוטונית יורדת (כנ''ל על $\ag$), דהיינו בגלל ש־$a_{2n} \ge \ag$ אז $\cos a_{2n} < \cos \ag$. משום ש־$\cos_{a_{2n}} < \cos \ag$ אך עדיין נמצא בסיבה בה $\cosx$ יורדת (כי $a_{2n} \in (0.5, 1]$) נקבל $\cos (\cos(a_{2n})) > \cos(\cos(\ag)) = \cos(\ag) = \ag$ כנדרש. 
		\end{itemize}
		סה''כ באינדוקציה הראינו שהפונקציה מונוטונית יורדת. היא מונוטונית יורדת וחסומה ולכן הגבול החלקי $a_{2n}$ מתכנס ל־$\ag$. באופן דומה $a_{2n + 1}$ מונוטונית עולה וחסומה, ולכן מתכנסת ל־$\ag$. ממשפט הכיסוי $a_n$ מתכנסת ל־$\ag$ כנדרש. 
	\end{proof}
	
	\ndoc
	
\end{document}
