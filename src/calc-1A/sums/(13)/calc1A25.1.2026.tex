\documentclass[]{../../../../tex/classes/styledArticle}
\usepackage{../../../../tex/packages/hebrewSupport}
\usepackage{../../../../tex/packages/mathShortcuts}
\usepackage{../../../../tex/packages/theoremsSupport}

\author{שחר פרץ}
\title{\textit{חדו''א 1א 13}}
\begin{document}
	\maketitle
	
	נחזור על שארית לגראנג': 
	
	נתבונן בפונקציה $f \co I \to \R$ ותהא $x_0 \in I$ פנימית. יהי $n \in \N^{+}$ ונניח כי $f$ גזירה $n + 1$ פעמים ב־$I$. (שימו לב: לא ב־$x_0$ אלא בכל הקטע). אז לכל $x \in I$ קיים $c$ בין $x$ ל־$x_0$ כך ש־$R_n(x) = \frac{f^{(n + 1)}(c)}{(n + 1)!}(x - x_0)^{n + 1}$. 
	
	נחזור על שארית פאנו: 
	
	תהי $\wg \in I \to \R$ כך ש־$\wg(x_0) = 0$. $\wg$ רציפה ב־$x_0$ וכן $R_n(x) = \wg(x)(x - x_0)^{n}$ לכל $x \in I$. 
	
	נתבונן בהכללה הבאה של שארית לגראנג': 
	\theo{יהי $ \le p \le n + 1$ ונתבונן בפונקציה $f \co I \to \R$ ותהא $x_0 \in I$ פנימית. יהי $n \in \N^{+}$ ונניח כי $f$ גזירה $n + 1$ פעמים ב־$I$. אז לכל $x \in I$ קיים $c$ בין $x_0$ ל־$x$ כך ש־$R_n(x) = \frac{f^{(n + 1)}(c)}{פ p \cdot n!}(x - x_0)^{p}(c - x_0)^{n + 1 - p}$}
	
	כאשר $p = n + 1$ נקבל בדיוק את שארית לגראנג'. כאשר $p = 1$, השארית נקראת שארית קושי. 
	
	\begin{proof}
		יהי $x \in I$. נגדיר $\phi, \psi \co I \to \R$ באופן הבא: 
		\begin{align*}
			\phi(t) := f)x) - f(t) - \frac{f'(t)}{1!}(x - t(x)) - \cdots - \frac{f^{(n)}(t)}{n!}(x - t)^{n} && \psi(t) = (x - t)^{p}
		\end{align*}
		
		נבחין בכמה דברים: ראשית כל $\phi(x) = \psi(x) = 0$. עוד נבחין $\phi(x_0) = R_n(x)$ ו־$\psi(x_0) = (x - x_0)^{p}$. לכל $t \in I$ נבחין שההנגזרת של $\phi$ היא כמו טור טלסקופי: 
		\[ \phi(t) = -f'(t) + f'(t) - f''(t)(x - t) + f'''(x - t) -\frac{f'''(t)}{2!}(x - t)^{2} + \cdots  = -\frac{f^{(n + 1)}(t)}{n!} \]
		קל יותר למצוא את $\psi'$ ולסכם ש־$\psi'(x) = -p(x - t)^{p -1}$. נשים לב שבקטע בין $x$ ל־$x_0$ שתי ה]פונקציות קציפות בקטע הסוגר, רציפות בקטע הפתוח, ו־$\psi$ אינה מתאפסת בקטע הפתוח. מקושי קיים $c$ בין $x$ ל־$x_0$ כך ש־: 
		\[ \frac{R_n(x)}{(x - x_0)^{2}} = \frac{\phi(x) - \phi(x)}{\psi(x) - \psi(y)} = \frac{\phi'(c)}{\psi'(c)} = \frac{\frac{f^{(n + 1)}(c)}{n!}(x - c)^{n}}{p(x - c)^{o - 1}} = \frac{f^{(n + 1)}(c)}{p \cdot n!}(x - c)^{n - p + 1} \]
		
	\end{proof}
	
	
	\exe{נגדיר $I = (-1, 1, f \co)$ גזירה מכל סדרה ומתקיים $\forall x \in (-1, \infty)$ לכל $n \in \N^{+}$ מתקיים $f^{(n)}(x) = (-1)^{n - 1}\frac{(n - 1)!}{(1 + x)^{n}}$}. אז עבור $x \in (-1, \inft)$ נקבל כי שארית לגראנג' של פולינום מק'לורן מסדר $n$ היא $R_n(x) = (-1)^{n}\frac{n!}{(n + 1)!(1 + c)^{n + 1}}c^{n + 1} = (-1)^n\cdot \frac{1}{n + 1} \cdot \frac{x^{n}}{(1 + c)^{n + 1}}$ עבור $c$ בין $x$ ל־$0$ כלשהו. 
	כאשר $0 \le x \le 1$ נגיד למשל ש־:  
	\[ \sof{R_n(x)} \le \frac{1}{n + 1} \cdot \frac{\sof{x}^{n + 1}}{(a + x)^{n + 1}} \le \frac{1}{n + 1} \toinf 0 \]
	מכאן $\sof{x^{n + 1}} \le 1$ וכן $(1 + c)^{n + 1} \ge 1$. 
	
	כאשר $-\frac{1}{2} \le x \le 0$ אז $\sof{R_n(x)} = \frac{1}{n + 1} \cdot \frac{\sof{x}^{n + 1}}{(1 + c)^{n + 1}}$ נבחין ש־$-\frac{1}{2} \le x < c < 0$	 ומכאן $\sof{x} \le \frac{1}{2} \le 1 + 2x < 1 + c < 1$ ןאז $\frac{\sof{x}}{1 + c} \le 1$. 
	
	סה''כ לכל $x \in \csb{-\frac{1}{2}, 1}$ מתקיים $R_n(x) \toinf 0$. 
	
	
	עבור $-1 > x < -\frac{1}{2}$ לא נוכל לחסום את $\frac{\sof x}{c + 1}$ על ידי $1$. ואז שארית לגראנג' תפסיק לעבוד. למקרה זה נשתמש בשארית קושי. לכל $x \in (-1, -0.5)$ ולכל $n \in \N$, קיים $x < c < 0$ כך ש־: 
	\[ \sof{R_n(x)} = \sof{\frac{\frac{(-1)^{n}n!}{(1 + c)^{n + 1}}}{n!}(x - c)^{n}(x - 0)} = \frac{\sof x}{1 + c} \cdot {\sof{\frac{x - c}{1 + c}}^{n}} = \cdots \]
	
	שימו לב! $c$ תלוי ב־$x$ בכל הדברים לעיל. יש כאן טריק קטן שפותר את זה: (ניעזר בכך ש־$c, x$ שליליים)
	\[ \sof{x - c} = \sof x - \sof c < \sof x - \sof x \sof c = \sof x (1 - c) = \sof x (1 + c) \]
	נחזור למעלה: 
	\[ \cdots < \frac{\sof{x}^{n + 1}}{1 + c} \toinf 0 \]
	נבחין ש־$1 + c$ חסום בין $0$ ל־$x$ (למרות שהוא עדיין תלוי ב־$n$!) ואנחנו מחלקים משהו מעריכי בקבוע, ולכן כל הסיפור לעיל הולך ל־$0$. 
	
	
	\exe{הוכיחו/הפריכו: 
	\begin{enumerate}
		\item \hfil $\limxz f'(x) = f'(0)$
		\item קיימת סדרה $\{x_n\}$ השואפת ל־$0$ עבורה $\limsi f'(x_n) = f'(0)$, וכן $x_n \neq 0$ לכל $n$. 
		\end{enumerate}}
	\begin{enumerate}
		\item נראה דוגמה נגדית ל־$1$. נגדיר: 
		\[ f(x) = \begin{cases}
			x^{2}\sin\cl{\frac{1}{x}} & x \neq 0 \\ 
			0 & x = 0
		\end{cases} \]
		אזי לכל $x \neq 0$ $f$ גזירה כמהפלה והרכבה של גזירות. ב־$0$: 
		\[ \limh \frac{f(0 + h) - f(0)}{h} = \limh h \sin \cl{\frac{1}{h}} = 0 \]
		לכן: 
		\[ f'(x) = \begin{cases}
			2x\sin \frac{1}{x} - \cos \frac{1}{x} & x \neq 0 \\
			0 & x = 0
		\end{cases} \]
		הגבול $\limxz 2\sin\cl{\frac{1}{x}} = 0$ וכמו כן ל־$\cos\frac{1}{x}$ ללא גבול ב־$0$, ולכן $f'$ ללא גבול ב־$0$ מאריתמטיקת גבולות. 
		\item זו הוכחה. נציג שני פתרונות. \begin{proof}[פתרון 1]
			נגדיר $x_1 =1$. 
			עבור $n \in \N$ נסמן $a = \frac{x_n}{2}$. ממשפט דרבו $f'$ מקיימת את תכונת דרבו (תכונת ערך הביניים). נסמן $\lg = \min\{f'(0) + \frac{1}{n}, f'(a)\}$ (בה''כ $f'(a) > f'(0)$, אחרת נחסר). קיים $x_{n + 1} \in [0, a)$ כך ש־$f'(x_{n + 1}) = \lg$. לכל $n \in \N$ נקבל $\sof{x_n} \le \frac{1}{2^{n}}$ וגם $\sof{f'(0) - f'(x_n)} \le \frac{1}{n}$. 
		\end{proof}
		עתה נציג פתרון נוסף. 
		\begin{proof}[פתרון 2]
			מהיינה $\limsi \frac{f(\frac{1}{n}) - f(0)}{\frac{1}{n}} = f'(0)$ (בחרנו ספציפית מבין כל הסדרות השואפות ל־$0$). לכל $n \in \N$, $f$ מקיימת את תנאי משפט לגראנג' ב־$[0, \frac{1}{n}]$ ולכן קיים $0 < x_n < \frac{1}{n}$ כך ש־: 
			\[ \frac{f\cl{\frac{1}{n}} - f(0)}{\frac{1}{n}} = f'(x_n) \]
			ו־$0 \neq x_n \to 0$ וכמובן $f'(x_n) \to f'(0)$. 
		\end{proof}
	\end{enumerate}
	
	עתה נתבונן בעוד שאלה שהייתה בתרגיל הבית: תהי $f \co \R \to \R$ פונקציה גזירה פעמיים ב־$x_0 \in \R$. הראו כי: 
	\[ \limh \frac{f(x_0 + h) - f(x_0 - h) - 2f(x_0)}{h^{2}} = f''(x_0) \]
	בבית עשינו כולנו עם לופיטל. שימו לב לנמק הכל ובשום פנים ובאופן לא לעשות לופיטל פעמיים (נתון שהיא גזירה רק פעמיים). 
	\begin{proof}[הוכחה באמצעות טיילור]
		נפתח את פולינום טיילור מסדר $2$ סביב $x_0$ משארית פאנו. קיימת $\wg \co \R \to \R$ כך ש־$\wg(x_0) = 0$, $\wg$ רציפה ב־$x_0$, וכן $\forall x \in \R\co f(x) = f(x_0) + f'(x_0)(x - x_0) + \frac{f''(x_0)}{2}(x - x_0)^{2} + \wg(x)(x - x_0)^{2}$. יהי $h \in \R\setminus \{0\}$. נבחין ש־: 
		\begin{gather*}
			f(x_0 + h) = f(x_0) + f'(x_0)h + \frac{f''(x_0)h^{2}}{2} + \wg(x - h)h^{2} \\
			f(x_0 - h) = f(x_0) + f'(x_0)(-h) + \frac{f''(x_0)h^{2}}{2} + \wg(x - h)h^{2}
		\end{gather*}
		נחזור לביטוי למעלה, ונקבל: 
		\[ \frac{f(x_0 + h) + f(x_0  - h) - 2f(x_0)}{h^{2}} = f''(x_0) + \wg(x_0 + h) + \wg(x_0 - h) \]
		ובגבול: 
		\[ \limh \frac{f(x_0 + h) + f(x_0  - h) - 2f(x_0)}{h^{2}} = f''(x_0) \]
	\end{proof}
	
	כך נראה הפתרון עם לופי: 
	\begin{proof}[הוכחה באמצעות לופי]
		נגדיר $g(h) = f(x_0 + h) + f(x_0  - h) - 2f(x_0)$ וכן $t(h) = h^{2}$. ידוע ש־$f$ גזירה ולכן רציפה ב־$x_0$, כלומר $\limh g(h) = 0$ וכמו כן $\limh t(h) = 0$. ידוע ש־$g, t$ גזירות בסביבת $0$, כיוון ש־$g$ גזירה פעמיים ב־$x_0$. $t'$ לא מתאפסת למעט ב־$0$, כי $t'(h) = 2h, g'(h) = f'(x_0 + h) - f'(x_0 -h)$. עכשיו נשתמש בכלל לופיטל:
		\begin{gather*}
			\limh \frac{g'(h)}{t'(h)} = \limh \frac{f'(x_0 + h) - f'(x_0 - h)}{2h} = \frac{1}{2}\cl{\limh \frac{f'(x_0 + h) - f'(x_0)}{h} + \frac{f'(x_0) - f'(x_0 - h)}{h}}
		\end{gather*}
		נחשב את הגבולות בנפרד:
		\begin{align*}
			\limh \frac{f'(x_0 + h) - f'(x_0)}{h} = f''(x_0) && \limh \frac{f'(x_0) - f'(x_0 - h)}{h} \overset{\ag = -h}{=} \lim_{\ag \to 0}\frac{f'(x_0) - f'(x_0 + \ag)}{-\ag} = f''(x_0)
		\end{align*}
		ואפילו כיצד לציין שהחלפת המשתנה נובעת ממשפט על גבולות והרכבה ולציין את תנאיו. 
	\end{proof}
	שימו לב – לעשות כאן לופיטל פעמיים זה פטאלי! זה לא נכון ולא נתון שהפונקציה מקיימת את התנאים של כלל לופיטל. 
	
	\exe{תהא $f\co \R \to \R$. נניח $f$ גזירה, $f'$ רציפה במ''ש ואינה חסומה. הראו כי $f$ אינה רציפה במ''ש. }\begin{proof}
		נניח בה''כ ש־$f'$ חסומה מלעיל (אחרת נעבוד עם $-f$). 
		
		אין לנו שום דרך כמעט להראות שפונקציה איננה רציפה במ''ש, אלא לפי הגדרה. לכן נתבונן ב־$\eg = 1$ כלשהו (מקסימום נתקן אותו אחכ), ותהא $\dg > 0$. $f'$ רציפה במ''ש ולכן קיים $\dg_1$ כך ש־$\forall x, y \in \R \co \sof{x - y} < \dg_1 \implies \sof{f'(x)- f'(y)} < \dg$. נסמן $r = \frac{1}{3}\min\{\dg_1, \dg, \frac{1}{\dg}\}$. קיים $t \in \R$ כך ש־$f'(t) > \frac{1}{r}$. לכל $x \in [t - r, t + r]$, מתקיים ש־$\sof{f'(x) - f'(t)} < \dg$. מלגראנג', קיים $c \in (t - r, t + r)$ כך ש־: 
		\[ \frac{f(t + r) - f(t - r)}{2r} = f'(c) > \frac{1}{r} - \dg \implies \sof{f(t + r) - f(t - r)} > 2 - 2 r \dg \ge 1 \]
		כמו כן $\sof{t + r - (t - r)}  = 2r < \dg$. סה''כ סתירה. 
	\end{proof}
	
	''גבול אחרון והביתה``
	
	\exe{נחשב את הגבול הבא: 
	\[ \climi x\cl{\cl{1 + \frac{1}{x}}^x - e} \]
	אפשר להתייחס ל־$x$ כאל $\frac{1}{1/x}$. ואז נעשה לופיטל ונקבל במונה: 
	\begin{gather*}
		\cl{1 + \frac{1}{x}}^{x} = e^{x \ln\cl{1 + \frac{1}{x}}} = e^{x\ln(1 + \frac{1}{x})}\cl{\ln\cl{1 + \frac{1}{x}} + x \cdot \frac{1}{1 + \frac{1}{x}} \cdot \cl{-\frac{1}{x^{2}}}} = 
	x^{2}\cl{1 + \frac{1}{x}}^{x}\cl{\ln\cl{1 + \frac{1}{x}} - \frac{1}{1 + x}}
	\end{gather*}
	וזה ממש לא עוזר. זה יעזור מתישהו, אבל זה יהיה כואב. ונצטרך הרבה לופי. 
	
	נבצע החלפת משתנים, כי כרגע הגבול הוא לאינסוף וזה לא מאפשר (בכלים הנוכחיים שלנו) להשתמש בטיילור. אי אפשר לעשות טיילור סביב אינסוף, אפשר להשתמש במשפט טיילור עבור פולינום טיילור סביב נקודה כלשהי. 
	
	לכן נבצע החלפת משתנים – $t = \frac{1}{x}$. ואז נקבל: 
	\[ = \lim_{t \to 0} \frac{1}{t}\cl{\cl{1 + t}^{\frac{1}{t}} - e} \]
	זו לא פונקציה שמוגדרת בכלל באפס. נגדיר למען הנוחות: 
	\[ f(t) = \begin{cases}
		(1 + t)^{\frac{1}{t}} & t \neq 0 \\
		e & \other
	\end{cases} \]
	נבחין שהיא רציפה. ואז נקבל את הגבול: 
	\[ \cdots = \lim_{t \to 0} \frac{f(t) - e}{t} \]
	זו ליטרלי הגדרת הנגזרת של $f$! אך הנגזרת של $f$ אינה מוגדרת ב־$0$. לכן טיילור (סביב $0$) לא עובד, באופן ישיר, כי טור הטיילור צריך ממש להחזיק את הנגזרות ביד כדי שנוכל לפתח אותו. 
	
	אבל, $f(t)$ רציפה וגזירה בכל $t \neq 0$. שם נקבל: 
	\[ f'(t) = \cl{e^{\frac{1}{t}\ln(1 + t)}}' = e^{\frac{1}{t}\ln(1 + t)}\cl{-\frac{1}{t^{2}}\ln(1 + t) + \frac{1}{t} \cdot \frac{1}{1 + t}} = \cl{1 + t}^{\frac{1}{t}}\cl{\frac{-(t + 1)\ln(1 + t) + t}{t^{2}(t + 1)}} = \xi \]
	ה־$t + 1$ בחילוק למטה באפס לא מפריע לנו, וגם לא ה־$(1 + t)^{\frac{1}{t}}$ ששואף ל־$e$ ואם כל השאר יעבוד אז נוכל פשוט להשתמש באריתמטיקה. נוציא אותם החוצה: 
	\[ \xi = (1 + t)^{\frac{1}{t}} \cdot \frac{1}{t + 1}\cl{\frac{t - t \ln (t + 1) - \ln (t + 1)}{t^{2}}} = (1 + t)^{\frac{1}{t}} \cdot \frac{1}{t + 1}\Bigg(\underbrace{\frac{t - \ln (1 + t)}{t^{2}}}_{\mathclap{לופיטל בצד}} - \underbrace{\frac{\ln(1 + t)}{t}}_{1}\Bigg ) = \xi \]
	
	הלופיטל בצד הפך להיות טיילור בצד: 
	\[ \frac{t - \ln(1 + t)}{t^{2}} = \frac{t - \frac{t^{2}}{2} + \wg(t)t^{2}}{t^{2}} = -\frac{1}{2} \]
	סה''כ: 
	\[ \lim_{t \to 0} f'(t) = \xi = -\frac{e}{2} \]
	זו הפואנטה, עכשיו צריך לתפור הכל. כמו שאמרנו הגבול הזה צריך לצאת הגבול המקורי כי הוא ליטרלי נגזרת לפי הגדרה. 
	ניזכר בטענה שראינו: תהא $f \co I \to \R$ גזירה. תהא $x_0$ בקטע ונניח שקיים וסופי הגבול $\limxz f'(x)$, אז הגבול שווה ל־$f'(x_0)$. או במילים אחרות, אם קיים גבול לנגזרת, היא רציפה (כי מדרבו אין נקודות אי־רציפות סליקות ואין נקודות אי־רציפות מסוג ראשון. כל האי רציפות \textit{רע}. זה נכון לגבי כל פונקציה שמקיימת את תכונת חרבו דרבו). 
	
	הנגזרת לא מוגדרת ב־$0$, ועשינו גבול של הנגזרת. אך הנגזרת לא מוגדרת ב־$t = 0$. מהמשפט, נבין שאם הגבול המקורי אכן קיים, אז הוא אכן שווה ל־$-\frac{e}{2}$. אז עכשיו רק נותר להראות שהגבול המקורי שווה ל־$-\frac{e}{2}$. ואפשר לדעת שהגבול קיים! היא גזירה בסביבה נקובה של $0$, כנ''ל לגבי $t$, ולכן אפשר לעשות לופיטל. למעשה לא היינו צריכים לעשות את כל הבלגן עם הנגזרת כי אחרי החלפת משתנים זה פותר. המרצה סתם רצה לדבר על המשפט לעיל. 
	}
	
	ההבדל בין לופיטל בטיילור – הגזירות בנקודה עצמה, בטיילור צריך רק אותו, בלופיטל לא צריך אותו בכלל. 
	
	\ndoc
	
	
\end{document}