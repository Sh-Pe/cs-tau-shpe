\documentclass[]{../../../../tex/classes/styledArticle}
\usepackage{../../../../tex/packages/hebrewSupport}
\usepackage{../../../../tex/packages/mathShortcuts}
\usepackage{../../../../tex/packages/theoremsSupport}

\newcommand\dd  {\mathrm d}
\newcommand\df  {\dd f}
\newcommand\tfl {\lim_{\mathclap{x \to x_0}}}

\author{שחר פרץ}
\title{\textit{חדו''א 1א 10}}
\begin{document}
	\maketitle
	נדבר על... דברים אני מניח. יאי 2026. שנה חדשה. יאי. 
	
	שנה שעברה עסקנו בתכונות גלובליות של פונקציות רציפות. עתה נתחיל לדבר על הנושא המכעט אחרון, גזירות. 
	
	\section*{גזירות}
	''למי אתה מאמין? לניוטון או לייבניץ?``
	
	''אני לא זוכר איך קוראים לך, כי אתה אף פעם לא מדבר איתי אלא רק עם האנשים הקרובים אליך``
	
	אז מכאן התחיל החדו''א. האינטואציה הגיאומטרית הוא מציאת ה־slope של המשיק בנקודה מסוימת. 
	
	\defi{בהינתן $f \co I \to \R$, וכן $x_0 \in I$ בפנים הקטע (איננה נקודת קצה). נאמר ש־$f$ \textit{גזירה ב־$x_0$} כאשר קיים וסופי הגבול $\limxz \frac{f(x) - f(x_0)}{x - x_0}$. }
	\rmark{גזירות היא תכונה נקודית, לוקאלית. }
	\noti{בהנחה שהגבול ב־$x_0$ של הפונקציה $f$ קיים, נסמן $\frac{\dd f}{\dx}(x_0)$ או $f'(x_0)$. }
	\theo{$f$ גזירה ב־$x_0$ אמ''מ קיים וסופי: 
	\[ \lim_{h \to 0}\frac{f(x_0 + h) - f(x_0)}{h} \]}
	''עוד הגדרה שקשורה לזה שבמממד אחד היא לא ממש makes sense``
	\defi{תהי $f \co I \to \R$ וכן $x_0 \in I$ בפנים הקטע. $f$ תקרא \textit{דיפרנציאבילית} ב־$x_0$ כאשר קיימת העתקה לינארית $T \co \R\to \R$ המקיימת שהגבול $\limxz \frac{f(x) - f(x_0) - T(x - x_0)}{x - x_0} = 0$. }
	במממד אחד זה לא ממש מעניין. מה זה אומר? נתחיל מהגבול של $\lim_{x \to x_0} f(x) - g(x) = 0$, שאומר שבגבול הן הולכות לאותו המקום. זה אומר שאפשר לעשות ''משפט השוואה``, אפשר להציב באחת ולקבל קירוב של השנייה. גם בהגדרה של דיפרנציאביליות יש לנו שתי פונקציות. מה המשמעות של כך ש־: 
	\[ \lim_{ x\to x_0} \frac{g(x) - f(x_)}{x - x_0} \]
	אומרת? זה אומר שלא רק ש־$g$ קירוב טוב ש־$f$, אלא גם שכאשר מחלקים ב־$x - x_0$ ששואף ל־$0$ הקירוב נשאר טוב. הקירוב הזה הולך לאפס יותר מהר מזה ש־$x - x_0$ הולך לאפס. ההגדרה של דיפרנביאביליות אומרת שאפשר לקרב את $f$ בנקודה ע''י פונקציה לינארית, והקירוב הזה יותר מהיר מ־$x - x_0$. 
	
	במשתנה אחד, $f$ גזירה ב־$x_0$ אמ''מ $f$ דיפרנציאבילית. ההעתקה הלינארית $T$ הזו נקראת \textit{הדיפרנציאל} של $f$ ב־$x_0$. 
	
	\theo{תהי $f \co I \to \R$ ותהי $x_0 \in I$ בפנים הקטע. אם $f$ גזירה ב־$x_0$ אז $f$ רציפה ב־$x_0$. }
	\begin{proof}
		נניח ש־$f$ גזירה ב־$x_0$ ונגדיר $h \co I \to \R$ ע''י: 
		\[ h(x) = \begin{cases}
			\frac{f(x) - f(x_0)}{x - x_0} & x \neq x_0 \\
			f'(x_0) & x = x_0
		\end{cases} \]
		לכל $x \in I$. נבחין ש־$f$ גיזרה ב־$x_0$ ולכן: 
		\[ \lim_{\mathclap{x \to x_0}} h(x) = \lim_{\mathclap{x \to x_0}} \frac{f(x) - f(x_0)}{x - x_0} = f'(x_0) = h(x) \]
		לכן $h$ רציפה ב־$x_0$. מאריתמטיקת גבולות נקבל: 
		\[ \lim_{\mathclap{x \to x_0}} f(x) - f(x_0) = \lim_{\mathclap{x \to x_0}} h(x) (x - x_0) \]
		ומכאן ש־$f$ רציפה ב־$f_0$. 
	\end{proof}
	
	\textbf{דוגמאות. }
	\begin{itemize}
		\item נתבונן בפונקציה $f \co \R \to \R$ המוגדרת ע''י $f(x) = x^{n}$. נבחין שלכל $x_0 \in \R$ מתקיים ש־$f$ גיזרה בו ומתקיים $f'(x_0) = nx_0^{n - 1}$. 
		\begin{proof}
			לול ראיתי את ההוכחה הזו במכינה של אודיסאה בכיתה ח'. יהי $x_0 \in \R$. ניעזר בבינום של ניוטון, אריתמטיקה ורציפות פולינומים. 
			\[ \lim_{\mathclap{x \to x_0}} \frac{x^{n} - x_0^{n}}{x - x_0} = \lim_{\mathclap{x \to x_0}} \sum_{j = 0}^{n - 1}x^{j}x_0^{n - 1 - j} = \sum_{j = 0}^{n - 1}\lim_{\mathclap{x \to x_0}}x^{j}x_0^{n - 1 - j} = \sum_{j = 0}^{n - 1}x^{n - 1} = nx_0^{n - 1} \]
		\end{proof}
		\item נבחין ש־M$\sin$ גזירה בכל $\R$ ונגזרתה $\cos$. \begin{proof}
			יהי $x_0$. לכל $x \neq x_0$: 
			\[ \tfl \frac{\sinx - \sin x_0}{x - x_0} = \tfl \frac{2\sin\cl{\frac{x - x_0}{2}}\cos\cl{\frac{x + x_0}{2}}}{x - x_0} = \tfl \frac{\sin\cl{\frac{x - x_0}{2}}}{\frac{x - x_0}{2}} \cdot \cos{\frac{x + x_0}{2}} \]
			מהרציפות של $\cos$ ב־$x_0$ ומהגבולות וההרכבה $\lim_{x \to x_0} \cos\cl{\frac{x + x_0}{2}} = \cos x_0$ ומהרציפות של $\frac{\sinx}{x}$ ב־$0$ ומגבולות וההרכבה $\lim_{x \to x_0}\frac{\sin\cl{x - x_0}}{\frac{x - x_0}{2}} = 1$ ומאריתמטייקה גבולות, נקבל: 
			\[ \tfl \frac{\sinx - \sin x_0}{x - x_0} = \cos x_0 \]
		\end{proof}
		
		\item \textbf{דוגמה 3. }יהי $x_0 \in \R$. נמצא את הנגזרת של $e^{x}$ ב־$x_0$. 
		\begin{proof}
			האמת את ההוכחה הזו ראיתי בכיתה ט' במתמטיקה ב'. יש לי אותה מוקלדת עם יותר פירוט בסיכום של $e$ במתמטיקה B. יהי $x_0 \in \R$. נבחין ש־: 
			\[ \frac{e^{x} - e^{x_0}}{x - x_0} = e^{x_0} \cdot \frac{e^{x - x_0} - 1}{x - x_0} \]
			ובפרט: 
			\[ \lim_{h \to 0} \frac{e^{h} - 1}{h} = 1 \]
			(כי ראינו את שיעור שעבר) ומגבולות והרכבה $\lim_{x \to x_0} \frac{e^{x - x_0} - 1}{x - x_0} = 1$. מאריתמטיקה סיימנו. 
		\end{proof}
		\item נגדיר $f \co \R\to \R$ ע''י $f(x) = xD(x)$. נוכיח ש־$f$ אינה גזירה באף נקודה ב־$\R$. \begin{proof}
			לכל $x_0 \neq 0$, הראינו ש־$f$ אינה רציפה ב־$x_0$. לכן היא אינה גזירה ב־$x_0$. נטפל עתה ב־$0$ (נראה שהיא אומנם רציפה אך לא גזירה בו). נתבונן ב־$\eg_0 = \frac{1}{2}$ ויהי $\dg > 0$. בקטע $(-\dg, \dg) \setminus \{0\}$  יש רציונלי $x$ ורציונלי $y$ כך ש־: 
			\[ \sof{\frac{f(x) - f(0)}{x - 0} - \frac{f(y) - f(0)}{y - 0}} = \sof{D(x) - D(y)} = 1 \ge \eg_0 \]
			מקריטריון קושי לא קיים $\lim_{x \to 0} \frac{f(x) - f(0)}{x - 0}$. 
		\end{proof}
		\item עבור $x^{2}D(x)$, היא אומנם עדיין לא רציפה ב־$x_0 \neq 0$, אבל ב־$0$ יקרו דברים קצת פחות מנוונים:
		\begin{proof}
			הראינו ש־$\limxz xD(x) = 0$ והראינו ש־$\limxz \frac{f(x) - f(0)}{x- 0} = \limxz \frac{x^{2}D(x)}{x} = \lim_{x \to x_0} xD(x) = 0$. 
		\end{proof}
	\end{itemize}
	
	\subsection*{נגזרות חד־צדדיות}
	\defi{תהא $f \co  I \to \R$ ותהא $x_0 \in I$ המקיימת $\exists \dg > 0 \co (x_0 - \dg, x_0) \subseteq I$. אז נאמר שנאמר ש־$f$ \textit{גזירה משמאל ב־$x_0$} כאשר קיים וסופי הגבול $\lim_{x \to x_0^{-}} \frac{f(x) - f(x_0)}{x - x_0}$. }
	\defi{\textit{נגזרת מימין} מוגדרת באופן דומה}
	\noti{נסמן את הגזירה משמאל ב־$f_-'(x_0)$ ומימין $f_+'(x_0)$. }
	
	
	\subsubsection*{אריתמטיקה של גזירות}
	\theo{יהיו $f, g \co I \to \R$ ותהי $x_0 \in I$ בפנים הקטע. נניח ש־$f, g$ גזירות ב־$x_0$. אז: 
	\begin{itemize}
		\item לכל $\ag, \bg \in \R$ מתקיי ם$\ag f + \bg g$ גזירה ב־$x_0$ וכן $(\ag f + \bg g)' = \ag f'(x_0) + \bg g'(x_0)$ (הנגזרת לינארית)
		\item מתקיים ש־$fg$ גזירה ב־$x_0$ ומתקיים ש־$(fg)'(x_0) = f'(x_0)g(x) + f(x_0)g'(x_0)$. 
		\item אם $g(x_0) \neq 0$ אז $\frac{f}{g}$ גזירה ב־$x_0$ ומתקיים: 
		\[ \cl{\frac{f}{g}}'\!\!(x_0) = \frac{f'(x_0)g(x_0) - f(x_0)g'(x_0)}{(g(x_0))^{2}} \]
	\end{itemize}}
	נוכיח את (2) ואת השאר לבית. 
	\begin{proof}
		לכל $x \neq x_0$ מתקיים ש־: 
		\begin{multline*}
			\tfl \frac{f(x)g(x) - f(x_0)g(x_0)}{x - x_0} = \tfl \frac{f(x) g(x) - f(x)g(x_0) + f(x)g(x_0) - f(x_0)g(x)}{x - x_0} \\
			= \tfl f(x) \frac{g(x) - g(x_0)}{x - x_0} + g(x_0) \frac{f(x) - f(x_0)}{x - x_0}
		\end{multline*}
		ידוע ש־$f$ גזירה ב־$x_0$ ולכן רציפה ב־$x_0$. לכן $\lim_{x \to x_0} f(x) = f(x_0)$. $g$ גזירה ב־$x_0$ ולכן $\lim_{x \to x_0} \frac{g(x) - g(x_0)}{x - x_0} = g'(x_0)$. מאריתמטיקה קיבלנו: 
		\[ \tfl f(x) \frac{g(x) - g(x_0)}{x - x_0} f(x_0)g(x_0) \quad\quad \tfl g(x_0) \frac{f(x) - f(x_0)}{x - x_0} = g(x_0)f'(x_0) \]
		מאריתמטיקה סיימנו: 
		\[ \tfl \frac{f(x)g(x) - f(x)g(x)}{x - x)} = f(x_0)g'(x_0) + g'(x_0)g(x_0) \]
		
	\end{proof}
	''זהו אתה פורש?``
	
	\subsection*{כלל השרשרת}
	\theo{תהא $f \co I \to J$ ותהא $g \co J \to \R$. נניח $x_0 \in I$ בפנים הקטע. נניח ש־$f$ גזירה ב־$x_0$ וגם $g$ גזירה ב־$x_0$. אז $g \circ f$ גיזרה ב־$x_0$ וכן $(g \circ f)'(x_0) = g'(f(x_0))f'(x_0)$.}
	\begin{proof}[הוכחה שגויה, ידועה בכינוייה הוכחהחהחה]
		\[ \tfl \frac{(g \circ f)(x) - (g \circ f)(x_0)}{x - x_0} = \tfl \frac{g(f(x)) - g(f(x_0))}{f(x) - f(x_0)} \cdot \frac{f(x) - f(x_0)}{x - x_0} = g'(f(x_0)) \cdot f'(x_0) \]
		מה הבעיה בהוכחהחהחה? שלא מובטח ש־$f(x) - f(x_0) \neq 0$. מותר להניח $x \neq x_0$ (כי אנחנו בגבול), אבל הטענה השנייה לא עובדת. לדוגמה עבור פונקציה קבועה ההוכחהחהחה לא עובדת. יש כאן עוד בעיה. במשפט של הרכבה, דרשנו שהגבול של הפונקציה הפנימית מקיימת כל מני דברים. לכן נצטרך לעשות חלוקה למקרים. 
	\end{proof}
	\rmark{שיגאות מעין אילו הרבה פעמים חומקות מתחת לרדאר. אבל גם הוכחה שגויה אפשר לתקן – אם נפצל למספיק מקרים נוכל לטפל בבעיה. }
	\begin{proof}[הוכחה נכונה]
		\begin{itemize}
			\item במקרה הראשון, $f'(x_0) \neq 0$. ואז הגבול $\lim_{x \to x_0} \frac{f(x) - f(x_0)}{x - x_0} \neq 0$ ולכן קיים $\dg > 0$ כך שלכל $x \in (x_0 - \dg, x_0 + \dg)$ ומכאן $\frac{f(x) -f(x_0)}{x - x_0} \neq 0$  ובפרט $f(x) \neq f(x_0)$. בקטע הזה אפשר לבצע את ההוכחהחהחה – מתקיימים תנאי המשפט על גבולות והרבה, ולכן הגבול: 
			\[ \tfl \frac{g(f(x)) - g(f(x_0))}{f(x) - f(x_0)} = \lim_{\mathclap{y \to x_0}} \frac{g(y) - g(f(x_0))}{y - f(x_0)} = g'(f(x_0)) \]
			$f$ גזירה ב־$f_0$ ולכן $\lim_{x \to x_0} \frac{f(x) - f(x_0)}{x - x_0} = f'(x_0)$. מאריתמטיקה קיבלנו $\lim_{x \to x_0} \frac{(g \circ f)(x) - (g \circ f)(x_0)}{x - x_0} = g'(f(x_0))f'(x_0)$. 
			\item אם $f'(x) = 0$. במקרה זה, לביטוי: 
			\[ \lim_{y \to f(x_0)} \frac{g(y) -g(f(x_0))}{y - f(x_0)} \]
			קיים וסופי (הגדרת הנגזרת). לכן קיים $M > 0$ כך שקיים $\dg > 0$ כך שלכל $y \in (f(x_0) - \dg, f(x_0) + \dg)$, מתקיים ש־$\sof{\frac{g(y) - g(f(x_0))}{y - f(x_0)}} < M$ (ברה יכולת פשוט להגיד שהדבר הזה חסום ע''י $M$ בסביבת $\dg$ נקובה ולגמור עניין). יהי $\eg > 0$. אז קיים $\dg_2 > 0$ כך שללכ $x \in (x_0 - \dg, x_0 + \dg)$, מתקיים $\sof{\frac{f(x) - f(x_0)}{x - x_0}} < \frac{\eg}{M}$ (הגדרת הגבול). $f$ רציפה ב־$x_0$ ולכן קיים $\dg_3 > 0$ כך שלכל $x \in (x_0 - \dg, x_0 + \dg)$ מתקיים $\sof{f(x) - f(x_0)} < \dg$. נסמן ב־$\eta = \min\{\dg_2, \dg_3\}$ ויהי $x \in (x_0 - \eta, x_0 + \eta)$. נחלק למקרים. 
			\begin{itemize}
				\item אם $f(x) = f(x_0)$, אז $\sof{\frac{g(f(x)) - g(f(x_0))}{x - x_0}} = 0 < \eg$ וסיימנו. 
				\item אחרת: 
				\[ \frac{(g \circ f)(x) - (g \circ f)(x_0)}{\sof{x - x_0}} = \sof{\frac{(g \circ f)(x) - (g \circ f)(x_0)}{f(x) - f(x_0)}}\sof{\frac{f(x) - f(x_0)}{x - x_0}} < M \cdot \frac{\eg}{M} = \eg \]
			\end{itemize}
			סה''כ: 
			\[ \tfl \frac{(g \circ f)(x) - (g \circ f)(x_0)}{x - x_0} = 0 = g'(f(x_0))f'(x) \]\envendproof
		\end{itemize}
	\end{proof}
	
	\subsection*{מסקנות נוספות}
	\theo{תהא $f \co I \to J$ פונקציה חח''ע ועל, כאשר $I, J$ קטעים פתוחים (אך לא בהכרח, סתם למרצה לא בא להתעסק עם הקצוות). אז $f\op$ גזירה בכל נקודה ב־$J$ ומתקיים $\forall y \in J \co (f\op(y))(y) = \frac{1}{f'(f\op(y))}$. }\begin{proof}[הוכחהחהחה]
		ידוע שלכל $x \in J$ מתקיים $(f \circ f\op)(x) = x$. מכלל השרשרת נקבל $f'(f\op(x))(f\op)'(x) = 1$. נחלק ונקבל את הדרוש. 
		
		מה הבעיה בהוכחהחהחה? כלל השרשרת דרש שהפונקציה גזירה בנקודה. לא הראינו את זה. 
	\end{proof}
	
	\begin{proof}[הוכחה נכונה]
		יהי $y \in J$. נניח $f\op(f'(y_0)) \neq 0$. מסמן $x_0 = f\op(f'(y_0))$. ידוע $f'(x_0) \neq 0$, לכן קיימת סביבה מנוקבת $U$ של $x_0$ כך שבה לכל $x \in U$ מתקיים $\frac{f(x) - f(x_0)}{x - x_0} \neq 0$. נגדיר $g \co U \to \R$. לכל $x \in U$ מתקיים: 
		\[ g(x) = \begin{cases}
			\frac{1}{\frac{f(x) - f(x_0)}{x - x_0}} & x \neq x_0 \\
			\frac{1}{f\op(x_0)} & x = x_0
		\end{cases} \]
		ניתן להבחין ש־$g$ רציפה, ובפרט רציפה ב־$x_0$. $f\op$ רציפה ב־$y_0$. לכן $g \circ f\op$ רציפה ב־$y_0$. כלומר: 
		\[ \lim_{y \to y_0} (g \circ f\op)(y) = (g \circ f\op)(y_0) = g(x_0) = \frac{1}{f'(x_0)} = \frac{1}{f'(f\op(y_0))} \]
		מצד שני, 
		\[ \lim_{y \to y_0} (g \circ f\op)(y) = \lim_{ y \to y_0} \frac{f'(y) - f\op(y_0)}{y - y_0} \]
		לכן $f\op$ גזירה ב־$y_0$ ומתקיים
		\[ (f\op)'(y_0) = \frac{1}{f'(f\op(y_0))} \]
		וסיימנו. 
	\end{proof}
	\textbf{דוגמה. }יהי $n \in \N^{+}$, ונגדיר $f(x) = \sqrt[n]{x}$ לכל $x \in (0, \inft)$. נשים לב ש־$\mathrm{Range} f = (0, \infty)$. נגדיר $g \co (0, \inft) \to (0, \inft)$ על ידי $g(x) = x^{n}$ לכל $x \in (0, \inft)$. אז $f = g\op$. לכן $f$ גזירה בכל נקודה $(0, \inft)$ ומתקיים לכל $y \in (0, \inft)$ ש־$f'(y) = \frac{1}{g'(f(y))}$. סה''כ: 
	\[ f'(y) = \frac{1}{g'(f(y))} = \frac{1}{n \cdot (f(y))^{n - 1}} = \frac{1}{n}y^{\frac{1 - n}{n}} = \frac{1}{n}y^{\frac{1}{n} - 1} \]
	
	\textbf{דוגמה. }יהיו $m, n \in \N^{+}$. נגדיר $f \co (0, \inft) \to (0, \inft)$ ע''י $f(x) = x^{\frac{m}{n}}$. אז אפשר לנסח $f(x) = \cl{x^{\frac{1}{n}}}^{m}$ ומכלל השרשרת נקבל ש־$f(x) = m\cl{x^{\frac{1}{n}}}^{m - 1} \cdot \frac{1}{n}x^{\frac{1}{n} - 1} = \frac{m}{n}x^{\frac{m}{n} - \frac{1}{n} + \frac{1}{n} + 1} = \frac{m}{n}x^{\frac{m}{n} - 1}$. 
	סה''כ באופן כללי $(x^{q})' = qx^{q - 1}$ לכל $q \in \Q_+$. 
	
	לבית, להוכיח ל־$\Q_-$ וכן ל־$\R$, ש־$(x^{r})' =  rx^{r - 1}$ לכל $r \in \R$. 
	
	\textbf{דוגמה. }יהי $a > 0$ לכל $x \in \R$. נקבל $f(x) = e^{x\ln a}$ ומכלל השרשרת קיבלנו (הבהרה, לא גזרנו $\ln$, גזרנו קבוע) $f'(x) = e^{x\ln a} \ln a = a^{x}\ln a$. 
	
	\textbf{דוגמה. }נגדיר $f(x) = \ln x$ לכל $x \in (0, \inft)$. נסמן $g(x) = e^{x}$ לכל $x \in \R$. נבחין ש־$g\op = f$ וכן $g'$ אינה מתאפסת באף נקודה. מכאן ש־: 
	\[ f'(x) - \frac{1}{g'(f(x))} = \frac{1}{e^{f(n)}} = \frac{1}{e^{\ln x}} = \frac{1}{x} \]
	
	\textbf{דוגמה. }נגדיר $f(x) = \arctan x$ לכל $x \in \R$. נבחין $g(x) = \tanx$ לכל $x \in (-\frac{\pi}{2}, \frac{\pi}{2})$. אז $g\op = f$ ולכל $x \in \cl{-\frac{\pi}{2}, \frac{\pi}{2}}$ מתקיים $g'(x) = \frac{1}{\cos^{2}x}\neq 0$. לכן $f$ גזירה בכל $\R$ ומתקיים: 
	\[ f'(x) = \frac{1}{g'(f(x))} = \frac{1}{\frac{1}{\cos^{2}\arctan(x)}} = \frac{1}{1 + x^{2}} \]
	
	\subsection*{תכונות גלובליות של פונקציות גזירות}
	
	\begin{Theorem}[המשפט הלא אחרון של פרמה]
		תהא $f \co I \to \R$ ותהי $x_0 \in I$ בפנים הקטע. נניח $f$ גזירה ב־$x_0$ ונניח של־$f$ יש קיצון מקומי ב־$x_0$. אז $f'(x_0) = 0$. 
	\end{Theorem}
	\rmark{המשפט הזה הוא חד־כיווני. לא כל נקודה סטרציונרית היא נקודת קיצון. }
	\defi{ל־$f$ יש מקסימום מקומי ב־$x_0$ כאשר קיים $\dg > 0$ כך שלכל $x \in (x_0 - \dg, x_0 + \dg)$ מתקיים $f(x) \le f(x_0)$. }
	\defi{מינימום מקומי בדומה. }
	\begin{proof}[הוכחה למשפט פרמה הלא אחרון]
		נניח $x_0$ מקסימום מקומי (ההוכחה עבור מינימום בדומה). $x_0$ פנימית בקטע ולכן מוגדרות (הנחת גזירות) ושוות הנגזרות החד־צדדיות בנקודה. קיים $\dg > 0$ כך שלכל $x \in (x_0 - \dg, x_0 + \dg)$ מתקיים $f(x) \le f(x_0)$. לכן לכל $x \in (x - \dg, x + \dg)$ נקבל: 
		\[ \frac{f(x) - f(x_0)}{x - x_0} \ge 0 \]
		(משפט לפיו גבול משמר א''ש חלש). לכן $f'(x_0) = \lim_{x \to x_0^{-}} \frac{f(x) - f(x_0)}{x - x_0} \ge 0$. באופן דומה מימין נקבל $f'(x_0)$. לכן $f'(x_0)$. 
	\end{proof}
	
	\subsubsection*{המשפט היסודיים של החשבון הדיפרנציאלי (להבדיל מהמשפט היסודי של החד''א)}
	\begin{Theorem}[משפט רול]
		תהא $f \co [a, b] \to \R$. נניח ש־$f$ רציפה בקטע ב־$[a, b]$ וכן גזירה ב־$(a, b)$, $f(a) = f(b)$. אז קיימת $c \in (a, b)$ שבה $f'(c) = 0$. 
	\end{Theorem}
	\begin{proof}
		\begin{enumerate}
			\item מקרה ראשון: נניח $f$ קבועה ב־$[a, b]$. נתבונן ב־$c = \frac{a + b}{2}$ ובסביבת  $c$ היא קבועה כלומר $f'(c) =0$ וסיימנו. 
			\item במקרה השני, קיים $x \in (a, b)$ כך שלכל $f(x) \ne f(a)$, בה''כ $f(x) > f(a)$, $f$ רציפה ב־$[a, b]$ ולכן לפי וויראשטראס יש לה מקסימום בקטע, כלומר קיימת $c \in [a, b]$ כך שלכל $x \in (a, b)$ מתקיים $f(x) \le f(c)$. ידוע $f(c) \ge f(x_0) > f(a)$ לכן $c \neq a$ וגם $c \in (a, b)$ ולכן לפי פרמה מכיוון ש־$f$ גזירה ב־$c$ נובע ש־$'f(c) = 0$. 
		\end{enumerate}
	\end{proof}
	
	\begin{Theorem}[משפט ערך הביניים של לגראנג']
		תהי $f \co [a, b] \to \R$ ונניח $f$ רציפה ב־$[a, b]$ וכן גזירה ב־$(a, b)$. אז קיימת $c \in (a, b)$ כך ש־$f'(c) = \frac{f(b) - f(a)}{b - a}$. 
	\end{Theorem}
	\rmark{מקרה פרטי של רול, עבור $a = b$. }
	\begin{proof}
		נגדיר: (נחסר את המיתר כדי להשתמש ברול)
		\[ h(x) = f(x) - f(a) - \frac{f(b) - f(a)}{b - a}(x - a) \]
		מאריתמטיקה $h$ רציפה  ב־$[a, b]$ וכן גזירה ב־$(a, b)$. כמו כן $h(a) = h(b) = 0$. לכן $h$ מקיימת את תנאי משפט רול ב־$[a, b]$ כלומר קיימת $c \in (a, b)$ כך ש־$h(c) = 0$, ומתקיים: 
		\[ 0 = h'(c) = f'(c) - \frac{f(b) - f(a)}{b - a} \implies f'(c) = \frac{f(b) - f(a)}{b - a} \]
		כנדרש. 
	\end{proof}
	
	
	\theo{תהא $f \co I \to \R$ ונניח כי $f$ גזירה בכל $I$ וכי לכל $x \in I$ מתקבל $f'(x) = 0$. הראו כי $f$ קבועה. }\begin{proof}
		יהיו $x, y \in I$ ונניח $x < y$. בקטע $[x, y]$ $f$ רציפה. בקטע $(x, y)$ $f$ גזירה. לכן קיימת $c \in (a, b)$ כך ש־: 
		\[ f'(c) = \frac{f(x) - f(y)}{x - y} \]
		ו־$f'(c) = 0$ כלומר $f(x) = f(y)$. 
	\end{proof}
	\rmark{הוכחה דומה מראה שאם $f'$ שלילית אז $f$ יורדת ואם $f'$ חיובית אז $f$ עולה. }
	
	\theo{תהא $f \co I \to \R$ ונניח כי $f$ גזירה בכל $I$. הראו ש־$f$ עולה ב־$I$ אמ''מ $\forall x \in I \co f'(x) \ge 0$. }
	\begin{proof}
		\begin{itemize}
			\item[$\impliedby$]אותה הפריקינג הוכחה. יהיו $x, y \in I$ ונניח $x < y$. $f$ רציפה ב־$[x, y]$ וכן גיזרה ב־$(x, y)$. מכאן ש־$f$ מקיימת את תנאי משפט לגראנג' וקיימת $c \in (x, y)$ כך ש־: 
			\[ 0 \le f'(x) = \frac{f(x) - f(y)}{x - y} \]
			כלומר $f(x) \le f(y)$ וסיימנו. 
			\item[$\implies$]יהי $x_0 \in I$. קיים $\dg > 0$ כך ש־$(x_0 - \dg, x_0 + \dg) \subseteq I$. מהגדרה $\lim_{x \to x_0} \frac{f(x) - f(x_0)}{x - x_0} = f'(x_0)$. ידוע $f(x) \le f(x_0)$ כי הנחנו שהיא עולה, ואז לכל $x \in (x_0 - \dg, x_0)$ קיבלנו $\frac{f(x) - f(x_0)}{x - x_0} \ge 0$. מכאן $f'(x_0) \ge 0$. ז
		\end{itemize}
	\end{proof}
	
	\ndoc
	
\end{document}