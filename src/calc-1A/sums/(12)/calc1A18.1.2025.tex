\documentclass[]{../../../../tex/classes/styledArticle}
\usepackage{../../../../tex/packages/hebrewSupport}
\usepackage{../../../../tex/packages/mathShortcuts}
\usepackage{../../../../tex/packages/theoremsSupport}


\author{שחר פרץ}
\title{\textit{חדו''א 12}}
\begin{document}
	\maketitle
	\theo{תהאנה $f, g \co A \to \R$. תהא $x_0$ נקודת הצטברות של $A$. נניח כי $\limxz f(x) = \ml \in (0, \infty)$ וכן $\limxz g(n) = m \in \R$. אזי $\limxz f(x)^{g(x)} = \ml^{m}$. }\begin{proof}
		מרציפות $\ln$ וממשפט על רציפות והרכבה, נקבל $\limxz \ln(f(x)) = \ln(\ml)$. מאריתמטיקת גבולות $\limxz g(x) \cdot \ln f(x) = m \ln \ml$. מרציפות $e^{x}$ וממשפט על רציפות והרכבה, נקבל: 
		\[ \climxz f(x)^{g(x)} = \climxz e^{g(x)\ln f(x)} = e^{m \ln l} = \ml^{m} \]
	\end{proof}
	\rmark{השתמשנו חזק בזה ש־$\ml$חיובי, כי $\ln$ רציפה רק בעבור מספרים חיוביים. }
	לא צריך להסביר את זה. אפשר להפעיל את זה במיידית. אם $\ml = 0$ צריך לעבוד יותר קשה. יש צורך גם לדבר בקצרה על זה שהגבול מוגדר, משום ש־$f(x)$ היא bounded away from zero. 
	
	\theo{תהא $f(x) \co I \to \R$ גזירה פעמיים ב־$x_0 \in I$ (בפנים הקטע). נניח כי $f'(x_0) = 0$. אם $f''(x_0) > 0$ אז $x_0$ מינימום. אם $f''(x_0) < 0$ אז $x_0$ מקסימום. }
	ה''בעיה`` בהוכחה של הטענה הזו, היא שאנחנו יודעים ש־$f$ גזירה \textbf{רק ב־$x_0$}. לכן אי אפשר לעשות לגראנג'. זכרו שהגדרנו מינימום מקומי כמצב בו קיימת סביבה שבה כל הנקודות גדולות או שוות לנקודה. \begin{proof}
		ידוע $f$ גזירה פעמיים ב־$x_0$ ולכן קיימת סביבה של $x_0$ שבה $f$ גזירה (אחרת הנגזרת השנייה בנקודה אינה מוגדרת). ללא הגבלת הכליות $f$ גזירה בכל $I$ (למה בה''כ כי אפשר פשוט לצמצם ולהגדיר $\tl f = f|_I$, או להגדיר דלתא ולעשות מינימומים). מהנתון: 
		\[ \lim_{x \to x_0^{-}} \frac{f'(x) - f'(x_0)}{x - x_0} = f''(x_0) > 0 \]
		ולכן קיים $\dg > 0$ כך שלכל $x_0 - \dg  < x < x_0$ מתקיים
		\[ \frac{f'(x) - f'(x_0)}{x - x_0} > \frac{f''(x_0)}{2} > 0 \]
		(כלומר הוא bounded away). לכן לכל $x_0 - \dg < x < x_0$ מתקיים $f'(x) < f'(x_0) = 0$. מנימוקים דומים קיים $\dg_2> 0$ כך שלכל $x_0 < x < x_0 + \dg_2$ מתקיים $\frac{f'(x) - f'(x_0)}{x - x_0} > \frac{f''(x_0)}{2} > 0$. 
		לכן לכל $x_0 < x < x_0 + \dg_2$  מתקיים $f'(x) > f'(x_0) = 0$. נתבונן ב־$\dg = \min(\dg_1, \dg_2)$. יהי $x$. נניח $x$ בסביבת $\dg$ של $x_0$. 
		\begin{itemize}
			\item אם $x = x_0$ אז $f(x) = f(x_0) \ge f(x_0)$ וסיימנו. 
			\item נניח $x_0 - \dg < x < x_0$ אז בקטע $[x, x_0]$, $f$ מקיימת את תנאי משפט לגראנג' ומכאן קיימת $x \in (x, x_0)$ כך ש־$\frac{f(x_0) - f(x)}{x_0 - x} = f'(c) < 0$. לכן, ומכיוון ש־$x_0 > x$, בהכרח $f(x_0) < f(x)$. 
			\item נניח $x_0 < x < x_0 + \dg$. בדומה. 
		\end{itemize}
	\end{proof}
	עיקרי ההוכחה: להראות שמכיוון שיש גבול, מכן אחד מהצדדים כל הנגזרת היא bounded away from zero, ומכיוון שהיא גזירה ורציפה באיזושהי סביבה, אפשר להפעיל לגראנג' ולקבל את מה שצריכים. 
	
	נוכל לתת הוכחה שקצת פחות נוגעת בגבולות ובדלתאות. ''פולינום טיילור, לא טור! לילי חזרה תשובה?``
	
	\begin{proof}[הוכחה נוספת]
		$f$ גזירה פעמיים ב־$x_0$, לכן לפי משפט קיימת $\wg \co I \to \R$ המקיימת כי $\wg$ רציפה ב־$x_0$, $\wg(x_0) = 0$ וגם לכל $x \in I$ מתקיים: 
		\[ f(x) = f(x_0) + f'(x_0)(x - x_0) + f''(x_0)(x - x_0)^{2} + \wg(x)(x - x_0)^{2} \]
		ידוע $\limxz \wg(x) = 0$. מהרציפות של $\wg$ ב־$x_0$ קיים $\dg > 0$ כך שלכל $x_0 - \dg < x < x_0 + \dg$ כך ש־$\wg(x) > -\frac{f''(x_0)}{4}$ (רציפות, כי אנחנו רוצים סביבה שאיננה נקובה). יהי $x$ בסביבת ה־$\dg$ של $x$. אז (מהמשוואה הקודמת): 
		\[ f(x) = f(x_0) + \underbrace{(x - x_0)^{2}}_{\ge 0} \cdot \underbrace{\cl{\frac{f''(x_0)}{2} + \wg(x)}}_{> \frac{f''(x_0)}{4} >  0} \ge f(x_0) \]
		וסיימנו. 
	\end{proof}
	ישנה גרסה מוכללת באינדוקציה לטענה זו. (או לא באינדוקציה אם עושים עם טיילור)
	
	\theo{יהי $n \in \Nodd$. תהא $f \co I \to \R$ גזירה $n + 1$ פעמים ב־$x_0$. נניח $f^{(i)}(x_0) = 0$ וגם $f^{(n + 1)}(x_0) \neq 0$. אז אם $f^{(n + 1)}(x_0) > 0$ אז יש ל־$f$ מינימום ב־$x_0$. אם $f^{(n + 1)(x_0) < 0}$ אז יש ל־$f$ מקסימום ב־$x_0$. באותם התנאים, אם $n \in \Neven$ אז אין קיצון, יש פיתול. }
	
	\exe{חשבו את 
	\[ \limz \frac{x^{2}\sinx}{x^{2} - \sin^{2}x} \]
	או הוכיחו שאינו קיים. }
	לופיטל יעבוד פה מתישהו. אבל good luck בלגזור את המונה. של עומד להתפשט ולגדול כמו סרטן בגוף של סבא שלי. אופציה אחת, ללהפוך את הלופיטל לנחמד, היא לבוא ולהגיד $x^{2}\sin^2x = x^{4} \cdot \frac{\sin^{2}x}{x^{2}}$. החלק הימני שואף ל־$1$, ולשאר אפשר לעשות לופיטל גם 4 פעמים וזה יהיה בסדר (מה איכפת לי לגזור מונום). זה חוקי מהטענה הבאה: נניח של־$f$ אין גבול ב־$x_0$, ונניח שהגבול $\limxz g(x) = \ml \neq 0$, אז $f \cdot g$ חסרת גבול ב־$x_0$. אם היינו מוציאים גבול שהוא איננו $0$, זה לא היה עובד, כלומר אם $\frac{\sin^{2}x}{x^{2}}$ היה שואף למקום אחר. במקום זה, נעשה טיילור. ולהלן הפתרון עם טיילור. 
	
	\begin{proof}[פתרון]
		נפתח את $\sin^{2}x$ לפולינום מסדר $4$ עם שארית פאנו. נגדיר $f(x) = \sin^{2}x$. אז: 
		\[ f(x) = \sin^{2}x \quad f'(x) = 2\sinx\cosx = \sin(2x) \quad f''(x) = 2 \cos(dx) \quad f'''(x) = -4\sin(2x) \quad f^{(4)}(x) = -8\cosx \]
		לכן קיימת $\wg \co \R\to \R$ כך ש־$\wg(0) = 0$, $\wg$ רציפה ב־$0$ וגם לכל $x$ מתקיים: 
		\[ \sin^{2}x = 0 + 0x + \frac{2}{2}x^{2} + 0x^{3} + \frac{-8}{4!}x^{4} + \wg(x)x^{4} = x^{2} - \frac{1}{3}x^{4} + \wg(x)x^{4} \]
		לכן לכל $x$, מתקיים: 
		\[ \frac{x^{2}\sin^{2}x}{x^{2} - \sin^{2}x} = \frac{x^{4} - \frac{1}{3}x^{6} + \wg(x)x^{6}}{\frac{1}{3}x^{4} - \wg(x)x^{4}} = \frac{1 - \frac{1}{3}x^{2} + \wg(x)}{\frac{1}{3} - \wg(x)} \rrr{x \to 0} \frac{1 - 0}{\frac{1}{3} - 0} = 3 \]
	\end{proof}
	למה היינו צריכים פולינום טיילור ממעלה 4? כי אחרת היינו מקבלים: 
	\[ \sin^{2}x = x^{2} + \wg(x)x^{2} \implies \frac{\cdots}{\cancel{x^{2} - x^{2}} + \wg(x)x^{2}} \]
	כלומר המכנה הוא $0$ ואנחנו עדיין בבעיה. 
	
	אגב, הנה הפתרון עם לופיטל. 
	\begin{proof}[לופיתרון]
		\[ \frac{x^{2} - \sin^{2}x}{x^{4}} = \frac{x^{4}\overbrace{\frac{\sin^{2}x}{x^{2}}}^{\to 1}}{x^{2} - \sin^2x} \slh \frac{2x - \sin(2x)}{x^{3}} \slh \frac{2 - 2\cos 2c}{12x^{2}} = \frac{1}{6} \cdot \frac{1 - \cosx 2x}{x^{2}} \]
		בשלב הזה אפשר גם לעשות זהויות טריגו ולעשות את זה סבבה. אם הייתם עושים שוב לופיטל ועושים $\frac{2\sin(2x)}{2x}$ זה כבר מוגזם ו''אני הייתי מוריד נקודות`` (המרצה). ואיך עושים בלי לופי? 
		\[ \frac{1 - \cos2x}{x^{2}} = \frac{2\sin^{2}x}{x^{2}} = 2 \cdot 1 \]
	\end{proof}
	
	\exe{חשבו את הגבול: 
	\[ \limz (1 + \arctan x - x)^{\frac{1}{x^{3}}} \]
	או הוכיחו שאינו קיים. }
	
	\begin{proof}[פתרון]
		לכל $x \neq 0$, מתקיים: 
		\[ (1 + \arctan x - x)^{\frac{1}{x^{3}}} = \cl{\cl{1 + \arctan x - x}^{\frac{1}{\arctan x - x}}}^{\frac{\arctan x - x}{x^{3}}} = e^{\frac{\arctan x - x}{x^{3}}} = \cdots \]
		כי כל הבפנוכו שואף ל־$e$ (ממשפט שהוכחנו בתרגילי הבית + היינה במבטא גרמני). אפשר לעשות פולינום טיילור. אפשר גם לעשות לופיטל. ידוע $\limz \arctan x - x = 0$ וגם $\limz x^{3} = 0$. הן גזירות ורציפות ואנחנו יודעים את הנגזרת ובלה בלה ולכן:
		\[ \frac{\arctan x - x}{x^{3}} \slh \frac{\frac{-x^{2}}{1 + x^{2}}}{3x^{2}} = -\frac{1}{3} \]
		אפשר גם לעשות טיילור ל־$\arctan$. נחזור למעלה: 
		\[ \cdots = e^{-\frac{1}{3}} \]
	\end{proof}
	
	\rmark{אל תציבו חצי גבול. הדבר הזה: 
	\[ \limz (2 + x)^{\frac{1}{x}} = \limz 2^{\frac{1}{x}} \]
	זה כמו הדבר הזה: 
	\[ \frac{\can 64}{1\can 6} = 4 \]
	במקרה של $\frac{1}{x^{2}}$ ולא $\frac{1}{x}$ זה היה עובד, כי זה ב־$0$ שואף ל־$+\inft$, אבל צריך לנמק את כל זה. כי אפשר להגיד: 
	\[ \limz (2 + x)^{\frac{1}{x^{2}}} = \limz e^{\frac{1}{x^{2}}\ln(2 +x)} \]
	וכל הדבר למעלה רציף ונחמד, ואפשר לעבוד איתו. 
	}
	
	\rmark{טור טיילור סביב $0$ קרוי טור מק'לורן. כנ''ל על טורים. }
	עתה נוכיח משפט משיעור שעבר. 
	\theo{תהא $f \co I \to \R$ ותהא $x_0 \in I$ נקודת םפנים. נניח $f$ גזירה $n$ פעמים ב־$x_0$. נסמן ב־$T_n$ את פולינום הטיילור של $f$ מסדר $n$ סביב $x_0$. נסמן ב־$R_n$ את השארית המתאימה. אז: 
	\[ \limxz \frac{R_n(x)}{(x - x_0)^{n}} = 0 \]}
	\begin{proof}
		הבחנה: $T_n'$ הוא פולינום טיילור של $f'$ מסדר $n - 1$ סביב $x_n$, ויתר על כן, $R_n'$ היא השארית המתאימה. הסיבה: 
		\[ \cl{\frac{f^{(k)(x_0)}}{k!}(x - x_k)^{k}}' = \frac{f^{(k)}(x_0)}{(k -1)!}(x -x_0)^{k} = \frac{(f')^{(k -1)(x_0)}}{(k - 1)!}(x -x_0)^{k} \]
		ההוכחה באינדוקציה על $n$. 
		\begin{itemize}
			\item בסיס: עבור $n = 1$ נקבל: 
			\[ \limxz \frac{R_1(x)}{x - x_0} = \limxz \frac{R_1(x) - R_1(x_0)}{x - x_0} = R_1'(x_0) = 0 \]
			\item נניח נכונות בעבור $n$ (\textbf{לא הנחנו ל־$\bm f$ ספציפית}). אז $T_{n + 1}'$ פולינום טיילור מסדר $n$ של $f'$ ו־$R_{n + 1}'$ השארית המתאימה (כי הנגזרת לינארית והכל). לכן: 
			\[ \limxz \frac{R'_{n + 1}(x)}{(x - x_0)^{n}} = 0 \]
			אפשר לעשות לופיטל. אפשר גם לעשות לגראנג'. יהי $\eg > 0$. מכאן שקיים $\dg > 0$ כך שלכל $x \in I$, אם $0< \sof{x - x_0} < \dg$, אז: 
			\[ \sof{\frac{R'_{n + 1}(x)}{(x - x_0)^{n}}} < \eg \]
			יהי $x \in I$. נניח $0 < \sof{x - x_0} < \dg$. בקטע שבין $x$ ל־$x_0$, $R_{n + 1}$ מקיימת את תנאי משפט לגארנג'. לכן קיים $c$ בין $x$ ל־$x_0$ כך ש־: 
			\[ \frac{R_{n + 1}(x) - R_{n + 1}(x_0)}{x - x_0} = R_{n + 1}'(c) \]
			סה''כ (ניעזר בכך ש־$R_{n + 1}(x_0) = 0$ ושהנגזרת בנקודה הזו היא $0$): 
			\[ \sof{\frac{R_{n + 1}(x)}{(x - x_0)^{n + 1}}} = \sof{\frac{\frac{R_{n + 1}(x) - R_{n + 1}(x_0)}{x - x_0}}{(x - x_0)^{n}}} = \sof{\frac{R'_{n + 1}(c)}{(x -x_0)^{n}}} < \sof{\frac{R'_{n + 1}(c)}{(c - x_0)^{n}}} < \eg \]
			ואז כנראה סיימנו. 
		\end{itemize}
	\end{proof}
	''לופיטל גורם לריפיון שכל. הוא גורם לסטונדטים לעשות דברים מטופשים``. ואז המרצה מסביר איך לופיטל זה כמו לחצות את הכביש לא במעבר חציה. 
	
	\subsection*{פולינום טיילור עם שארית לגראנג'}
	עד עכשיו עבדנו עם שארית פאנו. אנחנו רוצים יותר כי אנחנו גרידי. לא מספיק למרצה שהשארית שואפת ל־$0$ יותר מהר מ־$x^{n}$, הוא רוצה יותר מזה. 
	
	\noti{נגדיר את $C^{(n + 1)}(A)$ את קבוצת הפונקציות הגזירות ברציפות ב־$I$. }
	\theo{תהא $f \co I \to \R$ ותהא $x_0 \in \R$ בפנים הקטע. נניח כי $f$ גזירה $n + 1$ פעמים בכל $I$ ונגזרותיה רציפות (כלומר $f \in C^{(n + 1)}$). לכל $x \in I$ קיים $c$  בין $x_0$ ל־$x$ כך ש־: 
	\[ R_{n}(x) = \frac{f^{(n + 1)}(c)}{(n + 1)!}(x - x_0)^{n + 1} \]}
	ההוכחה נעשית ע''י משפט קושי. 
	
	איך נשתמש במשפט הזה? 
	\begin{enumerate}
		\item הערכת הקירוב. 
		
		\textbf{דוגמה: }נחשב את $\sin 1$ עם שגיאה של לכל היותר $\frac{1}{1000}$. נגדיר $f(x) = \sinx$ ונפתח את $f$ לפולינום מק'לורן מסדר $7$. 
		\[ \sinx = x - \frac{x^{3}}{3!} + \frac{x^{5}}{5!} - \frac{x^{7}}{7!} + R_7(x) \]
		כלומר: 
		\[ \sin1 = 1 - \frac{1}{6} + \frac{1}{120} - \frac{1}{7!} + R_7 \]
		תחשבו את זה להנאתכם בלי מחשבון. לפי פיתוח שארית לגראנג', קיים $c \in (0, 1)$ כך ש־: 
		\[ R_7(1) = \frac{f^{(8)(c)}}{8!}(1 - 0)^{8} \]
		מכיוון ש־$c \in (0, 1)$ בהכרח $\sof{f^{(8)}(c)} \le 1$. מכאן: 
		\[ \sof{R_7(1)} \le \frac{1}{8!} < \frac{1}{1000} \]
		אגב, השארית האמיתית היא משהו בסביבת $0.000002730839643$. 
		\item הוכחת התכנסות של {\bf \Large \textit{טור הטיילור}} לפונקציה עצמה. 
		
		\textbf{דוגמה: }
		נגדיר $f(x) = \sinx$. יהי $x \in \R$. יהי $\eg > 0$. קיים $n \in \N$ כך ש־$\sof{\frac{x^{n}}{n!}} < \eg$ (סטירלינג, וגם הוכחנו בלי. שימו לב שה־$n$ תלוי ב־$\eg$ וב־$x$). לפי פיתוח מק'לורן של סינוס עם שארית לגראנג': 
		\[ \sof{\sinx - \sumnk (-1)^{k}\frac{x^{2k + 1}}{(2k + 1)!}} = \sof{R_{2n + 1}(x)} \le \sof{\frac{1}{(2n + 2)!}x^{2n + 2}} < \eg \]
		קבענו את $x$. לכן אנחנו בטור חמודי: 
		\[ \sinx = \limsi \sumnk (-1)^{k}\frac{x^{2k + 1}}{(2k + 1)!} = \sum_{k = 0}^{\infty} (-1)^{k} \frac{x^{2k + 1}}{(2k + 1)!} \]
	\end{enumerate}
	
	\href{https://eincyclopedia.org/wiki/\%D7\%98\%D7\%95\%D7\%A8\_\%D7\%98\%D7\%99\%D7\%99\%D7\%9C\%D7\%95\%D7\%A8}{מבולבלים מטיילור? אני ממש ממליץ על הסיכום הבא (זה קישור לחיץ)}
	
	יש לטורי טיילור יותר כוח ממה שאנחנו רואים כאן. אגב, בהקשר ל־$\sinx$ שטור הטיילור שלה מתכנס, זה לא נכון לכל פונקציה. לדגוהמ אם מגדיר $f \co \R\to \R$. 
	\[ f(x) = \begin{cases}
		e^{-\frac{1}{x^{2}}} & x \neq 0 \\
		0 & \other
	\end{cases} \]
	אפשר להוכיח באינדוקציה של $f$, שיש לה נגזרת מסדר $n$ ב־$0$, ושלכל $n \in \N^{+}$ מתקיים $f^{(n)}(0) =0$. הוא לא מתכנס לפונקציה. נ.ב. זו פונקציה מוכרת עם שימושים בסטטיסטיקה או משהו כזה. 
	
	יש פונקציות שעבורן הטור לא מתכנס בכלל. לדוגמה $\frac{1}{1 - x}$ שטורה $\sum_{n = 0}^{\inft}x^{n}$, לא מתכנס בתחומים מסויימים. 
	
	
	טור הטיילור לא יכול להיות כיאוטי יותר מדי – הראינו שקיים רדיוס התכנסות, ואומנם לא ברור מה קורה בקצוות שלו, אבל בכלים של חדו''א 2א אפשר להראות שטור חזקות רציף בתחום הזה. 
	
	האמריקאים מזיזים את נושאת המטוסים שלהם מסין לאיראן. שיעור הבא תלוי במצב הרוח של טראמפ. 
	
	\begin{proof}
		באינדוקציה על $n \in \N^{+}$.
		\begin{itemize}
			\item \textbf{בסיס: }ב־$n = 0$ נקבל שפולינום הטיילור קבוע, וערכו $T_n = f(x_0)$. מכאן $R_0(x) = f(x)- f(x_0)$. נבחין ש־$f$ מקיימת את תנאי משפט לגראנג' בקטע שבין $x$ ל־$x_0$. לכן קיים $c$ בין $x$ ל־$x_0$ כך ש־: 
			\[ R_0(x) = f(x) - f(x_0) = f'(c)(x -x_0) \]
			\item \textbf{צעד: }נניח באינדוקציה על $n$ ונוכיח ל־$n + 1$. יהי $x_0 \neq x \in I$. $R_{n + 1}$ ו־$(x - x_0)^{n + 2}$ גזירות בין $x$ ל־$x_0$ והנגזרת של $(x - x_0)^{n + 2}$ אינה מתאפסת בקטע הפתוח שבין $x$ ל־$x_0$. לכן קיים $c$ בין $x$ ל־$x_0$, כך ש־: 
			\[ \frac{R_{n + 1}(x)}{(x - x_0)^{n + 2}} = \frac{R_{n + 1}(x) - R_{n + 1}(x_0)}{(x - x_0)^{n + 2} - 0^{n + 2}} = \frac{R_{n + 1}'(c)}{(n + 2)(c - x_0)^{n + 1}} \]
			זאת ממשפט קושי. $R_{n + 1}'$ היא השארית בפיתוח של $f'$ מסדר $n$ סביב $x_0$ (הוכחנו את זה כחלק מהוכחה הקודמת). מה.א. קיים $d$ בין $c$ ל־$x_0$ כך ש־: 
			\[ R'_{n + 1}(c) = \frac{(f')^{n + 1}(d)}{(n + 1)!}(c - x_0)^{n + 1} \]
			לכן: 
			\[ \frac{R_{n + 1}'(c)}{(c - x_0)^{n +1}} = \frac{f^{(n + 2)}(d)}{(n + 1)!} \]
			מציבים כל הדרך למעלה, מקבלים: 
			\[ \frac{R_{n + 1}(x)}{(x - x_0)^{n +1}} = \frac{f^{(n + 2)}(d)}{(n+2)!} \implies R_{n + 1} = \frac{f^{(n + 2)(d)}}{(n + 2)!}(x - x_0)^{n + 2} \]
		\end{itemize}\envendproof
	\end{proof}
	\defi{מסמנים ב־$C^{\inft}(A)$ את קבוצת הפונקציות הגזירות (ובפרט רציפות) מכל סדר ב־$A$. }
	\theo{תהא $f \in C^{\inft}(A)$. אם קיים $M > 0$ כך ש־$\forall n \in \N \forall x \in I \co \sof{f^{(n)}(x)}\le M$ (''הנגזרות חסומות באופן אחיד``), אז טור טיילור של $f$ מתכנס ל־$f$ בכל $I$. }
	
	\theo{טור הטיילור של $e^{x}$ מתכנס ל־$e^{x}$ בכל נקודה, כלומר $\forall x \in \R\co e^{x} = \sum_{n = 0}^{\inft}\frac{x^{n}}{n!}$. }
	לא נוכל להשתמש במשפט הקודם, כי הנגזרות לא חסומות. כן נוכל להוכיח התכנסות. 
	\begin{proof}
		יהי $\hat x \in \R$. נסמן $I = (\sof{\hat x} - 1, \sof{\hat x} + 1)$. בקטע $I$ כל הנגזרות של $e^{x}$, חסומות ע''י $e^{\sof{\hat x}} + 1$ (הסיבה שצריך ערך מוחלט: כי צריך קטע שכולל גם את $0$ וגם את $\hat x$, כי זו הנקודה סביבה הטור מפותח). לכן, מהמשפט, לכל $x \in I$ מתקיים $e^{x} = \sum_{i = 0}^{\inft}\frac{x^{n}}{n!}$. בפרט $e^{\hat x} = \sum_{i = 0}^{\inft}\frac{x^{n}}{n!}$. 
	\end{proof}
	במילים אחרות, מה שצריך בפועל זה שהנגזרות יהיו חסומות בכל קטע קומפקטי. רק שאת זה לא הפכו למשפט בקורס. 
	
	
	
	
\end{document}