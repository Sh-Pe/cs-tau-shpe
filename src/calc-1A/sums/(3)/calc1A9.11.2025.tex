\documentclass[]{../../../../tex/classes/styledArticle}
\usepackage{../../../../tex/packages/hebrewSupport}
\usepackage{../../../../tex/packages/mathShortcuts}
\usepackage{../../../../tex/packages/theoremsSupport}

\newcommand \limcsi {\limsi c_n}
\newcommand \az     {\aleph_0}

\author{שחר פרץ}
\title{\textit{חדו''א 1א 3}}
\begin{document}
	\maketitle
	תהאנה $\an, \bn$ סדרות. הוכיחו או הפריכו: 
	\begin{enumerate}[A.]
		\item אם $\an$ אינה מתכנסת, וגם $\bn$ אינה מתנכסת, אז $\an + \bn$ אינה מתכנסת. 
		
		\textbf{תשובה: }לא נכון. אפשר להראות שזה לא עובד, לדוגמה עבור $a_n = (-1)^{n}$ ו־$b_n = (-1)^{n + 1}$, הראינו ש־$\an$ אינה מתנכסת ובאופן דומה $\bn$ אינה מתכנסת. אבל, $\forall n \in \N \co a_n + b_n = 0$ ולכן $\limsi a_n + b_n = 0$
		
		\item אם $\an$ מתכנסת וגם $\bn$ אינה מתכנסת, אז $(a_n + b_n)_{i = 1}^{\inft}$ אינה מתכנסת. \textbf{תשובה: }
		
		\begin{proof}
			נניח ש־$\an$ מתכנסת וגם $\bn$ אינה מתכנסת. נניח בשלילה ש־$a_n + b_n$ מתכנסת. מכאן שיש גבול $\ml$ לסדרה, ומאריתמטיקה של גבול $\limsi a_n + b_n - \limsi a_n = \limsi (\cancel{a_n - a_n} + b_n) = \limsi b_n$ אך $\limsi b_n$ אינו מוגדר שכן $\bn$ לא מתכנסת. 
		\end{proof}
		\item אם $\an$ מתכנסת וגם $\bn$ אינה מתכנסת, אז $a_n \cdot b_n$ מתכנסת. 
		
		\textbf{תשובה: }בניגוד להוכחה הקודמת, צריך בשביל להוכיח לטעון ש־$\limsi a_n \neq 0$ כדי שנוכל לחלק. לא מפתיע אם כן שעבור $b_n = (-1)^{n}, a = 0$ נקבל סתירה שכן $\bn$ לא מתכנסת, אך $a_n \cdot b_n = 0$ הסדרה הקבוע. 
		
		\item לבית: כמו הסעיף הקודם אבל $\limsi b_n \neq 0$. 
	\end{enumerate}
	
	\theo{תהא $\an$ סדרה, יהי $\ml \in \R$. אם $\limasi = \ml$ אז $\limsi \sof{a_n} = \sof \ml$. }\begin{proof}
		נניח $\limasi = \ml$. יהי $\eg > 0$. אז קיים $N \in \N$ כך ש־$\forall n \le N \co \sof {a_n - \ml} < \eg$. נתבונן ב־$N$. יהי $n \ge N$. מא''ש המשולש ההפוך: 
		\[ \sof{\sof {a_n} - \sof \ml} \le \sof{a_n - \ml} < \eg \]
		מההגדרה $\limsi \sof{a_n} = \sof \ml$ כדרוש. 
	\end{proof}
	\rmark{הבעייתיות בכיוון השני היא אם $\an$ מחליפה סימן אינסוף פעמים. הוא נכון אם $\an$ שומרת סימן מגבול מסויים (מה ששקול לכך שיש לה גבול, ממשפט נחמד שהראינו בעבר). }
	
	\begin{Theorem}
		תהאנה $\an, \bn, \cn$ סדרות. נניח כי־: 
		\begin{enumerate}
			\item \hfil $\exists N \in \N .\,\forall n \ge N \co a_n \le c_n \le b_n$
			\item \hfil $\displaystyle \limasi = \ml  = \limbsi$
		\end{enumerate}
		אז $\limcsi = \ml$
	\end{Theorem}
	\begin{proof}
		יהי $\eg > 0$. קיים $N_2 \in \N$ כך ש־$\forall n \ge N_2 \co \sof{b_n - \ml} < \eg$, ובאופן דומה קיים $N_3 \in \N$ כך ש־$\forall n \in N_3\co \sof{a_n - \ml} < \eg$. נסמן $N = N_1$ (מהנתון). נתבונן ב־$N = \max\{N_1, N_2, N_3\}$. יהי $n \ge N$. אז נסיק $\ml - \eg < b_n < \ml + \eg$ וכן $\ml - \eg < a_n < \ml + \eg$ ואז: 
		\[ \ml - \eg < c_N \le c_n \le b_n < \ml + \eg \]
		כלומר $\sof{c_n - \ml} < \eg$ כדרוש. 
	\end{proof}
	
	\subsection{גבולות ושוויונות}
	\theo{תהנא $\an, \bn$ סדרות. יהיו $\ml, m \in \R$. נניח כי:
	\begin{enumerate}[(1)]
		\item לכל $n \in \N$, מתקיים $\an < \bn$. (\textit{הערה: }מספיק גם אם החל מ־$N$ כלשהו התנאי הזה מתקיים)
		\item מתקיים $\limasi = \ml$
		\item מתקיים $\limbsi = m$
	\end{enumerate}
אז $\ml \le m$}
	\begin{proof}
		נניח בשלילה $m < \ml$. נסמן $\eg = \frac{\ml - m}{2}$, בהכרח $\eg > 0$. לכן $\exists N_1 \in \N.\, \forall n \ge N_1 \co \sof{a_n - \ml} < \eg$. כמו כן $\exists N_2 \in \N.\, \forall n \ge N_2 \co \sof{b_n - m} < \eg$. נתבונן ב־$N = \max\{N_1, N_2\}$. שם, מתקיים: 
		\[ b_N < m + \eg = \frac{\ml + m}{2} = \ml - \eg < a_N \]
		בסתירה ל־$(1)$. וסיימנו. 
	\end{proof}
	למה היינו צריכים להניח בשלילה? כי עקרונית נרצה לקחת את $\frac{\sof{\ml - m}}{2}$ ולעבוד עם זה, ולהפעיל על זה את הגדרת הגבול, אבל זה יכול להיות $0$. לכן נרצה להניח בשלילה ש־$\ml > m$, כי כאן יש א''ש חזק ממש. 
	
	לועמת זאת, אם נניח ב־$(1)$ במקום זאת $\forall n \in N \co a_n < b_n$, עדיין נוכל לדעת $\ml \le m$ בלבד, למרות שהא''ש לכאורה חזק. לדוגמה, בעבור $a_n = \frac{1}{n}$ ו־$b_n = 0$ מתקיים שוויון חלש ולא חזק בגבול, על אף ש־$b_n < a_n$. 
	
	
	\theo{תהאנה $\an, \bn$ סדרות, ויהיו $\ml, m \in \R$. נניח ש־$\limasi = \ml, \limbsi = m$. נניח גם $\ml < m$. נוכיח שהחל ממקום מסויים $\exists N \in \N.\, \forall n \ge N \co a_n < b_n$. }
	הוכחה לבית. 
	
	\begin{Theorem}[משפט ויירשטראס הראשון]
		תהא $\an$ סדרה. אם $\an$ מונוטונית וחסומה, אז $\an$ מתכנסת. 
	\end{Theorem}
	\begin{proof}
		בה''כ נניח ש־$\an$ מונוטונית עולה (אחרת ההוכחה בדומה). ידוע ש־$\an$ חסומה, ובהכרח מלמעלה, ולכן לפי אקסיומת השלמות חיים לה חסם עליון, נסמנו $\ml = \sup a_n$. יהי $\eg > 0$. אז קיים $N \in \N$, כך ש־$a_n > \ml - \eg$. נתבונן ב־$N$. יהי $n \ge N$. אז: 
		\[ \ml - \eg < a_N \le a_n \le \ml < \ml + \eg \]
		כלומר $\sof{a_n - \ml} < \eg$. לכן $\an$ שואפת ל־$\ml$, כדרוש.  
	\end{proof}
	\defi{סדרה $\an$ תקרא \textit{בעלת גבול במובן הרחב} אם $(\exists \ml \in \R\co \limasi = \ml) \lor \limasi = \pm \infty$. }
	\theo{בהינתן סדרה מונוטונית לא חסומה, היא שואפת ל־$\pm \inft$. }
	\cola{תהי $\an$ מונוטונית. אז ל־$\an$ יש גבול במובן הרחב. }
	
	\subsection[$e$]{\hfil $e$}
	\theo{נגדיר $a_n = \cl{a + \frac{1}{n}}^{n}$ לכל $n \in \N$, ו־$b_n = \sumnk \frac{1}{k!}$ לכל $n \in \N$. אז:
	\begin{enumerate}
		\item $\an$ חסומה, מונוטונית עולה וחסומה ב־$3$. 
		\item $\bn$ חסומה, ומונוטונית עולה. 
		\item $\forall n \in \N \co a_n \le b_n$
		\item $\forall n \in \N.\, \exists k > n \co b_n \le a_{n + k}$
	\end{enumerate}}
	המסקנה מ־1, 2 הוא שיש להן גבול (ממשפט וויראשטראס). ממשפט אחר שהראינו, 3 ו־4 גוררים ש־$\an, \bn$ מתכנסות לאותו הגבול. נסמנו ב־$e$. 
	\defi{נסמן:
		\[e := \limsi \cl{1 + \frac{1}{n}}^{n} \dequad\ = \limsi \sumnk \frac{1}{k!}\]}
	
	\section*{תת־סדרות וגבולות חלקיים}
	\defi{תהי פונקציה $n_k \co \N \to \N$ סדרה עולה ממש של טבעיים, ותהא $\an$ סדרה. אז הסדרה $a_{(n_k)}$ נקראת \textit{תת־סדרה של} $\an$. פורמלית, זוהי הרכבה $a_n \circ n_k$. }
	\defi{$\ml$ יקרא \textit{גבול חלקי} של $\ml$ כאשר קיימת ת''ס של $\an$ המתכנסת ל־$\ml$. }
	\defi{$\pm\infty$ יקרא גבול חלקי של $\an$, כאשר קיימת ת''ס השואפת ל־$\pm\infty$. }
	
	לדוגמה, עבור $a_n = (-1)^{n}$, ו־$a_{2k}$ ת''ס של $\an$, אז $\limsi a_{2k} = 1$. לכן $1$ גבול חלקי של $\an$. באופן דומה $-1$ גבול חלקי של $\an$ ואפשר גם להוכיח יחידות. 
	
	\rmark{לעיתים, לגבולות חלקיים קוראים \textit{נקודות גבול}. }
	
	להלן משפט שקצת יוצא מתחומי החדו''א. 
	\begin{Theorem}[משפט הרקורסיה]
		תהא $f \in \N \times \R \to \R$. יהי איזשהו $a \in \R$. אז קיימת סדרה יחידה $\an$ המקיימת: 
		\[ \begin{cases}
			a_0 = a \\
			\forall n \in \N \co a_{n + 1} = f(n, a_{n})
		\end{cases} \]
	\end{Theorem}
	למה אנחנו צריכים את המשפט הזה? כי אם כותבים משהו כמו $a_0 = 2, \, a_{n + 1} = 2^{n}a_n + 1$ (בהקשר הזה $f(x, y) = 2^{x} y + 1$), למה שבכלל תהיה $\an$ שתקיים את תנאי הנסיגה הזה? המשפט הזה דואג לכך שנוסחאות נסיגה יהיו מוגדרות היטב (קיימות ויחידות בהינתן כלל נסיגה עם תנאי בסיס). אפשר להכליל באינדוקציה לפונקציות נסיגה מדרגה $k$־ית. 
	
	השבוע יעלה למודל תרגיל מודרך העוסק בזה. 
	
	\begin{Theorem}[משפט בוצלנו־וייראסטראס]
		לכל סדרה חסומה, יש ת''ס מתכנסת. 
	\end{Theorem}
	\lem{תהא $\an$ סדרה. נניח של־$\an$ אין איבר מסקימלי. אז יש לה תת סדרה מונוטונית עולה ממש. }
	\begin{proof}
		יהי $n \in \N$. נסמן $A_n = \{m \in \N \co m> n \land a_m > a_n\}$. מהיות $\an$ ללא איבר מקסימלי, לכן קיים $m \in \N$ כך ש־$a_m > \max\{a_1 \dots a_n\}$. בפרט $a_m > a_n$ ולכן $m \in A$. מכאן שבהכרח $A_n$ לא ריקה. 
		
		נגדיר $n_k \co \N\to \N$ ברקורסיה: 
		\[ \begin{cases}
			n_1 = 1 \\
			n_{k + 1} = \min A_{(n_k)}
		\end{cases} \]
		המינימום בהכרח מוגדר היטב מהיות $A_{(n_k)}$ לא ריקה. ממשפט הרקורסיה $(n_k)_{i = 1}^{\infty}$ מוגדרת היטב. כדי להראות שהיא ת''ס, יש להראות שהיא מונוטונית עולה חזק. ואכן מהגרדה $n_{k + 1} > n_k$. יתרה מכך, היא גם מונוטונית עולה חזק על $\an$ שכן מהגדרה $a_{n_{k + 1}} > a_{n_k}$. סה''כ $(a_{n_k})_{k = 1}^{\infty}$ ת''ס מונוטונית עולה ממש וסיימנו. 
	\end{proof}
	\rmark{מה שהבטיח לנו את קיום המינימום, פרט לכך שהקבוצה לא ריקה, הוא שהסדר על הטבעיים \textbf{סדר טוב}. }
	
	\lem{תהא $\an$ סדרה שבה אינסוף איברים שונים. אם ל־$\an$ אין ת''ס מונוטונית עולה ממש, אז יש לה ת''ס מונוטונית יורדת ממש. }
	\rmark{ת''ס של ת''ס היא ת''ס}
	\begin{proof}[הוכחת משפט בולצאנו־וייראשטראס]
		תהא $\an$ סדרה. נפריד למקרים. 
		\begin{itemize}
			\item נניח ש־$\{a_n\}_{n = 1}^{\infty}$ (הטווח של $\an$) סופית. אז קיים $\ml \in \R$ כך ש־$\sof{\{n \in \N \mid a_n = \ml\}} = \az$. נבנה ברקורסיה ת''ס $\{a_{n_k}\}$ עבורה $a_{n_k} = \ml$ לכל $k \in \N$. 
			\item אם $\{a_n\}_{n = 1}^{\infty}$ אין־סופי, אז מהלמה הקודמת קיימת ל־$\an$ ת''ס סדרה מונוטונית (ממש) $a_{n_k}$. בגלל ש־$\an$ חסומה אז בפרט $a_{n_k}$ חסומה. לפי משפט קודם כל סדרה מונוטונית חסומה היא מתכנסת, וסיימנו. 
		\end{itemize}
		סה''כ בשני המקרים מצאנו ת''ס מתכנסת. 
	\end{proof}
	משפט בולצאנו־וייראשטראס השתמש במשפט וייראשטראס (הראשון), שתלוי באקסיומת השלמות. משפט בוצאלנו־וייראשטראס תלוי באקסיומת השלמות!
	
	להלן הוכחה נוספת, קונסטקרטיבית אפילו פחות (לא שפונקציית בחירה זה קונסטקרטיבי במיוחד): 
	\begin{proof}[הוכחה נוספת לבולצאנו־וייראשטראס]
		נסמן ב־$A$ את התמונה של $\an$. אם $A$ סופית – כמו קודם. אחרת $A$ אינסופית. נגדיר את הקבוצה: 
		\[ B = \{x \in \R \co \sof{\{y \in A \mid y \le x\}} \ge \az\} \]
		$B$ חסומה מלמטה (למשל ע''י $\inf a_n$). היא לא ריקה, כי לדוגמה $\sup a_n \in B$. לכן ל־$B$ קיים חסם תחתון (אקסיומת השלמות). נסמן $\ag = \inf B$. יהי $\eg > 0$. אז קיים $b \in B$ כך ש־$b < \ag + \eg$. מתקיים $\ag - \eg < \ag \implies \ag - \eg \notin B$  לכן $\{y \in A \mid y \le b \le \ag + \eg\}$ אין־סופית, אבל $\{y \in A \mid y \le \ag - \eg\}$ סופית. לכן, $A_{\eg} = \{y \in A \co \sof{y - \ag} < \eg\}$ אינסופית!
		
		נסכם: לכל $\eg > 0$, ולכל $N \in \N$ קיים $n \ge \N$ כך ש־$\sof{a_n - \ag} < \eg$, בגלל ש־$A_\eg$ אינסופי.  וזו כבר הגדרה שקולה לקיום גבול חלקי, כמו שנראה בקרוב. 
	\end{proof}
	\theo{$\forall \eg > 0 .\, \forall N \in \N.\, \exists n \ge N \co \sof{a_n - \ag} < \eg$ אמ''מ לקבוצה יש גבול חלקי ב־$\ag$. } \begin{proof}
		\begin{itemize}
			\item[$\impliedby$]נניח את הטענה שנראית מפחיד. נגדיר: 
			\[ \begin{cases}
				n_1 = \min\{n \in \N \co \sof{a_n - \ag} < 1\} \\
				n_{k + 1} = \min \{n \in \N \co n < n_k \land \sof{a_n - \ag} < \frac{1}{k + 1}\}
			\end{cases} \]
			אז $a_{n_{k}}$ ת''ס של $\an$. יהי $\eg > 0$. קיים $K \in \N$ כך ש־$\frac{1}{K + 1} < \eg$. יהי $k \ge K$ אז: 
			\[ \sof{a_{n_k} - \ag} < \frac{1}{k} \le \eg \]
			וסה''כ $\limsi a_{n_k} = \ag$ וסיימנו. 
			\item[$\implies$]לבית
		\end{itemize}
	\end{proof}
	\rmark{המשפט לעיל הוא לא באמת משפט בקורס. צריך להוכיח אותו כל פעם מחדש. }
	
	''בשפה של בני אדם``, הטענה השקולה הזו אומרת שבכל קטע פתוח שמכיל את $\ag$ יש אינסוף איברים מהסדרה, [וההגדרה של גבול לא חלקי דורשת ש־] מחוץ אליו, יש מספר סופי של איברים. 
	
	\cola{לכל סדרה יש גבול חלקי במובן הרחב. }
	
	\theo{סדרה מתכנסת אמ''מ יש לה גבול חלקי יחיד. }\begin{proof}
		\begin{itemize}
			\item[$\impliedby$] בכיוון הראשון, נוכיח: תהא $\an$ סדרה. יהי $\ml \in \R$. אם $\limasi = \ml$ אז כל ת''ס של $\an$ מתכנסת ל־$\ml$. 
			
			כיוון זה לבית. שימו לב שצריך להפריד למקרים גבולות מתבדרים וכאלו שאינם. 
			
			\item[$\implies$] עתה, תהא $\an$ סדרה, ויהי $\ml \in \R$. אם כל ת''ס של $\an$ מתכנסת ל־$\ml$, אז $\limasi a_n = \ml$. 
			
			לבית גם כן. (זה טרוויאלי. $\an$ ת''ס של עצמה וסיימנו)
		\end{itemize}
	\end{proof}
	
	\theo{תהא $\an$ סדרה חסומה ויהי $\ml \in \R$. נניח כי כל ת''ס \textit{מתכנסת} של $a_n$ מתכנסת ל־$\ml$. אז $\limasi = \ml$. }
	\rmark{מה לא טרוויאלי כאן? אי אפשר פשוט לבחור את $\an$, שכן היא לא מתכנסת בהכרח (צריך להוכיח את זה). }
	\begin{proof}
		יהי $\eg > 0$. נסמן $A = \{n \in \N \co \sof{a_n - \ml} \ge \eg\}$. נניח בשלילה ש־$A$ אינסופית. נסמן $A_{+} := \{n \in \N\co a_n \ge \ml + \eg\}$ ו־$A_- = \{ n \in \N \co a_n \le \ml - \eg\}$. משום ש־$A = A_+ \cup A_-$, ללא הגבלת הכלליות, $A_+$ אינסופית. לכן קיימת ת''ס $a_{n_k}$ כך שלכל $k \in \N$, $n_{k} \in A_+$. $a_{n_k}$ חסומה ולכן יש לה ת''ס $a_{n_{k_j}}$ מתכנסת. נסמן את גבולה $m$. לכל $j \in \N$, מתקיים $\ml + \eg \le a_{n_{k_j}}$. לכן $m \ge \ml + \eg> \ml$, כלומר $m \neq \ml$ גבול חלקי של $\an$ בסתירה. 
	\end{proof}
	המטרה בלהכניח ש־$\an$ חסומה, היא לחסוך את הפיצול למקרה האין־סופי. 
	
	
	
	
	\ndoc
\end{document}