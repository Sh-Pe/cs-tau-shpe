\documentclass[]{../../../../tex/classes/styledArticle}
\usepackage{../../../../tex/packages/hebrewSupport}
\usepackage{../../../../tex/packages/mathShortcuts}
\usepackage{../../../../tex/packages/theoremsSupport}


\author{שחר פרץ}
\title{\textit{חדו''א 1א 9}}
\begin{document}
	\maketitle
	\subsection*{המשך רציפות}
	\textbf{אריתמטיקה של רציפות}
	מייבא אוטומאטית הכל מאריתמטיקה של גבולות פונקציות. 
	\theo{תהאנא $f, g \co A \subseteq \R \to \R$ ותהא $x_0 \in \R$. נניח כי $f$ רציפה ב־$x_0$ וכן $g$ רציפה ב־$x_0$. אז: 
	\begin{itemize}
		\item $f\pm g$ רציפה ב־$x_0$
		\item $f \cdot g$ רציפה ב־$x_0$. 
		\item אם $g(x_0) \neq 0$ אז $\frac{f}{g}$ רציפה ב־$x_0‏$. 
	\end{itemize}}

	\theo{תהאנה $f \co A \to B$ ו־$g \co B \to \R$, ותהא $x_0 \in A$. נניח כי $f$ רציפה ב־$x_0$ ו־$g$ רציפה ב־$f(x_0)$. אז $g \circ f$ רציפה ב־$x_0$. }
	
	\textbf{דוגמאות לפונקציות רציפות: }
	\begin{itemize}
		\item פולינומים (מראים שהזהות והקבועה רציפות, ואז מאריתמטיקה סיימנו). 
		\item הפונקציות הטריגונומטריות רציפות בכל נקודה בה הן מוגדרות. 
		\item הפונקציות הטריגונומטריות ההפוכות רציפות בכל נקודה בה הן מוגדרות. 
		\item הפונקציות המעריכיות רציפות ב־$\R$ (מהיינה וממשפט קודם שהגדיר היטב חזקה). 
		\item לכל $1 \neq a > 0$ הפונקציה $\log_a x$ רציפה ב־$(0, \infty)$. 
		\item הפונקציה $\sof x$ רציפה בכל $\R$. 
	\end{itemize}
	
	\textbf{גבולות חשובים}
	\theo{\hfil $\disty \lim_{x \to 0}\frac{\sinx}{x} = 1$}
	\begin{proof}[הוכחה אבל חצי כח]
		לא באמת אני יכול להעתיק כי יש כאן מעגל היחידה ודברים שאין לי כח להעתיק. ההוכחה לא פורמלית בכל מקרה. זו הוכחה מאוד סטנדרטית שיצא לי לראות בעבר ואני משוכנע שתוכלו למצוא הוכחות באינטרנט. שימו לב שלופיטל זה טיעון מעגלי. עקרונית מראים על מעגל היחידה באמצעות טיעונים גיאומטריים לא מוגדרים היטב על משולש עם זווית $x_{\mathrm{rad}}$ על המעגל, ש־$\sinx \le x \le \tanx$ ומכאן $1 \le \frac{x}{\sinx} \le \frac{1}{\cosx}$ וידוע מרציפות $\limsi \frac{1}{\cosx} = 1$ ומסנדוויץ' סיימנו. 
	\end{proof}
	
	\theo{\hfil $\disty \limz \frac{\ln(1 + x)}{x} = 1$}
	\begin{proof}
		די בקלות. לכל $x \in (-1, \infty) \setminus \{0\}$ נקבל: 
		\[ \frac{\ln(1 + x)}{x} = \ln\cl{\cl{1 + x}^\frac{1}{x}} \]
		מסדרות + היינה: 
		\[ \limz \cl{(1 + x)^{\frac{1}{x}}} = e \]
		הסלנג הוא ''להכניס את הגבול פנימה``, אבל זה רציפות והרכבה: 
		\[ \limz \ln\cl{(1 + x)^{\frac{1}{x}}} = \ln e = 1 \]
	\end{proof}
	\theo{\hfil $\disty \frac{e^{x} - 1}{x} = 1$}
	\begin{proof}
		נעשה מעברים אלגברים: 
		\[ \frac{e^{x} - 1}{x} = \frac{e^{x}}{\ln(e^{x} - 1 + 1)} \]
		נציב $t = e^{x} - 1$ (בפועל, משמעו הרכבה שחוקית רק מרציפות $\ln$): 
		\[ = \frac{\ln(1 + t)}{t} = 1 \]
		תוך שימוש בסעיף הקודם. 
	\end{proof}
	
	לבית חשבו את הגבול $\limsi \frac{(1 + x)^{\ag} - 1}{x}$. 
	
	\subsection*{תכונות גלובליות של פונקציות רציפות}
	\defi{פונקציה $f$ היא \textit{רציפה} אם היא רציפה בכל נקודה. }
	\theo{תהא  $f \co A \to \R$. אז $f$ רציפה אמ''מ לכל קבוצה פתוחה $V \subseteq \R$ קיימת קבוצה פתוחה $U \subseteq \R$ כך ש־$f\op(V) = U \cap A$. }
	\begin{proof}
		\begin{itemize}
			\item[$\implies$]תהא $V \subseteq \R$. תהא $x \in f\op(V)$. אחרת $f(x) \in V$ לכן לא קיים $\eg > 0$כך ש־$f(x)- \eg, f(x) + eg$. ידוע $f$ רציפה ב־ $x$ (מהנתון). לכן קיים $\dg_x > 0$ כך שלכל $y \in (x - \dg_x), d + \dg_x$ מתקיים $f(y) \in (x - \dg_x, x + \dg_x)$ לכן $(x - \dg_x, x + \dg_x) \cap A \subseteq f\op(V)$. 
			
			נגדיר: 
			\[ U = \bigcup_{\mathclap{x \in f\op(V)}}(x - \dg_x, x + \dg_x) \]
			נבחין ש־$U$ פתוחה שכן היא איחוד של קבוצות מבסיס הטופולוגיה. כמו כן לכל $x \in f\op(V)$ מתקיים $(x - \dg_x, x + \dg_x)$ לכן $U \cap A \subseteq f\op(V)$. בנוסף מהגדרת האיחוד, $f\op(V) \subseteq UA \cap A$. לכן $U \cap A = f\op(V)$. 
			\item[$\impliedby$]נניח שלכל $V$ פתוחה קיימת $U \subseteq \R$ פתוחה כך ש־$f\op(A)(V) = U \cap A$. יהי $x \in A$. יהי $\eg > 0$. $(x - \eg, x + \eg) \subseteq \R$ פתוחה ולכן קיימת $U \subseteq \R$ פתוחה כך ש־$U \cap A = f\op((f(x) - \eg, f(x) + \eg))$. יהי $x \in U \cap A$ לכן $x \in U$ ולכן לא קיים $\dg > 0$ כך ש־$(x - \dg, x + \dg) \subseteq U$. לכל $y \in A$ אם $\sof{y - x} < \dg$ אז $\sof{\sof{y} - \sof{x}} < \eg$. לכן $f$ רציפה וסיימנו. 
		\end{itemize}
	\end{proof}
	
	\defi{תהא $f \co I \to \R$ כאשר $I$ קטע. נאמר כי $f$ מקיימת \textit{תכונת דרבו} כאשר לכל $a, b \in R$ כך ש־$a < b$, לכל $\lg \in \R$ בין $f(a) \le \lg \le f(b)$. קיים $c \in [a, b]$ כך ש־$f(x) = \lg$. }
	\begin{Theorem}[משפט ערך הביניים]
		פונקציה רציפה מקיימת את תכונת דרבו. 
	\end{Theorem}
	\begin{proof}
		יהיו $a, b \in I$ ונניח ש־$a< b$. יהי $\lg \in \R$ כך ש־$f(a) \le \lg \le f(b)$. נבנה סדרת קטעים ברקורסיה: $a_1 =a, b_1 = b$ ואז צעד: 
		\[ \begin{cases}
			a_{n + 1} = \frac{a_n + b_n}{2}, b_{n + 1} = b_n & f\cl{\frac{a_n  + b_n}{2}} \le \lg \\
			a_{n  + 1} = a_n \land b_n = \frac{a_n + b_n}{2} & f\cl{\frac{a_n + b_n}{22}} > \lg
		\end{cases} \]
		\item 
		לכל $n \in \N$, נקבל $\sof{b_n - a_n} = \frac{b - a}{2^{n - 1}}$. לכל $n \in \N$ נקבל $f(a_n) \le \lg \le f(b_n)$ (אינדוקציה). ידוע $\limsi \frac{b - a}{2^{n - 1}} = 0$. לפי קנטור קיימת $ c\ni \R$כך ש־$\bigcup_{n = 1}^{\infty}[a_n, b_n] = \{c\}$. לכן $\limsi a_n = c$. מרציפות הגבול $\limsi a_n = c$ ומרציפות $\limsi I\lim f(a))$ לכל $n \in \N$, $f(a) \le \lg$ ולכן $f(x) \le \lg$. באופן דומה $f(c) \ge \lg$ ולכן $f(x) = \lg$. 
	\end{proof}
	\begin{proof}[הוכחה נוספת]
		אחרי יהיו יהי תהיינה, אם $f(a) = \lg$ סיימנו. אחרת $f(a) < \lg$. נגדיר $A = \{x \in [a, b] \co f(x) < \lg\}$. אז $A$ לא ריקה כי $f(a) < \lg$ כלומר $\sup A$ קיים מאקסיומת השלמות. נניח בשלילה ש־$f(\ag) < \lg$. מרציפות $f$ קיים $\dg > 0$ כך שלכל $x \n (\ag - \dg, \ag + \dg)$, מתקיים $\sof{f(x) - f(\ag)} < \frac{num}{\lg f(x2)}$ נובע $f(x) < \lg$ בסתירה למינימליות הסופרמום. מהצד השני נוכל להפעיל ותו הטיעון ההפוך. לכן $f(\ag) = \lg$. 
	\end{proof}
	\rmark{זה לא אמ''מ. להלן דוגמאות לפונקציות לא רציפות שמקיימות את תכונת ערך הביניים: }
	\begin{itemize}
		\item \textbf{פונקציית צימרמן: }בהינתן $r$, נגדיר שהיא תחזיר את הגבול של הממוצע החשבוני של הספרות במידה והוא קיים, אחרת $0$. 
		\item \textbf{פונקציה סימפטית מספיק: } $\sin \frac{1}{x}$ (שמחזירה $0$ ב־$0$ בשביל נוחות). היא מקיימת דרבו אך אינה רציפה כי אין לה גבול ב־$0$. 
	\end{itemize}
	
	\begin{Theorem}[משפט ווירשטראס (עוד אחד)]
		תהא $f \co A \to \R$ רציפה. אם $A$ קומפקטית (סגורה וחסומה) אז $f$ חסומה ומשיגה את חסמיה (יש לה מינימום ומקסימום). 
	\end{Theorem}
	\begin{proof}[חלק ראשון]
		נניח בשלילה ש־$f$ אינה חסומה. אז לכל $n \in \N$ קיים $x_n \in A$ כך ש־$\sof{f(x_n)} > n$. $A$ [הערה: $x_n$ מוגדרת היטב כי קיים יחס סדר טוב על הטבעיים] חסומה ולכן $x_n$ חסומה. יש לה ת''ס $x_{n_k}$ מתכנסת. נסמן את גבולה $x_0$. $A$ סגורה ולכן $x_0 \in A$ (סגירות סדרתית). לכן $\lim_{k \to \inft} x_{n_k} = x_0 \in A$ ומרציפות $\lim_{k \to \inft} f(x_{n_k}) = f(x_0) \in \R$ בסתירה לכך ש־$\limsi \sof{f(x_n)} = \inft$. לכן $f$ חסומה. 
	\end{proof}
	\begin{proof}[חלק שני]
		ידוע $f$ חסומה ולכן ניתן לסמן $f = \sup f(A)$. לכל $n \in \N$ קיים $y_n \in f(A)$ כך ש־$M - \frac{1}{n} \le y_n \le M$. לכל $n \in \N$ קיים $x_n \in A$ כך ש־$f(x_n) = y_n$. $A$ חסומה ולכן $x_n$ חסומה ומכאן שקיימת $x_{n_k}$ מ־BW שמתכנסת. נסמן גבולה $x_0$. $A$ סגורה ולכן $x_0 \in A$. מכאן ש־$M = \lim_{k \to \infty} y_{n_k} = \lim_{k \to \inft} f(x_{n_k}) = f(x_0)$. בדומה בעבור $\inf (f(A))$. 
	\end{proof}
	\rmark{בד''כ יציינו את זה על קטע סגור, שזה מקרה פרטי של קבוצה קומפקטית. צריך רק קומפקטיות – השתמשנו גם בכל התכונות, הסגירות והחסימות. }
	
	\theo{תהא $f \co I \to \R$ המקיימת תכונת דרבו. אז ל־$f$ אין נקודות אי־רציפות סליקות או מסוג ראשון. }\begin{proof}
		תהא $x_0 \in I$. נניח שקיים $\lim_{x \to x_0^{-}} f(x)$ וסופי. נסמנו $\ml$. נוכיח ש־$\ml = f(x_0)$. נניח בשלילה ש־$\ml < f(x)$ (כנ''ל לגבי גדול, בה''כ). קיים $\dg > 0$ כך שלכל $x \in I$ אם $x_0 - \ml < x < x_0$ אז $\sof{f(x) - \ml} < \frac{f(x_0) - \ml}{2}$. בקטע $[x_0 - \frac{\dg}{2}, x_0]$, מתקיים: 
		\[ f\cl{x_0 - \frac{\dg}{2}} < \frac{\ml + f(x_0)}{2} < f(x_0) \]
		מתכונת דרבו קיים $x_0 - \frac{\dg}{2} < y < x_0$ כך ש־$f(y) = \frac{\ml + f(x_0)}{2}$. כלומר $\sof{f(y) - \ml} \ge \frac{f(x_0) - \ml}{2}$ בסתירה. לכן $f(x_0) \le \ml$. באופן דומה $f(x_0) \ge \ml$. לכן $f(x_) = \ml$. באופן דומה, אם קיים וסופי הגבול $\lim_{x \to x_0^{+}} f(x) =: m$ אז $f(x) = m$. 
		
		מכאן שלא קיימות נקודות אי־רציפות סליקות ומסוג ראשון. 
	\end{proof}
	
	\cola{תהא $f \co I \to \R$. אם $f$ מקיימת תכונת דרבו ומונוטונית, היא בהכרח רציפה. }\begin{proof}
		תהא $f$ מונוטונית המקיימת את תכונת דרבו, מהמשפט הקודם אין לא נקודות אי־רציפות סליקות או מסוג ראשון. משום ש־$f$ מונוטונית, אין לה נקודות אי־רציפות מסוג שני (משפט קודם). מכאן של־$f$ אין נקודות אי־רציפות ולכן היא רציפה. 
	\end{proof}
	
	\rmark{עקרונית אפשר להגדיר את תכונת דרבו בעבור $A$ פתוחה ולהגדירה כך שכל קטע פתוח $I \subseteq A$ מקיים את דרבו כפי שהגדרנו אותה. }
	
	אם ננסה להוכיח את הרציפות של $\frac{1}{x}$, נצטרך לבחור $\dg = \min\{\frac{x_0}{2}, \frac{\eg x_0^{2}}{2}\}$
	\[ \sof{\frac{1}{x} - \frac{1}{x_0}} = \frac{\sof{x - x_0}}{x_0x} < \frac{\dg}{xx_0} < \frac{\dg}{x_0(x_0 - \dg)} < \frac{2\dg}{x_0^{2}} = \eg \]
	
	מאוד ברור שה־$\dg$ תלוי באיזה $x_0$ אנחנו בוחרים. זה גם ניכר מההגדרה של רציפות: ''לכל $x \in A$, ולכל $\eg > 0$, קיים $\dg > 0$ כך שלכל $y \in A$ אם לכל $\sof{y - x} < \dg$ אז $\sof{f(x) - f(y)} < \eg$``. 
	
	כאשר אנו אומרים ''במידה שווה``, הכוונה היא שה־$\dg$ לא תלוי בנקודה. דהיינו: 
	\defi{$f$ \textit{רציפה במידה שווה} אם לכל $\eg > 0$ קיים $\dg > 0$ כך שלכל $x, y \in A$ אם $\sof{x - y} < \dg$ אז $\sof{f(x) - f(y)} < \eg$. }
	
	''$\frac{1}{x}$ היא לא סימפטית`` – המרצה (לא פיזיקאי מוסמך). 
	
	\theo{אם $f$ רציפה במידה שווה ב־$A$ אז $f$ רציפה ב־$A$. }\begin{proof}
		כאילו דה
	\end{proof}
	אינטואציה: נדבר על זה בהמשך, אבל נגזרת חסומה אומר שהפונקציה רציפה במידה שווה. 
	
	לדוגמה, נראה ש־$f(x) = x^{2}$ אינה רציפה במידה שווה ב־$\R$, אך רציפה במידה שווה לכל קטע חסום ב־$\R$. 
	\begin{proof}
		יהי $M > 0$. נגדיר $f \co [-M, M] \to \R$ ע''י $f(x) = x^{2}$ לכל $x \in [-M, M]$. יהי $\eg > 0$. נבחר $\dg = \frac{\eg}{2M}$. יהיו $x, y \in [-M, M]$ ונניח $\sof{x - y} < \dg$. אז: 
		
		\[ \sof{x^{2} - y^{2}} = \sof{x - y}\sof{x + y} \le 2M\sof{x - y} < 2M\dg = \eg \]
	\end{proof}
	לא סתם בחרנו $\dg$ להיות $\eg$ כפול נקודת המקסימום של הנגזרת, אבל לא מדברים על זה. 
	\begin{proof}
		עתה נראה ש־$x^{2}$ אינה רציפה במידה שווה ב־$\R$. נבחר $\eg = 1$ ויהי $\dg > 0$. נבחר $y = x + \frac{\dg}{2}$. נבחר $x = \frac{1}{\dg}$. מכאן $\sof{x - y} < \dg$. 
		\[ \sof{x^{2} - y^{2}} = \sof{x - y}\sof{x + y} = \frac{\dg}{2}\cl{2x + \frac{\dg}{2}} > \frac{\dg}{2} \cdot 2x = 1 = \eg \]
	\end{proof}
	תרגיל טוב הוא להוכיח ש־$\sin x^{2}$ אינה רציפה במידה שווה ב־$\R$. 
	
	\theo{תהאנה $f, g \co A \to \R$. נניח כי $f$ רציפה במידה שווה ב־$A$ וגם $g$ רציפה במידה שווה ב־$A$. אז: 
	\begin{itemize}
		\item $f\pm g$ רציהפ במידה שווה ב־$A$. 
		\item אם $f$ ו־$g$ חסומות ב־$A$, אז $fg$ רציפה במידה שווה. 
	\end{itemize}}
	\begin{Theorem}[משפט קנטור (עוד אחד)]
		תהא $f \co A \to \R$. אם $f$ רציפה ב־$A$ וגם $A$ קומפקטית, אז $f$ רציפה במידה שווה ב־$A$. 
	\end{Theorem}\begin{proof}
		נניח בשלילה ש־$f$ אינה רציפה במידה שווה ב־$A$. אז קיים $\eg_0 > 0$ כך שלכל $n \in \N$ קיימים $x_n, y_n \in A$ כך ש־$\sof{x_n - y_n} < \frac{1}{n}$ וגם $\sof{f(x_n) - f(y_n)} \ge \eg_0$. אז $\limsi x_n - y_n = 0$. $A$ חסומה ולכן $x_n$ חסומה. לכן מ־BW קיימת לה ת''ס מתכנסת $x_{n_k}$. נסמן גבולה $x_0$. $A$ סגורה ולכן $x_0 \in A$. ידוע ש־$\lim_{k \to \inft} x_{n_k} - y_{n_k} = 0$ שכן כל ת''ס של סדרה מתכנסת מתכנסת לאותו הגבול. לכן מאריתמטיקה $\lim_{k \to \inft} y_{n_k} = x_0$. מהרציפות $\lim_{k \to \inft} f(x_{n_k}) = \lim_{k \to \inft} f(y_{n_k}) = f(x_0)$ בסתירה לכך ש־$\sof{f(x_{n_k}) - f(y_{n_k})} \ge \eg_0$ לכל $k \in \N$. לכן $f$ רציפה במידה שווה ב־$A$. 
	\end{proof}
	
	\theo{יהיו $a, b \in \R\cup\{\pm\inft\}$. נניח $a < b$. יהי $a < c < b$. תהא $f\co (a, b) \to \R$ ונניח $f$ רציפה במידה שווה ב־$(a, c)$ וכן $f$ רציפה במידה שווה ב־$(c, b]$, אז $f$ רציפה במידה שווה ב־$(a, b)$. }\begin{proof}
		יהי $\eg > 0$. ידוע ש־$f$ רב''ש ב־$(a, c]$ ולכן קיים $\dg_1 > 0$ כך ש־$\forall x, y \in (a, c]$ אם $\sof{x - y} < \dg_1$ אז $\sof{f(x) - f(y)} < \frac{\eg}{2}$. $f$ רציפה במ''ש ב־$[c, b)$ לכן קיים $\dg_2 > 0$ כך שלכל $x, y \in [c, b)$, אם $\sof{x - y}$ אז $\sof{f(x) - f(y)} < \frac{\eg}{2}$. נתבונן ב־$\dg = \min\{\dg_1, \dg_2\}$. יהיו $x, y \in (a, b)$. נניח $\sof{x - y} < \dg$. נפרק למקרים. 
		\begin{itemize}
			\item אם $x, y \ge c$ אז מכיוון ש־$\sof{x - y} < \dg \le \dg_2$ נובע ב־$\sof{f(x) - f(y)} < \frac{\eg}{2} < \eg$. 
			\item אם $x, y \le c$ אז מכיוון ש־$\sof{x - y} < \dg \le \dg_1$ נובע ב־$\sof{f(x) - f(y)} < \frac{\eg}{2} < \eg$. 
			\item אם $x \le c \le y$ אז $\sof{c - x} < \sof{y - x} < \dg_1$ ולכן $\sof{f(c) - f(x)} < \frac{\eg}{2}$. אז $\sof{y - c} < \sof{y - x} < \dg_2$ ולכן $\sof{f(c) - f(y)} < \frac{\eg}{2}$. ניעזר בא''ש במשולש: 
			\[ \sof{f(x) - f(y)} \le \sof{f(x) - f(c)} + \sof{f(c) - f(y)} = \frac{\eg}{2} + \frac{\eg}{2} < \eg \]
			\item נניח $y \le c \le x$. בדומה. 
		\end{itemize}\envendproof
	\end{proof}
	
	''אתה לא רוצה לשדר זלזול. מקרה 4 בדומה``. 
	
	\exe{נניח ש־$f$ רציפה במידה שווה ב־$(a, c)$ ו־$(b, c)$, ורציפה ב־$c$ (כאשר $a < c < b$). נוכיח ש־$f$ רציפה במידה שווה ב־$(a, b)$. }
	
	\theo{הפונקציה $\sqrt x$ רציפה במ''ש בקטע $[0, \infty)$. }\begin{proof}
		יהי $\eg > 0$. 
		\begin{itemize}
			\item יהיו $x, y \in [1, \infty)$. נניח $\sof{x - y} < \dg$ עבור $\dg = \eg$ ונקבל: 
			\[ \sof{\sqrt x - \sqrt y} = \frac{\sof{x - y}}{\sqrt x + \sqrt y} < \frac{\dg}{\sqrt x + \sqrt y} \le \dg = \eg \]
			\item בקטע $[0, 1]$ נקבל ש־$\sqrt x$ רציפה ומשום שהקטע חסום היא רציפה במידה שווה לפי קנטור. 
		\end{itemize}
		משום ש־$\sqrt x$ רציפה במ''ש ב־$[0, 1]$ ו־$(1, \infty)$ סה''כ מהמשפט הקודם היא רציפה במ''ש. 
	\end{proof}
	\theo{תהא $f \co [a, \infty) \to \R$. נניח $f$ רציפה וגם קיים וסופי $\lim_{x \to \inft}f(x)$. הראו כי $f$ רציפה במ''ש ב־$[a, \infty)$. }\begin{proof}
		ויהי $\eg > 0$. אז קיים $M > 0$ כך שלכל $x, y > M$ מתקיים $\sof{f(x) - f(y)} < \frac{\eg}{2}$ (קושי). הקטע $[a, M]$ הוא קטע קומפקטי, ומשום ש־$f$ רציפה בו ולפי קנטור $f$ רציפה בו במידה שווה. לכן קיים $\dg > 0$ כך שלכל $x, y \in [a, M]$ אם $\sof{x - y} < \dg$ אז $\sof{f(x) - f()} < \frac{\eg}{2}$. נתבונן ב־$\dg$. יהיו $x, y \in [a, \infty)$. נניח בה''כ $x \le y$. נפרק למקרים. 
		\begin{itemize}
			\item נניח $x \le y \le M$, מכיוון ש־$\sof{x - y} < \dg$ נובע ש־$\sof(f(x) - f(y)) < \frac{\eg}{2} < \eg$. 
			\item נניח $M \le x \le y$, נובע ש־$\sof(f(x) - f(y)) < \frac{\eg}{2} < \eg$. 
			\item אם $x \le M \le y$ מא''ש המשולש: 
			\[ \sof{f(x) - f(y)} = \sof{f(x) - f(M)} + \sof{f(M) - f(y)} < \frac{\eg}{2} + \frac{\eg}{2} = \eg \]
			לכן $f$ רציפה במידה שווה ב־$[a, \infty)$. 
		\end{itemize}
	\end{proof}
	\rmark{לא היה עובד להשתמש במשפט של האיחוד קטעים כאן – כי $M$ תלוי ב־$\eg$. }
	\rmark{זה לא אמ''מ. לדוגמה $\sqrt x$ או $x$. }
	
	\theo{יהי $a, b \in \R$ ונניח $a < b$. תהא $f \co (a, b) \to \R$ רציפה. אז $f$ רציפה במידה שווה ב־$(a, b)$ אמ''מ קיימים ל־$f$ הגבולות ב־$a$ וב־$b$ והסם סופיים. }\begin{proof}
		\begin{itemize}
			\item[$\implies$]נסמן $\ml = \lim_{x \to a^{+}} f(x)$ ו־$m = \lim_{x \to b^{-}}$. נגדיר: 
			\[ F \co [a, b] \to \R \quad F(x) = \begin{cases}
				\ml & x = a \\
				f(x) & x \in (a, b) \\
				m & x = b
			\end{cases} \]
			נבחין ש־$F$ רציפה ב־$[a, b]$ ולפי קנטור, $F$ רציפה במידה שווה ב־$[a, b]$. לכן $f = F|_{(a, b)}$ רציפה במידה שווה ב־$(a, b)$. 
			\item[$\impliedby$]רוצים להוכיח שקיים גבול סופי ואין לנו מושג מה הוא. כלומר זה כנראה קושי. נניח כי $f$ רציפה במ''ש ב־$(a, b)$. יהי $\eg > 0$ וידוע קיו ם$\dg > 0$ כך שלכל $x, y \in (a, b)$ אם $\sof{x - y} < \dg$ אז $\sof{f(x) - f(y)} < \eg$. נתבונן ב$\dg$. יהיו $x, y \in (a, a + \dg)$. אז $\sof{x - y} < \dg$ ולכן $\sof{f(x) - f(y)} < \eg$. לפי קריטריון קושי יש ל־$f$ גבול סופי ב־$a$ מימין. באופן דומה יש ל־$f$ גבול סופי משמאל ב־$b$. 
		\end{itemize}\envendproof
	\end{proof}
	
	
	
	
\end{document}