\documentclass[]{../../../../tex/classes/homework}
\usepackage{../../../../tex/packages/hebrewSupport}
\usepackage{../../../../tex/packages/mathShortcuts}
\input pdfmsym

\author{שחר פרץ}
\title{הרצאה/תרגול על בסיס שאלות מהפתוחה. }
\begin{document}
	\maketitle
	\textbf{מרצה/מתרגל: בן בסקין}
	
	התרגול יבוסס על פתרון שאלות מהעברית. 
	\section{}
	נתבונן בקבוצה הבאה: 
	\[ \ccb{\frac{x}{\floor{x\floor{x}}} \mid x \in [1, \inft)} \]
	הוכיחו שהיא חסומה. האם יש לה מינימום או מקסימום? 
	אפס הוא חסם מלרע, והיא קבוצה חיובית. עוד נבחין שהקבוצה מכילה את הקבוצה $\frac{1}{n}$. מכאן ש־$0$ האיניפימום ומשום שהוא אינו בקבוצה אין מינימום. נתחיל מזה. משום שהקבוצה חיובית $0$ חסם מלרע. נוכיח מקסימלי: יהי $\eg > 0$ ומארכימדיאניות $\eg > \frac{1}{n}$ עבור $n$ כלשהו וסיימנו. נבחין: 
	\begin{alignat*}{9}
		a - 1 &< \floor a && \le a \\ 
		x \floor x - 1 & < \floor{x \floor x} &&\le x \floor x \\
		x - 1 &< \floor x &&\le x \\
		x^{2} - x - 1 &< \floor{x - \floor x} &&\le x^{2} \\
		\frac{1}{x^{2}} &\le \frac{1}{\floor{x \floor x}} &&< \frac{1}{x^{2} - x - 1}
	\end{alignat*}
	סה''כ נדרוש $\floor{\frac{x}{\floor{x \floor x}}} < \frac{x}{x^{2} - x - 1}$. זה כנראה חסם גס מדי אז נעשה משהו אחר. בן מוחק את זה מהלוח. 
	\[ a \le \floor x \implies a^{2} \le ax \le \floor x x \implies a^{2} \le \floor{x \floor x} \implies \frac{x}{\floor {x \floor x}} < \frac{a + 1}{a^{2}} \]
	
	סה''כ מסתבר שהפתרון הרבה יותר פשוט מכל הטיוטה הזו. ממונוטוניות $\floor {\cdots}$ ועוד הסברים: 
	\[ \frac{x}{\floor{x\floor x}} \le \frac{x}{\floor x} \]
	את הביטוי מימין הרבה יותר קל לנתח. עבור $x \ge 2$ נקבל: $\frac{x}{\floor x} \le \frac{x}{x - 1} < 2$. אינטואיטיבית זה נכון ($\frac{3}{2}, \frac{4}{3}, \frac{5}{6}$) וכו'. פורמלית $\frac{x}{x - 1} < 2 \iff 2 < 2x - 2 \iff x > 2$ והנחנו $x > 2$. 
	
	עבור $1 \le x < 2$ מתקיים $\floor x = 1$ ואז $\frac{x}{\floor x} = x < 2$. 
	
	עתה הראינו ש־$\frac{x}{\floor x}$ חוסם מלמעלה את כל הסיפור, ולכן $2$ חסם מלעיל. אך האם הוא חסם עליון/סופרמום? כן – כי $\frac{x}{\floor x}$ מוכל בקבוצה. פורמלית לכל $\eg > 0$ נבחר $\eg' = \min \{\eg, \frac{1}{2}\}$ ואז אפשר להראות ש־$x = 2 - \frac{\eg'}{2}$. 
	
	(לא ממש טיפלנו במקרה בו $x = 2$ אבל עזבו). 
	
	מסקנות: גועל נפש והכל, אבל ברגע שמציבים קצת ערכים שלמים דברים הפכו להגיוניים. בתוך קטע בין מספרים שלמים יש מונוטוניות ובמספרים השלמים עצמם קפיצות. כל זה אפשר להבין מהצבות ומלבהות בפונקציה. 
	
	משום שהראינו שכל מספר קטן ממש מ־$2$ אז אין מקסימום. 
	
	\section{}
	תהא $(a_n)_{n = 1}^{\inft}$ ממשית המקיימת $a_1 = 0$ וכן $\forall n \in \N \co a_{n + 1} = \frac{1}{2 - a_n}$. 
	\begin{enumerate}[A.]
		\item הוכיחו שהסדרה מוגדרת היטב. 
		\begin{proof}
			נציב קצת ערכים ונקלוט די מהר שזה פשוט $\frac{n - 1}{n}$. אבל כרגע נפתור את זה בגישה של אין לנו מושג איך נראה האיבר הכללי בגלל נקודת מבט חינוכית של בן. באינדוקציה אפשר להראות ש־$0 \le a_n < 1$. זה מוכיח מוגדרות היטב שכן $2 - a_n \neq 0$ תמיד. עוד נבחין: 
			\[ 2 \ge 2 - a_n > 1 \implies 0 \le \frac{1}{2} \le \frac{1}{2 - a_n} < 1 \]
			כלומר זה חסום. זה יועיל בשאלה הבאה. 
		\end{proof}
		\item הוכיחו שהיא מתכנסת ומצאו את גבולה. \begin{proof}
			עכשיו אפשר להראות שזה מונוטוני עולה. נרצה לעשות וויראשטראס כלומר להראות מונוטוניות ולהגיד שהיא חסומה ומכאן שהיא מתכנסת. 
			\[ 0 < a_n < a_{n + 1} \iff 2 - a_n > 2 - a_{n + 1} > 0 \iff a_{n + 1} < a_{n + 1} \]
			זה מוכיח את הצעד של האינדוקציה. עכשיו כשאנו יודעים שזה מתכנס אפשר לפתור משוואה ריבועית כמו שראינו בתרגול וסיימנו (השורה התחתונה $L = \frac{1}{2 - L}$ ומראים $L = 1$). 
		\end{proof}
	\end{enumerate}
	
	\section{}
	תהא $f \co \R \to \R$ מונוטונית, כך ש־$\Q \subseteq \Img f$. הוכיחו שהיא רציפה. 
	\begin{proof}
		פתרון אחד הוא לעבוד לפי הגדרה. פתרון אחד הוא להשתמש שכל נקודת אי רציפות היא מסוג ראשון. מכאן ההוכחה פשוטה: יהי $x_0 \in \R$ נקודת אי רציפות, מכאן שהיא נקודת אי רציפות מסוג ראשון ולכן קיים לה $L_1$ גבול מימין ו־$L_2$ משמאל, ובינהם יש לפחות שני מספרים רציונליים, שאפילו אם $f(x_0)$ יהיה אחד מהם – לא נוכל לקבל את שניהם וסתירה. 
		
		אבל בן רוצה מטעמים חינוכיים להוכיח לפי הגדרה. יהי $x_0 \in \R$. יהי $\eg > 0$. בה''כ $f$ עולה. מצפיפות הרציונליים בממשיים קיימים $q_1, q_2 \in \Q$ כך ש־$f(x_0) - \eg < q_1 < f(x_0) < q_2 < f(x_0) + \eg$. יהיו $x_1, x_2 \in \R$ כך ש־$f(x_1) = q_1 \land f(x_2) = q_2$ שקיימים מהנתון $\Q\subseteq \Img \F$. ממונוטוניות $x_1 < x_2$. נגדיר $\dg = \min \{x_0 - x_1, x_2 - x_0\}$. יהא $x' \in (x_0 - \dg, x_0 + \dg) \setminus \{x_0\}$. נתבונן ב־$\sof{f(x') - f(x_0)}$. עבור $x' > x_0$, ממונוטוניות: 
		\[ 0 < \sof{f(x') - f(x_0)} = f(x') - f(x_0) < f(x_2) - f(0) < \eg \]
		המקרה השני בדומה וסיימנו. 
	\end{proof}
	בכל מקרה בפועל היינו צריכים רק רציונליים. 
	\section{}
	תהי $f \co [0, 1] \to \R$ גזירה המקיימת $\forall x \in [0, 1] \co 0 \le f'(x) \le 1$. הוכיחו שקיימת $c \in [0, 1]$ כך ש־$f'(c) = c$. (נגזרת בקצוות הקטע: משמעה גבול חד צדדי). 
	
	\begin{proof}
		נתבונן ב־$h(x) = f'(x) - x$ (נבחין $g \co [0, 1] \to \R$). נרצה לטעון שהיא מקיימת את ערך הביניים, אך לא ידוע (או נכון) שחיבור פונקציות שמקיימות את את תכונת ערך הביניים, מקיימת את ערך הביניים. נגדיר את $g(x) = f(x) - \frac{x^{2}}{2}$. נבחין $g'(x) = h(x)$ ומדרבו $f$ מקיימת את ערך הביניים. ונבחין ש־$f(0) > 0$ וכן $f(1) < 0$ ולכן סיימנו מערך הביניים. 
	\end{proof}
	
	נתבונן בינתיים בתרגיל עזר: תהא $f \co \R \to \R$ רציפה ומחזורית עם מחזור $T > 0$ ($\forall x \in \R\co f(x) = f(x + T)$). הוכיחו שקיימת $c \in \R$ כך ש־$f(c) = f(c + \frac{T}{2})$ (אפשר להכליל את זה לכל כפולה רציונלית ואז גם ממשית של המחזור, מה שבפועל אומר שלכל $\ag \in \R$ קיים $c \in \R$ כך ש־$f(c)= f(\ag)$). 
	
	למה? כי $f(x + \frac{T}{2})$ רציפה, ולכן גם $g(x) = f(x) - f\cl{x + \frac{T}{2}}$. נבחין $g(0) = f(0) - f\cl{\frac{T}{2}}$. עוד נבחין $g\cl{\frac{T}{2}} = f\cl{\frac{T}{2}} -f(T) = f\cl{\frac{T}{2}} - f(0) = -g(0)$. סה''כ מרציפות $g(c) = 0$ עבור $c$ כלשהו מערך הביניים. 
	
	\section{}
	יהיו $a, b \in \R$ וכן $a < b$. כמו כן תהא $f \co [a, b] \to \R$, המקיימת שלכל $x \in [a, b]$ קיים הגבול הנקודתי של $f$ ב־$x$. 
	
	הוכח/הפרך ש־$f$ חסומה ב־$[a, b]$. 
	
	\begin{proof}
		בה''כ נניח בשלילה ש־$f$ לא חסומה מלעיל (אחרת נסתכל על $-f$). לכן לכל $n \in \N$ קיים $x_n \in [a, b]$ שעבורו $f(x_n) > n$. בפרט $x_n \to \inft$. הסדרה $f(x_n) \subseteq [a, b]$ ובפרט חסומה. לכן מ־BW קיימת לה ת''ס $x_{n_k}$ המתכנסת בקטע (כאן משתמשים בקומפקטיות $[a, b]$ – זה לא טרוויאלי שההתכנסות היא בקטע). נסמן את גבולה ב־$c$. 
		
		מצד אחד, קיים הגבול במובן הצר $\ml := \lim_{x \to c} f(x)$. לכן לכל סדרה $(y_n)_{n = 1}^{\inft}$ המתכנסת ל־$c$ (ושונה ממנה כמעט תמיד וכל מה שצריך בהיינה) מתקיים $\limsi f(y_n) = \ml$. נבחין ש־$x_n$ כמעט תמיד שונה מ־$c$ (כי קיים $n \in \N$ כך ש־$f(c) < n$, ומכאן $x_{n_k}$ שונה תמיד מ־$c$ כת''ס שלה). בפרט בעבור $y_n = x_{n_k}$, ומכאן גבול $f(x_{n_k}) \to \ml \in \R$ התכנסות במובן הצר. אך $\lim_{k \to \inft} f(x_{n_k}) = \inft$ כת''ס של סדרה $f(x_n)$ השואפת לאינסוף (כי $x_n$ שואף לאינסוף ו־$f(x)$ לא חסומה). סתירה. 
	\end{proof}
	
	\section{}
	תהי $f \co [0, \infty) \to \R$  רציפה. נניח $f(0) = 1$. בנוסף $\forall x \in [0, \infty) \co f(x) \le \frac{x + 2}{x + 1}$. 
	
	לפני שנפתור אותה, נציג ווראיציה של השאלה הזו: 
	(''אני פתרתי אותה בתיכון כשאני למדתי חדו''א``) תהי $f \co [0, \inft) \to \R$ רציפה. נניח $f(0) = 1$. 
	\begin{enumerate}[A.]
		\item אם $\limsi f(x) = 0$. הוכיחו שהיא מקבלת מקסימום. זו שאלה קלאסית. פתרנו אותה בתרגולים/הרצאות. \begin{proof}
			נניח $\limsi f(x) = 0$. אז קיים $N \in \R$ כך שלכל $x > N$ מתקיים $f(x) < \frac{1}{2}$. $f$ רציפה ולכן רציפה על $[0, N]$. בפרט ממשפט וויראשטראס השני קיים $c \in [0, N]$ שעבורו $f(c)$ מקסימלית. נסמן $f(c) = M$. ידוע $f(0) = 1$ ולכן $M \ge 1$. עתה נוכיח ש־$M$ הוא המקסימום. יהי $x \in \R$. ואז מפלגים למקרים וסיימנו. 
		\end{proof}
		\item אם $\limsi f(x) = 1$. הוכיחו שהיא מקבלת מקסימום. יחסית דומה לג'. 
		\item השאלה שראינו קודם. בן נתן אינטואציה שאני לא אקליד. 
	\end{enumerate}
	
	
	\section{}
	תהא $f \co \R \to \R$. יהא $\ag \in \R \setminus \{0, \pm1\}$, כך שלכל $x \in \R$ מתקיים $f(x) = f(\ag x)$. נתון ש־$f$ רציפה ב־$f$. תוכיחו ש־$f$ קבועה. 
	
	\begin{proof}
		בה''כ $\ag > 0$ כי אחרת נסתכל על $\ag^{2}$. בה''כ $\ag \in (0, 1)$ אחרת נסתכל על $\frac{1}{\ag}$. יהא $x \in \R$. נבנה את הסדרה $a_0 = x$ וכן $a_n = \ag^{n}x$. נבחין באינדוקציה שלכל $n \in \N$ מתקיים $f(a_n) = f(x)$. מכאן ש־$f(a_n)$ סדרה קבועה. לכן $f(x) = \limsi f(a_n)$. ידוע $\limsi a_n = 0$ שכן זו סדרה הנדסית מתכנסת. עתה אפשר להשתמש בקירטריון היינה ולקבל: $f(x) = \limsi f(a_n) = f\cl{\limsi a_n} = f(0)$ כלומר $f(x)$ קבועה ב־$f(0)$ כנדרש. 
	\end{proof}
	
	\section{}
	תהא $a_n$ סדרה המקיימת שקיים $M \in \R$ כך ש־$\forall n \in \N \co \sumnko \sof{a_{k + 1} - a_k} < M$. הוכיחו ש־$a_n$ מתכנסת. 
	\begin{proof}
		נגדיר את $x_n = \sum_{i = 1}^{n} \sof{a_{k + 1} - a_k}$. זוהי סדרה מונוטונית (טור חיובי) עולה וחסומה (ב־$M$) ולכן מתכנסת, על כן קושי. בפרט קיים $N$ כך שעבור $n > m > N$ מתקיים $\sof{x_n - x_m} < \eg$, מקיים: 
		\[ \sof{x_n - x_m} = x_n- x_m = \sumnko \sof{a_{k + 1} -a_k} - \sum_{i = 1}^{m}\sof{a_{k + 1} - a_k} = \suum_{i = m + 1}^{n}\sof{a_{k + 1} - a_k} \ge \sof{\suum_{k = m + 1}^{n}a_{k + 1}  -a_k} = \sof{a_{n + 1} - a_n} \]
	\end{proof}
	\section{}
	תהא $a_n$ חסומה. נסמן ב־$S_a$ את קבוצת הגבולות החלקיים שלה. תהא $b_n$ סדרה שמתכנסת ל־$1$. נסמן ב־$S_{ab}$ את קבוצת הגבולות החלקיים של $(a_nb_n)$. הוכיחו: $S_a = S_{ab}$. 
	
	\section{}
	תהא: 
	\[ a_n = \begin{cases}
		2^{-\frac{1}{n} + (-1)^{n}} & \exists k \in \N, n = k^{2} \\
		n - \sqrt n & \other
	\end{cases} \]
	מצאו את קבוצת הגבולות החלקיים של $a_n$ הוא הוכיחו שאין. 
	
	\section{}
	נתבונן בסדרה: 
	\[ a_n = 1, 1, \frac{1}{2}, 1, \frac{1}{2}, \frac{1}{3}, 1, \frac{1}{2}, \frac{1}{3}, \frac{1}{4}, 1, \cdots \]
	מהי קבוצת הגבולות החלקיים שלה? אם יש $L$ גבול שאינו מהצורה $\frac{1}{n}$ והוא אינו $0$ אז: 
	\[ n < \frac{1}{L} < n + 1 \implies \frac{1}{n + 1} < L < \frac{1}{n} \]
	אז אפשר לקחת $\eg$ שהוא ה־distance בינהם. השטיק הוא ש־$\frac{1}{n}$־ים מרווחים ודיסקרטיים ולכן יש רווחים בינהם. 
	
	
	
	
	
\end{document}