%! ~~~ Packages Setup ~~~ 
\documentclass[]{article}
\usepackage{lipsum}
\usepackage{rotating}


% Math packages
\usepackage[usenames]{color}
\usepackage{forest}
\usepackage{ifxetex,ifluatex,amssymb,amsmath,mathrsfs,amsthm,witharrows,mathtools,mathdots,centernot}
\usepackage{amsmath}
\WithArrowsOptions{displaystyle}
\renewcommand{\qedsymbol}{$\blacksquare$} % end proofs with \blacksquare. Overwrites the defualts. 
\usepackage{cancel,bm}
\usepackage[thinc]{esdiff}


% tikz
\usepackage{tikz}
\usetikzlibrary{graphs}
\newcommand\sqw{1}
\newcommand\squ[4][1]{\fill[#4] (#2*\sqw,#3*\sqw) rectangle +(#1*\sqw,#1*\sqw);}


% code 
\usepackage{algorithm2e}
\usepackage{listings}
\usepackage{xcolor}

\definecolor{codegreen}{rgb}{0,0.35,0}
\definecolor{codegray}{rgb}{0.5,0.5,0.5}
\definecolor{codenumber}{rgb}{0.1,0.3,0.5}
\definecolor{codeblue}{rgb}{0,0,0.5}
\definecolor{codered}{rgb}{0.5,0.03,0.02}
\definecolor{codegray}{rgb}{0.96,0.96,0.96}

\lstdefinestyle{pythonstylesheet}{
	language=Java,
	emphstyle=\color{deepred},
	backgroundcolor=\color{codegray},
	keywordstyle=\color{deepblue}\bfseries\itshape,
	numberstyle=\scriptsize\color{codenumber},
	basicstyle=\ttfamily\footnotesize,
	commentstyle=\color{codegreen}\itshape,
	breakatwhitespace=false, 
	breaklines=true, 
	captionpos=b, 
	keepspaces=true, 
	numbers=left, 
	numbersep=5pt, 
	showspaces=false,                
	showstringspaces=false,
	showtabs=false, 
	tabsize=4, 
	morekeywords={as,assert,nonlocal,with,yield,self,True,False,None,AssertionError,ValueError,in,else},              % Add keywords here
	keywordstyle=\color{codeblue},
	emph={var, List, Iterable, Iterator},          % Custom highlighting
	emphstyle=\color{codered},
	stringstyle=\color{codegreen},
	showstringspaces=false,
	abovecaptionskip=0pt,belowcaptionskip =0pt,
	framextopmargin=-\topsep, 
}
\newcommand\pythonstyle{\lstset{pythonstylesheet}}
\newcommand\pyl[1]     {{\lstinline!#1!}}
\lstset{style=pythonstylesheet}

\usepackage[style=1,skipbelow=\topskip,skipabove=\topskip,framemethod=TikZ]{mdframed}
\definecolor{bggray}{rgb}{0.85, 0.85, 0.85}
\mdfsetup{leftmargin=0pt,rightmargin=0pt,innerleftmargin=15pt,backgroundcolor=codegray,middlelinewidth=0.5pt,skipabove=5pt,skipbelow=0pt,middlelinecolor=black,roundcorner=5}
\BeforeBeginEnvironment{lstlisting}{\begin{mdframed}\vspace{-0.4em}}
	\AfterEndEnvironment{lstlisting}{\vspace{-0.8em}\end{mdframed}}


% Design
\usepackage[labelfont=bf]{caption}
\usepackage[margin=0.6in]{geometry}
\usepackage{multicol}
\usepackage[skip=4pt, indent=0pt]{parskip}
\usepackage[normalem]{ulem}
\forestset{default}
\renewcommand\labelitemi{$\bullet$}
\usepackage{titlesec}
\titleformat{\section}[block]
{\fontsize{15}{15}}
{\sen \dotfill (\thesection)\she}
{0em}
{\MakeUppercase}
\usepackage{graphicx}
\graphicspath{ {./} }

\usepackage[colorlinks]{hyperref}
\definecolor{mgreen}{RGB}{25, 160, 50}
\definecolor{mblue}{RGB}{30, 60, 200}
\usepackage{hyperref}
\hypersetup{
	colorlinks=true,
	citecolor=mgreen,
	linkcolor=black,
	urlcolor=mblue,
	pdftitle={Document by Shahar Perets},
	%	pdfpagemode=FullScreen,
}
\usepackage{yfonts}
\def\gothstart#1{\noindent\smash{\lower3ex\hbox{\llap{\Huge\gothfamily#1}}}
	\parshape=3 3.1em \dimexpr\hsize-3.4em 3.4em \dimexpr\hsize-3.4em 0pt \hsize}
\def\frakstart#1{\noindent\smash{\lower3ex\hbox{\llap{\Huge\frakfamily#1}}}
	\parshape=3 1.5em \dimexpr\hsize-1.5em 2em \dimexpr\hsize-2em 0pt \hsize}



% Hebrew initialzing
\usepackage[bidi=basic]{babel}
\PassOptionsToPackage{no-math}{fontspec}
\babelprovide[main, import, Alph=letters]{hebrew}
\babelprovide[import]{english}
\babelfont[hebrew]{rm}{David CLM}
\babelfont[hebrew]{sf}{David CLM}
%\babelfont[english]{tt}{Monaspace Xenon}
\usepackage[shortlabels]{enumitem}
\newlist{hebenum}{enumerate}{1}

% Language Shortcuts
\newcommand\en[1] {\begin{otherlanguage}{english}#1\end{otherlanguage}}
\newcommand\he[1] {\she#1\sen}
\newcommand\sen   {\begin{otherlanguage}{english}}
	\newcommand\she   {\end{otherlanguage}}
\newcommand\del   {$ \!\! $}

\newcommand\npage {\vfil {\hfil \textbf{\textit{המשך בעמוד הבא}}} \hfil \vfil \pagebreak}
\newcommand\ndoc  {\dotfill \\ \vfil {\begin{center}
			{\textbf{\textit{שחר פרץ, 2025}} \\
				\scriptsize \textit{קומפל ב־}\en{\LaTeX}\,\textit{ ונוצר באמצעות תוכנה חופשית בלבד}}
	\end{center}} \vfil	}

\newcommand{\rn}[1]{
	\textup{\uppercase\expandafter{\romannumeral#1}}
}

\makeatletter
\newcommand{\skipitems}[1]{
	\addtocounter{\@enumctr}{#1}
}
\makeatother

%! ~~~ Math shortcuts ~~~

% Letters shortcuts
\newcommand\N     {\mathbb{N}}
\newcommand\Z     {\mathbb{Z}}
\newcommand\R     {\mathbb{R}}
\newcommand\Q     {\mathbb{Q}}
\newcommand\C     {\mathbb{C}}
\newcommand\One   {\mathit{1}}

\newcommand\ml    {\ell}
\newcommand\mj    {\jmath}
\newcommand\mi    {\imath}

\newcommand\powerset {\mathcal{P}}
\newcommand\ps    {\mathcal{P}}
\newcommand\pc    {\mathcal{P}}
\newcommand\ac    {\mathcal{A}}
\newcommand\bc    {\mathcal{B}}
\newcommand\cc    {\mathcal{C}}
\newcommand\dc    {\mathcal{D}}
\newcommand\ec    {\mathcal{E}}
\newcommand\fc    {\mathcal{F}}
\newcommand\nc    {\mathcal{N}}
\newcommand\vc    {\mathcal{V}} % Vance
\newcommand\sca   {\mathcal{S}} % \sc is already definded
\newcommand\rca   {\mathcal{R}} % \rc is already definded
\newcommand\zc    {\mathcal{Z}}

\newcommand\prm   {\mathrm{p}}
\newcommand\arm   {\mathrm{a}} % x86
\newcommand\brm   {\mathrm{b}}
\newcommand\crm   {\mathrm{c}}
\newcommand\drm   {\mathrm{d}}
\newcommand\erm   {\mathrm{e}}
\newcommand\frm   {\mathrm{f}}
\newcommand\nrm   {\mathrm{n}}
\newcommand\vrm   {\mathrm{v}}
\newcommand\srm   {\mathrm{s}}
\newcommand\rrm   {\mathrm{r}}

\newcommand\Si    {\Sigma}

% Logic & sets shorcuts
\newcommand\siff  {\longleftrightarrow}
\newcommand\ssiff {\leftrightarrow}
\newcommand\so    {\longrightarrow}
\newcommand\sso   {\rightarrow}

\newcommand\epsi  {\epsilon}
\newcommand\vepsi {\varepsilon}
\newcommand\vphi  {\varphi}
\newcommand\Neven {\N_{\mathrm{even}}}
\newcommand\Nodd  {\N_{\mathrm{odd }}}
\newcommand\Zeven {\Z_{\mathrm{even}}}
\newcommand\Zodd  {\Z_{\mathrm{odd }}}
\newcommand\Np    {\N_+}

% Text Shortcuts
\newcommand\open  {\big(}
\newcommand\qopen {\quad\big(}
\newcommand\close {\big)}
\newcommand\also  {\mathrm{, }}
\newcommand\defis {\mathrm{ definitions}}
\newcommand\given {\mathrm{given }}
\newcommand\case  {\mathrm{if }}
\newcommand\syx   {\mathrm{ syntax}}
\newcommand\rle   {\mathrm{ rule}}
\newcommand\other {\mathrm{else}}
\newcommand\set   {\ell et \text{ }}
\newcommand\ans   {\mathscr{A}\!\mathit{nswer}}

% Set theory shortcuts
\newcommand\ra    {\rangle}
\newcommand\la    {\langle}

\newcommand\oto   {\leftarrow}

\newcommand\QED   {\quad\quad\mathscr{Q.E.D.}\;\;\blacksquare}
\newcommand\QEF   {\quad\quad\mathscr{Q.E.F.}}
\newcommand\eQED  {\mathscr{Q.E.D.}\;\;\blacksquare}
\newcommand\eQEF  {\mathscr{Q.E.F.}}
\newcommand\jQED  {\mathscr{Q.E.D.}}

\DeclareMathOperator\dom   {dom}
\DeclareMathOperator\Img   {Im}
\DeclareMathOperator\range {range}

\newcommand\trio  {\triangle}

\newcommand\rc    {\right\rceil}
\newcommand\lc    {\left\lceil}
\newcommand\rf    {\right\rfloor}
\newcommand\lf    {\left\lfloor}
\newcommand\ceil  [1] {\lc #1 \rc}
\newcommand\floor [1] {\lf #1 \rf}

\newcommand\lex   {<_{lex}}

\newcommand\az    {\aleph_0}
\newcommand\uaz   {^{\aleph_0}}
\newcommand\al    {\aleph}
\newcommand\ual   {^\aleph}
\newcommand\taz   {2^{\aleph_0}}
\newcommand\utaz  { ^{\left (2^{\aleph_0} \right )}}
\newcommand\tal   {2^{\aleph}}
\newcommand\utal  { ^{\left (2^{\aleph} \right )}}
\newcommand\ttaz  {2^{\left (2^{\aleph_0}\right )}}

\newcommand\n     {$n$־יה\ }

% Math A&B shortcuts
\newcommand\logn  {\log n}
\newcommand\logx  {\log x}
\newcommand\lnx   {\ln x}
\newcommand\cosx  {\cos x}
\newcommand\sinx  {\sin x}
\newcommand\sint  {\sin \theta}
\newcommand\tanx  {\tan x}
\newcommand\tant  {\tan \theta}
\newcommand\sex   {\sec x}
\newcommand\sect  {\sec^2}
\newcommand\cotx  {\cot x}
\newcommand\cscx  {\csc x}
\newcommand\sinhx {\sinh x}
\newcommand\coshx {\cosh x}
\newcommand\tanhx {\tanh x}

\newcommand\seq   {\overset{!}{=}}
\newcommand\slh   {\overset{LH}{=}}
\newcommand\sle   {\overset{!}{\le}}
\newcommand\sge   {\overset{!}{\ge}}
\newcommand\sll   {\overset{!}{<}}
\newcommand\sgg   {\overset{!}{>}}

\newcommand\h     {\hat}
\newcommand\ve    {\vec}
\newcommand\lv    {\overrightarrow}
\newcommand\ol    {\overline}

\newcommand\mlcm  {\mathrm{lcm}}

\DeclareMathOperator{\sech}   {sech}
\DeclareMathOperator{\csch}   {csch}
\DeclareMathOperator{\arcsec} {arcsec}
\DeclareMathOperator{\arccot} {arcCot}
\DeclareMathOperator{\arccsc} {arcCsc}
\DeclareMathOperator{\arccosh}{arccosh}
\DeclareMathOperator{\arcsinh}{arcsinh}
\DeclareMathOperator{\arctanh}{arctanh}
\DeclareMathOperator{\arcsech}{arcsech}
\DeclareMathOperator{\arccsch}{arccsch}
\DeclareMathOperator{\arccoth}{arccoth}
\DeclareMathOperator{\atant}  {atan2} 
\DeclareMathOperator{\Sp}     {span} 
\DeclareMathOperator{\sgn}    {sgn} 
\DeclareMathOperator{\row}    {Row} 
\DeclareMathOperator{\adj}    {adj} 
\DeclareMathOperator{\rk}     {rank} 
\DeclareMathOperator{\col}    {Col} 
\DeclareMathOperator{\tr}     {tr}

\newcommand\dx    {\,\mathrm{d}x}
\newcommand\dt    {\,\mathrm{d}t}
\newcommand\dtt   {\,\mathrm{d}\theta}
\newcommand\du    {\,\mathrm{d}u}
\newcommand\dv    {\,\mathrm{d}v}
\newcommand\df    {\mathrm{d}f}
\newcommand\dfdx  {\diff{f}{x}}
\newcommand\dit   {\limhz \frac{f(x + h) - f(x)}{h}}

\newcommand\nt[1] {\frac{#1}{#1}}

\newcommand\limz  {\lim_{x \to 0}}
\newcommand\limxz {\lim_{x \to x_0}}
\newcommand\limi  {\lim_{x \to \infty}}
\newcommand\limh  {\lim_{x \to 0}}
\newcommand\limni {\lim_{x \to - \infty}}
\newcommand\limpmi{\lim_{x \to \pm \infty}}

\newcommand\ta    {\theta}
\newcommand\ap    {\alpha}

\newcommand\inft  {\infty}
\newcommand\ninf  {-\inf}

% Combinatorics shortcuts
\newcommand\sumnk     {\sum_{k = 0}^{n}}
\newcommand\sumni     {\sum_{i = 0}^{n}}
\newcommand\sumnko    {\sum_{k = 1}^{n}}
\newcommand\sumnio    {\sum_{i = 1}^{n}}
\newcommand\sumai     {\sum_{i = 1}^{n} A_i}
\newcommand\nsum[2]   {\reflectbox{\displaystyle\sum_{\reflectbox{\scriptsize$#1$}}^{\reflectbox{\scriptsize$#2$}}}}

\newcommand\bink      {\binom{n}{k}}
\newcommand\setn      {\{a_i\}^{2n}_{i = 1}}
\newcommand\setc[1]   {\{a_i\}^{#1}_{i = 1}}

\newcommand\cupain    {\bigcup_{i = 1}^{n} A_i}
\newcommand\cupai[1]  {\bigcup_{i = 1}^{#1} A_i}
\newcommand\cupiiai   {\bigcup_{i \in I} A_i}
\newcommand\capain    {\bigcap_{i = 1}^{n} A_i}
\newcommand\capai[1]  {\bigcap_{i = 1}^{#1} A_i}
\newcommand\capiiai   {\bigcap_{i \in I} A_i}

\newcommand\xot       {x_{1, 2}}
\newcommand\ano       {a_{n - 1}}
\newcommand\ant       {a_{n - 2}}

% Linear Algebra
\DeclareMathOperator{\chr}     {char}
\DeclareMathOperator{\diag}    {diag}
\DeclareMathOperator{\Hom}     {Hom}
\DeclareMathOperator{\Sym}     {Sym}
\DeclareMathOperator{\Asym}    {ASym}

\newcommand\lra       {\leftrightarrow}
\newcommand\chrf      {\chr(\F)}
\newcommand\F         {\mathbb{F}}
\newcommand\co        {\colon}
\newcommand\tmat[2]   {\cl{\begin{matrix}
			#1
		\end{matrix}\, \middle\vert\, \begin{matrix}
			#2
\end{matrix}}}

\makeatletter
\newcommand\rrr[1]    {\xxrightarrow{1}{#1}}
\newcommand\rrt[2]    {\xxrightarrow{1}[#2]{#1}}
\newcommand\mat[2]    {M_{#1\times#2}}
\newcommand\gmat      {\mat{m}{n}(\F)}
\newcommand\tomat     {\, \dequad \longrightarrow}
\newcommand\pms[1]    {\begin{pmatrix}
		#1
\end{pmatrix}}

\newcommand\norm[1]   {\left \vert \left \vert #1 \right \vert \right \vert}
\newcommand\snorm     {\left \vert \left \vert \cdot \right \vert \right \vert}
\newcommand\smut      {\left \la \cdot \mid \cdot \right \ra}
\newcommand\mut[2]    {\left \la #1 \,\middle\vert\, #2 \right \ra}

% someone's code from the internet: https://tex.stackexchange.com/questions/27545/custom-length-arrows-text-over-and-under
\makeatletter
\newlength\min@xx
\newcommand*\xxrightarrow[1]{\begingroup
	\settowidth\min@xx{$\m@th\scriptstyle#1$}
	\@xxrightarrow}
\newcommand*\@xxrightarrow[2][]{
	\sbox8{$\m@th\scriptstyle#1$}  % subscript
	\ifdim\wd8>\min@xx \min@xx=\wd8 \fi
	\sbox8{$\m@th\scriptstyle#2$} % superscript
	\ifdim\wd8>\min@xx \min@xx=\wd8 \fi
	\xrightarrow[{\mathmakebox[\min@xx]{\scriptstyle#1}}]
	{\mathmakebox[\min@xx]{\scriptstyle#2}}
	\endgroup}
\makeatother


% Greek Letters
\newcommand\ag        {\alpha}
\newcommand\bg        {\beta}
\newcommand\cg        {\gamma}
\newcommand\dg        {\delta}
\newcommand\eg        {\vepsi}
\newcommand\zg        {\zeta}
\newcommand\hg        {\eta}
\newcommand\tg        {\theta}
\newcommand\ig        {\iota}
\newcommand\kg        {\keppa}
\renewcommand\lg      {\lambda}
\newcommand\og        {\omicron}
\newcommand\rg        {\rho}
\newcommand\sg        {\sigma}
\newcommand\yg        {\usilon}
\newcommand\wg        {\omega}

\newcommand\Ag        {\Alpha}
\newcommand\Bg        {\Beta}
\newcommand\Cg        {\Gamma}
\newcommand\Dg        {\Delta}
\newcommand\Eg        {\Epsi}
\newcommand\Zg        {\Zeta}
\newcommand\Hg        {\Eta}
\newcommand\Tg        {\Theta}
\newcommand\Ig        {\Iota}
\newcommand\Kg        {\Keppa}
\newcommand\Lg        {\Lambda}
\newcommand\Og        {\Omicron}
\newcommand\Rg        {\Rho}
\newcommand\Sg        {\Sigma}
\newcommand\Yg        {\Usilon}
\newcommand\Wg        {\Omega}

% Other shortcuts
\newcommand\tl    {\tilde}
\newcommand\op    {^{-1}}

\newcommand\sof[1]    {\left | #1 \right |}
\newcommand\cl [1]    {\left ( #1 \right )}
\newcommand\csb[1]    {\left [ #1 \right ]}
\newcommand\ccb[1]    {\left \{ #1 \right \}}

\newcommand\bs        {\blacksquare}
\newcommand\dequad    {\!\!\!\!\!\!}
\newcommand\dequadd   {\dequad\duquad}

\renewcommand\phi     {\varphi}


% Theorems etc.
\definecolor{myblue}      {rgb}{0.2,0.35,0.7}
\definecolor{mygreen}     {rgb}{0.15,0.65,0.35}
\definecolor{myyellow}    {rgb}{0.0,0.4,0.5}
\definecolor{mycyan}      {rgb}{0.05,0.65,0.6}
\definecolor{myred}       {rgb}{0.05,0.1,0.75}
\definecolor{mymagenta}   {rgb}{0.1,0.7,0.1}

\theoremstyle{definition}
\newtheorem{Theorem}      {\color{myblue}משפט}
\newtheorem{Definition}   {\color{mygreen}הגדרה}
\newtheorem{Lemma}        {\color{myyellow}למה}
\newtheorem{Remark}       {\color{mycyan}הערה}
\newtheorem{Notion}       {\color{myred}סימון}
\newtheorem{Collary}      {\color{mymagenta}מסקנה}
\newtheorem{Exercise}     {\normalcolor תרגיל}

\newcommand\cola  [1]     {\begin{Collary}#1\end{Collary}}
\newcommand\theo  [1]     {\begin{Theorem}#1\end{Theorem}}
\newcommand\defi  [1]     {\begin{Definition}#1\end{Definition}}
\newcommand\rmark [1]     {\begin{Remark}#1\end{Remark}}
\newcommand\lem   [1]     {\begin{Lemma}#1\end{Lemma}}
\newcommand\noti  [1]     {\begin{Notion}#1\end{Notion}}
\newcommand\exe   [1]     {\begin{Exercise}#1\end{Exercise}}
\newcommand\exec  [1]     {\begin{Exercise}[נפוץ]#1\end{Exercise}}
\newcommand\exeh  [1]     {\begin{Exercise}[קשה]#1\end{Exercise}}

% Algorithems
\newcommand\sFunc [1] {\SetKwFunction{#1}{#1}}
\newcommand\sData [1] {\SetKwData{#1}{#1}}
\newcommand\sIO   [1] {\SetKwInOut{#1}{#1}}
\newcommand\ttt   [1] {\sen \texttt{#1} \she\,}
\newcommand\io    [2] {\Input{#1}\Output{#2}\BlankLine}

%! ~~~ Document ~~~

\author{שחר פרץ}
\title{\textit{חדו''א 1א 1}}
\begin{document}
	\maketitle
	
	\textbf{שם מרצה: }ליאור קמה
	
	\textbf{אימייל: }liorkammma@tauex.tau.ac.il
	
	ניוטון פיתח לראשונה את החדו''א ככלי לנתח ככובי לכת וקליעים. תוך כדי כך לייבניץ פיתח את החדו''א. ההגדרות לא היו פורמליות בכלל. זה השתנה לאחר פרדוקס ראסל, ולאחרי שזרם הפורמליזם של הילברט בגטינגן השתלט על הכל. 
	
	\section{\en{Intro}}
	\subsection{שדות סדורים שלם}
	דיברנו על מערכת המספרים הממשיים בלינארית. נדבר על הקבוצה $\R$. הקבוצה היחידה שניתנת לנו מהמשיים היא $\N$ (מהאקסיומות של תקבצ). מהטבעיים בונים את הקבוצות האחרות, כמו השלמים והרציונליים. לבנות את הממשיים זה יותר בלגן, זה לא קשה, בעיקר לוקח זמן. 
	
	אופציה אחרת, היא במקום לבנות את $\R$, ניגש בקבוצה האקסיומטית, כמו שראינו בתורת החוגים. נניח כל מני דברים על הקבוצה הזו, נקווה שהיא קיימת, ונוכיח כל מני טענות על גבי זה. 
	
	אינטואיטיבית נחשוב על זה כעל כל מספר שיכול להתבטא באורך של קטע. 
	
	יש לנו שתי פעולות, $+ \co \R \times \R \to \R$ ו־$\cdot \,\co \R \times \R \to \R$. עקרונית $+(3, 5)$ כיתוב חוקי, אבל כתיב פולני של $3 + 5$ מקובל מספיק. הקבוצה $\R$ היא חבורה בחיבור, חבורה בכפל, ודיסטרבוטיבית. כלומר לכל $x, y, z$ מתקיים: 
	\begin{enumerate}
		\item קומטטיביות: \hfill $\forall x, y \in \R \co x + y = y + x$
		\item אסוציאטיביות: \hfill $\forall x, y, z \in \R\co x + (y + z) = (x + y) + z$
		\item קיום איבר 0 (יחידת חיבור): \hfill $\exists 0 \in \R\co \forall x \in \R x + 0 = x$
		\item קיום נגדי (הופכי לחיבור): \hfill $\forall x \in \R \co \exists y \in \R\co x + y = 0$
	\end{enumerate}
	כבר בעזרת ההנחות האלו אפשר לעשות דברים. 
	\theo{לכל $x, y, z \in \R \co (x + y = z + y) \implies x = z$}\begin{proof}
		יהיו $x, y, z \in \R$. נניח $x+ y = z + y$. מ־4 קיים $t \in \R$ כך ש־$y + t = 0$. נרכיב את $+$ עם $t$ על שני האגפים ונקבל $(x + y) + t = (z + y) + t$ מח''ע הפונקציה. מ־2 נקבל $x + (y + t) = z + (y + t)$ כלומר $x + 0 = t + 0$ ולכן מ־3 $x = z$ כדרוש. 
	\end{proof}
	\cola{לכל $x \in \R$ קיים $y \in \R$ \textit{יחיד} כך ש־$x+  y = 0$. }
	\noti{יהי $x \in \R$. את \textit{ה}מספר $y$ המקיים $x+ y = 0$ נכנה \textit{הנגדי} של $x$ ונסמן $-x$. }
	
	נמשיך עתה עם אקסיומות כפל. 
	
	\begin{enumerate}
		\skipitems{4}
		\item קומוטטיביות: \hfill $\forall x, y \in \R\co x\cdot y = y \cdot x$
		\item אסוציאטיביות: \hfill $\forall x, y, z, \in \R\co (xy)z = x(yz)$
		\item קיום ניטרלי לחיבור (קיום יחידה בכפל): \hfill $x \cdot 1 = x$
		\item קיום הופכי בכפל: \hfill $\forall x \in \R\setminus\{0\}\co \exists y \in \R\co xy = 1$
	\end{enumerate}
	\theo{לכל $x, y, z \in \R$, אם $xy = zy \land y \neq 0$ אז $x = z$. }
	שימו לב לדרישה $y \neq 0$. 
	
	\begin{proof}
	 תרגיל לבית
	\end{proof}
	\cola{לכל $x \in \R\setminus\{0\}$ קיים $y \in \R \setminus \{0\}$ יחיד, כך ש־$xy = 1$. }
	\noti{יהי $x \in \R\setminus \{0\}$, את המספר המקיים $y \neq 0 \land xy = 1$ נכנה \textit{ההופכי} של $x$ ונסמן $x\op$. }
	
	עתה נוסיף את התכונה האחרונה שנדרשה מאיתנו: 
	\begin{enumerate}
		\skipitems{8}
		\item דיסטרבוטיביות: $\forall x, y, z \in \R\co x(y + z) = xy + xz$
	\end{enumerate}
	
	תשעת האקסיומות הללו מגדירות על $(\R, +, \cdot)$ מבנה הקרוי \textit{שדה}. הוא למעשה חוג עם הופכי בכפל, ומקיים כל מני תכונות נחמדות שראינו באלגברה לינארית 1א. 
	
	\theo{לכל $x \in \R$ מתקיים $x \cdot 0 = 0$. }\begin{proof}
		יהי $x \in \R$. לפי 3 $0 + 0 = 0$ כלומר $x \cdot (0 + 0) = x \cdot 0$ מהיות כפל פונקציה ולכן חד־ערכי. לפי 9 $x \cdot 0 + x \cdot 0 = x \cdot 0$. לפי 3 $x \cdot 0 + x \cdot 0 = x \cdot 0 + 0$. מהטענה הראשונה שהוכחנו $x \cdot 0 = 0$. 
	\end{proof}
	\theo{$\forall x \in \R\co (-1) \cdot x = -x$. }\begin{proof}
		יהי $x$. מטענה קודמת, $x \cdot 0 = 0$. מההגדרה, $1 + (-1) = 0$. לכן $x(1 + (-1)) = 0$. לפי 9, $x \cdot 1 + x \cdot (-1) = 0$. לפי 7 ו־5 $x + (-1)x = 0$. הוכחנו את יחידות הנגדי ולכן $(-1) \cdot x = -x$. 
	\end{proof}
	
	עתה, נגדיר \textit{יחס סדר} (כמ ושעשינו בבדידה 1). קבוצה $R$ קרויה \textit{יחס} אם $R \subseteq A \times A$ עבור $A$ כלשהו. ואכן, טוענים $< \subseteq \R\times \R$. במקום לכתוב $(2, 3) \in <$ נכתוב $2> 3$. 
	
	\begin{enumerate}
		\skipitems{9}
		\item אנטי־סימטריות חזקה: \hfill $\forall x, y \in \R \co x < y \implies x \not< y$
		\item טרנזטיביות: \hfill $\forall x, y, z \in \R \co (x < y \land y < z) \implies x < z$
		\item מליאות: \hfill $\forall x, y \in \R \co x < y \lor x = y \lor y < x$
		\item אדטיביות: \hfill $\forall x, y, z \in \R \co x < y \implies x + z < y+ z$
		\item ססקווי־כפליות: \hfill $\forall x, y, z \in \R\co (x < y \land 0 < z) \implies xz < yz$
	\end{enumerate}
	
	הקבוצה $(\R, +, \cdot, <)$ נקראת שדה סגור. 
	\theo{יהיו $x, y \in \R$. אם $x < y$ אז $-y < -x$. }\begin{proof}
		נניח $x < y$. לפי 13 $x + (-y) < y + (-y)$, כלומר $x + (-y) < 0$. לפי 1, 13 מתקיים $-x + (x + (-y)) < -x + 0$. לפי $2, 3$ מתקיים $(-x + x) + (-y) < -x$ וסה''כ $0 + (-y) < -x$ ומ־$3$ נקבל $-y < -x$ כדרוש. 
	\end{proof}
	
	\theo{לכל $x, y, z, w \in \R$, אם $x < y \land z < w$ אז $x + z < y +w$. }
	שימו לב שזה לא עובד בכפל, אלא אם מניחים שהכל חיובי (לבית). 
	
	יש לציין שגם $(\Q, +, \cdot, <)$ הוא יחס סדר סדור. 
	
	אז מה מיוחד ב־$\R$? תמתינו, אבל הרעיון הוא שהוא יותר ''רציף``. המהות של החשבון הדיפרנציאלי הוא הרצף הזה. את ה''נעילה`` הזו של האקסיומות כך שרק $\R$ יקימן (עד לכדי איזו') יתבצע ע''י הוספת אקסיומת השלמות. 
	
	\subsection{קבוצות חסומות וחסמים}
	\defi{תהא $A \subset \R$. יהי $\ag \in \R$. נאמר ש־$\ag$ \textit{חסם מלעיל} של $A$ אם לכל $a \in A$ מתקיים $a \le \ag$. }
	\defi{תהא $A \subseteq \R$. יהי $\ag \in \R$. נאמר ש־$\ag$ \textit{חסם מלרע} של $A$ אם לכל $a \in A$ מתקיים $\ag \le a$. }
	\defi{$A$ תקרא \textit{חסומה מלעיל} כאשר קיים לה חסם מלעיל. }
	\defi{$A$ תקרא \textit{חסומה מלרע} אם קיים לה חסם מלרע. }
	\defi{$A$ תקרא \textit{חסומה} אם היא חסומה מלעיל ומלרע. }
	\defi{$\ag$ ייקרא \textit{חסם עליון} (סופרמום) כאשר: 
	\begin{enumerate}
		\item $\ag$ חסם מלעיל, כלומר $\forall a \in A \co a \le \ag$
		\item החסימה הדוקה, כלומר $\forall \eg > 0\,\exists a \in A \co a > \ag - \epsi$
	\end{enumerate}}
	נבחין ש־2 \textit{לא} שקול ל''קיים $a\in A$ כך ש־$\ag = a$``. לדוגמה, $A = \{x \in \R\co x < 1\}$. מטרנזטיביות כל $\ag > 1$ הוא חסם עליון, אך רק אחד הוא סופרמום, על אף ש־$1 \notin A$. עם זאת, הכיוון השני עובד: אם $\ag \in A$ חסם עליון של $A$ (קוראים למספר כזה מקסימום), אז $\ag$ סופרמום. כלומר, מקסימום הוא סופרמום, אבל סופרמום לא בהכרח מקסימום. 
	
	האינטואציה ל־2 – לא משנה כמה מעט נוריד (כמה ה־$\eg$ קטן), ברגע שנוריד משהו מ־$\ag$, נקבל משהו שהוא כבר לא חסם מלעיל. כלומר, החסם העליון הוא ''החסם המלעיל הקטן ביותר``. כמו שנראה בהמשך, האינטואציה הזו אולי עוזרת להבין את ההגדרה, אבל היא אינטואציה מטעה מאוד. 
	
	\lem{1 חסם עליון של הקבוצה לעיל}\begin{proof}
		יהי $\ag \in A$. אז $a < 1$ ולכן $a \le 1$ ומכאן הוא חסם מלעיל. נותר להוכיח שהחסימה הדוקה. יהי $\eg > 0$. אז $0 < \frac{\vepsi}{2} < \eg$. לכן $1 > 1 - \frac{\eg}{2} > 1 - \eg$. לכן $1 - \frac{\eg}{2} \in A$, וגם $1 - \frac{\eg}{2} > 1 - \eg$. לכן $1$ חסם עליון. 
	\end{proof}
	\theo{תהא $A \subseteq \R$. אם יש ל־$A$ חסם עליון, יש לה חסם עליון יחיד. }\begin{proof}
		נניח $\ag$ חסם עליון של $A$ וגם $\bg$ חסם עליון של $A$. נניח בשלילה $\ag < \bg$. נסמן $\eg = \bg - \ag$ ומההנחה $\epsi > 0$. נקבל קיום $a \in A$ כך ש־$a > \bg - (\bg - \ag)$ ולכן $a > \ag$, בסתירה לכך ש־$\ag$ חסם מלעיל של $A$ חסם מלעיל של $A$. 
	\end{proof}
	
	\noti{תהר $A \subseteq \R$ קבוצה חסומה מלעיל. נסמן את החסם העליון של $A$ ב־$\sup A$. }
	
	לבית – תגדירו באופן דומה חסם תחתון. 
	% TODO
	\noti{חסם תחתון (שהגדרתם בבית) יקרא \textit{אינפימום} ויסומן ב־$\inf A$. }
	
	עתה, נוכל להגדיר את האקסיומה ה־15 של הממשיים. 
	\begin{enumerate}
		\skipitems{14}
		\item \textit{אקסיומת השלמות} (או \textit{אקסיומת החסם העליון}): לכל $A \subseteq \R$. אם $A \neq \varnothing$ וגם $A$ חסומה מלעיל, אז ל־$A$ קיים חסם עליון. 
	\end{enumerate}
	\lem{לכל $x \in \Q$, $x^2 \neq 2$. }(כלומר, $\sqrt2$ מספר אי־רציונלי) \begin{proof}
		יהי $x \in \Q$. נניח בשלילה $x^2 = 2$. קיימים $m, n \in \Z$ כך ש־$n \neq 0$ וגם $x = \frac{m}{n}$. ללא הגבלת הכלליות, $m$ אי־זוגי או $n$ אי־זוגי (לבית: לסגור את הפינה הזו באינדוקציה). לכן $\cl{\frac{m}{n}}^2 = 2$, כלומר $\frac{m^2}{n^2} = 2$. מכאן $m^2 = 2n^2$. לכן $m^2$ זוגי ולכן $m$ זוגי (כי ריבוע לא משנה גורמים ראשוניים). סה''כ קיים $k$ כך ש־$m = 2k$ כלומר $4k^2 = 2n^2$ ומכאן $n^2 = 2k^2$ ואז $n$ זוגי וסתירה. לכן $x^2 \neq 2$. 
	\end{proof}
	\lem{יהיו $x, y \in \R$, אם $x > 0 \land y > 0 \land x^2 < y^2$ אז $x < y$. }
	
	\theo{$(\Q, \cdot, +, <)$ אינה מקיימת את אקסיומת השלמות. }\begin{proof}
		נתבונן בקבוצה $A = \{\Q \ni x > 0 \co x^2 < 2\}$. נתבונן ב־1. $1 \in \Q$ וכמו כן $1 > 0$ וגם $1^2 < 2$ כלומר $1 \in A$ ו־$A \neq \varnothing$. נתבונן ב־2. נראה ש־2 חסם מלעיל. יהי $a \in A$. ידוע $a^2 < 2$. נפצל למקרים. מקרה $1$, נניח $a \ge 1$ ואז $a \le a^2 < 2$. מקרה 2, נניח $a<1$. אז $a<2$ וסיימנו. לכן $2$ חסם מלעיל של $A$ כלומר $A$ חסומה מלעיל. 
		
		נותר להוכיח שאין ל־$A$ חסם עליון. יהי $\ag \in \Q$. נראה ש־$\ag$ לא חסם עליון. ידוע ממשפט קודם $\ag^2 \neq 2$. לכן, $\ag^2 < 2 \lor \ag^2 > 2$. 
		\begin{itemize}
			\item אם $\ag^2 < 2$. 
			[טיוטה: היינו רוצים לקחת ממוצע חשבוני, עם $\sqrt 2$. אבל $\sqrt 2$ לא מוגדר. כלומר היינו רוצים למצוא $\dg$ כך ש־$(\ag + \dg)^2 >2$. זה יוצא $(\ag + \dg)^2 = \ag^2 + 2\ag \dg + \dg^2 < 2$. מכאן $2\ag\dg + \dg^2 < 2 - \ag^2$. בגלל ש־$2 - \ag^2$ קבוע חיובי יש לנו תקווה שזה אפשרי. נקווה $\dg < 1$ ואז $2\ag\dg + \dg^2 \le 2\ag\dg + \dg < 2 - \ag^2$ ואז $\dg < \frac{2 - \ag^2}{2\ag + 1}$, וברגע שנדע שזה לא $0$ בהכרח קיים $\dg > 0$ מתאים. ניקח את המינימום בין זה לבין $1$ ונגמור עניין – סוף טיוטה]. 
			\begin{itemize}
				\item אם $\ag < 0$ אז $\ag$ אינו חסם עליון, אחרת נסמן $\dg = \frac{1}{2} \min \csb{1, \frac{2 - \ag^2}{2\ag + 1}}$. אז $\dg \in \Q$ ו־$\dg > 0$, שכן מההנחה $2 - \ag^2 > 0$ ולכן $\dg \neq 0$. לכן $\ag + \dg \in \Q$ וגם $\ag + \dg > \ag > 0$. כמו כן, $(\ag + \dg)^2 = \ag^2 + 2\ag\dg + \dg^2$. ידוע $\dg \le \frac{1}{2}$ דהיינו $\dg^2 < \dg$. לכן: 
				\[ (\ag + \dg)^2 < \ag^2 + 2\ag\dg + \dg = \ag^2 + \dg(2\ag + 1) < \ag^2 + \frac{2 - \ag^2}{2\ag + 1}(2\ag + 1) = \ag^2 + 2 - \ag^2 = 2 \]
				לכן $\ag + \dg \in A$ כלומר $\ag$ אינו חסם מלעיל של $A$ ולכן איננו חסם עליון. 
				\item בדומה למקרה הקודם, אם $\ag \le 0$ אז $\ag$ אינו חסם עליון של $A$. נניח $\ag > 0$. [טיוטה: הפעם נעשה הפוך, נרצה למצוא $\dg > 0$ כך ש־$(\ag - \dg)^2 = \ag^2 - 2\ag\dg + \dg^2 > 2$ ומכאן $2\ag\dg - \dg^2 < \ag^2 - 2$ צ.ל. חייבים להניח $\dg < \ag$, בלי קשר $2 \ag \dg - \dg^2 < 2 \ag \dg < \ag^2 - 2$ וסה''כ $\dg < \frac{\ag^2 - 2}{2\ag}$.  – סוף טיוטה]
				
				נפנה לאשכרה הוכחה. נבחר $\dg = \frac{1}{2}\min \csb{\ag, \frac{\ag^2 - 2}{2\ag}}$. נראה ש־$\ag - \dg$ גם חסם מלעיל. אז $\dg < \ag$ ולכן $\ag - \dg >0$. כמו כן: 
				\[ (\ag - \dg)^2 = \ag ^2 - 2\ag \dg + \dg^2 > \ag^2 - 2 \ag \dg = \cdots \]
				ידוע $\dg < \frac{\ag^2 - 2}{2\ag}$ כלומר $2\ag\dg < \ag^2 - 2$ וגם $-2\ag\dg > 2 - \ag^2$. 
				\[ \cdots > \ag^2 + (2 - \ag^2) = 2 \]
				נותר להראות ש־$\ag - \dg$ אשכרה חסם עליון. יהי $a \in A$. אז $a^2 < 2 < (\ag - \dg)^2$. מהיות $\ag - \dg > 0$ כי בחרנו את $\dg$ כך ש־$\ag > \dg$, ידוע $\ag < \ag - \dg$ (מהלמה השנייה שהוכחנו). לכן $\ag - \dg$ חסם מלעיל של $A$, ו־$\ag$ אינו חסם עליון של $A$. 
			\end{itemize}
		\end{itemize}
	\end{proof}
	
	לסיכום – אקסיומת השלמות היא ההבדל המשמעותי בין $\R$ ל־$\Q$. לבינתיים, נניח ש־$\R$ שדה סדור מלא שמקיים את אקסיומת החסם העליון, ואפשר להראות קיום, ואף להראות שכל השדות המתאימים איזומורפים אחד לשני. 
	
	\theo{לכל $x \in \R$, אם $x > 0$ אז קיים $y \in \R$ \textit{יחיד} כך ש־$y > 0$ וגם $y^2 = x$. }
	\begin{proof}
		לא נוכיח במדויק, נוכיח רק בערך. נגדיר את $A = \{A \co a^2 < x\}$. ממש כמו שהוכחנו קודם, אפשר להראות ש־$A$ חסומה מלעיל, וב־$\R$ יש לה חסם עליון. צ.ל. שריבוע החסם העליון הזה, הוא $x$. 
	\end{proof}
	יש הכללה למשפט הזה: 
	\theo{לכל $x \in \R$, ולכל $n \in \N_+$, אם $x > 0$ אז קיים $y \in \R$ יחיד כך ש־$y > 0$ וגם $y^n = x$. }
	ההכלה הזו יותר מסובכת, וצריך בשביל זה את הבינום של ניוטון. זה הרבה עבודה ידנית. 
	\noti{נסמן את ה־$y$ היחיד שמקיים את המשפט לעיל ב־$\sqrt[n]{x}$. }
	כמה מילים לגבי חזקות. חזקות שלמות אפשר להגדיר רקורסיבית. חזקות רציונליות אפשר להגדיר בפחות או יותר באופן הבא: 
	\[ a^{\frac{m}{n}} = \sqrt[n]{a}^{m} \]
	שקיים מהמשפטים שלנו. בשביל ההגדרה הזו, צריך להראות שזה לא תלוי בייצוג של הרציונלי – לא איכפת לנו בעבור אילו $n, m$ אנו מגדירים את זה, כלומר $\forall m, n, k, \ml \co \frac{m}{n}  = \frac{k}{\ml} \implies \sqrt[n]{a^{m}} = \sqrt[\ml]{a^{k}}$. 
	
	
	\ndoc
\end{document}