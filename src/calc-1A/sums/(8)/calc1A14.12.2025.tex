\documentclass[]{../../../../tex/classes/styledArticle}
\usepackage{../../../../tex/packages/hebrewSupport}
\usepackage{../../../../tex/packages/mathShortcuts}
\usepackage{../../../../tex/packages/theoremsSupport}

\newcommand\limxo {\lim_{x \to x_0}}
\newcommand\lxo   {$x_0 \in \R$ נקודת הצטברות של $A$}
\newcommand\gf    {$f \co A \subseteq \R\to \R$\ }

\author{שחר פרץ}
\title{\textit{חדו''א 1א 8}}
\begin{document}
	\maketitle
	\subsection*{אריתמטיקה של גבולות}
	\theo{תהאנה $f, g \co \R\to \R$ ותהא $X_0 \in \R$ נקודת הצטברות של $A$. (בסיכומים של שירי ואסף, הם יגידו ש־$f, g \co I \setminus \{x_0\} \to \R$. זה מקרה מאוד פרטי – הם עוסקים בקטעים בלבד, במקום בקבוצות פתוחות). נניח כי יהיו $\ml, m \in \R$, ונניח כי $\limxo f (x) = \ml \land \limxo g(x) = m$. 
		\begin{enumerate}
			\item \hfil $\disty\forall \ag, \bg \in \R \co \limxo f(\ag g(x) + \bg g(x)) = \ag \ml + \bg m$
			\item \hfil $\disty\limxo f(x)g(x) = \ml m$
			\item \hfil $\disty m \neq 0 \implies (\exists \dg > 0 .\, \forall x \in A \co 0 < \sof{x - x_0} < \dg \implies g(x) \neq 0) \land \cl{\limxo \frac{f(x)}{g(x)} = \frac{\ml}{m}}$
		\end{enumerate}
	}
	
	\begin{proof}[הוכחת 3]
		נניח $m \neq 0$. ידוע $\limxo g(x) = m$. לכן קיים $\dg > 0$ כך שלכל $x \in A$, אם $0 < \sof{x - x_0} < \dg$ אז $\sof{g(x) - m} < \frac{m}{2}$. נתבונן ב־$\dg$. יהי $x \in A$. נניח $0 < \sof{x - x_0} < \dg$. אז $\sof{g(x) - m} < \frac{m}{2}$. לכן: 
		\[ \sof{g(x)} \ge \sof{m} - \sof{g(x) - m} > \sof m - \frac{\sof{m}}{2} = \frac{\sof m}{2} \]
		סיימנו את החלק הראשון של המשפט. עתה נותר להוכיח ש־$\limxo \frac{f(x)}{g(x)} = \frac{\ml}{m}$. תהא $\an$ סדרה המקיימת $\Img \an \subseteq A \setminus \{x_0\}$ וכן $\limsi a_n = x_0$. לפני היינה $\limsi f(a_n) = \ml$, וכן $\limsi g(a_n) = m$. ידוע $m \neq 0$, ולכן לפי אריתמטיקת גבולות של סדרות, נקבל: 
		\[ \limsi \frac{f(a_n)}{g(a_n)} = \frac{\ml}{m} \]
		לפי היינה (מהכיוון השני) סה''כ $\limxo \frac{f(x)}{g(x)} = \frac{\ml}{m}$. 
	\end{proof}
	
	אפשר להכליל את החלק הראשון של שלוש (זו אותה ההוכחה) ולקבל את המשפט הבא: 
	\theo{תהא $f \co A \subseteq \R \to \R$ ותהא $x_0 \in \R$ נקודת הצטברות של $A$. אם קיים ל־$f$ גבול סופי ב־$x_0$, קיימת סביבה נקובה של $x_0$ שבה $f$ חסומה. }
	
	''להיות חכם זה לדעת שעבניה זה פרי, ולהיות אינטליגנט זה לדעת לא להכניס אותו לסלט פירות``
	
	\rmark{המרצה רימה. לא בהכרח קיימת סביבה נקובה שמוכלת כולה ב־$A$. מההקשר, אפשר להבין שהכוונה ב''שבה`` היא כל נקודה שבתחום ההגדרה מוכלת בסביבה הזו. }
	
	
	\theo{תהאנה $f, g \in A \subseteq \R\to \R$ ונניח כי $A$ אינה חסומה מלעיל [כלומר אינסוף הוא נקודת הצטברות]. נניח כי $g$ חסומה וכי הגבול $\lim_{x \to \inft} f(x) = -\infty$. אז $\lim_{x \to \infty} f(x) + g(x) = -\infty$. }\begin{proof}
		$g$ חסומה לכן קיים $M > 0$ חסם שלה כך ש־$\forall x \in A \co \sof{g(x)} \le M$. [מה צ.ל.? שלכל $K > 0$ קיים $N > 0$ כך ש־$\forall x \in A$ אם $x > N$ אז $f(x) + g(x) < -K$] יהי $K > 0$. ידוע $\lim_{x \to \infty} f(x) = -\infty$ ולכן קיים $N > 0$ כך שלכל $ A \ni x > N$ מתרחש $f(x) < -K -M $. נניח $x > N$. אז $f(x) + g(x)< -K - M + M = -K$. לכן $\lim_{x \to \infty} f(x) + g(x) = -\infty$
	\end{proof}
	
	\theo{תהאנה $f, g \co A \subseteq \R\to \R$ ותהא $x_0 \in \R$ נקודת הצטברות של $A$. נניח כי קיימת סביבה נקובה של $x_0$ שבה לכל $x$, $f(x) \le g(x)$. נניח כי $\limxo f(x) = \infty$, אז $\limxo g(x) = +\infty$. }\begin{proof}
		מהנתון קיים $\dg > 0$ כך שלכל $x \in A$, אם $0  < \sof{x - x_0} < \dg$, אז $f(x) \le g(x)$. תהא $\an$ סדרה המקיימת $\Img \an \subseteq A \setminus \{x_0\}$ וכן $\limsi a_n = x_0$. קיים $N_1$ כך ש־$\forall n \ge N_1$ מתקיים $0 < \sof{a_n - x_0} < \dg$ (השתמשנו בהגדרת הגבול, כאשר ''ה־$\eg$ שלנו`` הוא $\dg$). ידוע $\limxo f(x) = \inft$ לכן לפי היינה $\limsi f(a_n) = \inft$. יהי $K>0$. קיים $N_2 \in \N$ כך ש־$\forall n \ge N_2 \co f(a_n) > K$. נתבונן ב־$N = \max\{N_1, N_2\}$. יהי $n \ge N$. אז $g(a_n) \ge f(a_n) > K$. לכן $\limsi f(a_n) = \infty$ ולכן מהיינה $\limxo g(x) = + \infty$. 
	\end{proof}
	
	\theo{תהנה $f, g, h \co A \subseteq \R\to \R$, ותהא $x_0 \in \R$ נקודת הצטברות של $A$. נניח כי קיימת סביבה נקובה של $x_0$ שבה לכל $x$ $h(x) \le f(x) \le g(x)$. יהי $\ml \in \R$. נניח $\limxo g(x) = \limxo h(x) = \ml$. אז $\limxo f(x) = \ml$. }
	\begin{proof}
		לבית – להוכיח עם הגדרת קושי, ועם הגדרת היינה. 
	\end{proof}
	
	\theo{תהאנה $f, g \co A \subseteq \R \to \R$ ותהא $x_0 \in \R$ נקודת הצטברות של $A$. יהיו $\ml, m \in \R$. נניח $\limxo f(x) = \ml \land \limxo g(x) = m$. 
	\begin{enumerate}
		\item אם קיימת סביבה של $x_0$, כל שלכל $x$ בה $f(x) \le g(x)$ אז $\ml \le m$. 
		\item אם $\ml < m$, אז קיימת סביבה נקובה של $x_0$ שבה לכל $x$ בה $f(x) < g(x)$. 
	\end{enumerate}}
	\begin{proof}[הוכחה ל־$2$]
		נניח $\ml < m$. קיים $\dg_1 > 0$ כך ש־$\forall x \in A$, אם $0 < \sof{x - x_0} < \dg_1$ אז $\sof{f(x) - \ml} < \frac{m - \ml}{2}$. באותו האופן קיים $\dg_2 > 0$ כך ש־$\forall x \in A$, אם $0 < \sof{x - x_0} < \dg_2$ אז $\sof{g(x) - m} < \frac{m - \ml}{2}$. נתבונן ב־$\dg = \min\{\dg_1, \dg_2\}$. יהי $x \in A$. נניח $0 < \sof{x - x_0} < \dg$. אז: 
		\[ f(x) < \ml + \frac{m - \ml}{2} = \frac{m + \ml}{2} = m - \frac{m - \ml }{2} < g(x) \]\envendproof
	\end{proof}
	ההוכחה של 1 מאוד דומה. 
	
	להלן משפט העונה לשם ''משפט על גבולות והרכבה``. 
	\theo{תהאנה $f \co A \subseteq \R\to B \subseteq \R, \ g \co B \to \R$. תהא \lxo. יהיו $y_0, \ml \in \R$. 	נניח כי: 
	\begin{enumerate}
		\item $\limxo f(x) = y_0$
		\item קיימת סביבה נקובה של $x_0$ שבה לכל $f(x) \neq y_0$. 
		\item $\lim_{x \to y_0} g(x) = \ml$
	\end{enumerate}
	אז $\limxo g \circ g(x) = \ml$. 
	}
	\rmark{גם כאן המרצה עשה עברה – יש כאן הנחה ש־$y_0$ נקודת הצטברות של $B$. זה בסדר, כי באמצעות 1 ו־2 אפשר להראות ש־$y_0$ נקודת הצטברות של $B$ בכל מקרה. }

	יש גם ניסוח עם קטעים, פחות בעייתי: 
	תהאנה $f \co I \setminus \{x_0\} \to J \setminus \{x_0\}$ וכן $g \co J \to \R$ (מקובל ש־$I, J$ מסמנים קטעים) ואז ממשיכים את שאר המשפט. אבל הניסוח הזה מקרה פרטי למדי. חשוב לדעת להתבטא כך כי ככה מלמדים בקורס ברגיל. 
	
	\begin{proof}
		ראשית כל, נצטרך לוודא שכל החרא שלנו מוגדר היטב. לשם כך נראה שמ־1 ו־2 אכן נובע ש־$y_0$ נקודת הצטברות של $B$. יהי $\eg > 0$. קיים $\dg_1 > 0$ כך שלכל $x \in A$ אם $0 < \sof{x - x_0} < \dg_1$ ואז $\sof{f(x) - y_0} < \eg$. קיים $\dg_2 > 0$ כך שלכל $x \in A$, אם $0 < \sof{x - x_0} < \dg$ אז $f(x) \neq y_0$. נסמן $\dg  = \min \{\dg_1, \dg_2\}$. $x_0$ נקודת הצטברות של $A$, לכן קיים $x\in A$ כך ש־$0 < \sof{x - x_0} < \dg$. נתבונן ב־$f(x)$. אז $f(x) \in B$ וכן $0 < \sof{f(x) - y_0} < \dg$ ולכן $y_0$ נקודת הצטברות של $B$ וסיימנו. 
		
		תהא $\an$ כך ש־$\Img \an \subseteq A \setminus \{x_0\}$ וגם $\limsi a_n = x_0$. אז לפי היינה $\limsi f(a_n) = y_0$. אז: 
		\begin{enumerate}
			\item מתקיים $\Img f(a_n) \subseteq B$. 
			\item כמעט תמיד $f(a_n) \neq y_0$ (זה מספיק להיינה. את זה גם צריך להוכיח, לבית). 
			\item בהכרח $\limsi a_n = y_0$. 
		\end{enumerate}
		לכן לפי היינה $\limsi g(f(a_n)) = \ml$, כלומר $\limsi (g \circ f)(a_n) = \ml$. לפי היינה $\limxo (f \circ g)(x_0) = \ml$. 
	\end{proof}
	
	\subsection*{גבולות חד־צדיים}
	\defi{תהא $f \co A \to B$ פונקציה. תהי ת''ק $C \subseteq A$. נגדיר $g \co C \to B$ על־ידי $g(x) = f(x)$ לכל $x \in B$. $g$ נקראת \textit{הצמצום של $f$ ל־$C$} ומסמנים $g = f|_C$. }
	
	ניתן היה אפשר להגדיר תת־סדרה של $\an$ (בדידה) כצמצום של הסדרה לקבוצה אינסופית של טבעיים. הטרמינולוגיה הזו לא צריכה שהסדר על התחום יהיה סדר טוב. לכן נוכל להכליל אותה ל־$\R$. 
	
	\theo{
	\begin{enumerate}
		\item תהא $A \subseteq \R$ ותהא $B \subseteq A$ ויהי $x_0 \in \R$. אם $x_0$ נקודת הצטברות של $B$ אז \lxo. 
		\item תהא $A \subseteq \R$ ותהאנה $B, C \subseteq A \setminus \{x_0\}$ כך ש־$B \cup C = A$. אם \lxo אז $x_0$ נקודת הצטברות של $B$ או ש־$x_0$ נקודת הצטברות של $C$ (ה''או`` לא בהכרח xor). 
	\end{enumerate}}
	\begin{proof}
		לבית
	\end{proof}
	
	מה שנעשה עכשיו על ת''קים ספציפיים, היה אפשר לעשות על כל תת־קבוצה. 
	
	נגדיר את הסימון הבא לסיכום הזה בלבד (הוא לא מקובל). תהא $A \subseteq \R$ ותהא $x_0 \in \R$ נקודת הצטברות של $A$. נסמן $A_{x_0^{+}} := \{x \in A \mid x > x_0\} = A \cap (x_0, +\infty)$. נגדיר את $A_{x_0^{-}} := \{x \in A \mid x< x_0\} = A \cap (-\infty, x_0)$. 
	
	מהמשפט הקודם, אם $x_0$ נקודת הצטברות של $A$, אז $x_0$ נקודת הצטברות של $A_{x_0^{+}}$ וכן של $A_{x_0^{-}}$. 
	
	\defi{תהא $f \co A \subseteq \R \to \R$ ותהא $x_0$ נקודת הצטברות של $A$. אם $x_0$ נקודת הצטברות של $A_{x_0^{+}}$ וגם קיים הגבול של $f|_{A_{x_0}^{+}}$ ב־$x_0$, אז נאמר של־$f$ יש גבול מימין ב־$x_0$ ונסמנו $\lim_{x \to x_0^{+}} f(x)$. }
	\defi{תהא $f \co A \subseteq \R \to \R$ ותהא $x_0$ נקודת הצטברות של $A$. אם $x_0$ נקודת הצטברות של $A_{x_0^{-}}$ וגם קיים הגבול של $f|_{A_{x_0}^{-}}$ ב־$x_0$, אז נאמר של־$f$ יש גבול מימין ב־$x_0$ ונסמנו $\lim_{x \to x_0^{-}} f(x)$. }
	
	
	הכל כמובן במובן הרחב. 
	
	\textbf{דוגמה. }נוכיח ש־$\lim_{x \to 0^{+}} \frac{1}{x} = +\infty$. \begin{proof}
		יהי $K > 0$. נתבונן ב־$\dg = \frac{1}{K}$. יהי $x > 0$ בסביבת הדלתא של $0$ (כלומר $x < \dg$), אז $x \in (0, \dg)$ ולכן $\frac{1}{k} > \frac{1}{\dg} = K$ מכאן $\lim_{x \to x_0^{+}} = + \infty$. 
	\end{proof}
	
	\theo{תהא $f \co A \subseteq \R \to \R$ ותהא \lxo. יהי $\ml \in \R$ ונניח $\lim_{x \to x_0^{+}} f(x) = \ml$. אז אם $x_0$ נקודת הצטברות של $A_{x_0^{-}}$, אז $\lim_{x \to x_0^{-}}f(x) = \ml$. אם $x_0$ נקודת הצטברות של $A_{x_0^{+}}$, אז $\lim_{x \to x_0^{+}}f(x) = \ml$. }
	\rmark{אין באמת סיבה להסתכל על $A_{x_0^{+}}$ ו־$A_{x_0^{-}}$. אפשר היה להגדיר ''גבול חלקי`` על קבוצה כללית ולטעון את המשפט הזה. היינו מקבלים משפט הומורפי לכך שכל הגבולות החלקיים של פונקציה בדידה מתכנסים לגבול יחיד כאשר היא מתכנסת. עוד הערה: בד''כ לא יכתבו ''$x_0$ נקודת הצטברות של $A \cap (x_0, \infty)$`` אלא ''אם יש משמעות לגבול משמאל ס־$x_0$``. }
	
	\theo{תהא \gf ותהא \lxo. יהי $\ml \in \R$. 
	\begin{enumerate}
		\item אם $x_0$ נקודת הצטברות של $A_{x_0^{+}}$ וכן נקודת הצטברות של $A_{x_0^{-}}$, אז $\lim_{x \to x_0^{-}} f(x) = \lim_{x \to x_0^{+}}f(x) = \ml$ גורר ש־$\limxo f(x) = \ml$. 
		
		אחרת [כלומר $x_0$ אינה נקודת הצטברות של אחת מהקבוצות]: 
		\item אם $x_0$ נקודת הצטברות של $A_{x_0^{-}}$ אז $\lim_{x \to x_0^{-}} = \ml$ גורר $\limxo f(x) = \ml$. [כלומר, אם אני יכול להגיע ל־$x_0$ רק מהצד השלילי – זה יקבע את הגבול]
		\item אם $x_0$ נקודת הצטברות של $A_{x_0^{+}}$ אז $\lim_{x \to x_0^{+}} = \ml$ גורר $\limxo f(x) = \ml$. [כלומר, אם אני יכול להגיע ל־$x_0$ רק מהצד החיובי – זה יקבע את הגבול]
	\end{enumerate}}
	
	''הוא ריחם על היאור, על החול במדבר... אבל לסלע הוא נתן זאפטה``
	
	נתחיל מלהוכיח את המשפט הקודם. \begin{proof}
		נניח ש־$x_0$ נקודת הצטברות של $A_{x_0^{-}}$. יהי $\eg > 0$. ידוע $\limxo f(x) = \ml$ לכן קיים לנו $\dg > 0$ כך שלכל $x \in A$ אם $0 < \sof{x - x_0} < \dg$ אז $\sof{f(x) - \ml} < \eg$. נתבונן ב־$\dg$. יהי $x \in A$ ונניח $x < x_0$. אז בפרט $0 < \sof{x - x_0} < \dg$ כלומר $\sof{f(x) - \ml} < \eg$ לכן $\lim_{x \to x_0^{-}} = \ml$. 
		
		החלק השני (החיובי) – בדומה. ובכך סיימנו. 
	\end{proof}
	
	עכשיו נחזור להוכיח את המשפט האחרון. 
	\begin{proof}[הוכחת 1]
		נניח $x_0$ נקודת הצטברות של $A_{x_0^{-}}$ וגם $x_0$ נקודת הצטברות של $A_{x_0^{+}}$. נניח שהגבול משמאל ומימין שניהם $\ml$. יהי $\eg >0$. קיים $\dg_1 > 0$ כך שלכל $x \in A$ אם $x_0 < x < x_0 + \dg$ אז $\sof{f(x) - \ml} < \eg$. קיים $\dg_2 > 0$ כך שלכל $x \in A$ אם $x_0  - \dg < x < x_0$ אז $\sof{f(x) - \ml} < \eg$. נתבונן ב־$\dg := \min \{\dg_1, \dg_2\}$. יהי $x \in A$. נניח $0 < \sof{x - x_0} < \dg$. אז $x_0 < x < x_0 + \dg$ או $x_0 - \dg < x < x_0$. לכן $\sof{f(x) - \ml} < \eg$. 
	\end{proof}
	\begin{proof}[הוכחת 2]
		נניח $x_0$ נקודת הצטברות של $A_{x_0^{-}}$ וגם $x_0$ אינה נקודת הצטברות של $A_{x_0^{+}}$. נניח $\lim_{x \to x_0^{-}} f(x) = \ml$. קיים $\dg_1 > 0$ כך ש־$A \cap (x_0, x + \dg_1) = \varnothing$. ידוע $\lim_{x \to x_0^{-}} f(x) = \ml$ לכן קיים $\dg_2 > 0$ כך שלכל $x \in A$, אם $x_0 - \dg_2 < x< x_0$ אז $\sof{f(x) -\ml} < \eg$. מכאן ממשיכים כמו ההוכחה הקודמת. 
	\end{proof}
	
	\subsection*{קריטריון קושי לקיום גבול של פונקציה}
	\theo{תהא \gf ותהא \lxo. ל־$f$ יש גבול סופי ב־$x_0$ אמ''מ לכל $\eg > 0$, קיים $\dg > 0$, כך שלכל $x, y \in A$ אם $0 < \sof{x - x_0} < \dg$ וגם $0 < \sof{y - x_0} < \dg$ אז $\sof{f(x) - f(y)} < \eg$. }
	ההוכחה זה פחות או יותר היינה עם קושי. 
	
	\subsection*{רציפות}
	רציפות וגזירות אלו שני המושגים שהחלו את החדו''א. בימים של לגראנג', ניוטון ולייבניץ הגדירו באמצעות זה שהפונקציה סימפטית מספיק ואפשר לצייר אותה על דף. ההגדרה הפורמלית היא \textbf{תכונה לוקאלית} – היא מוגדרת בעבור נקודה, לא בעבור כל הפונקציה. ישנן גם תכונות גלובליות, כמו ''בכל נקודה לפונקציה יש גבול`` או ''הפונקציה רציפה בכל התחום``. ההגדרה האינטואיטיבית של רציפות היא תכונה גלובלית. 
	
	\defi{תהא \gf ותהא $x_0 \in A$. נאמר ש־$f$ רציפה ב־$x_0$ אם: 
	\[ \forall \eg > 0.\, \exists \dg > 0.\ \forall x \in A \co \cl{\sof{x - x_0} < \dg} \implies \sof{f(x) - f(x_0)} < \eg \]}
	
	\rmark{כדי לדבר על רציפות בנקודה, חייבים לדבר על נקודה בתחום ההגדרה של הפונקציה. לא מספיקה נקודת התכנסות. מכאן גם, שאם יש חור בתחום ההגדרה, זה לא אומר שהפונקציה לא רציפה. לדוגמה, סדרות רציפות בכל נקודה. }
	
	''לקחתי את העפרון ודחפתי נקודות קצת על הגרף, וזהו! הכל רציף!!``` $\sim$ פיזיקאי כועס
	
	\theo{תהא \gf ותהא $x_0 \in A$. אם \lxo, אז $f$ רציפה ב־$x_0$ אמ''מ $\limxo f(x) = f(x_0)$. }
	
	כשמדברים על קטעים, המשפט הזה פשוט מספק הגדרה שקולה. זה לא עובד יותר כשיש נקודות מבודדות. 
	
	''הוא נחנק, אבל הוא בסדר?``
	\begin{proof}
		נניח ש־\lxo. 
		\begin{itemize}
			\item[$\impliedby$]נניח $f$ רציפה ב־$x_0$. יהי $\eg > 0$. מהגדרת הרציפות קיים $\dg > 0$ כך שלכל $x \in A$ אם $\sof{x - x_0} < \dg$ אז $\sof{f(x) - f(x_0)} < \eg$. נתבונן ב־$\dg$. יהי $x \in A$. נניח $0 < \sof{x - x_0} < \dg$. בפרט $\sof{x - x_0} < \dg$ ולכן $\sof{f(x) - f(x_0)} < \eg$ ומכאן $\limxo f(x) = f(x_0)$. 
			\item[$\implies$]נניח $\limxo f(x) = x_0$. נוכיח שהיא רציפה ב־$x_0$. יהי $\eg > 0$. מהגדרת הגבול קיים $\dg > 0$ כך שלכל $x \in A$ אם $0 < \sof(x - x_0) < \dg$ אז $\sof{f(x) - f(x_0)} < \eg$. נתבונן ב־$\dg$. יהי $x \in A$. נניח $\sof{x - x_0} < \dg$. אם $x = x_0$ אז $0 = \sof{f(x) - f(x_0)} < \eg$. אחרת $0 < \sof{x - x_0} < \eg$ ולכן $\sof{f(x) - f(x_0)} < \eg$ וסיימנו. 
		\end{itemize}\envendproof
	\end{proof}
	
	
	\textbf{הבחנה: }תהא \gf ונניח $x_0 \in A$ נקודת הצטברות של $A_{x_0^{+}}$ וכן של $A_{x_0^{-}}$. אז $f$ רציפה ב־$x_0$ אמ''מ מתקיימים שלושת התנאים הבאים: 
	\begin{enumerate}
		\item קיים ל־$f$ גבול סופי ב־$x_0$ משמאל, וקיים ל־$f$ גבול סופי ב־$x_0$ מימין
		\item שני הגבולות להלן שווים
		\item שני הגבולות להלן שווים ל־$f(x_0)$
	\end{enumerate}
	
	(זה בדיוק כמו להגיד את מה שכתוב במשפט למעלה)
	
	למה זה מנוסח כזה פרגמטי (עם פ' רפה)? כי לפעמים יעניין אותנו ''עד כמה $f$ רציפה בנקודה``. 
	
	\subsubsection*{מיון נקודות רציפות}
	\defi{תהא \gf ותהא $x_0 \in \R$. נניח ש־$f$ אינה רציפה בה. [מכאן, שבהכרח היא נקודת הצטברות – כי נקודה שאיננה נקודת הצטברות, היא רציפה. לכן אפשר לדבר על הגבול]. אז [הדוגמאות ל־$x_0 = 0$]: 
		\begin{itemize}
			\item אם 1-2 מתקיים (מהמיון לעיל) אז $x_0$ תקרא \textit{אי־רציפות סליקה}. \textit{לדוגמה: }
			\[ f(x) = \begin{cases}
				x^{2} & x \neq 0 \\
				67 & x = 0
			\end{cases} \]
			\item אחרת, אם רק 1 מתקיים, $x_0$ תקרא \textit{אי־רציפות מסוג ראשון}. \textit{לדוגמה: }
			\[ f(x) = \begin{cases}
				x^{2} & x > 0 \\
				x^{2} + 67 & x \le 0
			\end{cases} \]
			\item אחרת, רק 2 מתקיים, ו־$x_0$ תקרא \textit{אי־רציפות מסוג שני}. \textit{לדוגמה: }פונקציית דיריכלה, $\frac{1}{x}$. 
	\end{itemize}}
	מה המשמעות של אי־רציפות סליקה? שהפונקציה פחות או יותר רציפה בנקודה הזו, אבל ספציפית הנקודה הזו קופצת. 
	
	\theo{תהא $f \co I \to \R$ מונוטונית עולה. אז לכל $x_0 \in I$, יש ל־$f$ גבול סופי משמאל ב־$x_0$ וגם גבול סופי מימין. }
	זה למעשה משפט וויראשטראס בעבור סדרות. 
	\begin{proof}
		נסמן $A = \sup\{f(x) \mid x < x_0\}$. לכל $a \in A$, מתקיים $a \le f(x_0)$ ומהמונוטוניות של $f$. לכן $A$ חסומה. $A \neq \varnothing$ כי $x_0$ בתוך הקטע. לכן קיים ל־$A$ חסם עליון. נסמן $\ml = \sup A$. יהי $\eg > 0$. אז קיים $x < x_0$ כך ש־$f(x) > \ml - \eg$. נתבונן ב־$\dg  = x_0 - x$. יהי $x = x_0 - \dg < y < x_0$. אז $\ml - \eg < f(x) \le f(y) \le \ml < \ml + \eg$. לכן $\sof{f(y) - \ml} < \eg$. מכאן $\lim_{x \to x_0^{-}} f(x) = \ml$. בדומה יש ל־$f$ גבול מימין ב־$x_0$. 
	\end{proof}
	לפונקציה מונוטונית יש רק נקודות רציפות מסוג ראשון. מכאן שיש רק כמות בת־מנייה של נקודות רציפות. 
	
	
	

	
	

	
	
\end{document}