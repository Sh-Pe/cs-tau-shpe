%! ~~~ Packages Setup ~~~ 
\documentclass[]{article}


% Math packages
\usepackage[usenames]{color}
\usepackage{forest}
\usepackage{ifxetex,ifluatex,amsmath,amssymb,mathrsfs,amsthm,witharrows,mathtools}
\WithArrowsOptions{displaystyle}
\renewcommand{\qedsymbol}{$\blacksquare$} % end proofs with \blacksquare. Overwrites the defualts. 
\usepackage{cancel,bm}
\usepackage[thinc]{esdiff}


% tikz
\usepackage{tikz}
\newcommand\sqw{1}
\newcommand\squ[4][1]{\fill[#4] (#2*\sqw,#3*\sqw) rectangle +(#1*\sqw,#1*\sqw);}


% code 
\usepackage{listings}
\usepackage{xcolor}

\definecolor{codegreen}{rgb}{0,0.35,0}
\definecolor{codegray}{rgb}{0.5,0.5,0.5}
\definecolor{codenumber}{rgb}{0.1,0.3,0.5}
\definecolor{codeblue}{rgb}{0,0,0.5}
\definecolor{codered}{rgb}{0.5,0.03,0.02}
\definecolor{codegray}{rgb}{0.96,0.96,0.96}

\lstdefinestyle{pythonstylesheet}{
	language=Python,
	emphstyle=\color{deepred},
	backgroundcolor=\color{codegray},
	keywordstyle=\color{deepblue}\bfseries\itshape,
	numberstyle=\scriptsize\color{codenumber},
	basicstyle=\ttfamily\footnotesize,
	commentstyle=\color{codegreen}\itshape,
	breakatwhitespace=false, 
	breaklines=true, 
	captionpos=b, 
	keepspaces=true, 
	numbers=left, 
	numbersep=5pt, 
	showspaces=false,                
	showstringspaces=false,
	showtabs=false, 
	tabsize=4, 
	morekeywords={as,assert,nonlocal,with,yield,self,True,False,None,AssertionError,ValueError,in,else},              % Add keywords here
	keywordstyle=\color{codeblue},
	emph={object,type,isinstance,copy,deepcopy,zip,enumerate,reversed,list,set,len,dict,tuple,print,range,xrange,append,execfile,real,imag,reduce,str,repr,__init__,__add__,__mul__,__div__,__sub__,__call__,__getitem__,__setitem__,__eq__,__ne__,__nonzero__,__rmul__,__radd__,__repr__,__str__,__get__,__truediv__,__pow__,__name__,__future__,__all__,},          % Custom highlighting
	emphstyle=\color{codered},
	stringstyle=\color{codegreen},
	showstringspaces=false,
	abovecaptionskip=0pt,belowcaptionskip =0pt,
	framextopmargin=-\topsep, 
}
\newcommand\pythonstyle{\lstset{pythonstylesheet}}
\newcommand\pyl[1]     {{\lstinline!#1!}}
\lstset{style=pythonstylesheet}

\usepackage[style=1,skipbelow=\topskip,skipabove=\topskip,framemethod=TikZ]{mdframed}
\definecolor{bggray}{rgb}{0.85, 0.85, 0.85}
\mdfsetup{leftmargin=0pt,rightmargin=0pt,innerleftmargin=15pt,backgroundcolor=codegray,middlelinewidth=0.5pt,skipabove=5pt,skipbelow=0pt,middlelinecolor=black,roundcorner=5}
\BeforeBeginEnvironment{lstlisting}{\begin{mdframed}\vspace{-0.4em}}
	\AfterEndEnvironment{lstlisting}{\vspace{-0.8em}\end{mdframed}}


% Deisgn
\usepackage[labelfont=bf]{caption}
\usepackage[margin=0.6in]{geometry}
\usepackage{multicol}
\usepackage[skip=4pt, indent=0pt]{parskip}
\usepackage[normalem]{ulem}
\forestset{default}
\renewcommand\labelitemi{$\bullet$}
\usepackage{titlesec}
\titleformat{\section}[block]
{\fontsize{15}{15}}
{\en{\dotfill \thesection \dotfill}}
{0em}
{\MakeUppercase}
\usepackage{graphicx}
\graphicspath{ {./} }


% Hebrew initialzing
\usepackage[bidi=basic]{babel}
\PassOptionsToPackage{no-math}{fontspec}
\babelprovide[main, import, Alph=letters]{hebrew}
\babelprovide[import]{english}
\babelfont[hebrew]{rm}{David CLM}
\babelfont[hebrew]{sf}{David CLM}
\babelfont[english]{tt}{Monaspace Xenon}
\usepackage[shortlabels]{enumitem}
\newlist{hebenum}{enumerate}{1}

% Language Shortcuts
\newcommand\en[1] {\begin{otherlanguage}{english}#1\end{otherlanguage}}
\newcommand\sen   {\begin{otherlanguage}{english}}
	\newcommand\she   {\end{otherlanguage}}
\newcommand\del   {$ \!\! $}
\newcommand\ttt[1]{\en{\footnotesize\texttt{#1}\normalsize}}

\newcommand\npage {\vfil {\hfil \textbf{\textit{המשך בעמוד הבא}}} \hfil \vfil \pagebreak}
\newcommand\ndoc  {\dotfill \\ \vfil {\begin{center} {\textbf{\textit{שחר פרץ, 2024}} \\ \scriptsize \textit{נוצר באמצעות תוכנה חופשית בלבד}} \end{center}} \vfil	}

\newcommand{\rn}[1]{
	\textup{\uppercase\expandafter{\romannumeral#1}}
}

\makeatletter
\newcommand{\skipitems}[1]{
	\addtocounter{\@enumctr}{#1}
}
\makeatother

%! ~~~ Math shortcuts ~~~

% Letters shortcuts
\newcommand\N     {\mathbb{N}}
\newcommand\Z     {\mathbb{Z}}
\newcommand\R     {\mathbb{R}}
\newcommand\Q     {\mathbb{Q}}
\newcommand\C     {\mathbb{C}}

\newcommand\ml    {\ell}
\newcommand\mj    {\jmath}
\newcommand\mi    {\imath}

\newcommand\powerset {\mathcal{P}}
\newcommand\ps    {\mathcal{P}}
\newcommand\pc    {\mathcal{P}}
\newcommand\ac    {\mathcal{A}}
\newcommand\bc    {\mathcal{B}}
\newcommand\cc    {\mathcal{C}}
\newcommand\dc    {\mathcal{D}}
\newcommand\ec    {\mathcal{E}}
\newcommand\fc    {\mathcal{F}}
\newcommand\nc    {\mathcal{N}}
\newcommand\sca   {\mathcal{S}} % \sc is already definded
\newcommand\rca   {\mathcal{R}} % \rc is already definded

\newcommand\Si    {\Sigma}

% Logic & sets shorcuts
\newcommand\siff  {\longleftrightarrow}
\newcommand\ssiff {\leftrightarrow}
\newcommand\so    {\longrightarrow}
\newcommand\sso   {\rightarrow}

\newcommand\epsi  {\epsilon}
\newcommand\vepsi {\varepsilon}
\newcommand\vphi  {\varphi}
\newcommand\Neven {\N_{\mathrm{even}}}
\newcommand\Nodd  {\N_{\mathrm{odd }}}
\newcommand\Zeven {\Z_{\mathrm{even}}}
\newcommand\Zodd  {\Z_{\mathrm{odd }}}
\newcommand\Np    {\N_+}

% Text Shortcuts
\newcommand\open  {\big(}
\newcommand\qopen {\quad\big(}
\newcommand\close {\big)}
\newcommand\also  {\text{, }}
\newcommand\defi  {\text{ definition}}
\newcommand\defis {\text{ definitions}}
\newcommand\given {\text{given }}
\newcommand\case  {\text{if }}
\newcommand\syx   {\text{ syntax}}
\newcommand\rle   {\text{ rule}}
\newcommand\other {\text{else}}
\newcommand\set   {\ell et \text{ }}
\newcommand\ans   {\mathit{Ans.}}

% Set theory shortcuts
\newcommand\ra    {\rangle}
\newcommand\la    {\langle}

\newcommand\oto   {\leftarrow}

\newcommand\QED   {\quad\quad\mathscr{Q.E.D.}\;\;\blacksquare}
\newcommand\QEF   {\quad\quad\mathscr{Q.E.F.}}
\newcommand\eQED  {\mathscr{Q.E.D.}\;\;\blacksquare}
\newcommand\eQEF  {\mathscr{Q.E.F.}}
\newcommand\jQED  {\mathscr{Q.E.D.}}

\newcommand\dom   {\mathrm{dom}}
\newcommand\Img   {\mathrm{Im}}
\newcommand\range {\mathrm{range}}

\newcommand\trio  {\triangle}

\newcommand\rc    {\right\rceil}
\newcommand\lc    {\left\lceil}
\newcommand\rf    {\right\rfloor}
\newcommand\lf    {\left\lfloor}

\newcommand\lex   {<_{lex}}

\newcommand\az    {\aleph_0}
\newcommand\uaz   {^{\aleph_0}}
\newcommand\al    {\aleph}
\newcommand\ual   {^\aleph}
\newcommand\taz   {2^{\aleph_0}}
\newcommand\utaz  { ^{\left (2^{\aleph_0} \right )}}
\newcommand\tal   {2^{\aleph}}
\newcommand\utal  { ^{\left (2^{\aleph} \right )}}
\newcommand\ttaz  {2^{\left (2^{\aleph_0}\right )}}

\newcommand\n     {$n$־יה\ }

% Math A&B shortcuts
\newcommand\logn  {\log n}
\newcommand\cosx  {\cos x}
\newcommand\cost  {\cos \theta}
\newcommand\sinx  {\sin x}
\newcommand\sint  {\sin \theta}
\newcommand\tanx  {\tan x}
\newcommand\tant  {\tan \theta}
\newcommand\sex   {\sec x}
\newcommand\sect  {\sec^2}
\newcommand\cotx  {\cot x}
\newcommand\cscx  {\csc x}
\newcommand\sinhx {\sinh x}
\newcommand\coshx {\cosh x}
\newcommand\tanhx {\tanh x}

\newcommand\seq   {\overset{!}{=}}
\newcommand\sle   {\overset{!}{\le}}
\newcommand\sge   {\overset{!}{\ge}}
\newcommand\sll   {\overset{!}{<}}
\newcommand\sgg   {\overset{!}{>}}

\newcommand\h     {\hat}
\newcommand\ve    {\vec}
\newcommand\lv    {\overrightarrow}
\newcommand\ol    {\overline}

\newcommand\mlcm  {\mathrm{lcm}}

\DeclareMathOperator{\sech}   {sech}
\DeclareMathOperator{\csch}   {csch}
\DeclareMathOperator{\arcsec} {arcsec}
\DeclareMathOperator{\arccot} {arcCot}
\DeclareMathOperator{\arccsc} {arcCsc}
\DeclareMathOperator{\arccosh}{arccosh}
\DeclareMathOperator{\arcsinh}{arcsinh}
\DeclareMathOperator{\arctanh}{arctanh}
\DeclareMathOperator{\arcsech}{arcsech}
\DeclareMathOperator{\arccsch}{arccsch}
\DeclareMathOperator{\arccoth}{arccoth} 

\newcommand\dx    {\,\mathrm{d}x}
\newcommand\dt    {\,\mathrm{d}t}
\newcommand\dtt   {\,\mathrm{d}\theta}
\newcommand\df    {\mathrm{d}f}
\newcommand\dfdx  {\diff{f}{x}}
\newcommand\dit   {\limhz \frac{f(x + h) - f(x)}{h}}

\newcommand\nt[1] {\frac{#1}{#1}}

\newcommand\limz  {\lim_{x \to 0}}
\newcommand\limxz {\lim_{x \to x_0}}
\newcommand\limi  {\lim_{x \to \infty}}
\newcommand\limh  {\lim_{x \to 0}}
\newcommand\limni {\lim_{x \to - \infty}}
\newcommand\limpmi{\lim_{x \to \pm \infty}}

\newcommand\ta    {\theta}
\newcommand\ap    {\alpha}

\renewcommand\inf {\infty}
\newcommand  \ninf{-\inf}

% Combinatorics shortcuts
\newcommand\sumnk     {\sum_{k = 0}^{n}}
\newcommand\sumni     {\sum_{i = 0}^{n}}
\newcommand\sumnko    {\sum_{k = 1}^{n}}
\newcommand\sumnio    {\sum_{i = 1}^{n}}
\newcommand\sumai     {\sum_{i = 1}^{n} A_i}
\newcommand\nsum[2]   {\reflectbox{\displaystyle\sum_{\reflectbox{\scriptsize$#1$}}^{\reflectbox{\scriptsize$#2$}}}}

\newcommand\bink      {\binom{n}{k}}
\newcommand\setn      {\{a_i\}^{2n}_{i = 1}}
\newcommand\setc[1]   {\{a_i\}^{#1}_{i = 1}}

\newcommand\cupain    {\bigcup_{i = 1}^{n} A_i}
\newcommand\cupai[1]  {\bigcup_{i = 1}^{#1} A_i}
\newcommand\cupiiai   {\bigcup_{i \in I} A_i}
\newcommand\capain    {\bigcap_{i = 1}^{n} A_i}
\newcommand\capai[1]  {\bigcap_{i = 1}^{#1} A_i}
\newcommand\capiiai   {\bigcap_{i \in I} A_i}

\newcommand\xot       {x_{1, 2}}
\newcommand\ano       {a_{n - 1}}
\newcommand\ant       {a_{n - 2}}

% Other shortcuts
\newcommand\tl    {\tilde}
\newcommand\op    {^{-1}}

\newcommand\sof[1]    {\left | #1 \right |}
\newcommand\cl [1]    {\left ( #1 \right )}
\newcommand\csb[1]    {\left [ #1 \right ]}

\newcommand\bs    {\blacksquare}

\DeclareMathOperator{\frc}   {frc}

%! ~~~ Document ~~~

\author{שחר פרץ}
\title{תרגיל בית מספר 3 $\sim$ מבוא מורחב למדעי המחשב}

\begin{document}
	\maketitle
	\section{}
	\begin{enumerate}[A)]
		\item 
		\begin{enumerate}[1.]
			\item $\bm{64^{\log_4n}} = O(n^4)$. \textit{נוכיח. }צריך להוכיח קיום $a, n_0$ המקיים את אי־השוויון הבא. יהי $n \ge n_0$: 
			\[ 64^{\log_4n} \le a \cdot n^4 \iff 4^{3 \log_4n} \le an^4 \iff n^3 \le an^4 \iff n^3 = O(n^4) \]
			כאשר הטענה האחרונה ידועה מן החסמים שהוצגו בהרצאה. 
			\item $\bm{\log(n) = O(\log(\log(n^4)))}$. \textit{נסתור. }נניח בשלילה ונראה סתירה. 
			\[ \logn = O(\log(\log(n^4))) = O(\log(4\log(n))) = O(\underbrace{\log 4 }_{\mathclap{\mathrm{const.}}} + \logn) = O(\log\logn) \]
			אך מההירכיית החסמים שהוצגה בכיתה, $\log\log n$ הוא חסם אסימפטוטי תחתון ממש מ־$\logn$, ולכן לא ייתכן שיחסום אותו מלמעלה. 
			\item $\bm{\forall f_1(n) = O(g_1(n)), f_2(n) = O(g_2(n)). f_1 \circ f_2 = O((g_1 \circ g_2)(n))}$. \textit{נסתור. }נבחר: 
			\begin{alignat*}{9}
				f_1 &= 2^n, &&\quad g_1 &&= 2^n, &&&\quad n &&&= O(0.5n) \\
				f_2 &= n &&\quad g_2 &&= 0.5n &&&\quad 2^n &&&= O(2^n)
			\end{alignat*}
			נניח בשלילה שהטענה נכונה. מכאן: 
			\[ (f_1 \circ f_2)(n) = O(g_1 \circ g_2)(n) \implies 2^n = O(2^{0.5n}) \implies 2^n = O(\sqrt2^n) \]
			אך זו סתירה לטבלה שראינו לההרכיית חסמים אסימפטוטיים. 
			\item $\bm{f = O(6^n) \land g = 6^{O(n)} \implies g = O(f)}$. \textit{נסתור. }עבור $g = 6^{2n}$ (תקין כי $2n = O(n)$), וגם $6^n = O(6^n)$, יתקיים: 
			\[ g(n) = 12^n = O(6^n) = O(f(n)) \]
			לפי הטענה, אך זו סתירה להררכיית החסמים שראינו בהרצאה. 
		\end{enumerate}
		\item \begin{enumerate}[1.]
			\item יהיו $0 \le a_1, a_2, \dots $ סדרת מספרים אי־שליליים. נניח קיום קבועים $0 < b, c \le 1$ כך שלכל $n$ לפחות $n \cdot b$ מאיברי הסדרה הם בגודל של לפחות $c \cdot \max\{a_1, \dots a_n\}$. נוכיח $\sum_{i = 1}^{n}a_i = \Theta(n \cdot \max\{a_1, \dots a_n\})$. 
			
			\begin{proof}
				תהה $a_1, a_2, \dots$ סדרה של מספרים. יהי $n \in \N$. נניח קיום קבוע $b, c \in \N$ כך ש־$b \cdot n$ מאיברי הסדרה הם בגודל של לפחות $c \cdot \max\{a_1, \dots a_n\}$. בפרט, הסדרה $a_1, \dots, a_n$ תקיים את התנאי הזה. ידוע: 
				\[ \underbrace{cb }_{\mathclap{const.}}\cdot \; n \max\{a_1, \dots, a_n\} \le \sum_{i = 1}^{n} \begin{cases}
					a_i & a_i = c\max\{a_1, \dots a_n\} \\
					0 & \other
				\end{cases} \le \bm{\sum_{i = 1}^{n} a_i} \le \sum_{i = 1}^{n}\max\{a_1, \dots a_n\} = n \cdot \max\{a_1, \dots a_n\} \]
			\end{proof}
			סה"כ $\sum_{i = 1}^{n} a_i = \Theta(n \cdot \max\{a_1, \dots, a_n\})$ מהגדרת החסם כדרוש. 
		\item צ.ל. $n \logn = O(\log(n!))$. 
		\begin{proof}
			נתבונן בסדרה הבאה: 
			\[ a_i = \log i, \ 1 \le i \le n, \ \sum_{i = 0}^{n}a_i = \sumnk\log i = \log\cl{\prod_{k = 1}^{n}k} = \logn! \]
			עבור קבועים $b, c = 0.5$ יתקיים הא"ש הבאה בנכונות ל־$0.5n$ מהאיברים (בעיגול): 
			\[ \forall 0.5n \le i \le n. a_i \ge a_{0.5n} = \log(0.5n) = -1 + \log n \sge 0.5\logn = 0.5a_n = 0.5\max\{a_1, \dots, a_n\} \]
			כאשר הא"ש לעיל מתקיים לכל $n \ge 4$ מתוך מציאת נק' החיתוך בין הפונקציות בא"ש והיותן מונוטוניות עולות, אך אין זה משנה כי ניקח $n \to \inf$ בשביל לחשב חסם אסימפטוטי. סה"כ הוכחנו את התנאים לטענה לעיל כך ש־$\logn! = \Theta(n \cdot \max\{a_1, \dots a_n\}) = \Theta(n\logn)$. בפרט $n\logn = O(\logn!)$, כדרוש. 			
		\end{proof}
		\item נגדיר $p_k(n) = \sum_{i = 1}^{n}i^k$. צ.ל. לכל $k \ge 1$ מתקיים $p_k(n) = \Theta(n^{k + 1})$. 
		
		\begin{proof}
			
			נתבונן בסדרה $a_i = i^k, \ 1 \le i \le n$. נגדיר קבועים $c, b = 0.5$. עבור $n \cdot b = 0.5n$ מאיברי הסדרה (לפחות), הם $a_j \in \{a_{0.5n}, \dots a_n\}$ (בעיגול) יקיימו ממונוטוניות הסדרה: 
			$a_j \ge a_{0.5n} = (0.5n)^k \ge 0.5 \cdot n^k = c \cdot \max\{a_1, \dots a_n\}$ (כי $\forall k \ge 1. 0.5^k \le 0.5$), כלומר הסדרה עונה על התנאים של הטענה שהוכחה בסעיף 1. אזי: 
			\[ p_k(n) = \sum_{i = 1}^{n}a_i = \Theta(n \cdot \max\{a_1, \dots a_m\}) = \Theta(n \cdot n^k) = \Theta(n^{k + 1}) \]
			כדרוש. 
		\end{proof}
		\item צ.ל .$s := \sum_{i = 1}^{n}2^ii^k = \Theta(2^n\cdot n^k)$ \begin{proof} נוכיח את שני החסמים. 
		\begin{itemize}
			\item \textbf{חסם עליון: }נוכיח $\sum_{i = 1}^{n}2^ii^k = O(2^nn^k)$. ידוע: 
			\[ \forall n \ge \underbrace{1}_{n_0}. s = \sum_{i = 1}^{n}2^ii^k = \underbrace{\sum_{i = 1}^{n - 1}2^{i}(i)^k}_{\ge 0} + 2^nn^k \le \underbrace{1}_{a} \cdot 2^nn^k \]
			סה"כ הוכח החסם האסימפטוטי העליון בהתאם להגדרתו. 
			\item \textbf{חסם תחתון: }נוכיח $s = \Omega(2^nn^k)$. 
			\[ \forall n \ge \underbrace{1}_{n_0}. s = \sum_{i = 1}^{n}2^ii^k = \underbrace{\sum_{i = 1}^{n - 1}2^ii^k}_{\le 2^nn^k} + 2^nn^k \le \underbrace{2}_{a} \cdot 2^nn^k \]
			נוכיח את טענת העזר שהסתמכנו עליה, $\sum_{i = 1}^{n}2^ii^k \le 2^nn^k$. נוכיח באינדוקציה. \textit{בסיס $n = 1$: }יתקיים $2^2 \cdot 2^1 = \sum_{i = 1}^{1}2^ii^k$. \textit{צעד: }נניח באינדוקציה את נכונות הטענה על $n$ ונוכיח על $n + 1$: 
			\[ \sum_{i = 1}^{n}2^ii^k = \underbrace{\sum_{i = 1}^{n - 1}2^ii^k}_{\mathclap{\le 2^nn^k \text{ By Induction}}} + 2^nn^k \le 2 \cdot 2^nn^k = 2^{n + 1}n^k \le 2^{n + 1}(n + 1)^k \]
			כאשר המעבר האחרון נכון ממונוטוניות הפונקציה $x^k$. סה"כ האינדוקציה הושלמה, ובכך טענת העזר, כלומר האי־שוויון לעיל מתקיים ואכן $s = \Omega(2^nn^k)$ לפי הגדרה. 
		\end{itemize}
		מתוך הגדרת $\Theta$, הוכחנו $s = \Theta(2^nn^k)$ כדרוש. 
		\end{proof}
		\end{enumerate}
		\item \begin{enumerate}[1.]
			\item הלולאה תתחיל מ־$n$ ותגמור לאחר $k$ איטרציות בהן כל פעם $n$ קטן פי 2, עד ש-$n = 1$. נוכל למצוא את כמות האיטרציות הזו: 
			\[ \begin{WithArrows}
				n \cdot 0.5^k &= 1 \Arrow{$\cdot \, 0.5^{-k}$} \\
				n &= 2^k \Arrow{$\log_2$} \\
				\logn &= k
			\end{WithArrows} \]
			בפנים, יש לולאה שתלוייה ב־$n$, ושם בדיקת קיום $i$ בתוך $L$ – פעולה שבמקרה הגרוע ביותר תיאלץ לבצע מעבר על כל $L$, שגודלו $n$ (בהתאם למצב של לולאת ה־while). נחשב: 
			\[ \ans = \sum_{i = 1}^{\mathclap{\logn}}\cl{\frac{n}{2^i}\cdot (n + i) + 1} = \logn + n\sum_{i = 1}^{\mathclap{\logn}}\frac{(n + i)}{2^i} \le \logn + \sum_{i = 1}^{\inf}\frac{n + n}{2^i} = \logn + n \cdot 2n \cdot 1 = \bm{O(n^2)} \]
			\item כדי לנתח את סיבוכיות לולאת ה־while הפנימית כתלות ב־$n$, נדרוש ש־$k$ יוכפל $i$ פעמים ב־$2$ עד שיתשווה ל־$n$, ומכיוון שהוא מתחיל מ־1 אז $1 \cdot 2^i = n \implies i = \logn$. סה"כ נסכום בתהאם ל־range של הלולאות: 
			\[ \ans = \sum_{i = 0}^{n - 500}\sum_{j = 0}^{\log i}\logn = \log n \cdot \sum_{i = 0}^{n - 500}\log i \le \logn (n - 500)\logn = \bm{O(n\log^2n)} \]
			\item חיתוך של $L$ עד $i + 1$ ידרוש $i + 2$ פעולות, וכן יהיה אורכו. הוספה לרשימה, השוואה ושינוי איבר ברשימה מתבצעים ב־$O(1)$. 
			\[ \ans = \sum_{i = 0}^{n}(i + 2) = 2n + \sum_{i = 0}^{n}i = 2n + \frac{1}{2}(n^2 + n) = \bm{O(n^2)} \]
		\end{enumerate}
	\end{enumerate}
	\npage
	\section{}
	\begin{enumerate}[A)]
		\item \begin{enumerate}[a.]
			\item נרצה למצוא את הייצוב העשרוני של \ttt{0 10000000000 11000\dots0}: 
			\[ (-1)^0 \cdot 2^{2^{10} - 1023} \cdot \cl{1 + \frac{1}{2} + \frac{1}{4}} = 2 \cdot 1.75 = \bm{3} \]
			\item נרצה למצוא את הייצוג העשרוני של \ttt{1 10000000010 1000\dots00}: 
			\[ (-1)^1 \cdot 2^{2^{10} + 2^1 - 1023} \cdot \cl{1 + \frac{1}{2}} = 2^3 \cdot 1.5 = \bm{-12} \]
		\end{enumerate}
			\item למען הנוחות, נסמן $\exp = exponent, \ \frc = fraction$
			נניח $-1023 \le n \le 1024, n \in \N, \ sign = 0$. \begin{enumerate}[a.]
				\item צ.ל. בין כל זוג מספרים סמוכים בתחום $[2^n, 2^{n + 1})$ יש את אותו ההפרש. \textbf{נסמן את ההפרש הזה ב־}$\bm{\Delta^n}$. 
				\begin{proof}
					יהי $N_1, N_2 \in [2^n, 2^{n + 1})$ מספרים סמוכים שונים הניתנים לייצוג. למען הנוחות, נסמן $\Delta_{a, b} = N_b - N_a$. בה"כ נניח $N_1 < N_2$, ונוכיח $\Delta_{2, 3} = \Delta_{1, 2}$. ידוע מההרצאה שלכל float יש ייצוג יחיד, פרט ל־0. אזי, בתחום $[2^n, 2^{n + 1})$. הכרח הוא $\exp - 1023 = n$. נתבונן בייצוג הבינארי של $frc_{N_1}$. נדע, שהביט האחרון בו יהיה היחיד ששונה מזה של $\frc_{N_2}$, כי זהו השינוי המזערי ביותר שנוכל לבצע (עם שינוי של $\frac{1}{2^52}$, לפי הנוסחה לחישוב $\frc$). סה"כ: 
					\[ \Delta_{1, 2} = 1 \cdot 2^{\exp + 1023} \cdot (1 + \frc_{N_1}) - 1 \cdot 2^{n + 1023} \cdot \cl{1 + \frc_{N_1} + \frac{1}{2^{52}}} = 2^{n}\cl{\cancel{\frc_{N_1} - \frc_{N_1}} + \frac{1}{2^{52}}} = {\frac{2^{n}}{2^{52}}} = \bm{2^{n - 52}} \]
				\end{proof}
				\item נרצה למצוא פי כמה גדול הרווח בתחום $[2^{n + 1}, 2^{n + 2})$ מהרווח בתחום $[2^n, 2^{n + 1})$. נמצא את היחס: 
				\[ \frac{\Delta^{n + 1}}{\Delta^n} = \frac{2^{n + 1 - 52}}{2^{n - 52}} = 2^{n + 1 - n - 52 + 52} = \bm{2} \]
				\item אם היינו מוסיפים ביט נוסף ל־fraction, אז השינוי הכי קטן שיכולנו לבצע בו היה מוסיף $\frac{1}{2^{53}}$ לערך של $\frc$, ובהתאם לדרך בה חישבנו את ההפרש המינימלי בסעיף a – התשובה הייתה הופכת להיות $\frac{2^n}{2^{53}} = 2^{n - 53}$. הסעיף השני מדבר על יחס והחלוקה ב־$2^{53}$ מתבטלת (כמפורט בסעיף b על מספרים שונים) ולכן התוצאה תיוותר $2$. 
				\item בשני החישובים הראשונים, החישוב מתבצע בעבור אובייקט \ttt{int}, כזה שמשנה את גודלו בהתאם לכמות המקום בזכרון שיצרוך. באחרון, משום שהמספר \ttt{2.0} \ הוא מטיפוס \ttt{float}, אזי לאחר העלאה בחזקת $53$ פייתון ישמור על הטיפוס. ידוע שאפשר לייצג באופן מדוייק $2^{53}$ בזכרון בעבור $\exp = 52, \ \frc = 2^{52}$ (שאפשר לייצג בייצוג בינארי בגודל המתאים \ttt{00\dots110100}, \ttt{1\dots000} \ בהתאמה), אך נבחין לב שהרווח בתחום $[2^{52}, 2^{53})$ הוא $\Delta^{52} = 2^{52 - 52} = 2 \ge 1$, כלומר, המספר הבא שיהיה אפשר לייצג באמצעות \ttt{float}, יהיה $2^{52} + 2$ ולא $2^{52} + 1$, הוא החישוב שמתבקש מהמחשב לבצע. על־כן המחשב בוחר לעגל מטה למספר הקרוב ביותר שהוא יודע לייצג. 
			\end{enumerate}
		\end{enumerate}
	
	\npage
	\section{}
	\begin{enumerate}[A.]
		\skipitems{3}
		\item ראשית כל, נשקיע $5^k$ פעולות בשביל ליצור רשימה באורך הזה. נמיר כל מחורזת למספר (זה יקח $O(k)$ עבור $n$ מחרוזות כלומר $O(nk)$). אחרונה, תהיה לולאה שתרוץ $5^k$ פעמים ובתוכה פעולות בזמן ריצה קבוע, פרט ללולאה נוספת שתרוץ במהלך כל הרצת הקוד $n$ פעמים ללא תלות בלולאת האב שלה (היא תבצע ריצה על כל איבר בתשובה הסופית, שניתן להניח שהיא תקינה). סה"כ $O(5^k + nk + n) = O(kn + 5^k)$ תהיה הסיבוכיות. הפונקציה עומדת בדרישות הזכרון, כי כל מה שנשמר בזכרון זה זכרון העזר בגודל $5^k$, וערכים בגודל קבוע כמו משתנה החזרה של לולאות ה־\ttt{for}. 
		\skipitems{1}
		\item הפונקציה עומדת בדרישות הזמן כי היא כוללת לולאה שתרוץ $5^k$ פעמים, ובתוכה חישוב של \ttt{int\_to\_string} \ שיארך זמן ריצה של $k$ ולולאה נוספת שתרוץ ב־$O(n)$ מאיך שהיא בנויה, כלומר סה"כ הסיבוכיות היא $5^k(n + k) = O(5^knk)$ (כי כי כמובן שהחסם ההדוק קטן יותר). הפונקציה גם עומדת בדרישות הזכרון, כי פרט למשתנה העזר \ttt{ttt} \ שאורכו $k$ תווים, כל השאר יתפוש מקום קבוע בזכרון. 
	\end{enumerate}
	\npage
	\section{}
	\begin{enumerate}
		\skipitems{1}
		\item \begin{enumerate}[a.]
			\skipitems{1}
			\item בעבור כל אינדקס, נמיין את $k$ האיברים שלאחריו בזמן ממוצע $k\log k = O(1)$, כך שעבור האינדקס ה־$i$ כל האיברים לפניו כבר יהיו ממוינים (כלומר אין צורך למיין גם אותם), וסה"כ כך נמיין את הרשימה. סיבוכיות במקרה הכי גרוע $n \cdot k^2 = O(n)$ (כי $k$ קבוע וחסום ב־$100$). 
		\end{enumerate}
	\end{enumerate}
\end{document}