%! ~~~ Packages Setup ~~~ 
\documentclass[]{article}
\usepackage{lipsum}
\usepackage{rotating}


% Math packages
\usepackage[usenames]{color}
\usepackage{forest}
\usepackage{ifxetex,ifluatex,amssymb,amsmath,mathrsfs,amsthm,witharrows,mathtools,mathdots}
\usepackage{amsmath}
\WithArrowsOptions{displaystyle}
\renewcommand{\qedsymbol}{$\blacksquare$} % end proofs with \blacksquare. Overwrites the defualts. 
\usepackage{cancel,bm}
\usepackage[thinc]{esdiff}


% tikz
\usepackage{tikz}
\usetikzlibrary{graphs}
\newcommand\sqw{1}
\newcommand\squ[4][1]{\fill[#4] (#2*\sqw,#3*\sqw) rectangle +(#1*\sqw,#1*\sqw);}


% code 
\usepackage{algorithm2e}
\usepackage{listings}
\usepackage{xcolor}

\definecolor{codegreen}{rgb}{0,0.35,0}
\definecolor{codegray}{rgb}{0.5,0.5,0.5}
\definecolor{codenumber}{rgb}{0.1,0.3,0.5}
\definecolor{codeblue}{rgb}{0,0,0.5}
\definecolor{codered}{rgb}{0.5,0.03,0.02}
\definecolor{codegray}{rgb}{0.96,0.96,0.96}

\lstdefinestyle{pythonstylesheet}{
    language=Java,
    emphstyle=\color{deepred},
    backgroundcolor=\color{codegray},
    keywordstyle=\color{deepblue}\bfseries\itshape,
    numberstyle=\scriptsize\color{codenumber},
    basicstyle=\ttfamily\footnotesize,
    commentstyle=\color{codegreen}\itshape,
    breakatwhitespace=false, 
    breaklines=true, 
    captionpos=b, 
    keepspaces=true, 
    numbers=left, 
    numbersep=5pt, 
    showspaces=false,                
    showstringspaces=false,
    showtabs=false, 
    tabsize=4, 
    morekeywords={as,assert,nonlocal,with,yield,self,True,False,None,AssertionError,ValueError,in,else},              % Add keywords here
    keywordstyle=\color{codeblue},
    emph={var, List, Iterable, Iterator},          % Custom highlighting
    emphstyle=\color{codered},
    stringstyle=\color{codegreen},
    showstringspaces=false,
    abovecaptionskip=0pt,belowcaptionskip =0pt,
    framextopmargin=-\topsep, 
}
\newcommand\pythonstyle{\lstset{pythonstylesheet}}
\newcommand\pyl[1]     {{\lstinline!#1!}}
\lstset{style=pythonstylesheet}

\usepackage[style=1,skipbelow=\topskip,skipabove=\topskip,framemethod=TikZ]{mdframed}
\definecolor{bggray}{rgb}{0.85, 0.85, 0.85}
\mdfsetup{leftmargin=0pt,rightmargin=0pt,innerleftmargin=15pt,backgroundcolor=codegray,middlelinewidth=0.5pt,skipabove=5pt,skipbelow=0pt,middlelinecolor=black,roundcorner=5}
\BeforeBeginEnvironment{lstlisting}{\begin{mdframed}\vspace{-0.4em}}
    \AfterEndEnvironment{lstlisting}{\vspace{-0.8em}\end{mdframed}}


% Deisgn
\usepackage[labelfont=bf]{caption}
\usepackage[margin=0.6in]{geometry}
\usepackage{multicol}
\usepackage[skip=4pt, indent=0pt]{parskip}
\usepackage[normalem]{ulem}
\forestset{default}
\renewcommand\labelitemi{$\bullet$}
\usepackage{titlesec}
\titleformat{\section}[block]
{\fontsize{15}{15}}
{\sen \dotfill (\thesection)\she}
{0em}
{\MakeUppercase}
\usepackage{graphicx}
\graphicspath{ {./} }

\usepackage[colorlinks]{hyperref}
\definecolor{mgreen}{RGB}{25, 160, 50}
\definecolor{mblue}{RGB}{30, 60, 200}
\usepackage{hyperref}
\hypersetup{
    colorlinks=true,
    citecolor=mgreen,
    linkcolor=black,
    urlcolor=mblue,
    pdftitle={Document by Shahar Perets},
    %	pdfpagemode=FullScreen,
}


% Hebrew initialzing
\usepackage[bidi=basic]{babel}
\PassOptionsToPackage{no-math}{fontspec}
\babelprovide[main, import, Alph=letters]{hebrew}
\babelprovide[import]{english}
\babelfont[hebrew]{rm}{David CLM}
\babelfont[hebrew]{sf}{David CLM}
%\babelfont[english]{tt}{Monaspace Xenon}
\usepackage[shortlabels]{enumitem}
\newlist{hebenum}{enumerate}{1}

% Language Shortcuts
\newcommand\en[1] {\begin{otherlanguage}{english}#1\end{otherlanguage}}
\newcommand\he[1] {\she#1\sen}
\newcommand\sen   {\begin{otherlanguage}{english}}
    \newcommand\she   {\end{otherlanguage}}
\newcommand\del   {$ \!\! $}

\newcommand\npage {\vfil {\hfil \textbf{\textit{המשך בעמוד הבא}}} \hfil \vfil \pagebreak}
\newcommand\ndoc  {\dotfill \\ \vfil {\begin{center}
            {\textbf{\textit{שחר פרץ, 2025}} \\
                \scriptsize \textit{קומפל ב־}\en{\LaTeX}\,\textit{ ונוצר באמצעות תוכנה חופשית בלבד}}
    \end{center}} \vfil	}

\newcommand{\rn}[1]{
    \textup{\uppercase\expandafter{\romannumeral#1}}
}

\makeatletter
\newcommand{\skipitems}[1]{
    \addtocounter{\@enumctr}{#1}
}
\makeatother

%! ~~~ Math shortcuts ~~~

% Letters shortcuts
\newcommand\N     {\mathbb{N}}
\newcommand\Z     {\mathbb{Z}}
\newcommand\R     {\mathbb{R}}
\newcommand\Q     {\mathbb{Q}}
\newcommand\C     {\mathbb{C}}
\newcommand\One   {\mathit{1}}

\newcommand\ml    {\ell}
\newcommand\mj    {\jmath}
\newcommand\mi    {\imath}

\newcommand\powerset {\mathcal{P}}
\newcommand\ps    {\mathcal{P}}
\newcommand\pc    {\mathcal{P}}
\newcommand\ac    {\mathcal{A}}
\newcommand\bc    {\mathcal{B}}
\newcommand\cc    {\mathcal{C}}
\newcommand\dc    {\mathcal{D}}
\newcommand\ec    {\mathcal{E}}
\newcommand\fc    {\mathcal{F}}
\newcommand\zc    {\mathcal{Z}}
\newcommand\nc    {\mathcal{N}}
\newcommand\vc    {\mathcal{V}} % Vance
\newcommand\sca   {\mathcal{S}} % \sc is already definded
\newcommand\rca   {\mathcal{R}} % \rc is already definded

\newcommand\prm   {\mathrm{p}}
\newcommand\arm   {\mathrm{a}} % x86
\newcommand\brm   {\mathrm{b}}
\newcommand\crm   {\mathrm{c}}
\newcommand\drm   {\mathrm{d}}
\newcommand\erm   {\mathrm{e}}
\newcommand\frm   {\mathrm{f}}
\newcommand\nrm   {\mathrm{n}}
\newcommand\vrm   {\mathrm{v}}
\newcommand\srm   {\mathrm{s}}
\newcommand\rrm   {\mathrm{r}}

\newcommand\Si    {\Sigma}

% Logic & sets shorcuts
\newcommand\siff  {\longleftrightarrow}
\newcommand\ssiff {\leftrightarrow}
\newcommand\so    {\longrightarrow}
\newcommand\sso   {\rightarrow}

\newcommand\epsi  {\epsilon}
\newcommand\vepsi {\varepsilon}
\newcommand\vphi  {\varphi}
\newcommand\Neven {\N_{\mathrm{even}}}
\newcommand\Nodd  {\N_{\mathrm{odd }}}
\newcommand\Zeven {\Z_{\mathrm{even}}}
\newcommand\Zodd  {\Z_{\mathrm{odd }}}
\newcommand\Np    {\N_+}

% Text Shortcuts
\newcommand\open  {\big(}
\newcommand\qopen {\quad\big(}
\newcommand\close {\big)}
\newcommand\also  {\mathrm{, }}
\newcommand\defis {\mathrm{ definitions}}
\newcommand\given {\mathrm{given }}
\newcommand\case  {\mathrm{if }}
\newcommand\syx   {\mathrm{ syntax}}
\newcommand\rle   {\mathrm{ rule}}
\newcommand\other {\mathrm{else}}
\newcommand\set   {\ell et \text{ }}
\newcommand\ans   {\mathscr{A}\!\mathit{nswer}}

% Set theory shortcuts
\newcommand\ra    {\rangle}
\newcommand\la    {\langle}

\newcommand\oto   {\leftarrow}

\newcommand\QED   {\quad\quad\mathscr{Q.E.D.}\;\;\blacksquare}
\newcommand\QEF   {\quad\quad\mathscr{Q.E.F.}}
\newcommand\eQED  {\mathscr{Q.E.D.}\;\;\blacksquare}
\newcommand\eQEF  {\mathscr{Q.E.F.}}
\newcommand\jQED  {\mathscr{Q.E.D.}}

\DeclareMathOperator\dom   {dom}
\DeclareMathOperator\Img   {Im}
\DeclareMathOperator\range {range}

\newcommand\trio  {\triangle}

\newcommand\rc    {\right\rceil}
\newcommand\lc    {\left\lceil}
\newcommand\rf    {\right\rfloor}
\newcommand\lf    {\left\lfloor}
\newcommand\ceil  [1] {\lc #1 \rc}
\newcommand\floor [1] {\lf #1 \rf}

\newcommand\lex   {<_{lex}}

\newcommand\az    {\aleph_0}
\newcommand\uaz   {^{\aleph_0}}
\newcommand\al    {\aleph}
\newcommand\ual   {^\aleph}
\newcommand\taz   {2^{\aleph_0}}
\newcommand\utaz  { ^{\left (2^{\aleph_0} \right )}}
\newcommand\tal   {2^{\aleph}}
\newcommand\utal  { ^{\left (2^{\aleph} \right )}}
\newcommand\ttaz  {2^{\left (2^{\aleph_0}\right )}}

\newcommand\n     {$n$־יה\ }

% Math A&B shortcuts
\newcommand\logn  {\log n}
\newcommand\logx  {\log x}
\newcommand\lnx   {\ln x}
\newcommand\cosx  {\cos x}
\newcommand\sinx  {\sin x}
\newcommand\sint  {\sin \theta}
\newcommand\tanx  {\tan x}
\newcommand\tant  {\tan \theta}
\newcommand\sex   {\sec x}
\newcommand\sect  {\sec^2}
\newcommand\cotx  {\cot x}
\newcommand\cscx  {\csc x}
\newcommand\sinhx {\sinh x}
\newcommand\coshx {\cosh x}
\newcommand\tanhx {\tanh x}

\newcommand\seq   {\overset{!}{=}}
\newcommand\slh   {\overset{LH}{=}}
\newcommand\sle   {\overset{!}{\le}}
\newcommand\sge   {\overset{!}{\ge}}
\newcommand\sll   {\overset{!}{<}}
\newcommand\sgg   {\overset{!}{>}}

\newcommand\h     {\hat}
\newcommand\ve    {\vec}
\newcommand\lv    {\overrightarrow}
\newcommand\ol    {\overline}

\newcommand\mlcm  {\mathrm{lcm}}

\DeclareMathOperator{\sech}   {sech}
\DeclareMathOperator{\csch}   {csch}
\DeclareMathOperator{\arcsec} {arcsec}
\DeclareMathOperator{\arccot} {arcCot}
\DeclareMathOperator{\arccsc} {arcCsc}
\DeclareMathOperator{\arccosh}{arccosh}
\DeclareMathOperator{\arcsinh}{arcsinh}
\DeclareMathOperator{\arctanh}{arctanh}
\DeclareMathOperator{\arcsech}{arcsech}
\DeclareMathOperator{\arccsch}{arccsch}
\DeclareMathOperator{\arccoth}{arccoth}
\DeclareMathOperator{\atant}  {atan2} 
\DeclareMathOperator{\Sp}     {span} 
\DeclareMathOperator{\sgn}    {sgn} 
\DeclareMathOperator{\row}    {Row} 
\DeclareMathOperator{\adj}    {adj} 
\DeclareMathOperator{\rk}     {rank} 
\DeclareMathOperator{\col}    {Col} 
\DeclareMathOperator{\tr}     {tr}

\newcommand\dx    {\,\mathrm{d}x}
\newcommand\dt    {\,\mathrm{d}t}
\newcommand\dtt   {\,\mathrm{d}\theta}
\newcommand\du    {\,\mathrm{d}u}
\newcommand\dv    {\,\mathrm{d}v}
\newcommand\df    {\mathrm{d}f}
\newcommand\dfdx  {\diff{f}{x}}
\newcommand\dit   {\limhz \frac{f(x + h) - f(x)}{h}}

\newcommand\nt[1] {\frac{#1}{#1}}

\newcommand\limz  {\lim_{x \to 0}}
\newcommand\limxz {\lim_{x \to x_0}}
\newcommand\limi  {\lim_{x \to \infty}}
\newcommand\limh  {\lim_{x \to 0}}
\newcommand\limni {\lim_{x \to - \infty}}
\newcommand\limpmi{\lim_{x \to \pm \infty}}

\newcommand\ta    {\theta}
\newcommand\ap    {\alpha}

\renewcommand\inf {\infty}
\newcommand  \ninf{-\inf}

% Combinatorics shortcuts
\newcommand\sumnk     {\sum_{k = 0}^{n}}
\newcommand\sumni     {\sum_{i = 0}^{n}}
\newcommand\sumnko    {\sum_{k = 1}^{n}}
\newcommand\sumnio    {\sum_{i = 1}^{n}}
\newcommand\sumai     {\sum_{i = 1}^{n} A_i}
\newcommand\nsum[2]   {\reflectbox{\displaystyle\sum_{\reflectbox{\scriptsize$#1$}}^{\reflectbox{\scriptsize$#2$}}}}

\newcommand\bink      {\binom{n}{k}}
\newcommand\setn      {\{a_i\}^{2n}_{i = 1}}
\newcommand\setc[1]   {\{a_i\}^{#1}_{i = 1}}

\newcommand\cupain    {\bigcup_{i = 1}^{n} A_i}
\newcommand\cupai[1]  {\bigcup_{i = 1}^{#1} A_i}
\newcommand\cupiiai   {\bigcup_{i \in I} A_i}
\newcommand\capain    {\bigcap_{i = 1}^{n} A_i}
\newcommand\capai[1]  {\bigcap_{i = 1}^{#1} A_i}
\newcommand\capiiai   {\bigcap_{i \in I} A_i}

\newcommand\xot       {x_{1, 2}}
\newcommand\ano       {a_{n - 1}}
\newcommand\ant       {a_{n - 2}}

% Linear Algebra
\DeclareMathOperator{\chr}     {char}
\DeclareMathOperator{\diag}    {diag}
\DeclareMathOperator{\Hom}     {Hom}
\DeclareMathOperator{\Sym}     {Sym}
\DeclareMathOperator{\Asym}    {ASym}

\newcommand\lra       {\leftrightarrow}
\newcommand\chrf      {\chr(\F)}
\newcommand\F         {\mathbb{F}}
\newcommand\co        {\colon}
\newcommand\tmat[2]   {\cl{\begin{matrix}
            #1
        \end{matrix}\, \middle\vert\, \begin{matrix}
            #2
\end{matrix}}}

\makeatletter
\newcommand\rrr[1]    {\xxrightarrow{1}{#1}}
\newcommand\rrt[2]    {\xxrightarrow{1}[#2]{#1}}
\newcommand\mat[2]    {M_{#1\times#2}}
\newcommand\gmat      {\mat{m}{n}(\F)}
\newcommand\tomat     {\, \dequad \longrightarrow}
\newcommand\pms[1]    {\begin{pmatrix}
        #1
\end{pmatrix}}

% someone's code from the internet: https://tex.stackexchange.com/questions/27545/custom-length-arrows-text-over-and-under
\makeatletter
\newlength\min@xx
\newcommand*\xxrightarrow[1]{\begingroup
    \settowidth\min@xx{$\m@th\scriptstyle#1$}
    \@xxrightarrow}
\newcommand*\@xxrightarrow[2][]{
    \sbox8{$\m@th\scriptstyle#1$}  % subscript
    \ifdim\wd8>\min@xx \min@xx=\wd8 \fi
    \sbox8{$\m@th\scriptstyle#2$} % superscript
    \ifdim\wd8>\min@xx \min@xx=\wd8 \fi
    \xrightarrow[{\mathmakebox[\min@xx]{\scriptstyle#1}}]
    {\mathmakebox[\min@xx]{\scriptstyle#2}}
    \endgroup}
\makeatother


% Greek Letters
\newcommand\ag        {\alpha}
\newcommand\bg        {\beta}
\newcommand\cg        {\gamma}
\newcommand\dg        {\delta}
\newcommand\eg        {\epsi}
\newcommand\zg        {\zeta}
\newcommand\hg        {\eta}
\newcommand\tg        {\theta}
\newcommand\ig        {\iota}
\newcommand\kg        {\keppa}
\renewcommand\lg      {\lambda}
\newcommand\og        {\omicron}
\newcommand\rg        {\rho}
\newcommand\sg        {\sigma}
\newcommand\yg        {\usilon}
\newcommand\wg        {\omega}

\newcommand\Ag        {\Alpha}
\newcommand\Bg        {\Beta}
\newcommand\Cg        {\Gamma}
\newcommand\Dg        {\Delta}
\newcommand\Eg        {\Epsi}
\newcommand\Zg        {\Zeta}
\newcommand\Hg        {\Eta}
\newcommand\Tg        {\Theta}
\newcommand\Ig        {\Iota}
\newcommand\Kg        {\Keppa}
\newcommand\Lg        {\Lambda}
\newcommand\Og        {\Omicron}
\newcommand\Rg        {\Rho}
\newcommand\Sg        {\Sigma}
\newcommand\Yg        {\Usilon}
\newcommand\Wg        {\Omega}

% Other shortcuts
\newcommand\tl    {\tilde}
\newcommand\op    {^{-1}}

\newcommand\sof[1]    {\left | #1 \right |}
\newcommand\cl [1]    {\left ( #1 \right )}
\newcommand\csb[1]    {\left [ #1 \right ]}
\newcommand\ccb[1]    {\left \{ #1 \right \}}

\newcommand\bs        {\blacksquare}
\newcommand\dequad    {\!\!\!\!\!\!}
\newcommand\dequadd   {\dequad\duquad}

\renewcommand\phi     {\varphi}

\newtheorem{Theorem}{משפט}
\theoremstyle{definition}
\newtheorem{definition}{הגדרה}
\newtheorem{Lemma}{למה}
\newtheorem{Remark}{הערה}
\newtheorem{Notion}{סימון}
\newtheorem{Hence}{מסקנה}


\newcommand\theo  [1] {\begin{Theorem}#1\end{Theorem}}
\newcommand\defi  [1] {\begin{definition}#1\end{definition}}
\newcommand\rmark [1] {\begin{Remark}#1\end{Remark}}
\newcommand\lem   [1] {\begin{Lemma}#1\end{Lemma}}
\newcommand\noti  [1] {\begin{Notion}#1\end{Notion}}
\newcommand\hen  [1] {\begin{Hence}#1\end{Hence}}

% DS
\newcommand\limsi     {\limsup_{n \to \inf}}
\newcommand\limfi     {\liminf_{n \to \inf}}

\DeclareMathOperator\amort   {amort}
\DeclareMathOperator\worst   {worst}
\DeclareMathOperator\type    {type}
\DeclareMathOperator\cost    {cost}
\DeclareMathOperator\tim     {time}

\newcommand\dsList{
    \sFunc{List}
    \sFunc{Retrieve}
    \SetKwFunction{RetrieveFirst}{Retrieve-First}
    \SetKwFunction{RetrieveLast}{Retrieve-Last}
    \sFunc{Delete}
    \SetKwFunction{DeleteFirst}{Delete-First}
    \SetKwFunction{DeleteLast}{Delete-Last}
    \sFunc{Insert}
    \SetKwFunction{InsertFirst}{Insert-First}
    \SetKwFunction{InsertLast}{Insert-Last}
    \sFunc{Shift}
    \sFunc{Length}
    \sFunc{Concat}
    \sFunc{Plant}
    \sFunc{Split}
}
\newcommand\dsQueue{
    \sFunc{Queue}
    \sFunc{Enqueue}
    \sFunc{Head}
    \sFunc{Dequeue}
}
\newcommand\dsStack{
    \sFunc{Stack}
    \sFunc{Push}
    \sFunc{Top}
    \sFunc{Pop}
}
\newcommand\dsVector{
    \sFunc{Vector}
    \sFunc{Get}
    \sFunc{Set}
}
\newcommand\dsGraph{
    \sFunc{Graph}
    \sFunc{Edge}
    \SetKwFunction{AddEdge}{Add-Edge}
    \SetKwFunction{RemoveEdge}{Remove-Edge}
    \sFunc{InDeg} \sFunc{OutDeg}
}
\newcommand\importDs{
    \dsList
    \dsQueue
    \dsStack
    \dsVector
    \dsGraph
    \SetKwData{error}{\color{codered}error}
    \SetKwInOut{Input}{input}
    \SetKwInOut{Output}{output}
    \SetKwRepeat{Do}{do}{while}
    \SetKwData{Null}{\color{codeblue}null}
}


% Algorithems
\newcommand\sFunc [1] {\SetKwFunction{#1}{#1}}
\newcommand\sData [1] {\SetKwData{#1}{#1}}
\newcommand\sIO   [1] {\SetKwInOut{#1}{#1}}
\newcommand\ttt   [1] {\sen \texttt{#1} \she\,}
\newcommand\io    [2] {\Input{#1}\Output{#2}\BlankLine}

%! ~~~ Document ~~~

\author{שחר פרץ}
\title{\textit{אלגברה לינארית 2א 13}}
\begin{document}
    \maketitle
    \textbf{מרצה: }בן בסקין
    \subsection{צורת ג'ודן למטריצה כללית}
    
    צורת ג'ורדן לט''ל כללית: נניח ש־$f_T(x)$ מתפצל לחלוטין. כלומר
    \[ f_T(x) = \prod_{j}(x - \lg_j)^{n_j} = \prod_{i = 1}^{k}f_{|_{n_i}}(x) \]
    (הערה: $u_i$ מ''ו למרות שהשתמשתי ב־$u$ קטנה)
    כאשר $u_i$ האי־פריקים ביחס ל־$T$, ו־$T$ שמורים. היות שהם אי פריקים $f_{T|_{u_i}} = (x - \lg)^{n}$. נגדיר $S = T - \lg I$. אז $u_i$ – $T$־אינ' אממ הוא $S$־אינ'. 
    . אז $S_{|_{u_i}}$ היא ניל'. לכן ל־$u_i$ יש בסיס שרשראות $B$ שעבורו $[S_{|{u_i}}]_B$ מורכבת מבלוקי ג'ורדן ניל' כלומר: 
    \[ \csb{S_{|{u_i}}}_B = \pms{\square \\ & \square \\ && \ddots \\ &&&& \square} \]
    כאשר כל $\square$ מהצורה $J_n(0) \in M_n(\F)$. אז: 
    \[ \csb{T_{|{u_i}}}_B = \diag\{\square \dots \square\} \]
    כאשר כל בלוק מהצורה $J_n(\lg)$. 
    
    (המרצה לא כתב את זה אז אני מוסיף משהו משלי): ומכאן צורת הג'ורדן של המטריצה הזו זה פשוט בלוקים של הצמצומים בבסיס $B$ על גבי עוד מטריצת בלוקים. 
    
    \theo{צורת ג'ורדן היא יחידה עבור סדר הבלוקים. }אסטרטגיית הוכחה: ניקח צורת ג'ורדן עבור $T \co V \to V$ ונראה שהיא נקבעת מ־$T, V$ בלבד. 
    \begin{proof}
        תהא צורת ג'ורדן עבור $T$ תהא צורת ג'ורדן עבור $T$. קיים בסיס $B$ שעבורו: 
        \[ [T]_B = \diag\{\square_{\lg_1} \dots \square_{\lg_k}\} \]
        כאשר $\square_{\lg_i}$ זו דרך מוזרה לכתוב $J_k(\lg)$. 
        אז: 
        \[ V = \bigoplus_{i = 1}^{k} u_i = \bigoplus \bar v_{\lg}, \ \bar v_{\lg} = \bigoplus_{i = s}^{\ml} u_i \]
        כאשר $\bar v_{\lg}$ הוא סכום של אי־פריקים שעבורם $T - \lg I$ ניל'. תזכורת: 
        \[ \tl v_{\lg} := \bar v_{\lg} := \{v \in V \mid \exists n \in \N \co (T - \lg I)^nv = 0\} \]
        
        מה ניתן להגיד על הממדים של ה־$u_i$־ים שמרכיבים את $\tl v_\lg$? הממדים שלהם נקבעים ביחידות, עד כדי סדר, כי היות ש־$u_i$ הם $T$־אינ' הם גם $(T - \lg I)$־אינ', ולכן $[S_{|\tl v_\lg}]_{B_\lg}$ היא כ'ורדן ניל' ואז: 
        \[ \csb{T_{|\tl v_\lg}}_{B_\lg} = \csb{S_{|\tl v_\lg}}_{B_\lg} \]
    \end{proof}
    
    הגיון: המרחבים $v_\lg$ נקבעים ביחידות ללא תלות בפירוק שבחרנו. 
    
    הגיון אחר: כל בלוק מורכבת מהעתקות שבהן $T - \lg I$ ניל' (פירוק פרימרי). 
    
    הערה: בהוכחה בסיכום צריך להראות שה־$\Sp$ של הבלוקים הוא באמת $\tl v_\lg$. 
    
    הערה שלי/מרצה ביחס ללמה צריך את זה: כי באיזשהו מקום אם נבחר בסיסים שונים לפירוק אז יכול להיות שדברים מתחרבשים. 
    
    \subsection{חרבושים של סוף נושא}
    
    \textit{הסיפור של מה שעשינו עד עכשיו: }אנחנו חוקרים אופרטורים לינאריים, בצורה שתהיה נוחה להעלות את האופרטור בחזקה. הגענו למסקנה שהכי נוח כשזה לכסין. כשזה קורה, אנחנו יודעים איך לפרק. ראינו כמה אפיונים לזה – גיאומטרי, אלגברי וכו'. ניסינו לעשות מטריצה עם בלוקים על האלכסון במקום, לשם כך, נסתכל על המרחבים שרלוונטיים לבלוקים האלו בלבד. הבנו שבמקום לחקור את ה־$T$־אינ', נחקור את ה־$S$־אינ' (הניל' כמו שהגדרתי למעלה). הבנו שהם מורכבים מבלוקי ג'ורדן ניל' אלמטריים, עד לכדי סדר, ואז הרחבנו לצורה הכללית. עברנו דרך חוגים רק כדי להגיד שחוג הפולינום הוא תחום ראשי, ע''מ שנוכל להגדיר פולינום מינימלי המחלק כל פולינום אחר. לא באמת היה צריך חוגים. סתם המרצה רצה לרצוח אותנו. כל הדיבורים על פולינום מינימלי בזכות משפט קיילי־המילטון. 
    
    בסיכום אחר שיעלה למודל, [הזהרת הרבה דברים שהמרצה אמר בעפ ולא באמת הבנתי] מתחילים מלפרק את המרחב למרחבים $T$־ציקליים שלכולם יש פולינום אופייני משל עצמם. הראינו שאם נציב את האופרטור בפולינום האופייני של המטריצה המצורפת זה יתאפס (מה? איפה עשיתי את זה?). ומכאן הפולינום המינימלי של אופרטור הצמצום על המרחב הציקלי מחלק את הפולינום האופייני של ההעתקה שלו. המרצה: את מי איבדתי בשלב הזה? [כולנו]
    
    עכשיו הוא אומר להראות דרך אחרת לפתח צורת ג'ורדן: בגלל ש־$\zc(T, V) \subseteq V$ תמ''ו, ונוכל לקחת $\zc(T, V) \oplus W = V$ (נסמן $\zc(T, V) = U$) אז $f_T(x) = f_{T_{|U}} \cdot f_{T_{|W}}$ (סוף סוף משפט טרוויאלי) והמטריצה המצורפת האקראית ההיא ש־$f_{T_{|U}}$ הפולינום המינילי גם מאפס את $T_{|U}$ 
    והוא שווה ל־$\prod f_{T_{|U_i}}$ כלשהם. מהכיוון הזה אפשר להראות גם את קיילי המילטון, בלי לעבור דרך פיצול מקרים למשולשית/לא משולשית ומשום מה הרחבת שדות באמצע שאיכשהו גם את זה הוכחנו. 
    
    \section{\en{Bilinear Forms}}
    נושא חדששששש סוף סוף הזדמנות לא להיות out of the loop לחלוטין בכל מה שקשור ללקשור פולינומים בשרשראות מרחב ציקלי משהו
    
    \defi{יהי $V$ מ''ו מעל $\F$. פונקציונל לינארי $\vphi$ מעל $V$ הוא $\vphi \co V \to \F$. }
    \defi{יהיו $V, W$ מ''וים מעל $\F$. תבנית בי־לינארית על $V \times W$ הינה העתקה $f \co V \times W \to \F$ כך ש־$\forall v_0 \in V \ \forall w_0 \in W$ כך שהעתקות $w \mapsto f(v_0, w), \ v \mapsto (v, w_0)$ הן פונקציונליים לינאריים. }
    \theo{באופן שקול: \hfill $\forall v \in V, \ w \in W, \ \ag \in \F$: 
        \[ \begin{aligned}
            \forall v_1, v_2 \in V \co f(v_1 + v_2, w) &= f(v, w) + f(v_2, w) \\
            \forall w_1, w_2 \in W \co f(v, w_1, w_2) &= f(v, w_1) + f(v, w_2) \\
            f(\ag v, w) &= \ag f(v, w) = f(v, \ag w)
        \end{aligned} \]}
    
    בסופו של יום נתמקד בסוג מסויים של העתקות בילינאריות, הן מכפלות פנימיות. 
    
    בשביל העתקות $n$־לינאריות צריך טנסור $n - 1$ ממדי. זה לא נעים ויודעים מעט מאוד על האובייקטים הללו. בפרט, בעבור העתקה בי־לינארית נראה שנוכל לייצג אותה באמצעות מטריצות. בלי טנסורים ובלגנים – שזה נחמד, וזו הסיבה שאנו מתעסקים ספציפית עם העתקות בילינאריות. 
    
    \textbf{דוגמאות. }
    \begin{enumerate}
        \item תבנית ה־$0$: \hfill $\forall v,w \co f(v, w) = 0$ 
        \item נגדיר $V = W = \R^2$, אז \hfill $f\big(\binom{x}{y}, \binom{u}{v}\big) = 2xu + 5xv - 12yu$
        \item (חשוב)  על $\F^n$ \defi{לכל שדה $\F$ מוגדרת \textit{התבנית הבי־לינארית הסטנדרטית} היא: 
        \[ f\cl{\pms{x_1 \\ \vdots \\ x_n}, \pms{y_1 \\ \vdots \\ y_n}} = \sum_{i = 1}^{n}x_iy_i \]}
        \item יהיו $\phi \co V \to \F, \ \psi \co W \to \F$ פונקציונליים לינאריים \hfill $f(v, w) = \phi(v) \cdot \psi(w)$
        \item הכללה של 4: יהיו $\phi_1 \dots \phi_k \co V \to \F$ פונקציונליים לינאיריים וכן $\psi_1 \dots \psi_k \co W \to \F$ פונקציונליים לינארים. אז ההעתקה הבאה בילינארית: \hfill $f(v, w) = \sum_{i = 1}^{k}\phi_i(v) \psi_i(w)$
    \end{enumerate}
    הרעיון: ברגע שנקבע וקטור ספציפי נקבל לינאריות של הוקטור השני. 
    
    במקרה ש־$\F = \R$ לעיל, התבנית הבילינארית הסטנדרטית ``משרה'' את הגיאומטריה האוקלידית. כלומר $v \perp u \iff f(v, u) = 0$. 
    
    \textit{הערה: }בעתיד נראה שכל תבנית בילינארית נראית כמו 5. 
    
    \theo{נסמן את מרחב התבניות הבי־לינאריות על $V \times W$ בתור $B(V, W)$. זהו מ''ו מעל $\F$}
    אני ממש לא עומד להגדיר את החיבור והכפל בסקלר של המשפט הקודם כי זה טרוויאלי והמרצה כותב את זה בעיקר בשביל להטריל אותנו. 
    
    \textbf{דוגמה חשובה אחרת. }\theo{נסמן ש־$\dim V = n, \ \dim W = m$ ותהי $A \in M_{n \times m}(\F)$. יהי $\ac$ בסיס ל־$B$, $\bc$ בסיס ל־$W$ (``זה $\ac$ mathcal, אתם תסתדרו'' – המרצה ברגע שיש לו שני $A$־ים על הלוח)\[ f(u, w) = [v]_\ac^T \cdot A[w]_{\bc} \] העתקה בילינארית. }
    \begin{proof}
        נקבע $v$ כלשהו: 
        \[ [v]_\ac^T \cdot A = B \in M_{1 \times m}, \ g(w) = f(v, w), \ g(w_1 + w_2) = B[w_1 + w_2]_\bc = B[w_1]_\bc + B[w_2]_\bc \]
        כנ''ל עבור כפל בסקלר. 
        נקבע $w$, אז $C = A[w]_\bc \in M_{n \times 1}(\F)$: 
        \[ h(v) = f(v, w) \ h(v) = [v]_B^T \cdot C, \ h(v_1 + v_2) = [v_1 + v_2]_\bc^T = ([v_1]_\bc^T + [v_2]_\bc^T)C = h(v_1) + h(v_2) \]
    \end{proof}
    בסיכום של הקורס לא הניחו שהעברת וקטור לוקטור קורדינאטות היא ט''ל מסיבה כלשהי. 
    \defi{בהינתן תבנית בי־לינ' $f \co V \times W \to \F$ ונניח ש־$\ac$ בסיס ל־$V$, $\bc$ בסיס ל־$W$. נגדיר את המטריצה המייצגת את $f$ ביחס לבסיסים $\ac, \bc$ ע''י $A \in M_{n \times m}(\F)$ כאשר $(A)_{ij} = f(v_i, w_j)$ (נסמן $\ac = (v_i)_{i =1}^n \ \bc = (w_i)_{i = 1}^{m}$)}
    
    \theo{$f(v, w) = [v]_{\ac}^T A [w]_B$}
    \begin{proof}
        
        קיימים ויחידים $\ag_1 \dots \ag_n, \ \bg_1 \dots \bg_m \in \F$ כך ש־$v = \sum \ag_i v_i, \ w = \sum b_iw_i$. 
        כלומר: 
        \[ [v]_\ac^T = (\ag_1 \dots \ag_n), \ [w]_B = \pms{\bg_1 \\ \vdots \\ \bg_m} \]
        ומכאן פשוט נזרוק אלגברה: 
        \begin{align*}
            f(v, w) = &f\cl{\sum_{i = 1}^{n}\ag_i v_i, \ w} = \sum_{i = 1}^{n}\ag_i f(v_i, \ w) \\
            = &\sum_{i = 1}^{n}\ag_i f\cl{v, \ \sum_{j = 1}^{n}\bg_j w_j} = \sum_{i = 1}^{n}\ag_i \sum_{j = 1}^{m}\bg_j f(v_i, w_j) \\
            = &\sum_{\mathclap{i, j \in [n] \times [m]}} \ag_i f(v_i, w_j)\bg_j \\
            = &\cl{\sum_{i = 1}^{n}\ag_i a_{i1}, \sum_{i = 1}^{n}\ag_i a_{i2}, \vdots, \sum_{i = 1}^{n}\ag_i a_{im}}\pms{\bg_1 \\ \vdots \\ \bg_m} \\
            = &\,(\ag_1 \dots \ag_n)\pms{a_{11} &\cdots &a_{1m} \\ \vdots && \vdots \\ a_{n1} & \cdots & a_{nm}}\pms{b_1 \\ \vdots \\ b_{m}} 
        \end{align*}
    \end{proof}
    
    נאמץ לסיכום הזה את הסימון $[f]_{\ac, \bc}$ עבור המטריצה המייצגת של $f$ בי־לינארית. 
    \theo{עם עותם (ככה המרצה כתב) סימונים כמו קודם: 
    \[ \psi \co B(v, w) \to M_{n \times m}(\F), \ f \mapsto [f]_{\ac, \bc} \]
    אז $\psi$ איזו'. }
    \begin{proof}
        נסמן את $[f]_{\ac, \bc} = A$ ואת $[g]_{\ac, \bc} = B$. אז: 
        \begin{itemize}
            \item \textbf{לינאריות. }
                     \[ (\ps(f + g))_{ij} = (f + g)(v_i, w_j) = f(v_i, w_j) + g(v_i, w_j) = (A)_{ij} + (B)_{ij} = (\psi(f))_{ij} + (\psi(g))_{ij} \implies \psi(f + g) = \psi(f) + \psi(g) \]
            באופן דומה בעבור כפל בסקלר: 
            \[ (\psi(\ag f))_{ij} = \ag f(v_i, w_j) = \ag (\psi(f))_{ij} \implies \psi(\ag f) = \ag \psi(f) \]
            \item \textbf{חח''ע. }תהי $f \in \ker \psi$, אז: $\psi(f) = 0 \in M_{n \times m} \implies \forall i,j \in [n] \times [m]\co f(v_i, w_j) = 0$ ולכן $\forall v \in V, w \in W \co f(v, w) = \sum_{\mathclap{i, j \in [n] \times [m]}}\ag_i f(v_i, w_j)\bg_j = 0$
            (עם אותם הסימונים כמו קודם)
            \item \textbf{על. }תהי $A \n M_{n \times m}(\F)$. נגדיר $f(v, w) = [v]_\ac^T A[w]_\bc$ ואכן $f(v_i, w_j) = e_i^T A e_j = (A)_{ij}$. 
        \end{itemize}
    \end{proof}
    
    \theo{יהיו $V, W$ מ''וים מעל $\F$ נניח $\ac, \ac' \subseteq V$ בסיסים של $V$ וכן $\bc, \bc' \subseteq W$ בסיסים של $W$. תהי $f \in B(V, W)$. 
        תהי המייצגת של $f$ לפי $\ac, \bc$ היא $A$ ותהי $A'$ המייצגת בבסיסים $\ac', \bc'$. תהי $P$ מטריצת המעבר מ־$\ac$ ל־$\ac'$ ו־$Q$ מטריצת המעבר מ־$\bc$ ל־$\bc'$, אז $A' = P^T AQ$. 
        }
    \begin{proof}
        ידוע: 
        \[ P[v]_{\ac} = [v]_{\ac'}, \ Q[w]_\bc = [w]_{\bc'} \]
        מצד אחד: 
        \[ f(v, w) = [v]_\ac^T = [v]_{\ac}^T A [w]_\bc = (P[v]_{\ac '})^TAQ[w]_{\bc'} = [v]_{\ac'}^T \, P^T A Q \, [w]_{\bc'} \implies A' = P^T A Q \]
        כדרוש. 
    \end{proof}
    
    \defi{עבור $f \in B(V, W)$ נגדיר את $\rk f = \rk A$ כאשר $A$ מייצגת אותהת ביחס לבסיסים כלשהם. }
    \theo{$\rk f$ מוגדר היטב}
    \begin{proof}
        כפל בהפיכה לא משנה את דרגת המטריצה
    \end{proof}
    
    \begin{Hence}
        תהא $f \in B(V, W)$ ונניח $\rk f = r$. אז קיימים בסיסים $\ac, \bc$ של $W, V$ בהתאמה כך ש־$[f]_{\ac, \bc} = \binom{I_R \, 0}{0\,\,\,\, 0}$
    \end{Hence}
    הרעיון הוא לדרג את כל כיוון, שורות באמצעות transpose ועמודות באמצעות המטריצה השנייה. אפשר גם לקבע בסיס, ולדרג שורות ועמודות עד שיוצאים אפסים (הוכחה לא נראתה בכיתה). 
    
    ``חצי השעה הזו גרמה לי לשנוא מלבנים בצורה יוקדת'' – מעתה ואילך נתעסק במקרה בו $V = W$. נשתמש בבסיס יחיד. 
    
    \ndoc
\end{document}