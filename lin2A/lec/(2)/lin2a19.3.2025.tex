%! ~~~ Packages Setup ~~~ 
\documentclass[]{article}
\usepackage{lipsum}
\usepackage{rotating}


% Math packages
\usepackage[usenames]{color}
\usepackage{forest}
\usepackage{ifxetex,ifluatex,amsmath,amssymb,mathrsfs,amsthm,witharrows,mathtools,mathdots}
\WithArrowsOptions{displaystyle}
\renewcommand{\qedsymbol}{$\blacksquare$} % end proofs with \blacksquare. Overwrites the defualts. 
\usepackage{cancel,bm}
\usepackage[thinc]{esdiff}


% tikz
\usepackage{tikz}
\usetikzlibrary{graphs}
\newcommand\sqw{1}
\newcommand\squ[4][1]{\fill[#4] (#2*\sqw,#3*\sqw) rectangle +(#1*\sqw,#1*\sqw);}


% code 
\usepackage{listings}
\usepackage{xcolor}

\definecolor{codegreen}{rgb}{0,0.35,0}
\definecolor{codegray}{rgb}{0.5,0.5,0.5}
\definecolor{codenumber}{rgb}{0.1,0.3,0.5}
\definecolor{codeblue}{rgb}{0,0,0.5}
\definecolor{codered}{rgb}{0.5,0.03,0.02}
\definecolor{codegray}{rgb}{0.96,0.96,0.96}

\lstdefinestyle{pythonstylesheet}{
	language=Java,
	emphstyle=\color{deepred},
	backgroundcolor=\color{codegray},
	keywordstyle=\color{deepblue}\bfseries\itshape,
	numberstyle=\scriptsize\color{codenumber},
	basicstyle=\ttfamily\footnotesize,
	commentstyle=\color{codegreen}\itshape,
	breakatwhitespace=false, 
	breaklines=true, 
	captionpos=b, 
	keepspaces=true, 
	numbers=left, 
	numbersep=5pt, 
	showspaces=false,                
	showstringspaces=false,
	showtabs=false, 
	tabsize=4, 
	morekeywords={as,assert,nonlocal,with,yield,self,True,False,None,AssertionError,ValueError,in,else},              % Add keywords here
	keywordstyle=\color{codeblue},
	emph={var, List, Iterable, Iterator},          % Custom highlighting
	emphstyle=\color{codered},
	stringstyle=\color{codegreen},
	showstringspaces=false,
	abovecaptionskip=0pt,belowcaptionskip =0pt,
	framextopmargin=-\topsep, 
}
\newcommand\pythonstyle{\lstset{pythonstylesheet}}
\newcommand\pyl[1]     {{\lstinline!#1!}}
\lstset{style=pythonstylesheet}

\usepackage[style=1,skipbelow=\topskip,skipabove=\topskip,framemethod=TikZ]{mdframed}
\definecolor{bggray}{rgb}{0.85, 0.85, 0.85}
\mdfsetup{leftmargin=0pt,rightmargin=0pt,innerleftmargin=15pt,backgroundcolor=codegray,middlelinewidth=0.5pt,skipabove=5pt,skipbelow=0pt,middlelinecolor=black,roundcorner=5}
\BeforeBeginEnvironment{lstlisting}{\begin{mdframed}\vspace{-0.4em}}
	\AfterEndEnvironment{lstlisting}{\vspace{-0.8em}\end{mdframed}}


% Deisgn
\usepackage[labelfont=bf]{caption}
\usepackage[margin=0.6in]{geometry}
\usepackage{multicol}
\usepackage[skip=4pt, indent=0pt]{parskip}
\usepackage[normalem]{ulem}
\forestset{default}
\renewcommand\labelitemi{$\bullet$}
\usepackage{titlesec}
\titleformat{\section}[block]
{\fontsize{15}{15}}
{\sen \dotfill (\thesection)\she}
{0em}
{\MakeUppercase}
\usepackage{graphicx}
\graphicspath{ {./} }


% Hebrew initialzing
\usepackage[bidi=basic]{babel}
\PassOptionsToPackage{no-math}{fontspec}
\babelprovide[main, import, Alph=letters]{hebrew}
\babelprovide[import]{english}
\babelfont[hebrew]{rm}{David CLM}
\babelfont[hebrew]{sf}{David CLM}
\babelfont[english]{tt}{Monaspace Xenon}
\usepackage[shortlabels]{enumitem}
\newlist{hebenum}{enumerate}{1}

% Language Shortcuts
\newcommand\en[1] {\begin{otherlanguage}{english}#1\end{otherlanguage}}
\newcommand\sen   {\begin{otherlanguage}{english}}
	\newcommand\she   {\end{otherlanguage}}
\newcommand\del   {$ \!\! $}

\newcommand\npage {\vfil {\hfil \textbf{\textit{המשך בעמוד הבא}}} \hfil \vfil \pagebreak}
\newcommand\ndoc  {\dotfill \\ \vfil {\begin{center} {\textbf{\textit{שחר פרץ, 2024}} \\ \scriptsize \textit{נוצר באמצעות תוכנה חופשית בלבד}} \end{center}} \vfil	}

\newcommand{\rn}[1]{
	\textup{\uppercase\expandafter{\romannumeral#1}}
}

\makeatletter
\newcommand{\skipitems}[1]{
	\addtocounter{\@enumctr}{#1}
}
\makeatother

%! ~~~ Math shortcuts ~~~

% Letters shortcuts
\newcommand\N     {\mathbb{N}}
\newcommand\Z     {\mathbb{Z}}
\newcommand\R     {\mathbb{R}}
\newcommand\Q     {\mathbb{Q}}
\newcommand\C     {\mathbb{C}}

\newcommand\ml    {\ell}
\newcommand\mj    {\jmath}
\newcommand\mi    {\imath}

\newcommand\powerset {\mathcal{P}}
\newcommand\ps    {\mathcal{P}}
\newcommand\pc    {\mathcal{P}}
\newcommand\ac    {\mathcal{A}}
\newcommand\bc    {\mathcal{B}}
\newcommand\cc    {\mathcal{C}}
\newcommand\dc    {\mathcal{D}}
\newcommand\ec    {\mathcal{E}}
\newcommand\fc    {\mathcal{F}}
\newcommand\nc    {\mathcal{N}}
\newcommand\vc    {\mathcal{V}} % Vance
\newcommand\sca   {\mathcal{S}} % \sc is already definded
\newcommand\rca   {\mathcal{R}} % \rc is already definded

\newcommand\prm   {\mathrm{p}}
\newcommand\arm   {\mathrm{a}} % x86
\newcommand\brm   {\mathrm{b}}
\newcommand\crm   {\mathrm{c}}
\newcommand\drm   {\mathrm{d}}
\newcommand\erm   {\mathrm{e}}
\newcommand\frm   {\mathrm{f}}
\newcommand\nrm   {\mathrm{n}}
\newcommand\vrm   {\mathrm{v}}
\newcommand\srm   {\mathrm{s}}
\newcommand\rrm   {\mathrm{r}}

\newcommand\Si    {\Sigma}

% Logic & sets shorcuts
\newcommand\siff  {\longleftrightarrow}
\newcommand\ssiff {\leftrightarrow}
\newcommand\so    {\longrightarrow}
\newcommand\sso   {\rightarrow}

\newcommand\epsi  {\epsilon}
\newcommand\vepsi {\varepsilon}
\newcommand\vphi  {\varphi}
\newcommand\Neven {\N_{\mathrm{even}}}
\newcommand\Nodd  {\N_{\mathrm{odd }}}
\newcommand\Zeven {\Z_{\mathrm{even}}}
\newcommand\Zodd  {\Z_{\mathrm{odd }}}
\newcommand\Np    {\N_+}

% Text Shortcuts
\newcommand\open  {\big(}
\newcommand\qopen {\quad\big(}
\newcommand\close {\big)}
\newcommand\also  {\text{\en{, }}}
\newcommand\defi  {\text{\en{ definition}}}
\newcommand\defis {\text{\en{ definitions}}}
\newcommand\given {\text{\en{given }}}
\newcommand\case  {\text{\en{if }}}
\newcommand\syx   {\text{\en{ syntax}}}
\newcommand\rle   {\text{\en{ rule}}}
\newcommand\other {\text{\en{else}}}
\newcommand\set   {\ell et \text{ }}
\newcommand\ans   {\mathscr{A}\!\mathit{nswer}}

% Set theory shortcuts
\newcommand\ra    {\rangle}
\newcommand\la    {\langle}

\newcommand\oto   {\leftarrow}

\newcommand\QED   {\quad\quad\mathscr{Q.E.D.}\;\;\blacksquare}
\newcommand\QEF   {\quad\quad\mathscr{Q.E.F.}}
\newcommand\eQED  {\mathscr{Q.E.D.}\;\;\blacksquare}
\newcommand\eQEF  {\mathscr{Q.E.F.}}
\newcommand\jQED  {\mathscr{Q.E.D.}}

\DeclareMathOperator\dom   {dom}
\DeclareMathOperator\Img   {Im}
\DeclareMathOperator\range {range}

\newcommand\trio  {\triangle}

\newcommand\rc    {\right\rceil}
\newcommand\lc    {\left\lceil}
\newcommand\rf    {\right\rfloor}
\newcommand\lf    {\left\lfloor}

\newcommand\lex   {<_{lex}}

\newcommand\az    {\aleph_0}
\newcommand\uaz   {^{\aleph_0}}
\newcommand\al    {\aleph}
\newcommand\ual   {^\aleph}
\newcommand\taz   {2^{\aleph_0}}
\newcommand\utaz  { ^{\left (2^{\aleph_0} \right )}}
\newcommand\tal   {2^{\aleph}}
\newcommand\utal  { ^{\left (2^{\aleph} \right )}}
\newcommand\ttaz  {2^{\left (2^{\aleph_0}\right )}}

\newcommand\n     {$n$־יה\ }

% Math A&B shortcuts
\newcommand\logn  {\log n}
\newcommand\logx  {\log x}
\newcommand\lnx   {\ln x}
\newcommand\cosx  {\cos x}
\newcommand\cost  {\cos \theta}
\newcommand\sinx  {\sin x}
\newcommand\sint  {\sin \theta}
\newcommand\tanx  {\tan x}
\newcommand\tant  {\tan \theta}
\newcommand\sex   {\sec x}
\newcommand\sect  {\sec^2}
\newcommand\cotx  {\cot x}
\newcommand\cscx  {\csc x}
\newcommand\sinhx {\sinh x}
\newcommand\coshx {\cosh x}
\newcommand\tanhx {\tanh x}

\newcommand\seq   {\overset{!}{=}}
\newcommand\slh   {\overset{LH}{=}}
\newcommand\sle   {\overset{!}{\le}}
\newcommand\sge   {\overset{!}{\ge}}
\newcommand\sll   {\overset{!}{<}}
\newcommand\sgg   {\overset{!}{>}}

\newcommand\h     {\hat}
\newcommand\ve    {\vec}
\newcommand\lv    {\overrightarrow}
\newcommand\ol    {\overline}

\newcommand\mlcm  {\mathrm{lcm}}

\DeclareMathOperator{\sech}   {sech}
\DeclareMathOperator{\csch}   {csch}
\DeclareMathOperator{\arcsec} {arcsec}
\DeclareMathOperator{\arccot} {arcCot}
\DeclareMathOperator{\arccsc} {arcCsc}
\DeclareMathOperator{\arccosh}{arccosh}
\DeclareMathOperator{\arcsinh}{arcsinh}
\DeclareMathOperator{\arctanh}{arctanh}
\DeclareMathOperator{\arcsech}{arcsech}
\DeclareMathOperator{\arccsch}{arccsch}
\DeclareMathOperator{\arccoth}{arccoth}
\DeclareMathOperator{\atant}  {atan2} 

\newcommand\dx    {\,\mathrm{d}x}
\newcommand\dt    {\,\mathrm{d}t}
\newcommand\dtt   {\,\mathrm{d}\theta}
\newcommand\du    {\,\mathrm{d}u}
\newcommand\dv    {\,\mathrm{d}v}
\newcommand\df    {\mathrm{d}f}
\newcommand\dfdx  {\diff{f}{x}}
\newcommand\dit   {\limhz \frac{f(x + h) - f(x)}{h}}

\newcommand\nt[1] {\frac{#1}{#1}}

\newcommand\limz  {\lim_{x \to 0}}
\newcommand\limxz {\lim_{x \to x_0}}
\newcommand\limi  {\lim_{x \to \infty}}
\newcommand\limh  {\lim_{x \to 0}}
\newcommand\limni {\lim_{x \to - \infty}}
\newcommand\limpmi{\lim_{x \to \pm \infty}}

\newcommand\ta    {\theta}
\newcommand\ap    {\alpha}

\renewcommand\inf {\infty}
\newcommand  \ninf{-\inf}

% Combinatorics shortcuts
\newcommand\sumnk     {\sum_{k = 0}^{n}}
\newcommand\sumni     {\sum_{i = 0}^{n}}
\newcommand\sumnko    {\sum_{k = 1}^{n}}
\newcommand\sumnio    {\sum_{i = 1}^{n}}
\newcommand\sumai     {\sum_{i = 1}^{n} A_i}
\newcommand\nsum[2]   {\reflectbox{\displaystyle\sum_{\reflectbox{\scriptsize$#1$}}^{\reflectbox{\scriptsize$#2$}}}}

\newcommand\bink      {\binom{n}{k}}
\newcommand\setn      {\{a_i\}^{2n}_{i = 1}}
\newcommand\setc[1]   {\{a_i\}^{#1}_{i = 1}}

\newcommand\cupain    {\bigcup_{i = 1}^{n} A_i}
\newcommand\cupai[1]  {\bigcup_{i = 1}^{#1} A_i}
\newcommand\cupiiai   {\bigcup_{i \in I} A_i}
\newcommand\capain    {\bigcap_{i = 1}^{n} A_i}
\newcommand\capai[1]  {\bigcap_{i = 1}^{#1} A_i}
\newcommand\capiiai   {\bigcap_{i \in I} A_i}

\newcommand\xot       {x_{1, 2}}
\newcommand\ano       {a_{n - 1}}
\newcommand\ant       {a_{n - 2}}

% Linear Algebra
\DeclareMathOperator{\chr}    {char}
\DeclareMathOperator{\diag}    {diag}

\newcommand\lra       {\leftrightarrow}
\newcommand\chrf      {\chr(\F)}
\newcommand\F         {\mathbb{F}}
\newcommand\co        {\colon}
\newcommand\tmat[2]   {\cl{\begin{matrix}
			#1
		\end{matrix}\, \middle\vert\, \begin{matrix}
			#2
\end{matrix}}}

\makeatletter
\newcommand\rrr[1]    {\xxrightarrow{1}{#1}}
\newcommand\rrt[2]    {\xxrightarrow{1}[#2]{#1}}
\newcommand\mat[2]    {M_{#1\times#2}}
\newcommand\tomat     {\, \dequad \longrightarrow}
\newcommand\pms[1]    {\begin{pmatrix}
		#1
\end{pmatrix}}

\DeclareMathOperator{\Sp}     {span} 
\DeclareMathOperator{\sgn}    {sgn} 
\DeclareMathOperator{\row}    {Row} 
\DeclareMathOperator{\adj}    {adj} 
\DeclareMathOperator{\rk}     {rank} 
\DeclareMathOperator{\col}    {Col} 
\DeclareMathOperator{\tr}     {tr}

% someone's code from the internet: https://tex.stackexchange.com/questions/27545/custom-length-arrows-text-over-and-under
\makeatletter
\newlength\min@xx
\newcommand*\xxrightarrow[1]{\begingroup
	\settowidth\min@xx{$\m@th\scriptstyle#1$}
	\@xxrightarrow}
\newcommand*\@xxrightarrow[2][]{
	\sbox8{$\m@th\scriptstyle#1$}  % subscript
	\ifdim\wd8>\min@xx \min@xx=\wd8 \fi
	\sbox8{$\m@th\scriptstyle#2$} % superscript
	\ifdim\wd8>\min@xx \min@xx=\wd8 \fi
	\xrightarrow[{\mathmakebox[\min@xx]{\scriptstyle#1}}]
	{\mathmakebox[\min@xx]{\scriptstyle#2}}
	\endgroup}
\makeatother


% Greek Letters
\newcommand\ag        {\alpha}
\newcommand\bg        {\beta}
\newcommand\cg        {\gamma}
\newcommand\dg        {\delta}
\newcommand\eg        {\epsi}
\newcommand\zg        {\zeta}
\newcommand\hg        {\eta}
\newcommand\tg        {\theta}
\newcommand\ig        {\iota}
\newcommand\kg        {\keppa}
\renewcommand\lg      {\lambda}
\newcommand\og        {\omicron}
\newcommand\rg        {\rho}
\newcommand\sg        {\sigma}
\newcommand\yg        {\usilon}
\newcommand\wg        {\omega}

\newcommand\Ag        {\Alpha}
\newcommand\Bg        {\Beta}
\newcommand\Cg        {\Gamma}
\newcommand\Dg        {\Delta}
\newcommand\Eg        {\Epsi}
\newcommand\Zg        {\Zeta}
\newcommand\Hg        {\Eta}
\newcommand\Tg        {\Theta}
\newcommand\Ig        {\Iota}
\newcommand\Kg        {\Keppa}
\newcommand\Lg        {\Lambda}
\newcommand\Og        {\Omicron}
\newcommand\Rg        {\Rho}
\newcommand\Sg        {\Sigma}
\newcommand\Yg        {\Usilon}
\newcommand\Wg        {\Omega}

% Other shortcuts
\newcommand\tl    {\tilde}
\newcommand\op    {^{-1}}

\newcommand\sof[1]    {\left | #1 \right |}
\newcommand\cl [1]    {\left ( #1 \right )}
\newcommand\csb[1]    {\left [ #1 \right ]}
\newcommand\ccb[1]    {\left \{ #1 \right \}}

\newcommand\bs        {\blacksquare}
\newcommand\dequad    {\!\!\!\!\!\!}
\newcommand\dequadd   {\dequad\duquad}
\newcommand\wmid      {\;\middle\vert\;}

\renewcommand\phi     {\varphi}
\newcommand\bcl[1]    {\big(#1\big)}

%! ~~~ Document ~~~

\author{שחר פרץ}
\title{\textit{לינארית 2א 2}}
\begin{document}
	\maketitle
	\section{\en{Dalit stuff}}
	לגבי סוף השנה. בסוף יוני נגמר הסמסטר – יהיה מפגש, בהתחלה לבדנו ולאחר מכן עם ההורים. נדבר על המשמעות של בניית תוכנית אישית. בלי קשר דאי לקרוא את התוכנית של מדמח באינטרנט. נדייק כל אחד בפוקוס ואת התוכנית. בתוך המסלול הנוכחי אי אפשר להתקדם מעבר לשנה ב' ואי אפשר לססיים תואר. זהו אינו מסלול אקדמי והוא אינו משוייך לשום פקולטה או תואר. 
	
	נצטרך להגיע למצב בו מי שחושב או רוצה להתקדם לתוא רבהמשך (אף רחוק) צריך להיות בתוך מסלול המאפשר את זה. ישנו דבר שרלוונטי לנו (בדגש – יתכנו שינויים) והוא מסמך מסודר שכולל את התאריכים והדברים שנדרשים לקבלה למסלול לתואר. התוכנית אודיסאה כמו עסקת חבילה – כוללת את התוכן למעבר למסלול אקדמי והמרה. הדרישה – כל הציונים עוברים והממוצע מעל 85. צריך לעשות את כל הקורסים. בדיקת האפשרות להליך קבלה תדרש מסוף כיתה י"א. הציונים צריכים להיות טובים. כרגע בכל מקרה אין אופציה להצטיינות דיקן ולכן זה לא משנה שעושים מועדי ב' (בהנחה שהציונים טובים). לכן יש משמעות לכל רגע שאנחנו כאן. על כן לא תתאפשר הגעה באיחור וכו', ואם תהיה טיילת יהיה קשה ללמד. 
	
	אחרי שנפגש ביחד (כל השנה), כל מסלול (בנפרד), נפגש כל אחד עם ההורים (זאת לאחר מועדי א' אב', סוף אוגוסט או תחילת ספטמבר). נדבר גם על דו־חוגי (כמו מיכאל שרוצה לשלב מדמח עם הנדסת חשמל), ונבין איך מייצרים תוכנית שנוכל להתמודד איתה ולשמור על ההשגים, נוכל לבצעה במקביל לתיכון ותמלא את דרישותנו. 
	
	צריך ממוצע מעל 90 בשביל לקבל אישורים לחדו"א 2. כל עוד לא גומרים חדו"א – אין אפשרות לרוץ על קורסים בפקולטה. אבל "טכנית אפשר הכל" אז אני אופטימי. 
	
	\section{\en{Actual Linear Algebra}}
	\textbf{הגדרה. }יהי $T \co V \to V$ א"ל, נניח $\lg \in \F$ ע"ע, אז המרחב העצמי (מ"ע) של $\lg$ הוא: 
	\[ V_\lg := \{v \in V \mid Tv = \lg v\} \]
	\textbf{טענה. }$V_\lg$ תמ"ו של $V$. ראה תרגול. 
	
	\textbf{הגדרה. }יהי $T \co V \to V$ א"ל, ויהי $\lg \in \F$ ע"ע של $T$. נגדיר את ה\textit{ריבוי הגיאומטרי} של $\lg$ (ביחס ל־$T$) הוא $\dim V_\lg$. 
	
	[מספר דוגמאות שראינו בתרגול]. 
	
	\textbf{דוגמה.}
	יהי $V$ מ"ו ממימד $n$, $T \co V \to V$ א"ל. נניח קיו ם$v \in V $ המקיים $T^nv = v$, ו־$\{v, Tv, T^2v, \dots, T^{n - 1}v\}$ בסיס של $V$. ננסה להבין מהם הע"ע. 
	
	יהי $0 \neq u \in V$ ו"ע כך ש־$Tu = \lg u$. נראה כי $T^nu = u$. ידוע קיום $\ag_0, \dots, \ag_{n - 1} \in \F$ כך ש־$u = \sum \ag_iT^i(v)$. אז: 
	\[ \lg^nu = T^n(u) = \sum_{i = 0}^{n - 1}\ag_i\underbrace{T^{n + i}(v)}_{\mathclap{= T^i(T^nv) = T^iv}} = u \]
	
	ננסה להבין מי הם הוקטורים העצמיים. הם שורשי היחידה. זה תלוי שדה. 
	
	\textbf{משפט. }תהי $T \co V \to V$ א"ל, ונניח $A \subseteq V$ קבוצה של ו"ע של $T$ עם ע"ע שונים, אז $A$ בת"ל. הוכחה בתרגול. 
	
	\textbf{הגדרה. }יהי $T \co V \to V$ א"ל. נאמר ש־$T$ ניתן לכסון/לכסין אם קיים ל־$V$ בסיס של ו"ע של $T$. 
	
	\textbf{מסקנה. }אם $\dim V = n$ ול־$T$ יש $n$ ע"ע שונים אז $T$ לכסין. 
	
	\textit{הערה. }שימו לב – ייתכן מצב בו קיימים פחות מ־$n$ ע"ע שונים אך $T$ עדיין לכסין. דוגמה: $id, 0$
	
	\textbf{מסקנה. }תהי $T \co V \to V$ א"ל. נניח שלכל $\lg$ ע"א, ישנה $B_\lg \subseteq V_\lg$ בת"ל. אז $B = \bigcup_{\lg}B_\lg$ בת"ל. 
	
	\begin{proof}\,\!\!
		[הערה: ההוכחה הזו עובדת בעבור ההכללה לממדים שאינם נוצרים סופית]. ניקח צ"ל כלשהו שווה ל־0: 
		\begin{align*}
			\sum_{v_i \in B} \ag_iv_i &= 0 \\
			&= \sum_{\lg}\sum_{\lg_i}\ag_iv_{\lg, i} \\
			&\implies \sum_{\lg_j}\ag_i v_{\lg_{ji}} =: u_j \in V_{\lg_j} \\
			&\implies \sum_{j}u_j = 0
		\end{align*}
			קיבלנו צירוף ליניארי לא טרוויאלי של איברים במ"ע שונים (=עם ע"א שונים). אם אחד מהם אינו 0, קיבלנו סתירה למשפט. סה"כ קיבלנו שלכל $j$ מתקיים $\sum \ag_{ji}v_{ji} = 0$. בגלל ש־$v_{ji} \in B_j$ אז בת"ל ולכן כל הסקלרים 0. 
	\end{proof}

	\textbf{מסקנה. }יהי $T \co V \to V$ א"ל כך ש־$\dim V = n$. אז: 
	\[ \sum_\lg \dim V_\lg \le n \]
	שוויון אמ"מ $T$ לכסין. 
	
	\begin{proof}
		לכל $\lg$ יהא $B_\lg$ בסיס. אז $B = \sum_{\lg}B_\lg$ בת"ל. אז $n \ge |B| = \sum_\lg \dim V_\lg$ . 
		
		אם $T$ לכסין אז קיים בסיס של ו"ע כך שאכל אחד מהם מבין $V_\lg$ ושוויון. 
		
		מצד שני, אם יש שוויון אז $B$ קבוצה בת"ל של $n$ ו"ע ולכן בסיס ולכן $T$ לכסין. 
	\end{proof}
	
	\section{\en{The same stuff just for matrixes}}
	\textbf{הגדרה. }תהי $A \in M_n(\F)$. נאמר ש־$0 \neq v \in \F^n$ הוא ו"ע של $A$ עם ע"ע $\lg$ אם $Av = \lg v$. 
	
	\textbf{משפט. }תהי $T \co V \to V$ א"ל ויהי $B$ בסיס סדור, ו־$V$ נוצר סופי (לעיתים יקרא: סוף־ממדי). נניח $A = [T]_B$. אז $v \neq 0$ וקטור עצמי של $T$ עם ערך עצמי $\lg$ אמ"מ $[v]_B$ וקטור עצמי של $A$ עם ע"ע $\lg$. 
	
	\begin{proof}
		גרירה דו־כיוונית. 
		נניח $V$ ו"ע של $T$. אז $A[v]_B = [Tv]_B = [\lg_v]_B \lg[v]_B$. מהכיוון השני "לכו הפוך". 
	\end{proof}
	
	\textbf{הגדרה. }מטריצה $A \in M_n(\F)$ תקרא לכסינה/נתנת ללכסון אם היא דומה למטריצה אלכסונית $\Lambda \in M_n(\F)$ אלכסונית כך שקיימת $P \in M_n(\F)$ הפיכה שעבורה $\Lg = P\op AP$. 
	
	\textbf{משפט. }יהיו $A, P \in M_n(\F)$. נניח $P$ הפיכה. אז אם $P\op AP = \diag(\lg_1, \dots, \lg_n)$ אמ"מ עמודות $P$ הן ו,ע של $A$ עם ע"ע $\lg_1 \dots \lg_n$ בהתאמה. 
	
	\begin{proof}
		נסמן $P = (P_1 \dots P_n)$. אז: 
		\[ AP = (AP_1 \dots AP_n) = (\lg_1P_1 \dots \lg_nP_n) = P\underbrace{\pms{\lg_1 && \\ & \ddots & \\ && \lg_n}}_{\Lg} \]
		ולכן: 
		\[ P\op AP = P\op P\Lg \]
		ההוכחה מהכיוון השני היא לקרוא את זה מהצד השני. 
	\end{proof}
	
	"אני מקווה שראיתם שכפל באלכסונית מתחלפות". 
	"אני אמרתי שטות". 
	$\sim$ בן 
	
	\subsection{\en{Characteristic Polynomial}}
	\textbf{תרגיל. }תהי $A = \binom{-7 \, 8}{6 \, 7}$. מצאו ו"ע וע"ע של $A$ ולכסנו אם אפשר. 
	
	\textbf{פתרון. }מחפשים $\binom{0}{0} \neq \binom{x}{y} \in \R^2$ ו־$\lg \in \R$ כך ש־: 
	\begin{align*}
		A\pms{x \\ y} &= \lg \pms{x \\ y} \\
		A\pms{x \\ y} &= \lg I \pms{x \\ y} \\
		\lg I \pms{x \\ y} - A \pms{x \\ y} &= 0 \\
		\pms{\lg I - A} \pms{x \\ y} = 0
	\end{align*}
	סה"כ $\binom{x}{y}$ ו"ע עם ו"ע $\lg$ אמ"מ $\binom{x}{y} \in \ker (\lg I - A)$, אמ"מ $\lg I - A$ לא הפיכה, אמ"מ $\det(\lg I - A) = 0$ (AKA "הפולינום האופייני"). במקרה הזה: 
	\[ \lg I - A = \pms{\lg + 7 & -8 \\ -6 & \lg - 7} \implies \det(\lg I - A) = (\lg + 7)(\lg - 7) = \lg^2 - 1 \]
	לכן הע"ע הם $\pm 1$. נמצא את הו"ע. עבור $\lg = 1$, מתקיים: 
	\[ \pms{8 & 8 \\ -6 & -6}\pms{x \\ y} = 0 \]
	יש לנו חופש בחירה (ופתרון יחיד עד לכדי כפל בסקלר). במקרה הזה, נבחר $\binom{x}{y} = \pms{1 \\ -1}$. 
	
	עבור $\lg = -1$, יתקיים: 
	\[ \pms{6 & 8 \\ -6 & 8}\pms{x \\ y} = 0 \impliedby b\pms{x \\ y} = \pms{4 \\ -3} \]
	ראינו שהמלכסנת היא העמודות של הו"ע. אז: 
	\[ P = \pms{1 & 4 \\ -1 & -3} \]
	וסה"כ $P\op A P = I$. מכאן צריך למצוא את $P\op$. 
	
	\textbf{משפט. }תהי $A \in M_n(\F)$ אז $\lg \in \F$ ע"ע של $A$ אמ"מ $|\lg - A| = 0$. 
	
	\textbf{הגדרה. }תהי $A \in M_n(\F). $\textit{הפולינום האופייני} של $A$ מוגדר להיות: 
	\[ f_A(x) = |xI - A| \]
	
	\textbf{משפט. }תהי $A \in M_n(\F)$. אז $f_A(x)$ הוא פולינום מתוקן [=מקדם מוביל הוא 1] ממעלה $n$, המקדם של $x^{n - 1}$ הוא $- \tr A$ והמקדם החופשי הוא $(-1)^{n}|A|$. 
	
	
	
	
\end{document}