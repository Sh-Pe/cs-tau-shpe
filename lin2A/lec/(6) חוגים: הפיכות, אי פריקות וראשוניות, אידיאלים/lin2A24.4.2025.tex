%! ~~~ Packages Setup ~~~ 
\documentclass[]{article}
\usepackage{lipsum}
\usepackage{rotating}


% Math packages
\usepackage[usenames]{color}
\usepackage{forest}
\usepackage{ifxetex,ifluatex,amssymb,amsmath,mathrsfs,amsthm,witharrows,mathtools,mathdots}
\usepackage{amsmath}
\WithArrowsOptions{displaystyle}
\renewcommand{\qedsymbol}{$\blacksquare$} % end proofs with \blacksquare. Overwrites the defualts. 
\usepackage{cancel,bm}
\usepackage[thinc]{esdiff}


% tikz
\usepackage{tikz}
\usetikzlibrary{graphs}
\newcommand\sqw{1}
\newcommand\squ[4][1]{\fill[#4] (#2*\sqw,#3*\sqw) rectangle +(#1*\sqw,#1*\sqw);}


% code 
\usepackage{algorithm2e}
\usepackage{listings}
\usepackage{xcolor}

\definecolor{codegreen}{rgb}{0,0.35,0}
\definecolor{codegray}{rgb}{0.5,0.5,0.5}
\definecolor{codenumber}{rgb}{0.1,0.3,0.5}
\definecolor{codeblue}{rgb}{0,0,0.5}
\definecolor{codered}{rgb}{0.5,0.03,0.02}
\definecolor{codegray}{rgb}{0.96,0.96,0.96}

\lstdefinestyle{pythonstylesheet}{
	language=Java,
	emphstyle=\color{deepred},
	backgroundcolor=\color{codegray},
	keywordstyle=\color{deepblue}\bfseries\itshape,
	numberstyle=\scriptsize\color{codenumber},
	basicstyle=\ttfamily\footnotesize,
	commentstyle=\color{codegreen}\itshape,
	breakatwhitespace=false, 
	breaklines=true, 
	captionpos=b, 
	keepspaces=true, 
	numbers=left, 
	numbersep=5pt, 
	showspaces=false,                
	showstringspaces=false,
	showtabs=false, 
	tabsize=4, 
	morekeywords={as,assert,nonlocal,with,yield,self,True,False,None,AssertionError,ValueError,in,else},              % Add keywords here
	keywordstyle=\color{codeblue},
	emph={var, List, Iterable, Iterator},          % Custom highlighting
	emphstyle=\color{codered},
	stringstyle=\color{codegreen},
	showstringspaces=false,
	abovecaptionskip=0pt,belowcaptionskip =0pt,
	framextopmargin=-\topsep, 
}
\newcommand\pythonstyle{\lstset{pythonstylesheet}}
\newcommand\pyl[1]     {{\lstinline!#1!}}
\lstset{style=pythonstylesheet}

\usepackage[style=1,skipbelow=\topskip,skipabove=\topskip,framemethod=TikZ]{mdframed}
\definecolor{bggray}{rgb}{0.85, 0.85, 0.85}
\mdfsetup{leftmargin=0pt,rightmargin=0pt,innerleftmargin=15pt,backgroundcolor=codegray,middlelinewidth=0.5pt,skipabove=5pt,skipbelow=0pt,middlelinecolor=black,roundcorner=5}
\BeforeBeginEnvironment{lstlisting}{\begin{mdframed}\vspace{-0.4em}}
	\AfterEndEnvironment{lstlisting}{\vspace{-0.8em}\end{mdframed}}


% Deisgn
\usepackage[labelfont=bf]{caption}
\usepackage[margin=0.6in]{geometry}
\usepackage{multicol}
\usepackage[skip=4pt, indent=0pt]{parskip}
\usepackage[normalem]{ulem}
\forestset{default}
\renewcommand\labelitemi{$\bullet$}
\usepackage{titlesec}
\titleformat{\section}[block]
{\fontsize{15}{15}}
{\sen \dotfill (\thesection)\she}
{0em}
{\MakeUppercase}
\usepackage{graphicx}
\graphicspath{ {./} }

\usepackage[colorlinks]{hyperref}
\definecolor{mgreen}{RGB}{25, 160, 50}
\definecolor{mblue}{RGB}{30, 60, 200}
\usepackage{hyperref}
\hypersetup{
	colorlinks=true,
	citecolor=mgreen,
	linkcolor=black,
	urlcolor=mblue,
	pdftitle={AI Something},
	pdfpagemode=FullScreen,
}


% Hebrew initialzing
\usepackage[bidi=basic]{babel}
\PassOptionsToPackage{no-math}{fontspec}
\babelprovide[main, import, Alph=letters]{hebrew}
\babelprovide[import]{english}
\babelfont[hebrew]{rm}{David CLM}
\babelfont[hebrew]{sf}{David CLM}
%\babelfont[english]{tt}{Monaspace Xenon}
\usepackage[shortlabels]{enumitem}
\newlist{hebenum}{enumerate}{1}

% Language Shortcuts
\newcommand\en[1] {\begin{otherlanguage}{english}#1\end{otherlanguage}}
\newcommand\he[1] {\she#1\sen}
\newcommand\sen   {\begin{otherlanguage}{english}}
	\newcommand\she   {\end{otherlanguage}}
\newcommand\del   {$ \!\! $}

\newcommand\npage {\she\vfil {\hfil \textbf{\textit{המשך בעמוד הבא}}} \hfil \vfil \pagebreak\sen}
\newcommand\ndoc  {\dotfill \\ \vfil {\begin{center}
			{\textbf{\textit{שחר פרץ, 2025}} \\
				\scriptsize \textit{קומפל ב־}\en{\LaTeX}\,\textit{ ונוצר באמצעות תוכנה חופשית בלבד}}
	\end{center}} \vfil	}

\newcommand{\rn}[1]{
	\textup{\uppercase\expandafter{\romannumeral#1}}
}

\makeatletter
\newcommand{\skipitems}[1]{
	\addtocounter{\@enumctr}{#1}
}
\makeatother

%! ~~~ Math shortcuts ~~~

% Letters shortcuts
\newcommand\N     {\mathbb{N}}
\newcommand\Z     {\mathbb{Z}}
\newcommand\R     {\mathbb{R}}
\newcommand\Q     {\mathbb{Q}}
\newcommand\C     {\mathbb{C}}
\newcommand\One   {\mathit{1}}

\newcommand\ml    {\ell}
\newcommand\mj    {\jmath}
\newcommand\mi    {\imath}

\newcommand\powerset {\mathcal{P}}
\newcommand\ps    {\mathcal{P}}
\newcommand\pc    {\mathcal{P}}
\newcommand\ac    {\mathcal{A}}
\newcommand\bc    {\mathcal{B}}
\newcommand\cc    {\mathcal{C}}
\newcommand\dc    {\mathcal{D}}
\newcommand\ec    {\mathcal{E}}
\newcommand\fc    {\mathcal{F}}
\newcommand\nc    {\mathcal{N}}
\newcommand\vc    {\mathcal{V}} % Vance
\newcommand\sca   {\mathcal{S}} % \sc is already definded
\newcommand\rca   {\mathcal{R}} % \rc is already definded

\newcommand\prm   {\mathrm{p}}
\newcommand\arm   {\mathrm{a}} % x86
\newcommand\brm   {\mathrm{b}}
\newcommand\crm   {\mathrm{c}}
\newcommand\drm   {\mathrm{d}}
\newcommand\erm   {\mathrm{e}}
\newcommand\frm   {\mathrm{f}}
\newcommand\nrm   {\mathrm{n}}
\newcommand\vrm   {\mathrm{v}}
\newcommand\srm   {\mathrm{s}}
\newcommand\rrm   {\mathrm{r}}

\newcommand\Si    {\Sigma}

% Logic & sets shorcuts
\newcommand\siff  {\longleftrightarrow}
\newcommand\ssiff {\leftrightarrow}
\newcommand\so    {\longrightarrow}
\newcommand\sso   {\rightarrow}

\newcommand\epsi  {\epsilon}
\newcommand\vepsi {\varepsilon}
\newcommand\vphi  {\varphi}
\newcommand\Neven {\N_{\mathrm{even}}}
\newcommand\Nodd  {\N_{\mathrm{odd }}}
\newcommand\Zeven {\Z_{\mathrm{even}}}
\newcommand\Zodd  {\Z_{\mathrm{odd }}}
\newcommand\Np    {\N_+}

% Text Shortcuts
\newcommand\open  {\big(}
\newcommand\qopen {\quad\big(}
\newcommand\close {\big)}
\newcommand\also  {\text{, }}
\newcommand\defis {\text{ definitions}}
\newcommand\given {\text{given }}
\newcommand\case  {\text{if }}
\newcommand\syx   {\text{ syntax}}
\newcommand\rle   {\text{ rule}}
\newcommand\other {\text{else}}
\newcommand\set   {\ell et \text{ }}
\newcommand\ans   {\mathscr{A}\!\mathit{nswer}}

% Set theory shortcuts
\newcommand\ra    {\rangle}
\newcommand\la    {\langle}

\newcommand\oto   {\leftarrow}

\newcommand\QED   {\quad\quad\mathscr{Q.E.D.}\;\;\blacksquare}
\newcommand\QEF   {\quad\quad\mathscr{Q.E.F.}}
\newcommand\eQED  {\mathscr{Q.E.D.}\;\;\blacksquare}
\newcommand\eQEF  {\mathscr{Q.E.F.}}
\newcommand\jQED  {\mathscr{Q.E.D.}}

\DeclareMathOperator\dom   {dom}
\DeclareMathOperator\Img   {Im}
\DeclareMathOperator\range {range}

\newcommand\trio  {\triangle}

\newcommand\rc    {\right\rceil}
\newcommand\lc    {\left\lceil}
\newcommand\rf    {\right\rfloor}
\newcommand\lf    {\left\lfloor}
\newcommand\ceil  [1] {\lc #1 \rc}
\newcommand\floor [1] {\lf #1 \rf}

\newcommand\lex   {<_{lex}}

\newcommand\az    {\aleph_0}
\newcommand\uaz   {^{\aleph_0}}
\newcommand\al    {\aleph}
\newcommand\ual   {^\aleph}
\newcommand\taz   {2^{\aleph_0}}
\newcommand\utaz  { ^{\left (2^{\aleph_0} \right )}}
\newcommand\tal   {2^{\aleph}}
\newcommand\utal  { ^{\left (2^{\aleph} \right )}}
\newcommand\ttaz  {2^{\left (2^{\aleph_0}\right )}}

\newcommand\n     {$n$־יה\ }

% Math A&B shortcuts
\newcommand\logn  {\log n}
\newcommand\logx  {\log x}
\newcommand\lnx   {\ln x}
\newcommand\cosx  {\cos x}
\newcommand\sinx  {\sin x}
\newcommand\sint  {\sin \theta}
\newcommand\tanx  {\tan x}
\newcommand\tant  {\tan \theta}
\newcommand\sex   {\sec x}
\newcommand\sect  {\sec^2}
\newcommand\cotx  {\cot x}
\newcommand\cscx  {\csc x}
\newcommand\sinhx {\sinh x}
\newcommand\coshx {\cosh x}
\newcommand\tanhx {\tanh x}

\newcommand\seq   {\overset{!}{=}}
\newcommand\slh   {\overset{LH}{=}}
\newcommand\sle   {\overset{!}{\le}}
\newcommand\sge   {\overset{!}{\ge}}
\newcommand\sll   {\overset{!}{<}}
\newcommand\sgg   {\overset{!}{>}}

\newcommand\h     {\hat}
\newcommand\ve    {\vec}
\newcommand\lv    {\overrightarrow}
\newcommand\ol    {\overline}

\newcommand\mlcm  {\mathrm{lcm}}

\DeclareMathOperator{\sech}   {sech}
\DeclareMathOperator{\csch}   {csch}
\DeclareMathOperator{\arcsec} {arcsec}
\DeclareMathOperator{\arccot} {arcCot}
\DeclareMathOperator{\arccsc} {arcCsc}
\DeclareMathOperator{\arccosh}{arccosh}
\DeclareMathOperator{\arcsinh}{arcsinh}
\DeclareMathOperator{\arctanh}{arctanh}
\DeclareMathOperator{\arcsech}{arcsech}
\DeclareMathOperator{\arccsch}{arccsch}
\DeclareMathOperator{\arccoth}{arccoth}
\DeclareMathOperator{\atant}  {atan2} 
\DeclareMathOperator{\Sp}     {span} 
\DeclareMathOperator{\sgn}    {sgn} 
\DeclareMathOperator{\row}    {Row} 
\DeclareMathOperator{\adj}    {adj} 
\DeclareMathOperator{\rk}     {rank} 
\DeclareMathOperator{\col}    {Col} 
\DeclareMathOperator{\tr}     {tr}

\newcommand\dx    {\,\mathrm{d}x}
\newcommand\dt    {\,\mathrm{d}t}
\newcommand\dtt   {\,\mathrm{d}\theta}
\newcommand\du    {\,\mathrm{d}u}
\newcommand\dv    {\,\mathrm{d}v}
\newcommand\df    {\mathrm{d}f}
\newcommand\dfdx  {\diff{f}{x}}
\newcommand\dit   {\limhz \frac{f(x + h) - f(x)}{h}}

\newcommand\nt[1] {\frac{#1}{#1}}

\newcommand\limz  {\lim_{x \to 0}}
\newcommand\limxz {\lim_{x \to x_0}}
\newcommand\limi  {\lim_{x \to \infty}}
\newcommand\limh  {\lim_{x \to 0}}
\newcommand\limni {\lim_{x \to - \infty}}
\newcommand\limpmi{\lim_{x \to \pm \infty}}

\newcommand\ta    {\theta}
\newcommand\ap    {\alpha}

\renewcommand\inf {\infty}
\newcommand  \ninf{-\inf}

% Combinatorics shortcuts
\newcommand\sumnk     {\sum_{k = 0}^{n}}
\newcommand\sumni     {\sum_{i = 0}^{n}}
\newcommand\sumnko    {\sum_{k = 1}^{n}}
\newcommand\sumnio    {\sum_{i = 1}^{n}}
\newcommand\sumai     {\sum_{i = 1}^{n} A_i}
\newcommand\nsum[2]   {\reflectbox{\displaystyle\sum_{\reflectbox{\scriptsize$#1$}}^{\reflectbox{\scriptsize$#2$}}}}

\newcommand\bink      {\binom{n}{k}}
\newcommand\setn      {\{a_i\}^{2n}_{i = 1}}
\newcommand\setc[1]   {\{a_i\}^{#1}_{i = 1}}

\newcommand\cupain    {\bigcup_{i = 1}^{n} A_i}
\newcommand\cupai[1]  {\bigcup_{i = 1}^{#1} A_i}
\newcommand\cupiiai   {\bigcup_{i \in I} A_i}
\newcommand\capain    {\bigcap_{i = 1}^{n} A_i}
\newcommand\capai[1]  {\bigcap_{i = 1}^{#1} A_i}
\newcommand\capiiai   {\bigcap_{i \in I} A_i}

\newcommand\xot       {x_{1, 2}}
\newcommand\ano       {a_{n - 1}}
\newcommand\ant       {a_{n - 2}}

% Linear Algebra
\DeclareMathOperator{\chr}     {char}
\DeclareMathOperator{\diag}    {diag}
\DeclareMathOperator{\Hom}     {Hom}

\newcommand\lra       {\leftrightarrow}
\newcommand\chrf      {\chr(\F)}
\newcommand\F         {\mathbb{F}}
\newcommand\co        {\colon}
\newcommand\tmat[2]   {\cl{\begin{matrix}
			#1
		\end{matrix}\, \middle\vert\, \begin{matrix}
			#2
\end{matrix}}}

\makeatletter
\newcommand\rrr[1]    {\xxrightarrow{1}{#1}}
\newcommand\rrt[2]    {\xxrightarrow{1}[#2]{#1}}
\newcommand\mat[2]    {M_{#1\times#2}}
\newcommand\gmat      {\mat{m}{n}(\F)}
\newcommand\tomat     {\, \dequad \longrightarrow}
\newcommand\pms[1]    {\begin{pmatrix}
		#1
\end{pmatrix}}

% someone's code from the internet: https://tex.stackexchange.com/questions/27545/custom-length-arrows-text-over-and-under
\makeatletter
\newlength\min@xx
\newcommand*\xxrightarrow[1]{\begingroup
	\settowidth\min@xx{$\m@th\scriptstyle#1$}
	\@xxrightarrow}
\newcommand*\@xxrightarrow[2][]{
	\sbox8{$\m@th\scriptstyle#1$}  % subscript
	\ifdim\wd8>\min@xx \min@xx=\wd8 \fi
	\sbox8{$\m@th\scriptstyle#2$} % superscript
	\ifdim\wd8>\min@xx \min@xx=\wd8 \fi
	\xrightarrow[{\mathmakebox[\min@xx]{\scriptstyle#1}}]
	{\mathmakebox[\min@xx]{\scriptstyle#2}}
	\endgroup}
\makeatother


% Greek Letters
\newcommand\ag        {\alpha}
\newcommand\bg        {\beta}
\newcommand\cg        {\gamma}
\newcommand\dg        {\delta}
\newcommand\eg        {\epsi}
\newcommand\zg        {\zeta}
\newcommand\hg        {\eta}
\newcommand\tg        {\theta}
\newcommand\ig        {\iota}
\newcommand\kg        {\keppa}
\renewcommand\lg      {\lambda}
\newcommand\og        {\omicron}
\newcommand\rg        {\rho}
\newcommand\sg        {\sigma}
\newcommand\yg        {\usilon}
\newcommand\wg        {\omega}

\newcommand\Ag        {\Alpha}
\newcommand\Bg        {\Beta}
\newcommand\Cg        {\Gamma}
\newcommand\Dg        {\Delta}
\newcommand\Eg        {\Epsi}
\newcommand\Zg        {\Zeta}
\newcommand\Hg        {\Eta}
\newcommand\Tg        {\Theta}
\newcommand\Ig        {\Iota}
\newcommand\Kg        {\Keppa}
\newcommand\Lg        {\Lambda}
\newcommand\Og        {\Omicron}
\newcommand\Rg        {\Rho}
\newcommand\Sg        {\Sigma}
\newcommand\Yg        {\Usilon}
\newcommand\Wg        {\Omega}

% Other shortcuts
\newcommand\tl    {\tilde}
\newcommand\op    {^{-1}}

\newcommand\sof[1]    {\left | #1 \right |}
\newcommand\cl [1]    {\left ( #1 \right )}
\newcommand\csb[1]    {\left [ #1 \right ]}
\newcommand\ccb[1]    {\left \{ #1 \right \}}

\newcommand\bs        {\blacksquare}
\newcommand\dequad    {\!\!\!\!\!\!}
\newcommand\dequadd   {\dequad\duquad}

\renewcommand\phi     {\varphi}

\newtheorem{Theorem}{משפט}
\theoremstyle{definition}
\newtheorem{definition}{הגדרה}
\newtheorem{Lemma}{למה}
\newtheorem{Remark}{הערה}
\newtheorem{Notion}{סימון}
\newtheorem{Collary}{מסקנה}

\newcommand\cola [1] {\begin{Collary}#1\end{Collary}}
\newcommand\theo  [1] {\begin{Theorem}#1\end{Theorem}}
\newcommand\defi  [1] {\begin{definition}#1\end{definition}}
\newcommand\rmark [1] {\begin{Remark}#1\end{Remark}}
\newcommand\lem   [1] {\begin{Lemma}#1\end{Lemma}}
\newcommand\noti  [1] {\begin{Notion}#1\end{Notion}}

% DS
\newcommand\limsi     {\limsup_{n \to \inf}}
\newcommand\limfi     {\liminf_{n \to \inf}}

\DeclareMathOperator\amort   {amort}
\DeclareMathOperator\worst   {worst}
\DeclareMathOperator\type    {type}
\DeclareMathOperator\cost    {cost}

\newcommand\dsList{
	\sFunc{List}
	\sFunc{Retrieve}
	\SetKwFunction{RetrieveFirst}{Retrieve-First}
	\SetKwFunction{RetrieveLast}{Retrieve-Last}
	\sFunc{Delete}
	\SetKwFunction{DeleteFirst}{Delete-First}
	\SetKwFunction{DeleteLast}{Delete-Last}
	\sFunc{Insert}
	\SetKwFunction{InsertFirst}{Insert-First}
	\SetKwFunction{InsertLast}{Insert-Last}
	\sFunc{Shift}
	\sFunc{Length}
	\sFunc{Concat}
	\sFunc{Plant}
	\sFunc{Split}
}
\newcommand\dsQueue{
	\sFunc{Queue}
	\sFunc{Enqueue}
	\sFunc{Head}
	\sFunc{Dequeue}
}
\newcommand\dsStack{
	\sFunc{Stack}
	\sFunc{Push}
	\sFunc{Top}
	\sFunc{Pop}
}
\newcommand\dsVector{
	\sFunc{Vector}
	\sFunc{Get}
	\sFunc{Set}
}
\newcommand\dsGraph{
	\sFunc{Graph}
	\sFunc{Edge}
	\SetKwFunction{AddEdge}{Add-Edge}
	\SetKwFunction{RemoveEdge}{Remove-Edge}
	\sFunc{InDeg} \sFunc{OutDeg}
}
\newcommand\importDs{
	\dsList
	\dsQueue
	\dsStack
	\dsVector
	\dsGraph
	\SetKwData{error}{\color{codered}error}
	\SetKwInOut{Input}{input}
	\SetKwInOut{Output}{output}
	\SetKwRepeat{Do}{do}{while}
	\SetKwData{Null}{\color{codeblue}null}
}


% Algorithems
\newcommand\sFunc [1] {\SetKwFunction{#1}{#1}}
\newcommand\sData [1] {\SetKwData{#1}{#1}}
\newcommand\sIO   [1] {\SetKwInOut{#1}{#1}}
\newcommand\ttt   [1] {\sen \texttt{#1} \she\,}
\newcommand\io    [2] {\Input{#1}\Output{#2}\BlankLine}

%! ~~~ Document ~~~

\author{שחר פרץ}
\title{\textit{לינארית 2א $\sim$ חוגים ושאר ירקות}}
\begin{document}
	\maketitle
	\section{\en{Rings}}
	\defi{תחום שלמות הוא חוג קומוטטיבי עם יחידה ללא מחלקי $0$. }
	\defi{חוק ייקרא \textit{ללא מחלקי 0} אם: \hfill $\forall a, b \in \R \co ab = 0 \implies a = 0 \lor b = 0$}
	
	דוגמאות לחוגים עם מחלקי $0$: 
	\begin{itemize}
		\item $M_2(\R)$: הוכחה $a = b = \binom{0\, 1}{0\, 0}, \ a \cdot b = 0$
		\item $\Z/_{6\Z}$ הוכחה $2 \cdot 3 = 0$. 
	\end{itemize}
	
	\theo{בתחום שלמות יש את כלל הצמצום בכפל: אם $ab = ac \land a \neq 0$ אז $b = c$. }\begin{proof}
		\[ ab  \cdot ac = 0 \implies a(b \cdot c) = 0 \implies a = 0 \lor b - c = 0 \]
		בגלל ש־$a \neq 0$, אז $b - c = 0$. נוסיף את $c$ הנגדי של $-c$ ונקבל $b = c$. 
	\end{proof}
	
	דוגמאות לתחום שלמות: 
	\begin{itemize}
		\item שדות
		\item השלמים
		\item חוג הפולינומים
	\end{itemize}
	
	\theo{לכל $f, g \in \F[x]$, אם $g \neq 0$ אז קיימים ויחידים פולינומים $q, r \n \F[x]$ כך ש־$f = qg + r \land \deg r < \deg g$}
	
	\defi{נאמר שפולינום $q$ מחלק את $f$ אם $r = 0$ ומסמנים $q \mid f$. }
	
	\defi{חוק אוקלידי הוא חוג שמעליו אפשר לבצע פירוק פולינום כזה. }
	
	דוגמה לחוג שאינו אוקלידי: $\Z[\sqrt{-5}]$ הוא $\{a + b\sqrt{-5} \mid a, b \in \Z\}$. 
		
	\theo{חוג אוקלידי $\impliedby$ פריקות יחידה (דומה למשפט היסודי של האריתמטיקה). }
		
		לדוגמה בחוג לעיל $6 = 2 \cdot 3 = (1 + \sqrt{-5})(1 - \sqrt{-5})$ על אף ש־$2, 3$ אי־פריקים וכן $(1 + \sqrt{-5}), (1  - \sqrt{-5})$ אי פריקים. 
		
	\cola{\,
	\begin{itemize}
		\item $f(a) = 0 \iff (x - a) \mid f$ (משפט בזו)
		\item אם $\deg f = n > -\inf$, ל־$f$ לכל היותר $n$ שורשים כולל ריבוי. 
		\item נניח ש־$f, g \in \F[x]$ ו־$F \subseteq K$, כאשר $K$ שדה. אם $g \mid f$ מעל $K$ אז $g \mid f$ מעל $\F$. 
	\end{itemize}
	}
	\begin{proof}
		\begin{enumerate}\,
			\item \begin{itemize}
				\item[$\implies$] נניח $x - a \mid f$. אז קיים פולינום $g$ כך ש־$f = (x - a)g$. אז $f(a) = (a - a)g(a) = 0$. 
				\item[$\impliedby$] נניח $f(a) = 0$. אז קיימים $q, r \in \F[x]$ כך ש־$f = q(x - a)$ ועל כן $0 = f(a) = q(a)(a - a) + r(a) = 0$ ולכן $r(a) = 0$. משום ש־$r$ פולינום קבוע (דרגתו קטנה מ־1, כי חילקנו ב־$(x - a)$ מדרגה 1), אז $r(x) = 0$. 
			\end{itemize}
			\item אינדוקציה
			\item נוכיח ב"contrapositive": אנו יודעים ש־$P \to Q \iff \lnot Q \to \lnot P$. נניח ש־$g \nmid f$ מעל $\F$. קיימים $q, r \in \F[x]$ כך ש־$f =qg + r, \ r \neq 0$. הפירוק הזה הוא גם ב־$K[x]$. מיחידות $r$, נקבל ש־$g \nmid f$ כל מעל $K$. 
		\end{enumerate}
	\end{proof}
	
	\subsection{עוד על תחומי שלמות}
	\defi{יהי $R$ תחום שלמות, $a, b \in R$. נאמר ש־$a \mid b$ אם קיים $c \in \R$ כך ש־$ac = b$. }
	\defi{$u \in R$ נקרא \textit{הפיך} אם קיים $\ag \in R$ כך ש־$\ag u = 1$. }
	\theo{יהי $R$ תחום שלמות, $u \in R$ הפיך. יהי $a \in \R$. אז $u \mid a$. }\begin{proof}
		$1 \mid a, \ u \mid 1$. יחס החלוקה טרנזטיבי ולכן $u \mid a$. 
	\end{proof}
	\noti{קבוצת ההפיכים מוסמנת ב־$R^x$. }
	\textbf{דוגמאות. }
	\begin{enumerate}
		\item אם $R = \F$, אז $\F^x = \F \setminus \{0\}$
		\item אם $R = \Z$ אז $\Z^2 = \{\pm 1\}$
		\item אם $R = \F[x]$ אז $R^x = \F^x$ (ההתייחסות לסקלרים $\F$ היא כאל פונקציות קבועות)
	\end{enumerate}
	\defi{$a, b \in R$ נקראים \textit{חברים} אם קיים $u \in R^x$ הפיך כך ש־$a = ub$, ומסמנים $a \sim b$ }
	"אני אני חבר של עומר, ועומר חבר של מישהו בכיתה שאני לא מכיר, אני לא חבר של מי שאני לא מכיר." המרצה: "למה לא? תהיה חבר שלו". 
	
	\theo{יחס החברות הוא יחס שקילות. }\begin{proof}
		\begin{enumerate}[A.]
			\item $a \sim a$ כי $1 \in R^x$
			\item אם $a \sim b$ אז קיים $u \in R^x$ כך ש־$a = ub$. קיים ל־$u$ הופכי $\ag$ אז $\ag a + \ag u b = b$ ולכן $b \sim a$. 
			\item נניח $a \sim b \land b \sim c$, כי מכפלת ההופכיים הפיכה $a \sim c$ וסיימנו. 
		\end{enumerate}
	\end{proof}
	\theo{הופכי הוא יחיד} (אותה ההוכחה כמו בשדה. לא בהכרח בתחום שלמות, מעל כל חוג)\begin{proof}
				יהי $a \in R^x$ ו־$u, u'$ הופכיים שלו, אז:
		\[ u = u \cdot 1 = u \cdot a \cdot u' = 1 \cdot u' = u' \]
	\end{proof}
	\theo{אם $a \mid b$ וכם $b \mid a$ אז $a \mid b$ (בתחום שלמות). }\begin{proof}
		\begin{align*}
			a \mid b &\implies \exists c \in \R \co ac = b \\
			b \mid a &\implies \exists d \in \R\co bd = a
		\end{align*}
		לכן: 
		\[ ac = b \implies acd = a \implies a(cd - 1) = 0 \implies a = 0 \lor cd = 1 \]
	\end{proof}
	אם $a = 0$ אז $b = 0$ (ממש לפי הגדרה) ו־$\sim$ שקילות (רפליקסיביות). אחרת, $cd = 1$ ולכן $c$ הפיך, סה"כ $a \mid b$. 
	
	"אני חושב שבעברית קראו להם ידידים, לא רצו להתחייב לחברות ממש". 
	
	
	\defi{איבר $p \in R$ נקרא \textit{אי־פריק} אם מתקיים $p = ab \implies a \in R^x \lor b \in R^x$. }
	
	\defi{איבר $p \in R$ יקרא \textit{ראשוני} אם $p \mid (a \cdot b) \implies p \mid a \lor p \mid b$. }
	\textit{הערה: }איברים הפיכים לא נחשבים אי־פריקים או ראשוניים. הסיבה להגדרה: בשביל נכונות המשפט היסודי של האריתמטיקה (יחידות הפירוק לראשוניים). 
	
	\theo{בתחום שלמות כל ראשוני הוא אי פריק. }
	\textit{הערה: }שקילות לאו דווקא. 
	\begin{proof}
		יהי $p \in R$ ראשוני. יהיו $a, b \in R$ כך ש־$p = ab$. בה"כ $p \mid a$. אז קיים $c$ כך ש־$pc = a$ ולכן $pcb = p$. סה"כ $p \neq 0$ ולכן $cb = 1$ (ראה לעיל) ו־$b$ הפיך. 
	\end{proof}
	\theo{נניח שבתחום שלמות $R$, כל אי־פריק הוא גם ראשוני. אז $R$ תחום פריקות יחידה. }
	\defi{$R$ תחום פירוקת יחידה אם $\prod_{i = 1}^{n}p_i = \prod_{j = 1}^{m}q_i$ עבור $p_i, q_j$ ראשוניים, אז $m = n$, ועד לכדי סידור מחדש, לכל $i \in [n]$ $p_i \sim q_i$}
	
	ההוכחה: זהה לחלוטין לזו של המשפט היסודי. \begin{proof}
		באינדוקציה על $n+m$. בסיס: $n + m = 2$ ולכן $n = m = 1$ (כי מעפלה ריקה לא רלוונטית מאוד) אז $p = q$. נעבור לצעד. נניח שהטענה נכונה לכל $n + m < k$. נניח ש־$n + m = k$. אז $p_1 \mid \prod_{j = 1}^{m}q_j$. בה"כ $p_1 \mid q_1$. $q_1$ אי־פריק ולא הפיך. $p_1$ לא הפיך. לכן $p_1 \sim q_1$. אז עד כדי כפל בהופכי נקבל ש־$\prod_{i = 2}^{n} p_i = \prod_{j = 2}^{n} q_j$. \textit{הערה: }ראשוני כפול הפיך נשאר ראשוני. מכאן הקענו לדרוש וסיימנו (\textit{הערה שלי: }כאילו תכפילו בחברים ותקבלו את מה שצריך). 
	\end{proof}
	
	\defi{יהי $R$ תחום שלמות. תת־קבוצה $0 \neq I \subseteq R$ נקראת \textit{אידיאל} אם: 
	\begin{itemize}[A.]
		\item $\forall a, b \in I \co a + b \in I$ – סגירות לחיבור. 
		\item $\forall a \in I \, \forall b \in R \co ab \in I$ – תכונת הבליעה. [בפרט $0 \in I$]
	\end{itemize}}
	\textbf{דוגמאות: }\begin{enumerate}
		\item $0$ תמיד אידיאל, כך החוג תמדי אידיאל. 
		\item הזוגיים ב־$\Z$. 
		\item לכל $n \in \Z$, $n\Z$ אידיאל ($n$ כפול השלמים). הזוגיים מקרה פרטי. 
		\item $\la f \ra \subseteq \F[x]$ המוגדר לפי $\la f \ra := \{g \in \F[x] \mid f|g\}$
		\item הכללה של הקודמים: עבור $a \in R$ נסמן $\la a \ra := \{a \cdot b \mid b \in R\}$
		\item $I = \{f \in \F[x] \mid f(0) = 0\}$ (לעיתים מסומן $\forall a \in R \co aR = \la a \ra$)
		\item נוכל להכליל את 4 עוד: ("הכללה של הכללה היא הכללה. זה סגור להכללה. זה קורה הרבה במתמטיקה")
		\[ I = aR + bR = \{ar + bs \mid r, s \in R\} \]
		וניתן להכליל עוד באינדוקציה. 
	\end{enumerate}
	\defi{אידיאל $I$ נקרא \textit{ראשי} אם הוא מהצורה $aR$ עבור $a \in R$ כלשהו. }
	\defi{תחום שלמות נקרא \textit{ראשי} אם כל אידיאל שלו ראשי. }
	\theo{נניח ש־$R$ ראשי, אז כל אי־פריק ב־$R$ הוא ראשוני. }
	(תנאי מספיק אך לא הכרחי)
	
	\begin{proof}
		יהי $p \in R$ אי פריק. נראה שהוא ראשוני. נביט ב־$ab \in R$ כך ש־$p \mid ab$. נביט ב־$I = aR + bR$. מהיותו ראשי קיים איזשהו $c \in R$ כך ש־$I = cR$. אז $c \in I$ (כי $c \cdot 1 \in R$). אז $a, p \in I$. $a$ ברור. קיים $d \in R$ כך ש־$pd = a \cdot b$. למה $p \in I$ – המרצה דפק קליגמן ולא יודע להוכיח. 
		אחרי משהו כמו 5 דק' של בהיה בלוח הוא בה עם הדבר הבא: 
		\[ c \in I \implies \exists r, s \in \R\co ar + bs = c \]
		"אני אחושב על זה ואני אמשיך פעם הבאה". 
	\end{proof}
	
	
	
	\ndoc
\end{document}