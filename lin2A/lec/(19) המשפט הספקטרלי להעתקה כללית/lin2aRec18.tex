%! ~~~ Packages Setup ~~~ 
\documentclass[]{article}
\usepackage{lipsum}
\usepackage{rotating}


% Math packages
\usepackage[usenames]{color}
\usepackage{forest}
\usepackage{ifxetex,ifluatex,amssymb,amsmath,mathrsfs,amsthm,witharrows,mathtools,mathdots}
\usepackage{amsmath}
\WithArrowsOptions{displaystyle}
\renewcommand{\qedsymbol}{$\blacksquare$} % end proofs with \blacksquare. Overwrites the defualts. 
\usepackage{cancel,bm}
\usepackage[thinc]{esdiff}


% tikz
\usepackage{tikz}
\usetikzlibrary{graphs}
\newcommand\sqw{1}
\newcommand\squ[4][1]{\fill[#4] (#2*\sqw,#3*\sqw) rectangle +(#1*\sqw,#1*\sqw);}


% code 
\usepackage{algorithm2e}
\usepackage{listings}
\usepackage{xcolor}

\definecolor{codegreen}{rgb}{0,0.35,0}
\definecolor{codegray}{rgb}{0.5,0.5,0.5}
\definecolor{codenumber}{rgb}{0.1,0.3,0.5}
\definecolor{codeblue}{rgb}{0,0,0.5}
\definecolor{codered}{rgb}{0.5,0.03,0.02}
\definecolor{codegray}{rgb}{0.96,0.96,0.96}

\lstdefinestyle{pythonstylesheet}{
	language=Java,
	emphstyle=\color{deepred},
	backgroundcolor=\color{codegray},
	keywordstyle=\color{deepblue}\bfseries\itshape,
	numberstyle=\scriptsize\color{codenumber},
	basicstyle=\ttfamily\footnotesize,
	commentstyle=\color{codegreen}\itshape,
	breakatwhitespace=false, 
	breaklines=true, 
	captionpos=b, 
	keepspaces=true, 
	numbers=left, 
	numbersep=5pt, 
	showspaces=false,                
	showstringspaces=false,
	showtabs=false, 
	tabsize=4, 
	morekeywords={as,assert,nonlocal,with,yield,self,True,False,None,AssertionError,ValueError,in,else},              % Add keywords here
	keywordstyle=\color{codeblue},
	emph={var, List, Iterable, Iterator},          % Custom highlighting
	emphstyle=\color{codered},
	stringstyle=\color{codegreen},
	showstringspaces=false,
	abovecaptionskip=0pt,belowcaptionskip =0pt,
	framextopmargin=-\topsep, 
}
\newcommand\pythonstyle{\lstset{pythonstylesheet}}
\newcommand\pyl[1]     {{\lstinline!#1!}}
\lstset{style=pythonstylesheet}

\usepackage[style=1,skipbelow=\topskip,skipabove=\topskip,framemethod=TikZ]{mdframed}
\definecolor{bggray}{rgb}{0.85, 0.85, 0.85}
\mdfsetup{leftmargin=0pt,rightmargin=0pt,innerleftmargin=15pt,backgroundcolor=codegray,middlelinewidth=0.5pt,skipabove=5pt,skipbelow=0pt,middlelinecolor=black,roundcorner=5}
\BeforeBeginEnvironment{lstlisting}{\begin{mdframed}\vspace{-0.4em}}
	\AfterEndEnvironment{lstlisting}{\vspace{-0.8em}\end{mdframed}}


% Design
\usepackage[labelfont=bf]{caption}
\usepackage[margin=0.6in]{geometry}
\usepackage{multicol}
\usepackage[skip=4pt, indent=0pt]{parskip}
\usepackage[normalem]{ulem}
\forestset{default}
\renewcommand\labelitemi{$\bullet$}
\usepackage{titlesec}
\titleformat{\section}[block]
{\fontsize{15}{15}}
{\sen \dotfill (\thesection)\dotfill\she}
{0em}
{\MakeUppercase}
\usepackage{graphicx}
\graphicspath{ {./} }

\usepackage[colorlinks]{hyperref}
\definecolor{mgreen}{RGB}{25, 160, 50}
\definecolor{mblue}{RGB}{30, 60, 200}
\usepackage{hyperref}
\hypersetup{
	colorlinks=true,
	citecolor=mgreen,
	linkcolor=black,
	urlcolor=mblue,
	pdftitle={Document by Shahar Perets},
	%	pdfpagemode=FullScreen,
}
\usepackage{yfonts}
\def\gothstart#1{\noindent\smash{\lower3ex\hbox{\llap{\Huge\gothfamily#1}}}
	\parshape=3 3.1em \dimexpr\hsize-3.4em 3.4em \dimexpr\hsize-3.4em 0pt \hsize}
\def\frakstart#1{\noindent\smash{\lower3ex\hbox{\llap{\Huge\frakfamily#1}}}
	\parshape=3 1.5em \dimexpr\hsize-1.5em 2em \dimexpr\hsize-2em 0pt \hsize}



% Hebrew initialzing
\usepackage[bidi=basic]{babel}
\PassOptionsToPackage{no-math}{fontspec}
\babelprovide[main, import, Alph=letters]{hebrew}
\babelprovide[import]{english}
\babelfont[hebrew]{rm}{David CLM}
\babelfont[hebrew]{sf}{David CLM}
%\babelfont[english]{tt}{Monaspace Xenon}
\usepackage[shortlabels]{enumitem}
\newlist{hebenum}{enumerate}{1}

% Language Shortcuts
\newcommand\en[1] {\begin{otherlanguage}{english}#1\end{otherlanguage}}
\newcommand\he[1] {\she#1\sen}
\newcommand\sen   {\begin{otherlanguage}{english}}
	\newcommand\she   {\end{otherlanguage}}
\newcommand\del   {$ \!\! $}

\newcommand\npage {\vfil {\hfil \textbf{\textit{המשך בעמוד הבא}}} \hfil \vfil \pagebreak}
\newcommand\ndoc  {\dotfill \\ \vfil {\begin{center}
			{\textbf{\textit{שחר פרץ, 2025}} \\
				\scriptsize \textit{קומפל ב־}\en{\LaTeX}\,\textit{ ונוצר באמצעות תוכנה חופשית בלבד}}
	\end{center}} \vfil	}

\newcommand{\rn}[1]{
	\textup{\uppercase\expandafter{\romannumeral#1}}
}

\makeatletter
\newcommand{\skipitems}[1]{
	\addtocounter{\@enumctr}{#1}
}
\makeatother

%! ~~~ Math shortcuts ~~~

% Letters shortcuts
\newcommand\N     {\mathbb{N}}
\newcommand\Z     {\mathbb{Z}}
\newcommand\R     {\mathbb{R}}
\newcommand\Q     {\mathbb{Q}}
\newcommand\C     {\mathbb{C}}
\newcommand\One   {\mathit{1}}

\newcommand\ml    {\ell}
\newcommand\mj    {\jmath}
\newcommand\mi    {\imath}

\newcommand\powerset {\mathcal{P}}
\newcommand\ps    {\mathcal{P}}
\newcommand\pc    {\mathcal{P}}
\newcommand\ac    {\mathcal{A}}
\newcommand\bc    {\mathcal{B}}
\newcommand\cc    {\mathcal{C}}
\newcommand\dc    {\mathcal{D}}
\newcommand\ec    {\mathcal{E}}
\newcommand\fc    {\mathcal{F}}
\newcommand\nc    {\mathcal{N}}
\newcommand\vc    {\mathcal{V}} % Vance
\newcommand\sca   {\mathcal{S}} % \sc is already definded
\newcommand\rca   {\mathcal{R}} % \rc is already definded
\newcommand\zc    {\mathcal{Z}}

\newcommand\prm   {\mathrm{p}}
\newcommand\arm   {\mathrm{a}} % x86
\newcommand\brm   {\mathrm{b}}
\newcommand\crm   {\mathrm{c}}
\newcommand\drm   {\mathrm{d}}
\newcommand\erm   {\mathrm{e}}
\newcommand\frm   {\mathrm{f}}
\newcommand\nrm   {\mathrm{n}}
\newcommand\vrm   {\mathrm{v}}
\newcommand\srm   {\mathrm{s}}
\newcommand\rrm   {\mathrm{r}}

\newcommand\Si    {\Sigma}

% Logic & sets shorcuts
\newcommand\siff  {\longleftrightarrow}
\newcommand\ssiff {\leftrightarrow}
\newcommand\so    {\longrightarrow}
\newcommand\sso   {\rightarrow}

\newcommand\epsi  {\epsilon}
\newcommand\vepsi {\varepsilon}
\newcommand\vphi  {\varphi}
\newcommand\Neven {\N_{\mathrm{even}}}
\newcommand\Nodd  {\N_{\mathrm{odd }}}
\newcommand\Zeven {\Z_{\mathrm{even}}}
\newcommand\Zodd  {\Z_{\mathrm{odd }}}
\newcommand\Np    {\N_+}

% Text Shortcuts
\newcommand\open  {\big(}
\newcommand\qopen {\quad\big(}
\newcommand\close {\big)}
\newcommand\also  {\mathrm{, }}
\newcommand\defis {\mathrm{ definitions}}
\newcommand\given {\mathrm{given }}
\newcommand\case  {\mathrm{if }}
\newcommand\syx   {\mathrm{ syntax}}
\newcommand\rle   {\mathrm{ rule}}
\newcommand\other {\mathrm{else}}
\newcommand\set   {\ell et \text{ }}
\newcommand\ans   {\mathscr{A}\!\mathit{nswer}}

% Set theory shortcuts
\newcommand\ra    {\rangle}
\newcommand\la    {\langle}

\newcommand\oto   {\leftarrow}

\newcommand\QED   {\quad\quad\mathscr{Q.E.D.}\;\;\blacksquare}
\newcommand\QEF   {\quad\quad\mathscr{Q.E.F.}}
\newcommand\eQED  {\mathscr{Q.E.D.}\;\;\blacksquare}
\newcommand\eQEF  {\mathscr{Q.E.F.}}
\newcommand\jQED  {\mathscr{Q.E.D.}}

\DeclareMathOperator\dom   {dom}
\DeclareMathOperator\Img   {Im}
\DeclareMathOperator\range {range}

\newcommand\trio  {\triangle}

\newcommand\rc    {\right\rceil}
\newcommand\lc    {\left\lceil}
\newcommand\rf    {\right\rfloor}
\newcommand\lf    {\left\lfloor}
\newcommand\ceil  [1] {\lc #1 \rc}
\newcommand\floor [1] {\lf #1 \rf}

\newcommand\lex   {<_{lex}}

\newcommand\az    {\aleph_0}
\newcommand\uaz   {^{\aleph_0}}
\newcommand\al    {\aleph}
\newcommand\ual   {^\aleph}
\newcommand\taz   {2^{\aleph_0}}
\newcommand\utaz  { ^{\left (2^{\aleph_0} \right )}}
\newcommand\tal   {2^{\aleph}}
\newcommand\utal  { ^{\left (2^{\aleph} \right )}}
\newcommand\ttaz  {2^{\left (2^{\aleph_0}\right )}}

\newcommand\n     {$n$־יה\ }

% Math A&B shortcuts
\newcommand\logn  {\log n}
\newcommand\logx  {\log x}
\newcommand\lnx   {\ln x}
\newcommand\cosx  {\cos x}
\newcommand\sinx  {\sin x}
\newcommand\sint  {\sin \theta}
\newcommand\tanx  {\tan x}
\newcommand\tant  {\tan \theta}
\newcommand\sex   {\sec x}
\newcommand\sect  {\sec^2}
\newcommand\cotx  {\cot x}
\newcommand\cscx  {\csc x}
\newcommand\sinhx {\sinh x}
\newcommand\coshx {\cosh x}
\newcommand\tanhx {\tanh x}

\newcommand\seq   {\overset{!}{=}}
\newcommand\slh   {\overset{LH}{=}}
\newcommand\sle   {\overset{!}{\le}}
\newcommand\sge   {\overset{!}{\ge}}
\newcommand\sll   {\overset{!}{<}}
\newcommand\sgg   {\overset{!}{>}}

\newcommand\h     {\hat}
\newcommand\ve    {\vec}
\newcommand\lv    {\overrightarrow}
\newcommand\ol    {\overline}

\newcommand\mlcm  {\mathrm{lcm}}

\DeclareMathOperator{\sech}   {sech}
\DeclareMathOperator{\csch}   {csch}
\DeclareMathOperator{\arcsec} {arcsec}
\DeclareMathOperator{\arccot} {arcCot}
\DeclareMathOperator{\arccsc} {arcCsc}
\DeclareMathOperator{\arccosh}{arccosh}
\DeclareMathOperator{\arcsinh}{arcsinh}
\DeclareMathOperator{\arctanh}{arctanh}
\DeclareMathOperator{\arcsech}{arcsech}
\DeclareMathOperator{\arccsch}{arccsch}
\DeclareMathOperator{\arccoth}{arccoth}
\DeclareMathOperator{\atant}  {atan2} 
\DeclareMathOperator{\Sp}     {span} 
\DeclareMathOperator{\sgn}    {sgn} 
\DeclareMathOperator{\row}    {Row} 
\DeclareMathOperator{\adj}    {adj} 
\DeclareMathOperator{\rk}     {rank} 
\DeclareMathOperator{\col}    {Col} 
\DeclareMathOperator{\tr}     {tr}

\newcommand\dx    {\,\mathrm{d}x}
\newcommand\dt    {\,\mathrm{d}t}
\newcommand\dtt   {\,\mathrm{d}\theta}
\newcommand\du    {\,\mathrm{d}u}
\newcommand\dv    {\,\mathrm{d}v}
\newcommand\df    {\mathrm{d}f}
\newcommand\dfdx  {\diff{f}{x}}
\newcommand\dit   {\limhz \frac{f(x + h) - f(x)}{h}}

\newcommand\nt[1] {\frac{#1}{#1}}

\newcommand\limz  {\lim_{x \to 0}}
\newcommand\limxz {\lim_{x \to x_0}}
\newcommand\limi  {\lim_{x \to \infty}}
\newcommand\limh  {\lim_{x \to 0}}
\newcommand\limni {\lim_{x \to - \infty}}
\newcommand\limpmi{\lim_{x \to \pm \infty}}

\newcommand\ta    {\theta}
\newcommand\ap    {\alpha}

\renewcommand\inf {\infty}
\newcommand  \ninf{-\inf}

% Combinatorics shortcuts
\newcommand\sumnk     {\sum_{k = 0}^{n}}
\newcommand\sumni     {\sum_{i = 0}^{n}}
\newcommand\sumnko    {\sum_{k = 1}^{n}}
\newcommand\sumnio    {\sum_{i = 1}^{n}}
\newcommand\sumai     {\sum_{i = 1}^{n} A_i}
\newcommand\nsum[2]   {\reflectbox{\displaystyle\sum_{\reflectbox{\scriptsize$#1$}}^{\reflectbox{\scriptsize$#2$}}}}

\newcommand\bink      {\binom{n}{k}}
\newcommand\setn      {\{a_i\}^{2n}_{i = 1}}
\newcommand\setc[1]   {\{a_i\}^{#1}_{i = 1}}

\newcommand\cupain    {\bigcup_{i = 1}^{n} A_i}
\newcommand\cupai[1]  {\bigcup_{i = 1}^{#1} A_i}
\newcommand\cupiiai   {\bigcup_{i \in I} A_i}
\newcommand\capain    {\bigcap_{i = 1}^{n} A_i}
\newcommand\capai[1]  {\bigcap_{i = 1}^{#1} A_i}
\newcommand\capiiai   {\bigcap_{i \in I} A_i}

\newcommand\xot       {x_{1, 2}}
\newcommand\ano       {a_{n - 1}}
\newcommand\ant       {a_{n - 2}}

% Linear Algebra
\DeclareMathOperator{\chr}     {char}
\DeclareMathOperator{\diag}    {diag}
\DeclareMathOperator{\Hom}     {Hom}
\DeclareMathOperator{\Sym}     {Sym}
\DeclareMathOperator{\Asym}    {ASym}

\newcommand\lra       {\leftrightarrow}
\newcommand\chrf      {\chr(\F)}
\newcommand\F         {\mathbb{F}}
\newcommand\co        {\colon}
\newcommand\tmat[2]   {\cl{\begin{matrix}
			#1
		\end{matrix}\, \middle\vert\, \begin{matrix}
			#2
\end{matrix}}}

\makeatletter
\newcommand\rrr[1]    {\xxrightarrow{1}{#1}}
\newcommand\rrt[2]    {\xxrightarrow{1}[#2]{#1}}
\newcommand\mat[2]    {M_{#1\times#2}}
\newcommand\gmat      {\mat{m}{n}(\F)}
\newcommand\tomat     {\, \dequad \longrightarrow}
\newcommand\pms[1]    {\begin{pmatrix}
		#1
\end{pmatrix}}

\newcommand\norm[1]   {\left \vert \left \vert #1 \right \vert \right \vert}
\newcommand\snorm     {\left \vert \left \vert \cdot \right \vert \right \vert}
\newcommand\smut      {\left \la \cdot \mid \cdot \right \ra}
\newcommand\mut[2]    {\left \la #1 \,\middle\vert\, #2 \right \ra}

% someone's code from the internet: https://tex.stackexchange.com/questions/27545/custom-length-arrows-text-over-and-under
\makeatletter
\newlength\min@xx
\newcommand*\xxrightarrow[1]{\begingroup
	\settowidth\min@xx{$\m@th\scriptstyle#1$}
	\@xxrightarrow}
\newcommand*\@xxrightarrow[2][]{
	\sbox8{$\m@th\scriptstyle#1$}  % subscript
	\ifdim\wd8>\min@xx \min@xx=\wd8 \fi
	\sbox8{$\m@th\scriptstyle#2$} % superscript
	\ifdim\wd8>\min@xx \min@xx=\wd8 \fi
	\xrightarrow[{\mathmakebox[\min@xx]{\scriptstyle#1}}]
	{\mathmakebox[\min@xx]{\scriptstyle#2}}
	\endgroup}
\makeatother


% Greek Letters
\newcommand\ag        {\alpha}
\newcommand\bg        {\beta}
\newcommand\cg        {\gamma}
\newcommand\dg        {\delta}
\newcommand\eg        {\epsi}
\newcommand\zg        {\zeta}
\newcommand\hg        {\eta}
\newcommand\tg        {\theta}
\newcommand\ig        {\iota}
\newcommand\kg        {\keppa}
\renewcommand\lg      {\lambda}
\newcommand\og        {\omicron}
\newcommand\rg        {\rho}
\newcommand\sg        {\sigma}
\newcommand\yg        {\usilon}
\newcommand\wg        {\omega}

\newcommand\Ag        {\Alpha}
\newcommand\Bg        {\Beta}
\newcommand\Cg        {\Gamma}
\newcommand\Dg        {\Delta}
\newcommand\Eg        {\Epsi}
\newcommand\Zg        {\Zeta}
\newcommand\Hg        {\Eta}
\newcommand\Tg        {\Theta}
\newcommand\Ig        {\Iota}
\newcommand\Kg        {\Keppa}
\newcommand\Lg        {\Lambda}
\newcommand\Og        {\Omicron}
\newcommand\Rg        {\Rho}
\newcommand\Sg        {\Sigma}
\newcommand\Yg        {\Usilon}
\newcommand\Wg        {\Omega}

% Other shortcuts
\newcommand\tl    {\tilde}
\newcommand\op    {^{-1}}

\newcommand\sof[1]    {\left | #1 \right |}
\newcommand\cl [1]    {\left ( #1 \right )}
\newcommand\csb[1]    {\left [ #1 \right ]}
\newcommand\ccb[1]    {\left \{ #1 \right \}}

\newcommand\bs        {\blacksquare}
\newcommand\dequad    {\!\!\!\!\!\!}
\newcommand\dequadd   {\dequad\duquad}

\renewcommand\phi     {\varphi}

\newtheorem{Theorem}{משפט}
\theoremstyle{definition}
\newtheorem{definition}{הגדרה}
\newtheorem{Lemma}{למה}
\newtheorem{Remark}{הערה}
\newtheorem{Notion}{סימון}


\newcommand\theo  [1] {\begin{Theorem}#1\end{Theorem}}
\newcommand\defi  [1] {\begin{definition}#1\end{definition}}
\newcommand\rmark [1] {\begin{Remark}#1\end{Remark}}
\newcommand\lem   [1] {\begin{Lemma}#1\end{Lemma}}
\newcommand\noti  [1] {\begin{Notion}#1\end{Notion}}

% DS
\newcommand\limsi     {\limsup_{n \to \inf}}
\newcommand\limfi     {\liminf_{n \to \inf}}

\DeclareMathOperator\amort   {amort}
\DeclareMathOperator\worst   {worst}
\DeclareMathOperator\type    {type}
\DeclareMathOperator\cost    {cost}
\DeclareMathOperator\tim     {time}

\newcommand\dsList{
	\sFunc{List}
	\sFunc{Retrieve}
	\SetKwFunction{RetrieveFirst}{Retrieve-First}
	\SetKwFunction{RetrieveLast}{Retrieve-Last}
	\sFunc{Delete}
	\SetKwFunction{DeleteFirst}{Delete-First}
	\SetKwFunction{DeleteLast}{Delete-Last}
	\sFunc{Insert}
	\SetKwFunction{InsertFirst}{Insert-First}
	\SetKwFunction{InsertLast}{Insert-Last}
	\sFunc{Shift}
	\sFunc{Length}
	\sFunc{Concat}
	\sFunc{Plant}
	\sFunc{Split}
}
\newcommand\dsQueue{
	\sFunc{Queue}
	\sFunc{Enqueue}
	\sFunc{Head}
	\sFunc{Dequeue}
}
\newcommand\dsStack{
	\sFunc{Stack}
	\sFunc{Push}
	\sFunc{Top}
	\sFunc{Pop}
}
\newcommand\dsVector{
	\sFunc{Vector}
	\sFunc{Get}
	\sFunc{Set}
}
\newcommand\dsGraph{
	\sFunc{Graph}
	\sFunc{Edge}
	\SetKwFunction{AddEdge}{Add-Edge}
	\SetKwFunction{RemoveEdge}{Remove-Edge}
	\sFunc{InDeg} \sFunc{OutDeg}
}
\newcommand\importDs{
	\dsList
	\dsQueue
	\dsStack
	\dsVector
	\dsGraph
	\SetKwProg{Fn}{function}{ is}{end}
	\SetKwData{error}{\color{codered}error}
	\SetKwInOut{Input}{input}
	\SetKwInOut{Output}{output}
	\SetKwRepeat{Do}{do}{while}
	\SetKwData{Null}{\color{codegreen}null}
	\SetKwData{True}{\color{codeblue}true}
	\SetKwData{False}{\color{codeblue}false}
}


% Algorithems
\newcommand\sFunc [1] {\SetKwFunction{#1}{#1}}
\newcommand\sData [1] {\SetKwData{#1}{#1}}
\newcommand\sIO   [1] {\SetKwInOut{#1}{#1}}
\newcommand\ttt   [1] {\sen \texttt{#1} \she\,}
\newcommand\io    [2] {\Input{#1}\Output{#2}\BlankLine}

%! ~~~ Document ~~~

\author{שחר פרץ}
\title{\textit{לינארית 2א $\sim$ משפט הפירוק הספקטרלי להעתקות כלליות}}
\begin{document}
	\maketitle
	\textbf{מבוסס על הקלטה 18 של הקורס ב־2024-2025}
	
	\textbf{מרצה: ענת אמיר}
	
	המטרה: להבין לאילו העתקות בדיוק מתקיים המשפט הספקטרלי. מעל הממשיים, הבנו שאילו העתקות צמודות לעצמן. אז מה קורה מעל המרוכבים? 
	
	\theo{(משפט z). יהי $V$ ממ''פ סופי ויהי $\vphi \in V^*$. אז קיים ויחיד וקטור $u \in V$ שמקיים $\forall v \in V \co \vphi(v) = \mut{v}{u}$. }\begin{proof}
		\textbf{קיום. }יהי $B = (b_i)_{i = 1}^{n}$ בסיס אורתונורמלי של $V$ (הוכחנו קיום בהרצאות קודמות). נסמן $u = \sum_{i =1}^{n}\ol{\phi(b_i)} b_i$. בכדי להראות $\forall v \in V \co \phi(v) = \mut{v}{u}$ מספיק להראות תכונה זו לאברי הבסיס $B$, כלומר נראה ש־$\forall 1 \le j \le n \co \phi(b_j) = \mut{b_j}{u}$. ואכן: 
		\[ \mut{b_j}{u} = \mut{b_j}{\sumni \ol{\phi(b_i)}b_i} = \sum_{i = 1}^{n}\underbrace{\ol{\ol{\phi(b_i)}}}_{b_i}\underbrace{\mut{b_j}{b_i}}_{\dj_{ij}} = b_j \quad \top \]
		\textbf{יחידות: }אם קיים וקטור נוסף שעבורו $\forall v \in V \co \phi(v) = \mut{v}{w}$ אז בפרט עבור $v = u - w$ נקבל: 
		\[ \phi(v) = \mut{v}{w} = \mut{v}{u} \implies \mut{v}{u - w} = 0 \implies 0 = \mut{u - w}{u - w} = \norm{v - w}^{2} = 0 \implies v - w = 0 \implies v = w \]
		סה''כ הוכחנו קיום ויחידות. 
	\end{proof}
	
	\theo{יהי $V$ ממ''פ מנ''ס ותהי $T \co V \to V$ לינארית. אז קיימת ויחידה $T^* \co V \to V$ ומקיימת $\forall u ,v \in V \co \mut{Tu}{v} = \mut{u}{T^*v}$. }
	\defi{ההעתקה $T^*$ לעיל נקראת \textit{ההעתקה הצמודה ל־$T$}. }
	\begin{proof}
		לכל $v \in V$, נתבונן בפונקציונל הלינארי $\phi_V \in V^*$ המוגדר ע''י $\forall u \in V \co \phi_V(u) = \mut{Tu}{v}$. ממשפט ריס קיים ויחיד $T^*v \in V$ שעבורו $\forall u \in V \co \mut{Tu}{v} = \phi_V(u) = \mut{u}{T^*v}$. כלומר, ההעתקה $T^* \co V \to V$, קיימת ויחידה, ונותר להראות שהיא לינארית. עבור $v, w \in V$ ועבור $\ag, \bg \in \F$ מתקיים: 
		\[ \forall u \in V \co \mut{u}{T^* (\ag v + \bg w)} = \mut{Tu}{\ag v + \bg w} = \bar \ag \mut{Tu}{v} + \bar \bg \mut{Tu}{v} = \bar \ag \mut{u}{T^* v} + \bar \bg \mut{u}{T^*w} = \mut{u}{\ag T^*u + \bg T^*w} \]
		מסך נסיק ש־$T^*(\ag v + \bg w) = \ag T^* u + \bg T^* w$ מנימוקים דומים. 
	\end{proof}
	
	\textbf{דוגמאות. }
		מעל $\C^n$, עם המ''פ הסטנדרטי, נגדיר ט''ל $T_A \co \C^n \to \C^n$ עבור $A \in M_n(\C)$ מוגדרת ע''י $T_A(x) = Ax$. אז: 
		\[ \forall x, y \in \C^n \co \mut{T_A(x)}{y} = \mut{Ax}{y} = \ol{(Ax)^T} \cdot y = \ol{x^T}\cdot\ol{A^T}y\cdot = \ol{x^T}T_{\ol{A^T}}(y) = \mut{x}{T_{\ol{A^T}}y} \]
		כלומר, $(T_A)^* = T_{A^*}$ כאשר $A^* = \ol{A^T}$, וקראנו לה המטריצה הצמודה. 
	
	\textit{הערה: ההוכחה הזו הפוכה כי צריך לינאריות ברכיב הראשון ולא בשני והמרצה התבלבלה אבל אין לי כוח לתקן. עדיין $(T_A)^* = T_{A^*}$. }
	
	נבחין שהעתקה נקראת צמודה לעצמה אממ $T^* = T$. 
	
	עוד נבחין שעבור העתקה הסיבוב $T \co \R^2 \to \R^2$ בזווית $\tg$, מתקיים ש־$T^*$ היא הסיבוב ב־$-\tg$, וכן היא גם ההופכית לה. כלומר $(T_{\tg})^* = T_{-\tg} = (T_\tg)\op$. זו תכונה מאוד מועילה וגם נמציא לה שם במועד מאוחר יותר. 
	
	\theo{(תכונות ההעתקה הצמודה) יהי $V$ ממ''פ ותהיינה $T, S \co V \to V$ זוג העתקות לינאריות. נבחין ש־: 
	\begin{enumerate}[(A)]
		\item \hfil $(T^*)^* = T$
		\item \hfil $(T \circ S)^* = S^* \circ T^*$
		\item \hfil $(T + S)^* = T^* + S^*$
		\item \hfil $\forall \lg \in \F \co (\lg T)^* = \bar \lg (T^*)$
	\end{enumerate}}
	``זה אחד וחצי לינאריות''
	
	\begin{proof}\,
		\begin{enumerate}[A)]
			\item \hfil $\forall u, v \in V \co \mut{T^* u}{v} = \ol{\mut{v}{T^* u}}  = \ol{\mut{Tv}{u}}  = \mut{u}{Tv} \implies (T^*)^* = T$
			\item \hfil $\mut{(T \circ S) u}{v} = \mut{Su}{T^*v} = \mut{u}{S^* T^*} \implies (TS)^* = T^*S^*$
			\item \hfil $\mut{(T + S)u}{v} = \mut{Tu}{v} + \mut{Su}{v} = \mut{u}{T^*v} + \mut{u}{S^*v}  = \mut{u}{T^*v + S^*v}$
			\item כנ''ל
		\end{enumerate}
	\end{proof}
	
	\theo{יהי $V$ ממ''פ נ''ס ותהי $T \co V \to V$ לינאריות. אם $B = (b_i)_{i = 1}^{n}$ בסיס אורתוגונלי לו''ע של $T$, אז $\forall 1 \le i \le n$ ו''ע של ההעתקה הצמודה. }
	כלומר: אם מתקיים המשפט הספקטרלי, אז הבסיס שמלכסן אורתוגונלית את $T$ מלכסן אורתוגונלית את הצמודה. 
	\begin{proof}
		יהי $i \in [n]$ ונסמן בעבורו את $\lg_i$ הע''ע המתאים לו''ע $b_i$. עבור $i \neq j \in [n]$ נחשב את $\mut{b_i}{T^*b_j}$: 
		\[ \mut{b_i}{T^*b_j} = \ol{\mut{Tb_i}{b_j}} = \ol{\mut{\lg _i b_i}{b_j}} = \lg_i \mut{b_i}{b_j} = 0 \]
		לכן $T^*b_j \in (\Sp\{b_i\}_{i = 1}^{n})^{\perp} \seq \Sp\{b_j\}$. משיקולי ממדים, הפריסה מממד $n - 1$ ולכן המשלים האורתוגונלי שלו מממד $1$ ולכן השוויון. סה''כ $T^* b_j \in \Sp\{b_j\}$ ולכן $b_j$ ו''ע של $T^*$ כדרוש. 
	\end{proof}
	
	\textbf{מסקנה. }אאם $V$ ממ''פ נ''ס ו־$T \co V \to V$ ט''ל עם בסיס מלכסן אורתוגונלי, אז $T, T^*$ מתחלפות כלומר $TT^* = T^*T$. \begin{proof}
		לפי הטענה הקודמת כל $b_i$ הוא ו''ע משותף ל־$T$ ול־$T^*$, ולכן: 
		\[ TT^*(b_i) = T(T^*(b_i)) = \bg_i T^*(b_i) = \bg_i \dg_i b_i = \ag_i \bg_i b_i = \ag_i T^(b_i) = T^*T(b_i) \]
		העתקה מוגדרת לפי מה שהיא עושה לבסיס ולכן $TT^* = T^*T$. 
	\end{proof}
	\defi{העתקה כזו המקיימת $AA^* = A^*A$ נקראת \textit{נורמלית}). }
	
	עתה, ננסה להראות שכל העתקה נורמלית מקיימת את המפשט הספקטרלי. 
	
	\theo{(המשפט הספקטרלי) יהי $V$ ממ''פ נוצר סופית מעל $\C$, ותהי $T \co V \to V$ לינארית. אז קיים בסיס אורתוגונלי של ו''ע של $T$ אמ''מ $T$ נורמלית. }
	\lem{יהי $V$ ממ''פ ותהיינה $S_1, S_2 \co V \to V$  זוג ט''ל צמודות ולעמן ומתחלפות (כלומר $S_1S_2 = S_2 S_1$). אז קיים בסיס אורתוגונלי של $V$ שמורכב מו''עים משופים ל־$S_1$ ול־$S_2$. }\begin{proof}
		ידוע ש־$S_1$ צמודה לעצמה, לכן לפי המשפט הספקטרלי להעתקות צמודות לעצמן (לא מעגלי כי הוכח בנפרד בהרצאה הקודמת), קיים לה לכסון אורותגונלי ובפרט $S_1$ לכסינה. נציג את $V$ כ־$V = \bigoplus_{i = 1}^{m} \ker(S_1 - \lg_iI)$, כאשר $\lg _1 \dots \lg_m$ הע''עים השונים של $S_1$. לכל $1 \le i \le m$  מתקיים ש־$V_{\lg_i}$ (המרחב העצמי) הוא $S_1$־אינווריאנטי שהרי אם $v \in V_{\lg_i}$ ונחשב: 
		\[ S_1(S_2 v) = S_2(S_1 v) = S_2(\lg_i v) = \lg _i S_2v \implies S_2 v \in V_{\lg_i} \]
		כאשר $S_2|_{V_{\lg _i}} \co V_{\lg_i} \to V_{\lg_i}$ צמודה לעצמה, ולכן המפשט הספקטרלי לצמודות לעצמן אומר שבתוך $V_{\lg_i}$ ישנו בסיס אורתוגונלי של ו''עים מ־$S_2$. האיחוד של כל הבסיסים הללו מכל מ''ע של $S_1$ יהיה בסיס אורתוגונלי של ו''עים משותפים ל־$S_1$ ול־$S_2$. 
	\end{proof}
	\begin{proof}[הוכחת המשפט הספקטרלי. ]\,
		\begin{itemize}
			\item[$\implies$] לפי המסקנה הקודמת, אם ישנו לכסון אורתוגונלי $T$ בהכרח נורמלית. 
			\item[$\impliedby$] נגדיר $S_1 = \frac{T + T^*}{2}, \ S_2 = \frac{T - T^*}{2i}$. הן וודאי צמודות לעצמן מהלינאריות וכל השטויות ממקודם, והן גם מתחלפות אם תטרחו להכפיל אותן. מהטענה קיים ל־$V$ בסיס אורתוגונלי של ו''עים משותפים ל־$S_1, S_2$ ונסמנו $\{b_i\}^{n}_{i = 1}$ וגם $S_1b_i = \ag_i b_i, \ S_2b_i= \bg_i b_i$. אפשר גם לטעון ש־$\ag_i, \bg_i \in \R$ אבל זה לא מועיל לנו. נשים לב ש־$T = S_1 + iS_2$, כלומר $\forall i \in [n] \co T(b_i) = S_1(b_i) + iS_2(b_i) = \ag_i b_i + i\bg_i b_i = (\ag + i\bg_i)b_i$ וזהו בסיס אורתוגונלי של ו''עים של $T$. 
		\end{itemize}
	\end{proof}
	
	למעשה, הבנו מהפירוק של $S_1, S_2$ ש־$S_1$ נותנת את החלק הממשי של הע''ע ו־$S_2$ את החלק המדומה. 
	
	``אגב – לא השתמשתי במשפט היסודי של האלגברה''
	
	
	
	
	\ndoc
\end{document}