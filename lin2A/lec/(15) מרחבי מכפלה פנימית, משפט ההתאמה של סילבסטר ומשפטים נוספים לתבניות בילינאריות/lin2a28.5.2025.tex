%! ~~~ Packages Setup ~~~ 
\documentclass[]{article}
\usepackage{lipsum}
\usepackage{rotating}


% Math packages
\usepackage[usenames]{color}
\usepackage{forest}
\usepackage{ifxetex,ifluatex,amssymb,amsmath,mathrsfs,amsthm,witharrows,mathtools,mathdots}
\usepackage{amsmath}
\WithArrowsOptions{displaystyle}
\renewcommand{\qedsymbol}{$\blacksquare$} % end proofs with \blacksquare. Overwrites the defualts. 
\usepackage{cancel,bm}
\usepackage[thinc]{esdiff}


% tikz
\usepackage{tikz}
\usetikzlibrary{graphs}
\newcommand\sqw{1}
\newcommand\squ[4][1]{\fill[#4] (#2*\sqw,#3*\sqw) rectangle +(#1*\sqw,#1*\sqw);}


% code 
\usepackage{algorithm2e}
\usepackage{listings}
\usepackage{xcolor}

\definecolor{codegreen}{rgb}{0,0.35,0}
\definecolor{codegray}{rgb}{0.5,0.5,0.5}
\definecolor{codenumber}{rgb}{0.1,0.3,0.5}
\definecolor{codeblue}{rgb}{0,0,0.5}
\definecolor{codered}{rgb}{0.5,0.03,0.02}
\definecolor{codegray}{rgb}{0.96,0.96,0.96}

\lstdefinestyle{pythonstylesheet}{
    language=Java,
    emphstyle=\color{deepred},
    backgroundcolor=\color{codegray},
    keywordstyle=\color{deepblue}\bfseries\itshape,
    numberstyle=\scriptsize\color{codenumber},
    basicstyle=\ttfamily\footnotesize,
    commentstyle=\color{codegreen}\itshape,
    breakatwhitespace=false, 
    breaklines=true, 
    captionpos=b, 
    keepspaces=true, 
    numbers=left, 
    numbersep=5pt, 
    showspaces=false,                
    showstringspaces=false,
    showtabs=false, 
    tabsize=4, 
    morekeywords={as,assert,nonlocal,with,yield,self,True,False,None,AssertionError,ValueError,in,else},              % Add keywords here
    keywordstyle=\color{codeblue},
    emph={var, List, Iterable, Iterator},          % Custom highlighting
    emphstyle=\color{codered},
    stringstyle=\color{codegreen},
    showstringspaces=false,
    abovecaptionskip=0pt,belowcaptionskip =0pt,
    framextopmargin=-\topsep, 
}
\newcommand\pythonstyle{\lstset{pythonstylesheet}}
\newcommand\pyl[1]     {{\lstinline!#1!}}
\lstset{style=pythonstylesheet}

\usepackage[style=1,skipbelow=\topskip,skipabove=\topskip,framemethod=TikZ]{mdframed}
\definecolor{bggray}{rgb}{0.85, 0.85, 0.85}
\mdfsetup{leftmargin=0pt,rightmargin=0pt,innerleftmargin=15pt,backgroundcolor=codegray,middlelinewidth=0.5pt,skipabove=5pt,skipbelow=0pt,middlelinecolor=black,roundcorner=5}
\BeforeBeginEnvironment{lstlisting}{\begin{mdframed}\vspace{-0.4em}}
    \AfterEndEnvironment{lstlisting}{\vspace{-0.8em}\end{mdframed}}


% Design
\usepackage[labelfont=bf]{caption}
\usepackage[margin=0.6in]{geometry}
\usepackage{multicol}
\usepackage[skip=4pt, indent=0pt]{parskip}
\usepackage[normalem]{ulem}
\forestset{default}
\renewcommand\labelitemi{$\bullet$}
\usepackage{titlesec}
\titleformat{\section}[block]
{\fontsize{15}{15}}
{\sen \dotfill (\thesection)\she}
{0em}
{\MakeUppercase}
\usepackage{graphicx}
\graphicspath{ {./} }

\usepackage[colorlinks]{hyperref}
\definecolor{mgreen}{RGB}{25, 160, 50}
\definecolor{mblue}{RGB}{30, 60, 200}
\usepackage{hyperref}
\hypersetup{
    colorlinks=true,
    citecolor=mgreen,
    linkcolor=black,
    urlcolor=mblue,
    pdftitle={Document by Shahar Perets},
    %	pdfpagemode=FullScreen,
}
\usepackage{yfonts}
\def\gothstart#1{\noindent\smash{\lower3ex\hbox{\llap{\Huge\gothfamily#1}}}
    \parshape=3 3.1em \dimexpr\hsize-3.4em 3.4em \dimexpr\hsize-3.4em 0pt \hsize}
\def\frakstart#1{\noindent\smash{\lower3ex\hbox{\llap{\Huge\frakfamily#1}}}
    \parshape=3 1.5em \dimexpr\hsize-1.5em 2em \dimexpr\hsize-2em 0pt \hsize}



% Hebrew initialzing
\usepackage[bidi=basic]{babel}
\PassOptionsToPackage{no-math}{fontspec}
\babelprovide[main, import, Alph=letters]{hebrew}
\babelprovide[import]{english}
\babelfont[hebrew]{rm}{David CLM}
\babelfont[hebrew]{sf}{David CLM}
%\babelfont[english]{tt}{Monaspace Xenon}
\usepackage[shortlabels]{enumitem}
\newlist{hebenum}{enumerate}{1}

% Language Shortcuts
\newcommand\en[1] {\begin{otherlanguage}{english}#1\end{otherlanguage}}
\newcommand\he[1] {\she#1\sen}
\newcommand\sen   {\begin{otherlanguage}{english}}
    \newcommand\she   {\end{otherlanguage}}
\newcommand\del   {$ \!\! $}

\newcommand\npage {\vfil {\hfil \textbf{\textit{המשך בעמוד הבא}}} \hfil \vfil \pagebreak}
\newcommand\ndoc  {\dotfill \\ \vfil {\begin{center}
            {\textbf{\textit{שחר פרץ, 2025}} \\
                \scriptsize \textit{קומפל ב־}\en{\LaTeX}\,\textit{ ונוצר באמצעות תוכנה חופשית בלבד}}
    \end{center}} \vfil	}

\newcommand{\rn}[1]{
    \textup{\uppercase\expandafter{\romannumeral#1}}
}

\makeatletter
\newcommand{\skipitems}[1]{
    \addtocounter{\@enumctr}{#1}
}
\makeatother

%! ~~~ Math shortcuts ~~~

% Letters shortcuts
\newcommand\N     {\mathbb{N}}
\newcommand\Z     {\mathbb{Z}}
\newcommand\R     {\mathbb{R}}
\newcommand\Q     {\mathbb{Q}}
\newcommand\C     {\mathbb{C}}
\newcommand\One   {\mathit{1}}

\newcommand\ml    {\ell}
\newcommand\mj    {\jmath}
\newcommand\mi    {\imath}

\newcommand\powerset {\mathcal{P}}
\newcommand\ps    {\mathcal{P}}
\newcommand\pc    {\mathcal{P}}
\newcommand\ac    {\mathcal{A}}
\newcommand\bc    {\mathcal{B}}
\newcommand\cc    {\mathcal{C}}
\newcommand\dc    {\mathcal{D}}
\newcommand\ec    {\mathcal{E}}
\newcommand\fc    {\mathcal{F}}
\newcommand\nc    {\mathcal{N}}
\newcommand\vc    {\mathcal{V}} % Vance
\newcommand\sca   {\mathcal{S}} % \sc is already definded
\newcommand\rca   {\mathcal{R}} % \rc is already definded

\newcommand\prm   {\mathrm{p}}
\newcommand\arm   {\mathrm{a}} % x86
\newcommand\brm   {\mathrm{b}}
\newcommand\crm   {\mathrm{c}}
\newcommand\drm   {\mathrm{d}}
\newcommand\erm   {\mathrm{e}}
\newcommand\frm   {\mathrm{f}}
\newcommand\nrm   {\mathrm{n}}
\newcommand\vrm   {\mathrm{v}}
\newcommand\srm   {\mathrm{s}}
\newcommand\rrm   {\mathrm{r}}

\newcommand\Si    {\Sigma}

% Logic & sets shorcuts
\newcommand\siff  {\longleftrightarrow}
\newcommand\ssiff {\leftrightarrow}
\newcommand\so    {\longrightarrow}
\newcommand\sso   {\rightarrow}

\newcommand\epsi  {\epsilon}
\newcommand\vepsi {\varepsilon}
\newcommand\vphi  {\varphi}
\newcommand\Neven {\N_{\mathrm{even}}}
\newcommand\Nodd  {\N_{\mathrm{odd }}}
\newcommand\Zeven {\Z_{\mathrm{even}}}
\newcommand\Zodd  {\Z_{\mathrm{odd }}}
\newcommand\Np    {\N_+}

% Text Shortcuts
\newcommand\open  {\big(}
\newcommand\qopen {\quad\big(}
\newcommand\close {\big)}
\newcommand\also  {\mathrm{, }}
\newcommand\defis {\mathrm{ definitions}}
\newcommand\given {\mathrm{given }}
\newcommand\case  {\mathrm{if }}
\newcommand\syx   {\mathrm{ syntax}}
\newcommand\rle   {\mathrm{ rule}}
\newcommand\other {\mathrm{else}}
\newcommand\set   {\ell et \text{ }}
\newcommand\ans   {\mathscr{A}\!\mathit{nswer}}

% Set theory shortcuts
\newcommand\ra    {\rangle}
\newcommand\la    {\langle}

\newcommand\oto   {\leftarrow}

\newcommand\QED   {\quad\quad\mathscr{Q.E.D.}\;\;\blacksquare}
\newcommand\QEF   {\quad\quad\mathscr{Q.E.F.}}
\newcommand\eQED  {\mathscr{Q.E.D.}\;\;\blacksquare}
\newcommand\eQEF  {\mathscr{Q.E.F.}}
\newcommand\jQED  {\mathscr{Q.E.D.}}

\DeclareMathOperator\dom   {dom}
\DeclareMathOperator\Img   {Im}
\DeclareMathOperator\range {range}

\newcommand\trio  {\triangle}

\newcommand\rc    {\right\rceil}
\newcommand\lc    {\left\lceil}
\newcommand\rf    {\right\rfloor}
\newcommand\lf    {\left\lfloor}
\newcommand\ceil  [1] {\lc #1 \rc}
\newcommand\floor [1] {\lf #1 \rf}

\newcommand\lex   {<_{lex}}

\newcommand\az    {\aleph_0}
\newcommand\uaz   {^{\aleph_0}}
\newcommand\al    {\aleph}
\newcommand\ual   {^\aleph}
\newcommand\taz   {2^{\aleph_0}}
\newcommand\utaz  { ^{\left (2^{\aleph_0} \right )}}
\newcommand\tal   {2^{\aleph}}
\newcommand\utal  { ^{\left (2^{\aleph} \right )}}
\newcommand\ttaz  {2^{\left (2^{\aleph_0}\right )}}

\newcommand\n     {$n$־יה\ }

% Math A&B shortcuts
\newcommand\logn  {\log n}
\newcommand\logx  {\log x}
\newcommand\lnx   {\ln x}
\newcommand\cosx  {\cos x}
\newcommand\sinx  {\sin x}
\newcommand\sint  {\sin \theta}
\newcommand\tanx  {\tan x}
\newcommand\tant  {\tan \theta}
\newcommand\sex   {\sec x}
\newcommand\sect  {\sec^2}
\newcommand\cotx  {\cot x}
\newcommand\cscx  {\csc x}
\newcommand\sinhx {\sinh x}
\newcommand\coshx {\cosh x}
\newcommand\tanhx {\tanh x}

\newcommand\seq   {\overset{!}{=}}
\newcommand\slh   {\overset{LH}{=}}
\newcommand\sle   {\overset{!}{\le}}
\newcommand\sge   {\overset{!}{\ge}}
\newcommand\sll   {\overset{!}{<}}
\newcommand\sgg   {\overset{!}{>}}

\newcommand\h     {\hat}
\newcommand\ve    {\vec}
\newcommand\lv    {\overrightarrow}
\newcommand\ol    {\overline}

\newcommand\mlcm  {\mathrm{lcm}}

\DeclareMathOperator{\sech}   {sech}
\DeclareMathOperator{\csch}   {csch}
\DeclareMathOperator{\arcsec} {arcsec}
\DeclareMathOperator{\arccot} {arcCot}
\DeclareMathOperator{\arccsc} {arcCsc}
\DeclareMathOperator{\arccosh}{arccosh}
\DeclareMathOperator{\arcsinh}{arcsinh}
\DeclareMathOperator{\arctanh}{arctanh}
\DeclareMathOperator{\arcsech}{arcsech}
\DeclareMathOperator{\arccsch}{arccsch}
\DeclareMathOperator{\arccoth}{arccoth}
\DeclareMathOperator{\atant}  {atan2} 
\DeclareMathOperator{\Sp}     {span} 
\DeclareMathOperator{\sgn}    {sgn} 
\DeclareMathOperator{\row}    {Row} 
\DeclareMathOperator{\adj}    {adj} 
\DeclareMathOperator{\rk}     {rank} 
\DeclareMathOperator{\col}    {Col} 
\DeclareMathOperator{\tr}     {tr}

\newcommand\dx    {\,\mathrm{d}x}
\newcommand\dt    {\,\mathrm{d}t}
\newcommand\dtt   {\,\mathrm{d}\theta}
\newcommand\du    {\,\mathrm{d}u}
\newcommand\dv    {\,\mathrm{d}v}
\newcommand\df    {\mathrm{d}f}
\newcommand\dfdx  {\diff{f}{x}}
\newcommand\dit   {\limhz \frac{f(x + h) - f(x)}{h}}

\newcommand\nt[1] {\frac{#1}{#1}}

\newcommand\limz  {\lim_{x \to 0}}
\newcommand\limxz {\lim_{x \to x_0}}
\newcommand\limi  {\lim_{x \to \infty}}
\newcommand\limh  {\lim_{x \to 0}}
\newcommand\limni {\lim_{x \to - \infty}}
\newcommand\limpmi{\lim_{x \to \pm \infty}}

\newcommand\ta    {\theta}
\newcommand\ap    {\alpha}

\renewcommand\inf {\infty}
\newcommand  \ninf{-\inf}

% Combinatorics shortcuts
\newcommand\sumnk     {\sum_{k = 0}^{n}}
\newcommand\sumni     {\sum_{i = 0}^{n}}
\newcommand\sumnko    {\sum_{k = 1}^{n}}
\newcommand\sumnio    {\sum_{i = 1}^{n}}
\newcommand\sumai     {\sum_{i = 1}^{n} A_i}
\newcommand\nsum[2]   {\reflectbox{\displaystyle\sum_{\reflectbox{\scriptsize$#1$}}^{\reflectbox{\scriptsize$#2$}}}}

\newcommand\bink      {\binom{n}{k}}
\newcommand\setn      {\{a_i\}^{2n}_{i = 1}}
\newcommand\setc[1]   {\{a_i\}^{#1}_{i = 1}}

\newcommand\cupain    {\bigcup_{i = 1}^{n} A_i}
\newcommand\cupai[1]  {\bigcup_{i = 1}^{#1} A_i}
\newcommand\cupiiai   {\bigcup_{i \in I} A_i}
\newcommand\capain    {\bigcap_{i = 1}^{n} A_i}
\newcommand\capai[1]  {\bigcap_{i = 1}^{#1} A_i}
\newcommand\capiiai   {\bigcap_{i \in I} A_i}

\newcommand\xot       {x_{1, 2}}
\newcommand\ano       {a_{n - 1}}
\newcommand\ant       {a_{n - 2}}

% Linear Algebra
\DeclareMathOperator{\chr}     {char}
\DeclareMathOperator{\diag}    {diag}
\DeclareMathOperator{\Hom}     {Hom}
\DeclareMathOperator{\Sym}     {Sym}
\DeclareMathOperator{\Asym}    {ASym}

\newcommand\lra       {\leftrightarrow}
\newcommand\chrf      {\chr(\F)}
\newcommand\F         {\mathbb{F}}
\newcommand\co        {\colon}
\newcommand\tmat[2]   {\cl{\begin{matrix}
            #1
        \end{matrix}\, \middle\vert\, \begin{matrix}
            #2
\end{matrix}}}

\makeatletter
\newcommand\rrr[1]    {\xxrightarrow{1}{#1}}
\newcommand\rrt[2]    {\xxrightarrow{1}[#2]{#1}}
\newcommand\mat[2]    {M_{#1\times#2}}
\newcommand\gmat      {\mat{m}{n}(\F)}
\newcommand\tomat     {\, \dequad \longrightarrow}
\newcommand\pms[1]    {\begin{pmatrix}
        #1
\end{pmatrix}}

\newcommand\norm[1]   {\left \vert \left \vert #1 \right \vert \right \vert}

% someone's code from the internet: https://tex.stackexchange.com/questions/27545/custom-length-arrows-text-over-and-under
\makeatletter
\newlength\min@xx
\newcommand*\xxrightarrow[1]{\begingroup
    \settowidth\min@xx{$\m@th\scriptstyle#1$}
    \@xxrightarrow}
\newcommand*\@xxrightarrow[2][]{
    \sbox8{$\m@th\scriptstyle#1$}  % subscript
    \ifdim\wd8>\min@xx \min@xx=\wd8 \fi
    \sbox8{$\m@th\scriptstyle#2$} % superscript
    \ifdim\wd8>\min@xx \min@xx=\wd8 \fi
    \xrightarrow[{\mathmakebox[\min@xx]{\scriptstyle#1}}]
    {\mathmakebox[\min@xx]{\scriptstyle#2}}
    \endgroup}
\makeatother


% Greek Letters
\newcommand\ag        {\alpha}
\newcommand\bg        {\beta}
\newcommand\cg        {\gamma}
\newcommand\dg        {\delta}
\newcommand\eg        {\epsi}
\newcommand\zg        {\zeta}
\newcommand\hg        {\eta}
\newcommand\tg        {\theta}
\newcommand\ig        {\iota}
\newcommand\kg        {\keppa}
\renewcommand\lg      {\lambda}
\newcommand\og        {\omicron}
\newcommand\rg        {\rho}
\newcommand\sg        {\sigma}
\newcommand\yg        {\usilon}
\newcommand\wg        {\omega}

\newcommand\Ag        {\Alpha}
\newcommand\Bg        {\Beta}
\newcommand\Cg        {\Gamma}
\newcommand\Dg        {\Delta}
\newcommand\Eg        {\Epsi}
\newcommand\Zg        {\Zeta}
\newcommand\Hg        {\Eta}
\newcommand\Tg        {\Theta}
\newcommand\Ig        {\Iota}
\newcommand\Kg        {\Keppa}
\newcommand\Lg        {\Lambda}
\newcommand\Og        {\Omicron}
\newcommand\Rg        {\Rho}
\newcommand\Sg        {\Sigma}
\newcommand\Yg        {\Usilon}
\newcommand\Wg        {\Omega}

% Other shortcuts
\newcommand\tl    {\tilde}
\newcommand\op    {^{-1}}

\newcommand\sof[1]    {\left | #1 \right |}
\newcommand\cl [1]    {\left ( #1 \right )}
\newcommand\csb[1]    {\left [ #1 \right ]}
\newcommand\mut[2]    {\left \la #1 \,\middle\vert\, #2 \right \ra}
\newcommand\smut      {\left \la \cdot \mid \cdot \right \ra}
\newcommand\ccb[1]    {\left \{ #1 \right \}}

\newcommand\bs        {\blacksquare}
\newcommand\dequad    {\!\!\!\!\!\!}
\newcommand\dequadd   {\dequad\duquad}

\renewcommand\phi     {\varphi}

\newtheorem{Theorem}{משפט}
\theoremstyle{definition}
\newtheorem{definition}{הגדרה}
\newtheorem{Lemma}{למה}
\newtheorem{Remark}{הערה}
\newtheorem{Notion}{סימון}


\newcommand\theo  [1] {\begin{Theorem}#1\end{Theorem}}
\newcommand\defi  [1] {\begin{definition}#1\end{definition}}
\newcommand\rmark [1] {\begin{Remark}#1\end{Remark}}
\newcommand\lem   [1] {\begin{Lemma}#1\end{Lemma}}
\newcommand\noti  [1] {\begin{Notion}#1\end{Notion}}


%! ~~~ Document ~~~

\author{שחר פרץ}
\title{\textit{לינארית 2א 15}}
\begin{document}
    \maketitle
    \textbf{מרצה: }בן בסקין
    
    \theo{לכל $f$ תבנית סימ' קיימת מטריצה מייצגת מהצורה $\binom{I_r \, 0}{0 \,\,\, 0}$ כאשר $\F = \C$ (או סגור אלגברית כלשהו)}. 
    \begin{proof}
            
        נסמן את $\dim f = r$. עד כדי שינוי סדר איברי הבסיס, המטריצה המייצגת אלכסונית היא:
        \[ [f]_B = \pms{\diag(c_1 \dots c_r) & 0 \\ 0 & 0} \]
        כאשר $c_1 \dots c_r \neq 0$, ביחס לבסיס $B = (v_1 \dots v_r, \dots v_n)$. באופן כללי לכל $i \in \R$ נוכל להגדיר את $v_i' = \frac{v_i}{\sqrt{c_i}}$ כך ש־$f(v'_i, v'_i) = 1$ כי $f(v_i, v_i) = c_i$ ומליניאריות בכל אחת מהקורדינאטות. ולכן $B' = (v_1' \dots v'_r, v_{r + 1} \dots v_n)$ בסיס המקיים את הדרוש. 
        
        באותו האופן, אם $\F = \R$ (ולא $\C$) אז קיים בסיס שהמטריצה המייצגת לפיו היא: 
        \[ \pms{I_p & 0 & 0 \\ 0& -I_q & 0 \\0 & 0 & 0} \]
        בלוקים, כך ש־$p + q = r$. כאן נגדיר: 
        \[ f(v, v) = c < 0, \ v' = \frac{v}{\sqrt{|c|}}, \ f(v', v') = \frac{c}{|c|} = -1 \]
    \end{proof}
    
    בשיעורי הבית נראה ש־: נניח ש־$f$ אנטי־סימטרית לא מנוונת (לא תבנית ה־$0$), אז תמיד ישנה מטריצה מייצגת מהצורה (תחפשו ``מטריצה סימפלקטית'' בגוגל, זה קצת סיוט לעשות את זה בלאטך). הרעיון הוא אם: 
    \[ \hat I_n = \pms{ && 1 \\ & \reflectbox{$\ddots$} & \\ 1}, \ J = \pms{0 & & -\hat I_n \\ & \ddots & \\ \hat I_n && 0} \]
    אז $J$ סימפלקטית. 
    
    \defi{יהי $V$ מ''ו מעל $\R$ ו־$f$ תבנית בילינ' מעל $V$. נאמר ש־$f$ חיובית/אי־שלילית/שלילית/אי־חיובית אם $\forall 0 \neq v \in V $ מתקיים ש־$f(v, v) > 0$/$f(v, v) \ge 0$/$f(v, v) < 0$/$f(v, v)\le 0$}
    
    \theo{תהא $A$ מטריצה מייגצת של תבנית בי־ליניארית סימ', עם ערכים $0, -1, 1$ בלבד על האלכסון, מקיימת: 
    \begin{itemize}
        \item $f$ חיובית אמ''מ ישנם רק $1$־ים. 
        \item $f$ אי־שלילית אמ''מ ישנם רק $1$־ים ואפסים. 
        \item $f$ שלילית אמ''מ ישנם רק $-1$־ים
        \item $f$ חיובית אמ''מ ישנם רק $-1$־ים ואפסים. 
    \end{itemize}}
    \begin{proof}\,
        \begin{itemize}
            \item[$\impliedby$] ברור
            \item לכל $0 \neq v \in V$ קיימים ויחידים $\ag_1 \dots \ag_n \in \R$ כך ש־$v = \sum^n_{i = 1} \ag_i v_i$ ומתקיים $f(v, v) = \ag_i^2 f(v_{i, i})$ ולפי המקרה זה יסתדר יפה. 
        \end{itemize}
    \end{proof}
    
    \theo{\textbf{משפט ההתאמה של סילבסטר. }$p, q$ הנ''ל נקבעים ביחידות. }(תחזרו כמה משפטים למעלה למקרה בו $\F = \R$)
    \begin{proof}[ההוכחה של קרני. ]
        נכתבה ונמחקה מהלוח. שימו לב שה־$\tr$ לא נשמר בשינוי בסיס של תבניות בילינאריות, זה לא העתקות. ההוכחה שגויה. 
    \end{proof}
    \begin{proof}
        נסמן $B = (v_1 \dots v_p, u \dots u_q, w_1 \dots w_k)$ וכן $B' = (v_1' \dots v'_t, u'_1 \dots u'_s, w_1 \dots w_k)$ כי $t + s = p + q$. בה''כ $t \le p$, נניח בשלילה ש־$t < p$. נסמן $U = \Sp(v_1 \dots v_p)$. ידוע $f$ חיובית על $U$, וכן $\dim U = p$. נתבונן ב־$W = \Sp(u_1' \dots u_s', w_1 \dots w_k)$. אזי גם $f$ חיובית על $W$, ו־$\dim W = s + k$. בגלל ש־$U \cap W = \{0\}$ (כי אם לא, אז עבור $0 \neq v \in U \cap W$ נקבל $f(v, v) > 0$ כי $v \in U$ וכן $f(v, v) \le 0$ כי $v \in W$ וסתירה). ידוע ש־$U \oplus W \subseteq V$ תמ''ו וכן $\dim U + \dim W \le \dim V$. נציב ונקבל $p + s + k > t + s + k = \dim V$, סתירה. לכן $p, q$ נקבעים ביחידות. 
    \end{proof}
    
    \noti{ה־$(p, q)$ לעיל נקראים \textit{הסיגנטורה של $f$}. }
    
    \section{\en{Inner Product Vector Spaces}}
    \subsection{מעל $\R$}
    \textbf{מעתה ועד סוף הקורס, }מתקיים $\F = \R, \C$. 
    
    כל עוד נאמר ``$\F$'', זה נכון בעבור שני המקרים. אחרת, נפצל. 
    
    \defi{יהי $V$ מ''ו, \textit{מכפלה פנימית} מעל $\R$ היא תבנית בילינ' סימטרית חיובית מעל $V$, ומסומנת $f(v, u) = \la v, u \ra$ (ויש ספרים שמסמנים $\la v \mid u \ra$, בדומה לסימון של קוונטים), ונסמן $\la \cdot, \cdot \ra\co V \times V \to \R$. }
    
    בגלל שהיא לינארית סימטרית, נקבל $\forall v \in V \co \la v, v \ra \ge 0$ ו־$\la v, v \ra$ אמ''מ $v = 0$. 
    
    \textbf{דוגמה. }(המכפלה הפנימית הסטנדרטית על $\R^n$, AKA כפל סקלרי): 
    \[ \mut{\pms{x_1 \\ \vdots\\ x_n}}{\pms{y_1 \\ \vdots \\ y_n}} = \sum_{i = 1}^{n}x_iy_i \]
    
    \defi{אם $V$ מ''ו וקיימת $\smut \co V \times V \to \F$ מכפלה פנימית אז $(V, \smut)$ נקרא \textit{מרחב מכפלה פנימית}, ממ''פ. }
    
    \theo{$V = M_n(\R)$, אז $\mut{A}{B} = \tr(A \cdot B^T)$ אז $(V, \smut)$ ממ''פ. }
    
    \textbf{דוגמה מגניבה. }בהינתן $V = c[0, 1]$, מ''ו הפונקציות הממשיות הרציפות על $[0, 1]$, ו־$\mut{f}{g} = \int_0^1 f(x) \cdot g(x) \dx$ 
    \theo{(שהפליצו מחדו''א) אם $f \ge 0$ אינטרבילית (זה נשמע כמו מפלצת) על קטע $[a, b]$ וגם ישנה נקודה חיובית $c \in [a, b]$ שעבורה $f(x) \ge 0$ וגם $f$ רציפה ב־$c$, אז $\int^b_a f(x) \dx > 0$. }
    
    \subsection{מעל $\C$}
    ישנה בעיה עם חיובית: אם $v \in V$ כך ש־$\mut{v}{v} \ge 0$ אך $\mut{iv}{iv} = -1\mut{v}{v} < 0$ סתירה. לכן, במקום זאת, נשתמש בהגדרה הבאה: 
    \defi{יהי $V$ מ''ו מעל $\C$. מכפלה פנימית $\smut \co V \times V \to \C$ מקיימת: 
    \begin{itemize}
        \item ליניאירות ברכיב הראשון: אם נקבע $v$, אז $u \mapsto \mut{v}{u}$ לינארית. 
        \item ססקווי־ליניאריות ברכיב השני: \hfill $\mut{u_1 + u_2}{v} = \mut{u_1}{v} + \mut{u_2}{v} \land \mut{u}{\ag v} = \bar \ag \mut{u}{v}$ 
        
        כאשר $\bar \ag$ הצמוד המרוכב של $\ag$. 
        \item הרמטיות: \hfill $\mut{v}{u} = \ol{\mut{u}{v}}$
        \item \hfil $\forall 0 \neq v \in V \co \mut{v}{v} > 0 \land \mut{0}{0} = 0$
    \end{itemize}}
    למעשה – נבחין שאין צורך בממש ססקווי־ליניאיריות ברכיב השני וכן לא בתנאי $\mut{0}{0} = 0$, וההגדרה שקולה בעבור חיבוריות ברכיב השני בלבד,. זאת כי: 
    \[ \mut{u}{\ag v} = \ol{\mut{\ag v}{u}} = \ol{\ag \mut{v}{u}} = \bar \ag \cdot \ol{\mut{v}{u}} = \bar \ag \mut{v}{u} \]
    ומכאן נגרר ססקווי־ליניאריות, וכן $\mut{0}{0} = 0$ נובע ישירות מליניאריות ברכיב השני. 
    
    (אופס! בן הגדיר את זה לליניאירות ברכיב השני, כלומר הפוך, כי ככה עושים את זה בפתוחה. תיקנתי בסיכום אבל יכול להיות שיש משהו הפוך כי פספסתי. זה אמור להיות ליניארי ברכיב השני).
    
    \defi{\hfil $\bar B^T = B^*$}
    
    \defi{יהי ממ''פ $V$ מ''ו מעל $\F$. לכל $v \in V$ מגדירים את ה\textit{נורמה} של $v$ להיות $ \norm{v} = \sqrt{\mut{v}{v}} $}
    
    מאקסיומת החיוביות: 
    \[ \norm{v} \ge 0 \land (\norm{v} = 0 \iff v = 0) \]
    וכן: 
    \[ \norm{t \cdot v^2} = \mut{tv}{tu} = t \bar t\mut{v}{v} = \sof{t}\norm{v} \implies \norm{t \cdot v} = \sof{t} \cdot \norm{v} \]
    
    
    
    
    
    
    
    \ndoc
\end{document}