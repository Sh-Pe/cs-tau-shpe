%! ~~~ Packages Setup ~~~ 
\documentclass[]{article}
\usepackage{lipsum}
\usepackage{rotating}


% Math packages
\usepackage[usenames]{color}
\usepackage{forest}
\usepackage{ifxetex,ifluatex,amssymb,amsmath,mathrsfs,amsthm,witharrows,mathtools,mathdots}
\usepackage{amsmath}
\WithArrowsOptions{displaystyle}
\renewcommand{\qedsymbol}{$\blacksquare$} % end proofs with \blacksquare. Overwrites the defualts. 
\usepackage{cancel,bm}
\usepackage[thinc]{esdiff}


% tikz
\usepackage{tikz}
\usetikzlibrary{graphs}
\newcommand\sqw{1}
\newcommand\squ[4][1]{\fill[#4] (#2*\sqw,#3*\sqw) rectangle +(#1*\sqw,#1*\sqw);}


% code 
\usepackage{algorithm2e}
\usepackage{listings}
\usepackage{xcolor}

\definecolor{codegreen}{rgb}{0,0.35,0}
\definecolor{codegray}{rgb}{0.5,0.5,0.5}
\definecolor{codenumber}{rgb}{0.1,0.3,0.5}
\definecolor{codeblue}{rgb}{0,0,0.5}
\definecolor{codered}{rgb}{0.5,0.03,0.02}
\definecolor{codegray}{rgb}{0.96,0.96,0.96}

\lstdefinestyle{pythonstylesheet}{
	language=Java,
	emphstyle=\color{deepred},
	backgroundcolor=\color{codegray},
	keywordstyle=\color{deepblue}\bfseries\itshape,
	numberstyle=\scriptsize\color{codenumber},
	basicstyle=\ttfamily\footnotesize,
	commentstyle=\color{codegreen}\itshape,
	breakatwhitespace=false, 
	breaklines=true, 
	captionpos=b, 
	keepspaces=true, 
	numbers=left, 
	numbersep=5pt, 
	showspaces=false,                
	showstringspaces=false,
	showtabs=false, 
	tabsize=4, 
	morekeywords={as,assert,nonlocal,with,yield,self,True,False,None,AssertionError,ValueError,in,else},              % Add keywords here
	keywordstyle=\color{codeblue},
	emph={var, List, Iterable, Iterator},          % Custom highlighting
	emphstyle=\color{codered},
	stringstyle=\color{codegreen},
	showstringspaces=false,
	abovecaptionskip=0pt,belowcaptionskip =0pt,
	framextopmargin=-\topsep, 
}
\newcommand\pythonstyle{\lstset{pythonstylesheet}}
\newcommand\pyl[1]     {{\lstinline!#1!}}
\lstset{style=pythonstylesheet}

\usepackage[style=1,skipbelow=\topskip,skipabove=\topskip,framemethod=TikZ]{mdframed}
\definecolor{bggray}{rgb}{0.85, 0.85, 0.85}
\mdfsetup{leftmargin=0pt,rightmargin=0pt,innerleftmargin=15pt,backgroundcolor=codegray,middlelinewidth=0.5pt,skipabove=5pt,skipbelow=0pt,middlelinecolor=black,roundcorner=5}
\BeforeBeginEnvironment{lstlisting}{\begin{mdframed}\vspace{-0.4em}}
	\AfterEndEnvironment{lstlisting}{\vspace{-0.8em}\end{mdframed}}


% Design
\usepackage[labelfont=bf]{caption}
\usepackage[margin=0.6in]{geometry}
\usepackage{multicol}
\usepackage[skip=4pt, indent=0pt]{parskip}
\usepackage[normalem]{ulem}
\forestset{default}
\renewcommand\labelitemi{$\bullet$}
\usepackage{titlesec}
\titleformat{\section}[block]
{\fontsize{15}{15}}
{\sen \dotfill (\thesection)\she}
{0em}
{\MakeUppercase}
\usepackage{graphicx}
\graphicspath{ {./} }

\usepackage[colorlinks]{hyperref}
\definecolor{mgreen}{RGB}{25, 160, 50}
\definecolor{mblue}{RGB}{30, 60, 200}
\usepackage{hyperref}
\hypersetup{
	colorlinks=true,
	citecolor=mgreen,
	linkcolor=black,
	urlcolor=mblue,
	pdftitle={Document by Shahar Perets},
	%	pdfpagemode=FullScreen,
}
\usepackage{yfonts}
\def\gothstart#1{\noindent\smash{\lower3ex\hbox{\llap{\Huge\gothfamily#1}}}
	\parshape=3 3.1em \dimexpr\hsize-3.4em 3.4em \dimexpr\hsize-3.4em 0pt \hsize}
\def\frakstart#1{\noindent\smash{\lower3ex\hbox{\llap{\Huge\frakfamily#1}}}
	\parshape=3 1.5em \dimexpr\hsize-1.5em 2em \dimexpr\hsize-2em 0pt \hsize}



% Hebrew initialzing
\usepackage[bidi=basic]{babel}
\PassOptionsToPackage{no-math}{fontspec}
\babelprovide[main, import, Alph=letters]{hebrew}
\babelprovide[import]{english}
\babelfont[hebrew]{rm}{David CLM}
\babelfont[hebrew]{sf}{David CLM}
%\babelfont[english]{tt}{Monaspace Xenon}
\usepackage[shortlabels]{enumitem}
\newlist{hebenum}{enumerate}{1}

% Language Shortcuts
\newcommand\en[1] {\begin{otherlanguage}{english}#1\end{otherlanguage}}
\newcommand\he[1] {\she#1\sen}
\newcommand\sen   {\begin{otherlanguage}{english}}
	\newcommand\she   {\end{otherlanguage}}
\newcommand\del   {$ \!\! $}

\newcommand\npage {\vfil {\hfil \textbf{\textit{המשך בעמוד הבא}}} \hfil \vfil \pagebreak}
\newcommand\ndoc  {\dotfill \\ \vfil {\begin{center}
			{\textbf{\textit{שחר פרץ, 2025}} \\
				\scriptsize \textit{קומפל ב־}\en{\LaTeX}\,\textit{ ונוצר באמצעות תוכנה חופשית בלבד}}
	\end{center}} \vfil	}

\newcommand{\rn}[1]{
	\textup{\uppercase\expandafter{\romannumeral#1}}
}

\makeatletter
\newcommand{\skipitems}[1]{
	\addtocounter{\@enumctr}{#1}
}
\makeatother

%! ~~~ Math shortcuts ~~~

% Letters shortcuts
\newcommand\N     {\mathbb{N}}
\newcommand\Z     {\mathbb{Z}}
\newcommand\R     {\mathbb{R}}
\newcommand\Q     {\mathbb{Q}}
\newcommand\C     {\mathbb{C}}
\newcommand\One   {\mathit{1}}

\newcommand\ml    {\ell}
\newcommand\mj    {\jmath}
\newcommand\mi    {\imath}

\newcommand\powerset {\mathcal{P}}
\newcommand\ps    {\mathcal{P}}
\newcommand\pc    {\mathcal{P}}
\newcommand\ac    {\mathcal{A}}
\newcommand\bc    {\mathcal{B}}
\newcommand\cc    {\mathcal{C}}
\newcommand\dc    {\mathcal{D}}
\newcommand\ec    {\mathcal{E}}
\newcommand\fc    {\mathcal{F}}
\newcommand\nc    {\mathcal{N}}
\newcommand\vc    {\mathcal{V}} % Vance
\newcommand\sca   {\mathcal{S}} % \sc is already definded
\newcommand\rca   {\mathcal{R}} % \rc is already definded
\newcommand\zc    {\mathcal{Z}}

\newcommand\prm   {\mathrm{p}}
\newcommand\arm   {\mathrm{a}} % x86
\newcommand\brm   {\mathrm{b}}
\newcommand\crm   {\mathrm{c}}
\newcommand\drm   {\mathrm{d}}
\newcommand\erm   {\mathrm{e}}
\newcommand\frm   {\mathrm{f}}
\newcommand\nrm   {\mathrm{n}}
\newcommand\vrm   {\mathrm{v}}
\newcommand\srm   {\mathrm{s}}
\newcommand\rrm   {\mathrm{r}}

\newcommand\Si    {\Sigma}

% Logic & sets shorcuts
\newcommand\siff  {\longleftrightarrow}
\newcommand\ssiff {\leftrightarrow}
\newcommand\so    {\longrightarrow}
\newcommand\sso   {\rightarrow}

\newcommand\epsi  {\epsilon}
\newcommand\vepsi {\varepsilon}
\newcommand\vphi  {\varphi}
\newcommand\Neven {\N_{\mathrm{even}}}
\newcommand\Nodd  {\N_{\mathrm{odd }}}
\newcommand\Zeven {\Z_{\mathrm{even}}}
\newcommand\Zodd  {\Z_{\mathrm{odd }}}
\newcommand\Np    {\N_+}

% Text Shortcuts
\newcommand\open  {\big(}
\newcommand\qopen {\quad\big(}
\newcommand\close {\big)}
\newcommand\also  {\mathrm{, }}
\newcommand\defis {\mathrm{ definitions}}
\newcommand\given {\mathrm{given }}
\newcommand\case  {\mathrm{if }}
\newcommand\syx   {\mathrm{ syntax}}
\newcommand\rle   {\mathrm{ rule}}
\newcommand\other {\mathrm{else}}
\newcommand\set   {\ell et \text{ }}
\newcommand\ans   {\mathscr{A}\!\mathit{nswer}}

% Set theory shortcuts
\newcommand\ra    {\rangle}
\newcommand\la    {\langle}

\newcommand\oto   {\leftarrow}

\newcommand\QED   {\quad\quad\mathscr{Q.E.D.}\;\;\blacksquare}
\newcommand\QEF   {\quad\quad\mathscr{Q.E.F.}}
\newcommand\eQED  {\mathscr{Q.E.D.}\;\;\blacksquare}
\newcommand\eQEF  {\mathscr{Q.E.F.}}
\newcommand\jQED  {\mathscr{Q.E.D.}}

\DeclareMathOperator\dom   {dom}
\DeclareMathOperator\Img   {Im}
\DeclareMathOperator\range {range}

\newcommand\trio  {\triangle}

\newcommand\rc    {\right\rceil}
\newcommand\lc    {\left\lceil}
\newcommand\rf    {\right\rfloor}
\newcommand\lf    {\left\lfloor}
\newcommand\ceil  [1] {\lc #1 \rc}
\newcommand\floor [1] {\lf #1 \rf}

\newcommand\lex   {<_{lex}}

\newcommand\az    {\aleph_0}
\newcommand\uaz   {^{\aleph_0}}
\newcommand\al    {\aleph}
\newcommand\ual   {^\aleph}
\newcommand\taz   {2^{\aleph_0}}
\newcommand\utaz  { ^{\left (2^{\aleph_0} \right )}}
\newcommand\tal   {2^{\aleph}}
\newcommand\utal  { ^{\left (2^{\aleph} \right )}}
\newcommand\ttaz  {2^{\left (2^{\aleph_0}\right )}}

\newcommand\n     {$n$־יה\ }

% Math A&B shortcuts
\newcommand\logn  {\log n}
\newcommand\logx  {\log x}
\newcommand\lnx   {\ln x}
\newcommand\cosx  {\cos x}
\newcommand\sinx  {\sin x}
\newcommand\sint  {\sin \theta}
\newcommand\tanx  {\tan x}
\newcommand\tant  {\tan \theta}
\newcommand\sex   {\sec x}
\newcommand\sect  {\sec^2}
\newcommand\cotx  {\cot x}
\newcommand\cscx  {\csc x}
\newcommand\sinhx {\sinh x}
\newcommand\coshx {\cosh x}
\newcommand\tanhx {\tanh x}

\newcommand\seq   {\overset{!}{=}}
\newcommand\slh   {\overset{LH}{=}}
\newcommand\sle   {\overset{!}{\le}}
\newcommand\sge   {\overset{!}{\ge}}
\newcommand\sll   {\overset{!}{<}}
\newcommand\sgg   {\overset{!}{>}}

\newcommand\h     {\hat}
\newcommand\ve    {\vec}
\newcommand\lv    {\overrightarrow}
\newcommand\ol    {\overline}

\newcommand\mlcm  {\mathrm{lcm}}

\DeclareMathOperator{\sech}   {sech}
\DeclareMathOperator{\csch}   {csch}
\DeclareMathOperator{\arcsec} {arcsec}
\DeclareMathOperator{\arccot} {arcCot}
\DeclareMathOperator{\arccsc} {arcCsc}
\DeclareMathOperator{\arccosh}{arccosh}
\DeclareMathOperator{\arcsinh}{arcsinh}
\DeclareMathOperator{\arctanh}{arctanh}
\DeclareMathOperator{\arcsech}{arcsech}
\DeclareMathOperator{\arccsch}{arccsch}
\DeclareMathOperator{\arccoth}{arccoth}
\DeclareMathOperator{\atant}  {atan2} 
\DeclareMathOperator{\Sp}     {span} 
\DeclareMathOperator{\sgn}    {sgn} 
\DeclareMathOperator{\row}    {Row} 
\DeclareMathOperator{\adj}    {adj} 
\DeclareMathOperator{\rk}     {rank} 
\DeclareMathOperator{\col}    {Col} 
\DeclareMathOperator{\tr}     {tr}

\newcommand\dx    {\,\mathrm{d}x}
\newcommand\dt    {\,\mathrm{d}t}
\newcommand\dtt   {\,\mathrm{d}\theta}
\newcommand\du    {\,\mathrm{d}u}
\newcommand\dv    {\,\mathrm{d}v}
\newcommand\df    {\mathrm{d}f}
\newcommand\dfdx  {\diff{f}{x}}
\newcommand\dit   {\limhz \frac{f(x + h) - f(x)}{h}}

\newcommand\nt[1] {\frac{#1}{#1}}

\newcommand\limz  {\lim_{x \to 0}}
\newcommand\limxz {\lim_{x \to x_0}}
\newcommand\limi  {\lim_{x \to \infty}}
\newcommand\limh  {\lim_{x \to 0}}
\newcommand\limni {\lim_{x \to - \infty}}
\newcommand\limpmi{\lim_{x \to \pm \infty}}

\newcommand\ta    {\theta}
\newcommand\ap    {\alpha}

\renewcommand\inf {\infty}
\newcommand  \ninf{-\inf}

% Combinatorics shortcuts
\newcommand\sumnk     {\sum_{k = 0}^{n}}
\newcommand\sumni     {\sum_{i = 0}^{n}}
\newcommand\sumnko    {\sum_{k = 1}^{n}}
\newcommand\sumnio    {\sum_{i = 1}^{n}}
\newcommand\sumai     {\sum_{i = 1}^{n} A_i}
\newcommand\nsum[2]   {\reflectbox{\displaystyle\sum_{\reflectbox{\scriptsize$#1$}}^{\reflectbox{\scriptsize$#2$}}}}

\newcommand\bink      {\binom{n}{k}}
\newcommand\setn      {\{a_i\}^{2n}_{i = 1}}
\newcommand\setc[1]   {\{a_i\}^{#1}_{i = 1}}

\newcommand\cupain    {\bigcup_{i = 1}^{n} A_i}
\newcommand\cupai[1]  {\bigcup_{i = 1}^{#1} A_i}
\newcommand\cupiiai   {\bigcup_{i \in I} A_i}
\newcommand\capain    {\bigcap_{i = 1}^{n} A_i}
\newcommand\capai[1]  {\bigcap_{i = 1}^{#1} A_i}
\newcommand\capiiai   {\bigcap_{i \in I} A_i}

\newcommand\xot       {x_{1, 2}}
\newcommand\ano       {a_{n - 1}}
\newcommand\ant       {a_{n - 2}}

% Linear Algebra
\DeclareMathOperator{\chr}     {char}
\DeclareMathOperator{\diag}    {diag}
\DeclareMathOperator{\Hom}     {Hom}
\DeclareMathOperator{\Sym}     {Sym}
\DeclareMathOperator{\Asym}    {ASym}

\newcommand\lra       {\leftrightarrow}
\newcommand\chrf      {\chr(\F)}
\newcommand\F         {\mathbb{F}}
\newcommand\co        {\colon}
\newcommand\tmat[2]   {\cl{\begin{matrix}
			#1
		\end{matrix}\, \middle\vert\, \begin{matrix}
			#2
\end{matrix}}}

\makeatletter
\newcommand\rrr[1]    {\xxrightarrow{1}{#1}}
\newcommand\rrt[2]    {\xxrightarrow{1}[#2]{#1}}
\newcommand\mat[2]    {M_{#1\times#2}}
\newcommand\gmat      {\mat{m}{n}(\F)}
\newcommand\tomat     {\, \dequad \longrightarrow}
\newcommand\pms[1]    {\begin{pmatrix}
		#1
\end{pmatrix}}

\newcommand\norm[1]   {\left \vert \left \vert #1 \right \vert \right \vert}
\newcommand\snorm     {\left \vert \left \vert \cdot \right \vert \right \vert}
\newcommand\smut      {\left \la \cdot \mid \cdot \right \ra}
\newcommand\mut[2]    {\left \la #1 \,\middle\vert\, #2 \right \ra}

% someone's code from the internet: https://tex.stackexchange.com/questions/27545/custom-length-arrows-text-over-and-under
\makeatletter
\newlength\min@xx
\newcommand*\xxrightarrow[1]{\begingroup
	\settowidth\min@xx{$\m@th\scriptstyle#1$}
	\@xxrightarrow}
\newcommand*\@xxrightarrow[2][]{
	\sbox8{$\m@th\scriptstyle#1$}  % subscript
	\ifdim\wd8>\min@xx \min@xx=\wd8 \fi
	\sbox8{$\m@th\scriptstyle#2$} % superscript
	\ifdim\wd8>\min@xx \min@xx=\wd8 \fi
	\xrightarrow[{\mathmakebox[\min@xx]{\scriptstyle#1}}]
	{\mathmakebox[\min@xx]{\scriptstyle#2}}
	\endgroup}
\makeatother


% Greek Letters
\newcommand\ag        {\alpha}
\newcommand\bg        {\beta}
\newcommand\cg        {\gamma}
\newcommand\dg        {\delta}
\newcommand\eg        {\epsi}
\newcommand\zg        {\zeta}
\newcommand\hg        {\eta}
\newcommand\tg        {\theta}
\newcommand\ig        {\iota}
\newcommand\kg        {\keppa}
\renewcommand\lg      {\lambda}
\newcommand\og        {\omicron}
\newcommand\rg        {\rho}
\newcommand\sg        {\sigma}
\newcommand\yg        {\usilon}
\newcommand\wg        {\omega}

\newcommand\Ag        {\Alpha}
\newcommand\Bg        {\Beta}
\newcommand\Cg        {\Gamma}
\newcommand\Dg        {\Delta}
\newcommand\Eg        {\Epsi}
\newcommand\Zg        {\Zeta}
\newcommand\Hg        {\Eta}
\newcommand\Tg        {\Theta}
\newcommand\Ig        {\Iota}
\newcommand\Kg        {\Keppa}
\newcommand\Lg        {\Lambda}
\newcommand\Og        {\Omicron}
\newcommand\Rg        {\Rho}
\newcommand\Sg        {\Sigma}
\newcommand\Yg        {\Usilon}
\newcommand\Wg        {\Omega}

% Other shortcuts
\newcommand\tl    {\tilde}
\newcommand\op    {^{-1}}

\newcommand\sof[1]    {\left | #1 \right |}
\newcommand\cl [1]    {\left ( #1 \right )}
\newcommand\csb[1]    {\left [ #1 \right ]}
\newcommand\ccb[1]    {\left \{ #1 \right \}}

\newcommand\bs        {\blacksquare}
\newcommand\dequad    {\!\!\!\!\!\!}
\newcommand\dequadd   {\dequad\duquad}

\renewcommand\phi     {\varphi}

\newtheorem{Theorem}{משפט}
\theoremstyle{definition}
\newtheorem{definition}{הגדרה}
\newtheorem{Lemma}{למה}
\newtheorem{Remark}{הערה}
\newtheorem{Notion}{סימון}


\newcommand\theo  [1] {\begin{Theorem}#1\end{Theorem}}
\newcommand\defi  [1] {\begin{definition}#1\end{definition}}
\newcommand\rmark [1] {\begin{Remark}#1\end{Remark}}
\newcommand\lem   [1] {\begin{Lemma}#1\end{Lemma}}
\newcommand\noti  [1] {\begin{Notion}#1\end{Notion}}

% DS
\newcommand\limsi     {\limsup_{n \to \inf}}
\newcommand\limfi     {\liminf_{n \to \inf}}

\DeclareMathOperator\amort   {amort}
\DeclareMathOperator\worst   {worst}
\DeclareMathOperator\type    {type}
\DeclareMathOperator\cost    {cost}
\DeclareMathOperator\tim     {time}

\newcommand\dsList{
	\sFunc{List}
	\sFunc{Retrieve}
	\SetKwFunction{RetrieveFirst}{Retrieve-First}
	\SetKwFunction{RetrieveLast}{Retrieve-Last}
	\sFunc{Delete}
	\SetKwFunction{DeleteFirst}{Delete-First}
	\SetKwFunction{DeleteLast}{Delete-Last}
	\sFunc{Insert}
	\SetKwFunction{InsertFirst}{Insert-First}
	\SetKwFunction{InsertLast}{Insert-Last}
	\sFunc{Shift}
	\sFunc{Length}
	\sFunc{Concat}
	\sFunc{Plant}
	\sFunc{Split}
}
\newcommand\dsQueue{
	\sFunc{Queue}
	\sFunc{Enqueue}
	\sFunc{Head}
	\sFunc{Dequeue}
}
\newcommand\dsStack{
	\sFunc{Stack}
	\sFunc{Push}
	\sFunc{Top}
	\sFunc{Pop}
}
\newcommand\dsVector{
	\sFunc{Vector}
	\sFunc{Get}
	\sFunc{Set}
}
\newcommand\dsGraph{
	\sFunc{Graph}
	\sFunc{Edge}
	\SetKwFunction{AddEdge}{Add-Edge}
	\SetKwFunction{RemoveEdge}{Remove-Edge}
	\sFunc{InDeg} \sFunc{OutDeg}
}
\newcommand\importDs{
	\dsList
	\dsQueue
	\dsStack
	\dsVector
	\dsGraph
	\SetKwProg{Fn}{function}{ is}{end}
	\SetKwData{error}{\color{codered}error}
	\SetKwInOut{Input}{input}
	\SetKwInOut{Output}{output}
	\SetKwRepeat{Do}{do}{while}
	\SetKwData{Null}{\color{codegreen}null}
	\SetKwData{True}{\color{codeblue}true}
	\SetKwData{False}{\color{codeblue}false}
}


% Algorithems
\newcommand\sFunc [1] {\SetKwFunction{#1}{#1}}
\newcommand\sData [1] {\SetKwData{#1}{#1}}
\newcommand\sIO   [1] {\SetKwInOut{#1}{#1}}
\newcommand\ttt   [1] {\sen \texttt{#1} \she\,}
\newcommand\io    [2] {\Input{#1}\Output{#2}\BlankLine}

%! ~~~ Document ~~~

\author{\en{Shahar Perets}}
\title{\textit{לינארית 2א 21}}
\begin{document}
	\maketitle
	\theo{אם $A \in M_n(\F)$ נורמלית, אזי קיים פולינום $f(x) \in \R[x]$ כך ש־$A^* = f(A)$}
	
	דיברנו על המשפט הזה בשבוע שעבר, אבל ההוכחה הייתה קצת עקומה אז נחזור עליה. 
	\begin{proof}
		עבור $A$ נורמלית מהמשפט הספקטרלי קיים בסיס אורתונורמלי מלכסן ולכן קיימת $P$ הפיכה כך ש־$P\op APש = \diag(\lg_1 \dots \lg_n)$. לכן $P\op A^* P = \diag(\bar \lg_1 \dots \bar \lg_n)$. נשתמש במשפט לפיו יש פולינום $f \in \R[x]$ כך ש־$f(x_i) = \bar x_i$ ובפרט בעבור $x_i = \lg_i$ קיים פולינום עבורו $f(\lg_i) = \bar \lg_i$. אזי 
		\[ f(\diag(\lg_1 \dots \lg_n)) = \diag(f(\lg_1) \dots f(\lg_n)) = \diag(\bar \lg_1 \dots \bar \lg_n) \]
		\[ f(P\op A P) = P\op f(A) P = P \op A^* P \implies f(A) = A^* \]
		עוד נבחין ש־$\deg f = n - 1$. 
	\end{proof}
	
	ננסה להבין מי הן $A \in M_2(\R)$ שהן נורמליות. מעל $\C$ הן פשוט לכסינות. נבחין ש־: 
	\begin{align*}
		A = \pms{a & b \\ c & d}, \ A^* = f(A) \in \R_{\le1}[x] = \ag A + BI, \ A = \ag A^T + \bg I = \ag(\ag A + \bg I) + \bg I \\ \implies \pms{a & b \\ c & d} = \pms{\ag^2 a + \bg(\ag + 1) & \ag^2 b \\ \ag^2 c & \ag^2 d + \bg (\ag + 1)} \\
		\begin{cases}
			(b \land c \neq 0 ) \quad \ag = 1 \implies  A = A + 2 \bg I \implies \bg = 0, \ A = A^T \cancel{+ \bg I} \\
			(b \land c \neq 0) \quad \ag = -1 \implies A^T = - A  +\bg I  \implies A + A^T = \pms{\bg & 0 \\ 0 & \bg} \implies b = -c, \ A = \pms{a & b \\ -b & a} \\
			(b \lor c = 0) \implies A = \pms{a & 0 \\ 0 & d}
		\end{cases}
	\end{align*}
	המקרה השני – זה פשוט סיבובים, אבל בניפוח (כי הדטרמיננטה היא $a^2 - b^2$). 
	
	בכל מקרה, מסקנה מהמשפט הקודם. 
	\theo{אם $T \co V \to V$ נורמלית, אז $\exists f \in \R[x] \co f(T) = T^*$. }
	\begin{proof}
		נבחר בסיס א''נ $A^* = [T^*]_B, \impliedby A = [T]_B$. כבר הוכחנו שאם $T$ נורמלית אז $A$ נורמלית ולכן מהמשפט הקודם קיים $f$ מתאים כך ש־$[T^*]_B = A^* = f(A) = f([T]_B) = [f(T)]_B$. סה''כ $[T^*]_B = [f(T)]_B$ ומחח''ע העברת בסיס $T^* = f(T)$ כדרוש. 
	\end{proof}
	
	אם $T \co V \to V$ ט''ל, $U, W \subseteq V$ תמ''וים $T$־איוונריאנטי כך ש־$U \oplus W = B$. אם $\bc$ בסיס של $V$, כאשר קישא של הבסיס הוא הבסיס של $U$ אז: 
	\[ [T]_\bc = \pms{[T|_U]_\bc &  \\  & [T|_W]_\bc} \]
	בפרט בעבור ניצבים $U \subseteq V \implies V = U \oplus U^{\perp}$. ניעזר בכך כדי להוכיח את המשפט הבא: 
	\theo{אם $U \subseteq V$ תמ''ו אינוו' ביחס ל־$T$ אז $U^{\perp}$ הוא $T^*$־אינוו'. }\begin{proof}
		יהי $w \in U^{\perp}$. רוצים להראות $T^*w \in U^{\perp}$. יהי $u \in U$ אז: 
		\[ \mut{T^* w}{u} = \mut{w}{Tu} = \mut{w}{u'}, \ u' \in U \implies \mut{w}{u'} = 0 \quad \top \]
	\end{proof}
	\theo{בעבור $T \co V \to V$ נורמלית, אם היא $U$־אינו' אז גם $T^*$ הוא $U$־אינו'}
	\begin{proof}
		נבחין ש־$T^* = f(T)$ כלשהו, וכן $U$ הוא $T$־איוו' ולכן $U$ הוא $f(T)$־איוו' וכאן די גמרנו את ההוכחה. 
	\end{proof}
	מסימטריות $U^{\perp}$ הוא $T^*$, מהמשפט גם $(T^*)^*$ איונ' ולכן $T$־אינו'. 
	
	\theo{יהי $V$ מעל $\R$ מ''ו וכן $T \co V \to V$ ט''ל. אז קיים $U \subseteq V$ שהוא $T$־איונ' וממדו לכל היותר $T$. }
	\textit{הערה: }מעל $\C$ ``זה מטופש'' כי הפולינום מתפרק (ואז המרחב העצמי יקיים את זה). 
	\begin{proof}
		נפרק ל־$m_T(x)$ מינימלי ו־$g(x)$ גורם אי־פריק כך ש־$m_T(x) = g(x)h(x)$. לכל $g$ אי פריק ב־$\R$ הוא לינארי  הוא ממעלה $2$, מהמשפט היסודי של האגלברה ומהעובדה ש־$m_T(x) = 0 \implies m_T(\bar x) = 0$. 
		\begin{itemize}
			\item אם $g$ לינארי אז יש ע''ע ממשי של $T$ מה שנותא $U$ (שנפרש ע''י הו''ע) ממד $1$. 
			\item אם $\deg g = 2$ כמובן שניתן להניח $g$ מתוקן. ז''א $g(x) = x^2 + ax + b$ אז $g(T)$ אינו הפיך (מלמת החלוקה לפולינום מינימלי) כלומר $\exists 0 \neq v \in \ker g(T)$. 
			לכן: 
			\[ (T^2 + aT + bI)v = 0 \implies T^2v = -a T v - bv \]
			ולכן $U = \Sp(v, Tv)$ תמ''ו עם ממד לכל היותר 2 וגם נשמר תחת $T$. 
		\end{itemize}
	\end{proof}
	
	\textit{הערה: }בעבור נורמלית הטענה נכונה ללא תלות במשפט היסודי של האלגברה. 
	
	לכן, בעבור נורמליות, מהמשפט הספקטרלי ומטענות קודמות, עבור $T \co V \to V$ ממשית קיים בסיס א''נ $\bc$ של $V$ שבעבורו המטריצה המייצגת של $T$ היא מטריצת בלוקים $2 \times 2$ מצורה של $\pms{a & b \\ -b & a}$: 
	\[ [T]_\bc = \diag\cl{\pms{a_1 & b_1 \\ -b_1 & a_1} \cdots \pms{a_k & b_k \\ -a_k & b_k}, \ \lg_1 \cdots \lg_m} \]
	כאשר כמובן $2k + m = n$. 
	
	\section{\en{Unitary matrixes}}
	\defi{יהי $V$ ממ''פ. אז $T \co V \to V$ תקרא \textit{אוניטרית} (אם $\F = \C$) או \textit{אורתוגונלית} (אם $\F = \R$) אם $T^*T = I$ או במילים אחרות $T^* = T\op$. }
	ברור שט''ל כזו היא נורמלית. 
	\textbf{דוגמה. }עבור $T_\tg$ הסיבוב ב־$\tg$ מעלות, במישור $\R^2$, אז $(T_\tg)^* = T_{-\tg} = T\op_{\tg}$.  
	\textbf{דוגמה. }עבור $T$ שיקוף מתקים $T^2 = I$ וכן $T^* = T$ וסה''כ $T^* = T = T\op$. 
	
	\theo{התנאים הבאים על $T \co V \to V$ שקולים: 
	\begin{itemize}
		\item \hfil $T^* = T\op$
		\item \hfil $\forall v, u \co \mut{Tv}{Tu} = \mut{v}{u}$
		\item $T$ מעבירה כל בסיס א''נ של $V$ לבסיס א''נ של $V$
		\item $T$ מעבירה בסיס א''נ אחד של $V$ לבסיס א''נ של $V$ (כלומר, מספיק להראות שקיים בסיס יחידה שעובר לבסיס אחר). 
		\item \hfil $\forall v \in V \co \norm{Tv} = \norm{v}$
	\end{itemize}}
	כלומר: היא משמרת זווית (העתקה פנימית) וגודל. במילים אחרות, היא משמרת העתקה פנימית. 
	\begin{proof}נפרק לרצף גרירות
		\begin{enumerate}
			\item[$1 \to 2$] \hfil $T^* = T\op \implies \mut{Tv}{Tu} = \mut{v}{T^*Tu} = \mut{v}{u}$
			\item[$2 \to 3$] נאמר ש־$(v_1 \dots v_n)$ א''נ. צ.ל. $(Tv_i)_{i = 1}^{n}$ א''נ. לשם כך נצטרך להוכיח את שני התנאים – החלק של האורתו והחלק של הנורמלי. בשביל שניהם מספיק להוכיח ש־: 
				$\mut{Tv_i}{Tv_j} = \mut{v_i}{v_j}  = \dg_{ij}$
			\item[$3 \to 4$]טרוויאלי
			\item[$4 \to 5$]יהי $(v_1 \dots v_n)$ בסיס א''נ כך ש־$(Tv_1 \dots Tv_n)$ א''נ. אז: 
			\begin{align*}
				v = \sumni \ag_i v_i \implies &\norm{v}^2 = \mut{\sumni \ag_i v_i}{\sumni \ag_i v_i} = \sumni |\ag_i|^2 \\
				&\norm{Tv}^2 = \mut{T\cl{\sumni \ag_i v_i}}{T\cl{\sumni \ag_i v_i}} = \mut{\sumni \ag_i T(v_i)}{\sumni \ag_i T(v_i)} = \sum |\ag_i|^2
			\end{align*}
			\item[$5 \to 1$]מניחים $\forall v \in V \co \norm{Tv} = \norm{v}$. ידועות השקילויות הבאות: 
			\[ T^* = T\op \iff T^*T = I \iff T^*T - I = 0 \]
			בעבר ראינו את הטענה הבאה: נניח ש־$S$ צמודה לעצמה וכן ש־$\forall v \co \mut{Sv}{v} = 0$, אז $S = 0$. במקרה הזה: 
			\[ S := T^*T - I \implies S^* = (T^*T - I)^* = (T^*T)^* - I^* = T^*(T^*)^* - I =S \implies S^* = S \]
			והיא אכן צמודה לעצמה. עוד נבחין ש־: 
			\[ \mut{Sv}{v} = \mut{(T^* T - I)v}{v} =\mut{T^*Tv}{v} - \mut{v}{v} = \mut{Tv}{Tv} - \mut{v}{v} = \norm{Tv}^2 - \norm{v}^2 = 0 \]
			השוויון האחרון נכון מההנחה היחידה שלנו ש־$\norm{Tv} = \norm{v}$. סה''כ $TT^* - I = 0$. סה''כ הוכחנו $TT^* - I = 0$ שזה שקול ל־$T^* = T\op$ מהשקילויות לעיל כדרוש. 
		\end{enumerate}
	\end{proof}
	
	\theo{תהי $T \co V \to V$ אוני'אורתו', ו־$\lg$ ע''ע של $T$. אז $|\lg| = 1$} \begin{proof}
		יהי $v$ ו''ע של הע''ע $\lg$. אז: 
		\[ |\lg|^2 \mut{v}{v} = \lg\bar\lg \mut{v}{v} = \mut{Tv}{Tv} = \mut{v}{v} \]
	\end{proof}
	\defi{תהי $A \in M_n(\F)$. אז $A$ תיקרא \textit{אוניטרית}/\textit{אורתוגונלית} אם $A^* = A\op$. }
	\theo{אוניטרית: $A\ol{A^T} = I$}
	\theo{אורתוגונלית: $AA^T = I$}
	\textit{הערה: }אוניטרית בה מלשון unit – היא שומרת על הגודל, על וקטורי היחידה (ה־unit vectors). 
	\theo{יהי $\bc$ בסיס א''נ של $V$ ו־$T \co V \to V$ אז $T$ אוניטרית/אורתוגונלית אמ''מ $A = [T]_B$ אוניטרית/אורתוגונלית. }
	\begin{proof}
		\[ AA^* = [T]_\bc[T^*]_\bc = [TT^*]_\bc, I = AA^* \iff [TT^*]_\bc = I \iff TT^* = I \]
	\end{proof}
	
	הנושאים האחרונים למבחן – צורה קאנונית של העתקה נורמלית מעל הממשיים, וכן העתקות אורתוגונליות ואוניטריות. ללא צורה קאנונית של העתקה אורתוגונלית. לכן ההרצאות בשיעורים הבאים לא נכנסות לחומר. תודה איראן. 
	
	\textit{הערה. }\textit{איזומטריה} היא העתקה שמשמרת גדלים, ואיזומטריה אורתוגונלית היא פשוט אוניטרית. משום מה זה שם שמדברים עליו בע''פ אבל לא הגדירו מסודר. 
	\ndoc
\end{document}