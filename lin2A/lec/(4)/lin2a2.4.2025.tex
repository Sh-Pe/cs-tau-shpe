%! ~~~ Packages Setup ~~~ 
\documentclass[]{article}
\usepackage{lipsum}
\usepackage{rotating}


% Math packages
\usepackage[usenames]{color}
\usepackage{forest}
\usepackage{ifxetex,ifluatex,amssymb,amsmath,mathrsfs,amsthm,witharrows,mathtools,mathdots}
\usepackage{amsmath}
\WithArrowsOptions{displaystyle}
\renewcommand{\qedsymbol}{$\blacksquare$} % end proofs with \blacksquare. Overwrites the defualts. 
\usepackage{cancel,bm}
\usepackage[thinc]{esdiff}


% tikz
\usepackage{tikz}
\usetikzlibrary{graphs}
\newcommand\sqw{1}
\newcommand\squ[4][1]{\fill[#4] (#2*\sqw,#3*\sqw) rectangle +(#1*\sqw,#1*\sqw);}


% code 
\usepackage{listings}
\usepackage{xcolor}

\definecolor{codegreen}{rgb}{0,0.35,0}
\definecolor{codegray}{rgb}{0.5,0.5,0.5}
\definecolor{codenumber}{rgb}{0.1,0.3,0.5}
\definecolor{codeblue}{rgb}{0,0,0.5}
\definecolor{codered}{rgb}{0.5,0.03,0.02}
\definecolor{codegray}{rgb}{0.96,0.96,0.96}

\lstdefinestyle{pythonstylesheet}{
	language=Java,
	emphstyle=\color{deepred},
	backgroundcolor=\color{codegray},
	keywordstyle=\color{deepblue}\bfseries\itshape,
	numberstyle=\scriptsize\color{codenumber},
	basicstyle=\ttfamily\footnotesize,
	commentstyle=\color{codegreen}\itshape,
	breakatwhitespace=false, 
	breaklines=true, 
	captionpos=b, 
	keepspaces=true, 
	numbers=left, 
	numbersep=5pt, 
	showspaces=false,                
	showstringspaces=false,
	showtabs=false, 
	tabsize=4, 
	morekeywords={as,assert,nonlocal,with,yield,self,True,False,None,AssertionError,ValueError,in,else},              % Add keywords here
	keywordstyle=\color{codeblue},
	emph={var, List, Iterable, Iterator},          % Custom highlighting
	emphstyle=\color{codered},
	stringstyle=\color{codegreen},
	showstringspaces=false,
	abovecaptionskip=0pt,belowcaptionskip =0pt,
	framextopmargin=-\topsep, 
}
\newcommand\pythonstyle{\lstset{pythonstylesheet}}
\newcommand\pyl[1]     {{\lstinline!#1!}}
\lstset{style=pythonstylesheet}

\usepackage[style=1,skipbelow=\topskip,skipabove=\topskip,framemethod=TikZ]{mdframed}
\definecolor{bggray}{rgb}{0.85, 0.85, 0.85}
\mdfsetup{leftmargin=0pt,rightmargin=0pt,innerleftmargin=15pt,backgroundcolor=codegray,middlelinewidth=0.5pt,skipabove=5pt,skipbelow=0pt,middlelinecolor=black,roundcorner=5}
\BeforeBeginEnvironment{lstlisting}{\begin{mdframed}\vspace{-0.4em}}
	\AfterEndEnvironment{lstlisting}{\vspace{-0.8em}\end{mdframed}}


% Deisgn
\usepackage[labelfont=bf]{caption}
\usepackage[margin=0.6in]{geometry}
\usepackage{multicol}
\usepackage[skip=4pt, indent=0pt]{parskip}
\usepackage[normalem]{ulem}
\forestset{default}
\renewcommand\labelitemi{$\bullet$}
\usepackage{graphicx}
\graphicspath{ {./} }


% Hebrew initialzing
\usepackage[bidi=basic]{babel}
\PassOptionsToPackage{no-math}{fontspec}
\babelprovide[main, import, Alph=letters]{hebrew}
\babelprovide[import]{english}
\babelfont[hebrew]{rm}{David CLM}
\babelfont[hebrew]{sf}{David CLM}
\babelfont[english]{tt}{Monaspace Xenon}
\usepackage[shortlabels]{enumitem}
\newlist{hebenum}{enumerate}{1}

% Language Shortcuts
\newcommand\en[1] {\begin{otherlanguage}{english}#1\end{otherlanguage}}
\newcommand\sen   {\begin{otherlanguage}{english}}
	\newcommand\she   {\end{otherlanguage}}
\newcommand\del   {$ \!\! $}

\newcommand\npage {\vfil {\hfil \textbf{\textit{המשך בעמוד הבא}}} \hfil \vfil \pagebreak}
\newcommand\ndoc  {\dotfill \\ \vfil {\begin{center}
			{\textbf{\textit{שחר פרץ, 2025}} \\
				\scriptsize \textit{קומפל ב־}\en{\LaTeX}\,\textit{ ונוצר באמצעות תוכנה חופשית בלבד}}
	\end{center}} \vfil	}

\newcommand{\rn}[1]{
	\textup{\uppercase\expandafter{\romannumeral#1}}
}

\makeatletter
\newcommand{\skipitems}[1]{
	\addtocounter{\@enumctr}{#1}
}
\makeatother

%! ~~~ Math shortcuts ~~~

% Letters shortcuts
\newcommand\N     {\mathbb{N}}
\newcommand\Z     {\mathbb{Z}}
\newcommand\R     {\mathbb{R}}
\newcommand\Q     {\mathbb{Q}}
\newcommand\C     {\mathbb{C}}
\newcommand\One   {\mathit{1}}

\newcommand\ml    {\ell}
\newcommand\mj    {\jmath}
\newcommand\mi    {\imath}

\newcommand\powerset {\mathcal{P}}
\newcommand\ps    {\mathcal{P}}
\newcommand\pc    {\mathcal{P}}
\newcommand\ac    {\mathcal{A}}
\newcommand\bc    {\mathcal{B}}
\newcommand\cc    {\mathcal{C}}
\newcommand\dc    {\mathcal{D}}
\newcommand\ec    {\mathcal{E}}
\newcommand\fc    {\mathcal{F}}
\newcommand\nc    {\mathcal{N}}
\newcommand\vc    {\mathcal{V}} % Vance
\newcommand\sca   {\mathcal{S}} % \sc is already definded
\newcommand\rca   {\mathcal{R}} % \rc is already definded

\newcommand\prm   {\mathrm{p}}
\newcommand\arm   {\mathrm{a}} % x86
\newcommand\brm   {\mathrm{b}}
\newcommand\crm   {\mathrm{c}}
\newcommand\drm   {\mathrm{d}}
\newcommand\erm   {\mathrm{e}}
\newcommand\frm   {\mathrm{f}}
\newcommand\nrm   {\mathrm{n}}
\newcommand\vrm   {\mathrm{v}}
\newcommand\srm   {\mathrm{s}}
\newcommand\rrm   {\mathrm{r}}

\newcommand\Si    {\Sigma}

% Logic & sets shorcuts
\newcommand\siff  {\longleftrightarrow}
\newcommand\ssiff {\leftrightarrow}
\newcommand\so    {\longrightarrow}
\newcommand\sso   {\rightarrow}

\newcommand\epsi  {\epsilon}
\newcommand\vepsi {\varepsilon}
\newcommand\vphi  {\varphi}
\newcommand\Neven {\N_{\mathrm{even}}}
\newcommand\Nodd  {\N_{\mathrm{odd }}}
\newcommand\Zeven {\Z_{\mathrm{even}}}
\newcommand\Zodd  {\Z_{\mathrm{odd }}}
\newcommand\Np    {\N_+}

% Text Shortcuts
\newcommand\open  {\big(}
\newcommand\qopen {\quad\big(}
\newcommand\close {\big)}
\newcommand\also  {\text{, }}
\newcommand\defis {\text{ definitions}}
\newcommand\given {\text{given }}
\newcommand\case  {\text{if }}
\newcommand\syx   {\text{ syntax}}
\newcommand\rle   {\text{ rule}}
\newcommand\other {\text{else}}
\newcommand\set   {\ell et \text{ }}
\newcommand\ans   {\mathscr{A}\!\mathit{nswer}}

% Set theory shortcuts
\newcommand\ra    {\rangle}
\newcommand\la    {\langle}

\newcommand\oto   {\leftarrow}

\newcommand\QED   {\quad\quad\mathscr{Q.E.D.}\;\;\blacksquare}
\newcommand\QEF   {\quad\quad\mathscr{Q.E.F.}}
\newcommand\eQED  {\mathscr{Q.E.D.}\;\;\blacksquare}
\newcommand\eQEF  {\mathscr{Q.E.F.}}
\newcommand\jQED  {\mathscr{Q.E.D.}}

\DeclareMathOperator\dom   {dom}
\DeclareMathOperator\Img   {Im}
\DeclareMathOperator\range {range}

\newcommand\trio  {\triangle}

\newcommand\rc    {\right\rceil}
\newcommand\lc    {\left\lceil}
\newcommand\rf    {\right\rfloor}
\newcommand\lf    {\left\lfloor}

\newcommand\lex   {<_{lex}}

\newcommand\az    {\aleph_0}
\newcommand\uaz   {^{\aleph_0}}
\newcommand\al    {\aleph}
\newcommand\ual   {^\aleph}
\newcommand\taz   {2^{\aleph_0}}
\newcommand\utaz  { ^{\left (2^{\aleph_0} \right )}}
\newcommand\tal   {2^{\aleph}}
\newcommand\utal  { ^{\left (2^{\aleph} \right )}}
\newcommand\ttaz  {2^{\left (2^{\aleph_0}\right )}}

\newcommand\n     {$n$־יה\ }

% Math A&B shortcuts
\newcommand\logn  {\log n}
\newcommand\logx  {\log x}
\newcommand\lnx   {\ln x}
\newcommand\cosx  {\cos x}
\newcommand\sinx  {\sin x}
\newcommand\sint  {\sin \theta}
\newcommand\tanx  {\tan x}
\newcommand\tant  {\tan \theta}
\newcommand\sex   {\sec x}
\newcommand\sect  {\sec^2}
\newcommand\cotx  {\cot x}
\newcommand\cscx  {\csc x}
\newcommand\sinhx {\sinh x}
\newcommand\coshx {\cosh x}
\newcommand\tanhx {\tanh x}

\newcommand\seq   {\overset{!}{=}}
\newcommand\slh   {\overset{LH}{=}}
\newcommand\sle   {\overset{!}{\le}}
\newcommand\sge   {\overset{!}{\ge}}
\newcommand\sll   {\overset{!}{<}}
\newcommand\sgg   {\overset{!}{>}}

\newcommand\h     {\hat}
\newcommand\ve    {\vec}
\newcommand\lv    {\overrightarrow}
\newcommand\ol    {\overline}

\newcommand\mlcm  {\mathrm{lcm}}

\DeclareMathOperator{\sech}   {sech}
\DeclareMathOperator{\csch}   {csch}
\DeclareMathOperator{\arcsec} {arcsec}
\DeclareMathOperator{\arccot} {arcCot}
\DeclareMathOperator{\arccsc} {arcCsc}
\DeclareMathOperator{\arccosh}{arccosh}
\DeclareMathOperator{\arcsinh}{arcsinh}
\DeclareMathOperator{\arctanh}{arctanh}
\DeclareMathOperator{\arcsech}{arcsech}
\DeclareMathOperator{\arccsch}{arccsch}
\DeclareMathOperator{\arccoth}{arccoth}
\DeclareMathOperator{\atant}  {atan2} 
\DeclareMathOperator{\Sp}     {span} 
\DeclareMathOperator{\sgn}    {sgn} 
\DeclareMathOperator{\row}    {Row} 
\DeclareMathOperator{\adj}    {adj} 
\DeclareMathOperator{\rk}     {rank} 
\DeclareMathOperator{\col}    {Col} 
\DeclareMathOperator{\tr}     {tr}

\newcommand\dx    {\,\mathrm{d}x}
\newcommand\dt    {\,\mathrm{d}t}
\newcommand\dtt   {\,\mathrm{d}\theta}
\newcommand\du    {\,\mathrm{d}u}
\newcommand\dv    {\,\mathrm{d}v}
\newcommand\df    {\mathrm{d}f}
\newcommand\dfdx  {\diff{f}{x}}
\newcommand\dit   {\limhz \frac{f(x + h) - f(x)}{h}}

\newcommand\nt[1] {\frac{#1}{#1}}

\newcommand\limz  {\lim_{x \to 0}}
\newcommand\limxz {\lim_{x \to x_0}}
\newcommand\limi  {\lim_{x \to \infty}}
\newcommand\limh  {\lim_{x \to 0}}
\newcommand\limni {\lim_{x \to - \infty}}
\newcommand\limpmi{\lim_{x \to \pm \infty}}

\newcommand\ta    {\theta}
\newcommand\ap    {\alpha}

\renewcommand\inf {\infty}
\newcommand  \ninf{-\inf}

% Combinatorics shortcuts
\newcommand\sumnk     {\sum_{k = 0}^{n}}
\newcommand\sumni     {\sum_{i = 0}^{n}}
\newcommand\sumnko    {\sum_{k = 1}^{n}}
\newcommand\sumnio    {\sum_{i = 1}^{n}}
\newcommand\sumai     {\sum_{i = 1}^{n} A_i}
\newcommand\nsum[2]   {\reflectbox{\displaystyle\sum_{\reflectbox{\scriptsize$#1$}}^{\reflectbox{\scriptsize$#2$}}}}

\newcommand\bink      {\binom{n}{k}}
\newcommand\setn      {\{a_i\}^{2n}_{i = 1}}
\newcommand\setc[1]   {\{a_i\}^{#1}_{i = 1}}

\newcommand\cupain    {\bigcup_{i = 1}^{n} A_i}
\newcommand\cupai[1]  {\bigcup_{i = 1}^{#1} A_i}
\newcommand\cupiiai   {\bigcup_{i \in I} A_i}
\newcommand\capain    {\bigcap_{i = 1}^{n} A_i}
\newcommand\capai[1]  {\bigcap_{i = 1}^{#1} A_i}
\newcommand\capiiai   {\bigcap_{i \in I} A_i}

\newcommand\xot       {x_{1, 2}}
\newcommand\ano       {a_{n - 1}}
\newcommand\ant       {a_{n - 2}}

% Linear Algebra
\DeclareMathOperator{\chr}     {char}
\DeclareMathOperator{\diag}    {diag}
\DeclareMathOperator{\Hom}     {Hom}

\newcommand\lra       {\leftrightarrow}
\newcommand\chrf      {\chr(\F)}
\newcommand\F         {\mathbb{F}}
\newcommand\co        {\colon}
\newcommand\tmat[2]   {\cl{\begin{matrix}
			#1
		\end{matrix}\, \middle\vert\, \begin{matrix}
			#2
\end{matrix}}}

\makeatletter
\newcommand\rrr[1]    {\xxrightarrow{1}{#1}}
\newcommand\rrt[2]    {\xxrightarrow{1}[#2]{#1}}
\newcommand\mat[2]    {M_{#1\times#2}}
\newcommand\gmat      {\mat{m}{n}(\F)}
\newcommand\tomat     {\, \dequad \longrightarrow}
\newcommand\pms[1]    {\begin{pmatrix}
		#1
\end{pmatrix}}

% someone's code from the internet: https://tex.stackexchange.com/questions/27545/custom-length-arrows-text-over-and-under
\makeatletter
\newlength\min@xx
\newcommand*\xxrightarrow[1]{\begingroup
	\settowidth\min@xx{$\m@th\scriptstyle#1$}
	\@xxrightarrow}
\newcommand*\@xxrightarrow[2][]{
	\sbox8{$\m@th\scriptstyle#1$}  % subscript
	\ifdim\wd8>\min@xx \min@xx=\wd8 \fi
	\sbox8{$\m@th\scriptstyle#2$} % superscript
	\ifdim\wd8>\min@xx \min@xx=\wd8 \fi
	\xrightarrow[{\mathmakebox[\min@xx]{\scriptstyle#1}}]
	{\mathmakebox[\min@xx]{\scriptstyle#2}}
	\endgroup}
\makeatother


% Greek Letters
\newcommand\ag        {\alpha}
\newcommand\bg        {\beta}
\newcommand\cg        {\gamma}
\newcommand\dg        {\delta}
\newcommand\eg        {\epsi}
\newcommand\zg        {\zeta}
\newcommand\hg        {\eta}
\newcommand\tg        {\theta}
\newcommand\ig        {\iota}
\newcommand\kg        {\keppa}
\renewcommand\lg      {\lambda}
\newcommand\og        {\omicron}
\newcommand\rg        {\rho}
\newcommand\sg        {\sigma}
\newcommand\yg        {\usilon}
\newcommand\wg        {\omega}

\newcommand\Ag        {\Alpha}
\newcommand\Bg        {\Beta}
\newcommand\Cg        {\Gamma}
\newcommand\Dg        {\Delta}
\newcommand\Eg        {\Epsi}
\newcommand\Zg        {\Zeta}
\newcommand\Hg        {\Eta}
\newcommand\Tg        {\Theta}
\newcommand\Ig        {\Iota}
\newcommand\Kg        {\Keppa}
\newcommand\Lg        {\Lambda}
\newcommand\Og        {\Omicron}
\newcommand\Rg        {\Rho}
\newcommand\Sg        {\Sigma}
\newcommand\Yg        {\Usilon}
\newcommand\Wg        {\Omega}

% Other shortcuts
\newcommand\tl    {\tilde}
\newcommand\op    {^{-1}}

\newcommand\sof[1]    {\left | #1 \right |}
\newcommand\cl [1]    {\left ( #1 \right )}
\newcommand\csb[1]    {\left [ #1 \right ]}
\newcommand\ccb[1]    {\left \{ #1 \right \}}

\newcommand\bs        {\blacksquare}
\newcommand\dequad    {\!\!\!\!\!\!}
\newcommand\dequadd   {\dequad\duquad}

\renewcommand\phi     {\varphi}

\newtheorem{Theorem}{משפט}
\theoremstyle{definition}
\newtheorem{definition}{הגדרה}
\newtheorem{Lemma}{למה}
\newtheorem{Remark}{הערה}
\newtheorem{Notion}{סימון}

\newcommand\theo  [1] {\begin{Theorem}#1\end{Theorem}}
\newcommand\defi  [1] {\begin{definition}#1\end{definition}}
\newcommand\rmark [1] {\begin{Remark}#1\end{Remark}}
\newcommand\lem   [1] {\begin{Lemma}#1\end{Lemma}}
\newcommand\noti  [1] {\begin{Notion}#1\end{Notion}}

% DS
\DeclareMathOperator\amort   {amort}
\DeclareMathOperator\worst   {worst}
\DeclareMathOperator\type    {type}
\DeclareMathOperator\cost    {cost}

%! ~~~ Document ~~~

\author{שחר פרץ}
\title{\textit{ליניארית 2א 4}}
\begin{document}
	\maketitle
	\textbf{מרצה: }בן בסקין
	\section{על ההבדל בין פולינום לפולינום}
	נבחין ש־$\F[x]$ הוא מ"ו מעל $\F$. וכן $\F[x]$ הוא חוג חילופי עם יחידה. בחוג כפל לא חייב להיות קומטטיבי (נאמר, חוג המטריצות הריבועיות). אומנם קיימת יחידה (פולינום קבוע ב־1) אך אין הופכיים לשום דבר חוץ מלפונ' הקבועות. שזה מאוד חבל כי זה כמעט שדה. 
	
	לכן, נגדיר את $\F(x)$ – אוסף הפונקציות הרציונליות: 
	\[ \F(x) = \ccb{\frac{f(x)}{g(x)} \mid g(x) \neq 0} \]
	זהו שדה. אם נתבונן במטריצות דומות, יש הבדל בין להגיד $f_A(x) \in \F[x]$, אך אפשר לטעון $f_A(x) = |B|$ כש־$B \in M_n(\F(x))$. למה? כי $xI - A \in M_n(\F(x))$ (זה קצת מנוון כי איברי המטריצה הם או פולינומים קבועיים או ממעלה 1). משום שדטרמיננטה שולחת איבר לשדה, אז $|B| \in \F(x)$. כך למעשה נגיע לכך שפולינומים אופייניים שווים כשני איברים בתוך השדה, ולא רק באיך שהם מתנהגים ביחס לקבועיים. 
	
	דוגמה: 
	\[ A = \pms{0 & 1 \\ 0 & 0} \in M_2(\F_2), \ f(x) = x^3, \ g(x) = x, \ f, g \in \F_2 \to \F_2 \implies f = g \]
	אך: 
	\[ f(A) = A^3 = 0, \ g(A) = A \neq 0 \]
	זה לא רצוי. נבחין בשני שוויונות שונים – שוויון פונקציות, בהם $f = g$ מעל $\F_2$, ושוויון בשדה – בו $f -g \neq 0$ (כי $-x^2$ לא פולינום האפס, ואף מעל $\F_2$) ולכן ב־$\F_2(x)$ מתקיים $f \neq g$. 
	
	\section{על הקשר בין ריבוי גיאומטרי ואלגברי}
	\textit{הערה/טענה: }נניח שפ"א של $T$ או $A$: 
	\[ f_T(x) = \prod_{i = 1}^{k}(x - \lg_i)^{n_i} \]
	אז $d_{\lg_i} = n_i$ הריבוי האלגברי. על כן, נבחין כי $n_i \le d_{\lg_i}$. \begin{proof}
		{הוכחה: }
		\[ (x - \lg_i)^{n_i} \mid f_T(x) \implies f_T(x) = (x - \lg_i)^{n_i} \prod_{\overset{j \in [k]}{j \neq i}}(x - \lg_j)^{n_j} \]
		נניח בשלילה $d_{\lg_i} \ge n + 1$. אז: 
		\[ f_T(x) = \cdots = (x - \lg_i)q(x) \]
		נעביר אגפים מהשוויונות השונים ונוציא גורם משותף: 
		\[ (x - \lg_i)^{n}\cl{\overbrace{\prod_{\overset{j \in [k]}{j \neq i}}(x - \lg_j)^{n_j} - (x - \lg_i)q(x)}^{:=P(x)}} = 0 \]
		נדע כי $P(x)$ אינו פולינום האפס כי: 
		\[ P(\lg_i) = \prod_{\overset{j \in [k]}{j \neq i}}(\lg_i - \lg_j)^{n_j} \]
		שוויון בשדה $\F(x)$. וברור כי $(x - \lg_i)^{n}$ אינו פולינום האפס. אך אחד מהם הוא אפס משום שכפל שני איברי שדה שווה לאפס אמ"מ אחד מהם הוא אפס, וסתירה. 
	\end{proof}
	\textit{הערה. }בדוגמה שבטענה ראינו שמתקיים $\sum d_i = \sum n_i = n$ כאשר $n$ דרגת הפולינום. זה לא תמיד המצב. 
	דוגמה למצב בו זה לא קורה: $x^2(x^2 + 1) \in \R[x]$. סכום הריבויים האלגבריים הוא 2, אבל דרגת הפולינום היא 4. זה נכון מעל שדות סגורים אלגברית. 
	
	\textbf{טענה. }תהי $T \co V \to V$ ט"ל. אזי לכל ע"ע $\lg$ מתקיים $r_\lg \le f_\lg$. 
	\begin{proof}
		יהי $\lg$ ע"ע. אז $V_\lg = \{v \in V \mid Tv = \lg v\}$. יהי $B_\lg \subseteq V_\lg$ בסיס עבור $V_\lg$. נשלים אותו לבסיס $B$ של $V$. 
		\[ [T]_B = \pms{\lg & & 0 & * \\ & \lg &  & \\ 0 && \ddots & \\ *&&& C} \]
		ואז: 
		\[ f_T(x) = (x - \lg)^{r_\lg}C(x) \implies r_\lg \le d_\lg \]
	\end{proof}
	
	\textbf{משפט. }תהי $T \co V \to V$ ט"ל עם פ"א $f_T(x)$. אז $T$ לכסינה אמ"מ: 
	\begin{itemize}
		\item בעבור $k$ הע"ע שונים, $f_T(x) = \prod_{i = 1}^{k}(x - \lg_i)^{n_i}$
		\item לכל $\lg$ ע"ע של $T$ מתקיים $r_\lg = d_\lg$
	\end{itemize}
	(הבהרה: 1 לא גורר את 2. צריך את שניהם). 
	
	\begin{proof}\,
		\begin{itemize}
			\item[$\impliedby$]$T$ לכסינה ראינו ש־1 מתקיים. במקרה שלכסינה ראינו ש־$n = \sum r_{\lg_i} \le \sum d_{\lg_i} = n$ ולכן אם לאחד מבין הערכים העצמיים מתקיים $r_\lg \neq d_k$ אז מתקיים $r_k < d_k$ ונקבל סתירה לשוויונות  לעיל. 
			\item[$\implies$] 
			\begin{align*}
				1 &\implies \sum d_{\lg_i} = n \\
				2 &\implies \sum r_{\lg_i} = \sum d_{\lg_i} = n
			\end{align*}
		\end{itemize}
		וסה"כ $\sum r_{\lg_i} = n$ אמ"מ $T$ לכסינה. 
	\end{proof}
	
	\section{לכסון ושילוש}
	\subsection{פיבונאצ'י במרחב סופי}
	סדרת פיבונאצ'י: 
	\[ \pms{a_{n + 1} \\ a_n} = {\underbrace{\pms{1 & 1 \\ 1 & 0}}_{ = 0}}^{n}\pms{1 \\ 0} \]
	נניח שאנו מסתכלים מעל $\F_p$ כלשהו. אז הסדרה חייבת להיות מחזורית. \textbf{שאלה: }מתי מתקיים ש־$A^m = I$ (בעבור $m$ מינימלי)? במילים אחרות, מתי מתחילים מחזור. 
	
	היות שמספר הזוגות השונים עבור $\pms{a_{n + 1} \\ a_n}$ הוא $p^2$, אז $m \le p^2$. עבור $p = 7$: \hfill $0, 1, 1, 2, 3, 4, 5, 1, 6, 0, 6, 6, 5, 4, 2, 6, 1, 0, 1$ – כלומר עבור $p = 7$ יש מחזור באורך $m = 16$. (הערה: תירואטית עם המידע הנוכחי ייתכן ויהפוך למחזורי ולא יחזור להתחלה)
	
	\textbf{טענה. }אם $p$ ראשוני אז $p \equiv 1 \pmod 5$ אז אורך המחזור חסום מלעיל ע"י $p - 1$. 
	
	\begin{proof}
		תנאי מספיק (אך לא הכרחי) לקבלת מחזור באורך $k$ הוא $A^k = I$. אז: 
		\[ f_A(x) = x^2 - x - 1 \]
		יש דבר שנקרא "הדדיות ריבועית" (חומר קריאה רשות במודל) שמבטיחה שורש לפולינום להלן עבור $p$ כנ"ל. אכן יש לנו שני ע"ע שונים (אם קיים רק אחד אז סתירה מהיות הדיסקרימיננטה $5 = 0$ אך $p \not\equiv 1 \pmod 5$). לכן קיימת $P$ הפיכה כך ש־: 
		\[ P\op AP = \pms{\lg_1 & 0 \\ 0 & \lg_2} \]
		כך ש־$\lg_1, \lg_2 \neq 0$. משפט פרמה הקטן אומר ש־$\lg_1^{p - 1}  = \lg_2^{p - 1} = 1$. ואז $A^{p - 1} = I$. 
	\end{proof}
	
	\subsection{מבוא למשפט קיילי־המילטון}
	\textbf{הגדרה. }$T \co V \to V$ ט"ל ניתנת לשילוש אם קיים בסיס $B$ ל־$V$ כך ש־$[T]_B$ משולשית. 
	
	\textbf{הבחנה. }אם $T$ ניתנת לשילוש אז הפולינום האופייני שלה מתפרק לגורמים ליניארים (האם איברי האלכסון של הגרסה המשולשית). יהיה מעניין לשאול אם הכיוון השני מתקיים. 
	
	\textbf{משפט. }$T \co V \to V$ ט"ל. נניח ש־$f_T(x) = \prod_{i = 1}^{n}(x - \lg_i)$ (ניתנת לפירוק לגורמים ליניאריים) אז $T$ ניתנת לשילוש. \begin{proof}
		\textit{בסיס. }$n = 1$ היא כבר משולשית וסיימנו. \\
		\textit{צעד. }נניח שהטענה נכונה בעבור $n$ טבעי כלשהו, ונראה נכונות עבור $n + 1$. אז $f_T$ מתפרק לגורמים ליניאריים, לכן יש לו שורש. יהי $\lg$ ע"ע של $T$. בסיס $B$ של $V$ מקיים ש־$[T]_B$ משולשית עליונה (נסמן $B = (w_1 \dots w_{n + 1})$) $\iff$ אז $T(w_i) \in \Sp(w_1 \dots w_i)$. נגדיר את $w_1$ להיות ו"ע של $\lg$. נשלימו לבסיס $B^1$. 
		\[ [T]_B = \pms{\lg & & * && \\ 0 && \vdots \\ \vdots &\cdots & C & \cdots \\ 0 & &\vdots } \]
		אז ניתן לומר כי: 
		\[ f_T(x) = (x - \lg) f_C(x) \]
		נסמן $w = \Sp(w_2 \dots w_{n + 1})$. קיימת העתקה ליניארית $S \co W \to W$ כך ש־$f_S(x) = f_C(x)$. לפי ה"א קיים בסיס ל־$W$ הוא $B''$ שעבורו $S$ משולשית עליונה. נטען ש־$B = B'' \cup \{w_1\}$ ייתן את הדרוש. 
		\[ \forall w \in B'' \co (T - S)(w) = Tw - Sw = aw_1 + S(w) - S(w) = aw_1 \]
		(כלומר, השורה העליונה של $[T]_B$ "תרמה" את $aw_1$ בלבד)
		לכן: 
		\[ (T - S)w \subseteq \Sp(w_1) \]
		זה גורר שלכל $w \in W$ מליניאריות מתקיים ש־$(T - S)w \subseteq \Sp(w_1)$. סה"כ לכל $w \in B'' \cup \{w_1\}$ מתקיים $T(w_i) \in \Sp(w_1 \dots)$. 
	\end{proof}
	
	בהוכחה הזו, בנינו בסיס כך ש־: 
	\[ [T(w) - S(w)]_B = ae_1 \]
	
	\subsubsection{עוד מבוא לקיילי־המילטון}
	\textbf{הגדרה. }יהי $f(x) = \sum_{i = 0}^{d}a_ix^i \in \F[x]$, $V$ מ"ו מעל $\F$ נ"ס (נוצר סופית) וכן $T \co V \to V$ ט"ל. נגדיר: 
	\[ f(T) = \sum_{i = 0}^{d}a_iT^{i}, \ T^0 = id, \ T^{n} = T \circ T^{n - 1} \]
	כנ"ל עם מטריצות (ראה תרגול)
	
	\textbf{טענה. }אם $A = [T]_B$ ו־$f(x) \in \F[x]$, אז $[f(T)]_B = f(A)$. הוכחה נובעת מהתכונות $[TS]_B = AC, \ [T + S]_B = A + C, \ [\ag T]_B = \ag A, \ [S]_B = C<\ [T]_B = A$. 
	
	\textbf{טענה. }אם $f, g \in \F[x]$ ו־$T \co V \to V$ ט"ל, אז $(f \cdot g)(T) = f(T) \cdot g(T)$. באופן דומה $(f + g)(T) = f(T) + g(T)$. 
	
	לכן קל לראות ש־$f(T) = 0 \iff f(A) = 0$. 
	
	\textbf{מסקנה. }אם $A, C$ דומות אז $f(A) = 0 \iff f(C) = 0$. 
	
	\section{משפט קיילי־המילטון}
	\textbf{משפט קיילי־המילטון. }\textit{לכל $T\co V \to V$ ט"ל ($V$ נוצר סופית) ולכל $A \in M_n(\F)$ מתקיים: 
	\[ f_T(T) = 0, \ f_A(A) = 0 \]}
	
	\textbf{דוגמה. }(מנוונת) נתבונן ב־$D \co \F_n[x] \to \F_n[x]$ אופרטור הגזירה. ראינו $f_D(x) = x^{n + 1}$ (הפולינום האופייני). אז $f_D(D)(p) = p^{(n + 1)} = 0 \implies f_D(D) = 0$
	
\end{document}