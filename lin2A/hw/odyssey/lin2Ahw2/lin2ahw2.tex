%! ~~~ Packages Setup ~~~ 
\documentclass[]{article}
\usepackage{lipsum}
\usepackage{rotating}


% Math packages
\usepackage[usenames]{color}
\usepackage{forest}
\usepackage{ifxetex,ifluatex,amssymb,amsmath,mathrsfs,amsthm,witharrows,mathtools,mathdots}
\usepackage{amsmath}
\WithArrowsOptions{displaystyle}
\renewcommand{\qedsymbol}{$\blacksquare$} % end proofs with \blacksquare. Overwrites the defualts. 
\usepackage{cancel,bm}
\usepackage[thinc]{esdiff}


% tikz
\usepackage{tikz}
\usetikzlibrary{graphs}
\newcommand\sqw{1}
\newcommand\squ[4][1]{\fill[#4] (#2*\sqw,#3*\sqw) rectangle +(#1*\sqw,#1*\sqw);}


% code 
\usepackage{listings}
\usepackage{xcolor}

\definecolor{codegreen}{rgb}{0,0.35,0}
\definecolor{codegray}{rgb}{0.5,0.5,0.5}
\definecolor{codenumber}{rgb}{0.1,0.3,0.5}
\definecolor{codeblue}{rgb}{0,0,0.5}
\definecolor{codered}{rgb}{0.5,0.03,0.02}
\definecolor{codegray}{rgb}{0.96,0.96,0.96}

\lstdefinestyle{pythonstylesheet}{
	language=Java,
	emphstyle=\color{deepred},
	backgroundcolor=\color{codegray},
	keywordstyle=\color{deepblue}\bfseries\itshape,
	numberstyle=\scriptsize\color{codenumber},
	basicstyle=\ttfamily\footnotesize,
	commentstyle=\color{codegreen}\itshape,
	breakatwhitespace=false, 
	breaklines=true, 
	captionpos=b, 
	keepspaces=true, 
	numbers=left, 
	numbersep=5pt, 
	showspaces=false,                
	showstringspaces=false,
	showtabs=false, 
	tabsize=4, 
	morekeywords={as,assert,nonlocal,with,yield,self,True,False,None,AssertionError,ValueError,in,else},              % Add keywords here
	keywordstyle=\color{codeblue},
	emph={var, List, Iterable, Iterator},          % Custom highlighting
	emphstyle=\color{codered},
	stringstyle=\color{codegreen},
	showstringspaces=false,
	abovecaptionskip=0pt,belowcaptionskip =0pt,
	framextopmargin=-\topsep, 
}
\newcommand\pythonstyle{\lstset{pythonstylesheet}}
\newcommand\pyl[1]     {{\lstinline!#1!}}
\lstset{style=pythonstylesheet}

\usepackage[style=1,skipbelow=\topskip,skipabove=\topskip,framemethod=TikZ]{mdframed}
\definecolor{bggray}{rgb}{0.85, 0.85, 0.85}
\mdfsetup{leftmargin=0pt,rightmargin=0pt,innerleftmargin=15pt,backgroundcolor=codegray,middlelinewidth=0.5pt,skipabove=5pt,skipbelow=0pt,middlelinecolor=black,roundcorner=5}
\BeforeBeginEnvironment{lstlisting}{\begin{mdframed}\vspace{-0.4em}}
	\AfterEndEnvironment{lstlisting}{\vspace{-0.8em}\end{mdframed}}


% Deisgn
\usepackage[labelfont=bf]{caption}
\usepackage[margin=0.6in]{geometry}
\usepackage{multicol}
\usepackage[skip=4pt, indent=0pt]{parskip}
\usepackage[normalem]{ulem}
\forestset{default}
\renewcommand\labelitemi{$\bullet$}
\usepackage{titlesec}
\titleformat{\section}[block]
{\fontsize{15}{15}}
{\sen \dotfill (\thesection)\dotfill\she}
{0em}
{\MakeUppercase}
\usepackage{graphicx}
\graphicspath{ {./} }


% Hebrew initialzing
\usepackage[bidi=basic]{babel}
\PassOptionsToPackage{no-math}{fontspec}
\babelprovide[main, import, Alph=letters]{hebrew}
\babelprovide[import]{english}
\babelfont[hebrew]{rm}{David CLM}
\babelfont[hebrew]{sf}{David CLM}
\babelfont[english]{tt}{Monaspace Xenon}
\usepackage[shortlabels]{enumitem}
\newlist{hebenum}{enumerate}{1}

% Language Shortcuts
\newcommand\en[1] {\begin{otherlanguage}{english}#1\end{otherlanguage}}
\newcommand\sen   {\begin{otherlanguage}{english}}
	\newcommand\she   {\end{otherlanguage}}
\newcommand\del   {$ \!\! $}

\newcommand\npage {\vfil {\hfil \textbf{\textit{המשך בעמוד הבא}}} \hfil \vfil \pagebreak}
\newcommand\ndoc  {\dotfill \\ \vfil {\begin{center}
			{\textbf{\textit{שחר פרץ, 2025}} \\
				\scriptsize \textit{קומפל ב־}\en{\LaTeX}\,\textit{ ונוצר באמצעות תוכנה חופשית בלבד}}
	\end{center}} \vfil	}

\newcommand{\rn}[1]{
	\textup{\uppercase\expandafter{\romannumeral#1}}
}

\makeatletter
\newcommand{\skipitems}[1]{
	\addtocounter{\@enumctr}{#1}
}
\makeatother

%! ~~~ Math shortcuts ~~~

% Letters shortcuts
\newcommand\N     {\mathbb{N}}
\newcommand\Z     {\mathbb{Z}}
\newcommand\R     {\mathbb{R}}
\newcommand\Q     {\mathbb{Q}}
\newcommand\C     {\mathbb{C}}
\newcommand\One   {\mathit{1}}

\newcommand\ml    {\ell}
\newcommand\mj    {\jmath}
\newcommand\mi    {\imath}

\newcommand\powerset {\mathcal{P}}
\newcommand\ps    {\mathcal{P}}
\newcommand\pc    {\mathcal{P}}
\newcommand\ac    {\mathcal{A}}
\newcommand\bc    {\mathcal{B}}
\newcommand\cc    {\mathcal{C}}
\newcommand\dc    {\mathcal{D}}
\newcommand\ec    {\mathcal{E}}
\newcommand\fc    {\mathcal{F}}
\newcommand\nc    {\mathcal{N}}
\newcommand\vc    {\mathcal{V}} % Vance
\newcommand\sca   {\mathcal{S}} % \sc is already definded
\newcommand\rca   {\mathcal{R}} % \rc is already definded

\newcommand\prm   {\mathrm{p}}
\newcommand\arm   {\mathrm{a}} % x86
\newcommand\brm   {\mathrm{b}}
\newcommand\crm   {\mathrm{c}}
\newcommand\drm   {\mathrm{d}}
\newcommand\erm   {\mathrm{e}}
\newcommand\frm   {\mathrm{f}}
\newcommand\nrm   {\mathrm{n}}
\newcommand\vrm   {\mathrm{v}}
\newcommand\srm   {\mathrm{s}}
\newcommand\rrm   {\mathrm{r}}

\newcommand\Si    {\Sigma}

% Logic & sets shorcuts
\newcommand\siff  {\longleftrightarrow}
\newcommand\ssiff {\leftrightarrow}
\newcommand\so    {\longrightarrow}
\newcommand\sso   {\rightarrow}

\newcommand\epsi  {\epsilon}
\newcommand\vepsi {\varepsilon}
\newcommand\vphi  {\varphi}
\newcommand\Neven {\N_{\mathrm{even}}}
\newcommand\Nodd  {\N_{\mathrm{odd }}}
\newcommand\Zeven {\Z_{\mathrm{even}}}
\newcommand\Zodd  {\Z_{\mathrm{odd }}}
\newcommand\Np    {\N_+}

% Text Shortcuts
\newcommand\open  {\big(}
\newcommand\qopen {\quad\big(}
\newcommand\close {\big)}
\newcommand\also  {\text{, }}
\newcommand\defis {\text{ definitions}}
\newcommand\given {\text{given }}
\newcommand\case  {\text{if }}
\newcommand\syx   {\text{ syntax}}
\newcommand\rle   {\text{ rule}}
\newcommand\other {\text{else}}
\newcommand\set   {\ell et \text{ }}
\newcommand\ans   {\mathscr{A}\!\mathit{nswer}}

% Set theory shortcuts
\newcommand\ra    {\rangle}
\newcommand\la    {\langle}

\newcommand\oto   {\leftarrow}

\newcommand\QED   {\quad\quad\mathscr{Q.E.D.}\;\;\blacksquare}
\newcommand\QEF   {\quad\quad\mathscr{Q.E.F.}}
\newcommand\eQED  {\mathscr{Q.E.D.}\;\;\blacksquare}
\newcommand\eQEF  {\mathscr{Q.E.F.}}
\newcommand\jQED  {\mathscr{Q.E.D.}}

\DeclareMathOperator\dom   {dom}
\DeclareMathOperator\Img   {Im}
\DeclareMathOperator\range {range}

\newcommand\trio  {\triangle}

\newcommand\rc    {\right\rceil}
\newcommand\lc    {\left\lceil}
\newcommand\rf    {\right\rfloor}
\newcommand\lf    {\left\lfloor}

\newcommand\lex   {<_{lex}}

\newcommand\az    {\aleph_0}
\newcommand\uaz   {^{\aleph_0}}
\newcommand\al    {\aleph}
\newcommand\ual   {^\aleph}
\newcommand\taz   {2^{\aleph_0}}
\newcommand\utaz  { ^{\left (2^{\aleph_0} \right )}}
\newcommand\tal   {2^{\aleph}}
\newcommand\utal  { ^{\left (2^{\aleph} \right )}}
\newcommand\ttaz  {2^{\left (2^{\aleph_0}\right )}}

\newcommand\n     {$n$־יה\ }

% Math A&B shortcuts
\newcommand\logn  {\log n}
\newcommand\logx  {\log x}
\newcommand\lnx   {\ln x}
\newcommand\cosx  {\cos x}
\newcommand\sinx  {\sin x}
\newcommand\sint  {\sin \theta}
\newcommand\tanx  {\tan x}
\newcommand\tant  {\tan \theta}
\newcommand\sex   {\sec x}
\newcommand\sect  {\sec^2}
\newcommand\cotx  {\cot x}
\newcommand\cscx  {\csc x}
\newcommand\sinhx {\sinh x}
\newcommand\coshx {\cosh x}
\newcommand\tanhx {\tanh x}

\newcommand\seq   {\overset{!}{=}}
\newcommand\slh   {\overset{LH}{=}}
\newcommand\sle   {\overset{!}{\le}}
\newcommand\sge   {\overset{!}{\ge}}
\newcommand\sll   {\overset{!}{<}}
\newcommand\sgg   {\overset{!}{>}}

\newcommand\h     {\hat}
\newcommand\ve    {\vec}
\newcommand\lv    {\overrightarrow}
\newcommand\ol    {\overline}

\newcommand\mlcm  {\mathrm{lcm}}

\DeclareMathOperator{\sech}   {sech}
\DeclareMathOperator{\csch}   {csch}
\DeclareMathOperator{\arcsec} {arcsec}
\DeclareMathOperator{\arccot} {arcCot}
\DeclareMathOperator{\arccsc} {arcCsc}
\DeclareMathOperator{\arccosh}{arccosh}
\DeclareMathOperator{\arcsinh}{arcsinh}
\DeclareMathOperator{\arctanh}{arctanh}
\DeclareMathOperator{\arcsech}{arcsech}
\DeclareMathOperator{\arccsch}{arccsch}
\DeclareMathOperator{\arccoth}{arccoth}
\DeclareMathOperator{\atant}  {atan2} 
\DeclareMathOperator{\Sp}     {span} 
\DeclareMathOperator{\sgn}    {sgn} 
\DeclareMathOperator{\row}    {Row} 
\DeclareMathOperator{\adj}    {adj} 
\DeclareMathOperator{\rk}     {rank} 
\DeclareMathOperator{\col}    {Col} 
\DeclareMathOperator{\tr}     {tr}

\newcommand\dx    {\,\mathrm{d}x}
\newcommand\dt    {\,\mathrm{d}t}
\newcommand\dtt   {\,\mathrm{d}\theta}
\newcommand\du    {\,\mathrm{d}u}
\newcommand\dv    {\,\mathrm{d}v}
\newcommand\df    {\mathrm{d}f}
\newcommand\dfdx  {\diff{f}{x}}
\newcommand\dit   {\limhz \frac{f(x + h) - f(x)}{h}}

\newcommand\nt[1] {\frac{#1}{#1}}

\newcommand\limz  {\lim_{x \to 0}}
\newcommand\limxz {\lim_{x \to x_0}}
\newcommand\limi  {\lim_{x \to \infty}}
\newcommand\limh  {\lim_{x \to 0}}
\newcommand\limni {\lim_{x \to - \infty}}
\newcommand\limpmi{\lim_{x \to \pm \infty}}

\newcommand\ta    {\theta}
\newcommand\ap    {\alpha}

\renewcommand\inf {\infty}
\newcommand  \ninf{-\inf}

% Combinatorics shortcuts
\newcommand\sumnk     {\sum_{k = 0}^{n}}
\newcommand\sumni     {\sum_{i = 0}^{n}}
\newcommand\sumnko    {\sum_{k = 1}^{n}}
\newcommand\sumnio    {\sum_{i = 1}^{n}}
\newcommand\sumai     {\sum_{i = 1}^{n} A_i}
\newcommand\nsum[2]   {\reflectbox{\displaystyle\sum_{\reflectbox{\scriptsize$#1$}}^{\reflectbox{\scriptsize$#2$}}}}

\newcommand\bink      {\binom{n}{k}}
\newcommand\setn      {\{a_i\}^{2n}_{i = 1}}
\newcommand\setc[1]   {\{a_i\}^{#1}_{i = 1}}

\newcommand\cupain    {\bigcup_{i = 1}^{n} A_i}
\newcommand\cupai[1]  {\bigcup_{i = 1}^{#1} A_i}
\newcommand\cupiiai   {\bigcup_{i \in I} A_i}
\newcommand\capain    {\bigcap_{i = 1}^{n} A_i}
\newcommand\capai[1]  {\bigcap_{i = 1}^{#1} A_i}
\newcommand\capiiai   {\bigcap_{i \in I} A_i}

\newcommand\xot       {x_{1, 2}}
\newcommand\ano       {a_{n - 1}}
\newcommand\ant       {a_{n - 2}}

% Linear Algebra
\DeclareMathOperator{\chr}     {char}
\DeclareMathOperator{\diag}    {diag}
\DeclareMathOperator{\Hom}     {Hom}

\newcommand\lra       {\leftrightarrow}
\newcommand\chrf      {\chr(\F)}
\newcommand\F         {\mathbb{F}}
\newcommand\co        {\colon}
\newcommand\tmat[2]   {\cl{\begin{matrix}
			#1
		\end{matrix}\, \middle\vert\, \begin{matrix}
			#2
\end{matrix}}}

\makeatletter
\newcommand\rrr[1]    {\xxrightarrow{1}{#1}}
\newcommand\rrt[2]    {\xxrightarrow{1}[#2]{#1}}
\newcommand\mat[2]    {M_{#1\times#2}}
\newcommand\gmat      {\mat{m}{n}(\F)}
\newcommand\tomat     {\, \dequad \longrightarrow}
\newcommand\pms[1]    {\begin{pmatrix}
		#1
\end{pmatrix}}
\newcommand\bms[1]    {\begin{bmatrix}
		#1
\end{bmatrix}}
\newcommand\detms[1]  {\left\vert\begin{matrix}
		#1
\end{matrix}\right\vert}

% someone's code from the internet: https://tex.stackexchange.com/questions/27545/custom-length-arrows-text-over-and-under
\makeatletter
\newlength\min@xx
\newcommand*\xxrightarrow[1]{\begingroup
	\settowidth\min@xx{$\m@th\scriptstyle#1$}
	\@xxrightarrow}
\newcommand*\@xxrightarrow[2][]{
	\sbox8{$\m@th\scriptstyle#1$}  % subscript
	\ifdim\wd8>\min@xx \min@xx=\wd8 \fi
	\sbox8{$\m@th\scriptstyle#2$} % superscript
	\ifdim\wd8>\min@xx \min@xx=\wd8 \fi
	\xrightarrow[{\mathmakebox[\min@xx]{\scriptstyle#1}}]
	{\mathmakebox[\min@xx]{\scriptstyle#2}}
	\endgroup}
\makeatother


% Greek Letters
\newcommand\ag        {\alpha}
\newcommand\bg        {\beta}
\newcommand\cg        {\gamma}
\newcommand\dg        {\delta}
\newcommand\eg        {\epsi}
\newcommand\zg        {\zeta}
\newcommand\hg        {\eta}
\newcommand\tg        {\theta}
\newcommand\ig        {\iota}
\newcommand\kg        {\keppa}
\renewcommand\lg      {\lambda}
\newcommand\og        {\omicron}
\newcommand\rg        {\rho}
\newcommand\sg        {\sigma}
\newcommand\yg        {\usilon}
\newcommand\wg        {\omega}

\newcommand\Ag        {\Alpha}
\newcommand\Bg        {\Beta}
\newcommand\Cg        {\Gamma}
\newcommand\Dg        {\Delta}
\newcommand\Eg        {\Epsi}
\newcommand\Zg        {\Zeta}
\newcommand\Hg        {\Eta}
\newcommand\Tg        {\Theta}
\newcommand\Ig        {\Iota}
\newcommand\Kg        {\Keppa}
\newcommand\Lg        {\Lambda}
\newcommand\Og        {\Omicron}
\newcommand\Rg        {\Rho}
\newcommand\Sg        {\Sigma}
\newcommand\Yg        {\Usilon}
\newcommand\Wg        {\Omega}

% Other shortcuts
\newcommand\tl    {\tilde}
\newcommand\op    {^{-1}}

\newcommand\sof[1]    {\left | #1 \right |}
\newcommand\cl [1]    {\left ( #1 \right )}
\newcommand\csb[1]    {\left [ #1 \right ]}
\newcommand\ccb[1]    {\left \{ #1 \right \}}

\newcommand\bs        {\blacksquare}
\newcommand\dequad    {\!\!\!\!\!\!}
\newcommand\dequadd   {\dequad\duquad}

\renewcommand\phi     {\varphi}

\newtheorem{Theorem}{משפט}
\theoremstyle{definition}
\newtheorem{definition}{הגדרה}
\newtheorem{Lemma}{למה}
\newtheorem{Remark}{הערה}
\newtheorem{Notion}{סימון}

\newcommand\theo  [1] {\begin{Theorem}#1\end{Theorem}}
\newcommand\defi  [1] {\begin{definition}#1\end{definition}}
\newcommand\rmark [1] {\begin{Remark}#1\end{Remark}}
\newcommand\lem   [1] {\begin{Lemma}#1\end{Lemma}}
\newcommand\noti  [1] {\begin{Notion}#1\end{Notion}}

% DS
\DeclareMathOperator\amort   {amort}
\DeclareMathOperator\worst   {worst}
\DeclareMathOperator\type    {type}
\DeclareMathOperator\cost    {cost}

%! ~~~ Document ~~~

\author{שחר פרץ}
\title{\textit{תרגיל בית 2 $\sim$ אלגברה ליניארית 2א}}
\begin{document}
	\maketitle
	\section{}
	נלכסן את המטריצות הבאות: 
	\begin{enumerate}[A)]
		\item נלכסן את המטריצה $\pms{0 & 1 \\ 2 & -1}$. ראשית כל, נמצא את שורשי הפולינום האופייני כדי למצוא ע"עים. 
		\[ p_A(x) = \detms{-x & 1 \\ 2 & -1 - x} = -x(-1 - x) - 1 \cdot 2 = x^2 + x - 2 \]
		שורשי הפולינום האופייני הם $1, -2$. 
		
		נמצא ו"ע לע"ע $-2$: 
		\[ \nc (A + 2I) = \nc\pms{2 & 1 \\ 2 & 1} = \nc \pms{2 & 1 \\ 0 & 0} = \Sp\ccb{\pms{2 \\ -1}} \]
		וסה"כ $(2, -1)$ ו"ע לע"ע $-2$. נמצא ו"ע לע"ע $1$: 
		\[ \nc(A - I) = \pms{-1 & 1 \\ 2 & -2} = \nc\pms{-1 & 1 \\ 1 & -1} = \nc\pms{-1 & 1 \\ -1 & 1} = \nc\pms{1 & -1 \\ 0 & 0} = \Sp\ccb{\pms{1 \\ 1}} \]
		סה"כ $P = \pms{1 & 2 \\ 1 & -1}$ ו־$D = \pms{1 & 0 \\ 0 & 2}$ יקיימו $A = PDP\op$ כדרוש. 
		
		\item נלכסן את המטריצה: 
		\[ \pms{1 & -2 & 0 \\ 1 & 4 & 0 \\ 0 & 0 & -5} \]
		נמצא פולינום אופייני: 
		\[ p_A(x) = \det(A - xI) = \detms{1 - x & -2 & 0 \\ 1 & 4 - x & 0 \\ 0 & 0 & 5 - x} = (-5 - x)((1 - x)(4 - x) + 2) \]
		נמצא שורשים: 
		\[ \begin{cases}
			-5 - x = 0 \implies x = -5 \\
			x^2 - 5x + 6 = 0 \implies \xot = 3, 2
		\end{cases} \]
		סה"כ $-5, 3, 2$ ע"ע. נמצא ו"ע ל־$2$ הע"ע: 
		\[ \nc(A - 2I) = \nc\pms{-1 & -2 & 0 \\ 1 & 2 & 0 \\ 0 & 0 & -7} = \nc\pms{0 & 0 & 0 \\ 1 & 2 & 0 \\ 0 & 0 & -7} = \nc \pms{1 & 2 & 0 \\ 0 & 0 & 0 \\ 0 & 0 & 1} = \ccb{\pms{a \\ -2a \\ 0} \mid a, b \in \R} = \Sp\ccb{\pms{1 \\ -2 \\ 0}} \]
		ו"ע של $3$: 
		\[ \nc(A - 3I) = \nc\pms{-2 & -2 & 0 \\ 1 & 1 & 0 \\ 0 & 0 & -8} = \nc\pms{0 & 0 & 0 \\ 1 & 1 & 0 \\ 0 & 0 & 1} = \Sp\ccb{\pms{1 \\ - 1 \\ 0}} \]
		ו"ע $-5$ ע"ע: 
		\[ \nc(A + 5I) = \nc\pms{6 & -2 & 0 \\ 1& 9 & 0 \\ 0 & 0 & 0} = \nc\pms{0 & -56 \\ 1 & 9} = \nc\pms{0 & 1 & 0 \\ 1 & 0 & 0 \\ 0 & 0 & 0} = \Sp\pms{0 \\ 0 \\ 1} \]
		סה"כ: 
		\[ P = \pms{0 & 1 & 1\\ 0 & -1 & -2 \\ 1 & 0 & 0}, \ \Lg = \pms{-5 & 0 & 0 \\ 0 & 3 & 0 \\ 0 & 0 & 2}, \ A = P\Lg P\op \]
		
		
		\item נלכסן את המטריצה מעל $\Z_3$: 
		\[ A := \pms{1 & 1 & 0 \\ 1 & 1 & 0 \\ 0 & 0 & 1}, \ p_A(x) = \det(xI - A) = \detms{1 - x & 1 & 0 \\ 1 & 1 - x & 0 \\ 0 & 0 & 1 - x} = (1 - x)\detms{1 - x & 1 \\ 1 & 1 - x} = (1 - x)((1 - x)^{2} - 1) \]
		מצאנו את הפולינום. שורשיו: 
		\[ \begin{cases}
			1 - x = 0 \implies x = 1 \\
			(1 - x)^2 -1 = 0 \implies (1 - x)^2 = 1 \implies \begin{cases}
				1 - x = 1 \implies x = 0 \\
				1 - x = 2 \implies x = 2
			\end{cases}
		\end{cases} \]
		
		סה"כ ע"ע $x = 0, 1, 2$ כל השדה $\Z_3$ ע"ע. 
		
		נמצא ו"ע ל־$0$: 
		\[ \nc(A - 0I) = \nc\pms{1 & 1 & 0 \\ 1 & 1 & 0 \\ 0 & 0 & 1} = \nc\pms{1 & 1 & 0 \\ 0 & 0 & 0 \\ 0& 0 & 1} = \Sp\ccb{\pms{1 \\ -1 \\ 0}} \]
		
		נמצא ו"ע ל־$1$: 
		\[ \nc(A - 1I) = \nc\pms{0 & 1 & 0 \\ 1 & 0 & 0 \\ 0& 0 & 0} = \Sp\ccb{\pms{0 \\ 0 \\ 1}} \]
		נמצא ו"ע ל־$2$: 
		\[ \nc(A - 2I) = \nc\pms{2 & 1 & 0 \\ 1 & 2 & 0 \\ 0 & 0 & 2} = \nc\pms{0 & 0 & 0 \\ 1 & 2 & 0 \\ 0 & 0 & 1} = \Sp\ccb{\pms{1 \\ -2 \\ 0}} \]
		וסה"כ: 
		\[ \Lg := \pms{0 & 0 & 0 \\ 0 & 1 & 0 \\ 0 & 0 & 2}, \ P := \pms{1 & 0 & 1 \\ -1 & 0 & -2 \\ 0 & 1 & 0}, \ A = P\Lg P\op \]
	\end{enumerate}
	
	\section{}
	יהיו $A, B \in M_n(\F)$, נראה כי ל־$AB$ ו־$BA$ אותו הפולינום האופייני. 
	\begin{enumerate}[A)]
		\item ראשית כל נוכיח את הטענה במקרה ש־$A$ הפיכה. \begin{proof}
			משום ש־$A$ הפיכה אז קיימת $A\op$ הופכית לה. אזי: 
			\[ A \cdot BA \cdot A\op = AB \cdot \underbrace{AA\op}_{=I} = AB \]
			כלומר, מהגדרה, $BA$ דומה ל־$AB$. נוכיח שכל שתי מטריצות דומות $C = PDP\op$ בעלות אותו הפולינום האופייני: 
			\[ p_C(x) = \det(C - xI) = \det(PDP\op - xPP\op) = \det(P(D - xI)P\op) = \det (P P\op) \cdot \det (D - xI) = \det(D - xI) = p_D(x) \]
			סה"כ משום ש־$BA$ ו־$AB$ דומות, אז הפולינום האופייני שלהן זהה. 
		\end{proof}
		\item עתה, נראה את הטענה במקרה ו־$A = \pms{I_r & 0 \\ 0 & 0}$ מטריצת בלוקים. \begin{proof}
			נסמן $B = \pms{B_r & B_2 \\ B_3 & B_4}$ הבלוקים מהם $B$ מורכבת, כאשר $B_r$ בלוק $r \times r$. אז: 
			\[ AB = \pms{B_r & B_2 \\ B_3 & B_4}\pms{I_r & 0 \\ 0 & 0} = \pms{B_r & 0 \\ B_3 & 0}, \ BA = \pms{I_r & 0 \\ 0 & 0} \pms{B_r & B_2 \\ B_3 & B_4} = \pms{B_r & B_2 \\ 0 & 0} \]
			ולכן: 
			\begin{alignat*}{9}
				p_{AB}(x) &= \det\cl{\pms{B_r & 0 \\ B_3 & 0} - xI_n} &= \det\cl{\pms{B_r - xI_r & 0 \\ B_3 & -xI_{n - r}}} &= \det(B_r - I_rx)(-x)^{n - r} \\
				p_{BA}(x) &= \det\cl{\pms{B_r & B_2 \\ 0 & 0} - xI_n} &= \det\cl{\pms{B_r - xI_r & B_2 \\ 0 & -xI_{n - r}}} &= \det(B_r - I_rx)(-x)^{n - r}\\
				&\implies p_{AB}(x) = p_{BA}(x)
			\end{alignat*}
			כדרוש. 
		\end{proof}
		\item כעת, נוכיח את הטענה עבור המקרה הכללי. \begin{proof}
			יהיו $A, B$ מטריצות כלשהן מגודל $n$ מעל שדה $\F$ כלשהו. נסמן $r = \rk A$. נסמן $I_R = \pms{I_r & 0 \\ 0 & 0}$
			ידוע קיום $P, Q \in M_n(\F)$ הפיכות כך ש־$A = PI_RQ\op$, משום ש־$\rk I_R = \rk A$. אזי $AB = PI_RQ\op B$. גם ידוע ש־$Q\op$ הפיכה שכן $Q$ הפיכה וההפיכה להפיכה הפיכה. מסעיפים א' וב'־: 
			\[ p_{AB}(x) = p_{(PI_RQ\op B)}(x) \overset{\text{א'}}{=} p_{(I_RQ\op B P)}(x) \overset{\text{ב'}}{=} p_{(Q\op B P I_R)}(x) \overset{\text{א'}}{=} p_{(B P I_R Q\op)}(x) = p_{BA}(x) \]
			וסה"כ $p_{AB}(x) = p_{BA}(x)$ כדרוש. 
		\end{proof}
	\end{enumerate}
	
	\section{}
	יהי $p(x) = x^{n} + a_{n - 1}x^{n - 1} + \cdots + a_1x + a_0$ פולינום מתוקן. נגדיר את \textit{המטריצה המלוואה של $p$} להיות: 
	\[ A_p = \pms{-a_{n - 1} & -a_{n - 2} & \cdots & -a_1 & -a_0 \\
		1 & 0 & \cdots & 0 & 0 \\
		0 & 1 & \ddots  & 0 & 0 \\
		\vdots & \ddots & \ddots & \ddots & \vdots \\ 0 & \cdots & 0 & 1 & 0} \]
	\begin{enumerate}[A)]
		\item נראה כי $p_{A_p}(x) = p(x)$. \begin{proof}
			נמצא את הפולינום האופייני של $A_p$: 
			\begin{align*}
				\det (A_p - xI) &= \detms{-a_{n - 1} -x & -a_{n - 2} & \cdots & -a_1 & -a_0 \\
					1 & -x & \cdots & 0 & 0 \\
					0 & 1 & \ddots  & 0 & 0 \\
					\vdots & \ddots & \ddots & \ddots & \vdots \\ 0 & \cdots & 0 & 1 & -x} \\
					&= (-a_{n - 1} - x)\underbrace{\detms{
						-x & 0 & \dots & 0 \\ 1 & -x & \ddots & \vdots & \\ 0 & \ddots & \ddots & 0 \\ \vdots & \ddots & 1 & -x
						}}_{D_{n - 1}} + \sum_{i = 0}^{n - 2}-a_{i}D_i
			\end{align*}
			כאשר $D_i = \det((A_p - xI)_{1i})$. קל להראות באמצעות החלפות ומעברי שורה ש־$D_i = -x^i$. אז קיבלנו: 
			\[ = (-a_{n - 1} - x)x^{n - 1} + \sum_{i = 0}^{n - 2}(-a_i \cdot -x^{n}) = x^{n} + a_{n - 1}x^{n - 1} + \sum_{i = 0}^{n - 2}a_i x^{n} = p(x) \]
			כדרוש. 
		\end{proof}
		
		\hfil \Large \textbf{מעתה ואילך הכל טיוטה, ולא מומלץ לבדוק את זה} \normalsize
		
		
		\item נניח שלפולינום $p(x)$ יש $n$ שורשים שונים. נלכסן את $A_p$. 
		
		משום של־$p(x)$ יש $n$ שורשים וגם $p(x) = p_{A_p}(x)$ מסעיף קודם, אז שורשי $p(x)$ הם כל הע"ע של $A_p$. נסמנם $\bg_1 \dots \bg_n$. נמצא ו"ע ל־$\bg_i$: 
		ידוע: 
		\[ p(\bg_i) = \bg_i^n + a_{n - 1}\bg_i^{n - 1} + \cdots + a_0\bg_i^0 = 0 \]
		לכן: 
		\[ \nc(A_p - \bg_iI) = \nc\pms{
				-a_{n - 1} - \bg_i & -a_{n - 2} & \cdots & -a_1 & -a_0 \\
				1 & -\bg_i & \cdots & 0 & 0 \\
				0 & 1 & \ddots  & 0 & 0 \\
				\vdots & \ddots & \ddots & \ddots & \vdots \\ 0 & \cdots & 0 & 1 & -\bg_n
			} \]
			אם נכפול את השורה ה־$i$ ב־$\bg^{i - 1}$: 
		\begin{align*}
			=&\nc\pms{
				-a_{n - 1} - \bg_i & -a_{n - 2} & \cdots & -a_1 & -a_0 \\
				1 & -\bg_i^{n - 1} & \cdots & 0 & 0 \\
				0 & 1 & \ddots  & 0 & 0 \\
				\vdots & \ddots & \ddots & \ddots & \vdots \\ 0 & \cdots & 0 & 1 & -\bg_n
			} \rrt{\forall 1 < i \le n}{R_i \to R_i + R_{i - 1}} \\
			&\nc\pms{
				-p(0) & -a_{n - 2}\bg^{n - 2} & \cdots & -a_1\bg^1 & -a_0\bg^0 \\
				0 & -\bg_i^{n - 1} & \cdots & 0 & 0 \\
				0 & -\bg^{n - 2} & \ddots  & 0 & 0 \\
				\vdots & \ddots & \ddots & \ddots & \vdots \\ 0 & \cdots & 0 & -\bg_n^1 & -\bg_n^1
			}
		\end{align*}
		
		\item נתבונן בנוסחאת הנסיגה הבאה: 
		\[ f_{n} = 4f_{n-1} + 7_{n-2} - 10f_{n - 3} \]
		ננסח מחדש: 
		\[ f_{n} =  \]
	\end{enumerate}
	
	\newpage
	
	
\end{document}