%! ~~~ Packages Setup ~~~ 
\documentclass[a4paper]{article}
\usepackage{lipsum}
\usepackage{rotating}


% Math packages
\usepackage[usenames]{color}
\usepackage{forest}
\usepackage{ifxetex,ifluatex,amsmath,amssymb,mathrsfs,amsthm,witharrows,mathtools,mathdots}
\WithArrowsOptions{displaystyle}
\renewcommand{\qedsymbol}{$\blacksquare$} % end proofs with \blacksquare. Overwrites the defualts. 
\usepackage{cancel,bm}
\usepackage[thinc]{esdiff}


% tikz
\usepackage{tikz}
\usetikzlibrary{graphs}

% Design
\usepackage[labelfont=bf]{caption}
\usepackage[margin=0.85in]{geometry}
\usepackage{multicol}
\usepackage[skip=5pt, indent=0pt]{parskip}
\usepackage[normalem]{ulem}
\forestset{default}
\renewcommand\labelitemi{$\bullet$}
%\usepackage{dashrule}
\usepackage{titlesec}
\titleformat{\section}[block]
	{\fontsize{15}{15}}
	{\sen \dotfill \huge \bfseries\thesection \she}
	{0em}
	{}
\titleformat{\subsection}[block]
	{\large\itshape}
	{\normalfont\Large\bfseries\en{{\thesubsection}} \,\,$\sim$\,\,}
	{0em}
	{}
%	[\vspace{-13pt}\rule{\linewidth}{0.25pt}]
\titleformat{\subsubsection}[block]
	{\normalsize\bfseries}
	{\normalfont\large\bfseries\en{{\thesubsubsection}}}
	{1em}
	{}
	
%\titleformat{command}[shape]{format}{label}{sep}{before-code}[after-code]
\usepackage{graphicx}
\graphicspath{ {./} }
\usepackage{hyperref}
\hypersetup{
	colorlinks,
	citecolor=black,
	filecolor=black,
	linkcolor=black,
	urlcolor=blue
}

\usepackage{calc}
\usepackage{fancyhdr}
\pagestyle{fancy}
\newcommand\name{רשימות אלגברה לינארית 2א}
\fancyhead[C]{\textit{\textbf{\name}}}
\fancyhead[L,R]{}
\fancyfoot[R]{\textit{שחר פרץ, 2025}\, \en{{\Large({\thepage})}}}
\fancyfoot[L]{\textit{\rightmark}}
\fancyfoot[C]{}
\renewcommand{\headrule}{\vspace{-7pt}\hfil\rule{200pt}{1pt}}


% Hebrew initialzing
\usepackage[bidi=basic]{babel}
\PassOptionsToPackage{no-math}{fontspec}
\babelprovide[main, import, Alph=letters]{hebrew}
\babelprovide[import]{english}
\babelfont[hebrew]{rm}{David CLM}
\babelfont[hebrew]{sf}{David CLM}
\usepackage[shortlabels]{enumitem}
\newlist{hebenum}{enumerate}{1}

% Language Shortcuts
\newcommand\en[1] {\begin{otherlanguage}{english}#1\end{otherlanguage}}
\newcommand\sen   {\begin{otherlanguage}{english}}
	\newcommand\she   {\end{otherlanguage}}
\newcommand\del   {$ \!\! $}

\newcommand\npage {\vfil {\hfil \textbf{\textit{המשך בעמוד הבא}}} \hfil \vfil \pagebreak}
\newcommand\ndoc  {\dotfill \\ \vfil {\begin{center}
			{\textbf{\textit{שחר פרץ, 2025}} \\
				\scriptsize \textit{קומפל ב־}\en{\LaTeX}\,\textit{ ונוצר באמצעות תוכנה חופשית בלבד}}
	\end{center}} \vfil	}

\newcommand      {\rn}[1]{
	\textup{\uppercase\expandafter{\romannumeral#1}}
}

\newcommand\middleText[1] {\vfil {\hfil {#1}} \hfil \vfil \newpage}
\newcommand\envendproof{\vspace{-16pt}}

\makeatletter
\newcommand{\skipitems}[1]{
	\addtocounter{\@enumctr}{#1}
}
\makeatother

%! ~~~ Math shortcuts ~~~
	
% Letters shortcuts
\newcommand\N     {\mathbb{N}}
\newcommand\Z     {\mathbb{Z}}
\newcommand\R     {\mathbb{R}}
\newcommand\Q     {\mathbb{Q}}
\newcommand\C     {\mathbb{C}}
\newcommand\K     {\mathbb{K}}

\newcommand\ml    {\ell}
\newcommand\mj    {\jmath}
\newcommand\mi    {\imath}

\newcommand\powerset {\mathcal{P}}
\newcommand\ps    {\mathcal{P}}
\newcommand\pc    {\mathcal{P}}
\newcommand\ac    {\mathcal{A}}
\newcommand\bc    {\mathcal{B}}
\newcommand\cc    {\mathcal{C}}
\newcommand\dc    {\mathcal{D}}
\newcommand\ec    {\mathcal{E}}
\newcommand\fc    {\mathcal{F}}
\newcommand\nc    {\mathcal{N}}
\newcommand\vc    {\mathcal{V}} % Vance
\newcommand\sca   {\mathcal{S}} % \sc is already definded
\newcommand\rca   {\mathcal{R}} % \rc is already definded
\newcommand\zc    {\mathcal{Z}}

\newcommand\prm   {\mathrm{p}}
\newcommand\arm   {\mathrm{a}} % x86
\newcommand\brm   {\mathrm{b}}
\newcommand\crm   {\mathrm{c}}
\newcommand\drm   {\mathrm{d}}
\newcommand\erm   {\mathrm{e}}
\newcommand\frm   {\mathrm{f}}
\newcommand\nrm   {\mathrm{n}}
\newcommand\vrm   {\mathrm{v}}
\newcommand\srm   {\mathrm{s}}
\newcommand\rrm   {\mathrm{r}}

\newcommand\Si    {\Sigma}

% Logic & sets shorcuts
\newcommand\siff  {\longleftrightarrow}
\newcommand\ssiff {\leftrightarrow}
\newcommand\so    {\longrightarrow}
\newcommand\sso   {\rightarrow}

\newcommand\epsi  {\epsilon}
\newcommand\vepsi {\varepsilon}
\newcommand\vphi  {\varphi}
\newcommand\Neven {\N_{\mathrm{even}}}
\newcommand\Nodd  {\N_{\mathrm{odd }}}
\newcommand\Zeven {\Z_{\mathrm{even}}}
\newcommand\Zodd  {\Z_{\mathrm{odd }}}
\newcommand\Np    {\N_+}

% Text Shortcuts
\newcommand\open  {\big(}
\newcommand\qopen {\quad\big(}
\newcommand\close {\big)}
\newcommand\also  {\text{, }}
\newcommand\defis {\text{ definitions}}
\newcommand\given {\text{given }}
\newcommand\case  {\text{if }}
\newcommand\syx   {\text{ syntax}}
\newcommand\rle   {\text{ rule}}
\newcommand\other {\text{else}}
\newcommand\set   {\ell et \text{ }}
\newcommand\ans   {\mathscr{A}\!\mathit{nswer}}

% Set theory shortcuts
\newcommand\ra    {\rangle}
\newcommand\la    {\langle}

\newcommand\oto   {\leftarrow}

\newcommand\QED   {\quad\quad\mathscr{Q.E.D.}\;\;\blacksquare}
\newcommand\QEF   {\quad\quad\mathscr{Q.E.F.}}
\newcommand\eQED  {\mathscr{Q.E.D.}\;\;\blacksquare}
\newcommand\eQEF  {\mathscr{Q.E.F.}}
\newcommand\jQED  {\mathscr{Q.E.D.}}

\DeclareMathOperator\dom   {dom}
\DeclareMathOperator\Img   {Im}
\DeclareMathOperator\range {range}

\newcommand\trio  {\triangle}

\newcommand\rc    {\right\rceil}
\newcommand\lc    {\left\lceil}
\newcommand\rf    {\right\rfloor}
\newcommand\lf    {\left\lfloor}

\newcommand\n     {$n$־יה\ }

% Math A&B shortcuts
\newcommand\logn  {\log n}
\newcommand\logx  {\log x}
\newcommand\lnx   {\ln x}
\newcommand\cosx  {\cos x}
\newcommand\sinx  {\sin x}
\newcommand\tanx  {\tan x}

\newcommand\seq   {\overset{!}{=}}
\newcommand\slh   {\overset{LH}{=}}
\newcommand\sle   {\overset{!}{\le}}
\newcommand\sge   {\overset{!}{\ge}}
\newcommand\sll   {\overset{!}{<}}
\newcommand\sgg   {\overset{!}{>}}

\newcommand\h     {\hat}
\newcommand\ve    {\vec}
\newcommand\lv    {\overrightarrow}
\newcommand\ol    {\overline}

\newcommand\mlcm  {\mathrm{lcm}}

\newcommand\dx    {\,\mathrm{d}x}
\newcommand\dt    {\,\mathrm{d}t}

\newcommand\nt[1] {\frac{#1}{#1}}

\newcommand\limz  {\lim_{x \to 0}}
\newcommand\limxz {\lim_{x \to x_0}}
\newcommand\limi  {\lim_{x \to \infty}}
\newcommand\limh  {\lim_{x \to 0}}
\newcommand\limni {\lim_{x \to - \infty}}
\newcommand\limpmi{\lim_{x \to \pm \infty}}

\newcommand\ta    {\theta}
\newcommand\ap    {\alpha}

\renewcommand\inf {\infty}
\newcommand  \ninf{-\inf}

% Combinatorics shortcuts
\newcommand\sumnk     {\sum_{k = 0}^{n}}
\newcommand\sumni     {\sum_{i = 0}^{n}}
\newcommand\sumnko    {\sum_{k = 1}^{n}}
\newcommand\sumnio    {\sum_{i = 1}^{n}}
\newcommand\sumai     {\sum_{i = 1}^{n} A_i}
\newcommand\nsum[2]   {\reflectbox{\displaystyle\sum_{\reflectbox{\scriptsize$#1$}}^{\reflectbox{\scriptsize$#2$}}}}

\newcommand\bink      {\binom{n}{k}}
\newcommand\setn      {\{a_i\}^{2n}_{i = 1}}
\newcommand\setc[1]   {\{a_i\}^{#1}_{i = 1}}

\newcommand\cupain    {\bigcup_{i = 1}^{n} A_i}
\newcommand\cupai[1]  {\bigcup_{i = 1}^{#1} A_i}
\newcommand\cupiiai   {\bigcup_{i \in I} A_i}
\newcommand\capain    {\bigcap_{i = 1}^{n} A_i}
\newcommand\capai[1]  {\bigcap_{i = 1}^{#1} A_i}
\newcommand\capiiai   {\bigcap_{i \in I} A_i}

\newcommand\xot       {x_{1, 2}}
\newcommand\ano       {a_{n - 1}}
\newcommand\ant       {a_{n - 2}}

% Linear Algebra
\DeclareMathOperator{\chr}     {char}
\DeclareMathOperator{\diag}    {diag}
\DeclareMathOperator{\Hom}     {Hom}
\DeclareMathOperator{\Sp}      {span} 
\DeclareMathOperator{\sgn}     {sgn} 
\DeclareMathOperator{\row}     {Row} 
\DeclareMathOperator{\adj}     {adj} 
\DeclareMathOperator{\rk}      {rank} 
\DeclareMathOperator{\col}     {Col} 
\DeclareMathOperator{\tr}      {tr}
\DeclareMathOperator{\lcm}     {lcm}
\DeclareMathOperator{\sym}     {Sym}
\DeclareMathOperator{\asym}    {ASym}

\newcommand\lra       {\leftrightarrow}
\newcommand\chrf      {\chr(\F)}
\newcommand\F         {\mathbb{F}}
\newcommand\co        {\colon}
\newcommand\tmat[2]   {\cl{\begin{matrix}
			#1
		\end{matrix}\, \middle\vert\, \begin{matrix}
			#2
\end{matrix}}}

\makeatletter
\newcommand\rrr[1]    {\xxrightarrow{1}{#1}}
\newcommand\rrt[2]    {\xxrightarrow{1}[#2]{#1}}
\newcommand\mat[2]    {M_{#1\times#2}}
\newcommand\gmat      {\mat{m}{n}(\F)}
\newcommand\tomat     {\, \dequad \longrightarrow}
\newcommand\pms[1]    {\begin{pmatrix}
		#1
\end{pmatrix}}
\newcommand\bms[1]    {\begin{bmatrix}
		#1
\end{bmatrix}}
\newcommand\detms[1]  {\sof{\begin{matrix}
			#1
\end{matrix}}}

\newcommand\genein[1] {\tl \vc_{#1}}
\newcommand\norm[1]   {\left \vert \left \vert #1 \right \vert \right \vert}
\newcommand\snorm     {\left \vert \left \vert \cdot \right \vert \right \vert}
\newcommand\smut      {\left \la \cdot \mid \cdot \right \ra}
\newcommand\mut [2]   {\left \la #1 \,\middle\vert\, #2 \right \ra}
\newcommand{\fRed}     {f^{\mathrm{red}}}

% someone's code from the internet: https://tex.stackexchange.com/questions/27545/custom-length-arrows-text-over-and-under
\makeatletter
\newlength\min@xx
\newcommand*\xxrightarrow[1]{\begingroup
	\settowidth\min@xx{$\m@th\scriptstyle#1$}
	\@xxrightarrow}
\newcommand*\@xxrightarrow[2][]{
	\sbox8{$\m@th\scriptstyle#1$}  % subscript
	\ifdim\wd8>\min@xx \min@xx=\wd8 \fi
	\sbox8{$\m@th\scriptstyle#2$} % superscript
	\ifdim\wd8>\min@xx \min@xx=\wd8 \fi
	\xrightarrow[{\mathmakebox[\min@xx]{\scriptstyle#1}}]
	{\mathmakebox[\min@xx]{\scriptstyle#2}}
	\endgroup}
\makeatother


% Greek Letters
\newcommand\ag        {\alpha}
\newcommand\bg        {\beta}
\newcommand\cg        {\gamma}
\newcommand\dg        {\delta}
\newcommand\eg        {\epsi}
\newcommand\zg        {\zeta}
\newcommand\hg        {\eta}
\newcommand\tg        {\theta}
\newcommand\ig        {\iota}
\newcommand\kg        {\keppa}
\renewcommand\lg      {\lambda}
\newcommand\og        {\omicron}
\newcommand\rg        {\rho}
\newcommand\sg        {\sigma}
\newcommand\yg        {\usilon}
\newcommand\wg        {\omega}

\newcommand\Ag        {\Alpha}
\newcommand\Bg        {\Beta}
\newcommand\Cg        {\Gamma}
\newcommand\Dg        {\Delta}
\newcommand\Eg        {\Epsi}
\newcommand\Zg        {\Zeta}
\newcommand\Hg        {\Eta}
\newcommand\Tg        {\Theta}
\newcommand\Ig        {\Iota}
\newcommand\Kg        {\Keppa}
\newcommand\Lg        {\Lambda}
\newcommand\Og        {\Omicron}
\newcommand\Rg        {\Rho}
\newcommand\Sg        {\Sigma}
\newcommand\Yg        {\Usilon}
\newcommand\Wg        {\Omega}

% Other shortcuts
\newcommand\tl    {\tilde}
\newcommand\op    {^{-1}}

\newcommand\sof[1]    {\left | #1 \right |}
\newcommand\cl [1]    {\left ( #1 \right )}
\newcommand\csb[1]    {\left [ #1 \right ]}
\newcommand\ccb[1]    {\left \{ #1 \right \}}

\newcommand\bs        {\blacksquare}
\newcommand\dequad    {\!\!\!\!\!\!}
\newcommand\dequadd   {\dequad\duquad}

\renewcommand\phi     {\varphi}

\definecolor{myblue}{rgb}{0.2,0.35,0.7}
\definecolor{mygreen}{rgb}{0.15,0.65,0.35}
\definecolor{myyellow}{rgb}{0.0,0.4,0.5}
\definecolor{mycyan}{rgb}{0.05,0.65,0.6}
\definecolor{myred}{rgb}{0.05,0.1,0.75}
\definecolor{mymagenta}{rgb}{0.1,0.7,0.1}

\theoremstyle{definition}
\newtheorem{Theorem}{\color{myblue}משפט}
\newtheorem{Definition}{\color{mygreen}הגדרה}
\newtheorem{Lemma}{\color{myyellow}למה}
\newtheorem{Remark}{\color{mycyan}הערה}
\newtheorem{Notion}{\color{myred}סימון}
\newtheorem{Collary}{\color{mymagenta}מסקנה}

\newcommand\cola [1] {\begin{Collary}#1\end{Collary}}
\newcommand\theo  [1] {\begin{Theorem}#1\end{Theorem}}
\newcommand\defi  [1] {\begin{Definition}#1\end{Definition}}
\newcommand\rmark [1] {\begin{Remark}#1\end{Remark}}
\newcommand\lem   [1] {\begin{Lemma}#1\end{Lemma}}
\newcommand\noti  [1] {\begin{Notion}#1\end{Notion}}

%! ~~~ Document ~~~


\author{שחר פרץ}
\title{\name}
\date{2025 סמסטר ב'}
\begin{document}
	\renewcommand{\footrule}{\rule{\linewidth-19pt}{0.25pt}\vspace{-5pt}}
	\thispagestyle{empty}
	\,\!
	
	{\vspace{0.5\textheight-2em} 
		{
			\begin{center} 
				{
					\textbf{{\name}
				} \\ 
				\textit{שחר פרץ $\sim$ 2025B}}
			\end{center}
		}
	}
	
	\newpage
	\section[מבוא]{\en{Introduction}}
	\subsection{הרבה מילים שאפשר לדלג עליהן}
	סיכום זה לאלגברה לינארית 2א, נעשה במסגרת תוכנית אודיסאה, עם בן בסקין כמרצה. בגלל המון סיבות, כמו זו שאני לא לוקח את הקורס בסמסטר בו אני לומד אותו, והמלחמה עם איראן, הסיכום הזה כולל מגוון מקורות – ההרצאות של בן, הקלטות, סיכומים אחרים, הספר ''Linear Algebra Done Right``, ותרגולים של עומרי שדה־אור. פרט לכך הוספתי ציטוטים מן ההרצאה שמצאתי משעשעים. שכתבתי לחלוטין את הפרק על צורת ג'ורדן, במטרה לשפר את ההבנה של הקורא על הנושא, תוך הצגת שתי גישות להגדרת צורת ג'ורדן. 
	
	כנגד שלושה נושאים דיברה אלגברה לינארית 2א –
	\begin{enumerate}
		\item \textbf{אופרטורים ליניארים}, הן העתקות ממרחב לעצמו. 
		\item \textbf{תבניות בי־ליניאריות}, אובייקט מתמטי נוסף שניתן לייצג ע''י מטריצה. 
		\item \textbf{מרחבי מכפלה פנימית}, מרחבים בהם מוגדרת מעין תבנית ססקווי בי־לינארית שמאפשרת תיאור ``גודל'', ובהם יש ערך לפירוק מטריצות לכפל של מספר מטריצות שונות. 
	\end{enumerate}
	
	נוסף על שלושת הנושאים ה''רגילים'' של הקורס, מופיעה בסוף הרחבה של בן בסקין לגבי מרחבים דואלים. אני ממליץ בחום גם למי שלמד את הנושא בלינארית 1א לקרוא את הפרק עם מרחבים דואלים, משום שהוא קצר, ומראה קשרים חזקים (ומרתקים!) בין החומר הנלמד באלגברה לינארית 2א (כמו מרחבי מכפלה פנימית והעתקות צמודות) למרחבים דואלים. הוספתי בעצמי הרחבה לפירוק SVD. 
	
	באופן אישי, אני מוצא את הקורס די מעניין, ובמיוחד את הפרק האחרון בנוגע לפירוקים של מטריצות/העתקות. 
	
	הגרסה האחרונה של הסיכום תהיה זמינה \href{https://github.com/Sh-Pe/cs-tau-shpe/tree/master/lin2A/megasum}{בקישור הבא} כל עוד מיקרוסופט לא פשטו את הרגל. אם מצאתם בסיכום טעויות (החל בתקלדות, כלה בשגיהוט חטיב, ובטח טעויות מתמטיות) אשמח אם תפנו אלי בטלפון או במייל (perets.shahar@gmail.com), בטלפון (אם יש לכם אותו), או ב־issues ב־github (קישור בתחילת המשפט). 
	
	מקווה שתהנו מהסיכום ותמצאו אותו מועיל; 
	
	\hfill שחר פרץ, 19.7.2025
	
	
	\color{red}\textbf{אזהרה!}\color{black}\, נכון למצב הנוכחי, הסיכום הזה עדיין בעבודה לצריך לעבור הגהה. אתם מוזמנים להשתמש בו, אך אין לראות בו כרגע גרסה סופית ואמינה. 
	
	אני לא אחראי בשום צורה על דברים לא נכונים שכתבתי בטעות. פעמים רבות מרצים מדלגים עם שלבים ואני משלים מההבנה שלי, ולמרות שעברתי על הסיכום ואני משתדל שיהיה מדויק ככל האפשר, ייתכן שישנן טעויות. 
	
	\subsection{סימונים}
	בסיכום הבא נניח את הסימונים הבאים: 
	\begin{itemize}
		\item \hfil $[n] := \N \cap [0, n]$
		\item בהינתן $T \co V \to W$ העתקה ו־$U \subseteq V$ תמ''ו, נסמן $T(U) := \{Tu \mid u \in U\}$
		\item בהינתן $T \co V \to W$ העתקה ו־$v \in V$, נסמן $Tv := T(v)$
		\item בהינתן $A$ קבוצה עם יחס שקילות $\sim$, נסמן את קבוצת המנה ב־$A/_{\sim}$
		\item בפקולטה למתמטיקה בת''א מקובל להשתמש ב־$(v, w)$ בשביל מכפלה פנימית. בסיכום הזה אשתמש ב־$\mut{v}{w}$, גם כן סימון מקובל (בעיקר בפיזיקה), שאני חושב שנראה מגניב הרבה יותר. 
		\item נסמן שחלוף (transpose) ב־$A^{T}$ ולא $A^{t}$. 
		\item הטבעיים כוללים את $0$, ו־$\N_+$ הטבעיים החיוביים אינם. 
	\end{itemize}
	
	\newpage
	\setcounter{tocdepth}{3}
	\tableofcontents
	
	\middleText{\textit{בתיאבון}}
	\section[לכסון]{\en{Diagonalization}}
	\subsection{מבוא לפרק}
	\defi{נאמר ש־$A$ \textit{מטריצה אלכסונית}, אם: 
		\[ A = \pms{\lg_1 & & &\\ &\lg_2 && \\ &&\ddots & \\ &&&\lg_n} =: \diag(\lg_1, \lg_2 \dots \lg_n) \]}
	
	נאמר שישנה פעולה כשהי שנרצה להפעיל. נרצה לקרות מה יקרה לפעולה לאחר שנפעיל אותה מספר פעמים. כפל מטריצות היא פעולה מסדר גודל של $\mathcal{O}(n^3)$. אך, ישנן מטריצות שקל מאוד להעלות בריבוע, ובכך נוכל להפוך את ההליך לפשוט בהרבה, ואף לנסח אותו בצורה של נוסחה סגורה פשוטה. דוגמה מטריצה כזו היא מטריצה אלכסונית. ננסה למצוא דרך ,להמיר'' בין מטריצה ``רגילה'' למטריצה אלכסונית. 
	
	\defi{ו־$T$ ההעתקה אלכסונית: 
		\[ T_A^m(A) = A^m(v) = \pms{\lg_1^m && \\ & \ddots & \\ &&\lg_n^m} \]}
	למה זה מועיל? נזכר בסדרת פיבונצ'י. נתבונן בהעתקה הבאה: 
	\[ \pms{a_{n + 2} \\ a_{n + 1}} = \pms{1 & 1 \\ 1 & 0}\pms{a_{n + 1} \\ a_n} \implies \pms{a_n \\ a_{n - 1}} = \pms{1 & 1 \\ 1 & 0}^{n - 1}\pms{1 \\ 0} \]
	(בהנחת איברי בסיס $a_0 = 0, a_1 = 1$). 
	
	ואכן, מה חבל שלא כיף להכפיל את המטריצה $\binom{1\,1}{1\,0}$ בעצמה המון פעמים. מה נוכל לעשות? ננסה למצוא בסיס שבו $\binom{1 \, 1}{1\, 0}_B = (v_1, v_2)$ ו־$\binom{1 \, 1}{1 \, 0} = P\op \Lambda P$. [המשמעות של $\Lambda$ היא מטריצה אלכסונית כלשהית] אז נקבל: 
	\[ \pms{1 & 1 \\ 1 & 0}^{n} = \cl{P\op \Lambda P}^{n} = P\op\Lambda^nP \]
	(די קל להראות את השוויון האחרון באינדוקציה). במקרה כזה יהיה נורא נחמד כי אין בעיה להעלות לכסינה בחזקה. 
	
	הדבר הנחמד הבא שנוכל ליצור הוא צורת ג'ורדן – מטריצת בלוקים אלכסונית, ככה שנעלה בחזקה את הבלוקים במקום את כל המטריצה. נעשה זאת בהמשך הקורס. 
	
	\defi{\textit{אופרטור ליניארי} (א"ל) הוא ה"ל/טל ממרחב וקטורי $V$ לעצמו. }
	
	
	\subsection{ערכים עצמיים ווקטורים עצמיים לאופרטורים לינארים}
	\defi{יהי $T \co V \to V$ א"ל. אז $0 \neq v \in V$ נקרא \textit{וקטור עצמי} של $T$ (ו"ע) אם קיים $\lg \in \F$ כך ש־$T v =\lg v$. }
	
	\defi{$\lg$ מההגדרה הקודמת נקרא \textit{ערך עצמי} (ע"ע) של $T$, המתאים לו"ע $v$. }
	
	\textbf{שאלה. }יהי $T \co \F^n \to \F^n$. נניח ש־$V = (v_1 \dots v_n)$ בסיס של ו"ע של $\F^n$ [תיאורטית יכול להתקיים באופן ריק כי עדיין לא הראינו שקיים בסיס כזה] אז קיימת $P \in M_n(\F)$ הפיכה כך ש־$A$ המקיימת $Tv = Av$ לפי הבסיס הסטנדרטי, אז $[T]^B_B = \diag(\lg_1 \dots \lg_n)$, כאשר $\lg_1 \dots \lg_n$ ע"ע המתאימים לו"ע $v_1 \dots v_n$. 
	
	כדאי לדעת כי $\Hom(\F^n, \F^n) \approxeq M_{m \times n}(\F)$. מה המשמעות של איזומורפי ($\approxeq$)? בהינתן $A, B$ מבנים אלגברים כלשהם, נסמן $A \approxeq B$ אם קיימת $\phi \co A \to B$ העתקה חח''ע ועל שמשמרת את המבנה (כאשר המבנה שלנו מורכב מפעולות חיבור וכפל, העתקה כזו תהיה ליניארית). 
	
	\textbf{דוגמה. }אם $V, U$ מ"ו מעל $\F$, הם נקראים איזומורפים אם קיימת $\phi \co V \to U$ חח"ע ועל המקיימת
	\[ \forall \lg_1, \lg_2 \in \F\: \forall v_1, v_2 \in V \co \phi(\lg_1v_1 + \lg_2v_2) = \lg_1\phi(v_1) + \lg_2\phi(v_2) \]
	
	כלומר "המרנו`` שני מבנים אלגבריים, אבל לא באמת עשינו שום דבר – כל מבנה עדיין שומר על התכונות שלו. 
	\noti{בסוף הסיכום מופיעה הרחבה על תופעות מעין אלו. }
	
	\defi{יהי $T \co V \to V$ א"ל, נניח $\lg \in \F$ ע"ע, אז המרחב העצמי (מ"ע) של $\lg$ הוא: 
		\[ V_\lg := \{v \in V \mid Tv = \lg v\} \]}
	\theo{$V_\lg$ תמ"ו של $V$. }
	
	\defi{יהי $T \co V \to V$ א"ל, ויהי $\lg \in \F$ ע"ע של $T$. נגדיר את ה\textit{ריבוי הגיאומטרי} של $\lg$ (ביחס ל־$T$) הוא $\dim V_\lg$. }
	
	\textbf{דוגמה.}
	יהי $V$ מ"ו ממימד $n$, $T \co V \to V$ א"ל. נניח קיום $v \in V $ המקיים $T^nv = v$, ונניח $\{v, Tv, T^2v, \dots, T^{n - 1}v\}$ בסיס של $V$. ננסה להבין מהם הע"ע. 
	
	יהי $0 \neq u \in V$ ו"ע כך ש־$Tu = \lg u$. נראה כי $T^nu = u$. ידוע קיום $\ag_0, \dots, \ag_{n - 1} \in \F$ כך ש־$u = \sum \ag_iT^i(v)$. אז: 
	\[ \lg^nu = T^n(u) = \sum_{i = 0}^{n - 1}\ag_i\underbrace{T^{n + i}(v)}_{\mathclap{= T^i(T^nv) = T^iv}} = u \]
	
	נבחין שהוקטורים העצמיים הם שורשי היחידה. מי הם שורשי היחידה – זה תלוי שדה. 
	
	\cola{ערכים עצמיים תלויים בשדה. ערכים עצמיים של מטריצה מעל $\R$ יכולים להיות שונים בעבור אותה המטריצה מעל $\C$. דוגמה יותר פשוטה לכך היא העתקת הסיבוב ב־$\R^2$, שאין לה ו''עים מעל $\R$ אך יש כאלו מעל $\C$. }
	
	\theo{תהי $T \co V \to V$ א"ל, ונניח $A \subseteq V$ קבוצה של ו"ע של $T$ עם ע"ע שונים, אז $A$ בת"ל. הוכחה בתרגול. }
	
	\defi{יהי $T \co V \to V$ א"ל. נאמר ש־$T$ ניתן לכסון/לכסין אם קיים ל־$V$ בסיס של ו"ע של $T$. }
	
	\cola{אם $\dim V = n$ ול־$T$ יש $n$ ע"ע שונים אז $T$ לכסין. }
	
	\rmark{שימו לב – ייתכן מצב בו קיימים פחות מ־$n$ ע"ע שונים אך $T$ עדיין לכסין. דוגמה: $id, 0$. }
	
	\cola{תהי $T \co V \to V$ א"ל. נניח שלכל $\lg$ ע"א, ישנה $B_\lg \subseteq V_\lg$ בת"ל. אז $B = \bigcup_{\lg}B_\lg$ בת"ל. }
	
	\begin{proof}ניקח צירוף לינארי כלשהו שווה ל־0: 
		\begin{align*}
			\sum_{v_i \in B} \ag_iv_i &= 0 \\
			&= \sum_{\lg}\sum_{\lg_i}\ag_iv_{\lg, i} \\
			&\implies \sum_{\lg_j}\ag_i v_{\lg_{ji}} =: u_j \in V_{\lg_j} \\
			&\implies \sum_{j}u_j = 0
		\end{align*}
		קיבלנו צירוף ליניארי לא טרוויאלי של איברים במ"ע שונים (=עם ע"א שונים). אם אחד מהם אינו 0, קיבלנו סתירה למשפט. סה"כ קיבלנו שלכל $j$ מתקיים $\sum \ag_{ji}v_{ji} = 0$. בגלל ש־$v_{ji} \in B_j$ אז בת"ל ולכן כל הסקלרים 0. 
	\end{proof}
	\rmark{ההוכחה הזו עובדת בעבור ההכללה לממדים שאינם נוצרים סופית}
	
	\cola{יהי $T \co V \to V$ א"ל כך ש־$\dim V = n$. אז: 
		\[ \sum_\lg \dim V_\lg \le n \]}
	שוויון אמ"מ $T$ לכסין. 
	
	\begin{proof}
		לכל $\lg$ יהא $B_\lg$ בסיס. אז $B = \sum_{\lg}B_\lg$ בת"ל. אז $n \ge |B| = \sum_\lg \dim V_\lg$ . 
		
		אם $T$ לכסין אז קיים בסיס של ו"ע כך שאכל אחד מהם מבין $V_\lg$ ושוויון. 
		
		מצד שני, אם יש שוויון אז $B$ קבוצה בת"ל של $n$ ו"ע ולכן בסיס ולכן $T$ לכסין. 
	\end{proof}
	
	\subsection{ערכים עצמיים ווקטורים עצמיים למטריצות}
	\defi{תהי $A \in M_n(\F)$. נאמר ש־$0 \neq v \in \F^n$ הוא ו"ע של $A$ עם ע"ע $\lg$ אם $Av = \lg v$. }
	
	\theo{תהי $T \co V \to V$ א"ל ויהי $B$ בסיס סדור, ו־$V$ נוצר סופי (לעיתים יקרא: סוף־ממדי). נניח $A = [T]_B$. אז $v \neq 0$ וקטור עצמי של $T$ עם ערך עצמי $\lg$ אמ"מ $[v]_B$ וקטור עצמי של $A$ עם ע"ע $\lg$. }
	
	\begin{proof}
		גרירה דו־כיוונית. 
		נניח $V$ ו"ע של $T$. אז $A[v]_B = [Tv]_B = [\lg_v]_B \lg[v]_B$. מהכיוון השני "לכו הפוך". 
	\end{proof}
	
	\defi{מטריצה $A \in M_n(\F)$ תקרא לכסינה/נתנת ללכסון אם היא דומה למטריצה $\Lambda \in M_n(\F)$ אלכסונית, כלומר קיימת $P \in M_n(\F)$ הפיכה שעבורה $\Lg = P\op AP$. }
	
	\theo{יהיו $A, P \in M_n(\F)$. נניח $P$ הפיכה. אז אם $P\op AP = \diag(\lg_1, \dots, \lg_n)$ אמ"מ עמודות $P$ הן ו''ע של $A$ עם ע"ע $\lg_1 \dots \lg_n$ בהתאמה. }
	
	\begin{proof}
		נסמן $P = (P_1 \dots P_n)$ עמודותיה. אז: 
		\[ AP = (AP_1 \dots AP_n) = (\lg_1P_1 \dots \lg_nP_n) = P\underbrace{\pms{\lg_1 && \\ & \ddots & \\ && \lg_n}}_{\Lg} \]
		ולכן: 
		\[ P\op AP = P\op P\Lg \]
		ההוכחה מהכיוון השני היא לקרוא את זה מהצד השני. 
	\end{proof}
	
	"אני מקווה שראיתם שכפל מטריצה באלכסונית מתחלף" (הוא לא בהכרח). 
	"אני אמרתי שטות". $\sim$ בן 
	
	\theo{בהינתן העתקות $T, S$ שתיהן לכסינות לפי אותו הבסיס $B$ (לא בהכרח אותם הע''עים), אז $TS = ST$ מתחלפות. }
	\theo{המטריצה $\lg I$ עבור $\lg \in \F$ דומה רק לעצמה. }\begin{proof}
		בהינתן $P$ הפיכה, הכפל של $P$ עם $\lg I$ מתחלף בהכרח, ולכן $P\op \lg I P = P P\op \lg I = \lg I$ לכל מטריצה $P\op \lg I P$ דומה. 
	\end{proof}
	
	\subsection{פולינום אופייני}
	\textbf{תרגיל. }תהי $A = \binom{-7 \, 8}{\,\,6 \, 7}$. מצאו ו"ע וע"ע של $A$ ולכסנו אם אפשר. 
	
	\textbf{פתרון. }מחפשים $\binom{0}{0} \neq \binom{x}{y} \in \R^2$ ו־$\lg \in \R$ כך ש־: 
	\begin{align*}
		A\pms{x \\ y} &= \lg \pms{x \\ y} \\
		A\pms{x \\ y} &= \lg I \pms{x \\ y} \\
		\lg I \pms{x \\ y} - A \pms{x \\ y} &= 0 \\
		\pms{\lg I - A} \pms{x \\ y} = 0
	\end{align*}
	סה"כ $\binom{x}{y}$ ו"ע עם ו"ע $\lg$ אמ"מ $\binom{x}{y} \in \ker (\lg I - A)$, אמ"מ $\lg I - A$ לא הפיכה, אמ"מ $\det(\lg I - A) = 0$ (AKA "הפולינום האופייני"). במקרה הזה: 
	\[ \lg I - A = \pms{\lg + 7 & -8 \\ -6 & \lg - 7} \implies \det(\lg I - A) = (\lg + 7)(\lg - 7) = \lg^2 - 1 \]
	לכן הע"ע הם $\pm 1$. נמצא את הו"ע. עבור $\lg = 1$, מתקיים: 
	\[ \pms{8 & 8 \\ -6 & -6}\pms{x \\ y} = 0 \]
	יש לנו חופש בחירה (ופתרון יחיד עד לכדי כפל בסקלר). במקרה הזה, נבחר $\binom{x}{y} = \pms{1 \\ -1}$. 
	
	עבור $\lg = -1$, יתקיים: 
	\[ \pms{6 & 8 \\ -6 & 8}\pms{x \\ y} = 0 \impliedby b\pms{x \\ y} = \pms{4 \\ -3} \]
	ראינו שהמלכסנת היא העמודות של הו"ע. אז: 
	\[ P = \pms{1 & 4 \\ -1 & -3} \]
	וסה"כ $P\op A P = I$. מכאן צריך למצוא את $P\op$. 
	
	\theo{תהי $A \in M_n(\F)$ אז $\lg \in \F$ ע"ע של $A$ אמ"מ $|\lg I - A| = 0$. }
	
	\defi{תהי $A \in M_n(\F)$. \textit{הפולינום האופייני} של $A$ מוגדר להיות: 
		\[ f_A(x) = |xI - A| \]}
	
	\theo{תהי $A \in M_n(\F)$. אז $f_A(x)$ הוא פולינום מתוקן [=מקדם מוביל הוא 1] ממעלה $n$, המקדם של $x^{n - 1}$ הוא $- \tr A$ והמקדם החופשי הוא $(-1)^{n}|A|$. }
	
	\defi{בעבור $A \in M_n(\F)$ הפולינום האופייני של $A$ הוא $f_A(x) = \det(Ix - A)$. }
	
	ראינו ש־$v$ ו"ע של $A$ עם ערך עצמי $\lg$ אמ"מ $v \in \ker(\lg I - A)$, וכן $\lg$ ע"ע אמ"מ $\dim \ker \lg - A > 0$. 
	
	\theo{$f_A(x)$ פולינום מתוקן (מקדם מוביל 1) מדרגה $n$, המקדם של $x^{n - 1}$ הוא $-\tr A$, המקדם החופשי הוא $(-1)^{n}\det A$. }
	
	\begin{proof}\,
		\begin{itemize}
			\item\textbf{ תקינות הפולינום. }מבין $n!$ המחוברים, ישנו אחד יחיד שדרגתו היא $n$. הסיבה היא שמדטרמיננטה לפי תמורות, התמורה היחידה שתיצור איבר מסדר $x^n$ היא תמורת הזהות שתעבור על האלכסון. באינדוקציה על $n$, ונקבל: 
			\[ f_A(x) = (x - a_{11}) \cdots (x - a_{nn}) = x^{n} + \cdots \]
			באופן דומה אפשר להוכיח באינדוקציה באמצעות פיתוח לפי שורות: 
			\[ f_A(x) = (x - a_{11})|A_11| + \underbrace{a_{21}|A_21| - a_{31}|A_31| + \cdots + (-1)^{n + 1}a_{n1}|A_{n1}|}_{\,\!\text{מה.א. דרגה קטנה מ־$n$}} \]
			סה"כ גם כאן הראינו שהדרגה מתקבלת מהפולינום $\prod^{n}_{i = 1}(x - a_{ii})$, כלומר הפולינום האופייני מתוקן. 
			
			\item\textbf{המקדם של $x^{n - 1}$ הוא $\bm{-\tr A}$. } מקדמי $x^{n - 1}$ מגיעים גם הם רק מ־$\prod^{n}_{i = 1}(x - a_{ii})$ (הפולינום למעלה) שהם $-\tr A = \sum_{i = 1}^{n}-a_{ii}$ המקדם החופשי. 
			\item\textbf{המקדם החופשי. }מתקבל מהצבת $0$. $f_A(0) = \det(I \cdot 0 - A) = \det(-A) = (-1)^{n}\det A$. 
		\end{itemize}
	\end{proof}
	
	\textbf{דוגמאות. }\begin{enumerate}[A)]
		\item אם $A = \binom{a \, b}{c \, d}$ אז $f_A(x) = x^{2} - (a + d)x + ad - bc$ (נטו מהמשפט הקודם). 
		\item אם $A = \diag(\lg_1 \dots \lg_n)$ אז $f_A(x) = \prod_{i = 1}^{n}(x - \lg_i)$. 
		\item אם $A = \pms{\lg_1 & & * \\ & \ddots & \\  0 && \lg_n}$ אז גם כאן $f_A(x) = \prod_{i = 1}^{n}(x - \lg_i)$ אך כדאי לשים לב שמשולשית עליונה לא בהכרח דומה לאלכסונית עם אותם הקבועים. 
		\item אם $A = \pms{B & * \\ 0 & C}$ כאשר $B, C$ בלוקים ריבועיים אז $f_A(x) = f_B(x) \cdot f_C(x)$. 
	\end{enumerate}
	
	\defi{בהינתן $T \co V \to V$ ט"ל נגדיר את הפולינום האופייני שלה (פ"א) באופן הבא: נבחר בסיס $B$ למ"ו $V$, ונתבונן ב־$A = [T]_B$ ונגדיר את $f_T(x) := f_A(x)$. }
	
	"אתה פותר עכשיו שאלה משיעורי הבית`` "אל תדאג הבודק כבר שלח פתרון`` "מה?!``
	
	\theo{הפ"א של ט"ל מוגדר היטב. למטריצות דומות אותו פ"א. \textit{הוכחה. }בשיעורי הבית. ויש סיכוי שגם בדף הסיכום. }
	
	\textbf{דוגמה. }נתבונן בהעתקה $\R_n[x] \to \R_n[x], \ T(f) = f'$. נבחר בסיס $B = (1, x, \dots, x^{n})$. אז: 
	\[ [T]_B = \pms{0 & 1  & 0 & 0 & \dots & 0 \\ 0 & 0 & 2 & 0 & \dots & 0 \\ 0 & 0 & 0 & 3 & \ddots & \vdots \\ \vdots & \vdots & \vdots & \ddots & \ddots \\ 0 & 0 & 0 & 0 } \]
	אז: 
	\[ f_T(x) = \det\pms{x & -1 & 0 &\dots \\ & x & -2 & 0 &\dots \\ && \ddots &\ddots } = x^{n + 1} \]
	\theo{$T \co V \to V$ ט"ל, אז $\lg$ ע"ע של $T$ אמ"מ $f_T(\lg) = 0$. }
	\begin{proof}
		יהא $B \subseteq V$ בסיס של $V$. אז $A = [T]_B$ ואז $f_T(\lg) = 0$ אמ"מ $f_A(\lg) = 0$ אמ"מ $\lg$ ע"ע של $A$. 
	\end{proof}
	
	\defi{יהי $\lg \in \F$ ע"ע של $T$ (או $A$). \textit{הריבוי האלגברי} של $\lg$ הוא החזקה המקסימלית $d$ כך ש־$(x - \lg)^{d} \mid f_T(x)$ (חלוקת פולינומים). }
	
	\textbf{דוגמה. }בעבור $T$ היא העתקת גזירת פולינום, הפ''ע $f_T(x) = x^{n + 1}$ ולכן ע"ע יחיד הוא $0$. הריבוי האלגברי של $0$ הוא $n + 1$. הריבוי הגיאומטרי של $0$ הוא $1$. 
	
	\noti{נניח ש־$\lg$ ע"ע של $T$ (או $A$) אז $d_\lg$ הריבוי האלגברי של $\lg$ ו־$r_\lg$ הריבוי הגיאומטרי של $\lg$. }
	\subsection{על הקשר בין ריבוי גיאומטרי ואלגברי}
	
%	TODO: understand wtf
%	\theo{נניח שפ"א של $T$ או $A$: 
%		\[ f_T(x) = \prod_{i = 1}^{k}(x - \lg_i)^{n_i} \]
%		אז $d_{\lg_i} = n_i$ הריבוי האלגברי. על כן, נבחין כי $n_i \le d_{\lg_i}$. }
%		\begin{proof}
%		
%		\[ (x - \lg_i)^{n_i} \mid f_T(x) \implies f_T(x) = (x - \lg_i)^{n_i} \prod_{\overset{j \in [k]}{j \neq i}}(x - \lg_j)^{n_j} \]
%		נניח בשלילה $d_{\lg_i} \ge n + 1$. אז: 
%		\[ f_T(x) = \cdots = (x - \lg_i)q(x) \]
%		נעביר אגפים מהשוויונות השונים ונוציא גורם משותף: 
%		\[ (x - \lg_i)^{n}\Big(\overbrace{\prod_{\overset{j \in [k]}{j \neq i}}(x - \lg_j)^{n_j} - (x - \lg_i)q(x)}^{:=P(x)}\Big) = 0 \]
%		נדע כי $P(x)$ אינו פולינום האפס כי: 
%		\[ P(\lg_i) = \prod_{\overset{j \in [k]}{j \neq i}}(\lg_i - \lg_j)^{n_j} \]
%		שוויון בשדה $\F(x)$. וברור כי $(x - \lg_i)^{n}$ אינו פולינום האפס. אך אחד מהם הוא אפס משום שכפל שני איברי שדה שווה לאפס אמ"מ אחד מהם הוא אפס, וסתירה. 
%		\end{proof}
	\rmark{בדוגמה שבטענה ראינו שמתקיים $\sum d_i = \sum n_i = n$ כאשר $n$ דרגת הפולינום. זה לא תמיד המצב. 
	
	דוגמה למצב בו זה לא קורה: $x^2(x^2 + 1) \in \R[x]$. סכום הריבויים האלגבריים הוא 2, אבל דרגת הפולינום היא 4. זה נכון מעל שדות סגורים אלגברית. }
	
	
	\theo{תהי $T \co V \to V$ ט"ל. אזי לכל ע"ע $\lg$ מתקיים $r_\lg \le d_\lg$. }
	\begin{proof}
		יהי $\lg$ ע"ע. אז $V_\lg = \{v \in V \mid Tv = \lg v\}$. יהי $B_\lg \subseteq V_\lg$ בסיס עבור $V_\lg$. נשלים אותו לבסיס $B$ של $V$. 
		\[ [T]_B = \pms{\lg & & 0 & * \\ & \lg &  & \\ 0 && \ddots & \\ *&&& C} \]
		ואז: 
		\[ f_T(x) = (x - \lg)^{r_\lg}C(x) \implies r_\lg \le d_\lg \]
	\end{proof}
	
	\theo{תהי $T \co V \to V$ ט"ל עם פ"א $f_T(x)$. אז $T$ לכסינה אמ"מ שתי הטענות הבאות מתקיימות:
		\begin{enumerate}
			\item בעבור $k$ הע"ע שונים, $f_T(x) = \prod_{i = 1}^{k}(x - \lg_i)^{n_i}$
			\item לכל $\lg$ ע"ע של $T$ מתקיים $r_\lg = d_\lg$
		\end{enumerate}
	}
	(הבהרה: 1 לא גורר את 2. צריך את שניהם). 
	
	\begin{proof}\,
		\begin{itemize}
			\item[$\impliedby$]$T$ לכסינה ראינו ש־1 מתקיים. במקרה שלכסינה ראינו ש־$n = \sum r_{\lg_i} \le \sum d_{\lg_i} = n$ ולכן אם לאחד מבין הערכים העצמיים מתקיים $r_\lg \neq d_k$ אז מתקיים $r_k < d_k$ ונקבל סתירה לשוויונות  לעיל. 
			\item[$\implies$] 
			\begin{align*}
				1 &\implies \sum d_{\lg_i} = n \\
				2 &\implies \sum r_{\lg_i} = \sum d_{\lg_i} = n
			\end{align*}
		\end{itemize}
		וסה"כ $\sum r_{\lg_i} = n$ אמ"מ $T$ לכסינה. 
	\end{proof}
	
	\subsubsection{פיבונאצ'י בשדה סופי}
	סדרת פיבונאצ'י: 
	\[ \pms{a_{n + 1} \\ a_n} = {\underbrace{\pms{1 & 1 \\ 1 & 0}}_{ = 0}}^{n}\pms{1 \\ 0} \]
	נניח שאנו מסתכלים מעל $\F_p$ כלשהו. אז הסדרה חייבת להיות מחזורית. \textbf{שאלה: }מתי מתקיים ש־$A^m = I$ (בעבור $m$ מינימלי)? במילים אחרות, מתי מתחילים מחזור. 
	
	היות שמספר הזוגות השונים עבור $\pms{a_{n + 1} \\ a_n}$ הוא $p^2$, אז $m \le p^2$. עבור $p = 7$: \hfill $0, 1, 1, 2, 3, 4, 5, 1, 6, 0, 6, 6, 5, 4, 2, 6, 1, 0, 1$ – כלומר עבור $p = 7$ יש מחזור באורך $m = 16$.
	\rmark{תירואטית עם המידע הנוכחי ייתכן ויהפוך למחזורי ולא יחזור להתחלה}
	
	\textbf{טענה. }אם $p$ ראשוני אז $p \equiv 1 \pmod 5$ אז אורך המחזור חסום מלעיל ע"י $p - 1$. 
	
	\begin{proof}
		תנאי מספיק (אך לא הכרחי) לקבלת מחזור באורך $k$ הוא $A^k = I$. אז: 
		\[ f_A(x) = x^2 - x - 1 \]
		יש דבר שנקרא "הדדיות ריבועית" (חומר קריאה רשות במודל) שמבטיחה שורש לפולינום להלן עבור $p$ כנ"ל. אכן יש לנו שני ע"ע שונים (אם קיים רק אחד אז סתירה מהיות הדיסקרימיננטה $5 = 0$ אך $p \not\equiv 1 \pmod 5$). לכן קיימת $P$ הפיכה כך ש־: 
		\[ P\op AP = \pms{\lg_1 & 0 \\ 0 & \lg_2} \]
		כך ש־$\lg_1, \lg_2 \neq 0$. משפט פרמה הקטן אומר ש־$\lg_1^{p - 1}  = \lg_2^{p - 1} = 1$. ואז $A^{p - 1} = I$. 
	\end{proof}
	
	\subsection{שילוש}
	\defi{$T \co V \to V$ ט"ל ניתנת לשילוש אם קיים בסיס $B$ ל־$V$ כך ש־$[T]_B$ משולשית. }
	
	\rmark{אם $T$ ניתנת לשילוש אז הפולינום האופייני שלה מתפרק לגורמים ליניארים (האם איברי האלכסון של הגרסה המשולשית). יהיה מעניין לשאול אם הכיוון השני מתקיים. }
	
	\theo{$T \co V \to V$ ט"ל. נניח ש־$f_T(x) = \prod_{i = 1}^{n}(x - \lg_i)$ (ניתנת לפירוק לגורמים ליניאריים) אז $T$ ניתנת לשילוש. }
	\begin{proof}
		\textit{בסיס. }$n = 1$ היא כבר משולשית וסיימנו. \\
		\textit{צעד. }נניח שהטענה נכונה בעבור $n$ טבעי כלשהו, ונראה נכונות עבור $n + 1$. אז $f_T$ מתפרק לגורמים ליניאריים, לכן יש לו שורש. יהי $\lg$ ע"ע של $T$. בסיס $B$ של $V$ מקיים ש־$[T]_B$ משולשית עליונה (נסמן $B = (w_1 \dots w_{n + 1})$) $\iff$ אז $T(w_i) \in \Sp(w_1 \dots w_i)$. נגדיר את $w_1$ להיות ו"ע של $\lg$. נשלימו לבסיס $B^1$. 
		\[ [T]_B = \pms{\lg & & * && \\ 0 && \vdots \\ \vdots &\cdots & C & \cdots \\ 0 & &\vdots } \]
		אז ניתן לומר כי: 
		\[ f_T(x) = (x - \lg) f_C(x) \]
		נסמן $w = \Sp(w_2 \dots w_{n + 1})$. קיימת העתקה ליניארית $S \co W \to W$ כך ש־$f_S(x) = f_C(x)$. לפי ה"א קיים בסיס ל־$W$ הוא $B''$ שעבורו $S$ משולשית עליונה. נטען ש־$B = B'' \cup \{w_1\}$ ייתן את הדרוש. 
		\[ \forall w \in B'' \co (T - S)(w) = Tw - Sw = aw_1 + S(w) - S(w) = aw_1 \]
		(כלומר, השורה העליונה של $[T]_B$ "תרמה" את $aw_1$ בלבד)
		לכן: 
		\[ (T - S)w \subseteq \Sp(w_1) \]
		זה גורר שלכל $w \in W$ מליניאריות מתקיים ש־$(T - S)w \subseteq \Sp(w_1)$. סה"כ לכל $w \in B'' \cup \{w_1\}$ מתקיים $T(w_i) \in \Sp(w_1 \dots)$. 
	\end{proof}
	
	בהוכחה הזו, בנינו בסיס כך ש־: 
	\[ [T(w) - S(w)]_B = ae_1 \]
	
	\defi{מטריצה \textit{ניתנת לשילוש} אם היא  דומה למשולשית. }
	\theo{מטריצה $A$ ניתנת לשילוש, אמ"מ הפ"א האופייני שלה מתפצל לגורמים לינארים. }
	
	\renewcommand{\footrule}{\rule{\linewidth-26pt}{0.25pt}\vspace{-5pt}}
	\npage
	\subsection{על ההבדל בין פולינום לפולינום}
	נבחין ש־$\F[x]$ הוא מ"ו מעל $\F$. וכן $\F[x]$ הוא חוג חילופי עם יחידה. בחוג כפל לא חייב להיות קומטטיבי (נאמר, חוג המטריצות הריבועיות). אומנם קיימת יחידה (פולינום קבוע ב־1) אך אין הופכיים לשום דבר חוץ מלפונ' הקבועות. שזה מאוד חבל כי זה כמעט שדה. בהמשך, נגדיר את אוסף הפונקציות הרציונליות כדי להתגבר על כך. 
	
	אם נתבונן במטריצות דומות, יש הבדל בין להגיד $f_A(x) \in \F[x]$, אך אפשר לטעון $f_A(x) = |B|$ כש־$B \in M_n(\F(x))$. למה? כי $xI - A \in M_n(\F(x))$ (זה קצת מנוון כי איברי המטריצה הם או פולינומים קבועיים או ממעלה 1). משום שדטרמיננטה שולחת איבר לשדה, אז $|B| \in \F(x)$. כך למעשה נגיע לכך שפולינומים אופייניים שווים כשני איברים בתוך השדה, ולא רק באיך שהם מתנהגים ביחס לקבועיים. 
	
	דוגמה: 
	\[ A = \pms{0 & 1 \\ 0 & 0} \in M_2(\F_2), \ f(x) = x^3, \ g(x) = x, \ f, g \in \F_2 \to \F_2 \implies f = g \]
	אך: 
	\[ f(A) = A^3 = 0, \ g(A) = A \neq 0 \]
	זה לא רצוי. נבחין בשני שוויונות שונים – שוויון פונקציות, בהם $f = g$ מעל $\F_2$, ושוויון בשדה – בו $f -g \neq 0$ (כי $-x^2$ לא פולינום האפס, ואף מעל $\F_2$) ולכן ב־$\F_2(x)$ מתקיים $f \neq g$. 
	
	\subsection{משפט קיילי־המילטון}
	
	\defi{יהי $f(x) = \sum_{i = 0}^{d}a_ix^i \in \F[x]$, $V$ מ"ו מעל $\F$ נ"ס (נוצר סופית) וכן $T \co V \to V$ ט"ל. נגדיר: 
		\[ f(T) = \sum_{i = 0}^{d}a_iT^{i}, \ T^0 = id, \ T^{n} = T \circ T^{n - 1} \]
	כנ"ל עם מטריצות (ראה תרגול)
	}

	
	\textbf{טענה. }אם $A = [T]_B$ ו־$f(x) \in \F[x]$, אז $[f(T)]_B = f(A)$. הוכחה נובעת מהתכונות $[TS]_B = AC, \ [T + S]_B = A + C, \ [\ag T]_B = \ag A, \ [S]_B = C<\ [T]_B = A$. 
	
	\textbf{טענה. }אם $f, g \in \F[x]$ ו־$T \co V \to V$ ט"ל, אז $(f \cdot g)(T) = f(T) \cdot g(T)$. באופן דומה $(f + g)(T) = f(T) + g(T)$. 
	
	לכן קל לראות ש־$f(T) = 0 \iff f(A) = 0$. 
	
	\cola{אם $A, C$ דומות אז $f(A) = 0 \iff f(C) = 0$. }
	
	\textbf{דוגמה. }(מנוונת) נתבונן ב־$D \co \F_n[x] \to \F_n[x]$ אופרטור הגזירה. ראינו $f_D(x) = x^{n + 1}$ (הפולינום האופייני). אז נקבל: 
	\[ f_D(D)(p) = p^{(n + 1)} = 0 \implies f_D(D) = 0 \]
	
	\begin{Theorem}[משפט קיילי־המילטון]
		תהי $T \co V \to B$ ט"ל ($V$ נ"ס) או $A \in M_n(\F)$, ו־$f_A(x) = f_T(x)$ הפ"א, אז $f_T(T) = 0, \ f_A(A) = 0$. 
	\end{Theorem}
	
	\rmark{באנגלית: Cayley–Hamilton}
	
	\begin{proof}
		נוכיח את המשפט בשלושה שלבים – 
		\begin{itemize}
			\item נניח ש־$T$ ניתנת לשילוש. אזי, קיים בסיס $B =: (v_1 \dots v_n)$ כך ש־$[T]_B$ משולשית (עליונה). זאת מתקיים אמ"מ $\forall i \in [n] \co Tv_i \in \Sp(v_1 \dots v_i)$. נפנה להוכיח את משפט קיילי־המילטון למקרה זה. 
			
			\begin{proof}[תת־הוכחה.]\,
				\begin{itemize}
					\item \textit{בסיס:} בעבור $n = 1$, אז קיים $\lg \in \F$ כך ש־$f_T(T) = T - \lg I = 0$ (העתקה לינארית חד ממדית היא כפל בסקלר). בפרט $\forall v \in V \co (T - \lg)v = 0$. 
					\item \textit{צעד:} נניח ש־$B = (v_1 \dots v_n, v_{n + 1})$ שעבורו $[T]_B$ משולשית. נגדיר תמ"ו $W = \Sp(v_1 \dots v_n)$ כך ש־$\dim W \le \dim V$, ועבורו נכון $\forall w \in W \co Tw \in W$ (ניתן להראות שזה נכון עבור וקטורי הבסיס, ונכון לכל $w \in W$ מלינאריות). נגדיר $T|_W \co W \to W$ את הצמצום של $T$ ל־$W$. ידוע ש־$T|_w$ ניתנת לשילוש ולכן מקיימת את תנאי האינדוקציה. לכן, $\forall w \in W\co f_{T|_W}(T)(w) = 0$. אזי $f_{T|_W}(x) = \prod_{i = 1}^{n}(x - \lg_i)$ וסה"כ $f_{T}(x) = (x - \lg_{n + 1})f_{T|_W}(x)$ וקיבלנו $\forall w \in W \co f_T(T)(w) = 0$. 
					
					מספיק להראות ש־$\forall v \in V \co (T - \lg_{n + 1})v \in W$. למה? כי: 
					\[ f_T(T)(v) = \cl{\prod_{i = 1}^{n}(T - \lg_i)}(T - \lg_{n + 1})(v) \]
					מלינאריות, מספיק להראות ש־$(T - \lg_{n + 1})(v_{n + 1})\in W$, שכן זה מתקיים על כל בסיס אחר. אך זה ברור – עבור $[T]_B$ העמודה האחרונה היא: 
					\[ \pms{\ag_1 \\ \vdots \\ \ag_n \\ \lg_{n +1}} \]
					ולכן: 
					\[ T(v_{n + 1}) = \ag_1 v_1 + \cdots + \ag_n v_n + \lg_{n + 1} v_{n + 1} \implies (T - \lg_{n + 1})(v_{n + 1}) = \ag_1v_1 + \cdots + \ag_n v_n \in W \]
				\end{itemize}
			\end{proof}
			\item נוכיח בעבור מטריצה משולשית/ניתנת לשילוש. \begin{proof}[תת־הוכחה.]
				אם $A$ משולשית, אז $f_A(x) = f_{T_A}(x)$ כאשר $T_A \co \F^n \to \F^n$ המוגדרת ע"י $T_A(v) = Av$, ואז $T_A$ ניתנת לשילוש וסיימנו. 
				
				אם $A$ ניתנת לשילוש, אז היא דומה למשולשית, והן בעלות אותו הפולינום האופייני. אז חזרה לתחילת ההוכחה. 
			\end{proof}
			\item עבור $T$ כללית או $A$ כללית. \begin{proof}[תת־הוכחה. ]
				נניח $A = [T]_B$ עבור בסיס $B$, וידוע $f_T(x) = f_A(x)$. ידוע ש־$A$ ניתנת לשילוש אמ"מ $f_A(x)$ מתפצל. \textit{טענה מהעתיד הלא רחוק: }לכל שדה $\F$ קיים שדה $\F \subseteq \K$ סגור אלגברית (כל פולינום מעל שדה סגור אלגברית מפתצל). על כן, ניתן לחשוב על $A \in M_n(\F)$ כמו $A \in M_n(\K)$. הפולינום האופייני מעל $K$ הוא אותו הפולינום האופייני מעל $\F$. לכן הוא מתפצל (מעל $\K$), ולכן הוא דומה למשולשית, ומהמקרה הראשון $f_A(A) = 0$. זאת כי $f_A(A)$ לא תלוי בשדה עליו אנו עובדים, וסה"כ הוכחנו בעבור מטריצה כללית, ולכן לכל ט"ל. 
			\end{proof}
		\end{itemize}
	\end{proof}
	
	\theo{אם $A$ מייצגת של העתקה $T$, ו־$f \in \F[x]$, אז $f(A) = 0 \iff f(T) = 0$. }
	
	\begin{Remark}[בנוגע לשדות סגורים אלגברית]
		הטענה שלכל שדה יש שדה שסגור אלגברית – טענה שתלויה באקסיומת הבחירה. הסגור האלגברי הוא יחיד. \textbf{הטענה הזו לא נאמרת באופן רשמי בקורס} על אף שהרחבה לשדה סגור אלגברית מועילה מאוד בלינארית 2א באופן כללי. 
	\end{Remark}
	
	\npage
	\section[תורת החוגים]{\en{Ring Theory}}
	\subsection{מבוא והגדרות בסיסיות}
	אז, מה זה אובייקט אלגברי? הרעיון – "Data עם אקסיומות". אנו כבר מכירים רבים מהם: תמורות, חבורות, שדות, מרחבים וקטורים, ועוד. עתה נכיר אובייקט אלגברי בשם \textit{חוג}. 
	
	\defi{\textit{חוג עם יחידה} הוא קבוצה עם שתי פעולות, כפל וחיבור, ניטרלים לפעולות (0, 1) כך שמתקיימות כל אקסיומות השדה למעט (פוטנציאלית) קיום איבר הופכי, וקומטטיביות הכפל. }
	
	אנחנו נתעניין ספצפית בחוגים קומטטיבים, כלומר, בהם הכפל כן קומטטיבי. המטריצות הריבועיות מעל אותו הגודל, לדוגמה, הוא חוג שאיננו קומטטיבי. החוג ה"בסיסי ביותר" – חוג השלמים (אין הופכי) הוא חוג קומוטטיבי. ישנם חוגים בלי יחידה (לדוגמה הזוגיים בלי יחידה), שלא נדבר עליהם כלל. 
	
	
	\defi{תחום שלמות הוא חוג קומוטטיבי עם יחידה ללא מחלקי $0$. }
	\defi{חוק ייקרא \textit{ללא מחלקי 0} אם: \hfill $\forall a, b \in \R \co ab = 0 \implies a = 0 \lor b = 0$}
	
	דוגמאות לחוגים עם מחלקי $0$: 
	\begin{itemize}
		\item $M_2(\R)$: הוכחה $a = b = \binom{0\, 1}{0\, 0}, \ a \cdot b = 0$
		\item $\Z/_{6\Z}$ הוכחה $2 \cdot 3 = 0$. 
	\end{itemize}
	
	\theo{בתחום שלמות יש את כלל הצמצום בכפל: אם $ab = ac \land a \neq 0$ אז $b = c$. }\begin{proof}
		\[ ab  \cdot ac = 0 \implies a(b \cdot c) = 0 \implies a = 0 \lor b - c = 0 \]
		בגלל ש־$a \neq 0$, אז $b - c = 0$. נוסיף את $c$ הנגדי של $-c$ ונקבל $b = c$. 
	\end{proof}
	
	דוגמאות לתחום שלמות: 
	\begin{itemize}
		\item שדות
		\item השלמים
		\item חוג הפולינומים
	\end{itemize}
	
	\subsection{ראשוניות ואי־פריקות}
	\defi{יהי $R$ תחום שלמות, $a, b \in R$. נאמר ש־$a \mid b$ אם קיים $c \in \R$ כך ש־$ac = b$. }
	\defi{$u \in R$ נקרא \textit{הפיך} אם קיים $\ag \in R$ כך ש־$\ag u = 1$. }
	\theo{יהי $R$ תחום שלמות, $u \in R$ הפיך. יהי $a \in \R$. אז $u \mid a$. }\begin{proof}
		$1 \mid a, \ u \mid 1$. יחס החלוקה טרנזטיבי ולכן $u \mid a$. 
	\end{proof}
	\noti{קבוצת ההפיכים מוסמנת ב־$R^x$. }
	\textbf{דוגמאות. }
	\begin{enumerate}
		\item אם $R = \F$, אז $\F^x = \F \setminus \{0\}$
		\item אם $R = \Z$ אז $\Z^2 = \{\pm 1\}$
		\item אם $R = \F[x]$ אז $R^x = \F^x$ (ההתייחסות לסקלרים $\F$ היא כאל פונקציות קבועות)
	\end{enumerate}
	\defi{$a, b \in R$ נקראים \textit{חברים} אם קיים $u \in R^x$ הפיך כך ש־$a = ub$, ומסמנים $a \sim b$ }
	
	\theo{יחס החברות הוא יחס שקילות. }
	\begin{proof}\,
		\begin{enumerate}[A.]
			\item $a \sim a$ כי $1 \in R^x$
			\item אם $a \sim b$ אז קיים $u \in R^x$ כך ש־$a = ub$. קיים ל־$u$ הופכי $\ag$ אז $\ag a + \ag u b = b$ ולכן $b \sim a$. 
			\item נניח $a \sim b \land b \sim c$, כי מכפלת ההופכיים הפיכה $a \sim c$ וסיימנו. 
		\end{enumerate}
	\end{proof}
	"אני אני חבר של עומר, ועומר חבר של מישהו בכיתה שאני לא מכיר, אני לא חבר של מי שאני לא מכיר." המרצה: "למה לא? תהיה חבר שלו". 
	
	
	\theo{הופכי הוא יחיד} (אותה ההוכחה כמו בשדות. לא בהכרח בתחום שלמות, מעל כל חוג)\begin{proof}
		יהי $a \in R^x$ ו־$u, u'$ הופכיים שלו, אז:
		\[ u = u \cdot 1 = u \cdot a \cdot u' = 1 \cdot u' = u' \]
	\end{proof}
	\theo{אם $a \mid b$ וכם $b \mid a$ אז $a \mid b$ (בתחום שלמות). }\begin{proof}
		\begin{align*}
			a \mid b &\implies \exists c \in \R \co ac = b \\
			b \mid a &\implies \exists d \in \R\co bd = a
		\end{align*}
		לכן: 
		\[ ac = b \implies acd = a \implies a(cd - 1) = 0 \implies a = 0 \lor cd = 1 \]
	\end{proof}
	אם $a = 0$ אז $b = 0$ (ממש לפי הגדרה) ו־$\sim$ שקילות (רפליקסיביות). אחרת, $cd = 1$ ולכן $c$ הפיך, סה"כ $a \mid b$. 
	
	"אני חושב שבעברית קראו להם ידידים, לא רצו להתחייב לחברות ממש". 
	
	
	\defi{איבר $p \in R$ נקרא \textit{אי־פריק} אם מתקיים $p = ab \implies a \in R^x \lor b \in R^x$. }
	
	\defi{איבר $p \in R$ יקרא \textit{ראשוני} אם $p \mid (a \cdot b) \implies p \mid a \lor p \mid b$. }
	\rmark{איברים הפיכים לא נחשבים אי־פריקים או ראשוניים. הסיבה להגדרה: בשביל נכונות המשפט היסודי של האריתמטיקה (יחידות הפירוק לראשוניים). }
	
	\theo{בתחום שלמות כל ראשוני הוא אי פריק. }
	\textit{הערה: }שקילות לאו דווקא. 
	\begin{proof}
		יהי $p \in R$ ראשוני. יהיו $a, b \in R$ כך ש־$p = ab$. בה"כ $p \mid a$. אז קיים $c$ כך ש־$pc = a$ ולכן $pcb = p$. סה"כ $p \neq 0$ ולכן $cb = 1$ (ראה לעיל) ו־$b$ הפיך. 
	\end{proof}
	\defi{$R$ תחום פריקות יחידה אם $\prod_{i = 1}^{n}p_i = \prod_{j = 1}^{m}q_i$ עבור $p_i, q_j$ ראשוניים, אז $m = n$, ועד לכדי סידור מחדש, לכל $i \in [n]$ $p_i \sim q_i$. }
	\theo{נניח שבתחום שלמות $R$, כל אי־פריק הוא גם ראשוני. אז $R$ תחום פריקות יחידה. }
	
	ההוכחה: זהה לחלוטין לזו של המשפט היסודי. \begin{proof}
		באינדוקציה על $n+m$. בסיס: $n + m = 2$ ולכן $n = m = 1$ (כי מעפלה ריקה לא רלוונטית מאוד) אז $p = q$. נעבור לצעד. נניח שהטענה נכונה לכל $n + m < k$. נניח ש־$n + m = k$. אז $p_1 \mid \prod_{j = 1}^{m}q_j$. בה"כ $p_1 \mid q_1$. $q_1$ אי־פריק ולא הפיך. $p_1$ לא הפיך. לכן $p_1 \sim q_1$. אז עד כדי כפל בהופכי נקבל ש־$\prod_{i = 2}^{n} p_i = \prod_{j = 2}^{n} q_j$. \textit{הערה: }ראשוני כפול הפיך נשאר ראשוני. מכאן הקענו לדרוש וסיימנו (\textit{הערה שלי: }כאילו תכפילו בחברים ותקבלו את מה שצריך). 
	\end{proof}
	
	\defi{יהי $R$ תחום שלמות. תת־קבוצה $0 \neq I \subseteq R$ נקראת \textit{אידיאל} אם: 
		\begin{enumerate}[A.]
			\item $\forall a, b \in I \co a + b \in I$ – סגירות לחיבור. 
			\item $\forall a \in I \, \forall b \in R \co ab \in I$ – תכונת הבליעה. [בפרט $0 \in I$]
	\end{enumerate}}
	\textbf{דוגמאות: }\begin{enumerate}
		\item $0$ תמיד אידיאל, וכן החוג כולו תמיד אידיאל. 
		\item הזוגיים ב־$\Z$. 
		\item לכל $n \in \Z$, $n\Z$ אידיאל ($n$ כפול השלמים). הזוגיים לדוגמה, מקרה פרטי הוא $2\Z$. 
		\item $\la f \ra \subseteq \F[x]$ המוגדר לפי $\la f \ra := \{g \in \F[x] \mid f|g\}$
		\item הכללה של הקודמים: עבור $a \in R$ נסמן $\la a \ra := \{a \cdot b \mid b \in R\}$ הוא אידאל. 
		\item $I = \{f \in \F[x] \mid f(0) = 0\}$ (לעיתים מסומן $\forall a \in R \co aR = \la a \ra$)
		\item נוכל להכליל את 4 עוד: ("הכללה של הכללה היא הכללה. זה סגור להכללה. זה קורה הרבה במתמטיקה")
		\[ I = aR + bR = \{ar + bs \mid r, s \in R\} \]
		וניתן להכליל עוד באינדוקציה. 
	\end{enumerate}
	\defi{אידיאל $I$ נקרא \textit{ראשי} אם הוא מהצורה $aR$ עבור $a \in R$ כלשהו. }
	\noti{$ Ra =: (a) =: \la a \ra =: \{ar \mid r \in R\}$}
	\defi{תחום שלמות נקרא \textit{ראשי} אם כל אידיאל שלו ראשי. }
	
	\rmark{אנחנו סימנו אידיאל ב־$aR$ ובקורס מסמנים $Ra$, באופן כללי אפשר לדבר על אידיאל שמאלי ואידיאל ימני. תזכורת: $I \subseteq R$ היא אידיאל אם היא סגורה לחיבור ומקיימת את תכונת הבליעה. בתחום ראשי כל אידיאל הוא אידיאל ראשי. }
	
	\theo{ב־$\{0\} \neq R$ תחום ראשי אז כל אי פריק הוא ראשוני. }
	(תנאי מספיק אך לא הכרחי)
	\begin{proof}
		יהי $o$ אי פריק (א"פ). יהיו $a, b \in \R$ כך ש־$p \mid ab$. תיקון [משבוע שעבר]: במקום $I = Ra + Rb$, נשתמש ב־$I = Ra + Rp$. 
		בכלל ש־$R$ תחום ראשי, קיים $c \in R$ כך ש־$I = Rc$, ו־$a, p \in I$ כלומר $c \mid a \land c \mid p$. $p$ א"פ ולכן $c \sim p$ או $c$ הפיך. 
		\begin{itemize}
			\item $c$ הפיך $\impliedby$ $ \in I$ $\impliedby$ $R = R \cdot 1 \in I \subseteq R$ $\impliedby$ $I = R$. קיימים $r, s \in R$ כך ש־$ra + sp = 1$. נכפיל ב־$b$ ונקבל $rab + spb = b$ וסה"כ $p \mid b$. 
			\item אם $c \sim p$, אז $p \mid c \land c \mid a$ ולכן $p \mid a$. 
		\end{itemize}
	\end{proof}
	
	\cola{אם $R$ תחום שלמות ראשי אזי יש פריקות יחידה למכפלה של אי פריקים עד כדי חברות. }
	
	\theo{יהיו $a, b \in R$, אז $a, b$ ייקראו זרים אם $\forall c \in R\co c \mid a \land c \mid b \implies c \in R^x$}
	\defi{יהי $g \in R$ כך ש־:
		\begin{enumerate}
			\item $g \mid a \land g \mid b$
			\item $\forall \ml \in R \co \ml \mid a \land \ml \mid b$
			\item $\ml \mid g$
		\end{enumerate}
		אז $g$ כנ"ל הוא הגורם המשותף המקסימלי של $a, b$, הוא $\gcd(a, b)$}
	
	\theo{יהי $R$ תחום שלמות ויהיו $a, b \in R$. נניח שקיימים $r, s \in R$ כך ש־$g = ra + sb$ אשר מחלק את $a, b$. אז: 
		\begin{enumerate}
			\item $\gcd(a, b) = g$
			\item ה־$\gcd$ מוגדר ביחידות עד לכדי חברות. 
			\item בתחום ראשי, לכל $a, b$ קיים $g$ כנ"ל. 
	\end{enumerate}}
	(הערה: רק 3 באמת דורש תחום ראשי)
	\begin{proof}\,
		\begin{enumerate}
			\item יהי $\ml \mid a, b$ אז $\ml \mid ra, sb$ וסה"כ $\ml \mid g$. 
			\item מ־1 (בערך) אם $g, g'$ מקיימים את היותם $\gcd$ אז $g' \mid g \land g' \mid g$ ולכן $g \sim g'$. 
			\item נסמן $I = Ra + Rb$. אז $I = Rg$, וקיימים $r, s \in R$ כך ש־$ra + sb = g$ ולכן $a, b \in I$ אז $g \mid a, b$ וסיימנו מ־$1$. 
		\end{enumerate}
	\end{proof}
	\cola{בתחום ראשי, אם $a, b$ זרים אז $\exists r, s \in R\co ra + sb = 1$ (אלגוריתם אוקלידס המורחב). }
	\theo{$\F[x]$ תחום ראשי. }
	\begin{proof}
		יהי $I \subseteq \F[x]$ אידיאל. אם $I = \{0\}$, הוא ראשי. אחרת, $I \neq \{0\}$, ואז: יהי $0 \neq p \in I$ פולינום מדרגה מינימלית, ויהי $f \in I$. אז קיימים $q, r \in \F[x]$ כך ש־$f = qp + r$. ידוע $\deg r < \deg p$. בגלל ש־$f \in I \land p \in I$ אז $f - qp \in I$. אם $r$ אינו $0$, קיבלנו סתירה למינימליות הדרגה של $p$. 
	\end{proof}
	הוכחה זהה עובדת בשביל להראות ש־$\Z$ תחום ראשי, אך עם דרגה במקום ערך מוחלט. 
	
	\defi{תחום שלמות נקרא \textit{אוקלידי} אם קיימת $N \co R \setminus \{0\} \to \N_+$ כך ש־$\forall a, b \in R\setminus \{0\} \co \exists u, r \in R \co a = ub + r$ כאשר $r \neq 0$ או $N(b) > N(r)$, ו־$N$ סאב־כפלית כלומר $\forall 0 \neq a, b \in R \co N(a) \le N(ab)$. }
	ברגע שיש לנו את ההגדרה של תחום אוקלידי, $N$ הפונקציה שתשתמש אותנו בשביל להראות את ההוכחה שכל תחום אוקלידי הוא תחום ראשי (בדומה לערך מוחלט או $\deg$ בהוכחות קודמות). ההפך נכון תחת השערת רימן המוכללת (לא ראיתם את זה צץ, נכון?). 
	
	אינטואציה לחוג אוקלידי היא ``חלוקה עם שארית'', כאשר פונקצית הגודל $N$ דורשת שהשארית תהיה ``אופטימלית''. בחוג הפולינומים $N = \deg$ (פרטים בהמשך), ובחוג המספרים השלמים $N = \sof{\cdot}$. 
	
	דוגמה לחוג שאינו אוקלידי: $\Z[\sqrt{-5}]$ הוא $\{a + b\sqrt{-5} \mid a, b \in \Z\}$. 
	
	\theo{חוג אוקלידי $\impliedby$ פריקות יחידה (דומה למשפט היסודי של האריתמטיקה). }
	\theo{חוג אוקלידי $\impliedby$ תחום שלמות. }(הוכחה בויקיפדיה)
	
	לדוגמה בחוג לעיל $6 = 2 \cdot 3 = (1 + \sqrt{-5})(1 - \sqrt{-5})$ על אף ש־$2, 3$ אי־פריקים וכן $(1 + \sqrt{-5}), (1  - \sqrt{-5})$ אי פריקים. 
	
	
	\textbf{דוגמה }\textit{(חוג השלמות של גאוס)}\textbf{. }
	\[ R = \Z[i] = \{a + bi \mid a, b \in \Z\} \]
	הנורמה: 
	\[ N \co \R\to \Z_{\ge 0}, \ N(a + bi) = a^2 + b^2 = |a + bi|^2 \]
	בדומה להוכחה לפיה הערך המוחלט של מורכב הוא כפלי, ניתן להראות ש־$N$ כפלית. מי הם ההפיכים ב־$\Z[i]$? מי שמקיים $\ag \bg = 1$, כלומר: 
	\[ N(\ag)N(\bg) = 1 = N(1), \ \ag = a + bi, \ a^2 + b^2 = 1 \implies a + bi = \pm 1, \pm i \]
	
	בהנחה שמוגדרת נורמה כזו, החוג הוא אוקלידי (תנאי זה הכרחי אך לא מספיק). 
	
	\theo{יהי $p \in \Z$ ראשוני. התנאים הבאים שקולים: 
		\begin{itemize}
			\item $p$ פריק ב־$\Z[i]$
			\item $p = m^2 + n^2$ עבור $n, m \in \Z$
			\item $p = 2$ או $p \equiv 1 \pmod{4} $
			\item קיימים $r, s \in R$ כך ש־$ra + sb = 1$
	\end{itemize}}
	
	שימו לב ש־$\Z$ בתוך $\Z[i]$ לא סגורים לבליעה. 
	
	\defi{$I \subseteq R$ אידיאל נקרא \textit{ראשוני} אם $\forall a, b \in R \co (a \cdot b) \subseteq P$, אז $(a) \in P \lor (b) \in P$. }
	\defi{אידיאל $I \subseteq R$ נקרא \textit{אי־פריק} אם $\forall a, b \in R$ אם $I = (a \cdot b)$ אז $I = (a)$ או $I = (b)$. }
	ראינו, שבתחום ראשי אי פריק=ראשוני. ניתן להראות באופן שקול כי: 
	\theo{$R$ תחום ראשי, אז $I$ ראשוני אמ''מ $I$ אי־פריק. }
	\defi{יהי $R$ תחום שלמות [אפשר להתעסק גם עם אידיאל ימני ושמאלי] ונניח ש־$I \subseteq R$ אידיאל. אז $R/_I := \{a + I \mid a \in R\}$ הוא חוג (בהגדרת $a + I = \{a + i \mid i \in I\}$ חיבור מנות), כאשר הפעולות: 
		\begin{itemize}
			\item $(a + I) + (b + I) = (a + b) + I$
			\item $(a + I)(b + I) = ab + I$
	\end{itemize}}
	צריך להוכיח שזה לא תלוי בנציגים (הנציגים $a, b$) והכל אבל בן לא עומד לעשות את זה. נשאר כתרגיל בעבור הקורא. 
	
	\subsection{הרחבת שדות}
	\theo{בתחום ראשי $R$, אם $I$ אידיאל אי־פריק, אז $R/_I$ שדה. }
	
	\textbf{דוגמאות. }
	\begin{itemize}
		\item $\Z/_{\la p \ra}$ שדה. 
		\item $\R[x]$ תחום ראשי, ידוע $x^2 + 1$ אי־פריק. לכן $\R[x]/_{\la x^2 + 1\ra}\cong \C$. הרעיון: נוכל להסתכל על $p$ פולינום המבוטא כמו: 
		\[ p(x) = q(x) \cdot (x^2 + 1) + ax + b = ax +b + \la x^2 + 1\ra \]
		ואם נכפיל שני יצורים כאלו: 
		\[ ((ax + b) + I) + (cx + d + I)) = acx^2 + (ad + bc)x + bd + I \]
		אך ידוע ש־$x^2 + 1 = 0$ (כי זה האידיאל שלנו) עד לכדי נציג, כלומר מתקיים שוויון ל־$bd - ac + (ad + bc)x + I$ זהו כפל מרוכבים. אפשר גם להראות קיום הופכי וכו'. 
	\end{itemize}
	\begin{proof}
		יהי $0 \neq a + I \in R/_I$. אם $a \neq 0$, אז ב־$R$ מתקיים $p \nmid a$ אי $p$ א''פ (אם הוא היה מחלק את $a$ אז $a = 0$) ולכן $p, a$ זרים (כי האידיאל אי פריק וכו'). אז קיימים $r, s \in R$ כך ש־$ar + ps = 1$. סה''כ: 
		\[ ar + ps + I = 1 + I \implies ar + I = 1 + I \quad \top \]
		לכן $r + I$ הופכי של $a + I$ וסיימנו. 
	\end{proof}
	(למעשה זה אמ''מ – הכיוון השני תרגיל בעבור הקורא). 
	
	\defi{יהי $R$ תחום שלמות, $a_1 \dots a_n \in R$ ו־$\ml = \lcm(a_1 \dots a_n)$ אמ''מ: 
		\begin{enumerate}
			\item \hfil $\forall i \in [n] \co a_i \mid \ml$
			\item \hfil $\forall b \in R \co \forall i \in [n] \co a_i \mid b \so \ml \mid b$
	\end{enumerate}}
	\textbf{דוגמה. }$R = \Z, \ \lcm(2, 6, 5) = 30$. 
	
	\theo{יהי $\F$ שדה ויהי $f \in \F[x]$ פולינום אי־פריק ממעלה $\deg f > 1$. אז קיים $\K \supseteq \F$ כך שב־$\K$ יש ל־$f$ שורש. }
	ההוכחה למשפט קונסטקרטיבית, ובה צריך להראות שהקבוצה: 
	\[ \K = \{p(A_f) \mid p \in \F[x]\} \]
	עם חיבור וכפל מטריצות, היא שדה. השיכון $\ag \mapsto \ag I$ משכן את $\F \mapsto \K$. 
	
	
	\theo{(ללא הוכחה בקורס, תלוי באקסיומת הבחירה) לכל שדה $\F$ קיים ויחיד שדה $\F \subseteq \K$ סגור אלגברית. }
	דוגמה: $\R$ ו־$\C$. 
	
	\subsection{חוג הפולינומים}
	(תת־פרק זה לקוח מתרגול בקורס)
	\defi{ה\textit{דרגה} של הפולינום היא $\deg(f) := \max\{n \in \N \mid a_n \neq 0\}$, ומגדירים $\deg(0) = -\inf$}
	\theo{
	\[ \deg(fg) = \deg f + \deg g \quad \deg(d + g) = \max\{\deg f, \deg g\} \]}
	
	\rmark{חוג הפולינומים הוא חוג אוקלידי כי פונקציית הגודל $N = \deg f$ מקיימת את התנאי של חוג אוקלידי. לכן ממשפט הוא תחום ראשי. }
	\cola{לכל $f, g \in \F[x]$, אם $g \neq 0$ אז קיימים ויחידים פולינומים $q, r \in \F[x]$ כך ש־$f = qg + r \land \deg r < \deg g$. }
	
	\defi{נאמר שפולינום $q$ מחלק את $f$ אם $r = 0$ ומסמנים $q \mid f$. }
	
	\cola{\,
		\begin{enumerate}
			\item $f(a) = 0 \iff (x - a) \mid f$ (משפט בזו)
			\item אם $\deg f = n > -\inf$, ל־$f$ לכל היותר $n$ שורשים כולל ריבוי. 
			\item נניח ש־$f, g \in \F[x]$ ו־$\F \subseteq \K$, כאשר $K$ שדה. אם $g \mid f$ מעל $K$ אז $g \mid f$ מעל $\F$. 
		\end{enumerate}
	}
	\begin{proof}\,
		\begin{enumerate}
			\item הוכחה למשפט בזו: 
			\begin{itemize}
				\item[$\implies$] נניח $(x - a) \mid f$. אז קיים פולינום $g$ כך ש־$f = (x - a)g$, קרי $f(a) = (a - a)g(a) = 0$. 
				\item[$\impliedby$] נניח $f(a) = 0$. אז קיימים $q, r \in \F[x]$ כך ש־$f = q(x - a)$ ועל כן $0 = f(a) = q(a)(a - a) + r(a) = 0$ ולכן $r(a) = 0$. משום ש־$r$ פולינום קבוע (דרגתו קטנה מ־1, כי חילקנו ב־$(x - a)$ מדרגה 1), אז $r(x) = 0$. 
			\end{itemize}
			\item אינדוקציה
			\item נוכיח ב־"contrapositive": אנו יודעים ש־$P \to Q \iff \lnot Q \to \lnot P$. נניח ש־$g \nmid f$ מעל $\F$. קיימים $q, r \in \F[x]$ כך ש־$f =qg + r, \ r \neq 0$. הפירוק הזה הוא גם ב־$\K[x]$. מיחידות $r$, נקבל ש־$g \nmid f$ כל מעל $K$. 
		\end{enumerate}
	\end{proof}
	''לא הנחתי בשלילה, הוכחתי בקונטראפוסיטיב``
	
	\theo{בהינתן $f \in \F[x]$ ו־$r \in \N, \lg \in \F$ אז $\lg$ יקרא שורש מריבוי $r$ של $f$ אם $(x - \lg)^{n + 1} \nmid f \land (x - \lg)^{n} \mid f$. }
	\theo{\[ \forall f\in \F[x] \co f(\lg) = 0 \implies \exists g \in \F[x] \co f(x) = (x - \lg)g(x) \]}
	\theo{(באינדוקציה על הטענה הקודמת) בהינתן $\F$ שדה סגור אלגברית: 
		\[ \forall f \in \F[x] \co \exists (\ag_i)_{i = 1}^{n} \in \F, a_n \in \F \co a_n \cdot f = \prod_{i = 1}^{n}(x - \ag_i) \]}
	\theo{(מסקנה מהטענה הקודמת שניתן להוכיח באינדוקציה ללא הרחבת שדות) לפולינום $f \in \F[x]$ שאינו אפס יש לכל היותר $\deg f$ שורשים. }
	
	\rmark{שימו לב! כל המסקנות שלנו על תחומים ראשיים תקפים גם על פולינומים. בפרט, ניתן לכתוב כל פולינום $\F[x]$ כמכפלה של גורמים אי־פריקים ב־$\F[x]$ (אם $\F$ סגור אלברית, אלו גורמים לינאריים) עד לכדי סדר וחברות (קבועים). }
	\rmark{שימו לב שחלק ניכר מהמשפטים לעיל נכונים בעבור פולינומים מעל \textit{שדה} ולא מעל \textit{כל חוג} (בפרט, המשפט לפיו חוג הפולינומים תחום אוקלידי). }
	עתה נציג משפט פשוט אך מועיל ממתמטיקה B, שלעיתים משמש לניחוש שורשי פולינום ע''מ לפרקו. 
	\theo{יהי $p = \sumnio \ag_i x_i \in \Z[x]$ פולינום עם מקדמים שלמים. יהי $\frac{a}{b} \in \Q$ כך ש־$p\cl{\frac{a}{b}} = 0$ שורש, ובה''כ $\gcd(a, b) = 1$ (אחרת ניתן לצמצם). אזי $a \mid \ag_0 \land b \mid \ag_n$. }
	
	\cola{$\forall A \in M_n(\F) \,\, \forall k \ge n \,\, \exists p(c) \in \F_{n - 1}[x] \co A^{k} = p(A)$. }
	מסקנה זו נובעת מאלגוריתם לביטוי $A^{n + c}$ כקומבינציה לינארית של $I \cdots A^{n - 1}$ שמופיע בסוף הסיכום. 
	
	\subsubsection{פונקציות רציונליות ומספרים אלגבריים}
	\textbf{אינטואציה: }הרעיון של פונקציה רציונלית היא להיות ``פולינום חלקי פולינום''. נפרמל את הדבר הזה בעבור מרחב פולינומים מעל כל שדה. 
	
	\theo{בהינתן $\F$ שדה הקבוצה $\{(f, g) \mid f, g \in \F[x], g \neq 0\}$ משרה את יחס השקילות הבא: 
	\[ (f, g) \sim (\tl f, \tl g) \iff f \cdot \tl g = \tl f \cdot g \]}
	\noti{נסמן כל איבר במחלקת השקילות ע''י $\frac{f}{g}$ שמייצגים אותו. }
	\defi{\textit{שדה הפונקציות הרציונליות} הוא הקבוצה $Q[x]$ היא אוסף מחלקות השקילות של $\sim$ מהמשפט הקודם, עם פעולות החיבור והכפל הבאות: 
	\[ \frac{f}{g} \cdot \frac{\tl f}{\tl g} = \frac{f \tl f}{g \tl g} \land \frac{f}{g} + \frac{\tl f}{\tl g} = \frac{f \tl g + g \tl f}{g \tl g} \]}
	\lem{הגדרות הפעולות לעיל מוגדרות היטב (כלומר הן לא תללויות בנציגים)}
	\theo{$Q[x]$ שדה, כאשר $\frac{0}{1}$ הניטרלי לחיבור ו־$\frac{1}{1}$ הניטרלי לכפל. }
	\textbf{המלצה. }לקרוא שוב את פרק 2.1, ''על ההבדל בין פולינום לפולינום``, בו נבחין שלמרות ש־$\sof{\F_2 \to \F_2} = 4$, ישנם אינסוף פולינומים מעל השדה הזה. 
	
	\textbf{אינטואציה. }למעשה, נרצה להגיד שדה הפונקציות הרציונליות הוא איזומורפית (קאנונית, ולכן נתייחס אליו כאילו הוא שווה) ל־: 
	\[ Q[x] \cong \ccb{\frac{f(x)}{g(x)} \mid f(x), \,\smash{\underbrace{g(x)}_{\neq 0}} \in \F[x]} \]
	כאשר $\F[x]$ חוג הפולינומים מעל השדה $\F$. עוד כדאי לציין ש־$Q[x]$ מכיל עותק של $\F[x]$ (עד לכדי איזומורפיזם) בעבור $g = 1$ פולינום היחידה. כמובן ש''איזומורפיזם`` בהקשר הזה מדבר על העתקה (לא בהכרח לינארית) שמשמרת את פעולות החוג. 
	
	\theo{לכל $p$ ראשוני $\forall x \in \F_p \co x^{p} = x$. }
	\rmark{זוהי מסקנה ישירה מהמשפט הקטן של פרמה. }
	
	\defi{מספר מרוכב $\ag \in \C$ יקרא \textit{מספר אלגברי} אם קיים פולינום $0 \neq f \in \Q[x]$ כך ש־$f(\ag) = 0$. }
	\defi{מספר מרוכב שאינו אגלברי יקרא מספר \textit{טרנסצנדנטי}. }
	\textbf{דוגמאות. }נבחין ש־$\sqrt \ag$ הוא אלגברי כי הוא שורש של $x^2 - \ag$. קיימות הוכחות לפיהן $e$ ו־$\pi$ הם מספרים טרנסצדנטיים. 
	\theo{בהניתן $0 \neq V \subseteq \C$ תמ''ו מעל $\C$, אם $\forall x \in \C\co xV \subseteq V$ אז $x$ אלגברי. }
	\begin{proof}
		נגדיר $T_x \co V \to V$ כך ש־$T_x(v) = xv$ (ההעתקה מוגדרת היטב מהנתון). אזי $f_T(T) = 0$. אזי: 
		\[ f_T(t) =: \sumnio a_nt^{n} \implies 0_V = f(T)v = \sumnio a_nT^{n}v = \cl{\sumnio a_nx^{n}}v = f(x)v \]
		בפרט עבור $v \in V \setminus \{0\}$ יתקיים $f(x) = 0$ ולכן $x$ אלגברי. 
	\end{proof}
	
	
	
	
	\npage
	\section[פירוק פרימרי]{\en{Primary Decomposition}}
	\subsection{מרחבים $T$־שמורים וציקליים}
	 
	\defi{נניח ש־$V$ מ''ו מעל $\F$, ו־$T \co V \to V$ ט''ל. אז $U \subseteq V$ תמ''ו נקרא $T$־אינווריאנטי/$T$־שמור/ה אם לכל $u \in U$ מתקיים $T(u) \in U$. }
	\textbf{דוגמאות. }$V, \{0\}$ הם $T$־אינווריאנטים. גם המ''ע (המרחבים העצמיים) הם $T$־אינוואריאנטי. 
	
	\rmark{שימו לב: אם $U \subseteq V$ תמ''ו אינווריאנטי, אז $T|_U \co U \to U$ ט''ל. }
	
	\rmark{נניח ש־$u_1 \dots u_k$ בסיס ל־$U$ כנ''ל, ו־$W \subseteq V$ תמ''ו כך ש־$U \oplus W = V$, ונגיד ש־$w_{k + 1} \dots w_n$ בסיס ל־$W$, אז $B = (u_1 \dots u_k, w_{k + 1} \dots w_n)$ מקיים: 
		\[ [T]_B = \pms{[T|_{U}] & A \\ 0 & B} \]
		(כאשר $[T|_{U}] \in M_{k}$ ו־$B \in M_k$). 
		ותחת ההנחה שאכן $T$ הוא $U$־איוואריאנטי ו־$W$־איווריאנטי, אפשר לייצג אותו באמצעות שתי מטריצות מייצגות על האלכסון (ראה הוכחת המשפט הבא)
	}
	
	\begin{Theorem}
		יהי $V$ מ''ו, $U, W$ תמ''וים ונניח $U \oplus W = V$ וגם $U, W$ הם $T$־איוואריאנטים. אז $p_T(x) = p_{T|_U}(x) \cdot p_{T|_W}(x)$
	\end{Theorem}
	\begin{proof}
		משום ש־$U \oplus W = V$, קיים בסיס $B = (u_1 \dots u_k, w_{k + 1} \dots w_n)$ כך ש־$u_1 \dots u_k$ בסיס ל־$U$ ו־$w_{k + 1} \dots w_{n}$ בסיס ל־$W$. נבחין, שבייצוג תחת הבסיס הזה: 
		\[ [T]_B = \pms{[T|_U]_B & 0_{n \times (n - k)} \\ 0_{(n - k) \times n} & [T|_W]_B} \]
		זאת כי לכל $v \in V$ ניתן לייצגו בצורה יחידה כסכום של $u \in U, w \in W$ כך ש־$v = u + w$, כלומר $Tv = Tu + Tw$. ואכן תחת העתקת הקורדינאטות מהגדרת כפל וקטור במטריצה הטענה לעיל מתקיימת. 
		כלומר: 
		\[ p_T(x) = \detms{Ix - [T|_U]_B & 0 \\ 0 & Ix - [T|_W]_B} = \sof{Ix - [T|_U]_B} \cdot \sof{Ix - [T|_W]_B} = p_{T|_U}(x) \cdot p_{T|_W}(x) \]
		כדרוש. 
	\end{proof}
	\begin{Theorem}
		בהינתן $U_1 \dots U_k$ מרחבים $T$־איוואריאנטים כך ש־$\bigoplus_{i = 1}^{k} U_i = V$, מתקיים $p_T(x) = \prod_{i = 1}^{k} p_{T|_{U_i}}$. 
	\end{Theorem}
	\begin{proof}
		באינדוקציה על המשפט הקודם. 
	\end{proof}
	
	\defi{יהי $V$ מ''ו מעל $\F$, \ $T \co V \to V$ ט''ל ו־$v \in V$ וקטור. אז \textit{תת־המרחב־הציקלי} הנוצר מ־$T$ ע''י $T$ הוא 
	\[ \mathcal{Z}(T, v) := \Sp\{T^k(v) \mid k \in \N_0\} \]}
	\theo{\,\begin{itemize}
			\item $\zc(T, v)$ תמ''ו של $V$ – טרוויאלי.
			\item $\zc(T, v)$ תמ''ו $T$־איוואריאנטי – טרוויאלי גם. 
	\end{itemize}}
	
	עתה נציג משהו נחמד. אם $V$ נוצר סופית, גם $\zc(T, v)$ נ''ס. נגיד שיהיה $k \in \N_0$ מינימלי, כך שמתקיים $\zc(T, v) = \Sp\{v, Tv, \dots T^{k - 1} v\}$. אז $T^kv \in \zc(T, v)$. לכן קיימים $a_0 \dots a_{k - 1}$ כך ש־$T^kv + a_{k - 1}T^{k - 1}v + \cdots + a_0v = 0$. ניתן לקחת כבסיס את $v, Tv \dots T^{k - 1} v$ של $\zc(T, v)$. אז: 
	\[ [T]_B = \pms{0 & 0 & \cdots & 0 & -a_0
		\\ 1 & 0 & \cdots & 0&-a_1
		\\ 0 & 1 & \vdots & 0& \vdots
		\\ 0 & 0  & \cdots & 1 &-a_{n - 1}} \]
	כאשר השורה האחרונה כי: 
	\[ T(T^{n - 1}v)  = -\sum_{k = 0}^{n - 1}a_kT^kv \]
	\defi{$A_f = [T]_B$ היא \textit{המטריצה המצורפת} לפולינום $f(x) = x^{n} + a_{n - 1}x^{n - 1} + \cdots + a_0$. }
	
	\subsection{הפולינום המינימלי}
	
	דיברנו על הפולינום האופייני $f_A = f_T = \det(Ix - A)$. עוד ציינו בהינתו מטריצה, המטריצה המצורפת $A_f$ מקיימת $f_{A_f} = f(x)$. 
	
	
	\theo{תהי $A \in M_n(\F)$, נביט בקבוצה $I_A = \{p \in \F[x] \co p(A) = 0\}$, אז $I_A \subseteq \F[x]$ אידיאל, קיים ויחיד ב־$I_A$ פולינום מתוקן בעל דרגה מינימלית. }
	\defi{$I_A$ לעיל יקרא \textit{הפולינום המינימלי}. }
	\begin{proof}
		נבחין כי $0 \in I_A$. סגירות לחיבור – ברור. תכונת הבליעה – גם ברור. סה''כ אידיאל. 
		$\F[x]$ תחום שלמות ולכן נוצר ע''י פולינום יחיד $I_A = (p)$. אם $I_A = (p) = (p')$ אז $p \sim p'$. אם נקבע אותו להיות מתוקן אז הוא יחיד (חברות בשדה הפולינומים נבדלת ע''י כפל בפולינום קבוע).  לפולינום הנ''ל נקרא הפולינום המינימלי של $A$ הוא $m_A$. באותו האופן, עבור $T \co V \to V$ ט''ל ניתן להגדיר את $M_T$. 
	\end{proof}
	
	\noti{$m_A$ יהיה הפולינום המינימלי של המטריצה $A$. }
	
	\rmark{אם $A \in M_n(\F)$ ו־$p \in \F[x]$ כך ש־$p(A) = 0$, אז $p \in I_A$ ומתקיים $m_A \mid p$. }
	
	\rmark{אנו יודעים ש־$m_A \mid f_A$ כי ממשפט קיילי המילטון $f_A(A) = m_A(A) = 0$, ו־$f_A \in I_A$ כאשר $I_A$ האידיאל של המאפסים של $A$. מהיות מרחב הפולינומים תחום שלמות, $m_A \mid f_A$ כדרוש. }
	
	\textbf{דוגמה. }עבור $A = I_n$ אז $f_A = (x - 1)^{n}$ ו־$m_a = (x - 1)$. לא בהכרח $m_a = f_a$, אל לפעמים כן – לדגומה בעבור $D \co \F[x] \to \F[x]$ אופרטור הגזירה מתקיים $f_D = x^{n + 1}$ וכן $m_D = x^{n + 1}$ כי יש פולינומים שנדרש לכזור $n$ פעמים ע''מ לקבל $0$, לדוגמה $x^{n}$. 
	
	\theo{תהא $A =A_f$ המטריצה המצורפת ל־$A$. אז $f_A = m_A$.}
	\theo{אם $A$ מייצגת את $T \co V \to V$ אז $m_A = m_T$ (כלומר, הפולינום המינימלי לא תלוי בבחירת בסיס).}
	\begin{proof}
		נבחר בסיס ל־$V$, $B$. יהי $p \in \F[x]$. אז $[p(T)]_B = p([T]_B)$. שני האגפים מתאפסים ביחד, ולכן $I_A = I_T$. 
	\end{proof}
	\rmark{נניח ש־$A$ לכסינה והע''ע השונים הם $\lg_1 \dots \lg_k$ (כלומר $f_A = \prod_{i = 1}^{k}(x - \lg_i)^{r^i}$) אז $m_A = \prod_{i = 1}^{k}(x - \lg_i)$. }
	
	\begin{proof}
		בה''כ $A$ אלכסונית, $A = \diag(\lg_1 \dots \lg_k)$ עם חזרות. נבחין ש־$\prod_{i = 1}^{k} (x - \lg_i) = 0$ (הסברים בהמשך). $A$ מייצגת העתקה $T \co V \to V$ ול־$V$ יש בסיס של ו''עים $B = (v_1 \dots v_n)$.  אז $\cl{\prod_{i = 1}^{k}(T - \lg_i)}(v_j) = 0$ כי $v_j$ מתאים ל־$\lg_i$ כלשהו וכך זה מתאפס. ידוע $m_a \mid \prod_{i = 1}^{k}(x - \lg_i)$. אם נוריד את אחד המכופלים אז הע''ע שירד לא יתאפס/לא יאפס את הוקטור העצמי המצאים, כלומר כל הגורמים הלינארים דרושים כדי לאפס את $T$, ומכאן המינימליות והשוויון ל־$m_A$. 
	\end{proof}
%	למעשה, $d$ הנ''ל הוא המינימלי שעבורו ניתן לבטא את $A^{d}$ כצ''ל של חזקות נמוכות יותר. 
	
	\rmark{אם $A \in M_n(\F)$, ו־$\F \subseteq \K$, אז ניתן לחשוב על $A \in M_n(\K)$ ו־$m_A$ לא משתנה ללא תלות בשדה. }
	\theo{אם $g, h \in \F[x]$ ו־$T \co V \to V$ ט''ל אז $g(T), h(T)$ מתחלפות. }
	\begin{proof}
		\[ \big(g(T) \circ h(T)\big)(v) = (g \cdot h)(T)(v) = (h \cdot g)(T)(v) = \big(h(T) \circ g(T) \big)(v) \]
	\end{proof}
	
	\begin{Lemma}[למת המחלק של פולינום מינימלי]
		יהי $m_T$ הפולינום המינימלי של ט''ל $T \co V \to V$. אם $f(x) \mid m_T(x)$ וגם $\deg f > 0$ אז $f(T)$ אינו הפיך. 
	\end{Lemma}
	\begin{proof}
		בכלל ש־$f \mid m_T$ אז קיים $g\in \F[x]$ כך ש־$f \cdot g = m_T$. נניח בשלילה ש־$f(T)$ הפיכה. אז: 
		\[ f(T) \circ g(T) = \underbrace{m_T(T)}_{0} \implies \underbrace{f(T)\op \circ (0)}_{0} = g(T) \]
		ידוע: 
		\[ \deg m_T = \underbrace{\deg f}_{>0} + \deg g \implies \deg g< \deg m_T \]
		בה''כ $g$ מתוקן וקיבלנו סתירה למינימליות של $m_T$, אלא אם כן $g(x)$ פולינום ה־$0$ אבל אז $m_T = 0$ בסתירה להגדרתו של פולינום מינימלי. 
	\end{proof}
	הוכחה זהה עבור מחלק של $m_A$, עבור $A$ מטריצה. 
	\theo{אם $\lg$ ע''ע של $T$ אז בהינתן $p(T) = 0$ מתקיים $p(\lg) = 0$. }\begin{proof}
		קיים $v \neq 0$ ו''ע כלומר $Tv = \lg$, ולכן: 
		\[ p(x) = \sum_{i = 0}^{n}a_ix^i \quad 0 = 0 v = p(T)(v) = \cl{\sum_{i = 0}^{n}a_iT^{i}}\cl{v} = \sum_{i = 0}^{n}a_iT^{i}(v) = \cl{\sum_{i = 0}^{n}a_i\lg^i}v = p(\lg)v \] 
		מהיות $v \neq 0$ נקבל $p(\lg) = 0$ כדרוש. 
	\end{proof}
	``זה טבעוני, זה טבעוני וזה ממששש טבעוני''. ``מה זה אומר שזה לא טבעוני? יש בזה קצת ביצה''. 
	\theo{$\lg$ ע''ע של $T$ אמ''מ $m_T(\lg) = 0$. } \begin{proof}
		כיוון אחד הוא מקרה פרטי של המשפט הקודם. מהכיוון השני, ידוע $m_T(\lg) = 0$. לפי משפט בזו $(x - \lg) | m_T(x)$. ידוע $m_T | f_T$ וסה''כ $(x - \lg) | f_T$ וסה''כ $\lg$ ע''ע של $T$. 
	\end{proof}
	
	\theo{\hfil $m_A(x) \mid f_A(x) \mid (m_A(x))^{n}$}
	\begin{proof}
		נותר להוכיח $f_A(x) | (m_A(x))^{n}$ (השאר ממשפטים קודמים). ידוע שפולינום מינימלי/אופייני נשארים זהים מעל כל שדה שמכיל את $\F$. לכן, ניתן להניח שהוא מתפרק לגורמים לינאריים. ראינו שאם $f, g \in \F[x]$, $\F \subseteq \K$ ומתקיים $f \mid g$ מעל $\K$, אז $f \mid g$ מעל $\F$. אז: 
		\[ \cl{\sum n_i = n} \quad\quad f_A = \prod_{i = 1}^{k}(x - \lg_i)^{n_i}, \ m_A(x) = \prod_{i = 1}^{k}(x - \lg_i)^{m_i} \quad (1 \le m_i \le n_i) \ (m_a(x))^{n} = \prod_{i = 1}^{k}(x - \lg_i)^{n \mid m_i} \]
		בגלל ש־$1 \le m_i \implies n \le m_i \cdot n$ אז מצאנו $f_A\mid m_A^{n}$. 
	\end{proof}
	הוכחה זהה עבור $T \co V \to V$ עם $\dim V = n$. 
	\begin{Collary}[שימושית!]
		נניח ש־$g \mid f_A$. נניח ש־$g$ אי פריק. אז $g \mid m_A$. 
	\end{Collary}\begin{proof}
		\[ g \mid f_A \mid (m_A)^{n} \]
		ידוע $g$ אי פריק, ולכן ראשוני (כי $\F[x]$ תחום ראשי) ולכן $g \mid m_A$. 
	\end{proof}
	\theo{נניח ש־$A$ בלוקים עם בלוקים על האלכסון, $A = \diag(A_1 \dots A_k)$ כך ש־$\forall i \in [k] \co A_i \in M_{n_i}(\F), \ \sum n_i = n$, אז מתקיים $m_A = \lcm(m_{A_1} \dots m_{A_k})$. }
	במקרה שלנו, ה־$\lcm$ הנ''ל הוא הפולינום בעל הדרגה המינימלית שמתחלק בכל ה־$m_{A_i}(x)$־ים. באופן כללי, $\lcm(m_{a_1} \dots m_{a_n})$ מתקבל כיוצר של אידיאל החיתוך בתחום הראשי. כלומר: 
	\[ I = (\lcm(A_1 \dots A_k)) = \bigcap_{i = 1}^{n}R m_{a_i} \]
	(הבהרת הסימון: $ Ra = (a) = \la a \ra $). 
	\begin{proof}[הוכחה (למשפט לעיל). ]
		לכל $g \in \F[x]$ מתקיים: 
		\[ g(A) = \pms{g(A_1) & & \\ & \ddots & \\ && g(A_k)} \]
		בבירור מתקיים $g(A) = 0$ אמ''מ $\forall i \in [k] \co g(A_i) =0$. לכן $\forall i \in [k] \co m_{A_i} | g$. מהגדרת ה־$\lcm$ סיימנו. 
	\end{proof}
	\cola{תהי ט''ל $T \co V \to V$ ו־$V$ מונ''ס, אז בהינתן $U_1 \dots U_k$ מרחבים $T$־שמורים כך ש־$V = \bigoplus_{i = 1}^{k}U_i$, אזי $m_T = \lcm(\{m_{T|_{U_i}} \co i \in [k]\})$. }
	
	\theo{נניח ש־$T, S \co V \to V$ ט''לים. אז: 
		\begin{enumerate}
			\item אם $T, S$ מתחלפות, אז $\Img S, \ \ker S$ הם $T$־אינווריאנטים (ולהפך). 
			\item אם $T, S$ מתחלפות ו־$S \subseteq W$ תמ''ו הוא $T$־אינוואריאנטי, אז גם $S(W)$ הוא $T$־איוואריאנטי. 
			\item אם $W_1, W_2 \subseteq V$ הם $T$־איוואריאנטי אז גם $W_1 + W_2, \ W_1 \cap W_2$ הם $T$־איוואריאנטי. 
			\item אם $f \in \F[x]$ ו־$W \subseteq V$ תמ''ו $T$־איוואריאנטי, אז $W$ גם $f(T)$־איוואריאנטי. 
	\end{enumerate}}
	\begin{proof}\,
		\begin{enumerate}
			\item יהא $v \in \Img S$, אז קיים $u \in V$ כך ש־$S(u) = v$: 
			\[ Tv = T(S(u)) = (T \circ S)(u) = (S \circ T)(u) = S(T(u)) \in \Img S \implies Tv \in \Img S \]
			ועבור $v \in \ker S$: 
			\[ S(T(v)) = (ST)v = (TS)v = T(S(v)) = T(0) = 0 \implies Tv \in \ker S \]
			\item יהי $v \in S(W)$. קיים $w \in W$ כך ש־$v = S(w)$
			\[ Tv = T(S(w)) = S(T(w)) \in S(W) \]
			כי $Tw \in W$. 
			\item ראינו בתרגול הקודם
			\item יהי $w \in W$. 
			\[ f = \sum_{i = 1}^{n}a_ix^{i}, \ f(T)w = \cl{\sum_{i = 0}^{n}a_iT^{i}}(w) = \sum_{i = 0}^{n}a_iT^{i}(w) \]
			באינדוקציה $T^{i}(w) \in W$. $W$ תמ''ו ולכן סגור וסיימנו. 
		\end{enumerate}
	\end{proof}
	
	\subsection{ניסוח והוכחת משפט הפירוק הפרימרי}
	%TODO next line
	%TODO לעבור על הסיכומים המקוריים, נראה כאילו חסר הוכחות
	\begin{Theorem}[מקרה הבסיס של משפט הפירוק הפרימרי]
		(``מאוד חשוב'') יהי $V$ מ''ו מעל $\F$. נניח $T \co V \to V$ ט''ל. נניח $f(T) = 0$. נניח ש־$f = g \cdot h$ עבור $\gcd(g, h) = 1$. אז: 
		\[ V = \ker g(T) \oplus \ker h(T) \]
		ואם $f = m_T$, אז $g, h$ הם הפולינומים המינימליים לצמצום $T$ על תת־המרחבים לעיל בהתאמה. 
	\end{Theorem}
	הבהרת הכוונה ב''פולינום המינימלי לצמצום $T$ על תתי המרחבים'': בהינתן $T = U \oplus W$, $T_u = T_{\mid_U} \co U \to U$ ובאופן דומה $T_w$, אז $m_T = m_{T_U} \cdot m_{T_W}$. 
	\begin{proof}\,
		\begin{itemize}
			\item ידוע $h= g \cdot h$ ולכן $\exists a(x), b(x) \in \F[x]$ כך ש־$a(x)g(x) + b(x)h(x) = 1$, כך ש־: 
			\[ \underbrace{(a(T) \circ g(T))(v)}_{\in \ker h(T)} + \underbrace{(b(T) \circ h(T))(v)}_{\in \ker g(T)} = V \]
			
			הטענה ש־$(aT \circ gT)v \in \ker hT$ נובעת מכך ש־: 
			\[ (hT)((aT \circ gT)v) = hT((ag(T))v) = (hag)Tv = ((agh)T)v = ((af)T)v = (aT)(fT)v = (a(T) \cdot 0)v = 0v = 0 \]
			(זאת כי כפל פולינומים קוממטיבי, כל עולמות הדיון אסוציאטיביים, וכאשר ההעתקה $aT$ תקבל את $fT = 0$ היא תחזיר אפס וסה''כ $0v = 0$ כדרוש). מהכיוון השני: 
			\[ (gT)((bT \circ hT)v) = gT((bh(T))v) = (gbh)Tv = ((bgh)T)v = ((bf)T)v = (bT)(fT)v = (b(T) \cdot 0)v = 0v = 0 \]
			כלומר אכן $(aT \circ gT) \subseteq \ker hT$ ו־$(bT \circ hT) \subseteq \ker gT$. מהשוויון לעיל סה''כ אכן $V = \ker h(T) + \ker g(T)$. הסכום אכן ישר שכן: 
			\[ \forall v \in \ker gT \cap \ker hT \co 0 + 0 = (aT \circ gT)v + (bT + hT)v = v \]
			דהיינו, $\ker g(T) \oplus \ker h(T) = V$ כדרוש מהחלק הראשון של המשפט. 
			\item עתה נוכיח את החלק השני של המשפט. נניח $f = m_T$, ונסמן: 
			\begin{align*}
				W_2 &= \ker h(T) \dequad&\dequad W_1 &= \ker g(T) \\
				T_2 &=  T|_{{W_2}} \dequad&\dequad T_1 &= T|_{{W_1}}
			\end{align*}
			וכן $B_1$ בסיס ל־$W_1$, $B_2$ ל־$W_2$. לכן $B = B_1 \uplus B_2$ בסיס ל־$V$. משום שהראינו ש־$W_1, W_2$ הם $T$־אינוואריאנטי (כי $gT, hT$ מתחלפות): 
			\[ [T]_B = \pms{[T_1] & 0 \\ 0 & [T_2]} \]
			מהמשפט שראינו, $m_T = \lcm(m_{T_1}, m_{T_2})$. ברור ש־$m_{T_1} | g$ וגם $m_{T_2} | h$. אז: 
			\[ \deg m_{T_1} = \deg g + \deg h \ge \deg m_{T_1} + \deg m_{T_2} = \deg(m_{T_1} \cdot m_{T_2}) \ge \deg(\lcm(m_{T_1}, m_{T_2})) = \deg m_T \]
			ולכן כל ה''אשים לעיל הדוקים ושוויון בכל מקום. 
			\[ \deg m_{T_1} \le \deg g \land \deg m_{T_2} \le \deg h \]
			אם אחד מהשווינות לא הדוקים, אז: 
			\[ \deg m_{T_1} + \deg m_{T_2} < \deg g + \deg h \]
			וסתירה למה שהראינו. לכן: 
			\[ (m_{T_1} | g \land \deg m_{T_1} = \deg g) \implies m_{T_1} \sim g \]
			אבל שניהם מתוקנים ולכן שווים. כנ''ל עבור $m_{T_2} = h$. 
		\end{itemize}
		סה''כ הוכחנו את כל חלקי המשפט, כדרוש. 
	\end{proof}
	
	
	\textbf{דוגמה. }נסמן $f(x) = x^{2}(x - 1)^{3}$, $f(T) = 0$. החלק הראשון של המשפט אומר $V = \ker T^2 \oplus \ker (T - I)^3$. החלק השני אומר שאם $f = m_T$ אז $x^2$ הוא הפולינום המינימלי של $T|_{{\ker T^2}}$ וכן $(x - 1)^{3}$ המינילי של $T|_{{T - I}^{3}}$. 
	
	\begin{Theorem}[משפט הפירוק הפרימרי]
		יהיו $T \co V \to V$, $m_T$ הפולינום המינימלי של $T$, ונניח ש־: 
		\[ m_T = g_1 \cdots g_s \quad \forall i \neq j \co \gcd(g_i, g_j) = 1 \]
		אז: 
		\[ V = \bigoplus_{i = 1}^{s} \ker g_i(T) \]
		ובנוסף $g_i$ הוא הפולינום המינימלי של $T|_{{\ker g_i(T)}}$. 
	\end{Theorem}
	''יש לו שם מפוצץ אז הוא כנראה חשוב``
	\begin{proof}באינדוקציה על $s$
		\begin{itemize}
			\item[בסיס: ]עבור $s = 2$ המשפט שהוכחנו. 
			\item[צעד: ]נסמן: 
			\[ h(x) = g_s(x), \ g(x) = \prod_{i = 1}^{s - 1} g_i(x)\]
			ואז: 
			\[ \forall i \neq j \co \gcd(g_i, g_j) = 1 \implies \gcd(g, h) = 1 \]
			מהמשפט שקיבלנו: 
			\[ V = \ker g(T) \oplus \ker h(T) \,\overset{\mathclap{\text{ה.א.}}}{\implies}\,\, \bigoplus_{i = 1}^{s}\ker g_i(T) \]
			וכדי להוכיח את החלק השני של המשפט, נגדיר $m_{T|_{\ker g_i}} = g_i$: 
			%            \[ \tl T \co T|_{{\ker g_i}} \]
			\[ \ker h(T) = \bigoplus_{i = 1}^{s} \ker g_i (T_{|{\ker h(T)}}) = \bigoplus_{i = 1}^{s} \ker g_i(T) \]
		\end{itemize}
	\end{proof}
	\rmark{בהתאם למקרה הבסיס, מספיק היה להניח $f = g_1 \cdots g_s$, ולא היה באמת צורך להניח $f = m_T$ ספציפית, אם רק רוצים להראות קיום פירוק (ולא צריך להראות להראות ש־$g_i$ הם הפולינומים המינימליים לצמצום $T$ על התמ''וים). למעשה נשתמש בגרסה מוחלשת זו של משפט הפירוק הפרימרי. }
	
	\begin{Theorem}[תוצאה 1 ממשפט הפירוק הפרימרי]
		$T$ לכסינה אמ''מ $m_T = \prod_{i = 1}^{s}(x - \lg_i)$ מתפרק לגורמים לינארים $i \neq j \implies \lg_i \neq \lg_j$ שונים זה מזה. 
	\end{Theorem}
	\begin{proof}\,
		\begin{itemize}
			\item[$\implies$]לפי המשפט, אם נסמן $g_i = (x - \lg_i)$: 
			\[ V = \bigoplus_{i = 1}^{s} \ker (T - \lg_i I) \]
			כלומר $V$ סכום ישר של המ''ע של $\lg_1 \dots \lg_s$. לכל מרחב עצמי מממד $k_i$ קיימים $v_1 \dots v_{k_i}$ בסיס כך ש־$Tv_{j} = \lg_i v_j$ ($j \in [k_i]$), ומהסכום הישר ידוע $\sum_{i = 1}^{s}k_i = n$ ומהיות איחוד בסיסים של מ''ע גם בסיס (כי המ''ע זרים)	מצאנו בסיס מלכסן הוא אוסף הבסיסים של המ''עים. 
			\item[$\impliedby$]אם $T$ לכסינה, אז הפולינום המינימלי הוא ה־$\lcm$ של הפולינומים המינימליים של הבלוקים על האלכסון. הבלוקים על האלכסון הם $\lg_i$ הע''ע מגודל $1$, ולכן ה־$\lcm$ שלהם הוא מכפלת $x - \lg_i$ כאשר $\lg_i$ הע''עים השונים, וסה''כ $m_T$ מכפלת גורמים לינאריים שונים. 
		\end{itemize}
	\end{proof}
	
	\begin{Theorem}[תוצאה 2 ממשפט הפירוק הפרימרי]
		נניח $T \co V \to V$ לכסינה, וקיים $W \subseteq V$ תמ''ו $T$־שמור. אז $T|_{W}$ לכסינה. 
	\end{Theorem}\begin{proof}
		נסמן $S = T|_{W}$. אנחנו יודעים $m_T(T) = 0$ ולכן $m_T(S) = 0$. ידוע $m_S \mid m_T = \prod_{i = 1}^{r}(x - \lg_i)$ ולכן $m_S$ מתפרק לגורמים לינארים זרים, סה''כ $S$ לכסינה. 
	\end{proof}
	
	\textbf{סיכום. }$T \co V \to V$ ט''ל, ו־: 
	\[ \forall i \neq j \co \gcd(g_i, g_j) =1 \land m_T(x) = \prod_{i = 1}^{s}g_i \]
	אז: 
	\[ V = \bigoplus_{i = 1}^{s} \ker(g_i(T)) \land \forall i \co m_{T|_{\ker g_i(T)}} \!\!= g_i \]
	
	
	\npage	
	\section[צורת ג'ורדן]{\en{Jordan Form}}
	\subsection{מציאת שורשי פולינום אופייני ממעלה חמישית ואילך}
	נבחין בבעיה: $A = M_5(\Z)$, קבעו אם היא לכסינה מעל $\C$. 
	\begin{itemize}
		\item נחשב את $f_A(x)$
		\item נמצא שורשים, אלו הם הע''ע
		\item לכל ע''ע נחשב את $v_\lg$
		\item אם סכום הממדים מסעיף ג' הוא 5, אז היא לכסינה
		\item $T$ לכסינה אמ''מ קיים בסיס ו''ע אמ''מ ריבוי גיאומטרי = ריבוי אלגברי
	\end{itemize}
	
	אבל (המתמטיקאי, לא מילת הניגוד ולא מילה נרדפת ליגון) הוכיח שאין פתרונות לפולינומים ממעלה חמישית ויותר, וגלואה מצא דוגמאות לפולינומים שאי אפשר לבצע עליהם נוסחאת שורשים ופיתח את התורה של הרחבת שדות לשם כך. 
	
	היוונים העתיקים ביססו את הגיאומטריה שלהם באמצעות דברים שאפשר לבדוק עם שדה ומחוגה. באמצעות כלים של גלואה אפשר לראות מה אפשר לעשות עם האמצעים האלו, ולהוכיח שלוש בעיות שהיוונים לא הצליחו לפתור – האם אפשר לרבע את המעגל (האם אפשר לבנות באמצעות שדה ומחוגה ריבוע ששטחו שווה לשטח המעגל), או במילים אחרות, האם אפשר למצוא את $\sqrt\pi$ – אי אפשר כי זה לא מספר אלגברי. שאלות אחרות היו, בהינתן קובייה, האם אני יכול למצוא קובייה בנפח כפול? באותה המידה אי אפשר למצוא את $\sqrt3$. שאלה אחרת הייתה האם אפשר לחלק זווית ל־3 חלקים שווים. 
	
	גלואה הראה שכדי לעשות את זה צריך למצוא שורשים שלישיים של כל מני דברים, שבאמצעות סרגל ומחוגה אי אפשר לעשות זאת. בעיות שהיו פתוחות לעולם המתמטי במשך אלפי שנים נפתרו בעזרת אותן התורות. 
	
	אבל ניאלץ להאביל את משפחתו עליו  כשמת משחפת בגיל 26. גלואה מת בגיל 21 מדו־קרב. 
	\begin{Collary}[מסקנת הבדיעבד של גלואה]
		לא ללכת לדו־קרב. 
	\end{Collary}
	\begin{proof}
		ההוכחה מתקדמת ועוסקת בתורת גלואה. 
	\end{proof}
	
	\defi{בהינתן $f(x) = \prod_{k}(x - \lg_k)^{r_k} \quad \forall i \neq j \co \lg_i \neq \lg_j$. אז $f^{\mathrm{red}} := \prod_{k}(x - \lg_k)$}
	
	\theo{
	\[ f^{\mathrm{red}} = \frac{f}{\gcd(f, f')} \]}
	\begin{proof}
		נשאר כתרגיל בעבור הקורא. (נתנו לנו את זה בשיעורי הבית)
	\end{proof}
	
	\theo{$A$ לכסינה אמ''מ $f_A^{\mathrm{red}}(A) = 0$}
	
	\lem{$f_A^{\mathrm{red}} \mid m_A$ ושוויון אמ''מ $A$ לכסינה. }
	
	\begin{proof}[הוכחת הלמה. ]
		יהיו $\lg_1 \dots \lg_r$ הע''ע של $A$ (אפשר בה''כ להרחיב שדה כדי שהם יהיו קיימים). אז אם $f_A(x) = \prod_{i = 1}^{r}(x - \lg_i)^{s_i}$ ומתקיים $m_A(x) = \prod_{i = 1}^{r}(x - \lg_i)^{r_i}$ וידוע $1 \le r_i \le s_i$ ולכן $f_A^{\mathrm{red}} \mid m_A$. 
		
		עתה נוכיח את החלק השני של הלמה (השוויון). אם $A$ לכסינה, אז יש בסיס של ו''עים ועם $\lg$ הוא ע''ע של ו''ע בבסיס $B$ אז $Av_\lg - \lg v_\lg = 0$, ולכן $f_A^{\mathrm{red}}(A) = 0$ וסה''כ $m_A \mid f_A$. 
		
		אם $f_A^{\mathrm{red}} = m_A$ אז $m_A$ מכפלה של גורמים לינארית זרים, וראינו גרירה ללכסינות. 
	\end{proof}
	
	\begin{proof}[הוכחת המשפט באמצעות הלמה.]
		$A$ לכסינה אמ''מ $m_A = f_A^{\mathrm{red}}$, ואנחנו יודעים כי $m_A(A) = 0$ ולכן $A$ לכסינה אמ''מ $f_A^{\mathrm{red}}(A) = 0$. 
	\end{proof}
	
	\subsection{צורת ג'ורדן לאופרטור לינארי נילפוטנטי}
	\subsubsection{נילפוטנטיות}
	
	\textbf{מטרה: }בהינתן $אT \co V \to V$ נרצה לפרק את $V$ לסכומים ישרים של מרחבים $T$־אינווריאנטים, קטנים ככל האפשר. 
	
	\defi{יהי $T \co V \to V$ ט''ל. נאמר ש־$V$ \textit{פריק ל־$T$} אם קיימים $U, W \subseteq V$ תמ''וים כך ש: 
		\[ V = U \oplus W \quad \land \quad \dim U, \dim W > 0 \quad \land \quad U, W \ \text{\en{are $T$-invariant}} \]}
	
	\textbf{מעתה ואילך} (עד סוף הנושא)\textbf{, }נניח ש־$f_T(x)$ מתפצל מעל $f$ לגורמים לינארים (כלומר, נרחיב לשדה סגור אלגברית). 
	
	\defi{$T \co V \to V$ ט''ל. $T$ נקראת \textit{העתקה נילפוטנטית} אם קיים $n \in \N$ כך ש־$T^{n} = 0$. באופן דומה $A$ תקרא \textit{מטריצה נילפוטנטית} אם $\exists n \in \N \co A^{n} = 0$. }
	\defi{עבור $n$ המינימלי שעבורו $T^{n} = 0 / A^{n} = 0$, אז $n$ הנ''ל נקרא \textit{דרגת הנילפוטנטיות} של $T/A$, ומסמנים $n(T)/n(A)$. }
	``נילפוטנטית בא מלשון null. הרעיון: דבר מה שמתבטל. 
	
	\begin{Theorem}[תוצאה 3 ממשפט הפירוק הפרימרי]
		בהינתן $V$ אי־פריק ביחס ל־$T$, ובהנחה ש־$f_T(x)$ מתפצל לגורמים לינאריים, אז $m_T(x) = (x - \lg)^{r}$. נוסף על כך $T - \lg I$ נילפוטנטית ו־$n(T - \lg I) = r$. 
	\end{Theorem}
	\begin{proof}
		נפרק למקרים. 
		\begin{itemize}
			\item אם $m_T(x)$ לא מתפרק, הוא בהכרח לא קבוע אחרת $m_T(T) \neq 0$ וסתירה, לכן $m_T(x) = (x - \lg)$ לינארי כלשהו (אם לא לינארי ניתן לפרק לגורמים לינארים ואז $m_T$ מתפרק וסתירה). 
			\item אם $m_T(x)$ מתפרק, אז נוציא גורם לינארי אחד ונקבל $m_T = g_1 \cdots g_i$ כאשר $g$ לינארי, דהיינו ממשפט הפירוק הפרימרי, נניח בשלילה $g_i \neq g_j$ ומהיות $m_T$ מתוקן נקבל $\gcd(g_i, g_j) = 1$ כלומר $V = \ker \bigoplus_{i = 1}^{s}\ker g_i(T)$ ולכן $V$ פריק וסתירה. דהיינו $g_i = g_j$ וסה''כ $m_T(x)$ הוא מהצורה $m_T(x) = g_i^{s} = (x - \lg)^{s}$ כדרוש. 
		\end{itemize}
		עתה ניגש להוכיח את החלק השני של ההוכחה. משום ש־$m_T(x) = (x - \lg)^{r}$, אזי $0 = M_T(T) = (T - \lg I)^{r}$ ולכן $n(T - \lg I) \le r$, ומהמינימליות של $m_T$ נסיק $n(T - \lg I) = r$ כדרוש. 
	\end{proof}
	נסמן $S = T - \lg I$ בהקשר לעיל. עוד כדאי להבחין ש־$V$ הוא $S$־איווריאנטי (\textbf{אך לא בהכרח אי־פריק ביחס ל־$S$}) שכן $S(V) = T(V) - \lg V \in V$ מסגירות לכפל בסקלר $\lg$ ולחיבור נגדי. 
	
	מה למדנו? שמשום שאנו יכולים לפרק (ממשפט הפירוק הפרימרי) את $T$ למרחבים $T$־איווריאנטיים פריקים מינימליים, אז לכל $U_i$ כזה נוכל להגדיר $S_i = T - \lg_i I$ כזו כך שהיא נילפוטנטית. אם נוכל להבין טוב מה $S_i$ עושה למרחב שהיא שמורה עליו, נוכל להבין באופן כללי מה ההעתקה $T$ עושה לכל אחד מהמרחבים אליהם פריקנו אותה. 	
	
	\lem{תהי $T$ העתקה כללית, אז אם $\ker T^{i}  = \ker T^{i + 1}$ לכל $j \ge i$ מתקיים $\ker T^{i} = \ker T^{j}$. }
	\lem{תהי $T$ העתקה כללית, אז $\forall i > j \co \ker T^{i} \supseteq \ker T^{j} \land \Img T^{i} \subseteq \Img T^{j}$. }
	
	\theo{תהי $T \co V \to V$ העתקה מעל מ''ונסים, $\dim V = n$, אז קיים $\fc(T) \in [n]$ כך ש־
		$\forall i \in \N \co \ker T^{\fc(T)} = \ker T^{\fc(T) + i} \land \Img T^{\fc(T)} = \Img T^{\fc(T) + i}$. }
	\begin{proof}
		מלמה $5$, בהכרח: 
		\[ \ker T \subseteq \ker T^{2} \subseteq \ker T^{3} \subseteq \cdots \subseteq T^{i} \subseteq \cdots \subseteq V \]
		נניח בשלילה שכל ההכלות עד $i = n$ חלשות, ממשפט נסיק: 
		\[ \dim \ker T < \dim \ker T^{2} < \cdots < \dim \ker T^{i} \le n \]
		כלומר יש $n$ מספרים טבעיים שונים בין $\ker T$ ובין $n$ (לא כולל) ולכן $\dim \ker T < 0$ וסתירה. דהיינו קיים $\fc(T)$ כך ש־$\ker T^{\fc(T)} = \ker T^{\fc(T) + 1}$ ומלמה 4 נקבל ש־$\forall i \ge \fc(T) \co \ker T^{\fc(T)}  = \ker T^{i}$. ניכר ש־$\Img T^{\fc(T)} \supseteq \Img T^{i}$ לכל $i \ge \fc(T)$, וממשפט הממדים $\dim \Img T^{\fc(T)}  = \dim \Img T^{i}$ ולכן $\forall i \ge \fc(T) \co \Img T^{\fc(T)} = \Img T^{i}$ כדרוש. 
	\end{proof}
	\theo{בהינתן $T$ העתקה נילפוטנטית, אז $\fc(T) = n(T)$. }
	\noti{$\fc(T)$ לעיל סימון (שמקובל אך ורק בסיכום הזה), וקרוי ה־''fitting index`` של $T$. }
	
	
	\subsubsection{שרשאות וציקליות}
	
	\defi{קבוצה מהצורה $\{v, Tv \cdots T^kv\}$ כאשר $T^{k + 1}v = 0$ והוא המינימלי, נקרתא \textit{שרשרת}. }
	
	\theo{$T \co V \to V$ נילפוטנטית, אז כל שרשרת היא בת''ל. }
	\begin{proof}
		יהיו $\ag_0 \dots \ag_k \in \F$ כך ש־$\sum_{i = 0}^{k}\ag_i T^{(i)}(v) = 0$. נניח בשלילה שהצירוף אינו טרוויאלי. אז קיים $j$ מינימלי שעבורו $\ag_j \neq 0$. נניח $n$ המקסימלי שלא מאפס. אז: 
		\[ T^{n - j}\cl{\sum \ag_i T^{(i)}(v)} = T^{n - j}\cl{\sum_{i = j}^{k}\ag_iT^{i}(v)} = \ag_jT^{n - 1}(v) + 0 = 0 \]
		אבל $\ag_j, T^{n - 1} \neq 0$ וזו סתירה. 
	\end{proof}
	
	\textbf{תזכורת. }תמ''ו שקיים לו בסיס שהוא שרשרת, נקרא ציקלי. 
	
	\textbf{אנטי־דוגמה: }ישנם מ''וים שאינם $T$־ציקליים. למשל: 
	\[ V = \ccb{f \co \R^2 \to \R \mid f\pms{x \\ y} - P(x) + h(y) \mid \text{פולינומים ממעלה $\le n$} \ p, h} \]
	ו־$T$ אופרטור הגזירה הפורמלית. 
	כדי ש־$V$ יהיה ציקלי, צריך למצוא בסיס ציקלי שממדו הוא לכל היותר דרגת הנילפוטנטיות. נבחין ש־$n(T) = n  + 1$, וידוע ש־$\dim V = 2n + 1$, ולכן שרשרת מקסימלית באורך $n + 1$ ולכן לא יכול להיות בסיס שרשרת. לכן $V$ אינו $T$־ציקלי. 
	
	\rmark{יהי $T \co V \to V$ נילפוטנטית ו־$\dim V = n$ אז $n(T) \le n$ וישנו שוויון אמ''מ $V$ ציקלי. }
	
	\rmark{אם $T \co V \to V$ נילפוטנטית ו־$V$ ציקלי אז $V$ אי־פריק ל־$T$. }
	\begin{proof}
		נניח בשלילה שישנו פירוק לא טרוויאלי של $V$ ל־$T$. אז $V = U \oplus W$ לא טרוויאלים. נסמן $\dim U = k, \dim W = \ml$ וידוע $k, \ml < n$. בה''כ $k \ge \ml$. נסמן $B_v = (v, Tv \dots T^{n - 1}v)$. קיימים (ויחידים) $u \in U, w \in W$ כך ש־$v = u + w$. אז: 
		\[ T^{k}v = T^ku + T^kw \]
		אך משום ש־$T$ נילפוטנטית אז $T|_{U}, T|_{W}$ נילפוטנטית גם כן. ידוע $n(T|_{U}), n(T|_{W}) \le k$ ולכן בפרט $T^{k}(u) = T^{k}(w) = 0$ ולכן $T^{k}v = 0$ אבל $k < n \land T^k(v) \in B_v$. 
	\end{proof}
	
	\theo{תהי $T \co V \to V$ נילפוטנטית ונניח $U$ תמ''ו של $V$ הוא $T$־אינוואריאנטי וציקלי, אז עבור $T_{_U} =: S$: 
		\begin{enumerate}
			\item $\dim U \le n(T)$
			\item $\Img(T_{_U}) = T(U)$ ציקלי ו־$\dim T(U) = \dim U - 1$
	\end{enumerate}}
	\begin{proof}\,
		\begin{enumerate}
			\item $n(T) \ge n(T|_{U})$ וגם $\dim U = n(T_{_W})$
			\item $T(u) = T(\Sp(v, \dots T^{k}v)) = \Sp (Tv \dots T(T^k v))$ ז''א $T(U) = \Sp(Tv \dots T^k v)$, זו קבוצה בת''ל ופורתש את $T(U)$ ולכן $\dim T(U) = \dim U - 1$. 
		\end{enumerate}
	\end{proof}
	
	\defi{$U \subseteq V$ תמ''ו ציקלי ייקרא ציקלי מקסימלי אם $\dim U = n(T)$. }
	\theo{לכל $V$ מ''ו, $T \co V \to V$ נילפוטנטית קיים תמ''ו ציקלי מקסימלי. }
	\begin{proof}
		קיים $v \in V$ כך ש־$T^{n(T) - 1}v \neq 0$. אז $v \neq 0$ ו־$v, Tv, \dots T^{n(T) - 1}$ ומטעה מקודם בת''ל ולכן $\Sp(v \dots T^{n(T) -1})$ תמ''ו ציקלי מקסימלי. 
	\end{proof}
	
	
	\theo{נניח $U \subseteq V$ תמ''ו ציקלי מקסימלי. אזי: 
		\begin{enumerate}
			\item אם $T(U) \subseteq T(V)$ הוא גם ציקלי מקסימלי. 
			\item $U \cap T(V) = T(U)$ 
	\end{enumerate}}
	\begin{proof}\,
		\begin{enumerate}
			\item $U$ – ציקלי. לכן $\dim T(U) = \dim U - 1$. 
			טענה: 
			\[ \dim T(U) = n\cl{T|_{{T(V)}}} = n(T) - 1 \]
			וסיימנו. 
			\item ידוע $T(U) \subseteq U$ כי $U$ ציקלי ולכן שמור, וכן $U \subseteq V$ והסקנו $T(U) \subseteq T(V)$, לכן $T(U) \subseteq U \cap T(V)$
			
			עתה נוכיח שוויון באמצעות שיקולי ממד. אם לא היה שוויון אז: 
			\[ T(U) \subsetneq U \cap T(V) \subseteq U \implies \dim T(U) < \dim (U \cap T(V)) \le \dim (T(U)) + 1 \implies U \cap T(V) = U \]
			זו סתירה כי: 
			\[ U \cap T(V) \subseteq T(V) \implies n(T|_{{T(v)}}) = n(T) - 1 \implies n(T) = \dim U \le n(T) - 1 \]
			
		\end{enumerate}
	\end{proof}
	
	\subsubsection{ניסוח צורת מייקל ג'ורדן לאופרטור נילפוטנטי}
	\begin{Theorem}[המשלים הישר לתמ''ו ציקלי מקסימלי]
		נניח $T \co V \to V$ ט''ל לינארית נילפוטנטית, $U \subseteq V$ תמ''ו ציקלי מקסימלי אז קיים $W \subseteq V$ תמ''ו $T$־איוואריאנטי כך ש־$V = U \oplus W$. 
	\end{Theorem}
	\begin{proof}
		נוכיח באינדוקציה על $n = n(T)$. 
		\begin{itemize}
			\item[בסיס: ]אם $n(T) = 1$ אז $T = 0$ אז כל $W \subseteq V$ הוא $T$־איוואריאנטי. והיות שכל קבוצה בת''ל ניתנת להשלמה לבסיס, אז $U = \Sp(v)$ אז $W = \Sp(v_2 \dots v_m)$ כאשר $B_V = (v := v_1 \dots v_m)$. 
			\item[צעד: ](''צעד, מעבר, אותו דבר, תקראו לזה איך שבא לכם``) נניח שאנו יודעים את נכונות הטענה עבור $n = n(T) - 1$. נוכיח עבור $n = n(T)$. נצמצם את $T$ ל־$T|_{{T(V)}}$. ידוע $T(U) \subseteq T(V)$ ציקלי מקסימלי. לכן, לפי ה.א. קיים $W_1$ הוא $T$־איוואריאנטי כך ש־$T(V) = T(U) \oplus W_1$. 
			
			נגדיר $W_2 = \{v \in V \mid Tv \in W_1\}$. אז \lem{(''למה א``) $U + W_2 = V$ (לאו דווקא סכום ישר) וגם $U \cap W_1 = \{0\}$. }
			\lem{(``למה ב'') בהינתן $W_1 \subseteq W_2$ ו־$U \subseteq V$ תמ''ו כך ש־$U + W_2 = V$ וגם $U \cap W_1 = \{0\}$ אז קיים $W' \subseteq V$ כך ש־$W_1 \subseteq W' \subseteq W_2$ וגם $U \oplus W' = V$} 
			
			נניח שהוכחנו את הלמות. יהי $w \in W_1$ אז $T(w) \in W_1$ ולכן $w \in W_2$ ולכן $W_1 \subseteq W_2$.          
			אז מצאנו $W'$ תמ''ו של $V$ כך ש־$W_1 \subseteq W' \subseteq W_2$. יהי $w \in W'$ בפרט $w \in W_2$ ולכן $T(w) = W_1 \subseteq W'$. 
		\end{itemize}
		ולכן מש''ל המשפט. 
	\end{proof}
	
	הוכחת למה ב' היא תרגיל רגיל בלינארית 1א שאין ערך להביא את הוכחתו. 
	
	הוכחת למה א' גם היא לא מעניינת במיוחד, אבל אותה המרצה כן הוכיח: \begin{proof}
		יהי $v \in V$, נביט ב־$T(v)$. קיימים $u \in U, w_1 \in W_1$ כך ש־: 
		\[ Tv = Tu + w_1 \implies Tv - Tu = w_1 \implies T(v - u) = w_1 \in W_1 \]
		ידוע $v = v - u + u$. לכן $T(v - u) \in W_1 \implies v - u \in W_2$. 
		
		אזי משהו $V = U + W_2$ ו־$W_1 \subseteq T(V)$ ולכן: 
		\[ U \cap W_1 = U \cap (T(V) \cap W_1) \]
		ידוע ש־: 
		\[ U \cap W_1 = (U \cap T(V)) \cap W_1 \implies U \cap T(V) = T(U) \implies T(V) = T(U) \oplus W_1 \implies U \cap W_1 = T(U) \cap W_1 = \{0\} \]
	\end{proof}
	
	\cola{$T \co V \to V$ ט''ל נילפוטנטית אז $V$ אי־פריק ל־$T$ אמ''מ $V$ ציקלי. }
	\begin{proof}\,
		\begin{itemize}
			\item[$\implies$]זהו משפט שכבר הוכחנו
			\item[$\impliedby$]נניח $V$ אי־פריק. אז קיים $U \subseteq V$ תמ''ו ציקלי מקסימלי. לפי המשפט קיים $W \subseteq V$ תמ''ו $T$־איוואריאנטי כך ש־$T = U \oplus W$. ידוע $U, W$ תמ''וים איוואריאנטי. אם $U = \{0\}$ אז $V = 0$ ובפרט ציקלי. אחרת, מאי־פריקות $V$ ל־$T$, נסיק ש־$W = \{0\}$ ולכן $V = U$ ציקלי. 
		\end{itemize}
	\end{proof}
	
	\begin{Theorem}[משפט ג'ורדן בעבור $T$ נילפוטנטית 1]
		תהי $T \co V \to V$ נילפוטנטית אז קיים פירוק של $V$ לסכום ישר של $V = \bigoplus U_i$ כאשר $U_i$ הם $T$־ציקליים. 
	\end{Theorem}
	\begin{proof}
		באינדוקציה על המשפט הקודם: נמצא ב־$V$ ציקלי מקסימילי כלשהו. אז קיים $W \subseteq V$ תמ''ו $T$־שמור כך ש־$V = U_1 \oplus W$. ידוע $T|_{W} \co W \to W$ נילפוטנטית, וכעת באינדוקציה שלמה על $\dim V$. 
	\end{proof}
	
	\begin{Theorem}[משפט ג'ורדן בעבור ט''ל נילפוטנטית 2]
		עבוד $T \co V \to V$ נילפוטנטית, קיים בסיס $B$ של $V$ שהוא איחור של שרשראות. 
	\end{Theorem}
	\cola{בעבור $B$ בסיס מג'רדן, נסיק: 
		\[ [T]_B = \pms{\square & 0 & \dots &0 \\ 0 & \square & \ddots & \vdots \\ \vdots & \ddots & \ddots & 0 \\ 0 & \cdots & 0 & \square} \]
		כאשר הבלוקים הם בלוקי ג'ורדן אלמנטרי. בלוק ג'ורדן מהצורה הבאה יקרא אלמנטרי ויסומן: 
		\[ J_n(0) = \pms{\vert &  & \vert \\ T(v) & \cdots & T(T^kv) \\ \vert & & \vert} = \pms{0 & 0 & \cdots & 0 \\ 1 & \ddots & \ddots & \vdots \\ \vdots & \ddots & 0 & 0 \\ 0 & 0 & 1 & 0} \]
		(יש ספרים בהם זה ה־transpose של זה). }
	
	\begin{Theorem}[יחידות צורת ג'ורדן בעבור ט''ל נילפוטנטית]
		עבור $T \co V \to V$ נילפוטנטית, אז בכל הפירוקים של $V = \bigoplus U_i$ עבור $U_i$ ציקליים (אי־פריקים) אז מספר תתי־המרחב מממיד נתון הוא זהה עבור כל פירוק. 
	\end{Theorem}
	
	\begin{proof}
		באינדוקציה על $n  = n(T)$. 
		\begin{itemize}
			\item עבור $n = 1$, העתקת ה־‏$0$. $V$ מתפרק לסכום ישר של מרחבים מממד $1$. 
			\item צעד, נניח נכונות עבור $n \in \N$. נניח ש־$n(T) = n + 1$. נסמן פירוק: 
			\[ V = \bigoplus_{i = 1}^{k} U_i = \bigoplus_{i = 1}^{\ml} W_i \]
			נסדר את $(u_i)_{i =1}^{k}$ לפי גודל מימד, ונניח שרשימת הגדלים היא: 
			\[ (\underbrace{1, 1, \dots 1 }_{\times s}< a_1 \le \dots \le a_{p}) \implies s + p := k \]
			רשימת הממדים מגודל $1$ ועוד כל השאר. נעשה כנ''ל עבור $(w_i)_{i = 1}^{\ml}$ ונקבל: 
			\[ (\underbrace{1, 1, \dots 1}_{\times t} < b_1 \le \dots \le b_r) \implies t + r := \ml \]
			ידוע:
			\[ T(v) = \bigoplus_{i = 1}^{k} T(U_i) = \bigoplus_{i = 1}^{k} T(W_i), \quad n\cl{T|_{{T(v)}}} = n, \quad p = r, \quad \forall i \co a_i - 1 = b_i - 1 \implies a_i = b_i \]
			(הפירוק ל־$s$ ו־$t$ דרוש כדי שהפירוק לעיל לא יכלול אפסים כאשר מפעילים את $T$)
			(ידוע $a_i - 1 = b_i - 1$ כי אינדקס הנילפוטנטיות קטן ב־1 בהחלת $T$)
			
			משפט הממדים השני אומר ש־: 
			\[ a_i = \dim U_i = \dim \ker T|_{{U_i}} + \underbrace{\dim \Img T|_{{U_i}}}_{a_i - 1} \implies \dim \ker T|_{{U_i}} = 1 \]
			מהטענה השנייה בלמה: 
			\[ \begin{aligned}
				\ker T = \bigoplus_{i = 1}^{k} \ker T|_{{U_i}} \implies \dim \ker T &= \sum_{i = 1}^k \dim \ker T|_{{U_i}} = k \\
				&= \sum_{i = 1}^{k} \dim \ker T_{W_i} = \ml
			\end{aligned}\implies k = \ml \implies s = t \]
		\end{itemize}
	\end{proof}
	
	במילים אחרות: כל מטריצה מייצגת של ט''ל נילפוטנטית דומה למטריצת ג'ורדן יחידה, עד כדי שינוי סדר הבלוקים. 
	
	(למה זה שקול? כי הגודל של בלוק הוא הממד של התמ''ו שנפרש ע''י וקטורי הבסיס שמתאימים לעמודות הללו)
	
	למעשה, בכך הבנו לחלוטין כיצד העתקות נילפוטנטיות מתנהגות. עשינו רדוקציה למקרה הפרטי של נילפוטננטית, ועתה ננסה להבין את המקרה הכללי. ניעזר בתוצאה 3 ממשפט הפירוק הפרימרי לשם כך. 
	
	\lem{נניח $V = \bigoplus_{i = 1}^{k} U_i$ כאשר $U_i$ הוא $T$־איוואריאנטי (אין צורך להניח נילפוטנטיות), אז: 
		\begin{enumerate}[A.]
			\item \hfil $\displaystyle T(V) = \bigoplus_{i = 1}^{k}T(U_i)$
			\item \hfil $\displaystyle \ker T = \bigoplus_{i = 1}^k \ker T(U_i)$
	\end{enumerate}}
	הוכחה: נותר כתרגיל בעבור הקורא. 
	
	\subsection{צורת ג'ורדן לאופרטור לינארי כללי}
	
	\defi{בלוק ג'ורדן אלמנטרי עם ערך $\lg$ הוא בלוק מהצורה: 
		\[ J_n(\lg) = J_n(0) + \lg I_n = J_n(0) = \pms{\lg & 0 & \cdots & 0 \\ 1 & \ddots & \ddots & \vdots \\ \vdots & \ddots & \lg & 0 \\ 0 & 0 & 1 & \lg} \]
	}
	%	למה זה הגיוני? כי הסיבה שעשינו מלכתחילה רדוקציה ל־$T$ נילפוטנטית היא כי $T - \lg I$ היא $T$־אינוו' ונילפוטנטית, ועתה רק נותר להוסיף את ה־$\lg I$ חזרה לקבלת המקרה הכללי. 
	\defi{בהינתן $T \co V \to V$, בסיס $B$ נקרא \textit{בסיס מג'רדן} אם $[T]_B$ היא מטריצה עם בלוקי ג'ורדן מינימליים על האלכסון. }
	
	
	\begin{Theorem}[משפט ג'ורדן]
		לכל העתקה $T \co V \to V$ כאשר $V$ מונ''ס מעל שדה סגור אלגברית $\K$, קיים בסיס מג'רדן. 
	\end{Theorem}
	
	\textbf{מה עומד לקרות? }
	\begin{enumerate}
		\item נפרק את המרחב $V$ לתתי־מרחבים, שכל אחד מהם משוייך לערך עצמי $\lg_i$. נעשה זאת בשתי גישות – הראשונה באמצעות משפט הפירוק הפרימרי, והשנייה באמצעות פירוק למרחבים עצמיים מוכללים (שני הפירוקים מניבים את אותם המרחבים). 
		\item נתבונן על המרחבים האלו, ונסיק שיש העתקה ציקלית עליהם, שאנחנו כבר מכירים את צורת הג'ורדן שלה. היא תאפשר לנו לפרק את המרחבים שקיבלנו לתתי־מרחבים ציקליים, עם בסיס שרשרת שנותן לנו צורת ג'ורדן. 
	\end{enumerate}
	
	\subsubsection{בעזרת פירוק פרימרי}
	
	ראשית כל, נוכיח את משפט ג'ורדן באמצעות משפט הפירוק הפרימרי שכבר ראינו. 
	\begin{proof}[הוכחה באמצעות פירוק פרימרי]
		נניח ש־$f_T(x)$ מתפצל לחלוטין. מהגרסה החלשה של משפר הפירוק הפרימרי (ראה הערה תחתיו), ממשפט קיילי־המילטון $f_T$ מאפס את $T$, ותחת הסימון $f_T(x) =: \prod_{i = 1}^{s}(x - \lg_i)^{d_{\lg}}$  מתקיים: 
		\[ V = \bigoplus_{i = 1}^{s} \underbrace{\ker ((T - \lg_i)^{d_{\lg}})}_{U_i} \]
		כאשר $U_1 \dots U_n$ הם $T$־איווארינאטים. 
		משום ש־$U_i$ האי־פריקים ביחס ל־$T$, ו־$T$ שמורים. היות שהם אי פריקים $f_{T|_{U_i}} = (x - \lg)^{n}$. נגדיר $S = T - \lg I$. אז $U_i$ הוא $T$־איוואריאנטי אמ''מ הוא $S$־איוואריאנטי (טענה שראינו בעבר). 
		ראינו ש־$S|_{U_i}$ היא נילפוטנטית שכן ממשפט הפירוק $(T - \lg_i)^{r_i}$ מאפס את $T|_{U_i}$ (אך לא בהכרח מינימלי, שכן $f_T$ לא בהכרח מינימלי) ולכן $(T - \lg_i)^{r_i} = S^{r_i} = f|_{U_i}(T) = 0$, כלומר $S|_{U_i}$ נילפוטנטית. ל־$S|_{U_i}$ הוכחנו קיום צורת ג'ורדן, משמע קיים לה בסיס מג'רדן $\bc_i$ כך ש־: 
		\[ [S|_{U_i}]_{\bc_i} = [T|_{U_i} - \lg I_V] = [T|_{U_i}]_{\bc_i} - \lg I \implies [T|_{U_i}]_{B_i} = \diag(J_{a_1}(0) \dots J_{a_n}(0)) + \lg_i I = \diag(J_{\lg_i}(0) \dots J_{a_n}(\lg_i)) \]
		
		%	לכן ל־$u_i$ יש בסיס שרשראות $B$ שעבורו $[S_{|{U_i}}]_B$ מורכבת מבלוקי ג'ורדן נילפוטנטיים כלומר: 
		%	\[ \csb{S|_{{U_i}}}_B = \pms{J_{j_1}(0) \\ & J_{j_2}(0) \\ && \ddots \\ &&&& J_{j_{p}}}(0) \]
		%	ולכן נקבל: 
		%	\[  \]
		לכן, נוכל לשרשר את הבלוקים הללו ולקבל $\bc = \bigcup_{i = 1}^{s} \bc_i$, המקיים: 
		\[ \csb{T|_{{U_i}}}_{\bc} = \diag\big\{[T|_{U_1}]_{\bc_1} \dots [T|_{U_s}]_{\bc_s} \big\} \]
		משום שכל אחד מ־$[T|_{U_i}]_{B_i}$ הוא בלוק ג'ורדן בעצמו, סה''כ נקבל: 
		\[ [T]_B = \diag(J_1(\lg_1) \dots J_n(\lg_j)) \]
		זוהי צורת הג'ורדן של מטריצה כללית. 
	\end{proof}
	במילים אחרות – נעזרנו בפירוק פרימרי ע''מ לפרק את המרחב למרחבים $T$־איווראינטים פריקים מינימליים (בהמשך נראה שאלו \textit{המרחבים העצמיים המוכללים} של $T$, שמקיימים כל מיני תכונות נחמדות) ואת המרחבים אליהם פירקנו, ניתחנו בעזרת צורת ג'ורדן להעתקות נילפוטנטיות. 
	
	\begin{Theorem}[יחידות צורת ג'ורדן הכללית]
		צורת ג'ורדן היא יחידה עד כדי סדר בלוקים. 
	\end{Theorem}
	\begin{proof}
		תהא צורת ג'ורדן עבור $T$ תהא צורת ג'ורדן עבור $T$. קיים בסיס $B$ שעבורו: 
		\[ [T]_B = \diag\{J_1(\lg_1) \dots J_k(\lg_k)\} \]
		אז: 
		\[ V = \bigoplus_{i = 1}^{k} U_i = \bigoplus \bar v_{\lg}, \ \bar v_{\lg} = \bigoplus_{i = s}^{\ml} U_i \]
		כאשר $\bar v_{\lg}$ הוא סכום של אי־פריקים שעבורם $T - \lg I$ נילפוטנטית. 
		
		מה ניתן להגיד על הממדים של ה־$u_i$־ים שמרכיבים את $\tl \vc_\lg$? הממדים שלהם נקבעים ביחידות, עד כדי סדר, כי היות ש־$u_i$ הם $T$־איוואריאנטי הם גם $(T - \lg I)$־איוואריאנטי, ולכן $[S_{|\tl \vc_\lg}]_{B_\lg}$ היא ג'ורדן נילפוטנטית ואז: 
		\[ \csb{T_{|\tl \vc_\lg}}_{B_\lg} = \csb{S_{|\tl \vc_\lg}}_{B_\lg} \]
	\end{proof}
	%TODO
	הגיון: המרחבים $v_\lg$ נקבעים ביחידות ללא תלות בפירוק שבחרנו. 
	
	הגיון אחר: כל בלוק מורכבת מהעתקות שבהן $T - \lg I$ נילפוטנטית (פירוק פרימרי). 
	
	\subsubsection{בעזרת מרחביים עצמיים מוכללים}
	בגישה הזו נוכל לפתח את צורת ג'ורדן למטריצה כללית ללא צורך בפירוק פרימרי, פולינום מינימלי, משפט קיילי־המילטון וכו'. זו גישה יותר אלמנטרית ופשוטה, ואם מבינים אותה האלגוריתם המסורבל למציאת צורת ג'ורדן הופך לאינטואיטיבי בהרבה. 
	\defi{\textit{המרחב העצמי המוכלל} של $\lg$ הוא מ''ו: 
		\[ \genein{\lg} := \bar \vc_{\lg} := \{v \in V \mid \exists n \in \N \co (T - \lg I)^nv = 0\} \]}
	\theo{המרחב העצמי המוכלל הוא מ''ו. }
	\cola{באופן מידי נסיק $V_{\lg} \subseteq \genein{\lg}$. }
	\defi{\textit{וקטור עצמי מוכלל} הוא וקטור $v \in V$ כך ש־$\exists i \in [n] \co T^{(i)}v = \lg v$. }
	
	\rmark{החלק הזה ואילך, אז סוף הפרק, הינו הרחבה שלי בלבד ואילו אינם משפטים המופיעים בקורס. עם זאת, המשפטים להלן מאפשרים להבין בצורה הרבה יותר טובה את צורת ג'ורדן, ולעיתים קרובות תצטרכו להוכיח אותם בעצמכם. }
	\rmark{מרגישים אבודים? אני ממליץ על \href{https://youtu.be/1L4hvyHGKsE}{הסרטון הבא}. }
	\theo{תהי העתקה $T$ כללית ו־$\lg \in \F$ סקלר, אז $\genein{\lg} = \nc(T - \lg I)^{\dim V}$ (כאשר $\nc$ המרחב המאפס/הקרנל של המטריצה)}
	\begin{proof}
		נוכיח באמצעות הכלה דו כיוונית. הכיוון $\nc(T - \lg I)^{\dim V} \subseteq \genein{\lg}$ טרוויאלי. יהי $v \in \nc(T - \lg I)^{j}$, אם $j < \dim V$ אז $(T - \lg)^{j}v = 0$ ולכן $(T - \lg I)^{\dim V}v = 0$ וסה''כ $v \in \nc(T - \lg I)^{\dim V}$, אחרת $j > \dim V$ ואז הוכחנו $\nc(T - \lg I)^{j} = \nc(T - \lg I)^{\dim V}$. נסיק מעקרון ההחלפה: 
		\[ \tl \vc_{\lg} = \bigcup_{j = 1}^{\inf}\nc(T - \lg I)^{j} = \bigcup_{j = 1}^{\mathclap{\dim V}}\nc(T - \lg I)^{j} \,\,\cup\,\, \bigcup_{\mathclap{j = \dim V}}^{\inf}(T - \lg I)^{j} = \nc (T - \lg I)^{\dim V} \]
	\end{proof}
	\theo{בהינתן $v$ ו''ע מוכלל של $T$, קיים (מהגדרה) ויחיד $\lg_i$ כך ש־$v \in \genein{\lg_i}$. }
	\begin{proof}
		ההוכחה בעיקר אלגברית ולא מעניינת במיוחד, יש צורך לפתח את הבינום של ניוטון. 
	\end{proof}
	מסתבר, שאפשר לפרק את המרחב למרחביים עצמיים מוכללים, ומשם אפשר להסיק מה קורה בהם ביתר פרטים בעזרת העתקות נילפוטנטיות. 
	
	\begin{Theorem}הטענות הבאות מתקיימות: 
		\begin{enumerate}
			\item $\genein{\lg_i}$ הוא $T$־איווריאנטי. 
			\item $(T - \lg_i I)|_{\genein{\lg_i}}$ נילפוטנטית. 
			\item מעל שדה סגור אלגברית, הריבוי האלגברי $d_{\lg_i}$ הוא $\dim \genein{\lg_i}$
		\end{enumerate}
	\end{Theorem}
	\begin{proof}\,
		\begin{enumerate}
			\item יהי $v \in \genein{\lg_i}$, אז קיים $k \le \fc(T)$ כך ש־$(T - \lg I)^{k}v = 0$. נפעיל את $T$ על שני האגפים ונקבל $(T - \lg I)^{k + 1}v = T(0) = 0$ ולכן $Tv \in \genein{\lg_i}$. סה''כ $\genein{\lg_i}$ הוא $T$־איווריאנטי. 
			\item נגדיר $S = (T - \lg_i I)|_{\tl \vc_{\lg i}}$. לכן לכל $v \in \dom S = \genein{\lg_i}$ מתקיים $\exists k_v \co (T - \lg_i I)^{k_v} = S^{k_v} = 0$. משום ש־$k_v \le \fc(T) \le n$ נוכל לטעון ש־$\forall v \in \dom S \co S^{n}v = 0$, ומהגדרה $S$ נילפוטנטית כדרוש. 
			\item (הוכחה זו נכתבה בעזרתו האדיבה של chatGPT) נסמן $U = \genein{\lg_i}$, ונרחיב את הבסיס של $U$ לבסיס של $V$ כך שנוצר מ''ו $W$ כך ש־$U \oplus W = V$. ממשפט $p_T(x) = p_{T|_W}(x) \cdot p_{T|_U}(x)$. מסעיף קודם ידוע ש־$S:= (T - \lg_i I)$ ש־$S|_{U}$ נילפוטנטית, לכן $n$ כך ש־$S|_{U}^{n} = 0$. נכתוב את $T$ באופן הבא: $S|_{U} = T|_U - \lg I \implies T|_U = S|_{U} + \lg I$. נקבל שתי הבחנות: 
			\begin{itemize}
				\item $\lg_i$ הוא הע''ע היחיד של $T|_U$, והוא ע''ע $\lg_i \in U = \genein{\lg_i}$ ולכן $\lg_i$ ע''ע של $T|_U$, והיחידות נובעת מכך שכל ע''ע מוכלל שייך לע''ע יחיד של $T$. 
				\item $S|_W$ הפיכה, שכן בבירור $W \supseteq \ker S|_W \subseteq \ker(T - \lg_i I) = V_{\lg_i} \subseteq \genein{\lg_i} = U$ ולכן $\ker S \subseteq W \cap U = \{0\}$. 
			\end{itemize}
			נסיק משתי הטענות הללו שתי מסקנות: 
			\begin{itemize}
				\item מהיות $\lg_i$ הע''ע היחיד של $T|_U$, ומהיות $\deg p_{T|_U} = \dim U$, ויחדיו עם ההנחה שאנחנו בשדה סגור אלגברית, $p_{T|_U}$ בהכרח מורכב מ־$\dim U$ גורמים לינאריים שהם $(x - \lg_i)$. 
				\item $\lg_i$ איננו ע''ע של $T|_W$, בגלל שאם (בשלילה) $\lg_i$ ע''ע של $T|_W$ עם ו''ע $v$ אז $\lg_i v = T|_W(v) = Sv + \lg_i v$ ומחיסור אגפים נקבל $S|_Wv = 0$, כלומר $v = 0$ (כי $S|_W$ הפיכה) ואז $v$ לא ו''ע וסתירה. 
			\end{itemize}
			סה''כ, מהיות $p_T(x) = p_{T|_W}(x) \cdot p_{T|_U}(x)$, נקבל שהריבוי האלגברי של $(x - \lg_i)$ בא אך ורק מ־$p|_{T|_U}$ ושם הריבוי הוא $\dim U$, כלומר סה''כ הריבוי האלגברי של $\lg_i$ בהעתקה $T$ הוא $\dim U = \dim \genein{\lg_i}$ כדרוש. 
		\end{enumerate}
	\end{proof}
	\defi{$v$ הוא \textit{ו''ע עצמי מורחב של $\lg_i$ מדרגה $k$} אם הוא ו''ע עצמי מורחב של $\lg_i$ כך ש־$v \in \ker(T - \lg I)^{k} \setminus \ker(T - \lg I)^{k - 1}$, כאשר בסיס $k = 1$ מוגדר להיות $v \in V_{\lg_i}$. }
	
	
	\begin{Theorem}[פירוק המרחב למרחבים עצמיים מוכללים]
		נניח שאנחנו במ''ו סגור אלגברית (אפשר להרחיב לכזה במידת הצורך). אז ל־$T$ יש ע''עים $\lg_1 \dots \lg_k$ כלשהם. בהינתן $V$ מ''ו ו־$T$ העתקה לינארית, מההרחבה יש לה ערכים עצמיים $\lg_1 \dots \lg_k$ כלשהם. אזי: 
		\[ V = \bigoplus_{i = 1}^{k} \genein{\lg_i} \]
	\end{Theorem}
	\begin{proof}
		נתחיל מלהוכיח שהחיתוך בין שני מרחבים עצמיים מוכללים ריק. זה נובע ישירות מכך שכל שני ע''ע עצמיים מוכללים שייכים לע''ע רגיל יחיד של $T$. ניעזר בכך ש־$\dim \genein{\lg_i} = d_{\lg_i}$, ונקבל: 
		\[ \begin{cases}
			\sumnio \dim \genein{\lg_i} = \sumnio d_{\lg_i} = n \\
			\forall i \in [k] \co \genein{\lg_i} \subseteq V \\
			\forall i, j \in [k] \co \genein{\lg_i} \cap \genein{\lg_j} = \{0\}
		\end{cases} \]
		כאשר $d_{\lg_i}$ הריבוי האלגברי של $\lg_i$, וידוע סכום הריבויים האלגבריים הוא $n$ שכן $p_T(x)$ פולינום ממעלה $n$. לכן ממשפט יש סכום ישר כדרוש. 
	\end{proof}
	
	עתה נוכיח מחדש את משפט ג'ורדן, אך הפעם ללא תלות בפולינום מינימלי ופירוק פרימרי. 
	\begin{proof}[הוכחה באמצעות מרחבים עצמיים מוכללים]
		תהי העתקה $T$. מפריקות הפולינום האופייני יש לה $\lg_1 \dots \lg_k$ ע''עים כלשהם. ממשפט: 
		\[ V = \bigoplus_{i = 1}^{k} \genein{\lg_i} \]
		עוד ידוע שההעתקה $S_i = (T - \lg_i)_{\genein{\lg_i}}$ נילפוטנטית. כבר הוכחנו את צורת ג'ורדן עבור העתקות נילפוטנטיות ולכן ל־$S_i$ קיים בסיס מג'רדן $\bc_i$. נבחין ש־: 
		\[ T|_{\genein{\lg_i}} = S_i + \lg_i \implies \csb{T|_{\genein{\lg_i}}}_{\bc_i} = \underbrace{\diag\{J_{a_1}(0) \dots J_{a_\ml}(0)\}}_{[S_i]_{\bc_i}} + \lg I = \diag\big(J_{a_1}(\lg_i) \dots J_{a_\ml}(\lg_i)\big) \]
		ולכן אפשר לשרשר את הבסיסים לכדי בסיס מג'רדן: $\bc = \bigcup_{i = 1}^{k} \bc_i$, ואכן: 
		\[ [T]_\bc = \diag\cl{\csb{T|_{\genein{\lg_i}}}_{\bc_i} \mid i \in [k]} = \diag\big(J(\lg_1) \dots J(\lg_1) \dots J(\lg_k) \dots J(\lg_k)\big) \]
		שרשור של בלוקי ג'ורדן. 	
	\end{proof}
	
	\rmark{מיחידות צורת ג'ורדן, הצורה המתקבלת מפירוק פרימרי ומפירוק למרחבים עצמיים מוכללים היא זהה. דרך אחרת לראות את זה, היא שהמרחבים אליהם פירקנו פרימרית שהם $(T - \lg_i)^{d_{\lg}} = \genein{\lg_i}$ בכל מקרה. }
	
	\subsection{תוצאות מצורת ג'ורדן}
	\theo{כמות בלוקי הג'ורדן לע''ע $\lg$ היא הריבוי הגיאומטרי. }
	\theo{כמות הוקטורים בבסיס המג'רדן המשוייכים ל־$\lg$ הוא הריבוי האלגברי $d_{\lg}$ (ניסוח אחר: סכום גדלי הבלוקים השייכים ל־$\lg$ בצורת הג'ורדן הוא $d_{\lg}$). }
	\begin{proof}
		ראינו בצורת ג'ורדן בעזרת פירוק למרחבים עצמיים מוכללים, שמספר הוקטורים השייכים ל־$\lg$ הוא $\dim \genein{\lg}$ וידוע שזה מ''ו מממד $d_{\lg}$. סה''כ הראינו את הדרוש. 
	\end{proof}
	
	\theo{בלוק הגו'רדן המשויך ל־$\lg$ הגדול ביותר, הוא הריבוי של $(x - \lg)$ בפולינום $m_T(x)$. } \begin{proof}
		ראינו שבלוק הג'ורדן $J_a(\lg)$ מגיע מפירוק ג'ורדן של $S = (T - \lg)_{\genein{\lg}}$. הבלוק הכי גדול בצורת הגו'רדן של $S$ נילפוטנטית, היא השרשרת הכי ארוכה של $S$ ב־$\genein{\lg}$. משום ש־$\genein{\lg_i} = \{\ker S^{k} \mid k \in [n]\} = S^{n(S)}$, השרשרת הארוכה ביותר האפשרית היא $v, Sv \dots S^{n(S)}v$ והיא קיימת כי הסדרה הזו בת''ל עבור $v$ כלשהו (אחרת $n(S)$ לא החזקה המינימלית שמאפסת את $S$ וסתירה). 
		
		ראינו ש־$m_T$ הוא ה־$\lcm$ של הצמצום של $T$ למרחבים $T$־אינווראינטים, ומשום שכל $\genein{\lg_k}$ בעל פולינום אופייני $(x - \lg_k)^{d_{\lg_i}}$, ו־$\gcd((x - \lg_i)^{k}, (x - \lg_j)^{m}) = 1$, ובגלל ש־$m_{\smash{T|_{\genein{\lg}}}}\!\!(x)$ מחלק את $p_{\smash{T|_{\genein{\lg}}}}\!\!(x)$, אז $m_{\smash{T|_{\genein{\lg_i}}}} \!\! (x) = (x - \lg)^{k}$ עבור $k \in [r_\lg]$ כלשהו. 
		 אז: 
		\[ \forall i \neq j \in [k] \co \gcd\cl{T|_{\genein{\lg_i}}(x), T|_{\genein{\lg_j}}(x)} = 1 \]
		דהיינו, ה־$\lcm$ זה פשוט כפל של הפולינומים המינימליים של $T|_{\genein{\lg}}$. לכן, תחת הסימון $m_\lg$ להיות הריבוי של $\lg$ בפולינום $m_T$, בהכרח $m_{T|_{\genein{\lg}}}\!\!(x) = (x - \lg)^{m_i}$. מהגדרת פולינום מינימלי, $m_\lg$ הוא המינימלי כך ש־$(T - \lg)^{m_i} = 0$ כלומר $m_\lg$ המינימלי כך ש־$S^{m_\lg} = 0$. סה''כ $m_\lg$ דרגת הנילפוטנטיות של $S$. הראנו ש־$n(S)$ השרשרת המקסימלית בצורת הג'ורדן של $S$, וסה''כ בלוק הגו'רדן הגדול ביותר של $J(\lg)$ הוא $m_\lg$ הריבוי של $(x - \lg)$ ב־$m_T(x)$. 
	\end{proof}
	\theo{תהי $A \in M_n(\K)$ מטריצה, כאשר $\K$ סגור אלגברית. אז $A \sim A^{T}$. }
	\begin{proof}
		ממשפט ג'ורדן ל־$A$ יש צורת ג'ורדן $\Lg$, כלומר קיימת $P$ הפיכה כך ש־$P\op \Lg P = A$ ו־$\Lg$ מטריצה אלכסונית עם בלוקי גו'רדן. נבחין בכך ש־$A^{T} = (P\op \Lg P)^{T} = P^{T}\Lg^{T}(P\op)^{T}$, כלומר $A^{T} \sim \Lg^{T} \land A \sim \Lg$. נותר להוכיח $\Lg \sim \Lg^{T}$, כלומר, כל בלוק ג'ורדן $J_i(\lg) \sim J_i(\lg)^{T}$. טענה זו אכן מתקיימת בעבור מעבר לבסיס הסדור $(e_1 \dots e_n) \to (e_n \dots e_1)$. סה''כ אכן כל מטריצה דומה לשחלוף שלה. 
	\end{proof}
	
	
	\npage
	\section[תבניות בי־לינאריות]{\en{Bi-Linear Forms}}
	\subsection{הגדרות בסיסיות בעבור תבניות בי־לינאריות כלליות}
	
	\defi{יהי $V$ מ''ו מעל $\F$. פונקציונל לינארי $\vphi$ מעל $V$ הוא $\vphi \co V \to \F$. }
	\rmark{ראה הרחבה על פונקציונלים לינארים ומרחבים דואלים בסוף הסיכום. }
	\defi{יהיו $V, W$ מ''וים מעל $\F$. תבנית בי־לינארית על $V \times W$ הינה העתקה $f \co V \times W \to \F$ כך ש־$\forall v_0 \in V \ \forall w_0 \in W$ כך שהעתקות $w \mapsto f(v_0, w), \ v \mapsto (v, w_0)$ הן פונקציונליים לינאריים. }
	אינטואיטיבית, זו ההעתקה לינארית בכל אחת מהקורדינאטות בנפרד (בדומה לדוגמה לדטרמיננטה, שהיא העתקה מולטי־לינארית ולינארית בכל אחת מהשורות בנפרד)
	\theo{הטענה הבאה שקולה לכך ש־$f$ בי־לינארית. יהיו $\forall v \in V, \ w \in W, \ \ag \in \F$: 
		\[ \begin{aligned}
			\forall v_1, v_2 \in V \co f(v_1 + v_2, w) &= f(v, w) + f(v_2, w) \\
			\forall w_1, w_2 \in W \co f(v, w_1, w_2) &= f(v, w_1) + f(v, w_2) \\
			f(\ag v, w) &= \ag f(v, w) = f(v, \ag w)
		\end{aligned} \]}
	
	בשביל העתקות $n$־לינאריות צריך טנזור $n$ ממדי. זה לא נעים ויודעים מעט מאוד על האובייקטים הללו. בפרט, בעבור העתקה בי־לינארית נראה שנוכל לייצג אותה באמצעות מטריצות, בלי טנזור ובלגנים – שזה נחמד, וזו אחת הסיבות שאנו מתעסקים ספציפית עם העתקות בי־לינאריות (פרט לכך שמאוחר יותר נעסוק גם במכפלות פנימיות, וחלק מהתוצאות על ההעתקות בי־לינאריות יעזרו לנו להגיד דברים על מטריצות). 
	
	\textbf{דוגמאות. }
	\begin{enumerate}
		\item תבנית ה־$0$: \hfill $\forall v,w \co f(v, w) = 0$ 
		\item נגדיר $V = W = \R^2$, אז \hfill $f\big(\binom{x}{y}, \binom{u}{v}\big) = 2xu + 5xv - 12yu$
		\item (חשוב)  על $\F^n$: \defi{לכל שדה $\F$ מוגדרת \textit{התבנית הבי־לינארית הסטנדרטית} היא: 
			\[ f\cl{\pms{x_1 \\ \vdots \\ x_n}, \pms{y_1 \\ \vdots \\ y_n}} = \sum_{i = 1}^{n}x_iy_i \]}
		\item יהיו $\phi \co V \to \F, \ \psi \co W \to \F$ פונקציונליים לינאריים: \hfill $f(v, w) = \phi(v) \cdot \psi(w)$
		\item הכללה של 4: יהיו $\phi_1 \dots \phi_k \co V \to \F$ פונקציונליים לינאריים וכן $\psi_1 \dots \psi_k \co W \to \F$ פונקציונליים לינאריים. אז ההעתקה הבאה בי־לינארית: \hfill $f(v, w) = \sum_{i = 1}^{k}\phi_i(v) \psi_i(w)$
	\end{enumerate}
	הרעיון: ברגע שנקבע וקטור ספציפי נקבל לינאריות של הוקטור השני. 
	
	\rmark{במקרה ש־$\F = \R$ לעיל, התבנית הבי־לינארית הסטנדרטית משרה את הגיאומטריה האוקלידית. כלומר $v \perp u \iff f(v, u) = 0$. }
	
	\rmark{בעתיד נראה שכל תבנית בי־לינארית נראית כמו מקרה 5. }
	
	\theo{נסמן את מרחב התבניות הבי־לינאריות על $V \times W$ בתור $B(V, W)$. זהו מ''ו מעל $\F$. }
	אני ממש לא עומד להגדיר את החיבור והכפל בסקלר של המשפט הקודם כי זה טרוויאלי והמרצה כותב את זה בעיקר בשביל להטריל אותנו. 
	
	\textbf{דוגמה חשובה אחרת. }\theo{נסמן ש־$\dim V = n, \ \dim W = m$ ותהי $A \in M_{n \times m}(\F)$. יהי $\ac$ בסיס ל־$B$, $\bc$ בסיס ל־$W$. אז: \[ f(u, w) = [v]_\ac^T \cdot A \cdot [w]_{\bc} \] העתקה בי־לינארית. }
	\begin{proof}
		נְקַבֶּע $v$ כלשהו:
		\[ [v]_\ac^T \cdot A =: B \in M_{1 \times m}, \ g(w) := f(v, w) = B[w]_\bc \]
		נוכיח ש־$g$ לינארית: 
		\[ \forall w_1, w_2 \in V, \ \lg_1, \lg_2 \in \F \co  g(\lg_1w_1 + \lg_2w_2) = B[\lg_1w_1 + \lg_2w_2]_\bc = \lg_1(B[w_1]_\bc) + \lg_2(B[w_2]_\bc) = \lg_1g(w_1) + \lg_2 g(w_2) \]
		נְקַבֶּע $w$, ובאופן דומה נגדיר $C = A[w]_\bc \in M_{n \times 1}(\F)$ ו־$h(v) := f(v, w) = [v]_B^{T}C$: 
		\[  \forall v_1, v_2 \in V, \ \lg_1, \lg_2 \in \F \co h(\lg_1v_1 + \lg_2v_2) = [\lg_1v_1 + \lg_2v_2]_\bc^T = \lg_1([v_1]_\bc^TC) + \lg_2([v_2]_\bc^TC) = h(v_1) + h(v_2) \]
	\end{proof}
	(''זה $\ac$ mathcal, אתם תסתדרו`` – המרצה ברגע שיש לו שני $A$־ים על הלוח)
	
	\defi{בהינתן תבנית בי־לינארית $f \co V \times W \to \F$ ונניח ש־$\ac$ בסיס ל־$V$, $\bc$ בסיס ל־$W$. נגדיר את המטריצה המייצגת את $f$ ביחס לבסיסים $\ac, \bc$ ע''י $A \in M_{n \times m}(\F)$ כאשר $(A)_{ij} = f(v_i, w_j)$ (נסמן $\ac = (v_i)_{i =1}^n \ \bc = (w_i)_{i = 1}^{m}$)}
	
	\theo{$f(v, w) = [v]_{\ac}^T A [w]_\bc$}
	\begin{proof}
		
		קיימים ויחידים $\ag_1 \dots \ag_n, \ \bg_1 \dots \bg_m \in \F$ כך ש־$v = \sum \ag_i v_i, \ w = \sum b_iw_i$. 
		כלומר: 
		\[ [v]_\ac^T = (\ag_1 \dots \ag_n), \ [w]_B = \pms{\bg_1 \\ \vdots \\ \bg_m} \]
		ומכאן פשוט נזרוק אלגברה: 
		\begin{align*}
			f(v, w) = &f\cl{\sum_{i = 1}^{n}\ag_i v_i, \ w} = \sum_{i = 1}^{n}\ag_i f(v_i, \ w) \\
			= &\sum_{i = 1}^{n}\ag_i f\cl{v, \ \sum_{j = 1}^{n}\bg_j w_j} = \sum_{i = 1}^{n}\ag_i \cl{\sum_{j = 1}^{m}\bg_j f(v_i, w_j)} \\
			= &\sum_{\mathclap{i, j \in [n] \times [m]}} \ag_i f(v_i, w_j)\bg_j \\
			= &\cl{\sum_{i = 1}^{n}\ag_i a_{i1}, \ \sum_{i = 1}^{n}\ag_i a_{i2}, \ \dots, \ \sum_{i = 1}^{n}\ag_i a_{im}}\pms{\bg_1 \\ \vdots \\ \bg_m} \\
			= &\,(\ag_1 \dots \ag_n)\pms{a_{11} &\cdots &a_{1m} \\ \vdots && \vdots \\ a_{n1} & \cdots & a_{nm}}\pms{\bg_1 \\ \vdots \\ \bg_{m}} 
		\end{align*}
	\end{proof}
	
	
	\noti{נאמץ לסיכום הזה את הסימון $[f]_{\ac, \bc}$ עבור המטריצה המייצגת של $f$ בי־לינארית. }
	(זהו \textbf{אינו} סימון רשמי בקורס אם כי בהחלט צריך להיות)
	\theo{עם אותם הסימונים כמו קודם: 
		\[ \psi \co B(v, w) \to M_{n \times m}(\F), \ f \mapsto [f]_{\ac, \bc} \]
		אז $\psi$ איזו'. }
	\begin{proof}
		נסמן את $[f]_{\ac, \bc} = A$ ואת $[g]_{\ac, \bc} = B$. אז: 
		\begin{itemize}
			\item \textbf{לינאריות. }
			\begin{align*}
				(\ps(f + g))_{ij} &= (f + g)(v_i, w_j) \\
				&= f(v_i, w_j) + g(v_i, w_j) \\
				&= (A)_{ij} + (B)_{ij} \\
				&= (\psi(f))_{ij} + (\psi(g))_{ij} \implies \psi(f + g) \\
				&= \psi(f) + \psi(g)
			\end{align*}
			באופן דומה בעבור כפל בסקלר: 
			\[ (\psi(\ag f))_{ij} = \ag f(v_i, w_j) = \ag (\psi(f))_{ij} \implies \psi(\ag f) = \ag \psi(f) \]
			\item \textbf{חח''ע. }תהי $f \in \ker \psi$, אז: $\psi(f) = 0 \in M_{n \times m} \implies \forall i,j \in [n] \times [m]\co f(v_i, w_j) = 0$ ולכן $\forall v \in V, w \in W \co f(v, w) = \sum_{{i, j \in [n] \times [m]}}\ag_i f(v_i, w_j)\bg_j = 0$
			(עם אותם הסימונים כמו קודם)
			\item \textbf{על. }תהי $A \n M_{n \times m}(\F)$. נגדיר $f(v, w) = [v]_\ac^T A[w]_\bc$ ואכן $f(v_i, w_j) = e_i^T A e_j = (A)_{ij}$. 
		\end{itemize}
	\end{proof}
	
	
	\textbf{תזכורת }(מלינארית 1)\textbf{. }מטריצת המעבר מבסיס $\bc$ לבסיס $\cc$ מוגדרת להיות $[I]^{\bc}_{\cc}$, היא מטריצה הפיכה, ומתקיים השוויון $[I]^{\cc}_{\bc}[T]^{\bc}_{\bc}[I]^{\bc}_{\cc} = [T]^{\cc}_{\cc}$. 
	\theo{יהיו $V, W$ מ''וים מעל $\F$ נניח $\ac, \ac' \subseteq V$ בסיסים של $V$ וכן $\bc, \bc' \subseteq W$ בסיסים של $W$. תהי $f \in B(V, W)$. 
		תהי המייצגת של $f$ לפי $\ac, \bc$ היא $A$ ותהי $A'$ המייצגת בבסיסים $\ac', \bc'$. תהי $P$ מטריצת המעבר מ־$\ac$ ל־$\ac'$ ו־$Q$ מטריצת המעבר מ־$\bc$ ל־$\bc'$, אז $A' = P^T AQ$. 
	}
	\begin{proof}
		ידוע: 
		\[ P[v]_{\ac} = [v]_{\ac'}, \ Q[w]_\bc = [w]_{\bc'} \]
		ואכן: 
		\[ f(v, w) = [v]_{\ac}^T A [w]_\bc = (P[v]_{\ac '})^TA(Q[w]_{\bc'}) = [v]_{\ac'}^T \, (P^T A Q) \, [w]_{\bc'} \implies A' = P^T A Q \]
		כדרוש. 
	\end{proof}
	
	\defi{עבור $f \in B(V, W)$ נגדיר את $\rk f = \rk A$ כאשר $A$ מייצגת אותה ביחס לבסיסים כלשהם. }
	\theo{$\rk f$ מוגדר היטב. }
	\begin{proof}
		כפל בהפיכה לא משנה את דרגת המטריצה (ו־transpose של מטריצה הוא הפיך), ומטריצת שינוי הבסיס הפיכה, דהיינו כפל מטריצות שינוי הבסיס לא משנות את דרגת המטריצה ולכן לכל שני נציגים אותה הדרגה. 
	\end{proof}
	
	\cola{תהא $f \in B(V, W)$ ונניח $\rk f = r$. אז קיימים בסיסים $\ac, \bc$ של $W, V$ בהתאמה כך ש־$[f]_{\ac, \bc} = \binom{I_R \, 0}{0\,\,\,\, 0}$}
	הרעיון הוא לדרג את כל כיוון, שורות באמצעות transpose ועמודות באמצעות המטריצה השנייה. אפשר גם לקבע בסיס, ולדרג שורות ועמודות עד שיוצאים אפסים (הוכחה לא נראתה בכיתה). 
	
	''חצי השעה הזו גרמה לי לשנוא מלבנים בצורה יוקדת`` – מעתה ואילך נתעסק במקרה בו $V = W$. נשתמש בבסיס יחיד. 
	
	\subsection{חפיפה וסימטריות}
	\defi{יהיו $A, A' \in M_n(\F)$, נאמר שהן \textit{חופפות} אם קיימת הפיכה $P \in M_n(\F)$ כך ש־$A' = P^TAP$. }
	\theo{מטריצות חופפות אמ''מ הן מייצגות את אותה התבנית הבי־לינארית. }
	\theo{אם $A, A'$ חופפות, אז: 
		\begin{enumerate}
			\item \hfil $\rk A = \rk A^T$
			\item \hfil $\exists 0 \neq c \in \F \co \det A' = c^2 \det A$
	\end{enumerate}}
	\begin{proof}
		הגדרנו $\rk f$ כאשר $f$ בי־לינארית להיות הדרגה של המייצגת את התבנית, וראינו שהיא לא תלויה בבסיס. בכך למעשה כבר הוכחנו את 1. עבור 2, מתקיים $A' = P^TAP$ ו־$P$ הפיכה (ולמעשה מטריצת מעבר בסיס) ולכן אם נסמן $c = |P| = |P^T|$ מתקיים: 
		\[ |A'| = |P^TAP| = |P^T| \cdot |A| \cdot |P| = \sof{P}^2\sof{A} = c^2 |A| \]
	\end{proof}
	\rmark{יש שדות שמעליהם טענה 2 לא מעניינת במיוחד (שדות עבורם יש שורש לכל מספר, כמו $\C$). }
	
	\defi{תבנית $f$ מעל $V$ נקראת \textit{סימטרית} אם: \hfill $\forall v, w \in V \co f(v, w) = f(w, v)$}
	\defi{תבנית $f$ מעל $V$ נקראת \textit{אנטי־סימטרית} אם: \hfill $\forall v, w \in V \co f(v, w) = -f(w, v)$}
	\begin{Theorem}[פירוק תבנית בי־לינארית לחלק סימטרי וחלק אנטי־סימטרי]
		אם $\chr \F \neq 2$, בהינתן תבנית $f \co V \times V \to \F$ בי־לינארית, קיימות $\phi, \psi \co V\times V \to \F$ בי־לינאריות כך ש־$\phi$ סימטרית, $\psi$ אנטי־סימטרית ו־$f = \phi + \psi$. 
	\end{Theorem}
	\begin{proof}
		נבחין שאם $\chr \F \neq 2$, ניתן להגדיר את: 
		\[ \phi(v, w) = \frac{f(v, w) + f(w, v)}{2}, \ \psi(v, w) = \frac{f(v, w) - f(w, v)}{2} \]
		מתקיים ש־$\phi$ סימטרית ו־$\psi$ אנטי־סימטרית וכן $f = \phi + \psi$. 
	\end{proof}
	
	\theo{תהי $f$ תבנית בי־לינארית על $V$, ו־$B = (v_i)_{i = 1}^{n}$ בסיס ל־$B$. נניח $A =(a_{ij})^{n}_{i, j = 1}$ המייצגת את $f$ ביחס ל־$B$. אז $f$ סימטרית/אנטי־סימטרית אמ''מ $A$ סימטרית/אנטי־סימטרית. }
	
	\begin{proof}\,
		\begin{itemize}
			\item[$\implies$] אם $f$ סימטרית/אנטי־סימטרית, אז: 
			\begin{align*}
				a_{ij} = f(v_i, v_j) &\overset{\sym}{=} f(v_j, v_i) = a_{ji} \\
				a_{ij} = f(v_i, v_j) \,\,&\!\!\overset{\asym}{=} -f(v_j, v_i) = -a_{ji}
			\end{align*}
			\item[$\impliedby$] אם $A$ סימטרית אז: 
			\[ f(v, w) = [u]_B^TA[w]_B \overset{(1)}{=} ([u]_B^TA[w]_B)^T = [w]_B^TA^T([u]_B^T)^T = [w]^T_BA[u]_B = f(w, v) \]
			כאשר $(1)$ מתקיים כי transpose למטריצה מגודל $1 \times 1$ מחזיר אותו הדבר. וכן במקרה האנטי־סימטרי: 
			\[ f(u, w) = [w]^T_B(-A)[u]_B = -[w]_B^TA[u]_B = -(w, u) \]
		\end{itemize}
	\end{proof}
	
	\subsection{תבנית ריבועית}
	\defi{תהא $f$ תבנית על $V$. התבנית הריבועית: 
		\[ Q_p \co V \to \F, \ Q_f(v) = f(v, v) \]}
	היא \textit{לא} העתקה לינארית. היא תבנית ריבועית. \textbf{דוגמאות: }
	\begin{itemize}
		\item \hfil $\displaystyle f\cl{\pms{x \\ y}, \ \pms{u \\ v}} = xv + uy \implies Q\cl{\pms{x \\ y}} = xy + yx = 2xy$
		\item \hfil $\displaystyle f\cl{\pms{x \\ y}, \ \pms{u \\ v}} = xv - uy \implies Q\cl{\pms{x \\ y}} = xy - yx = 0$
		\item התבנית הסטנדרטית: 
		\[ f\cl{\pms{x \\ y}, \pms{u \\ v}} = xu + yv \implies Q_f\pms{x \\ y} = x^2 + y^2 \]
	\end{itemize}
	
	\noti{עבור תבנית בי־לינארית $f$ על $V$, נגדיר את $\h f(u, v) = f(v, u)$}
	אם $f$ סימטרית נבחין ש־$Q_f = Q_{\h f}$
	
	\begin{Theorem}[שחזור תבנית בי־לינארית מתבנית ריבועית]
		תהי $f$ תבנית בילי' סימטרית על $V$, ונניח ש־$\chr \F \neq 2$, אז: 
		\begin{enumerate}
			\item \hfil $\displaystyle f(v, w) = \frac{Q_f(v + w) - Q_f(v) - Q_f(w)}{2}$
			\item אם $f$ איינה תבנית ה־$0$ אז קיים $0 \neq v \in V$ כך ש־$Q_f(v) \neq 0$. 
		\end{enumerate}
	\end{Theorem}
	\begin{proof}
		\begin{align*}
			Q_f(v + w) - Q_f(v) - Q_f(w) = &f(v + w, v + w) - f(v, v) - f(w, w) \\
			= &\quad\, f(v, v) + f(v, w) \\
			&-f(w, v) + f(w, w) \\
			&-f(v, v) - f(w, w) \\
			\overset{\sym}{=} \!\!&\,\, 2f(v, w)
		\end{align*}
		
		ובכך הוכחנו את 1. 
		עתה נוכיח את 2. נניח $\forall v \in V \co Q_f(v) = 0$ אז
		\[ \forall v, w, \in V \co f(v, w) = \frac{Q_f(v + w) - Q_f(v) - Q_f(w)}{2} = 0  \]
		
		אז
		\[ \textstyle f(\binom{x}{y}, \ \binom{u}{v}) = xv + yu \implies Q_f = 0 \land f \neq 0 \]
	\end{proof}
	
	\rmark{אין ממש טעם להגדיר תבנית ריבועית על תבנית בי־לינארית שאיננה סימטרית. הסיבה? כל מטריצה יכולה להיות מפורקת לחלק סימטרי וחלק אנטי־סימטרי, החלק האנטי־סימטרי לא ישפיע על התבנית הריבועית (כי אלכסון אפס במטריצה המייצגת) ורק החלק הסימטרי ישאר בכל מקרה. זה לא שאי־אפשר, זה פשוט לא מעניין במיוחד. }
	
	
	\subsection{משפט ההתאמה של סילבסטר}
	
	\theo{נניח $\chr F \neq 2$, ו־$f$ סימטרית על $V$. אז קיים בסיס ל־$V$ הוא $B = (v_i)_{i = 1}^{n}$ כך ש־$[f]_B$ אלכסונית. אם $\F = \R$, אז האיברים על האלכסון יהיו $\{1, -1, 0\}$ ולא רק $\{1, 0\}$. }
	
	\textit{תזכורת: }$[f]_B$ סימון המוגדר בסיכום זה בלבד. בקורס מדברים על ''המטריצה המייצגת של בי־לינארית`` במילים מפורשות. 
	
	\begin{proof}
		באינדוקציה על $n$. בסיס $n = 1$ ברור. אם $f$ תבנית ה־$0$, אז כל בסיס שנבחר מתאים. אחרת, קיים $0 \neq v \in V$ כך ש־$Q_f(v) \neq 0$. נגדיר $U = \{u \in V \mid f(u, v) = 0\}$. תמ''ו כי גרעין של ה''ל (כי קיבענו את $v$). מה התמונה של ההעתקה? $f(v, v) = Q_f(v) \neq 0$. לכן תמונת ההעתקה היא כל $\F$, וממדה $1$. ידוע $U$ תמ''ו מממד $n - 1$. אז $f_{|U} \co U\times U \to \F$ לכסינה ולכן קיים בסיס $B_U$ כך ש־$[f_{|U}]$ אלכסונית. נגדיר את $B = \{v\} \cup B_U$ נבחין שהיא מהצורה: 
		\[ \pms{* & 0& \cdots \\ 0 & [f_{|U}]_B& - \\ \vdots & \vert} \]
	\end{proof}
	
	\theo{לכל $f$ תבנית סימטרית קיימת מטריצה מייצגת מהצורה $\binom{I_r \, 0}{0 \,\,\, 0}$ כאשר $\F = \C$ (או סגור אלגברית כלשהו). }
	\textit{אינטואציה להוכחה. }ננרמל את המטריצה, נבחין שחלוקה ב־$c$ של השורה ה־$i$ ניאלץ להפעיל גם עם העמודה ה־$i$, כלומר את $a_{i, i}$ נחלק ב־$c^{2}$ בצורה הזו (זאת כי כאשר $P^TAP$ הגדרת חפיפה, ו־$P$ מדרגת שורות, $P^{T}$ מדרגת עמודות). 
	\begin{proof}		
		נסמן את $\dim f = r$. עד כדי שינוי סדר איברי הבסיס, המטריצה המייצגת אלכסונית היא:
		\[ [f]_B = \pms{\diag(c_1 \dots c_r) & 0 \\ 0 & 0} \]
		כאשר $c_1 \dots c_r \neq 0$, ביחס לבסיס $B = (v_1 \dots v_r, \dots v_n)$. באופן כללי לכל $i \in \R$ נוכל להגדיר את $v_i' = \frac{v_i}{\sqrt{c_i}}$ כך ש־$f(v'_i, v'_i) = 1$ כי $f(v_i, v_i) = c_i$ ומליניאריות בכל אחת מהקורדינאטות. בשל כך $B' = (v_1' \dots v'_r, v_{r + 1} \dots v_n)$ בסיס המקיים את הדרוש. 
		
		באותו האופן, אם $\F = \R$ (ולא $\C$) אז קיים בסיס שהמטריצה המייצגת לפיו היא: 
		\[ \pms{I_p & 0 & 0 \\ 0& -I_q & 0 \\0 & 0 & 0} \]
		בלוקים, כך ש־$p + q = r$. כאן נגדיר: 
		\[ f(v, v) = c < 0, \ v' = \frac{v}{\sqrt{|c|}}, \ f(v', v') = \frac{c}{|c|} = -1 \]
	\end{proof}
	
	\defi{יהי $V$ מ''ו מעל $\R$ ו־$f$ תבנית בי־לינארית מעל $V$. נאמר ש־$f$ חיובית/אי־שלילית/שלילית/אי־חיובית אם $\forall 0 \neq v \in V $ מתקיים ש־$f(v, v) > 0$/$f(v, v) \ge 0$/$f(v, v) < 0$/$f(v, v)\le 0$}
	
	\theo{תהא $A$ מטריצה מייגצת של תבנית בי־ליניארית סימטרית, עם ערכים $0, -1, 1$ בלבד על האלכסון, מקיימת: 
		\begin{itemize}
			\item $f$ חיובית אמ''מ ישנם רק $1$־ים. 
			\item $f$ אי־שלילית אמ''מ ישנם רק $1$־ים ואפסים. 
			\item $f$ שלילית אמ''מ ישנם רק $-1$־ים
			\item $f$ חיובית אמ''מ ישנם רק $-1$־ים ואפסים. 
	\end{itemize}}
	\begin{proof}\,
		\begin{itemize}
			\item[$\impliedby$] טרוויאלי
			\item[$\implies$] לכל $0 \neq v \in V$ קיימים ויחידים $\ag_1 \dots \ag_n \in \R$ כך ש־$v = \sum^n_{i = 1} \ag_i v_i$ ומתקיים $f(v, v) = \ag_i^2 f(v_{i, i})$ ולפי המקרה זה יסתדר יפה. 
		\end{itemize}
	\end{proof}
	
	\begin{Theorem}[משפט ההתאמה של סילבסטר]
		$p, q$ הנ''ל נקבעים ביחידות. 
	\end{Theorem}
	(תחזרו כמה משפטים למעלה למקרה בו $\F = \R$)
	\begin{proof}[גישה שגויה להוכחה. ]
		הוכחה באמצעות $\tr$ לא עובדת. בניגוד ליחס הדמיון להעתקות לינאריות, ביחס החפיפה להעתקות בי־לינאריות ה־$\tr$ לא נשמר. 
	\end{proof}
	\begin{proof}[הוכחה תקינה. ]
		נסמן $B = (v_1 \dots v_p, u \dots u_q, w_1 \dots w_k)$ וכן $B' = (v_1' \dots v'_t, u'_1 \dots u'_s, w_1 \dots w_k)$ כי $t + s = p + q$. בה''כ $t \le p$, נניח בשלילה ש־$t < p$. נסמן $U = \Sp(v_1 \dots v_p)$. ידוע $f$ חיובית על $U$, וכן $\dim U = p$. נתבונן ב־$W = \Sp(u_1' \dots u_s', w_1 \dots w_k)$. אזי גם $f$ חיובית על $W$, ו־$\dim W = s + k$. בגלל ש־$U \cap W = \{0\}$ (כי אם לא, אז עבור $0 \neq v \in U \cap W$ נקבל $f(v, v) > 0$ כי $v \in U$ וכן $f(v, v) \le 0$ כי $v \in W$ וסתירה). ידוע ש־$U \oplus W \subseteq V$ תמ''ו וכן $\dim U + \dim W \le \dim V$. נציב ונקבל $p + s + k > t + s + k = \dim V$, סתירה. לכן $p, q$ נקבעים ביחידות. 
	\end{proof}
	
	\noti{ה־$(p, q)$ לעיל נקראים \textit{הסינגטורה של $f$}. }
	(תזהרו, הסינגטורה תתקוף אותנו אח''כ)
	
	\npage
	\section[מרחבי מכפלה פנימית]{\en{Inner Product Vector Spaces}}
	\subsection{הגדרה כללית}
	\subsubsection{מעל $\R$}
	\textbf{מעתה ועד סוף הקורס, }מתקיים $\F = \R, \C$. 
	כל עוד נאמר ``$\F$'', זה נכון בעבור שני המקרים. אחרת, נפצל. 
	
	\defi{יהי $V$ מ''ו, \textit{מכפלה פנימית} מעל $\R$ היא תבנית בי־לינארית סימטרית חיובית מעל $V$, ומסומנת $f(v, u) = \la v, u \ra$ (ויש ספרים שמסמנים $\la v \mid u \ra$, $\smut\co V \times V \to \R$. }
	\noti{בקורס מסמנים $\la \cdot, \cdot \ra$ אבל אני מגניב אז אני משתמש ב־$\smut$. }
	
	\lem{$\forall v \in V \co \mut{v}{v} \ge 0$ ו־$\la v, v \ra$ אמ''מ $v = 0$. }הוכחה מסימטריה. 
	
	\textbf{דוגמה. }(\textit{המכפלה הפנימית הסטנדרטית} על $\R^n$, AKA כפל סקלרי): 
	\[ \mut{\pms{x_1 \\ \vdots\\ x_n}}{\pms{y_1 \\ \vdots \\ y_n}} = \sum_{i = 1}^{n}x_iy_i \]
	
	\defi{אם $V$ מ''ו וקיימת $\smut \co V \times V \to \F$ מכפלה פנימית אז $(V, \smut)$ נקרא \textit{מרחב מכפלה פנימית}, ממ''פ. }
	
	\theo{$V = M_n(\R)$, אז $\mut{A}{B} = \tr(A \cdot B^T)$ אז $(V, \smut)$ ממ''פ. }
	
	\textbf{דוגמה מגניבה. }בהינתן $V = [0, 1]$, מ''ו הפונקציות הממשיות הרציפות על $[0, 1]$, ו־$\mut{f}{g} = \int_0^1 f(x) \cdot g(x) \dx$ 
	\theo{(שהפליצו מחדו''א) אם $f \ge 0$ אינטרבילית על קטע $[a, b]$ וגם ישנה נקודה חיובית $c \in [a, b]$ שעבורה $f(x) \ge 0$ וגם $f$ רציפה ב־$c$, אז $\int^b_a f(x) \dx > 0$. }
	
	\subsubsection{מעל $\C$}
	ישנה בעיה עם חיובית: אם $v \in V$ כך ש־$\mut{v}{v} \ge 0$ אך $\mut{iv}{iv} = -1\mut{v}{v} < 0$ סתירה. לכן, במקום זאת, נשתמש בהגדרה הבאה: 
	\defi{יהי $V$ מ''ו מעל $\C$. מכפלה פנימית $\smut \co V \times V \to \C$ מקיימת: 
		\begin{itemize}
			\item ליניאירות ברכיב הראשון: אם נקבע $v$, אז $u \mapsto \mut{v}{u}$ לינארית. 
			\item ססקווי־ליניאריות ברכיב השני (במקום לינאריות): \hfill $\mut{u_1 + u_2}{v} = \mut{u_1}{v} + \mut{u_2}{v} \land \mut{u}{\ag v} = \bar \ag \mut{u}{v}$ 
			
			כאשר $\bar \ag$ הצמוד המרוכב של $\ag$. 
			\item הרמטיות (במקום סימטריות): \hfill $\mut{v}{u} = \ol{\mut{u}{v}}$
			\item חיוביות ואנאיזוטרופיות: \hfill $\forall 0 \neq v \in V \co \mut{v}{v} > 0 \land \mut{0}{0} = 0$
	\end{itemize}}
	למעשה – נבחין שאין צורך בממש ססקווי־ליניאיריות ברכיב השני וכן לא בתנאי $\mut{0}{0} = 0$, וההגדרה שקולה בעבור חיבוריות ברכיב השני בלבד, זאת כי: 
	\[ \mut{u}{\ag v} = \ol{\mut{\ag v}{u}} = \ol{\ag \mut{v}{u}} = \bar \ag \cdot \ol{\mut{v}{u}} = \bar \ag \mut{v}{u} \]
	ומכאן נגרר ססקווי־ליניאריות, וכן $\mut{0}{0} = 0$ נובע ישירות מליניאריות ברכיב השני. 
	
	\rmark{באוניברסיטאות אחרות מקובל להגדיר לינאריות ברכיב השני ולא בראשון. זה לא באמת משנה. }
	
	
	\defi{\hfil $\ol{B^T} = B^*$}
	
	\begin{Definition}[הגדרה נחמדה]
		יהי ממ''פ $V$ מ''ו מעל $\F$. לכל $v \in V$ מגדירים את ה\textit{נורמה} של $v$ להיות $ \norm{v} = \sqrt{\mut{v}{v}} $. 
	\end{Definition}
	
	\theo{הנורמה כפלית וחיובית. }
	\begin{proof}
		מאקסיומת החיוביות: 
		\[ \norm{v} \ge 0 \land (\norm{v} = 0 \iff v = 0) \]
		וכן: 
		\[ \norm{t \cdot v^2} = \mut{tv}{tu} = t \bar t\mut{v}{v} = \sof{t}\norm{v} \implies \norm{t \cdot v} = \sof{t} \cdot \norm{v} \]
	\end{proof}
	
	\defi{יהי $V$ מ''ו מעל $\F$, ו־$\snorm \co V \to \R_{\ge 0}$, אז $(V, \snorm)$ יקרא \textit{מרחב נורמי}. }
	\theo{(\textit{''נוסחאת הפולריזציה'')} בהינתן $(V, \snorm)$ מרחב נורמי, ניתן לשחזר את המכפלה הפנימית, באמצעות הנוסחה הבאה: 
		
		\textbf{גרסה מעל $\bm{\R}$: }
		\[ \forall v, u \in V \co \mut{v}{u} = \frac{1}{4}(\norm{u + v}^2 + \norm{u - v}^2) \]
		\textbf{גרסה מעל $\C$: }
		\[ \mut{u}{v} = \frac{1}{4}\Big(\norm{u + v}^2 - \norm{u - v}^2 + i\norm{u + iv} - i\norm{u + iv}\Big) \]
	}
	\begin{proof}[הוכחה (ל־$\bm{\C}$)]
		\begin{align*}
			\mut{u + v}{u + v} =\,& \norm{u}^2 + \mut{u}{v} + \mut{v}{u} + \norm{v}^2 \\
			=\,& \norm{u}^2 + \norm{v}^2 + 2\Re(\mut{v}{u}) \\
			\mut{v - u}{v - u} = \, & \norm{u}^2 + \norm{v}^2 - 2\Re(\mut{v}{u}) \\
			\mut{u + iv}{u + iv} =\,& \norm{u}^2 + \norm{v}^2 + \mut{u}{iv} + \mut{iv}{u}  \\
			=\,& \norm{u}^2 + \norm{v}^2 - i\mut{u}{v} + i\ol{\mut{u}{v}} \\
			=\,& \norm{u}^2 + \norm{v}^2 - i(2 \Im\mut{u}{v}) \\
			\norm{u - iv} =\,& \norm{u} + \norm{v} - \mut{u}{iv} - \mut{iv}{u} \\
			=\,& \norm{u} + \norm{v} - 2\Im(\mut{v}{u})
		\end{align*}
		וסה''כ אם נציב בנוסחה, אחרי שחישבנו את כל אבירה, הכל יצטמצם וש־$\mut{u}{v}$ אכן שווה לדרוש. 
	\end{proof}
	
	במילים אחרות, באותה המידה שתבניות שמתבניות בי־לינאריות ותבניות ריבועיות אפשר להסיק אחת מהשניה, אפשר גם ממכפלה פנימית להסיק נורמה ולהפך. אזי, ממ''פ ומרחב נומרי הם די שקולים. 
	
	\subsection{אורתוגונליות}
	
	\defi{בהינתן $(v, \smut)$ ממ''פ, לכל $v \in V$ נאמר ש־$u$ \textit{מאונך ל־$v$} (או \textit{אורתוגונלי ל־} אם אנחנו מרגישים מפונפנים) ונסמן $u\perp v$ אם $\mut{u}{v} = 0$. }
	
	\rmark{אם $u \perp v$ אז $v \perp u$. (כי צמוד של $0$ הוא $0$). }
	
	\subsubsection{משפט פיתגורס ותוצאותיו}
	
	\begin{Theorem}[משפט פיתגורס]
		(מאוד מועיל) יהי $V$ ממ''פ כך ש-$v, u \in V$ אורתוגונלים, אז $\norm{v + u}^2 = \norm{v}^2 + \norm{u}^2$
	\end{Theorem}\begin{proof} משום שהם מאונכים מתקיים $\mut{v}{u} = 0$. נפתח אלגברה: 
		\[ \norm{v + u}^2 = \mut{v + u}{v + u} = \norm{v}^2 + \cancel{\mut{v}{u}} + \cancel{\mut{u}{v}} + \norm{u^2} = \norm{v}^2 + \norm{u}^2 \quad \top \]
	\end{proof}
	
	\rmark{בעבור $v = \R^n$ מ''פ סטנדרטית אז $\norm{v}$ מזדהה עם מושג הגודל של וקטור בגיואמטריה רגילה. }
	
	\rmark{בתוך $\R^n$ הוקטורים הסטנדרטיים מאונכים אחד לשני (במכפלה הפנימית הסטנדרטית) ולכן $\mut{e_i}{e_j} = \delta_{ij}$ כאשר $\delta_{ij}$ הדלתא של כרוניקר. באינדוקציה על משפט פיתגורס נקבל ש־: }
	
	\[ v = \sum_{i = 1}^{n}\ag_i e^i \implies \norm{v} = \sum_{i = 1}^{n}\ag_i^2 \]
	שזה בדיוק מושג הגודל בגיאומטריה אוקלידית. 
	
	\rmark{מעל $\R$ מקבלים אמ''מ למשפט פיתגורס, מעל $\C$ לאו דווקא. }
	
	\theo{(אי שוויון קושי־שוורץ)
		\[ \forall v, u \in V \co \sof{\mut{u}{v}} \le \norm{u}\cdot\norm{v} \]
		ושוויון אמ''מ $u, v$ ת''ל. }
	\rmark{זה בפרט נכון בגיאומטריה סטנדרטית ממשפט הקוסינוסים. }
	
	\begin{proof}
		אם $v$ או $u$ הם $0$, אז מתקקבל שוויון. 
		\textit{טענת עזר: }קיים איזשהו $\ag \in \F$ כך ש־$u - \ag v \perp v$. נסמן $v_u = \ag v$ כאשר נמצא אותו. 
		\textit{הוכחת טענת העזר. }נחפש כזה: 
		\[ \mut{u - \ag v}{v} = 0 \impliedby \mut{v}{u} - \ag \norm{v}^2 = 0 \impliedby \ag = \frac{\mut{v}{u}}{\norm{v}^2} \]
		כדרוש. (מותר לחלק בנורמה כי הם לא $0$). ניעזר במשפט פיתגורס: 
		\begin{align*}
			&\begin{cases}
				u - \ag v \perp v \\ u - \ag v \perp v
			\end{cases}\dequad  \\
			\implies &\norm{u}^2 = \norm{(u - \ag v + \ag v)}^2 = \smash{\overbrace{\norm{u - \ag v}^2}^{\ge 0}} + |\ag|^2\norm{v}^2 \\ 
				&\ge |\ag|\cdot\norm{v}^2 = \frac{|\mut{v}{u}|^2}{(\norm{v}^2)^2} = \norm{v}^2 = \frac{|\mut{u}{v}|^2}{\norm{v}^2} \\
			\implies &\sof{\mut{v}{u}}^2 \le \norm{v} \cdot \norm{u}
		\end{align*}
		בפרט $\norm{u - \ag v}^2 = 0$ אמ''מ הם תלויים לינארית ומכאן הכיוון השני של המשפט. 
	\end{proof}
	
	\rmark{המשפט למעלה לא אומר כלום כי מגדירים קוסינוס לפי המכפלה הפנימית הסטנדרטית. }
	
	\textbf{דוגמאות. }
	\begin{enumerate}
		\item ממכפלה פנימית סטנדרטית: 
		\[ \sof{\sum_{i = 1}^{n}\ag_i \bg_i}^2 \le \cl{\sum_{i = 1}^{n}\ag_i^2} \cl{\sum_{i = 1}^{n}\bg_i^2} \]
		\item  נניח $f, g [0, 1] \to \R$ רציפות אז: 
		\[ \sof{\int^1_0 f(t)g(t) \dt}^2 \le \int_{0}^{1}f^2(t) \dt \cdot\int^1_0 g^2(t) \dt \]
		כאשר $f^2 = f \cdot f$ (לא הרכבה). 
		\item אי־שוויון המשולש: 
		\[ \forall u, v \in V \co \norm{u + v} \le \norm{u} + \norm{v} \]
		ושוויון אמ''מ אחד מהם הוא $0$ או אם הם כפולה חיובית אחד של השני (לא שקול לתלויים לינארית – יכולה להיות כפולה שלילית). 
	\end{enumerate}
	\begin{proof}[הוכחה (לאי שוויון המשולש)]
		\textit{תזכורת: עבור $\zc \in \C$ מתקיים $\sof{\zc}^2 = (\Re \zc)^2 + (\Im \zc)^2$}
		\[ \norm{u + v}^2 = \norm{u}^2 + \norm{v}^2 + 2\Re(\mut{u}{v}) \le \norm{u}^2 + \norm{v}^2 + 2\sof{\mut{u}{v}}  \]
		ושוויון אמ''מ $u$ הוא אפס או כפולה חיובית של $v$. מקושי־שוורץ: 
		\[ \le \norm{u}^2 + 2\norm{u}\norm{v} + \norm{v}^2 = \cl{\norm{u} + \norm{v}}^2 \]
	\end{proof}
	
	\subsection{מרחבים ניצבים והיטלים}
	\noti{יהי $(V, \smut)$ ממ''פ. יהיו $S, T \subseteq V$. נסמן: 
		\begin{enumerate}[A.]
			\item \hfil $u \in V \co \cl{u \perp S \iff (\forall v \in S\co u \perp v)}$
			\item \hfil $S \perp T \iff \forall v \in S \,\, \forall u \in T \co v \perp u$ 
			\item \hfil $S^{\perp} := \{v \in V \mid v \perp S\}$
	\end{enumerate}}
	\defi{$T^{\perp}$ הוא תת־המרחב הניצב ל־$T$. }
	\theo{יהיו $S, T \subseteq V$ קבוצות, ו־$U, W \subseteq V$ תמ''וים. אז: 
		\begin{enumerate}[A.]
			\item $v \perp S$ אמ''מ $v \perp \Sp(S)$
			\item $S^{\perp} \subseteq V$ תמ''ו
			\item אם $S \subseteq T$ אז $T^\perp \subseteq S^\perp$
			\item \hfil $U \oplus U^{\perp} = V$
			\item \hfil $\cl{S^\perp}^\perp = \Sp S$
			\item \hfil $(U + W)^{\perp} = U^{\perp} \cap W^{\perp}$
			\item \hfil $(U \cap W)^{\perp} = U^{\perp} + W^{\perp}$
	\end{enumerate}}
	\begin{proof}[הוכחה (לג'). ]
		\[ \forall v \perp T \co c \perp S \implies v \in S^{\perp} \]
	\end{proof}
	\rmark{שוויון בג' מתקיים אמ''מ $\Sp S = \Sp T$. }
	
	\defi{משפחה של וקטורים $A \subseteq V$ נקראת \textit{אורתוגונלית} אם $\forall u \neq v \in V\co u \perp v$}
	\rmark{אם $A$ משפחה אורתוגונלית וגם $0 \notin A$ אז ניתן לייצור ממנה משפחה של וקטורים אורתוגונלים שהם גם וקטורי יחידה, ע''י נרמול. }
	\defi{משפחה של וקטורים $A \subseteq V$ נקראת \textit{אורתונורמלית}, אם היא אורתוגונלית ובנוסף כל הוקטורים הם וקטורי יחידה. }
	
	\defi{יהי $U \subseteq V$ תמ''ו. יהא $v \in V$. אז \textit{ההטלה האורתוגונלית} של $V$ על $U$ היא $p_U(v)$ הוא וקטור המקיים: 
		\begin{itemize}
			\item \hfil $p_U(v) \in U$ 
			\item \hfil $v - p_U(v) \in U^\perp$
	\end{itemize}}
	\theo{בסימונים לעיל, $\forall u \in U \co \norm{v - u} \ge \norm{v - p_U(v)}$ ושוויון אמ''מ $u = p_U(v)$. }
	
	\begin{proof}
		יהי $u \in U$. ידוע $p_U(v) \in U$. אזי $u - P_u(v) \in U$. כמו כן ידוע $p_U(v) - v \perp u$. אזי בפרט $\mut{u - p_U(v)}{p_U(v) - v}$. נתבונן ב־: 
		\[ \norm{u - v}^2 = \norm{(u - p_U(v)) + (p_U(v) - v)}^2 \overset{\text{פית'}}{=} \norm{u - p_U(v)}^2 + \norm{v - p_U(v)}^2 \]
		וסה''כ $\norm{v - u}^2 \ge \norm{v - p_U(v)}^2$. ושוויון אמ''מ $\norm{u- p_U(v)} = 0$ אמ''מ $u  = p_U(v)$. 
	\end{proof}
	עתה נוכיח את יחידות ההטלה האורתוגונלית (קיום נוכיח בהמשך באופן קונסטקרטיבי)
	\theo{ההטלה הניצבת, היא יחידה. }\begin{proof}
		יהיו $p_U(v)$ וכן $p'_U(v)$ הטלות של $v$ על $U$. מהטענה: 
		\[ \norm{v - p_U(v)} \le \norm{v - p'_U(v)} \]
		אבל בהחלפת תפקידים מקבלים את אי־השוויון ההפוך. לכן יש שוויון נורמות. מהמשפט לעיל $p_U(v) = p'_U(v)$. 
	\end{proof}
	
	\theo{תהי $A \subseteq V$ משפחה אורתוגונלית ללא $0$. אז היא בת''ל. }\begin{proof}
		יהיו $v_1 \dots v_n \in A$ וכן $\ag_1 \dots \ag_n \in \F$, כך ש־$\sumni \ag_i v_i = 0$. יהי $i \in [n]$. אז: 
		\[ 0 = \mut{0}{v_j} = \mut{\sumni \ag_i v_i}{v_j} = \sumni \ag_i \mut{v_i}{v_j} = \ag _j \underbrace{\norm{v_j}^2}_{\neq 0} \implies \ag_j = 0 \]
		כאשר השוויון האחרון מהיות הקבוצה אורתוגונלית. 
	\end{proof}
	
	\begin{Theorem}[קיום היטל אורתוגונלי]
		נניח ש־$u \subseteq V$ תמ''ו. נניח $U$ נ''ס וכן $B = (e_1 \dots e_n)$ בסיס אורתונורמלי של $U$ (כלשהם, לא בהכרח סטנדרטיים כי גם לא בהכרח $\F^n$). אז 
		\[ \forall v \in V \co p_U(v) = \sum_{i = 1}^{m}\mut{v}{e_i}e_i \]
	\end{Theorem}
	\begin{proof}
		צ.ל. $p_U(v) \in U$ וגם $\forall u \in U \co \mut{v - p_U(v)}{u} = 0$ אך לגבי התנאי האחרון די להוכיח $\forall j \in [n] \co \mut{v_i p_U(v)}{e_j}  = 0$. החלק הראשון ברור, נותר להוכיח: 
		\[ \mut{v - p_U(v)}{e_j} = \mut{v}{e_j} - \mut{p_u(v)}{e_j} =: * \]
		ידוע: 
		\[ \mut{p_U(v)}{e_j} = \mut{\sum_{i = 1}^{m}\mut{v_i}{e_i} e_i}{e_j} = \sum_{i = 1}^{m}\mut{v_i}{e_i} \cdot \mut{e_i}{e_j} = \sum_{i = 1}^{m}\mut{v}{e_i}\delta_{ij} = \mut{v}{e_j} \]
		נחזור לשוויון לעיל: 
		\[ *= \mut{v}{e_j} - \mut{v}{e_j} = 0 \]
		כדרוש. 
	\end{proof}
	(בכך הוכחנו את קיום $p_U(v)$ לכל מ''ו נ''ס, אם נשלב את זה עם המשפט הבא)
	
	\subsubsection{אלגוריתם גרהם־שמידט}
	\begin{Theorem}[אלגוריתם גרהם־שמידט]
		תהי $(b_1 \dots b_k)$ קבוצה סדורה בת''ל של וקטורים בממ''ס $V$. אז בכל משפחה א''נ $(u_1 \dots u_k)$ כך ש־$\Sp(b_1 \dots b_k) = \Sp(u_1 \dots u_k)$. 
	\end{Theorem}
	\textbf{מסקנות מהמשפט. }לכל ממ''ס נ''ס קיים בסיס א''נ (=אורתונורמלי). יתרה מזאת, בהינתן בסיס $B = (b_1 \dots b_n)$ ניתן להופכו לבסיס א''נ $(u_1 \dots u_n)$ המקיים $\forall k \in [n] \co \Sp (b_1 \dots b_k) = \Sp(u_1 \dots u_k)$. 
	
	\begin{proof}
		בנייה באינדוקציה. נגדיר עבור $k = 1$ את $u_1 = b_1''$. מתקיים $\Sp u_1 = \Sp b_1$ וכן $\{u_1\}$ קבוצה א''נ. נניח שבנינו את $k$ האיברים הראשונים, נבנה את האיבר ה־$k + 1$ (כלומר את $u_{k + 1}$). במילים אחרות, הנחנו $u_1 \dots u_k$ אורתונורמלית וגם $\Sp (u_1 \dots u_k) = \Sp(b_1 \dots b_k) = U$. 
		
		מהסעיף הקודם $p_U(b_{k + 1})$ קיים, וגם $b_{k + 1} - p_U(b_{k + 1}) \neq 0$ מהבנייה. נגדיר $u_{k + 1} = (b_{k + 1} - p_U(b_{k + 1}))$. בצורה מפורשת: 
		\[ u_{k + 1} = \frac{b_{k + 1} - \sum_{i = 1}^{k}\mut{b_{k + 1}}{u_i}u_i}{\norm{b_{k + 1} - \sum_{i = 1}^{k}\mut{b_{k + 1}}{u_i}u_i}} \]
		מהגדרת $p_U(b_{k + 1})$, מתקיים $b_{k + 1} - p_U(b_{k + 1}) \in U^{\perp}$ ולכן גם $u_{k + 1} \in U^{\perp}$ ולכן $(u_1 \dots u_{k + 1}$ משפחה א''נ. 
		\[ b_1 \dots b_k = \overbrace{\Sp(u_1 \dots u_{k + 1})}^{\,\!\text{בת''ל}} \]
		נשאר להוכיח ש־$b_{k + 1} \in \Sp(u_1 \dots u_{k + 1})$. זה מספיק משום שאז נקבל $\Sp(b_1 \dots b_{k + 1}) \subseteq \Sp(u_1 \dots u_{k + 1})$. אבל הם שווי ממד ולכן שווים. סה''כ: 
		\[ b_{k + 1} = \norm{b_{k + 1} - p_U(b_{b + 1})} \cdot u_{k + 1} + \sum_{i = 1}^{k}\mut{b_{k + 1}}{u_i}u_i \implies b_{k + 1} \in \Sp(u_1 \dots u_{k + 1}) \]
		מש''ל. 
	\end{proof}
	
	\theo{יהי $V$ מ''ו $U \subseteq V$. נניח שלכל $v \in V$ מוגדר $p_U(v)$ (בפרט כל מ''ו נ''ס). אז $p_U \co V \to V$ המוגדרת לפי $v \mapsto p_U(v)$ העתקה לינארית. }\begin{proof}
		יהיו $v, v' \in V, \ag \in \F$. ידוע $v - p_U(v), v' - p_U(v') \in U^{\perp}$ ועל כן: 
		\[ (v - p_U(v)) + \ag(v' - p_u(v')) \in U^T \implies (v + \ag v') - (p_U(v) + \ag p_U(v')) \in U^T \implies (v + \ag v') - p_U(v + \ag v') \in U^{T} \]
		מה מקיים היטל וקטור? ראשית ההיטל ב־$U$, ושנית $v$ פחות ההיטל מאונך. הוכחנו שבהינתן היטל, הוא יחיד. והראינו ש־$(v + \ag v') - p_U(v + \ag v')$ מקיים את זה, ולכן אם יש וקטור אחד אז הוא יחיד, וסה''כ שווים וליניארית. 
	\end{proof}
	
	\theo{\hfil $\min_{u \in U}\norm{v - u} = \norm{v - p_U(v)}$}
	בניסוח אחר: ההיטל $p_U(v)$ הוא הוקטור הכי קרוב ל־$v$ ב־$U$. בתרגול צוין שזוהי דרך למצוא את הפתרון ``הכי קרוב'' למערכת משוואות לינארית שאין לה פתרון. 
	\defi{\textit{הפתרון האופטימלי} למערכת משוואות $(A \mid b)$ הוא $p_{\col A}(b)$ (כאשר $\col A$ מ''ו העמודות). }
	
	
	\subsection{צמידות}
	\subsubsection{העתקה צמודה לעצמה}
	
	\defi{$V$ ממ''פ ו־$T \co V \to V$ ט''ל. אז $T$ נקראת \textit{סימטרית} ($\F = \R$) או \textit{הרמטית} ($\F = \C$) אם $\forall u, v \in V\la Tu \mid v \ra = \la u, Tv \ra$
		באופן כללי, העתקה כזו תקרא \textit{צמודה לעצמה}. }
	
	\textbf{דוגמה. }(המקרה בפרטי בממ''פ המשרה את הגיאו' האוקלידית) עבור $V = \R^{n}$, $\smut$ מ''פ סטנדרטית, ו־ $A \in M_n(\R)$ מתקיים $T_A \co V \to V$ ט''ל, היא צמודה לעצמה אם: ידוע $\mut{v}{u} = v^{T}u$: 
	\[ \mut{T_Av}{u} = (Av)^{T}u = v^{T}A^Tu = \mut{v}{A^{T}u} \]
	ז''א אם $A = A^{T}$ אז $T_A$ סימטרית, כלומר $A$ מטריצה סימטרית. גם הכיוון השני נכון: אם $T \co V \to V$ סימטרית אז ע''י בחירת בסיס נקבל $[T]^{B}_B$ גם היא סימטרית. 
	
	\textbf{דוגמה נוספת} (בדמות משפט)
	\theo{ההעתקה $v \mapsto p_U(v)$ עבור $U$ תמ''ו כלשהו, היא ההיטל, צמודה לעצמה. }
	
	\theo{העתקה סימטרית אמ''מ היא דומה למטריצה סימטרית. }
	
	\theo{יהיו $T, S \co V \to V$ צמודות לעצמן. אז: 
		\begin{enumerate}
			\item $\ag T, T + S$ צמודות לעצמן. 
			\item המכפלה $S\circ T$ צמודה לעצמה אמ''מ $ST = TS$
			\item אם $p$ פולינום מעל $\F$ אז $p(T)$ צמודה לעצמה. 
	\end{enumerate}}
	קל לראות ש־$1 + 2 \implies 3$. $1$ נובע ישירות מהגדרה. $1$ טרוויאלי. נוכיח את $2$. \begin{proof}[הוכחה ל־2.]
		נניח $S \circ T$ צמודה לעצמה. בהנחות המשפט ידוע $S, T$ צמודות לעצמן. נקבל: 
		\[ \mut{(S \circ T)v}{u} = \mut{v}{STu} = \mut{Sv}{Tu} = \mut{TSv}{u} \implies \mut{(ST - TS)v}{u} = 0 \quad \forall v, u \]
		נסיק: 
		\[ \implies \forall v\mut{(ST - TS)v}{(ST - TS)v} = 0  \implies (ST - TS)v = 0 \implies STv = TSv \implies \top \]
		מהכיוון השני, אם $TS = ST$ אז מהיות $S, T$ צמודות לעצמן: 
		\[ \mut{STv}{u} = \mut{S(Tv)}{u} = \mut{Tv}{Su} = \mut{v}{TSu} = \mut{v}{STu} \]
	\end{proof}
	\defi{$T \co V \to V$ תקרא חיובית/אי־שלילית/שלילית/אי־חיובית אם: 
		\begin{multicols}{2}
				 חיובית: $\mut{Tv}{v} > 0$ \\
				שלילית: $\mut{Tv}{v} < 0$ \\
				אי־שלילית: $\mut{Tv}{v} \ge 0$ \\
				אי־חיובית: $\mut{Tv}{v} \le 0$
		\end{multicols}
	}
	\theo{אם $T$ חיובית/שלילית, אז היא הפיכה} \begin{proof}
		נניח ש־$T$ לא הפיכה, נניח בשלילה שהיא חיובית. קיים $0 \neq v \in V$. אז $v \in \ker T$, אז $\mut{Tv}{v} = \mut{0}{v} = 0$, בסתירה לכך ש־$T$ חיובית. 
	\end{proof}
	\theo{נניח ש־$S$ צמודה לעצמה, אז $S^{2}$ צמודה לעצמה ואי־שלילית. }
	\begin{proof}
		ממשפט קודם $S^{2}$ צמודה לעצמה. נוכיח אי־שלילית: 
		\[ \forall 0 \neq v \in V \co \mut{S^{2}v}{v} = \mut{Sv}{Sv} = \norm{Sv}^{2} \ge 0 \]
	\end{proof}
	\defi{פולינום $p \in \R[x]$ יקרא חיובי אם $\forall x \in \R\co p(x) > 0$. }
	\theo{נניח $p(x) \in \R[x]$ חיובי, ו־$T \co V \to V$ צמודה לעצמה, אז $p(T)$ חיובית גם־כן, וצמודה לעצמה. }
	\lem{אם $p \in \R[x]$ אי־שלילי, אז קיימים $g_1 \dots g_k \in \R[x]$ וכן $0 \ge c \in \R$ כך ש־$p(x) = \sum_{i = 1}^{k}g^2_i(x) + c$, ו־$c \neq 0$ אמ''מ $p$ חיובי. }
	רעיון להוכחת הלמה: מעל $\C$ זה מתפרק, ונוכל לכתוב $p(x) = a_n\prod_{j = 1}^{s} (x - i\ag_j)(x + i\ag_j)$ (מעל $\R$ כל פולינום מתפרק לגורמים ריבועיים, ואם כל שורשיו מרוכבים, כל גורמיו ריבועיים). הרעיון הוא להוכיח את הטענה ש־$g^2 h \bar h = g_1^2 + g_2^2$.
	\begin{proof}[הוכחה (של המשפט, לא של הלמה). ]
		יהי $0 \neq v \in V$. אז: 
		\[ \mut{p(T)v}{v} = \underbrace{\mut{\sum_{i = 1}^{k}g_i^2(T)v}{v}}_{\mathclap{\sum_{i = 1}^{k} \mut{g_i^2(T)v}{v}\ge 0}} + \overbrace{c\mut{v}{v}}^{c\norm{v}^2 > 0} \ge 0 \]
	\end{proof}
	\cola{אם $T \co V \to V$ צמודה לעצמה ו־$p(x) \in \R[x]$ פולינום חיובי, אז $p(T)$ הפיכה. }
	
	\theo{נניח ש־$T \co V \to V$ סימטרית (צמודה לעצמה מעל $\R$/המייצגת סימטרית) ויהי $m_T(x)$ הפולינום המינימלי של $T$. אז $m_T$ מתפרק לגורמים לינארים. בנוסף, הם שונים זה מזה. }
	\begin{proof}
		נניח בשלילה קיום $p \mid m_T$ ו־$\deg p \ge 2$, $p $ אי־פריק. בה''כ נניח ש־$p$ חיובי (אין לו שורש ב־$\R$, לכן נמצא כולו מעל/מתחת לציר ה־$x$). אז אפשר לכתוב את $m_T$ כ־$m_T = p \cdot g$ כלשהו. ידוע $p(T) \neq 0$ כי $m_T$ מינימלי מדרגה גבוהה יותר. 
		אזי: 
		\[ 0 = m_T(T) = \underbrace{p(T)}_{\neq 0} \cdot g(T) \implies g(T) = 0 \]
		בסתירה למינימליות של $m_T$. סה''כ $m_T$ אכן מתפרק לגורמים לינארים. עתה יש להראות שהגורמים הלינארים שלו זרים. נניח ש־$T$ סימטרית. ניעזר בלמה המופיע מיד אחרי ההוכחה הזו. נניח בשלילה שהם לא כולם שונים, אז $m_T(x) = (x - \lg)^2g(x)$ ואז: 
		\[ 0 = m_T(T)v = (T - \lg I)^{2}g(T) \implies \omega = g(T)v, \ (T - \lg I)^2\omega = 0 \]
		לכן בפרט $(T - \lg I)\omega = 0$ מהסעיף הקודם. סה''כ $\forall v \in V \co (T - \lg I)g(T) = 0$ וסתירה למינימליות. 
	\end{proof}
	
	\cola{$T$ סימטרית היא לכסינה. }
	זכרו מסקנה זו להמשך. היא תהפוך להיות להגיונית כאשר נדבר על המשפט הספקטרלי מעל $\R$. 
	
	\lem{נניח $T$ סמטרית ו־$\lg \in \R$, אם $(T - \lg I)^{2} = 0$ אז $T - \lg I = 0$. }\begin{proof}
		ידוע: 
		\[ \forall v \co 0 = \mut{(T - \lg I)^{2}v}{v} = \mut{(T - \lg I)v}{(T - \lg I)v} = \norm{(T - \lg I)v}^2 \implies (T - \lg I)v = 0 \]
	\end{proof}
	
	\theo{אם $V$ ממ''פ ו־$T \co V \to V$ ט''ל צמודה לעצמה, אז הע''ע של $T$ ממשיים. }\begin{proof}
		יהי $0 \neq v \in V$ ו''ע של $T$ שמתאים לע''ע $\lg$. נחשב: 
		\[ \lg v\norm{v}^2 = \mut{\lg v}{v} = \mut{T v}{v} = \mut{v}{Tv} = \mut{v}{\lg v} = \ol \lg \norm{v} \]
		ידוע $v \neq 0$ ולכן $\norm{v} \neq 0$ ונסיק $\lg v = \bar \lg $ ולכן $\lg \in \R$. 
	\end{proof}
	
	\theo{אם $V$ ממ''פ ו־$T \co V \to V$ ט''ל צמודה לעצמה, אז כל זוג $0 \neq u, v \in V$ ע''ע שונים, המתאימים לערכים $\ag, \bg \in \C$, מאונכים זה לזה. }\begin{proof}
		למעשה, מהטענה הקודמת $\ag, \bg \in \R$. כאן $Tu = \ag u, \ Tv = \bg v$, כאשר $\ag = \bg$. נחשב: 
		\[ \ag \mut{u}{v} = \mut{\ag u}{v} = \mut{Tu}{v} = \mut{u}{Tv} = \mut{u}{\bg v} = \bar \bg \mut{u}{v} = \bg \mut{u}{v} \]
		בגלל ש־$\bg \in \R$ מתקיים $\bg = \bar \bg$. ולכן $(\ag - \bg)\mut{v}{u} = 0$ מהעברת אגף וסה''כ $\mut{u}{v} = 0$ ואכן $u \perp v$. 
	\end{proof}
	
	\rmark{בחלק הבא יש שימוש קל במרחבים דואלים. בעבור סטונדטים שבעבורם מרחבים דואלים לא נכלל כחלק מלינארית 1א, אני ממליץ לקרוא את החלק הראשון של מרחבים דואלים בסוף הסיכום. }
	
	
	\begin{Theorem}[משפט ריס]
		יהי $V$ ממ''פ סופי ויהי $\vphi \in V^*$. אז קיים ויחיד וקטור $u \in V$ שמקיים $\forall v \in V \co \vphi(v) = \mut{v}{u}$. 
	\end{Theorem}\begin{proof}\,
		
		\textbf{קיום. }יהי $B = (b_i)_{i = 1}^{n}$ בסיס אורתונורמלי של $V$ (הוכחנו קיום בהרצאות קודמות). נסמן $u = \sum_{i =1}^{n}\ol{\phi(b_i)} b_i$. בכדי להראות $\forall v \in V \co \phi(v) = \mut{v}{u}$ מספיק להראות תכונה זו לאברי הבסיס $B$, כלומר נראה ש־$\forall 1 \le j \le n \co \phi(b_j) = \mut{b_j}{u}$. ואכן: 
		\[ \mut{b_j}{u} = \mut{b_j}{\sumni \ol{\phi(b_i)}b_i} = \sum_{i = 1}^{n}\underbrace{\ol{\ol{\phi(b_i)}}}_{b_i}\underbrace{\mut{b_j}{b_i}}_{\dg_{ij}} = b_j \quad \top \]
		\textbf{יחידות: }אם קיים וקטור נוסף שעבורו $\forall v \in V \co \phi(v) = \mut{v}{w}$ אז בפרט עבור $v = u - w$ נקבל: 
		\begin{align*}
			         &\phi(v) = \mut{v}{w} = \mut{v}{u} \\
			\implies &\mut{v}{u - w} = 0 \\ 
			\implies &0 = \mut{u - w}{u - w} = \norm{v - w}^{2} = 0 \\ 
			\implies &v - w = 0 \\
			\implies &v = w
		\end{align*}
		סה''כ הוכחנו קיום ויחידות כדרוש. 
	\end{proof}
	
	\subsubsection{העתקה צמודה להעתקה}
	\theo{יהי $V$ ממ''פ מנ''ס ותהי $T \co V \to V$ לינארית. אז קיימת ויחידה $T^* \co V \to V$ ומקיימת $\forall u ,v \in V \co \mut{Tu}{v} = \mut{u}{T^*v}$. }
	
	\begin{proof}
		לכל $v \in V$, נתבונן בפונקציונל הלינארי $\phi_V \in V^*$ המוגדר ע''י $\forall u \in V \co \phi_V(u) = \mut{Tu}{v}$. ממשפט ריס קיים ויחיד $T^*v \in V$ שעבורו $\forall u \in V \co \mut{Tu}{v} = \phi_V(u) = \mut{u}{T^*v}$. כלומר, ההעתקה $T^* \co V \to V$, קיימת ויחידה, ונותר להראות שהיא לינארית. עבור $v, w \in V$ ועבור $\ag, \bg \in \F$ מתקיים: 
		\begin{align*}
			\forall u \in V \co \quad & \mut{u}{T^* (\ag v + \bg w)} \\
			= &\mut{Tu}{\ag v + \bg w} \\
			= &\bar \ag \mut{Tu}{v} + \bar \bg \mut{Tu}{v} \\ 
			= &\bar \ag \mut{u}{T^* v} + \bar \bg \mut{u}{T^*w} \\
			= &\mut{u}{\ag T^*u + \bg T^*w}
		\end{align*}
		מסך נסיק ש־$T^*(\ag v + \bg w) = \ag T^* u + \bg T^* w$ מנימוקים דומים. 
	\end{proof}
	
	\defi{ההעתקה $T^*$ לעיל נקראת \textit{ההעתקה הצמודה ל־$T$}. }
	
	\textbf{דוגמאות. }
	מעל $\C^n$, עם המ''פ הסטנדרטי, נגדיר ט''ל $T_A \co \C^n \to \C^n$ עבור $A \in M_n(\C)$ מוגדרת ע''י $T_A(x) = Ax$. אז: 
	\[ \forall x, y \in \C^n \co \mut{T_A(x)}{y} = \mut{Ax}{y} = \ol{(Ax)^T} \cdot y = \ol{x^T}\cdot\ol{A^T}y\cdot = \ol{x^T}T_{\ol{A^T}}(y) = \mut{x}{T_{\ol{A^T}}y} \]
	כלומר, $(T_A)^* = T_{A^*}$ כאשר $A^* = \ol{A^T}$, וקראנו לה המטריצה הצמודה. 
	
	נבחין שהעתקה נקראת צמודה לעצמה אמ''מ $T^* = T$. 
	
	עוד נבחין שעבור העתקה הסיבוב $T \co \R^2 \to \R^2$ בזווית $\tg$, מתקיים ש־$T^*$ היא הסיבוב ב־$-\tg$, וכן היא גם ההופכית לה. כלומר $(T_{\tg})^* = T_{-\tg} = (T_\tg)\op$. זו תכונה מאוד מועילה וגם נמציא לה שם במועד מאוחר יותר. 
	
	\begin{Theorem}[תכונות ההעתקה הצמודה]
		יהי $V$ ממ''פ ותהיינה $T, S \co V \to V$ זוג העתקות לינאריות. נבחין ש־: 
		\begin{enumerate}[(A)]
			\item \hfil $(T^*)^* = T$
			\item \hfil $(T \circ S)^* = S^* \circ T^*$
			\item \hfil $(T + S)^* = T^* + S^*$
			\item \hfil $\forall \lg \in \F \co (\lg T)^* = \bar \lg (T^*)$
		\end{enumerate}
	\end{Theorem}
	
	\begin{proof}\,
		\begin{enumerate}[A)]
			\item \hfil $\forall u, v \in V \co \mut{T^* u}{v} = \ol{\mut{v}{T^* u}}  = \ol{\mut{Tv}{u}}  = \mut{u}{Tv} \implies (T^*)^* = T$
			\item \hfil $\mut{(T \circ S) u}{v} = \mut{Su}{T^*v} = \mut{u}{S^* T^*} \implies (TS)^* = T^*S^*$
			\item \hfil $\mut{(T + S)u}{v} = \mut{Tu}{v} + \mut{Su}{v} = \mut{u}{T^*v} + \mut{u}{S^*v}  = \mut{u}{T^*v + S^*v}$
			\item \hfil $\mut{(\lg T)u}{v} = \lg\mut{Tu}{v} = \lg \mut{u}{Tv} = \mut{u}{(\bar \lg T)v}$
		\end{enumerate}
		\envendproof
	\end{proof}
	
	\noti{העתקה צמודה לעצמה לעיתים קרובות (בעיקר בפיזיקה) מסמנים ב־$T^{\dag}$. באופן דומה גם מטריצה צמודה מסמנים ב־$A^{\dag}$. }
	
	\theo{בהינתן $B$ אורתונורמלי של $V$ אז $[T^*]_B = [T]^*_B$ (שימו לב: האחד צמוד מטריציוני, והשני העתקה צמודה)}
	
	\theo{$T$ צמודה לעצמה אמ''מ $\forall v \in V \co \mut{Tv}{v} \in \R$. }\begin{proof}
		\begin{alignat*}{4}
			&&& \text{ צמודה לעצמה} T \\
			& \iff && T = T^* \iff T - T^* = 0 \\
			& \iff && \forall v \in V \co \mut{(T - T^*)v}{v} = 0 \\
			& \iff && \forall v \in V \co \mut{Tv}{v} - \mut{T^*v}{v} = 0 \\
			& \iff && \forall v \in V \co \mut{Tv}{v} - \ol{\mut{Tv}{v}} = 0 \\
			& \iff && \forall v \in V \co \Re (\mut{Tv}{v}) + \Im(\mut{Tv}{v}) - \Re(\mut{Tv}{v}) + \Im(\mut{Tv}{v}) = 0 \\
			& \iff && \forall v \in V \co 2\Im(\mut{Tv}{v}) = 0 \iff \forall v \in V \co \Im(\mut{Tv}{v}) = 0 \iff \forall v \in V \co \mut{Tv}{v} \in \R \quad \top
		\end{alignat*}
		\envendproof
	\end{proof}
	
	
	\npage
	\section[פירוקים]{\en{Decompositions}}
	\subsection{המשפט הספקטרלי להעתקות}
	\subsubsection{ניסוח המשפט הספקטרלי להעתקות צמודות לעצמן}
	
	\begin{Theorem}[המשפט הספקטרלי להעתקה לינארית צמודה לעצמה]
		יהי $V$ ממ''פ ממימד סופי, ותהי $T \co V \to V$ ט''ל צמודה לעצמה. אז קיים ל־$V$ בסיס אורתוגונלי (או אורתונורמלי) שמורכב מו''ע של $T$. 
	\end{Theorem} \begin{proof}
		יהי $m_T(x)$ הפולינום המינימלי של $T$. נציג $m_T(x) = \prod_{i = 1}^{m}(x - \lg_i)^{d_i}$ כאשר $\lg_1 \dots \lg_n$ הע''ע השונים של $T$. מהטענה הקודמת $\lg_1 \dots \lg_n \in \R$. [הערה: התמשתנו במשפט היסודי של האלגברה מעל המרוכבים, והסקנו פירוק מעל $\R$]. בכדי להראות ש־$T$ לכסינה, עלינו להוכיח ש־$\forall 1 \le i \le m \co d_i = 1$. נניח בשלילה שזה לא מתקיים, אזי $m_T(x) = (x - \lg)^2 \cdot p(x)$ כאשר $\lg$ ע''ע כלשהו. כעת, לכל $v \in V$ מתקיים מהיות $T$ צמודה לעצמה (כלומר גם $p(T)$ צמוד לעצמו): 
		\begin{multline*}
			0 = \underbrace{m_T(T)}_{=0}(v) \implies 0 = \mut{m_T(T)(v)}{p(T)(v)} = \mut{(T - \lg I)(p(T)v)}{p(T)v} =\\ \mut{(T - \lg I)(p(T)v)}{(T - \lg I)(p(T)v)} = \norm{(T - \lg I)^2(p(T)v)}^2 = 0
		\end{multline*}
		ולכן $\forall v \in V \co (T - \lg I)(p(T)v) = 0$ ולכן $((x - \lg)(p(x))(T) = 0$ בסתירה למינימליות של $m_T(x)$. נאמר, מכפלת גורמים לינארים שונים, ולכן $T$ לכסינה, ונוכל לפרק את $V$ באמצעות: 
		\[ V = \bigoplus_{i = 1}^{m} \ker (T - \lg_i I) \]
		והמרחכים העצמיים הללו אורתוגונליים זה לזה, מטענה שהוכחנו. 
		נבנה בסיס $B_i \subseteq \ker (T - \lg_i)$ וסה''כ $\bigcup_{i = 1}^{m} B_i$ בסיס אורתוגונלי מלכסן של $T$. 
	\end{proof}
	\theo{יהי $V$ נ''ס מעל $\R$ ותהי $T \co V \to V$ ט''ל. אז $T$ צמודה לעצמה אמ''מ קיים לה בסיס אורתוגונלי מלכסן. }\begin{proof}
		מכיוון אחד, הוכחנו באמצעות המשפט הספקטרלי להעתקות לינאריות צמודות לעצמן. 
		מהכיוון השני, נניח שקיים ל־$V$ בסיס אורתוגונלי מלכסן של ו''ע של $T$. ננרמל לבסיס אורתונורמלי $B = (b_i)_{i = 1}^{n}$ של ו''ע של $T$, המתאימים ל־$\lg_1 \dots \lg_n$. עבור $v, u \in V$, נציג: 
		\[ u = \sum_{i = 1}^{n}\ag_i b_i, \ v = \sum_{i = 1}^{n}\bg_i b_i \]
		ונחשב: 
		\[ \mut{Tu}{v} = \mut{T\cl{\sum_{i = 1}^{m}\ag_i b_i}}{\sum_{i = 1}^{m}\bg_i b_i} = \sum_{i = 1}^{n}\sum_{j = 1}^{n}\ag_i \bg_j \mut{T b_i}{b_j} = \sum_{i = 1}^{n}\sum_{j = 1}^{n}\ag_i \bg_j \lg_i \underbrace{\mut{b_i}{b_j}}_{\dg_{ij}} = \sumni \ag_i \bg_i \lg_i \]
		מהצד השני: 
		\[ \mut{u}{Tv} = \mut{\sumni \ag_i b_i}{T\cl{\sumni \bg_i b_i}} = \sumni \sum_{j = 1}^{n}\ag_i \bg_i \mut{b_i}{Tb_j} = \sum_{i = 1}^{n}\sum_{j = 1}^{n}\ag_i \bg_j \lg_j \underbrace{\mut{b_i}{b_j}}_{\dg_{ij}} = \sum_{i = 1}^{n}\ag_i \bg_j \lg_i \]
		מטרנזטיביות שוויון, הראינו ש־$\mut{Tu}{v} = \mut{u}{Tv}$ ולכן $T$ צמודה לעצמה. השוויון לדלתא של כקוניקר נכונה מאורתוגונליות איברי הבסיס, והבי־לינאריות כי אנחנו מעל הממשיים. המשפט לא נכון מעל מהרוכבים. 
	\end{proof}
	הוכחה שהמשפט לא נכון מעל המרוכבים: ההעתקה $T(x) = ix$ היא העתקה סקלרית לינארית, לכן כל וקטור הוא ו''ע וכל בסיס מלכסן, בסיס אורתונורמלי כלשהו יהיה בסיס מלכסן על אף שההעתקה לא צמודה לעצמה, אלא אנטי־הרמיטית. 
	
	
	\subsubsection{ניסוח המשפט הספקטרלי בעבור העתקה כללית}
	
	המטרה: להבין לאילו העתקות בדיוק מתקיים המשפט הספקטרלי. מעל הממשיים, הבנו שאילו העתקות צמודות לעצמן. אז מה קורה מעל המרוכבים? 
	
	\theo{יהי $V$ ממ''פ נ''ס ותהי $T \co V \to V$ לינאריות. אם $B = (b_i)_{i = 1}^{n}$ בסיס אורתוגונלי לו''ע של $T$, אז $\forall 1 \le i \le n$ ו''ע של ההעתקה הצמודה. }
	כלומר: אם מתקיים המשפט הספקטרלי, אז הבסיס שמלכסן אורתוגונלית את $T$ מלכסן אורתוגונלית את הצמודה. 
	\begin{proof}
		יהי $i \in [n]$ ונסמן בעבורו את $\lg_i$ הע''ע המתאים לו''ע $b_i$. עבור $i \neq j \in [n]$ נחשב את $\mut{b_i}{T^*b_j}$: 
		\[ \mut{b_i}{T^*b_j} = \ol{\mut{Tb_i}{b_j}} = \ol{\mut{\lg _i b_i}{b_j}} = \lg_i \mut{b_i}{b_j} = 0 \]
		לכן $T^*b_j \in (\Sp\{b_i\}_{i = 1}^{n})^{\perp} \seq \Sp\{b_j\}$. משיקולי ממדים, הפריסה מממד $n - 1$ ולכן המשלים האורתוגונלי שלו מממד $1$ ולכן השוויון. סה''כ $T^* b_j \in \Sp\{b_j\}$ ולכן $b_j$ ו''ע של $T^*$ כדרוש. 
	\end{proof}
	
	\textbf{מסקנה. }אאם $V$ ממ''פ נ''ס ו־$T \co V \to V$ ט''ל עם בסיס מלכסן אורתוגונלי, אז $T, T^*$ מתחלפות כלומר $TT^* = T^*T$. \begin{proof}
		לפי הטענה הקודמת כל $b_i$ הוא ו''ע משותף ל־$T$ ול־$T^*$, ולכן: 
		\[ TT^*(b_i) = T(T^*(b_i)) = \bg_i T^*(b_i) = \bg_i \dg_i b_i = \ag_i \bg_i b_i = \ag_i T^(b_i) = T^*T(b_i) \]
		העתקה מוגדרת לפי מה שהיא עושה לבסיס ולכן $TT^* = T^*T$. 
	\end{proof}
	\defi{העתקה כזו המקיימת $AA^* = A^*A$ נקראת \textit{נורמלית} (או ``\textit{נורמאלית}'' בעברית של שנות ה־60). }
	
	מעתה ואילך, ננסה להראות שכל העתקה נורמלית מקיימת את התנאי של המשפט הספקטרלי (כלומר ניתן ללכסנה אורתוגונלית)
	
	\theo{(המשפט הספקטרלי) יהי $V$ ממ''פ נוצר סופית מעל $\C$, ותהי $T \co V \to V$ לינארית. אז קיים בסיס אורתוגונלי של ו''ע של $T$ אמ''מ $T$ נורמלית. }
	\lem{יהי $V$ ממ''פ ותהיינה $S_1, S_2 \co V \to V$  זוג ט''ל צמודות ולעמן ומתחלפות (כלומר $S_1S_2 = S_2 S_1$). אז קיים בסיס אורתוגונלי של $V$ שמורכב מו''עים משופים ל־$S_1$ ול־$S_2$. }\begin{proof}
		ידוע ש־$S_1$ צמודה לעצמה, לכן לפי המשפט הספקטרלי להעתקות צמודות לעצמן (לא מעגלי כי הוכח בנפרד בהרצאה הקודמת), קיים לה לכסון אורותגונלי ובפרט $S_1$ לכסינה. נציג את $V$ כ־$V = \bigoplus_{i = 1}^{m} \ker(S_1 - \lg_iI)$, כאשר $\lg _1 \dots \lg_m$ הע''עים השונים של $S_1$. לכל $1 \le i \le m$  מתקיים ש־$V_{\lg_i}$ (המרחב העצמי) הוא $S_1$־אינווריאנטי שהרי אם $v \in V_{\lg_i}$ ונחשב: 
		\[ S_1(S_2 v) = S_2(S_1 v) = S_2(\lg_i v) = \lg _i S_2v \implies S_2 v \in V_{\lg_i} \]
		כאשר $S_2|_{V_{\lg _i}} \co V_{\lg_i} \to V_{\lg_i}$ צמודה לעצמה, ולכן המפשט הספקטרלי לצמודות לעצמן אומר שבתוך $V_{\lg_i}$ ישנו בסיס אורתוגונלי של ו''עים מ־$S_2$. האיחוד של כל הבסיסים הללו מכל מ''ע של $S_1$ יהיה בסיס אורתוגונלי של ו''עים משותפים ל־$S_1$ ול־$S_2$. 
	\end{proof}
	\begin{proof}[הוכחת המשפט הספקטרלי. ]\,
		\begin{itemize}
			\item[$\implies$] לפי המסקנה הקודמת, אם ישנו לכסון אורתוגונלי $T$ בהכרח נורמלית. 
			\item[$\impliedby$] נגדיר $S_1 = \frac{T + T^*}{2}, \ S_2 = \frac{T - T^*}{2i}$. הן וודאי צמודות לעצמן מהלינאריות וכל השטויות ממקודם, והן גם מתחלפות אם תטרחו להכפיל אותן. מהטענה קיים ל־$V$ בסיס אורתוגונלי של ו''עים משותפים ל־$S_1, S_2$ ונסמנו $\{b_i\}^{n}_{i = 1}$ וגם $S_1b_i = \ag_i b_i, \ S_2b_i= \bg_i b_i$. אפשר גם לטעון ש־$\ag_i, \bg_i \in \R$ אבל זה לא מועיל לנו. נשים לב ש־$T = S_1 + iS_2$, כלומר $\forall i \in [n] \co T(b_i) = S_1(b_i) + iS_2(b_i) = \ag_i b_i + i\bg_i b_i = (\ag + i\bg_i)b_i$ וזהו בסיס אורתוגונלי של ו''עים של $T$. 
		\end{itemize}
	\end{proof}
	
	למעשה, הבנו מהפירוק של $S_1, S_2$ ש־$S_1$ נותנת את החלק הממשי של הע''ע ו־$S_2$ את החלק המדומה. 
	
	``אגב – לא השתמשתי במשפט היסודי של האלגברה''
	
	\textbf{נסכם: }יש לנו שתי גרסאות של המשפט הספקטרלי: 
	
	\textbf{משפט}(המשפט הספקטרלי מעל $\R$)\textbf{. }{$T$ סימטרית אמ''מ קיים בסיס א''נ של ו''ע. }
	
	\textbf{משפט}{(המשפט הספקטרלי מעל $\C$)\textbf{. } $T$ נורמלית אמ''מ קיים בסיס א''נ של ו''ע. }
	
	משום שמטריצה הרמיטית (וצמודה לעצמה באופן כללי) היא בפרט נורמלית כי מטריצה מתחלפת עם עצמה, נסיק שלצמודה לעצמה קיים בסיס אורתוגונלי מלכסן (בעמוד הכיוון ההפוך לא נכון מעל המרוכבים, שם הההעתקה יכולה להיות נורמלית ולא סתם הרמיטית). 
	
	\subsubsection{תוצאות ממשפט הפירוק הספקטרלי}
	
	\theo{תהי $T \co V \to V$ ט''ל, ו־$V$ ממ''פ מעל $\F \in \{\R, \C\}$, ויהי $B$ בסיס א''נ של $V$. אזי אם $A = [T]_B$: 
		\[ [T^*]_B = A = ([T]_B)^* \]}  
	\begin{proof}
		נזכר ש־: 
		\[ [T]_B = \pms{\vert &  & \vert \\ [Te_1]_B & \cdots & [Te_n]_B \\ \vert &  & \vert} \]
		נסמן $B = \{e_i\}_{i = 1}^{n}$ בסיס. נבחין ש־: 
		\[ Te_j = \sumni a_{ij}e_i, \ a_{ij} = \mut{Te_j}{e_i} \]
		נסמן ב־$C$ את המטריצה המייצגת $[T^*]_B$: 
		\[ c_{ij} = \mut{T^*e_j}{e_i} \]
		ונחשב: 
		\[ c_ij = \mut{T^*e_j}{e_i} = \mut{e_j}{Te_i} = \ol{\mut{Te_i}{e_j}} = a_{ij} \]
		
	\end{proof}
	
	\textbf{מסקנה: }אם $A$ נורמלית אז $T_A$ נורמלית מעל $\F^n$ אם הסטנדרטית. בפרט מתקיים עליה המשפט הספקטרלי. גם אם $A$ ממשית, הע''ע עלולים להמצא מעל $\C$ (אלא אם היא צמודה לעצמה, ואז הם מעל $\R$). 
	
	\theo{יהיו $x_1 \dots x_n, y_1 \dots y_n \in \R$. נניח $\forall i, j \in [n] \co i \neq j \implies x_i \neq x_j$. אז $\exists! p \in \R_{\le n - 1}[x] \co \forall i \in [n] \co p(x_i) = y_i$ עד לכדי חברות (באופן שקול: נניח $p$ מתוקן) }
	
	\begin{proof}
		ידוע שהפולינום מהצורה $p(x)  = \sum_{k = 0}^{n - 1} a_k x^k  = (1, x, x^2, \dots x^{n - 1}) (a_0 \dots a_{n - 1})^T$. למעשה, נקבל את מטריצת ונדרמונד: 
		\[ \underbrace{\pms{1 & x_1 & x_1^2 & \cdots & x_1^n \\ 1 & x_2 & x_2^2 & \cdots & x_2^{n - 1} \\ \vdots & \vdots & \vdots & \ddots & \vdots \\ 1 & x_n & x_n^2 & \cdots & x_n^{n - 1}}}_{\vc}\underbrace{\pms{a_0 \\ a_1 \\ \vdots \\ a_{n - 1}}}_{a} = \underbrace{\pms{y_1 \\ y_2 \\ \vdots \\ y_n}}_{y} \]
		וידוע שהדטרמיננטה של $\vc$ היא מטריצת ונדרמונד היא $\prod_{i < j} (x_i - x_j)$, שאיננה אפס מההנחה ש־$\forall i \neq j \co x_i \neq x_j$, ולכן למערכת המשוואות $(\vc \mid y)$ קיים ויחיד פתרון, הוא $a$, שמגדיר באופן יחיד את מקדמי הפולינום. 
		
		אם $x_i = y_i$ בפולינום לעיל, אז $\forall a \in \C \co f(\bar a) = \ol{f(a)} \implies f \in \R[x]$. הוכחה: נניח בשלילה, אז $\exists \ag \in \C\setminus \R \co f(\ag) = 0$ כך ש־$f(\bar \ag) = 0$. אזי $0 \neq f(\bar \ag) = \ol{f(\ag)}  = 0$ וזו סתירה. 
	\end{proof}
	
	\rmark{הפולינום שמקיים זאת נקרא \textit{פולינום לגראנג'} והוא בונה אינטרפולציה די נחמדה אך יקרה חישובית. ניתן לחשב את הפולינום מפורשות באופן הבא: 
	\[ p(x) = \sum_{i = 1}^{n}\cl{y_i \prod_{j = 1}^{n}\frac{x - x_j}{x_i - x_j}} \]}
	
	\theo{תהי $A \in M_n(\F)$ נורמלית, אז קיים פולינום $f(x) \in \R[x]$ כך ש־$A^* = f(A)$. }
	\textit{הערה: }באופן כללי \textbf{לא} נכון שאם $A, B$ מתחלפות אז $\exists f(x) \in \F[x] \co f(A) = B$. 
	\begin{proof}
		עבור $A$ נורמלית מהמשפט הספקטרלי קיים בסיס אורתונורמלי מלכסן ולכן קיימת $P$ הפיכה כך ש־$P\op APש = \diag(\lg_1 \dots \lg_n)$. לכן $P\op A^* P = \diag(\bar \lg_1 \dots \bar \lg_n)$. נשתמש במשפט לפיו יש פולינום $f \in \R[x]$ כך ש־$f(x_i) = \bar x_i$ ובפרט בעבור $x_i = \lg_i$ קיים פולינום עבורו $f(\lg_i) = \bar \lg_i$. אזי 
		\[ f(\diag(\lg_1 \dots \lg_n)) = \diag(f(\lg_1) \dots f(\lg_n)) = \diag(\bar \lg_1 \dots \bar \lg_n) \]
		\[ f(P\op A P) = P\op f(A) P = P \op A^* P \implies f(A) = A^* \]
		עוד נבחין ש־$\deg f = n - 1$. 
	\end{proof}
	
	ננסה להבין מי הן $A \in M_2(\R)$ שהן נורמליות. מעל $\C$ הן פשוט לכסינות. נבחין ש־: 
	\begin{align*}
		A = \pms{a & b \\ c & d}, \ A^* = f(A) \in \R_{\le1}[x] = \ag A + BI, \ A = \ag A^T + \bg I = \ag(\ag A + \bg I) + \bg I \\ \implies \pms{a & b \\ c & d} = \pms{\ag^2 a + \bg(\ag + 1) & \ag^2 b \\ \ag^2 c & \ag^2 d + \bg (\ag + 1)} \\
		\begin{cases}
			(b \land c \neq 0 ) \quad \ag = 1 \implies  A = A + 2 \bg I \implies \bg = 0, \ A = A^T \cancel{+ \bg I} \\
			(b \land c \neq 0) \quad \ag = -1 \implies A^T = - A  +\bg I  \implies A + A^T = \pms{\bg & 0 \\ 0 & \bg} \implies b = -c, \ A = \pms{a & b \\ -b & a} \\
			(b \lor c = 0) \implies A = \pms{a & 0 \\ 0 & d}
		\end{cases}
	\end{align*}
	המקרה השני – זה פשוט סיבובים, אבל בניפוח (כי הדטרמיננטה היא $a^2 - b^2$). 
	
	בכל מקרה, מסקנה מהמשפט הקודם. 
	\theo{אם $T \co V \to V$ נורמלית, אז $\exists f \in \R[x] \co f(T) = T^*$. }
	\begin{proof}
		נבחר בסיס א''נ $A^* = [T^*]_B, \impliedby A = [T]_B$. כבר הוכחנו שאם $T$ נורמלית אז $A$ נורמלית ולכן מהמשפט הקודם קיים $f$ מתאים כך ש־$[T^*]_B = A^* = f(A) = f([T]_B) = [f(T)]_B$. סה''כ $[T^*]_B = [f(T)]_B$ ומחח''ע העברת בסיס $T^* = f(T)$ כדרוש. 
	\end{proof}
	
	אם $T \co V \to V$ ט''ל, $U, W \subseteq V$ תמ''וים $T$־איוונריאנטי כך ש־$U \oplus W = B$. אם $\bc$ בסיס של $V$, כאשר קישא של הבסיס הוא הבסיס של $U$ אז: 
	\[ [T]_\bc = \pms{[T|_U]_\bc &  \\  & [T|_W]_\bc} \]
	בפרט בעבור ניצבים $U \subseteq V \implies V = U \oplus U^{\perp}$. ניעזר בכך כדי להוכיח את המשפט הבא: 
	\theo{אם $U \subseteq V$ תמ''ו אינוואריאנטי ביחס ל־$T$ אז $U^{\perp}$ הוא $T^*$־אינוואריאנטי. }\begin{proof}
		יהי $w \in U^{\perp}$. רוצים להראות $T^*w \in U^{\perp}$. יהי $u \in U$ אז: 
		\[ \mut{T^* w}{u} = \mut{w}{Tu} = \mut{w}{u'}, \ u' \in U \implies \mut{w}{u'} = 0 \quad \top \]
	\end{proof}
	\theo{בעבור $T \co V \to V$ נורמלית, אם היא $U$־אינוואריאנטי אז גם $T^*$ הוא $U$־אינווארינטי}
	\begin{proof}
		נבחין ש־$T^* = f(T)$ כלשהו, וכן $U$ הוא $T$־אינוואריאנטי ולכן $U$ הוא $f(T)$־איוו' וכאן די גמרנו את ההוכחה. 
	\end{proof}
	מסימטריות $U^{\perp}$ הוא $T^*$, מהמשפט גם $(T^*)^*$ איונ' ולכן $T$־אינוואריאנטי. 
	
	%TODO
	\theo{יהי $V$ מעל $\R$ מ''ו וכן $T \co V \to V$ ט''ל. אז קיים $U \subseteq V$ שהוא $T$־איונ' וממדו לכל היותר $2$. }
	\textit{הערה: }מעל $\C$ ``זה מטופש'' כי הפולינום מתפרק (ואז המרחב העצמי יקיים את זה). 
	\begin{proof}
		נפרק ל־$m_T(x)$ מינימלי ו־$g(x)$ גורם אי־פריק כך ש־$m_T(x) = g(x)h(x)$. לכל $g$ אי פריק ב־$\R$ הוא לינארי  הוא ממעלה $2$, מהמשפט היסודי של האגלברה ומהעובדה ש־$m_T(x) = 0 \implies m_T(\bar x) = 0$. 
		\begin{itemize}
			\item אם $g$ לינארי אז יש ע''ע ממשי של $T$ מה שנותר $U$ (שנפרש ע''י הו''ע) ממד $1$. 
			\item אם $\deg g = 2$ כמובן שניתן להניח $g$ מתוקן. ז''א $g(x) = x^2 + ax + b$ אז $g(T)$ אינו הפיך (מלמת החלוקה לפולינום מינימלי) כלומר $\exists 0 \neq v \in \ker g(T)$. 
			לכן: 
			\[ (T^2 + aT + bI)v = 0 \implies T^2v = -a T v - bv \]
			ולכן $U = \Sp(v, Tv)$ תמ''ו עם ממד לכל היותר 2 וגם נשמר תחת $T$. 
		\end{itemize}
	\end{proof}
	
	\textit{הערה: }בעבור נורמלית הטענה נכונה ללא תלות במשפט היסודי של האלגברה. 
	
	לכן, בעבור נורמליות, מהמשפט הספקטרלי ומטענות קודמות, עבור $T \co V \to V$ ממשית קיים בסיס א''נ $\bc$ של $V$ שבעבורו המטריצה המייצגת של $T$ היא מטריצת בלוקים $2 \times 2$ מצורה של $\pms{a & b \\ -b & a}$: 
	\[ [T]_\bc = \diag\cl{\pms{a_1 & b_1 \\ -b_1 & a_1} \cdots \pms{a_k & b_k \\ -a_k & b_k}, \ \lg_1 \cdots \lg_m} \]
	כאשר כמובן $2k + m = n$. 
	
	
	
	\subsection{מטריצות אוניטריות}
	\defi{יהי $V$ ממ''פ. אז $T \co V \to V$ תקרא \textit{אוניטרית} (אם $\F = \C$) או \textit{אורתוגונלית} (אם $\F = \R$) אם $T^*T = I$ או במילים אחרות $T^* = T\op$ (מהגדרת הפיכה). }
	ברור שט''ל כזו היא נורמלית. 
	\textbf{דוגמה. }עבור $T_\tg$ הסיבוב ב־$\tg$ מעלות, במישור $\R^2$, אז $(T_\tg)^* = T_{-\tg} = T\op_{\tg}$.  
	\textbf{דוגמה. }עבור $T$ שיקוף מתקים $T^2 = I$ וכן $T^* = T$ וסה''כ $T^* = T = T\op$. 

	\textbf{משפט. }$T$ איזומטריה אמ''מ מתקיים אחד מבין הבאים: 
	\begin{enumerate}
		\item (ההגדרה) \hfil $T^* = T\op$
		\item \hfil $TT^* = T^*T = I$
		\item \hfil $\forall u, v \in V \co \mut{Tu}{Tv} = \mut{u}{v}$
		\item $T$ מעבירה כל בסיס א''נ של $V$ לבסיס א''נ של $V$
		\item $T$ מעבירה בסיס א''נ אחד של $V$ לבסיס א''נ של $V$ לבסיס א''נ [מקרה פרטי של 4 בצורה טרוויאלית, אך גם שקול!]
		\item \hfil $\forall v \in V \co \norm{Tv} = \norm{v}$
	\end{enumerate}
	כלומר: היא משמרת זווית (העתקה פנימית) וגודל. 
	
	
	\defi{העתקה $T \co V \to V$ (כאשר $V$ ממ''פ) תקרא \textit{איזומטריה} אם $\forall v \in V \co \norm{v} = \norm{Tv}$}
	
	באופן כללי אוניטרית/אורתוגונלית שקולות לאיזומטריה ליניארית (כלומר שם כללי לאורתוגונליות/אוניטריות יהיה איזומטריות). 
	
	\rmark{איזומטריה, גם מחוץ לאלגברה לינארית, היא פונקציה שמשמרת נורמה/גודל. }
	
	\rmark{אפשר להסתכל על מתכונה 4 על איזומטריות לינאריות כעל איזומורפיזם של ממ''פים. }
	
	\begin{proof}נפרק לרצף גרירות
		\begin{enumerate}
			\item[$1 \to 2$] \hfil $T^* = T\op \implies \mut{Tv}{Tu} = \mut{v}{T^*Tu} = \mut{v}{u}$
			\item[$2 \to 3$] נאמר ש־$(v_1 \dots v_n)$ א''נ. צ.ל. $(Tv_i)_{i = 1}^{n}$ א''נ. לשם כך נצטרך להוכיח את שני התנאים – החלק של האורתו והחלק של הנורמלי. בשביל שניהם מספיק להוכיח ש־: 
			$\mut{Tv_i}{Tv_j} = \mut{v_i}{v_j}  = \dg_{ij}$
			\item[$3 \to 4$]טרוויאלי
			\item[$4 \to 5$]יהי $(v_1 \dots v_n)$ בסיס א''נ כך ש־$(Tv_1 \dots Tv_n)$ א''נ. אז: 
			\begin{align*}
				v = \sumni \ag_i v_i \implies &\norm{v}^2 = \mut{\sumni \ag_i v_i}{\sumni \ag_i v_i} = \sumni |\ag_i|^2 \\
				&\norm{Tv}^2 = \mut{T\cl{\sumni \ag_i v_i}}{T\cl{\sumni \ag_i v_i}} = \mut{\sumni \ag_i T(v_i)}{\sumni \ag_i T(v_i)} = \sum |\ag_i|^2
			\end{align*}
			\item[$5 \to 1$]מניחים $\forall v \in V \co \norm{Tv} = \norm{v}$. ידועות השקילויות הבאות: 
			\[ T^* = T\op \iff T^*T = I \iff T^*T - I = 0 \]
			בעבר ראינו את הטענה הבאה: נניח ש־$S$ צמודה לעצמה וכן ש־$\forall v \co \mut{Sv}{v} = 0$, אז $S = 0$. במקרה הזה: 
			\[ S := T^*T - I \implies S^* = (T^*T - I)^* = (T^*T)^* - I^* = T^*(T^*)^* - I =S \implies S^* = S \]
			והיא אכן צמודה לעצמה. עוד נבחין ש־: 
			\[ \mut{Sv}{v} = \mut{(T^* T - I)v}{v} =\mut{T^*Tv}{v} - \mut{v}{v} = \mut{Tv}{Tv} - \mut{v}{v} = \norm{Tv}^2 - \norm{v}^2 = 0 \]
			השוויון האחרון נכון מההנחה היחידה שלנו ש־$\norm{Tv} = \norm{v}$. סה''כ $TT^* - I = 0$. סה''כ הוכחנו $TT^* - I = 0$ שזה שקול ל־$T^* = T\op$ מהשקילויות לעיל כדרוש. 
		\end{enumerate}
	\end{proof}
	
	\theo{תהי $T \co V \to V$ אוני'אורתו', ו־$\lg$ ע''ע של $T$. אז $|\lg| = 1$} \begin{proof}
		יהי $v$ ו''ע של הע''ע $\lg$. אז: 
		\[ |\lg|^2 \mut{v}{v} = \lg\bar\lg \mut{v}{v} = \mut{Tv}{Tv} = \mut{v}{v} \]
	\end{proof}
	\defi{תהי $A \in M_n(\F)$. אז $A$ תיקרא \textit{אוניטרית}/\textit{אורתוגונלית} אם $A^* = A\op$. }
	\theo{אוניטרית אמ''מ $A\ol{A^T} = I$. }
	\theo{אורתוגונלית אמ''מ $AA^T = I$. }
	\rmark{אוניטרית בה מלשון unit – היא שומרת על הגודל, על וקטורי היחידה (ה־unit vectors). }
	\theo{יהי $\bc$ בסיס א''נ של $V$ ו־$T \co V \to V$ אז $T$ אוניטרית/אורתוגונלית אמ''מ $A = [T]_B$ אוניטרית/אורתוגונלית. }
	\begin{proof}
		\[ AA^* = [T]_\bc[T^*]_\bc = [TT^*]_\bc, I = AA^* \iff [TT^*]_\bc = I \iff TT^* = I \]
	\end{proof}
	
	``היה לי מרצה בפתוחה שכתב דבר לא מדויק בסיכום, ואז הוריד נקודות לסטודנטים שהסתמכו על זה. הוא אמר שזה מתמטיקה, אתם אחראים להבין מה נכון או לא – גם אם כתבתי שטויות''. 
	
	\noti{א''נ = אוניטרית בהקשר של מטריצות (בהקשר של מרחבים – אורתונורמלי)}
	
	\theo{התאים הבאים שקולים על $A \in M_n(\F)$. 
		\begin{enumerate}
			\item $A$ א''נ
			\item שורות $A$ מהוות בסיס א''נ של $\F^n$ (ביחס למכפלה הפנימית הסטנדרטית)
			\item עמודות $A$ מהוות בסיס א''נ של $\F^n$. 
			\item (ביחס למכפלה הפנימית הסטנדרטית) \hfill $\forall u, v \in \F^n \co \mut{Au}{Av} = \mut{u}{v}$
			\item (ביחס למכפלה הפנימית הסטנדרטית) \hfill $\forall v \in \F^n \co \norm{Av} = \norm{n}$
	\end{enumerate}}
	\textit{הערה שלא קשורה למשפט: }נאמר ש־$[T^*]_B = [T]_B^*$ אמ''מ $B$ בסיס \textbf{א''נ}. עבור בסיס שאינו א''נ זה לא בהכרח מתקיים. 
	
	\textit{הערה נוספת: }זה בערך אמ''מ כי יש כמה מקרי קצה כמו מטריצת האפס. 
	
	\begin{proof}\,
		\begin{itemize}
			\item[$1 \lra 2$] נוכיח את הגרירה הראושנה
			\[ \pms{- & v_1 & - \\ & \vdots \\ - & v_n & -}\pms{\vert & & \vert\\ \bar v_1^T & \cdots & \bar v_n^T \\ \vert & & \vert} = AA^* = I \iff \text{א''נ}\, A \implies v_i \bar v_j^T = \dg_{ij} \] 
			הטענה האחרונה שקולה לכך ש־$v_1 \dots v_n$ בסיס א''נ (ביחס למ''פ הסטנדטית של $\F^n$)
			\item[$1 \lra 3$]מספיק להוכיח $A$ א''נ אמ''מ $A^T$ א''נ. מסימטריה ($(A^T)^T = A$) למעשה מספיק להוכיח $A$ א''נ גורר $A^T$ א''נ. נוכיח: 
			\[ A^*A = I \implies A^T\bar A = I \implies (A^T)^* = \bar A \implies A^T(A^T)^* = I \]
			\item[$4 \lra 1$]נתבונן ב־$T_A \co \F^n \to \F^n$ כאשר $\ec$ הבסיס הסטנדרטי. אז $[T_A]_{\ec} = A$. אז $T_A$ א''נ אמ''מ $[T_A]_{\ec} =: A$. אז: 
			\[ \mut{Au}{Av} = \mut{T_Au}{T_Av} = \mut{u}{v} \]
			\item[$5 \lra 1$] אותה הדרך כמו קודם. 
		\end{itemize}
	\end{proof}
	
	\subsubsection{צורה קאנונית למטריצה אוניטרית}
	\textbf{שאלה. }מהן המטריצות $A \in M_2(\R)$ האורתוגונליות? \begin{proof}[התשובה]
		בהינתן $A = \binom{a\, b}{c\, d}$ מהיות העמודות והשורות מהוות בסיס א''נ:
		\[ \begin{cases}
			a^2 + b^2 = 1 \\
			c^2 + d^2 = 1 \\
			a^c + c^2 = 1 \\
		\end{cases} \implies a = \cos\ta, \ b = \sin\ta \]
		עוד נבחין ש־$ac + bd = 0$ כי: 
		\[ AA^T = I \implies \pms{a & b \\ c & d}\pms{a & c \\ b & d} = \pms{a^2 + b^2 & ac + bd \\ ac + bd & c^2 + d^2} = I \]
		סה''כ מכך ש־$a^c + c^2 = 1$ ו־$b^2 + d^2 = 1$ נקבל שתי צורות אפשריות: 
		\[ A_1:= A = \pms{\cos\ta & \sin\ta \\ \sin\ta & -\cos\ta} \lor A_2:= A = \pms{\cos\ta&\sin\ta \\ -\sin\ta & \cos\ta} \]
		נבחין ש־$A_2$ הוא סיבוב ב־$\ta$, ו־$A_1$ שיקוף ניצב ביחס ל־$\frac{\ta}{2}$. זה לא מפתיע שכן $\det A_1 = -1, \ \det A_2 = 1$. 	
	\end{proof}
	``דרך נוספת לראות את זה'': 
	\[ a = \cos\ta \implies b = \sin\ta, \ c = \sin\vphi \implies d = \cos\vphi \]
	אז (עד לכדי סיבוב)
	\[ \cos\ta \sin\vphi + \sin\ta \cos\vphi \implies \sin(\ta + \phi) = 0 \implies \ta + \phi = 0 \lor \ta + \phi = \pi \]
	במקרה הראשון ש־$\phi = \ta$ קיבלנו סיבוב, ובמקרה השני נקבל ש־$\vphi = \pi - \ta$ ואז $\sin(\pi - \ta) = \sin \ta$ כדרוש. 
	
	ננסה להבין יותר טוב למה הן מסובבות בצורה הזו. $A_2$ מטריצה מוכרת אך $A_1$ פחות. נתבונן בפולינום האופייני שלה: 
	\[ f_{A_1}(x) = \sof{\begin{matrix}
			x - \cos\ta & -\sin\ta \\ -\sin\ta & x + \cos\ta
	\end{matrix}} = x^2 - \cos^2 \ta - \sin^2\ta = (x + 1)(x - 1) \]
	אזי הע''ע $-1, +1$ (שימו לב ש־$A_2$ לא לכסינה מעל $\R$). 
	\begin{multline*}
		A\pms{\cos\frac{\ta}{2}\\\sin\frac{\ta}{2}} = \pms{\cos\ta & \sin\ta \\ \sin\ta & -\cos\ta} \pms{\cos\frac{\ta}{2} \\ \sin\frac{\ta}{2}} = \pms{\cos \ta \cos \frac{\ta}{2} + \sin\ta \sin\frac{\ta}{2} \\ \sin\ta\cos\frac{\ta}{2} - \cos\ta\sin\frac{\ta}{2}} = \pms{\cos\cl{\ta - \frac{\ta}{2}} \\ \sin\cl{\ta - \frac{\ta}{2}}} \\
		= \pms{\cos\ta/2 \\ \sin\ta/2} \quad \cl{\frac{\ta}{2} \mapsto \frac{\ta}{2} + \pi}
		= \pms{\cos\cl{\frac{\ta}{2} - \frac{\pi}{2}} \\ \sin\cl{\frac{\ta}{2} - \frac{\pi}{2}}} = \pms{-\cos\cl{\frac{\ta}{2} + \frac{\pi}{2}} \\ -\sin\cl{\frac{\ta}{2} + \frac{\pi}{2}}} 
	\end{multline*}
	[אני ממש חלש בטריגו ואני מקווה שאני לא מסכם דברים לא נכונים. תבדקו אותי פעמיים כאן בחלק הזה. גם המרצה עשה את ההחלפה המוזרה של $\ta/2 \to \ta/2 + \pi/2$]. 
	
	''אם הייתם רוצים תקופות מבחנים נורמליות הייתם צריכים להיוולד בזמן אחר``
	
	''ומה, אתם חושבים שאחרי שהפקולטה דחתה בשבוע [את המבחנים] היא תיאמה את זה עם הפקולטות האחרות? הם דיברו איתם כמה ימים אח''כ``
	
	\begin{Collary}[הצורה הנורמלית של ט''ל אורתוגונלית]
		תהי $T \co V \to V$ אורתוגונלית. אז קיים בסיס א''נ של $V$, שביחס אליו המטריצה המייצגת את $T$ היא מהצורה: 
		\[ \pms{A_{\ta_1} \\ &\ddots \\ &&A_{\ta_n} \\ &&& 1 \\ &&&&\ddots \\ &&&&&1 \\ &&&&&&-1 \\ &&&&&&& \ddots \\ &&&&&&&&-1} \]
		כאשר: 
		\[ A_{\ta_i} = \pms{\cos \ta_i & \sin \ta_i \\ -\sin \ta_i & \cos \ta_i} \]
		(אוניטרית לא מעיינת כי היא לכסינה)
	\end{Collary}
	
	\textbf{הגיון: }
	
	אורתוגונלית, לכן נורמלית, לכן נראית בצורה של בלוקים $2 \times 2$ של ע''ע. הע''ע מגודל $1$ כי היא אורתוגונלית, והם חייבים להיות ממשיים על מעגל היחידה הממשי. המטריצה $A_{\ta}$ חייבת להיות אורתוגונלית מגודל $2 \times 2$ כי כל תמ''ו שם הוא $T$־אינוואריאנטי, כלומר אפשר לחלק את $T$ לבלוקים מתאימים ובפרט $T$ המצומצמת גם אורתוגונלית, ולכן $A_{\ta}$ סה''כ אורתוגונלית. כאשר $A_{\ta_i} = \pms{a & b \\ -b & a}$ ו־$b \neq 0$ נשארנו עם המטריצות הללו. 
	
	
	\begin{proof}
		ידוע עבור נורמלית: 
		\[ A = \pms{\bms{a_1 & b_1 \\ -b_1 & a_1} \\ & \ddots \\ &&\bms{a_m & b_m \\ -b_m & a_m} \\ &&& \lg_1 \\ &&&&\ddots \\ &&&&&\lg_k} \]
		נסמן $\square_i = \binom{\,\, a_i \, b_i}{-b_i \, a_i}$.
		במקרה הזה משום שהיא אורתוגונלית על $\R$ אז $\lg_i = \pm1$ כי $|\lg_i| = 1$. נתבונן במטריצה $\square_i$ כלשהי, אז $\square_i$ הנפרש ע''י $u_k, u_{k + 1} =: U$ מקיים: 
		\[ [T_{|U}]_{B_U} = \square_i = \pms{a_i & b_i \\ -b_i & a_i}\quad [Tu_k]_{U_B} = \pms{a \\ -b}, \quad [Tu_{k + 1}] = \pms{b \\ a} \]
		ומשום שהצמצום של אורתוגונלית על מ''ו $T$־אינוואריאנטי היא עדיין אורתוגונלית, והיא בהכרח מהצורה של מטריצת הסיבוב לעיל. המטריצה של שיקוף וסיבוב ב־$\frac{\ta}{2}$ לכסינה ולכן להפוך לע''ע $\lg_1 \dots \lg_n$ (עד לכדי סדר איברי בסיס) שהם בהכרח מגודל $\pm 1$ בכל מקרה, ויבלעו בשאר הע''ע, ובכך סיימנו. 
	\end{proof}
	
	אבל האם הייצוג יחיד? ננסה להבין את יחידות הייצוג עבור נורמלית כללית, ומשם לגזור על אורתוגונלית. 
	
	\theo{כל שתי מטריצות בצורה לעיל שמייצגות את אותה $T \co V \to V$ נורמלית, שוות עד כדי סדר הבלוקים על האלכסון. }
	
	(יש כאן מה להוכיח רק בעבור $\R$, שכן מעל $\C$ לכסין). 
	\begin{proof}
		ידוע שבעבור $\lg_1 \dots \lg_k$ ע''עים: 
		\[ f_T(x) = \cl{\prod (x - \lg_i)} \cl{\prod (x^2 - 2a_ix + a_i^2 + b_i^2)} \]
		כאשר המכפלה הראשונה באה מהע''עים והשניה מהריבועים $\square_i$. נבחין שלכל תמ''ו $a_i$ נקבבע ביחידות, ולכן $b_i$ נקבל ביחידות עד כדי סימן (נסיק זאת מהפולינום האופייני). ברור שהע''עים נקבעים ביחידות עוד מההרצאות הראשונות. 
	\end{proof}
	אז מאיפה בה שינוי הכיוון של $b$, בעבור מטריצות אורתוגונליות? כלומר, מדוע $A_{\ta_i}$ שקולה ל־$A_{-\ta_i}$? זאת כי הן דומות באמצעות ההעתקה שהופכת את הצירים, מה ששקול ללהחליף את עמודות $A_{\ta_i}$. 
	
	\textbf{תרגיל. }חשבו: (כולל פתרון)
	\[ \pms{-1 & 0 \\ 0 & -1}\pms{a & b \\ -b & a}\pms{1 & 0 \\ 0 & -1} = \pms{a & -b \\ b & a} \]
	מכאן נסיק שאכן המטריצות להלן דומות עד לכדי שינוי בסיס, וזו הסיבה שלא איכפת לנו מהסימן של $b$. 
	
	\rmark{למעשה, משום שהמטריצות $\square_i$ אינן פריקות למרחבים אינווראינטים קטנים יותר, ומיחידות הפירוק של הפולינום האופייני, הצורה הקאנונית שהוצגה לעיל היא צורת הג'ורדן שלה, ודי בכך כדי להוכיח את היחידות (כי הוכחנו את יחידות צורת ג'ורדן). }
	
	\subsubsection{המשפט הספקטלי בניסוח מטריציוני}
	\begin{Theorem}[המשפט הספקטרלי ``בשפה קצת מטרציונית'']
		תהי $A \in M_n(\F)$ מטריצה סימטרית (מעל $\R$)/נורמלית (מעל $\C$). אז קיימת  מטריצה $P$ אורתוגונלית/אוניטרית (בהתאם לשדה ממנו יצאנו), ומטריצה אלכסונית $D$ כך ש־
		$ A = P\op D P $
	\end{Theorem}
	כלומר – מטריצת מעבר הבסיס של המשפט הספקטרלי, שמעביר אותנו לפירוק הספקטרלי, היא איזומטריה. למעשה חיזקנו את המשפט הספקטרלי – המעבר לבסיס המלכסן, מסתבר להיות מיוצג ע''י מטריצות איזומטריות. 
	
	המרצה מדגיש שלא השתמשנו במשפט הזה בכלל על בסיסים ועל וקטורים – אפשר לתאר עולם הדיון של המטריצות, משום שהוא עולם דיון הומורפי להעתקות ולמרחבים וקטורים, בלי לדבר בכלל על העתקות ומרחבים וקטורים. המשפט מתאר באופן טהור מטריצות בלבד. 
	
	\lem{תהי $A \in M_n(\F)$ מטריצה ריבועית, וכן $\{e_1 \dots e_n\}$ בסיס א''נ של $V$. נניח ש־$A$ היא מטריצת המעבר מבסיס $\{e_1 \dots e_n\} \to \{v_1 \dots v_n\}$. אז $A$ איזומטריה אמ''מ $\{v_1 \dots v_n\}$ בסיס אורתונורמלי. }
	\begin{proof}[הוכחת המשפט. ]
		תהי $T_A \co \F^n \to \F^n$ כך ש־$T_A(x) = Ax$. אז $A= [T_A]_\ec$ כאשר $\ec = \{e_1 \dots e_n\}$ הבסיס הסטנדרטי. ידוע של־$T_A$ יש בסיס אורתונורמלי מלכסן, כלומר קיים בסיס א''נ $\bc$ כך ש־$[T_A]_\bc = D$ כאשר $D$ אלכסונית כלשהית. נבחין ש־$[T_A]_\bc = [Id]^\ec_\bc [T_A]_\ec [Id]_\ec^\bc$, נסמן $P = [Id]^\ec_B$ ונבחין ש־$[T_A]_\bc = PAP\op$ ומהלמה $P$ מטריצת מעבר מבסיס א''נ לבסיס א''נ ולכן איזומטריה. נכפיל בהופכיות ונקבל $A = P\op DP$. 
	\end{proof}
	
	``יאללה הפסקה? לא!''
	
	\subsection{פירוק פולארי}
	\subsubsection{מבוא, וקישור לתבניות בי־לינאריות}
	\rmark{במקרה של $\F= \R$ נקבל ש־}
	\[ A = P\op DP \implies PP^T = I \implies P\op = P^T \implies A = P^TDP \]
	מה שמחזיר אותנו לתבניות בי־לינאריות. נוכל לקשר את זה לסינגטורה. זאת כי $A$ לא רק דומה, אלא גם חופפת ל־$D$. גם מעל $\C$ נקבל דברים דומים, אך לא במדויק, שכן מכפלה פנימית מעל $\C$ היא ססקווי־בי־לינארית ולא בי־לינארית רגילה. 
	
	\theo{עבור $A \in M_n(\C)$ נורמלית, אז
		\begin{itemize}
			\item $A^* = A$ (צמודה לעצמה) אמ''מ כל הע''עים שלה ממשיים. 
			\item $A^* = A\op$ אמ''מ כל הע''ע שלה מנורמה $1$. 
	\end{itemize}}
	את הכיוון $\impliedby$ כבר הוכחנו. נותר להוכיח את הכיוון השני. 
	\begin{itemize}
		\item נניח שכל הערכים העצמיים ממשיים, ו־$A$ נורמלית. נוכל להשתמש במשפט הספקטרלי עליה: לכן קיימת מטריצה אוניטרית $P$ ואלכסונית $\Lg$ כך ש־$A = P\op \Lg P$. ידוע $\Lg \in M_n(\R)$ כי אלו הע''ע מההנחה. נבחין ש־: 
		\[ A^* = P^* \Lg^* (P\op)^* = P\op \Lg P = A \]
		כי $PP^* =I$ ו־$\Lg$ אוניטרית (אז ה־transpose לא עושה שום דבר) מעל $\R$ (אז ההצמדה לא עושה שום דבר). 
		\item נניח $A$ נורמלית וכל הע''ע מנורמה $1$. נוכיח $A$ אוניטרית. בעבור הפירוק הספקטרלי לעיל $A = P\op \Lg P$ נקבל כאן ש־$\Lg$ אוניטרית, ומהמשפט הספקטרלי $P$ אונטרית גם כן. $A$ מכפלה של 3 אוניטריות ולכן אוניטרית. 
		
		(הסיבה שמכפלה של אוניטריות היא אונטרית: בעבור $A, B$ א''נ מתקיים
		\[ \forall v \in V \co \mut{ABv}{ABv} = \mut{Bv}{Bv} = \mut{v}{v} \]
		משמרת מכפלה פנימית, וזה שקול להיותה אוניטרית ממשפט לעיל)
	\end{itemize}
	
	\textbf{תזכורת: }אם $V$ ממ''פ מעל $\F$, אז $T \co V \to V$ תקרא \textit{חיובית} או \textit{אי־שלילית} (וכו') אם $T = T^*$ וגם $\forall v \neq 0 \co \mut{Tv}{Tv} \ge\!/\!> 0$. 
	
	\theo{נניח ש־$A = A^* \in M_n(\F)$, אז התנאים הבאים שקולים (קיצור מוכר: TFAE, the following are equaivlent): 
		\begin{enumerate}
			\item $T_A$ חיובית/אי שלילית על $\F^n$. 
			\item לכל $T \co V \to V$ ובסיס א''נ $B$ כך ש־$A = [T]_B$, $T$ חיובית/אי שלילית. 
			\item קיימים $T \co V \to V$ חיובית/אי שלילית ו־$B$ בסיס, כך ש־$A = [T]_B$. 
			\item הע''ע של $A$ (יודעים ממשיים כי צמודה לעצמה) חיובים/אי שליליים. 
	\end{enumerate}}
	\begin{proof}
		מספיק לטעון זאת כדי להוכיח את השקילויות של 1, 2, 3: 
		\[ \mut{Tv}{v}_V = \mut{[Tv]_B}{[v]_B}_{\F^n} = \mut{A[v]_B}{[v]_B}_{\F^n} \]
		בשביל $1 \to 2$, ידוע שהאגף הימני גדול מ־$0$ מההנחה שהיא חיובית/אי שלילית על $\F^n$, ומכאן הראנו שהמיוצגת בכל בסיס חיובית כדרוש. בשביל $3 \to 1$, נפעיל טיעונים דומים מהאגף השמאלי במקום. הגרירה $2 \to 3$ ברורה. סה''כ הראינו את $1\lra2\lra 3$. 
		
		עתה נוכיח שקילות בין $1$ ל־$4$. 
		\begin{itemize}
			\item[$1 \to 4$] יהי $\lg \in \R$ ע''ע של $A$ (נוכל להניח ממשי כי $A$ צמודה לעצמה)
			\[ \mut{Av}{v} = \lg\norm{v}^2 > 0 \implies \lg > 0 \]
			\item[$4 \to 1$] יהי $B = (v_1 \dots v_n)$ בסיס א''נ של ו''ע, ויהי $V \ni v = \sum \ag_i v_i$. נקבל: 
			\[ \mut{T_Av}{v} = \mut{Av}{v} = \mut{A\sumnio \ag_i v_i}{\sumnio \ag_i v_i} = \sum \lg_i |\ag_i|^2 > 0 \]
		\end{itemize}
	\end{proof}
	
	\textbf{תזכורת: }מעל $\R$, הוכחנו שלכל תבנית סימטרית, יש ייצוג יחיד באמצעות מטריצה אלכסונית עם $-1, 1, 0$ על האלכסון. 
	\noti{הסיגנטורה של $f$ תסומן ע''י $\sg_-(f), \sg_0(f), \sg_+(f)$ כמספר האפסים, האחדים וה־$-1$ ב־$f$. }
	
	\textbf{המשך תזכורת: }כל מטריצה סימטרית חופפת למטריצה יחידה מהצורה לעיל. 
	
	\theo{נניח ש־$A$ מייצגת את התבנית הסימטרית $f$ (עולם הדיון מעל $\R$). אז, אם הסיגנטורה $\sg_+ = \#(\lg \mid \lg > 0)$ עבור $\lg$ ע''ע עם חזרות. באופן דומה $\sg_- = \#(\lg \mid \lg < 0)$ וכו'. }
	
	\begin{proof}
		משום ש־$A$ מייצגת סימטרית אז $A$ סימטרית. לפי המשפט הספקטרלי קיימת $P$ אורתוגונלית ו־$\Lg$ אלכסונית כך ש־$A = P\op\Lg P = P^T\Lg P$. $A$ דומה לאלכסונית וחופפת אליה. בעזרת נרמול המטריצה $\Lg$ האלכסונית (ניתן לבצע תהליך נרמול באמצעות פעולות שקולות תחת חפיפה), היא חופפת למטריצה מהצורה $\diag(1 \dots 1, -1 \ldots -1, 0 \ldots 0)$ כאשר הסימן נקבע לפני הנרמול. 
	\end{proof}
	
	\theo{תהי $T \co V \to V$ צמודה לעצמה ואי שלילית $\mut{Tv}{v} \ge 0$, אז קיימת ויחידה $R \co V \to V$ אי־שלילית צמודה לעצמה כך ש־$R^2 = T$. }
	
	\begin{proof}
		\textbf{קיום. }
		מהמשפט הספקטרלי קיים בסיס א''נ של ו''ע להעתקה אי־שלילית כל הע''ע הם אי־שליליים. 
		\[ [T]^{B}_B = \diag(\lg_1 \dots \lg_n) \quad [R]_B^B = \diag(\sqrt \lg_1 \dots \sqrt \lg_n) \]
		(ראינו זאת בתרגול). עוד נבחין ש־$R$ צמודה לעצמה כי ע''ע ממשיים. 
		
		\textbf{יחידות. }נבחין שכל ו''ע של $T$ הוא ו''ע של $R$: יהי $i \in [n]$, ו־$B = (e_1 \dots e_n)$ בסיס מלכסן, ואז עבור $R$ צמודה לעצמה כלשהי מתקיים: אז ו''ע של $R$ עם ע''ע $\sqrt \lg$ הוא ו''ע של $T$ עם ע''ע $\lg$ כי: 
		\[ \lg v = R^2v = Tv \implies Rv = \sqrt \lg \]
		הגרירה נכונה מאי־שליליות $R$ שהמשפט מניח עליה יחידות. כלומר הערכים העצמיים של $R$ כלשהי (לא בהכרח זו שברחנו בהוכחת הקיום) נקבעים ביחידות מע''ע של $T$. בסיס של ו''ע של $T$ הוא בסיס ו''ע של $R$, סה''כ ראינו איך $R$ פועלת על בסיס ו''ע כלשהו של $T$ מה שקובע ביחידות את $R$. 
	\end{proof}
	
	\noti{את ה־$R$ לעיל נסמן $\sqrt{T} := R$. }
	
	\begin{Collary}[פירוק שולסקי]
		לכל $A$ צמודה לעצמה ואי־שלילית חיובית קיים פירוק של מטריצה $R$ משולשית עליונה כך ש־$A=RR^*$. בעזרת פירוק פולארי שנראה במשפט הבא, נוכל להראות ש־$R$ משולשית עליונה. 
	\end{Collary}
	
	\subsubsection{ניסוח פירוק פולארי בעבור העתקות}
	\begin{Theorem}[פירוק הפולארי]
		תהי $T \co V \to V$ הפיכה, אז קיימות $R \to V \to V$ חיובית וצמודה לעצמה ו־$U \co V \to V$ אוניטרית כך ש־$T = RU$. 
	\end{Theorem}
	\rmark{לא הנחנו $T$ צמודה לעצמה. הפירוק נכון להעתקה הפיכה כללית. }
	\begin{proof}
		נגדיר $S = TT^*$. נבחין ש־$S$ צמודה לעצמה וחיובית: 
		\[ \forall V \ni v \neq 0 \co \mut{Sv}{v} = \mut{TT^* v}{v} = \mut{T^* v}{T^*v} = \norm{T^*v} > 0 \]
		האי־שוויון האחרון נכון כי $\ker T = \{0\}$, ממשפט קודם $\ker T^* = \ker T = \{0\}$, ו־$v \neq 0$. יצא שזה חיובי ולכן בפרט ממשי, כלומר היא צמודה לעצמה וחיובית. 
		
		קיימת ויחידה $R \co V \to V$ צמודה וחיובית כך ש־$S = R^2$. כל ערכיה העצמיים של $R = \sqrt S$ אינם $0$, ולכן היא הפיכה (ראינו בהוכחה של קיומה שהיא לכסינה יחדיו עם $S$). 
		
		נגדיר $U = R\op T$. נותר להראות ש־$U$ אוניטרית. 
		\[ U^*U = (R\op T)^*(R\op T) = T^*\underbrace{(R\op)^*}_{R\op}R\op T = T^*(R\op)^2 T = T^*S\op T = T^*(TT^*)\op T = I \]
		כדרוש. הטענה $(R\op)^* = R\op$ נכונה משום ש־$R$ צמודה לעצמה. 
	\end{proof}
	
	\begin{Remark}[לגבי יחידות]
		אם $T$ אינה הפיכה, מקבלי םש־$R$ יחידה אבל $U$ אינה. בשביל לא הפיכות נצטרך להצטמצם לבסיס של התמונה ועליו לפרק כמתואר לעיל. במקרה של הפיכות אז $T = RU = R \tl U$ ואז נקבל $R$ הפיכה כלומר $U = \tl U$ וגם $U$ הפיכה. 
	\end{Remark}
	
	עתה נראה ש־$R$ נקבעת ביחידות (בניגוד ליחידות $U$ – יחידות $R$ נכונה גם בעבור פירוק פולארי של העתקה שאיננה הפיכה): 
	\begin{proof}
		\[ TT^* = RU(RU)^* = RUU^*R^* = R^2 \]
		כלומר $R$ היא בכל פירוק שורש, והראינו קודם את יחידות השורש. 
	\end{proof}
	\rmark{קיים גם פירוק כנ''ל מהצורה $T = UR$. }
	\begin{proof}
		באותו האופן שפירקנו את $T$, נוכל לפרק את $T^* = \tl R \tl U$ פירוק פולארי. נפעיל $^*$ על שני האגפים ונקבל: 
		\[ T^* =\tl R \tl U \implies T = (T^*)^* = \tl U^*\tl R^* = \tl U\op \tl R \]
		נסמן $\tl R =: R, \ \tl U\op =: U$ וסה''כ $T = UR$ כדרוש. 
	\end{proof}
	\lem{עבור $T \co V \to V$ אז ל־$TT^*, T^*T$
		נגדיר $S = TT^*$. נבחין ש־$S$ צמודה לעצמה וחיובית: 
		\[ \forall V \ni v \neq 0 \co \mut{Sv}{v} = \mut{TT^* v}{v} = \mut{T^* v}{T^*v} = \norm{T^*v} > 0 \] יש אותם הערכים העצמיים. }
	\begin{proof}ניעזר בפירוק הפולארי: 
		\begin{align*}
			TT^* &= RUU^*R^* \\
			&= R^2 \\
			TT^* &= U\op R^2U
		\end{align*}
		סה''כ $TT^*, T^*T$ הן העתקות דומות ולכן יש להן את אותם הערכים העצמיים. 
	\end{proof}
	
	\rmark{אז איך זה קשור לפולארי? $R$ האי־שלילית היא ``הגודל'', בעוד $U$ האוניטרית לא משנה גודל – היא ה''זווית''. }
	
	
	\subsubsection{ניסוח פירוק פולארי בעבור מטריצות}
	\theo{(פירוק פולארי עבור מטריצות) תהי $A \in M_n(\F)$ הפיכה, אז קיימות $U, R \in M_n(\F)$ כאשר $U$ א''נ ו־$R$ חיובית צמודה לעצמה כך ש־$A = UR$. }\begin{proof}
		נסתכל על $A^*A$. היא חיובית וצמודה לעצמה (בדומה לעיל). אז $A^*A = P\op D P$, כאשר $D$ אלגסונית חיובית. כאשר $R = P\op \sqrt D P, \ R^2 = AA^*$. היא קיימת ויחידה מאותה הוכחה בדיוק להעתקות. 
	\end{proof}
	
	\subsection{פירוק SVD}
	\rmark{SVD הינו קיצור של Singular Value Decomposition. }
	\theo{(פירוק לערכים סינגולריים למטריצה – SVD) לכל מטריצה $A \in M_n(\F)$ קיימות מטריצות אוניטריות $U, V$ ומטריצה אלכסונית עם ערכים אי־שלילייים כך ש־$A = UDV$. }
	\begin{proof}
		ידוע שניתן לכתוב $A = \tl UR$ פירוק פולארי. משום ש־$R$ צמודה לעצמה, ניתן לפרקה ספקטרלית ל־$V$ אוניטרית ו־$D$ אלכסונית אי־שלילית (כי $R$ אי־שלילית) כך ש־$R = V\op DV$. סה''כ: 
		\[ A = \underbrace{\tl UV\op}_{=: U} DV = UDV \quad \top \]
		כי $\tl UV\op$ מכפלה של אוניטריות ולכן $U$ אוניטרית כנדרש. 
	\end{proof}
	
	\rmark{\begin{align*}
			AA^* &= (UDV) V^*D^*U^* = UD^2U\op \\
			A^*A &= V\op D^2 V
	\end{align*}}
	
	
	\defi{הערכים העצמיים האי־שליליים של $A^*A$ נקראים \textit{הערכים הסינגולריים} והם נקבעים ביחידות ע''י $A$. }
	הערכים הסינגולרים הם גם הע''ע של $R^2$ הפירוק בפולארי וכן הע''ע של $D^2$ בפירוק SVD. 
	
	\rmark{פירוק SVD יחיד למטריצה הפיכה. }
	\rmark{במסדרת הקורס הזה, ראינו פירוק SVD של מטריצה ריבועית בלבד. חלק מהחוזקה של פירוק SVD נובע מקיומו למטריצות שאינן בהכרח ריבועיות, דהיינו, לכל מטריצה סופית. יש לו מגוון רחב של שימושים במדעי המחשב. }
	
	\rmark{כאן בערך נגמר החומר בקורס. עם זאת, חסר מאוד הסברים על השימושים של פירוק SVD ועל מסקנות די חשובות ממנו. לכן הוספתי קצת הרחבות עליו, תוך הסתמכות על הספר ``Linear Algebra Done Right''. }
	
	\npage
	\section[אלגוריתמים נפוצים]{\en{Common Algorithms}}
	בניגוד לפרק הבא, הפרק הזה לא מעניין בכלל. הוא מסכם בקצרה אלגוריתמים מועילים שרואים בתרגולים וכדאי לזכור (\textbf{אין} כאן סיכום מלא של התרגולים). 
	\subsection{אלגוריתמים מרכזיים}
	\subsubsection{לכסון}
	\textbf{שלבים: }בהינתן $A \in M_n(\F)$ מטריצה. 
	\begin{itemize}
		\item נחשב את $f_A$
		\item נמצא את שורשי $f_A$. אם אנו מתקשים למצוא את שורשי הפולינום, נמצא את $\fRed$. 
		\item לכל ע''ע $\lg_i$, נמצא בסיס למרחב העצמי באמצעות חישוב $\nc(\lg_i I - A)$. איברי הבסיס יהיו הו''עים בעבור הע''ע $\lg_i$. 
		\item סה''כ $\diag(\lg_1 \dots \lg_n)$ המטריצה האלכסונית המתקבלת ע''י מטריצת מעבר הבסיס הנתונה ע''י הו''עים מהשלב הקודם. 
	\end{itemize}
	\subsubsection{ג'ירדון}
	\textbf{ג'רדון מטריצה כללית}
	תהי $A \in M_n(\F)$ מטריצה שהפולינום האופייני שלה $f_A(x) = \prod_{j = 1}^{m}(x - \lg_j)^{r_j}$ (בה''כ מתקיים מעל הרחבה לשדה סגור אלגברית). לכל $j \in [m]$ נבצע את הפעולות הבאות: 
	\begin{itemize}
		\item נמצא את הפולינום $f_A(x)$ האופייני ונפרק אותו לכדי גורמים לינאריים. 
		\item נחשב את $V_{\lg_j}^{(i)} := \nc((A - \lg_j)^{\ell_j})$ עד שנקבל $\dim\cl{V_{\lg_j}^{(\ml_j)}} = m_i$ (המרחב העצמי המוכלל). 
		
		\textit{הערה: }אפשר באופן חלופי לחשב את הפולינום המינימלי, שכן ראינו ש־$m_i$ הריבוי של $\lg_i$ ב־$m_T(x)$. 
		\item נחזור על האלגו' למציאת צורת ג'ורדן למטריצה נילפוטנטית: 
		\begin{itemize}
			\item נגדיר $B_{\lg_j} = \varnothing$
			\item לכל $i \in [\ell_j]$ נבצע: 
			\begin{itemize}
				\item נמצא בסיס כלשהו $C_{\lg_j}^{(i)-}$ של $V_{\lg_j}^{(i - 1)}$. 
				\item נוסיף ל־$C_{\lg_j}^{(i)-}$ את $B \cap (V_{\lg_j}^{(i)} \setminus V_{\lg_j}^{(i - 1)})$
				\item נשלים את $C_{\lg_j}^{(i)-}$ לבסיס של $V_{\lg_j}^{(i)}$. נסמן ב־$C_{\lg_j}^{(i)+}$. 
				\item נוסיף ל־$B_{\lg_j}$ את $\ccb{(A - \lg_jI)^{k}v \mid 0 \le k < i, \ v \in C_{\lg_j}^{(i)+}}$. 
			\end{itemize}
			\item נגדיר $B = \bigcup_{j = 1}^{m} B_{\lg_j}$ הבסיס המג'רדן. 
		\end{itemize}
	\end{itemize}
	
	\textbf{מציאת $J_n(\lg)^{m}$: }ידוע $J_n(\lg) = \lg I_n + J_n(0)$ ולכן מהיות $\lg I, J_n(0)$ מתחלפות נקבל מנוסחת הבינום של ניוטון: 
	\[ (J_n(\lg)^{m})_{ij} = \begin{cases}
		0 & j > i \\ 0 & j < i - m \\ \binom{m}{i - j}\lg^{m - (i - g)} & \other
	\end{cases} \]
	דהיינו: 
	\[ J_n(\lg)^{m} = \pms{
		\binom{m}{0}\lg^{m} \\ 
		\binom{m}{1}\lg^{m - 1} & \binom{m}{0}\lg^{m} \\
		\vdots & \ddots & \ddots \\
		\binom{m}{m}\lg^{0} & \binom{m}{m - 1}\lg^{1} &\ddots& \ddots \\
		0 & \binom{m}{m}\lg^{0} &\cdots&\ddots& \binom{m}{0}\lg^{m} \\
		\vdots & 0 & \binom{m}{m}\lg^{0} &\cdots& \binom{m}{m - 1}\lg^{1} & \binom{m}{0}\lg^{m} \\
		0 & \vdots & \ddots & \ddots
	} \]
	\subsubsection{אלגוריתם גראם־שמידט}
	נרצה למצוא בסיס אורתונורמלי/אורתוגונלי לממ''פ כלשהו. יהי בסיס $B = v_1 \dots v_n$ של $V$. 
	\begin{itemize}
		\item \textbf{למציאת בסיס אורתוגונלי}: נגדיר לכל $i \in [n]$
		\[ \tl v_i = v_i - \sum_{j = 1}^{i - 1}\frac{\mut{v_i}{\tl v_i}}{\mut{\tl v_i}{\tl v_j}} \cdot \tl v_j \]
		ואז $(\tl v_1 \dots \tl v_n)$ בסיס אורתוגונלי (הבחנה: התהליך רקורסיבי, נתחיל מ־$i = 1$ ונסיים ב־$i = n$). במידת הצורך נוכל לנרמל בסוף ע''י הגדרת: 
		\[ \bar v_i = \frac{\tl v_i}{\norm{\tl v_i}} \]
		ואז $(\bar v_1 \dots \bar v_n)$ אורתונורמלי מסיבות ברורות. 
		\item \textbf{מציאת בסיס אורתונורמלי}: (פחות יציב נומרית מאשר למצוא אורתונורמלי ואז לנרמל, אך יותר קל חישובית) נגדיר לכל $i \in [n]$: 
		\[ \bar v_i = \frac{1}{\norm{v_i}} \cl{v_i - \sum_{j = 1}^{i - 1}\mut{v_i}{\bar v_i} \cdot \bar v_i} \]
		בצורה זו נוכל לנרמל תוך כדי התהליך. 
	\end{itemize}
	\subsection{מספר אלגוריתמים נוספים}
	\begin{itemize}
		\item \textbf{אלגוריתם אוקלידס לתחום ראשי} (בפרט בעבור פולינומים): 
		\item \textbf{נרמול וקטור: }נגדיר $v = \frac{v}{\norm{v}}$ הוא $v$ מנורמל. 
		\item \textbf{בדיקת $\bm{T}$־איווריאנטיות: }בהינתן $B$ בסיס של $W \subseteq V$ נחשב את $T(B)$ ונבדוק האם $T(B) \subseteq W$ ע''י מעבר על כל איבר בסיס ודירוג. 
		\item \textbf{חישוב $\bm{A\op, A^{n + c}}$ באמצעות משפט קיילי־המילטון: }ידוע $f_A(A) = 0$, ואם נשאר גורם חופשי $\ag_nA^{n} + \cdot \ag_0 A^{0} = 0$ אז נוכל להעביר אגפים ולקבל: $I = A^0 = A\cl{\frac{\sum_{k = 1}^{n}\ag_kA^{k - 1}}{\ag_0}}$ ולכן $A\op = \frac{1}{\ag_0}\cl{\sumnko \ag_k A^{k - 1}}$. כדי לחשב את $A^{n + c}$ תחילה נחשב את $A^{n}$ באמצעות העברת אגפים וקבלת $A^{n} = -\sum_{k = 0}^{n - 1} \ag_kA^{k}$ (כי $\ag_n = 1$ כי הפולינום מתוקן). עתה, נכפול ב־$A$ בדיוק $c$ פעמים, ומשום שידוע $A^{n}$, בכל חלוקה שבא נקבל $A^{n + 1}$ נוכל להוציא גורם משותף ולקבל $A(A^{n}) = \sumnko \bg_kA^{k}$ ביטוי שהכפל הגבוהה ביותר בו תמיד $A^{n - 1}$. סה''כ נוכל לבטא את $A^{n + c}$ כקומבינציה לינארית של $I \cdots A^{n - 1}$, שעבור מספרי $n$ קטנים קל לחשב. 
		\item \textbf{ייצוג בבסיס אורתוגונלי: }לכל $u \in V$ בהינתן $(v_1 \dots v_n)$ בסיס אורתוגונלי, מתקיים $u = \sumnio \frac{\mut{u}{v_i}}{\norm{v_i}^{2}} \cdot v_i$ (אין צורך לחלק בנורמה בעבור אורתונורמלי). 
		\item \textbf{מציאת היטל אורתוגונלי: }בהינתן $(u_1 \dots u_n)$ בסיס אורתוגונלי של $U$ תמ''ו, אז $p_U(v) = \sum_{i = 1}^{k}\frac{\mut{v}{u_i}}{\norm{u_i}^2}u_i$ (גם כאן אין צורך לחלק בנורמה בעבור בסיס אורתונורמלי). 
		\item \textbf{מציאת ללכסון אוניטרי/אורתוגונלי} (אם קיים ממשפט הפירוק הספקטרלי)\textbf{: }
		\begin{itemize}
			\item נמצא את הע''ע של ההעתקה. 
			\item לכל ע''ע, נמצא בסיס עצמי של ו''ע ואז נבצע עליו בראם־שמידט כדי לקבל וקטורים אורתוגונליים/אורתונורמליים. 
			\item נשרשר את הבסיסים לקבלת בסיס אורתוגונלי/אורתונורמלי מלכסן. 
		\end{itemize}
	\end{itemize}
	
	
	{\dotfill \\ \vfil {\begin{center}
				{\Large \textbf{\textit{סוף הקורס $\sim$ 2025B}}\vspace{3pt} \\
					\normalsize הסיכום לא נגמר – יש הרחבה על דואלים בעמוד הבא
					\\
					\scriptsize \textit{קומפל ב־}\en{\LaTeX}\,\textit{ ונוצר באמצעות תוכנה חופשית בלבד}}
		\end{center}} \vfil	}
	\pagebreak
	
	\section[מרחבים דואלים]{\en{Dual Spaces}}
	\subsection{הגדרות בסיסיות}
	
	\defi{בהינתן $V$ מ''ו מעל $\F$, נגדיר $V^* = \hom(V, F)$. }
	\textbf{הבנה. }אם $\dim V = n$ אז $\dim V^* = n$. לכן $V \cong V^*$. לא נכון במקרה הסוף ממדי. 
	
	\lem{יהי $B = (v_i)_{i = 1}^{n}$ בסיס ל־$V$. אז \hfill $\forall i \in [n] \co \exists \psi_i \in V^* \co \forall j \in [n] \co \psi_i(v_j) = \dg_{ij}$}
	
	\theo{יהי $V$ נ''ס ו־$B = (v_i)_{i = 1}^{n}$ אז קיים ויחיד בסיס $B^* = (\psi_i)_{i = 1}^{n}$ המקיים $\forall i, j \in [n] \co \psi_i(v_j) = \dg_{ij}$. }\begin{proof}
		נבחין שהבדרנו העתקה לינארית $\phi \co B \to V^*$ והיא מגדירה ביחידות $\psi$ לינארית   $\psi \co V \to V^*$ המקיימת את הנרש. ברור שהבנייה של $\phi_i$ קיימת ויחידה כי היא מוגדר לפי מה קורה לבסיס. נותר להוכיח שזה אכן בסיס. יהיו $\ag_1 \dots \ag_n \in \F$ כך ש־$\sum\ag_i \psi_i = 0$. (האפס הזה הוא פונקציונל האפס). יהי $j \in [n]$. אז $0(v_j) = 0 = \cl{\sumni \ag_i \psi_i}v_j = \sumni \ag_i \psi_i(v_j) = \sum\ag_i \dg_{ij} = \ag_j$ וסה''כ $\ag_j = 0$. 
	\end{proof}
	
	נבחין שאפשר להגדיר: 
	\defi{$V^{**} = \hom(V^*, \F)$}
	ואכן $\dim V < \inf$ אז: 
	\[ V \cong V^* \cong V^{**} \]
	במקרה הזה, בניגוד לאיזו' הקודם, יש איזו' ``טבעי'' (קאנוני), כלומר לא תלוי באף בסיס. 
	
	\theo{קיים איזומורפיזם קאנוני בין $V$ ל־$V^{**}$. }
	\begin{proof}נגדיר את האיזו' הבא: 
		\[ \psi \co V \to V^{**} \quad \psi(v) =: \bar v \quad \forall \psi \in V^* \co \bar v(\psi) = \psi(v) \]
		נוכיח שהוא איזו': 
		\begin{itemize}
			\item \textbf{ט''ל: }יהיו $\ag, \bg \in \F, \ v, u \in V$. אז: 
			\[ \psi(\ag v + \bg u) = \overline{\ag v + \bg u} \seq \ag \bar v + \bg \bar u \]
			נוכיח זאת: 
			\[ \overline{\ag v + \bg u}(\phi) = \phi(\ag v + \bg u) = \ag \phi(v) + \bg \phi(u) = \ag v(\phi) + \bg \bar u(\phi) = (\ag \bar v + \bg \bar u)(\phi) = (\ag \psi(v) + \bg \psi(u))(\phi) \]
			\item \textbf{חח''ע: }יהי $v \in \ker \psi$. רוצים להראות $v = 0$. 
			\[ \forall \phi \in V^* \co \bar (\phi) = 0 \implies \forall \phi \in V^* \co \phi(v) = 0 \]
			אם $v$ אינו וקטור האפס, נשלימו לבסיס $V = (v_i)_{i = 1}^{n}$ ואם $\phi_1 \dots \phi_n$ בסיס הדואלי אז $\phi_1(v) = 1$ אבל אז $0 = \bar v(\phi_1) = 1$ וסתירה. 
			\item \textbf{על: }משוויון ממדים $\dim V^{**} = \dim V$. 
		\end{itemize}
	\end{proof}
	כלומר, הפונקציונלים בדואלי השני הם למעשה פונקציונלים שלוקחים איזשהו פונקציונל בדואלי הראשון ומציבים בו וקטור קבוע. 
	
	\subsection{איזומורפיות למרחבי מכפלה פנימית}
	\subsubsection{העתקה צמודה (דואלית)}
	\noti{לכל $v \in V$ ו־$\phi \in V^*$ נסמן $\phi(v) = (\phi, v)$}
	\rmark{סימון זה הגיוני משום שהכנסת וקטור לפונרציונל דואלי איזומורפי למכפלה פנימית. }
	\theo{יהיו $V, W$ מ''וים נוצרים סופית מעל $\F$, $T \co V \to W$. אז קיימת ויחידה $T^* \co W^* \to V^*$ כך ש־$(\psi, T(v)) = (T^*(\psi), v)$. }
	אם לצייר דיאגרמה: 
	\[ V \overset{T}{\to} W \cong W^* \overset{T^*}{\to} V^* \cong V \]
	(תנסו לצייר את זה בריבוע, ש־$V, W$ למעלה ו־$V^*, W^*$ למטה, כדי להבין ויזולאית למה זה הופך את החצים)
	
	ברמה המטא־מתמטית, בתורת הקטגוריות, יש דבר שנקרא פנקטור – דרך לזהות בין אובייקטים שונים במתמטיקה. מה שהוא עושה, לדוגמה, זה להעביר את $\hom(V, W)$ – מרחבים וקטרים סוף ממדיים – למרחב המטריצות. הדבר הזה נקרא פנקטור קו־וראיינטי. במקרה לעיל, זהו פנקטור קונטרא־ווריאנטי – שימוש ב־$T^*$ הופך את החצים. (הרחבה של המרצה)
	
	אז אפשר להגדיר פנקטור אבל במקום זה נעשה את זה בשפה שאנחנו מכירים – לינארית 1א. בהינתן $\psi \in W^*$, נרצה למצוא $T^*(\psi) \in V^*$. נגדיר: 
	\[ T^*(\psi) = \psi \circ T \]
	ברור מדוע $T^* \co W^* \to V^*$. בעצם, זהו איזומורפיזם (``בשפת הפנקטורים'') קאנוני. עוד קודם לכן ידענו (בגלל ממדים) שהם איזומורפים, אך לא מצאנו את האיזומורפיזם ולא ראינו שהוא קאנוני. 
	\[ \tau \co \hom(V, W) \to \hom(W^*, V^*) \quad \tau(T) = T^*  \]
	היא איזומורפיזם. 
	
	(הערה: תודה למרצה שנענה לבקשתי ולא השתמש ב־\en{\slash phi} אחרי שעשיתי \en{\slash renewcommand\slash phi\{\slash varphi\}})
	\begin{proof}[הוכחת לינאריות]
		יהיו $T, S \in \hom(V, W)$, $\ag \in \F$. אז: 
		\[ \tau(\ag T + S) = (T + \ag S)^* \]
		יהי $\psi \in W^*$, אז: 
		\[ (T + \ag S)^*(\psi) = \psi \circ (T + \ag S) \]
		יש למעלה פונקציונל ב־$V^*$. ננסה להבין מה הוא עושה על $V$. יהי $v \in V$: 
		\begin{align*}
			[\psi (T + \ag S)](v) &= \psi((T + \ag S)v) \\ 
			&= \psi(T)v + \ag \psi(S)(v) = (\psi\circ T + \ag \psi\circ S)(v) \\
			&= ((T^* + \ag S^*)\circ (\psi))v \\
			&= (\tau(T) + \ag \tau(S))(\psi)(v)
		\end{align*}
		סה''כ קיבלנו לינאריות: 
		\[ \tau(T + \ag S) = \tau(T) + \ag \tau(S) \]
	\end{proof}
	נוכל להוכיח זאת יותר בפשטות עם הנוטציה של ``המכפלה הפנימית'' שהגדרנו לעיל, $(\phi, v)$. 
	עתה נוכיח ש־$\tau$ לא רק לינארית, אלא מוגדרת היטב. 
	\begin{proof}\,
		\begin{itemize}
			\item \textbf{חח''ע: }תהי $T \in \ker \tau$, אז $\tau(T) = T^* = 0_{\hom(W^*, V^*)}$. נרצה להראות ש־$T$ העתקה האפס. נניח בשלילה ש־$T \neq 0$. אז קיים $v' \in V$ כך ש־$T(v') \neq 0$. נשלימו לבסיס – $(T(v) = w_1, w_2 \dots w_n)$ בסיס ל־$W$. יהי $(\psi_1 \dots \psi_n)$ הבסיס הדואלי. אז:
			\[ \tau(T)(\psi_1) = T^*(\psi_1) = \psi_1  \]
			אז: 
			\[ 0 = \tau(T)(\psi_1)(v') = T^*(\psi_1)(v') = \psi_1 \circ T(v') = \psi_1(w_1) = 1 \]
			סתירה. לכן $\ker \tau = \{0\}$ ולכן $\tau$ חח''ע. 
			\item \textbf{על: }גם כאן משוויון ממדים
		\end{itemize}
	\end{proof}
	
	\textbf{שאלה ממבחן שבן עשה. }(``את השאלה הזו לא פתרתי בזמן המבחן, ואני די מתבייש כי אפשר לפתור אותה באמצעות כלים הרבה יותר פשוטים'' ``חה חה'' ``לא חח''ע זה חד־חד ערכי'') יהיו $V, W$ מ''ו מעל $\F$ ו־$(w_1 \dots w_n)$ בסיס של $W$. תהי $T \co V \to V$. הוכיחו שקיימים $\phi_1 \dots \phi_n \in V^*$ כך שלכל $v \in V$ מתקיים: 
	\[ T(v) = \sumni \phi_i(v) w_i \]
	\textbf{שימו לב: }בניגוד למה שבן עשה במבחן, $V$ לא בהכרח נוצר סופית. 
	\begin{proof}[הוכחת ראש בקיר.]
		לכל $v \in V$ קיימים ויחידים $\ag_1 \dots \ag_n$ כך ש־$T(v) = \sumni \ag_i w_i$. נגדיר $\forall i \in [n] \co \phi_i(v) = \ag_i$. זה לינארי. 
	\end{proof}
	
	\begin{proof}[הוכחה ``מתוחכמת''.]
		``אני אהבתי את ההוכחה שלי'': נתבונן בבסיס הדואלי $B^* = (\psi_1 \dots \psi_n)$ שמקיים את הדלתא של קרונקר והכל. 
		נגדיר $T^*(\psi_i) =: \phi_i$. יהי $v \in V$. קיימים ויחידים $\ag_1 \dots \ag_n$ כך ש־$T(v) = \sumni \ag_i w_i$. אז: 
		\[ \sumni \phi_i(v)w_i = \sum T^*(\psi_i)(v) w_i \]
		צ.ל. $\ag_i = T^*(\psi_i)(v)$. אך נבחין שהגדרנו: 
		\[ T^*(\psi_i)(v) = \psi_i(T(v)) = \psi_i\cl{\sumni \ag_j w_j} = \ag_j \]
		
	\end{proof}
	
	``הפכת למרצה במתמטיקה כדי להתנקם באחותך?'' ``כן.''
	
	\subsubsection{המאפס הדואלי ומרחב אורתוגונלי}
	\defi{יהי $V$ מ''ו נוצר סופית. יהי $S \subseteq V$ קבוצה. נגדיר $\{\phi \in V^* \mid \forall v \in S \co \phi(v) = 0\} =: S^0 \subseteq V^*$. }
	
	\textbf{דוגמאות. } \hfil $\{0\}^0 = V^*, \ V^0 = \{0\}$
	
	\theo{\,
		\begin{enumerate}
			\item $S^0$ תמ''ו של $V^*$. 
			\item \hfil $(\Sp S)^0 = S^0$
			\item \hfil $S \subseteq S' \implies S'^0 \subseteq S^0$
	\end{enumerate}}
	
	\theo{יהי $V$ נ''ס, $U \subseteq V$ תמ''ו. אז $\dim U + \dim U^0 = n$}
	באופן דומה אפשר להמשיך ולעשות: 
	\[ \dim U^0 + \dim U^{**} = n \]
	וכן: 
	\[ U \cong U^{**} \]
	איזומורפיזם קאנוני. זאת כי:
	\[ \forall \phi \in U^0 \, \forall u \in U \co \phi(u) = 0 \]
	ומי אלו הוקטורים שיאפסו את $\phi$ שמאפס את $u$? הוקטורים ב־$U$ עד לכדי האיזומורפיזם הקאנוני מ־$U$ ל־$U^{**}$. 
	
	נבחין ש־: 
	\[ [T^*]^{\bc^*}_{\ac^*} = ([T]^{\ac}_{\bc})^T \]
	כאשר $\ac$ בסיס ל־$V$, $\ac^*$ ל־$V^*$, $\bc$ ל־$W$, $\bc^*$ ל־$W^*$. 
	
	``כוס אמא של קושי'' – בן על זה שקושי גילה משפט כלשהו לפניו. 
	
	\ndoc
	
	
	
	
	
\end{document}