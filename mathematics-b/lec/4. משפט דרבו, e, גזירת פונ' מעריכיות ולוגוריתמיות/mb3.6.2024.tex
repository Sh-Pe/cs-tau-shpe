\documentclass[]{article}

% Math packages
\usepackage[usenames]{color}
\usepackage{forest}
\usepackage{ifxetex,ifluatex,amsmath,amssymb,mathrsfs,amsthm,witharrows,mathtools}
\WithArrowsOptions{displaystyle}
\renewcommand{\qedsymbol}{$\blacksquare$} % end proofs with \blacksquare. Overwrites the defualts. 
\usepackage{cancel,bm}

% tikz
\usepackage{tikz,pgfplots}
\newcommand\sqw{1}
\newcommand\squ[4][1]{\fill[#4] (#2*\sqw,#3*\sqw) rectangle +(#1*\sqw,#1*\sqw);}


% code 
\usepackage{listings}
\usepackage{xcolor}

\definecolor{codegreen}{rgb}{0,0.35,0}
\definecolor{codegray}{rgb}{0.5,0.5,0.5}
\definecolor{codenumber}{rgb}{0.1,0.3,0.5}
\definecolor{deepblue}{rgb}{0,0,0.5}
\definecolor{deepred}{rgb}{0.5,0.03,0.02}

\lstdefinestyle{pythonstylesheet}{
	language=Python,
	morekeywords={}
	emphstyle=\color{deepred},
	backgroundcolor=\color{white},   
	commentstyle=\color{codegreen}\itshape,
	keywordstyle=\color{deepblue}\bfseries\itshape,
	numberstyle=\tiny\color{codenumber},
	basicstyle=\ttfamily\footnotesize,
	breakatwhitespace=false, 
	breaklines=true, 
	captionpos=b, 
	keepspaces=true, 
	numbers=left, 
	numbersep=5pt, 
	showspaces=false,                
	showstringspaces=false,
	showtabs=false, 
	tabsize=2, 
	morekeywords={object,type,isinstance,copy,deepcopy,zip,enumerate,reversed,list,set,len,dict,tuple,range,xrange,append,execfile,real,imag,reduce,str,repr},              % Add keywords here
	keywordstyle=\color{deepblue},
	emph={__init__,__add__,__mul__,__div__,__sub__,__call__,__getitem__,__setitem__,__eq__,__ne__,__nonzero__,__rmul__,__radd__,__repr__,__str__,__get__,__truediv__,__pow__,__name__,__future__,__all__,as,assert,nonlocal,with,yield,self,True,False,None},          % Custom highlighting
	emphstyle=\color{deepred},
	stringstyle=\color{deepgreen},
	showstringspaces=false
}
\newcommand\pythonstyle{\lstset{pythonstylesheet}}
\newcommand\pyl[1]     {{\pythonstyle\lstinline!#1!}}
\lstset{style=pythonstylesheet}


% Deisgn
\usepackage[labelfont=bf]{caption}
\usepackage[margin=0.6in]{geometry}
\usepackage{multicol}
\usepackage[skip=4pt, indent=0pt]{parskip}
\usepackage[normalem]{ulem}
\forestset{default}
\renewcommand\labelitemi{$\bullet$}
\usepackage{titlesec}


% Hebrew initialzing
\usepackage{polyglossia}
\setmainlanguage{hebrew}
\setotherlanguage{english}
\newfontfamily\hebrewfont[Script=Hebrew, Ligatures=TeX]{David CLM}
\usepackage[shortlabels]{enumitem}
\newlist{hebenum}{enumerate}{1}
\setlist[hebenum,1]{
	labelindent=\parindent,
	label={{\hebrewfont{\protect\hebrewnumeral{\value{hebenumi}}}}.}
}

% Language Shortcuts
\newcommand\en[1] {\selectlanguage{english}#1\selectlanguage{hebrew}}
\newcommand\sen   {\selectlanguage{english}}
\newcommand\she   {\selectlanguage{hebrew}}
\newcommand\del   {$ \!\! $}
\newcommand\ttt[1]{\en{\texttt{#1}}}

\newcommand\npage {\vfil {\hfil \textbf{\textit{המשך בעמוד הבא}}} \hfil \vfil}
\newcommand\ndoc  {\dotfill \\ \vfil \hfil \textbf{\textit{שחר פרץ, 2024}} \hfil \vfil}

\newcommand{\rn}[1]{
	\textup{\uppercase\expandafter{\romannumeral#1}}
}


%! ~~~ Math shortcuts ~~~

% Letters shortcuts
\newcommand\N     {\mathbb{N}}
\newcommand\Z     {\mathbb{Z}}
\newcommand\R     {\mathbb{R}}
\newcommand\Q     {\mathbb{Q}}
\newcommand\C     {\mathbb{C}}

\newcommand\ml    {\ell}
\newcommand\mj    {\jmath}
\newcommand\mi    {\imath}

\newcommand\powerset {\mathcal{P}}
\newcommand\ps    {\mathcal{P}}
\newcommand\pc    {\mathcal{P}}
\newcommand\ac    {\mathcal{A}}
\newcommand\bc    {\mathcal{B}}
\newcommand\cc    {\mathcal{C}}
\newcommand\dc    {\mathcal{D}}
\newcommand\ec    {\mathcal{E}}
\newcommand\fc    {\mathcal{F}}
\newcommand\nc    {\mathcal{N}}
\newcommand\sca   {\mathcal{S}} % \sc is already definded
\newcommand\rca   {\mathcal{R}} % \rc is already definded

\newcommand\Si    {\Sigma}

% Logic & sets shorcuts
\newcommand\siff  {\longleftrightarrow}
\newcommand\ssiff {\leftrightarrow}
\newcommand\so    {\longrightarrow}
\newcommand\sso   {\rightarrow}

\newcommand\epsi  {\epsilon}
\newcommand\vepsi {\varepsilon}
\newcommand\vphi  {\varphi}
\newcommand\Neven {\N_{\mathrm{even}}}
\newcommand\Nodd  {\N_{\mathrm{odd }}}
\newcommand\Zeven {\Z_{\mathrm{even}}}
\newcommand\Zodd  {\Z_{\mathrm{odd }}}
\newcommand\Np    {\N_+}

% Text Shortcuts
\newcommand\open  {\big(}
\newcommand\qopen {\quad\big(}
\newcommand\close {\big)}
\newcommand\also  {\text{, }}
\newcommand\defi  {\text{ definition}}
\newcommand\defis {\text{ definitions}}
\newcommand\given {\text{given }}
\newcommand\case  {\text{if }}
\newcommand\syx   {\text{ syntax}}
\newcommand\rle   {\text{ rule}}
\newcommand\other {\text{else}}
\newcommand\set   {\ell et \text{ }}
\newcommand\ans   {\mathit{Ans.}}

% Set theory shortcuts
\newcommand\ra    {\rangle}
\newcommand\la    {\langle}

\newcommand\oto   {\leftarrow}

\newcommand\QED   {\quad\quad\mathscr{Q.E.D.}\;\;\blacksquare}
\newcommand\QEF   {\quad\quad\mathscr{Q.E.F.}}
\newcommand\eQED  {\mathscr{Q.E.D.}\;\;\blacksquare}
\newcommand\eQEF  {\mathscr{Q.E.F.}}
\newcommand\jQED  {\mathscr{Q.E.D.}}

\newcommand\dom   {\text{dom}}
\newcommand\Img   {\text{Im}}
\newcommand\range {\text{range}}

\newcommand\trio  {\triangle}

\newcommand\rc    {\right\rceil}
\newcommand\lc    {\left\lceil}
\newcommand\rf    {\right\rfloor}
\newcommand\lf    {\left\lfloor}

\newcommand\lex   {<_{lex}}

\newcommand\az    {\aleph_0}
\newcommand\uaz   {^{\aleph_0}}
\newcommand\al    {\aleph}
\newcommand\ual   {^\aleph}
\newcommand\taz   {2^{\aleph_0}}
\newcommand\utaz  { ^{\left (2^{\aleph_0} \right )}}
\newcommand\tal   {2^{\aleph}}
\newcommand\utal  { ^{\left (2^{\aleph} \right )}}
\newcommand\ttaz  {2^{\left (2^{\aleph_0}\right )}}

\newcommand\n     {$n$־יה\ }

% Math A&B shortcuts
\newcommand\logn  {\log n}
\newcommand\cosx  {\cos x}
\newcommand\cost  {\cos \theta}
\newcommand\sinx  {\sin x}
\newcommand\sint  {\sin \theta}
\newcommand\tanx  {\tan x}
\newcommand\tant  {\tan \theta}
\newcommand\dx    {\,\mathrm{d}x}

\newcommand\seq   {\overset{!}{=}}
\newcommand\sseq  {\overset{?}{=}}
\newcommand\sle   {\overset{!}{\le}}
\newcommand\sge   {\overset{!}{\ge}}
\newcommand\sll   {\overset{!}{<}}
\newcommand\sgg   {\overset{!}{>}}

\newcommand\h     {\hat}
\newcommand\ve    {\vec}
\newcommand\lv    {\overrightarrow}
\newcommand\ol    {\overline}

\newcommand\mlcm  {\mathrm{lcm}}

\newcommand\limz  {\lim_{x \to 0}}
\newcommand\limhz {\lim_{h \to 0}}
\newcommand\limxz {\lim_{x \to x_0}}
\newcommand\limi  {\lim_{x \to \infty}}
\newcommand\limni {\lim_{x \to - \infty}}
\newcommand\limpmi{\lim_{x \to \pm \infty}}

\newcommand\ta    {\theta}
\newcommand\ap    {\alpha}

\renewcommand\inf {\infty}
\newcommand  \ninf{-\inf}

% Combinatorics shortcuts
\newcommand\sumnk     {\sum_{k = 0}^{n}}
\newcommand\sumni     {\sum_{i = 0}^{n}}
\newcommand\sumnko    {\sum_{k = 1}^{n}}
\newcommand\sumnio    {\sum_{i = 1}^{n}}
\newcommand\sumai     {\sum_{i = 1}^{n} A_i}
\newcommand\nsum[2]   {\reflectbox{\displaystyle\sum_{\reflectbox{\scriptsize$#1$}}^{\reflectbox{\scriptsize$#2$}}}}

\newcommand\bink      {\binom{n}{k}}

\newcommand\cupain    {\bigcup_{i = 1}^{n} A_i}
\newcommand\cupai[1]  {\bigcup_{i = 1}^{#1} A_i}
\newcommand\cupiiai   {\bigcup_{i \in I} A_i}

\newcommand\sof[1]    {\left | #1 \right |}
\newcommand\cl [1]    {\left ( #1 \right )}

\newcommand\xot       {x_{1, 2}}
\newcommand\ano       {a_{n - 1}}
\newcommand\ant       {a_{n - 2}}

% Other shortcuts
\newcommand\tl    {\tilde}
\newcommand\op    {^{-1}}

\newcommand\bs    {\blacksquare}

%! ~~~ Document ~~~

\author{שחר פרץ}
\title{מתמטיקה B $\sim$ עברי נגר $\sim$ משפט דרבו, $e$, גזירת פונקציות לוגוריתמיות ומערכיות}

\begin{document}
	\maketitle
	
	נותר לנו להבין איך גוזרים כמה פונקציות נוספות. לדוגמה: $2^{x}, \log_2x$. 
	\section{$e$}
	נגיד ויש לנו $100\%$ ריבית בשנה: $1 \to 2$. 
	
	בנק אחר, יגיד שיש לנו $50\%$ כל חצי שנה: $1 \to 1.5 \to 2.25 $. בכלליות: 
	\[ e_n = \left (1 + \frac{1}{n}\right )^{n} \]
	שזה המספר בו הוא יכפיל את הסכום בשנה, בעבור $\frac{1}{n} \cdot 100\% $ ריבית. 
	
	\textbf{טענה: }הסדרה $e_n$ היא מונוטונית עולה. נשתמש ב־$AMGM$ (א''ש הממוצעים גיאומטרי־אריתמטי): 
	\[ 1 + \frac{1}{n + 1} = \frac{1 + n(1 + \frac{1}{n})}{n + 1} \ge \cl{1 \cdot \cl{1 + \frac{1}{n}}^{n}}^{1/n + 1} \implies e_{n + 1} \ge e_n \]
	
	\textbf{טענה: }$e_n \le 3$: ניעזר בבינום של ניוטון. 
	\[ e_n = \sumnk \bink \frac{1}{n^{k}} \]
	נוכל לחסום זאת, כי: 
	\[ \bink \frac{1}{n^{k}} = \frac{n(n - 1) \cdots (n - k + 1)}{k!} \cdot \frac{1}{n^{k}} = \frac{n(n - 1) \cdots (n - k + 1)}{n \cdot n \cdots n} \frac{1}{k!} \le \frac{1}{k!} \le \frac{1}{2^{k - 1}} \]
	נחזור לביטוי המקורי: 
	\[ e_n = \sumnk \bink \frac{1}{n^{k}} \le 1 + \sumnko \frac{1}{^{k - 1}} \le 1 + \sum_{k = 1}^{\inf} 2^{-k + 1} = 3 \]
	כאשר השוויון האחרון מטור גיאומטרי. 
	
	ישנו משפט, שסדרה מונוטנית עולה החסומה מלמעלה היא בעלת גבול. נסמנו: 
	\[ \limi e_n \equiv e \]
	ולמרות שטכנית עברנו עם $n \in \N$, זה לא משנה ברמת הגבול. 
	
	\textbf{טענה: }$e_n \ge 2$: 
	\[ e_n = \sumnk \bink \frac{1}{n^{k}} \ge \frac{n}{0}\frac{1}{n^{0}} + \frac{n}{1}\frac{1}{n} \ge 2 \]
	
	\textit{הערה: }אם נחשב אותו, נקבל $e \approx 2.718\dots$ (הוא אי־רציונלי). 
	
	\section{נגזרות}
	אז למה איכפת לנו מ־$e$?
	
	נסמן: 
	\begin{align}
		f(x) = \log_ax, \ f'(x) &= \lim_{h \to 0}\frac{f(x + h) - f(x)}{h}\\
		&=\limhz \frac{\log_a(1 + \frac{b}{x})}{h} \\
		&=\limhz \frac{1}{x}\cdot \frac{x}{h} \cdot \log_a\left (1 + \frac{h}{x} \right ) \\
		&= \lim_{t \to \inf} \frac{1}{x}\log_a\cl{\cl{1+\frac{1}{t}}^{t}} \quad \cl{t = \frac{x}{h}} \\
		&= \frac{1}{x} \underbrace{\lim_{t \to \inf}\log_a \left [ \cl{1 + \frac{1}{t}} \right ]^{t}}_{C_a}\\
		&= C_a \frac{1}{x} = \frac{1}{x}\log_a e = \frac{1}{\ln a}\cdot \frac{1}{x}
	\end{align}
	אם נציב, נקבל $(\ln x)' = \frac{1}{x}$. 
	
	עתה נוכל לגזור לוגוריתמים. לדוגמה: 
	\[ x = \log_a(a^{x}) \implies 1 = \frac{1}{\ln a} \cdot \frac{1}{x} \cdot (a^{x})' \implies \bm{(a^{x})' = \ln a \cdot a^{x}} \]
	אם נציב $a = e$, נקבל: $(e^{x})' = e^{x}$. 
	
	\textbf{טענה: }אם $f'(x) = f(x)$ בקטע, אז קיים קבוע $c \in \R$ כך ש־$f(x) = ce^{x}$. 
	
	מומלץ לנסות להוכיח את הטענה. 
	
	\section{תרגילים}
	גזרו את הפונקציה הבאה: 
	\[ [e^{x^2\cosx}]' =e^{x^2\cosx} \cdot [x^2 \cosx]' = e^{x^2\cosx}(-x^2\sinx + 2x\cosx) = e^{x^2\cosx}x(-x\sinx + 2\cosx) \]
	דוגמה 2: 
	\[ f(x) = x^{x} = e^{x \ln x} \implies f'(x) = e^{x \ln x} \cdot (\ln x + x \cdot \frac{1}{x}) = x^{x}(\ln x + 1) \]
	
	דוגמה 3: 
	\[ (\sinh(x))' = \left [\frac{e^{x} - e^{-x}}{2}\right ] = \cosh x = \frac{e^x+ e^{-x}}{2} \]
	באופן דומה: $ (\cosh x)' = \sinh $ ו־: 
	\[ \tanh x = \frac{\sinh x}{\cosh x} \implies (\tan x)' = \frac{\cosh^2 x - \sinh^2x}{\cosh^x} = \frac{1}{\cosh^x} \]
	הערה: שיעור שעבר נכתב על הלוח את הנגזרת $\tanx' = -\cos^{-2}x$ אך זה רק $\cos^{-2}x$. 
	
	\section{ללא כותרת}
	נוכל לדעת אם פונקציה היא מונוטונית עולה או מונוטונית יורדת בהתאם לסימן. לדוגמה, לפונקציה $x^2$ תהיה הנגזרת $2x$. הנגזרת של $x^2$ חיובית כאשר $x$ חיובי ולהיפך שהוא שלילי, ולכן היא מונוטונית עולה ויורדת בהתאם לסימן. באופן דומה, $x^3$ נגזרתה תהיה $3x^2$, פונקציה חיובית בכל תחום, לכן הפונקציה עולה בכל תחום. 
	
	נוכל לגזור שוב את $x^2$ ולקבל שהנגזרת השנייה, תהיה $2$, כלומר הנגזרת עולה בכל תחום – קצב גידול הפונקציה הולך וגדל. באופן דומה, עבור $x^3$ עבורה השלישית $6x$. 
	
	מכאן, ש־$f''(x) > 0$ יגרור שהפונקציה תהיה (קעורה) $\cup$ ואם $f''<0 $ אז הפונקציה קמורה $\cap$. יש לשנן לפני המבחן את השמות שיכול להיות שהופיעו לא נכון כאן (בגלל זה נכתבו בסוגריים). 
	
	המשיק לפונקציה בנקודה, הוא הישר שעובר בנקודה ושיפועו כנגזרת הפונקציה. אפשר להרחיב את הגדרת המשיק לפונקציות שאין להם בהכרח נגזרת בנקודה (לדוגמה $|x|$ כאשר $x = 0$) אך לא נתעסק בכך כאן. 
	
	לדוגמה, המשיק ל־$e^x$ בנקודה $(t, e^{x})$ הוא $y =  e^{t}(x - t) + e^{t}$. 
	
	\section{מינימום מקסימום}
	
	\subsection{משפט דרבו (Darbuax)}
	אם $f$ גזירה בקטע, אז $f'$ מקיימת את תכונת ערך הביניים. 
	\subsection{חזרה לקיצון לוקאלי}
	הנגזרת בנקודה בנק' קיצון לוקאלי (נקודה לפניה הפונקציה עולה ולאחריה יורדת), מתאפסת, לפי משפט דרבו. [הכנס כאן ערימה של דוגמאות למציאת קיצון שאני כבר מכיר]. המשוואה $f'(x) = 0$ תהיה תנאי מספיק למציאת קיצון בנקודה. 
	
\end{document}