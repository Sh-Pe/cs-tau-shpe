\documentclass[]{article}

% Math packages
\usepackage[usenames]{color}
\usepackage{forest}
\usepackage{ifxetex,ifluatex,amsmath,amssymb,mathrsfs,amsthm,witharrows}
\WithArrowsOptions{displaystyle}
\renewcommand{\qedsymbol}{$\blacksquare$} % end proofs with \blacksquare. Overwrites the defualts. 
\usepackage{cancel,bm}

% Deisgn
\usepackage[labelfont=bf]{caption}
\usepackage[margin=0.6in]{geometry}
\usepackage{multicol}
\usepackage[skip=4pt, indent=0pt]{parskip}
\usepackage[normalem]{ulem}
\forestset{default preamble={for tree={circle, draw}}}
\renewcommand\labelitemi{$\bullet$}

% Hebrew initialzing
\usepackage{polyglossia}
\setmainlanguage{hebrew}
\setotherlanguage{english}
\newfontfamily\hebrewfont[Script=Hebrew, Ligatures=TeX]{David CLM}
\usepackage[shortlabels]{enumitem}
\newlist{hebenum}{enumerate}{1}
\setlist[hebenum,1]{
	labelindent=\parindent,
	label={{\hebrewfont{\protect\hebrewnumeral{\value{hebenumi}}}}.}
}

% Language Shortcuts
\newcommand\en[1] {\selectlanguage{english}#1\selectlanguage{hebrew}}
\newcommand\sen   {\selectlanguage{english}}
\newcommand\she   {\selectlanguage{hebrew}}
\newcommand\del   {$ \!\! $}
\newcommand\ttt[1]{\en{\texttt{#1}}}

%! ~~~ Math shortcuts ~~~

% Letters shortcuts
\newcommand\N     {\mathbb{N}}
\newcommand\Z     {\mathbb{Z}}
\newcommand\R     {\mathbb{R}}
\newcommand\Q     {\mathbb{Q}}
\newcommand\C     {\mathbb{C}}

\newcommand\ml    {\ell}
\newcommand\mj    {\jmath}
\newcommand\mi    {\imath}

\newcommand\powerset {\mathcal{P}}
\newcommand\ps    {\mathcal{P}}
\newcommand\pc    {\mathcal{P}}
\newcommand\ac    {\mathcal{A}}
\newcommand\bc    {\mathcal{B}}
\newcommand\cc    {\mathcal{C}}
\newcommand\dc    {\mathcal{D}}
\newcommand\ec    {\mathcal{E}}
\newcommand\fc    {\mathcal{F}}
\newcommand\nc    {\mathcal{N}}
\newcommand\sca   {\mathcal{S}} % \sc is already definded
\newcommand\rca   {\mathcal{R}} % \rc is already definded

% Logic & sets shorcuts
\newcommand\siff  {\longleftrightarrow}
\newcommand\ssiff {\leftrightarrow}
\newcommand\so    {\longrightarrow}
\newcommand\sso   {\rightarrow}

\newcommand\epsi  {\epsilon}
\newcommand\vepsi {\varepsilon}
\newcommand\vphi  {\varphi}
\newcommand\Neven {\N_{\mathrm{even}}}
\newcommand\Nodd  {\N_{\mathrm{odd }}}
\newcommand\Zeven {\Z_{\mathrm{even}}}
\newcommand\Zodd  {\Z_{\mathrm{odd }}}
\newcommand\Np    {\N_+}

% Text Shortcuts
\newcommand\open  {\big(}
\newcommand\qopen {\quad\big(}
\newcommand\close {\big)}
\newcommand\also  {\text{, }}
\newcommand\defi  {\text{ definition}}
\newcommand\defis {\text{ definitions}}
\newcommand\given {\text{given }}
\newcommand\case  {\text{if }}
\newcommand\syx   {\text{ syntax}}
\newcommand\rle   {\text{ rule}}
\newcommand\other {\text{else}}
\newcommand\set   {\ell et \text{ }}
\newcommand\ans   {\mathit{Ans.}}

% Set theory shortcuts
\newcommand\ra    {\rangle}
\newcommand\la    {\langle}

\newcommand\oto   {\leftarrow}

\newcommand\QED   {\quad\quad\mathscr{Q.E.D.}\;\;\blacksquare}
\newcommand\QEF   {\quad\quad\mathscr{Q.E.F.}}
\newcommand\eQED  {\mathscr{Q.E.D.}\;\;\blacksquare}
\newcommand\eQEF  {\mathscr{Q.E.F.}}
\newcommand\jQED  {\mathscr{Q.E.D.}}

\newcommand\dom   {\text{dom}}
\newcommand\Img   {\text{Im}}
\newcommand\range {\text{range}}

\newcommand\trio  {\triangle}

\newcommand\rc    {\right\rceil}
\newcommand\lc    {\left\lceil}
\newcommand\rf    {\right\rfloor}
\newcommand\lf    {\left\lfloor}

\newcommand\lex   {<_{lex}}

\newcommand\az    {\aleph_0}
\newcommand\uaz   {^{\aleph_0}}
\newcommand\al    {\aleph}
\newcommand\ual   {^\aleph}
\newcommand\taz   {2^{\aleph_0}}
\newcommand\utaz  { ^{\left (2^{\aleph_0} \right )}}
\newcommand\tal   {2^{\aleph}}
\newcommand\utal  { ^{\left (2^{\aleph} \right )}}
\newcommand\ttaz  {2^{\left (2^{\aleph_0}\right )}}

\newcommand\n     {$n$־יה\ }

% Math A&B shortcuts
\newcommand\logn  {\log n}
\newcommand\cosx  {\cos x}
\newcommand\sinx  {\sin x}
\newcommand\tanx  {\tan x}
\newcommand\dx    {\,\mathrm{d}x}

\newcommand\seq   {\overset{!}{=}}
\newcommand\sle   {\overset{!}{\le}}
\newcommand\sge   {\overset{!}{\ge}}
\newcommand\sll   {\overset{!}{<}}
\newcommand\sgg   {\overset{!}{>}}

\newcommand\h     {\hat}
\newcommand\ve    {\vec}
\newcommand\lv    {\overrightarrow}

\newcommand\mlcm  {\mathrm{lcm}}

\newcommand\limz  {\lim_{x \to 0}}
\newcommand\limi  {\lim_{x \to \infty}}
\newcommand\limni {\lim_{x \to - \infty}}

\newcommand\ninf  {-\inf}

% Combinatorics shortcuts
\newcommand\sumnk     {\sum_{k = 0}^{n}}
\newcommand\sumni     {\sum_{i = 0}^{n}}
\newcommand\sumnko    {\sum_{k = 1}^{n}}
\newcommand\sumnio    {\sum_{i = 1}^{n}}
\newcommand\sumai     {\sum_{i = 1}^{n} A_i}
\newcommand\nsum[2]   {\reflectbox{\displaystyle\sum_{\reflectbox{\scriptsize$#1$}}^{\reflectbox{\scriptsize$#2$}}}}

\newcommand\bink      {\binom{n}{k}}

\newcommand\cupain    {\bigcup_{i = 1}^{n} A_i}
\newcommand\cupai[1]  {\bigcup_{i = 1}^{#1} A_i}
\newcommand\cupiiai   {\bigcup_{i \in I} A_i}

\newcommand\sof[1]    {\left | #1 \right |}

% Other shortcuts
\newcommand\tl    {\tilde}
\newcommand\op    {^{-1}}

\newcommand\bs    {\blacksquare}

%! ~~~ Document ~~~

\title{מתמטיקה B $\sim$ עוד על מרוכבים, גבולות בסיסיים}
\author{שחר פרץ}
\date{20 ליוני 2024}

\begin{document}
	\maketitle
	
	\section{מתמטיקה B – הרחבה על מרוכבים}
	\subsection{תרגיל פתיחה}
	\textbf{שאלה: }פתרו את המשוואה הבאה: 
	\[ 3x^2 + 2x + 10 = 0 \]
	\textbf{פתרון: }
	\[ x_{1, 2} = \frac{-2 \pm \sqrt{2^2  - 4 \cdot 3 \cdot 10}}{2 \cdot 3} = \frac{-2 \pm i \sqrt{116}}{6} = -\frac{1}{3} \pm \sqrt{29}i \]
	
	באופן דומה: 
	\[ x^2 + (5 - 3i)x + (4 - 9i) \implies x = \frac{1}{2}(-5 + 3i \pm \underbrace{\sqrt{(5 - 3i)^2 - 4(4 - 9i)})}_{\sqrt{25 - 9 - 30i - 16 + 36i = 6i}} = \frac{1}{2}(i5 + 3i \pm \sqrt6 \sqrt i) \]
	תזכורת: 
	\[ \sqrt i = \pm \frac{1 + i}{\sqrt2} \]
	
	נשים לב שפתורנות המשוואה הראשונה צמודים זה של זה. נגיע לטענה: 
	\subsection{חרא כזה או אחר}
	הבהרה: $\R[x]$ מתייחס לפולינמים במשתנה $x$ עם מקדמים ממשיים. 
	
	\textbf{טענה: }לכל פולינום ממשי $p(x) \in \R[x]$, מתקיים ש־$\forall z \in \C. p(z) = 0 \iff p(\bar z) = 0$. 
	
	\begin{proof}
		נסמן את הפולינום $p(x) = a_nx^n + \dots + a_0 $ כאשר $a_i \in \R$, ונניח $p(z) = 0$. צ.ל. $p(\bar z) = 0$. נציב ונקבל: 
		\[p(\bar z) = a_n(\bar z)^n + \dots + a_k(\bar z)^k + \dots + a_0 = a_n\bar{z^n} + \dots + a_k\bar{z^k} + \dots + a_0 =\bar{a_n z^n} + \dots \bar{a_kz^k} + \dots + \bar{a_0} = \bar{a_nz^n + \dots + a_0} = \bar{p(z)} = 0 \]
		\textbf{טענה: }צמוד של מכפלה הוא מכפלת הצמודים. (השתמשנו בטענה זו בזמן ההוכחה והיא תופיע גם בשיעורי הבית)
		הגרירה השנייה מתקיימת מסימטריה. 
	\end{proof}
	
	ננסה לעשות טרינום לכל הפולינומים. בשביל לעשות זאת, נחלק פולינומים. 
	\[ \frac{p(x)}{x - x_0} = ? \]
	ננסה להוכיח קיום $q(x)$ כך ש־$p(x) = q(x)(x - x_0) + r$, ונרצה למצוא את $q(x)$. נסמן $q(x) = b_{n = 1}x^{n + 1} + \dots  + b_0 $
	\[ q(x)(x - x_0) = \sumnk x^k (-x_0b_k + b_{k - 1}) \]
	לכן: 
	\[ a_n = b_{n - 1}, a_{n - 1} = b_{n - 2} - x_0b_{n - 1} \ \cdots \ a_k = b_{k - 1} - x_0b_k \ \cdots \ a_0 = r - x_0b_0 \]
	
	עכשיו, ננסה להגדיר חילוק פולינומים בעבור פולינומים שהם לא ליניאריים. 
	\textit{דוגמה לחילוק ארוך שכרגע על דף}
	
	\textbf{משפט: }$p(x_0) = 0 \iff (x - x_0) \mid p(x)$
	\begin{proof}
		\textbf{כיוון ראשון: }נניח $(x - x_0) \mid p(x) \implies p(x) = q(x)(x - x_0)$ ואז $p(x_0) = q(x_0) \cdot 0 = 0$ \\
		\textbf{כיוון שני: }נניח $p(x_0) = 0$. נחלק עם שארית $p(x) = q(x)(x - x_0) + r$. ידוע $0 = p(x_0) = r$ כלומר אין שארית ולכן $(x - x_0) \mid p(x)$. 
	\end{proof}
	
	\subsection{המשפט היסודי של האלגברה}
	``זה בעצם משפט באנליזה'' $\sim$ עברי
	
	\large \textit{\textbf{לכל פולינום לא קבוע מעל }$\bm{\C}$\textbf{ קיים שורש ב־}$\bm{\C}$.}
	\normalsize	כלומר $\C$ סגור אלגברית, בשפה גבוהה. 
	לא קבוע = לא ממעלה 0
	
	יהי $p(x) \in \C[x]$. לכן ניתן לרשום אותו בצורה של $p(x) = (x - x_1)q(x)$. אך גם ל־$q(x)$ יש שורש! לכן נוקבל להמשיך לכתוב זאת בצורה הזו באינדוקציה, ולקבל $p(x) = (x - x_1)(x - x_2) \dots (x - x_n) \cdot a_n $. למעשה, כך הוכחנו באינדוקציה, שלכל פולינום מעל המרוכבים אפשר לכתוב בצורה של גורמים ליניאריים או משהו כזה. יש לו $n$ שורשים (ולכל היותר $n$ שורשים שונים, כי ייתכן ואחד מהם מופיע פעמיים). 
	
	\textit{הערה: }הפירוק הזה הוא יחיד. אפשר להוכיח זאת באמצעות כלים של חדו''א. 
	
	\subsection{ועכשיו לאפילו עוד חרא על מעגלים}
	להלן משוואה, לא פולינומיאלית: 
	\[ |z + 3i| = 3|z| \]
	נרצה למצוא את כל הפתרונות המתאימים. ננסה לפתור. נציב $z = a + bi$, כאשר $a, b \in \R$. נעלה את שני האגפים בריבוע (מותר כי שני האגפים אי־שליליים) ונקבל: 
	\[ |z + 3i|^2 = 9|z|^2 \implies |a + bi + 3i|^2 = 9|a + bi|^2 \implies a^2 + (b + 3)^2 = 9(a^2 + b^2) \]
	טוב נתחיל לעשות אנטרים שזה יהיה קריא: 
	\begin{gather*}
		3b^2 - 6b - 8 + 8a^2 = 0 \\
		b^2 - \frac{3}{4}b - \frac{9}{8} + a^2 = 0 \\
		\left ( b - \frac{3}{8} \right )^2 +  -\frac{9}{8} - \left (\frac{3}{8} \right )^2  + a^2 = 0 \\
		\left (b - \frac{3}{8} \right ) + a^2 = \frac{9}{8} + \frac{9}{64} = \frac{81}{64} = \left (\frac{9}{8}\right )^2
	\end{gather*}
	
	למעשה, קיבלנו שכל הנקודות המתאימות מציירות מעגל על מרחב המרוכבים (תציירו ב־desmos או תיצרו איתי קשר אם זה לא ברור). מרכזו יהיה ב־$0 + \frac{3}{8}i$, ורדיוסו $\frac{9}{8}$. 
	
	\subsection{סתם תרגיל אקראי}
	תרגיל: פתרו את המשוואה $\bar z = 2i(z - 1)$. נציב $z = a + bi$. נקבל: 
	\begin{gather}
		a - bi = 2i(a + bi - 1) \\
		a - bi = 2ai - b - 2i \\
		a - 2ai = -b -bi - 2i \\
		a(1 - 2i) = - (b + bi + 2i) \\
		a = \frac{b + bi + 2i}{1 - 2i} \\
		z = \frac{b + bi + 2i}{1 - 2i} + bi
	\end{gather}
	
	וזו לא השיטה. צריך להעביר את כל המספרים לצד שמאל, ולקבל: 
	\begin{gather}
		\begin{cases}
			a + 2b = 0 \\
			-b - 2(a - 1) = 0
		\end{cases}
		\implies a = \frac{4}{3}, b = \frac{-2}{3} \implies z = \frac{4}{3} - \frac{2}{3}i
	\end{gather}
	
	\section{חדו''א}
	``גבולות וכאלו''
	
	חלק מהחומר יהיה לא פורמלי, בניגוד לקורס בחדו''א שיהיה פורמלי מאוד. 
	
	\textbf{גבול: }לאן פונקציה ``הולכת'' בנקודה מסויימת. 
	
	\[ \lim_{x \to x_0} f(x) = L \]
	משמעו שככל ש־$x$ יותר קרוב ל־$x_0 $ הפונקציה יותר קרובה ל־$L$. הגבול של הפונקציה לא תלוי בערך של הפונקציה בנקודה. 
	לדוגמה $\lim_{x \to 3} x^2 = 9$ וגם $\lim_{x \to 0} \sin x = 0$. ישנם גם גבולות חד צדדיים: $\lim_{x \to 0^+} \sqrt{x} = 0 $ (כלומר, ככל שהגבול יותר קרוב ל־0 מצד ימין אז הוא יתקרב ל־0). 
	ניקח לדוגמה את הפונקציה: 
	$g(x) = \begin{cases}
		7 &x \le x_0 \\
		5 &x > x_0
	\end{cases}$
	אז: 
	\[ \lim_{x \to x_0^+} g(x) = 7, \ \lim_{x \to x_0^-} g(x) = 5, \ \lim_{x \to x_0} g(x) \in \emptyset \]
	
	באופן דומה: 
	\[ h(x) = \begin{cases}
		1 & x \neq 3 \\
		10 & x = 3
	\end{cases} \]
	אזי: 
	\[ \lim_{x \to 3} h(x) = 1 \]
	עוד גבולות: 
	\[ \lim_{x \to 2}\frac{x^2 - 4}{x - 2} = \lim_{x \to 2} x + 2 = 4 \]
	פרט לאי־רציפות, לא יהיה גבול בנקודה מסויימת. לדוגמה: 
	\[ \lim_{x \to 0} \sin \frac{1}{x} \in \emptyset, \ \lim_{x \to 1} \sin \frac{1}{x} = \sin 1 \]
	האינטואציה לכך היא שככל השסינוס יתקרב ל־0 הוא יעשה יותר מחרוזים. לא משנה כמה נתקרב, היא תתבלגן עוד יותר. אפשר לצייר על דסמוס כדי להבהיר. 
	לפונקציית ירכלה אין גבול באף נקודה: 
	\[ D(x) = \begin{cases}
		1 & x \in \Q  \\
	0 & x \in \R \setminus \Q
	\end{cases} \implies \lim_{x \to a} D(x) \in \emptyset \]
	כמו דוגמאות שכבר ראינו: 
	\[ \lim_{x \to 0}\frac{1}{x} \in \emptyset, \ \lim_{x \to 0^\pm}\frac{1}{x} \pm \infty \]
	
	אם הפונקציה רציפה, פשוט אפשר להציב. אפשר גם לדבר על גבולות אינסופיים: 
	\[ \lim_{x \to \infty} = \infty, \ \limi \frac{1}{x} = 0, \ \limni x^3 = - \infty \]
	
	
\end{document}