\documentclass[]{article}

% Math packages
\usepackage[usenames]{color}
\usepackage{forest}
\usepackage{ifxetex,ifluatex,amsmath,amssymb,mathrsfs,amsthm,witharrows}
\WithArrowsOptions{displaystyle}
\renewcommand{\qedsymbol}{$\blacksquare$} % end proofs with \blacksquare. Overwrites the defualts. 
\usepackage{cancel,bm}

% Deisgn
\usepackage[labelfont=bf]{caption}
\usepackage[margin=0.6in]{geometry}
\usepackage{multicol}
\usepackage[skip=4pt, indent=0pt]{parskip}
\usepackage[normalem]{ulem}
\forestset{default preamble={for tree={circle, draw}}}
\renewcommand\labelitemi{$\bullet$}

% Hebrew initialzing
\usepackage{polyglossia}
\setmainlanguage{hebrew}
\setotherlanguage{english}
\newfontfamily\hebrewfont[Script=Hebrew, Ligatures=TeX]{David CLM}
\usepackage[shortlabels]{enumitem}
\newlist{hebenum}{enumerate}{1}
\setlist[hebenum,1]{
	labelindent=\parindent,
	label={{\hebrewfont{\protect\hebrewnumeral{\value{hebenumi}}}}.}
}

% Language Shortcuts
\newcommand\en[1] {\selectlanguage{english}#1\selectlanguage{hebrew}}
\newcommand\sen   {\selectlanguage{english}}
\newcommand\she   {\selectlanguage{hebrew}}
\newcommand\del   {$ \!\! $}
\newcommand\ttt[1]{\en{\texttt{#1}}}

%! ~~~ Math shortcuts ~~~

% Letters shortcuts
\newcommand\N     {\mathbb{N}}
\newcommand\Z     {\mathbb{Z}}
\newcommand\R     {\mathbb{R}}
\newcommand\Q     {\mathbb{Q}}
\newcommand\C     {\mathbb{C}}

\newcommand\ml    {\ell}
\newcommand\mj    {\jmath}
\newcommand\mi    {\imath}

\newcommand\powerset {\mathcal{P}}
\newcommand\ps    {\mathcal{P}}
\newcommand\pc    {\mathcal{P}}
\newcommand\ac    {\mathcal{A}}
\newcommand\bc    {\mathcal{B}}
\newcommand\cc    {\mathcal{C}}
\newcommand\dc    {\mathcal{D}}
\newcommand\ec    {\mathcal{E}}
\newcommand\fc    {\mathcal{F}}
\newcommand\nc    {\mathcal{N}}
\newcommand\sca   {\mathcal{S}} % \sc is already definded
\newcommand\rca   {\mathcal{R}} % \rc is already definded

% Logic & sets shorcuts
\newcommand\siff  {\longleftrightarrow}
\newcommand\ssiff {\leftrightarrow}
\newcommand\so    {\longrightarrow}
\newcommand\sso   {\rightarrow}

\newcommand\epsi  {\epsilon}
\newcommand\vepsi {\varepsilon}
\newcommand\vphi  {\varphi}
\newcommand\Neven {\N_{\mathrm{even}}}
\newcommand\Nodd  {\N_{\mathrm{odd }}}
\newcommand\Zeven {\Z_{\mathrm{even}}}
\newcommand\Zodd  {\Z_{\mathrm{odd }}}
\newcommand\Np    {\N_+}

% Text Shortcuts
\newcommand\open  {\big(}
\newcommand\qopen {\quad\big(}
\newcommand\close {\big)}
\newcommand\also  {\text{, }}
\newcommand\defi  {\text{ definition}}
\newcommand\defis {\text{ definitions}}
\newcommand\given {\text{given }}
\newcommand\case  {\text{if }}
\newcommand\syx   {\text{ syntax}}
\newcommand\rle   {\text{ rule}}
\newcommand\other {\text{else}}
\newcommand\set   {\ell et \text{ }}
\newcommand\ans   {\mathit{Ans.}}

% Set theory shortcuts
\newcommand\ra    {\rangle}
\newcommand\la    {\langle}

\newcommand\oto   {\leftarrow}

\newcommand\QED   {\quad\quad\mathscr{Q.E.D.}\;\;\blacksquare}
\newcommand\QEF   {\quad\quad\mathscr{Q.E.F.}}
\newcommand\eQED  {\mathscr{Q.E.D.}\;\;\blacksquare}
\newcommand\eQEF  {\mathscr{Q.E.F.}}
\newcommand\jQED  {\mathscr{Q.E.D.}}

\newcommand\dom   {\text{dom}}
\newcommand\Img   {\text{Im}}
\newcommand\range {\text{range}}

\newcommand\trio  {\triangle}

\newcommand\rc    {\right\rceil}
\newcommand\lc    {\left\lceil}
\newcommand\rf    {\right\rfloor}
\newcommand\lf    {\left\lfloor}

\newcommand\lex   {<_{lex}}

\newcommand\az    {\aleph_0}
\newcommand\uaz   {^{\aleph_0}}
\newcommand\al    {\aleph}
\newcommand\ual   {^\aleph}
\newcommand\taz   {2^{\aleph_0}}
\newcommand\utaz  { ^{\left (2^{\aleph_0} \right )}}
\newcommand\tal   {2^{\aleph}}
\newcommand\utal  { ^{\left (2^{\aleph} \right )}}
\newcommand\ttaz  {2^{\left (2^{\aleph_0}\right )}}

\newcommand\n     {$n$־יה\ }

% Math A&B shortcuts
\newcommand\logn  {\log n}
\newcommand\cosx  {\cos x}
\newcommand\sinx  {\sin x}
\newcommand\tanx  {\tan x}
\newcommand\dx    {\,\mathrm{d}x}

\newcommand\seq   {\overset{!}{=}}
\newcommand\sle   {\overset{!}{\le}}
\newcommand\sge   {\overset{!}{\ge}}
\newcommand\sll   {\overset{!}{<}}
\newcommand\sgg   {\overset{!}{>}}

\newcommand\h     {\hat}
\newcommand\ve    {\vec}
\newcommand\lv    {\overrightarrow}

\newcommand\mlcm  {\mathrm{lcm}}

\newcommand\limz  {\lim_{x \to 0}}
\newcommand\limxz {\lim_{x \to x_0}}
\newcommand\limi  {\lim_{x \to \infty}}
\newcommand\limni {\lim_{x \to - \infty}}

\renewcommand\inf {\infty}
\newcommand\ninf  {-\inf}

% Combinatorics shortcuts
\newcommand\sumnk     {\sum_{k = 0}^{n}}
\newcommand\sumni     {\sum_{i = 0}^{n}}
\newcommand\sumnko    {\sum_{k = 1}^{n}}
\newcommand\sumnio    {\sum_{i = 1}^{n}}
\newcommand\sumai     {\sum_{i = 1}^{n} A_i}
\newcommand\nsum[2]   {\reflectbox{\displaystyle\sum_{\reflectbox{\scriptsize$#1$}}^{\reflectbox{\scriptsize$#2$}}}}

\newcommand\bink      {\binom{n}{k}}

\newcommand\cupain    {\bigcup_{i = 1}^{n} A_i}
\newcommand\cupai[1]  {\bigcup_{i = 1}^{#1} A_i}
\newcommand\cupiiai   {\bigcup_{i \in I} A_i}

\newcommand\sof[1]    {\left | #1 \right |}

% Other shortcuts
\newcommand\tl    {\tilde}
\newcommand\op    {^{-1}}

\newcommand\bs    {\blacksquare}

%! ~~~ Document ~~~


\author{שחר פרץ}
\title{מתמטיקה B $\sim$ הרבה דברים}
\date{27 למאי 2024}

\begin{document}
	\maketitle
	\section{עוד על גבולות}
	\subsection{משפטים נוספים}
	\textbf{הגדרה: }$f \colon I \to \R$ (כאשר $I $ קטע) יש את תכונת ערך הביניים אם לכל $a, b \in I$ קיים $c$ בין $b$ ל־$a$ כך ש־$f(c)$ בין $f(b)$ ל־$f(c)$. 
	
	\textbf{משפט ערך הביניים: }בפונקציות רציפות מתקיימת תכונת ערך הביניים. 
	
	יש דוגמאות ``מאוד סוציופתיות'' שמקיימות את תכונת ערך הביניים אך אינן רציפות. 
	
	אופסי. הניסוח של עברי נוראי וכולל משום מה את פונקציית ירכלה. נגדיר מחדש את תכונת ערך הביניים. 
	... אם לכל $a, b \in I$ ולכל $\xi$ בין $f(a)$ ל־$f(b)$ קיים $c$ בין $b$ ל־$a$ כך ש־$f(c) = \xi $. 
	
	\textbf{משפט הסנדוויץ': }אם מתקיים ש־$\lim_{x \to a} f(x) = \lim_{x \to a} h(x) = L$ וגם $f \le g \le h$ בסביבה נקובה של $a$ אז $\lim_{x \to a} g(x) = L$. ידוע גם בשם ``משפט שני שוטרים ושיכור''. 
	
	\section{שימושים במשפטים האלו}
	\subsection{}
	נניח ואנו מעוניינים לחשב את הגבול $\lim_{x \to 0} \frac{\sinx}{x}$. 
	
	מתוך הגבול היחיד הזה, שאותו נחשב באמצעות כלים גיאוטרים, נוכל לקבל את הגבולות האחרים שיעניינו אותנו בנוגע לגיאומטריה. 
	
	נתבונן במעגל היחידה, כאשר $x$ שואף ל־$0$. ראה סרטוט. השטח של המשולש הירוק הוא $\frac{1}{2} \sinx $ ושטח הגזרה הוא $\pi \cdot \frac{x}{2\pi}$. השטח של המשולש הירוק יהיה $\frac{1}{2}\tanx$. מתוך זה ברור מהסרטוט, $\frac{1}{2} \sinx < \pi \cdot \frac{x}{2\pi } < \frac{1}{2} \tanx$. מכמה מעברים אלגברים נקבל $\sinx < x < \tanx$. מצג אחד $\sinx <x \implies \frac{\sinx}{x} < 1 $, ומצד שני $x < \tanx  = \frac{\sinx}{\cosx} \implies \frac{\sinx}{x} > \cosx$. לפי משפט הסנדוויץ, בגלל ש־$\limz 1 = \limz \cosx = 1$ ממשפט הסנדוויץ'; $\limz \frac{\sinx}{x} = 1 $. [הערה: האי־שוויונות האלו צריכים להתקיים רק בסביבה של $x \to 0$]
	
	\textit{הערה: }הנחנו ש־$x$ חיובי, אך $\frac{\sinx}{x}$ זוגית ולכן זה לא משנה. 
	
	\textit{הערה 2: }הנחנו גם ש־$x$ ברדיאנים. נרחיב עבור כל מידה: $\limz \frac{\sin(ax)}{x} = \limz a \cdot \frac{\sin ax}{ax}\overset{t = ax}{=} \lim_{t \to 0}a \frac{\sin t}{t} = a \cdot 1 = a$
	
	ובפרט עבור מעלות. 
	
	\subsection{שימוש במשפט}
	\begin{itemize}
		\item נשתמש באותו הגבול עכשיו: 
		\[ \lim_{x \to 0} \frac{1 - \cosx }{x^2} = \frac{2\sin^2 \left (\frac{x}{2}\right )}{x^2} = 2 \left (\frac{\sin \frac{x}{2}}{x^2} \right )^2 = 2 \cdot \left (\frac{1}{2} \right ) = \frac{1}{2} \] 
		\item 
		\[ \limz \frac{\sin ax}{\sin bx} = \limz \frac{\sin ax}{x} \cdot \frac{x}{\sin bx} = a \cdot \frac{1}{b} = \frac{a}{b} \]
		\item תרגיל: 
		\[ \lim_{x \to \inf} \sin \left [\underbrace{\frac{x^2 + 3}{\sqrt{x^6 + 18x - 1}}}_{u \to 0}\right ] = \lim_{u \to 0^+} \sin (u) = 0 \]
	\end{itemize}
	
	\section{פונקציות היפר־טריגונומטריות}
	ישנה הפונקציה $e^x$ המעריכית. 
	חלקה האי־זוגי: $\frac{e^x - e^{-x}}{2}$ / החלק הזוגי: $ \frac{e^x + e^{-x}}{2} $
	
	החלק הזוגי יהיה $\cosh$ והאי־זוגי $\sinh$. ואכן חיבורם יהיה $e^x$. 
	
	גרפים אני לא עושה כאן, תעשו על דסמוס. 
	
	באופן דומה $\tanh x = \frac{\sinh x}{\cosh x}$. 
	
	טענה: $\sinh$ תהיה מונוטונית עולה בתחומה החיובי: 
	\[ \underbrace{\cosh x+ \cosh y}_{\frac{e^x - e^y + e^{_x} - e^{-y}}{2}} = \frac{1}{2}e^x(1 - e^{-x - y})(1 - e^{-x + y}) \] 
	אם $x > y > 0$ אז $-x + y $ שלילי כלומר $e^{-x + y } < 1 $. באופן דומה $1 - e^{-x - y }$ קטן מ־1. סה''כ מכפילים שני עגפים חיוביים וגמרנו. 
	
	משום שהפונקציה אי־זוגית אז זה מוכיח גם את המונוטוניות מהצד השני. 
	\subsection{אז למה קוראים להן טריגונומטריות?}
	\[ \cosh^2 x = \frac{e^{2x} + 2 + e^{-2x}}{4} = \frac{1}{2} + \frac{e^{2x} + e^{-2x}}{4} = \frac{1}{2} + \frac{1}{2}\cosh 2x = \frac{1 + \cosh 2x}{2} \]
	\[ \sinh^2 = \frac{e^{2x} - 2 + e^{-2x}}{4}= -\frac{1}{2} + \frac{1}{2}\cosh 2x = \frac{\cosh 2x - 1}{2} \]
	\[ \cosh^2x - \sinh^2x = 1 \]
	\begin{align}
		\sinh(x + y) &= \frac{e^{x + y}- e ^{-x - y}}{2} = \frac{e^{x + y} - e^{x - y} + e^{x - y} - e^{-x - y}}{2} \\
		&= e^x \frac{e^y - e^{-y}}{2} + e^{-y} \frac{e^x - e^{-x}}{2} \\
		&= e^x \frac{e^y - e^{-y}}{2} + e^{-y} \frac{e^x - e^{-x}}{2} + e^{-x}\frac{e^y - e^{-y}}{2} - e^{-x}\frac{e^y - e^{-y}}{2}
	\end{align}
	אוקי עכשיו עברי החליט שהוא מעדיף לפתור את זה בדרך אחרת: 
	\begin{align}
		\sinh(x + y) &= \frac{1}{4}(2 e^{x + y} - 2e^{-x - y} \bm{- e^{x - y} - e^{y - x} + e^{x - y} + e^{y - x}}) \\
		&= \frac{1}{2}\left [ e^x\frac{e^y - e^{-y}}{2} + e^y \frac{e^x - e^{-x}}{2} + e^{-y} \frac{e^x-e^{-x}}{2} + e^{-x}\frac{e^y - e^{-y}}{2} \right ] \\
		&= \frac{e^x - e^{-y}}{2} \cdot \frac{e^x + e^{-x}}{2} + \frac{e^x - e^{-x}}{2} \cdot \frac{e^y + e^{-y}}{2}\\
		&= \sinh y \cdot \cosh x + \sinh x \cosh y
	\end{align}
	\textit{(בבולד מה שעברי הוסיף והחסיר אך לא משנה את הערך)}
	כלומר, הזהויות שאנו מכירים מהפונקציות הטריגונומטריות מתנהגות באופן דומה לפונקציות הטריגונומטריות, עד לכדי סימן. בהמשך נראה למה. 
	
	\subsection{תרגיל}
	\begin{align}
		\cosh x \cdot \cosh y + \sinh x \cdot \sinh y &\seq \cosh(x + y) \\
		\cosh x \cdot \cosh y + \sinh x \cdot \sinh y &= \frac{e^x + e^{-x}}{2} \cdot \frac{e^y + e^{-y}}{2} + \frac{e^x - e^{-x}}{2} \cdot \frac{e^y - e^{-y}}{2}\\
		&=\frac{1}{2}\left [ 2e^x + 2e^{-x} + 2e^y + 2e^{-y} \right ] \\
		&=e^{x + y} + e^{-x - y} = 2\cosh(x + y)
	\end{align}
	טוב עשיתי טעות ואין לי זמן למצוא איפה
	
	\subsection{הפונקציות ההופכיות לפונקציות ההיפר־טריגונומטריות}
	\[ \begin{WithArrows}
		y &= \sinh x= \frac{e^x - e^{-x}}{2} \Arrow{$\cdot e^x$} \\
		ye^x &= \frac{(e^x)^2 - 1}{2} \\
		(e^x)^2 &- 2ye^x - 1 \\
		e^x &= y \pm \sqrt{y^2 + 1} \\
		x &= \sinh^{-1} = \ln(y + \sqrt{y^2 + 1})
	\end{WithArrows} \]
	
	\section{נגזרות}
	בקורס חדו''א מגיעים לנגזרות בשבוע 7 אז תגידו תודה. עברי אומר שהוא יסתמך על החומר של המכינה ``של השמחות''. 
	\[ f'(x) = \frac{\mathrm{d}f}{\dx} = \lim_{h \to 0} \frac{f(x + h) - f(x)}{h} \]
	$h$ מייצג את $\dx$
	נקודות על נגזרות: 
	\begin{enumerate}
		\item ליניאריות (בגלל שגבול מקיים את התכונות הללו): 
		\[ (f + g)' = f' + g', \quad (a \cdot f') = a \cdot f' \]
		\item מכפלה ומנה: 
		\begin{align}
			(f \cdot g) &= f'g + fg' \\
			\left (\frac{f}{g}\right )' &= \frac{f'g - g'f}{g^2} \\
			(x^n)' = nx^{n - 1}, \quad \sinx' = \cosx, &\quad \cosx' = -\sinx, \quad \tanx' = - \frac{1}{\cos^x}\\
			(f \circ g) &= f'(g(x)) \cdot g'(x)
		\end{align}
		וזה נכון גם על אובייקטים לא קומטטביים כמו וקטורים (לפעמים). 
		אפשר לגבר גם על נגזרת, ראשונה, שנייה, שלישית וכו':
		\[ (\sinx)' = \cosx, \ (\sinx)'' = -\sinx, \ (\sinx)''' = -\cosx, \ (\sinx)'''' = \sinx \]
		את הנגזרת ה־$n$־ית של פונקצייה כללית, נסמן ב־: 
		\[ f^{\underbrace{''''''}_{\times n}} = f^{(n)}(x) = \frac{d^nf}{dx^n} \]
		לפיכך, נוכל לכתוב: 
		\[ (\sinx)^{(n)} = \begin{cases}
			\cos x & n \equiv 1 \\
			-\sinx & n \equiv 2 \\
			-\cosx & n \equiv 3 \\
			\sinx  & n \equiv 0
		\end{cases} \mod 4 \]
	\end{enumerate}
		\subsection{נגזרת סתומה}
		באנגלית – diffrenetation implicit (המלצה של עברי: לתרגם את זה דרך ויקיפדיה)
		\[ xy = 1 \quad \quad y = \frac{1}{x} \quad \quad y' = - \frac{1}{x^2} \]
		הטענה, היא שהיה אפשר לעבוד גם בסדר ההפוך: 
		\[ y + y'x = (xy)' = 1' = 0 \implies y' = -\frac{y}{x} = -\frac{1}{x^2} \]
		אז למה זה מועיל? 
		\begin{align}
			x^3y(x)^5 + 3x &= 8y(x)^3 + 1 \\
			3x^2y^5 + 5y^4y'x^3 +3 &= 8 \cdot 3y^2y'+ 0 \\
			y' = \frac{3x^2y^3 + 3}{246^2 - 5y^4x^3}
		\end{align}
		
		יש מקרים שבהם אי אפשר לגזור עבור $y$ אך בגלל קסם באמצעות גזירה סתומה זה יעבוד. במקרים אחרים סתם יהיה יותר נוח להציב בנגזרת הסתומה במקרים לכזור מפלצות. 
		\subsubsection{''תרגילונון''}
		\begin{align}
			y^2 + x\sin y &= \cosx^2 \\
			2y y' + \sin y + x \cdot \cos y \cdot y' &= -\sin(x^2) \cdot 2x \\
			\implies y' &= \frac{-2x\sin(x^2) - \sin y}{2y + \cos y \cdot x}
		\end{align}
		\subsection{שימוש בעולם האמיתי}
		ידוע לנו הנגזרת של $x^a$. נרצה להרחיב עבור $a = \frac{n}{m}$ רציונלי ($n \in \Z, \ m \in \N$). כלומר, $y = x^\frac{n}{m}$ (תחת ההנחה שהיא מוגדרת). 
		\begin{alignat}{9}
			&y = x^{\frac{n}{m}} \implies y^m = x^n \\
			\implies & my^{m - 1}y' = nx^{n - 1} \implies y' = \frac{nx^{n - 1}}{my^{m - 1}} \\
			\implies & y' = \frac{nx^{n - 1}}{m x^{n - \frac{n}{m}}} = \frac{n}{m}x^{\frac{n}{m} - 1} \\
			\implies &ax^{a - 1}
		\end{alignat}
		כלומר הנוסחה עובדת גם עבור אי־רציונליים. אם נגדיר כמו שצריך ממשיים אז זה יעבוד גם עליהם. וכנ''ל על מרוכבים. 
		
		\subsection{עוד דוגמאות}
		לפי חוקי חזקות: 
		\[ (\sqrt{x}) = \frac{1}{2} \cdot \frac{1}{\sqrt{x}} \]
		תרגיל: תגזרו: 
		\begin{align}
			y(x) &= \cos(x^{5 / 3} \tan^{7 / 2}(x)) \\
			y'(x) &= -\sin(x^{5 / 3} \tan^{7 / 2}(x)) \cdot \frac{5}{3}x^{2 / 3} \tan^{7/ 2}x \cdot \frac{7}{2} \cdot \tan^{5 / 2}(x) \cdot \frac{1}{\cos^2x}
		\end{align}
		
\end{document}