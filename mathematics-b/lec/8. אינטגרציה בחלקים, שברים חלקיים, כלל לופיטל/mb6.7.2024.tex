%! ~~~ Packages Setup ~~~ 
\documentclass[]{article}


% Math packages
\usepackage[usenames]{color}
\usepackage{forest}
\usepackage{ifxetex,ifluatex,amsmath,amssymb,mathrsfs,amsthm,witharrows,mathtools}
\WithArrowsOptions{displaystyle}
\renewcommand{\qedsymbol}{$\blacksquare$} % end proofs with \blacksquare. Overwrites the defualts. 
\usepackage{cancel,bm}
\usepackage[thinc]{esdiff}


% tikz
\usepackage{tikz}
\newcommand\sqw{1}
\newcommand\squ[4][1]{\fill[#4] (#2*\sqw,#3*\sqw) rectangle +(#1*\sqw,#1*\sqw);}


% code 
\usepackage{listings}
\usepackage{xcolor}

\definecolor{codegreen}{rgb}{0,0.35,0}
\definecolor{codegray}{rgb}{0.5,0.5,0.5}
\definecolor{codenumber}{rgb}{0.1,0.3,0.5}
\definecolor{codeblue}{rgb}{0,0,0.5}
\definecolor{codered}{rgb}{0.5,0.03,0.02}
\definecolor{codegray}{rgb}{0.96,0.96,0.96}

\lstdefinestyle{pythonstylesheet}{
	language=Python,
	emphstyle=\color{deepred},
	backgroundcolor=\color{codegray},
	keywordstyle=\color{deepblue}\bfseries\itshape,
	numberstyle=\scriptsize\color{codenumber},
	basicstyle=\ttfamily\footnotesize,
	commentstyle=\color{codegreen}\itshape,
	breakatwhitespace=false, 
	breaklines=true, 
	captionpos=b, 
	keepspaces=true, 
	numbers=left, 
	numbersep=5pt, 
	showspaces=false,                
	showstringspaces=false,
	showtabs=false, 
	tabsize=4, 
	morekeywords={as,assert,nonlocal,with,yield,self,True,False,None,AssertionError,ValueError,in,else},              % Add keywords here
	keywordstyle=\color{codeblue},
	emph={object,type,isinstance,copy,deepcopy,zip,enumerate,reversed,list,set,len,dict,tuple,print,range,xrange,append,execfile,real,imag,reduce,str,repr,__init__,__add__,__mul__,__div__,__sub__,__call__,__getitem__,__setitem__,__eq__,__ne__,__nonzero__,__rmul__,__radd__,__repr__,__str__,__get__,__truediv__,__pow__,__name__,__future__,__all__,},          % Custom highlighting
	emphstyle=\color{codered},
	stringstyle=\color{codegreen},
	showstringspaces=false,
	abovecaptionskip=0pt,belowcaptionskip =0pt,
	framextopmargin=-\topsep, 
}
\newcommand\pythonstyle{\lstset{pythonstylesheet}}
\newcommand\pyl[1]     {{\lstinline!#1!}}
\lstset{style=pythonstylesheet}

\usepackage[style=1,skipbelow=\topskip,skipabove=\topskip,framemethod=TikZ]{mdframed}
\definecolor{bggray}{rgb}{0.85, 0.85, 0.85}
\mdfsetup{leftmargin=0pt,rightmargin=0pt,innerleftmargin=15pt,backgroundcolor=codegray,middlelinewidth=0.5pt,skipabove=5pt,skipbelow=0pt,middlelinecolor=black,roundcorner=5}
\BeforeBeginEnvironment{lstlisting}{\begin{mdframed}\vspace{-0.4em}}
	\AfterEndEnvironment{lstlisting}{\vspace{-0.8em}\end{mdframed}}


% Deisgn
\usepackage[labelfont=bf]{caption}
\usepackage[margin=0.6in]{geometry}
\usepackage{multicol}
\usepackage[skip=4pt, indent=0pt]{parskip}
\usepackage[normalem]{ulem}
\forestset{default}
\renewcommand\labelitemi{$\bullet$}
\usepackage{titlesec}
\titleformat{\section}[block]
{\fontsize{15}{15}}
{\sen \dotfill (\thesection) \she}
{0em}
{\MakeUppercase}
\usepackage{graphicx}
\graphicspath{ {./} }


% Hebrew initialzing
\usepackage[bidi=basic]{babel}
\PassOptionsToPackage{no-math}{fontspec}
\babelprovide[main, import]{hebrew}
\babelprovide[import]{english}
\babelfont[hebrew]{rm}{David CLM}
\babelfont[hebrew]{sf}{David CLM}
\babelfont[english]{tt}{Monaspace Xenon}
\usepackage[shortlabels]{enumitem}
\newlist{hebenum}{enumerate}{1}

% Language Shortcuts
\newcommand\en[1] {\begin{otherlanguage}{english}#1\end{otherlanguage}}
\newcommand\sen   {\begin{otherlanguage}{english}}
	\newcommand\she   {\end{otherlanguage}}
\newcommand\del   {$ \!\! $}
\newcommand\ttt[1]{\en{\footnotesize\texttt{#1}\normalsize}}

\newcommand\npage {\vfil {\hfil \textbf{\textit{המשך בעמוד הבא}}} \hfil \vfil \pagebreak}
\newcommand\ndoc  {\dotfill \\ \vfil {\begin{center} {\textbf{\textit{שחר פרץ, 2024}} \\ \scriptsize \textit{נוצר באמצעות תוכנה חופשית בלבד}} \end{center}} \vfil	}

\newcommand{\rn}[1]{
	\textup{\uppercase\expandafter{\romannumeral#1}}
}

\makeatletter
\newcommand{\skipitems}[1]{
	\addtocounter{\@enumctr}{#1}
}
\makeatother

%! ~~~ Math shortcuts ~~~

% Letters shortcuts
\newcommand\N     {\mathbb{N}}
\newcommand\Z     {\mathbb{Z}}
\newcommand\R     {\mathbb{R}}
\newcommand\Q     {\mathbb{Q}}
\newcommand\C     {\mathbb{C}}

\newcommand\ml    {\ell}
\newcommand\mj    {\jmath}
\newcommand\mi    {\imath}

\newcommand\powerset {\mathcal{P}}
\newcommand\ps    {\mathcal{P}}
\newcommand\pc    {\mathcal{P}}
\newcommand\ac    {\mathcal{A}}
\newcommand\bc    {\mathcal{B}}
\newcommand\cc    {\mathcal{C}}
\newcommand\dc    {\mathcal{D}}
\newcommand\ec    {\mathcal{E}}
\newcommand\fc    {\mathcal{F}}
\newcommand\nc    {\mathcal{N}}
\newcommand\sca   {\mathcal{S}} % \sc is already definded
\newcommand\rca   {\mathcal{R}} % \rc is already definded

\newcommand\Si    {\Sigma}

% Logic & sets shorcuts
\newcommand\siff  {\longleftrightarrow}
\newcommand\ssiff {\leftrightarrow}
\newcommand\so    {\longrightarrow}
\newcommand\sso   {\rightarrow}

\newcommand\epsi  {\epsilon}
\newcommand\vepsi {\varepsilon}
\newcommand\vphi  {\varphi}
\newcommand\Neven {\N_{\mathrm{even}}}
\newcommand\Nodd  {\N_{\mathrm{odd }}}
\newcommand\Zeven {\Z_{\mathrm{even}}}
\newcommand\Zodd  {\Z_{\mathrm{odd }}}
\newcommand\Np    {\N_+}

% Text Shortcuts
\newcommand\open  {\big(}
\newcommand\qopen {\quad\big(}
\newcommand\close {\big)}
\newcommand\also  {\text{, }}
\newcommand\defi  {\text{ definition}}
\newcommand\defis {\text{ definitions}}
\newcommand\given {\text{given }}
\newcommand\case  {\text{if }}
\newcommand\syx   {\text{ syntax}}
\newcommand\rle   {\text{ rule}}
\newcommand\other {\text{else}}
\newcommand\set   {\ell et \text{ }}
\newcommand\ans   {\mathit{Ans.}}

% Set theory shortcuts
\newcommand\ra    {\rangle}
\newcommand\la    {\langle}

\newcommand\oto   {\leftarrow}

\newcommand\QED   {\quad\quad\mathscr{Q.E.D.}\;\;\blacksquare}
\newcommand\QEF   {\quad\quad\mathscr{Q.E.F.}}
\newcommand\eQED  {\mathscr{Q.E.D.}\;\;\blacksquare}
\newcommand\eQEF  {\mathscr{Q.E.F.}}
\newcommand\jQED  {\mathscr{Q.E.D.}}

\newcommand\dom   {\mathrm{dom}}
\newcommand\Img   {\mathrm{Im}}
\newcommand\range {\mathrm{range}}

\newcommand\trio  {\triangle}

\newcommand\rc    {\right\rceil}
\newcommand\lc    {\left\lceil}
\newcommand\rf    {\right\rfloor}
\newcommand\lf    {\left\lfloor}

\newcommand\lex   {<_{lex}}

\newcommand\az    {\aleph_0}
\newcommand\uaz   {^{\aleph_0}}
\newcommand\al    {\aleph}
\newcommand\ual   {^\aleph}
\newcommand\taz   {2^{\aleph_0}}
\newcommand\utaz  { ^{\left (2^{\aleph_0} \right )}}
\newcommand\tal   {2^{\aleph}}
\newcommand\utal  { ^{\left (2^{\aleph} \right )}}
\newcommand\ttaz  {2^{\left (2^{\aleph_0}\right )}}

\newcommand\n     {$n$־יה\ }

% Math A&B shortcuts
\newcommand\logn  {\log n}
\newcommand\cosx  {\cos x}
\newcommand\cost  {\cos \theta}
\newcommand\sinx  {\sin x}
\newcommand\sint  {\sin \theta}
\newcommand\tanx  {\tan x}
\newcommand\tant  {\tan \theta}
\newcommand\sex   {\sec x}
\newcommand\sect  {\sec^2}
\newcommand\cotx  {\cot x}
\newcommand\cscx  {\csc x}
\newcommand\sinhx {\sinh x}
\newcommand\coshx {\cosh x}
\newcommand\tanhx {\tanh x}

\newcommand\seq   {\overset{!}{=}}
\newcommand\sqq   {\overset{?}{=}}
\newcommand\slh   {\overset{LH}{=}}
\newcommand\sle   {\overset{!}{\le}}
\newcommand\sge   {\overset{!}{\ge}}
\newcommand\sll   {\overset{!}{<}}
\newcommand\sgg   {\overset{!}{>}}

\newcommand\h     {\hat}
\newcommand\ve    {\vec}
\newcommand\lv    {\overrightarrow}
\newcommand\ol    {\overline}

\newcommand\mlcm  {\mathrm{lcm}}

\DeclareMathOperator{\sech}   {sech}
\DeclareMathOperator{\csch}   {csch}
\DeclareMathOperator{\arcsec} {arcsec}
\DeclareMathOperator{\arccot} {arcCot}
\DeclareMathOperator{\arccsc} {arcCsc}
\DeclareMathOperator{\arccosh}{arccosh}
\DeclareMathOperator{\arcsinh}{arcsinh}
\DeclareMathOperator{\arctanh}{arctanh}
\DeclareMathOperator{\arcsech}{arcsech}
\DeclareMathOperator{\arccsch}{arccsch}
\DeclareMathOperator{\arccoth}{arccoth} 

\newcommand\dx    {\,\mathrm{d}x}
\newcommand\dt    {\,\mathrm{d}t}
\newcommand\dtt   {\,\mathrm{d}\theta}
\newcommand\df    {\mathrm{d}f}
\newcommand\dfdx  {\diff{f}{x}}
\newcommand\dit   {\limhz \frac{f(x + h) - f(x)}{h}}

\newcommand\nt[1] {\frac{#1}{#1}}

\newcommand\limz  {\lim_{x \to 0}}
\newcommand\limxz {\lim_{x \to x_0}}
\newcommand\limi  {\lim_{x \to \infty}}
\newcommand\limh  {\lim_{x \to 0}}
\newcommand\limni {\lim_{x \to - \infty}}
\newcommand\limpmi{\lim_{x \to \pm \infty}}

\newcommand\ta    {\theta}
\newcommand\ap    {\alpha}

\renewcommand\inf {\infty}
\newcommand  \ninf{-\inf}

% Combinatorics shortcuts
\newcommand\sumnk     {\sum_{k = 0}^{n}}
\newcommand\sumni     {\sum_{i = 0}^{n}}
\newcommand\sumnko    {\sum_{k = 1}^{n}}
\newcommand\sumnio    {\sum_{i = 1}^{n}}
\newcommand\sumai     {\sum_{i = 1}^{n} A_i}
\newcommand\nsum[2]   {\reflectbox{\displaystyle\sum_{\reflectbox{\scriptsize$#1$}}^{\reflectbox{\scriptsize$#2$}}}}

\newcommand\bink      {\binom{n}{k}}
\newcommand\setn      {\{a_i\}^{2n}_{i = 1}}
\newcommand\setc[1]   {\{a_i\}^{#1}_{i = 1}}

\newcommand\cupain    {\bigcup_{i = 1}^{n} A_i}
\newcommand\cupai[1]  {\bigcup_{i = 1}^{#1} A_i}
\newcommand\cupiiai   {\bigcup_{i \in I} A_i}
\newcommand\capain    {\bigcap_{i = 1}^{n} A_i}
\newcommand\capai[1]  {\bigcap_{i = 1}^{#1} A_i}
\newcommand\capiiai   {\bigcap_{i \in I} A_i}

\newcommand\xot       {x_{1, 2}}
\newcommand\ano       {a_{n - 1}}
\newcommand\ant       {a_{n - 2}}

% Other shortcuts
\newcommand\tl    {\tilde}
\newcommand\op    {^{-1}}

\newcommand\sof[1]    {\left | #1 \right |}
\newcommand\cl [1]    {\left ( #1 \right )}
\newcommand\csb[1]    {\left [ #1 \right ]}

\newcommand\bs    {\blacksquare}

%! ~~~ Document ~~~

\author{שחר פרץ}
\title{מתמטיקה B $\sim$ עברי נגר $\sim$ משהו}
\date{6 ליולי 2024}

\begin{document}
	\maketitle
	\section{\en{Intergration By Parts (IBP)}}
	הגישה היחידה שיש לנו פרט לשיטת ההצבה. נובע מכלל המכפלה. 
	\[ (fg)' = f'g + fg' \implies f'g = fg' - (fg)' \implies \int f'(x)(gx)\dx = f(x)g(x)\dx - \int (f(x)g(x))'\dx \]	
	לדוגמה: 
	\[ \int  x\sinx \dx = \csb{\begin{cases}
			g(x) = x & f(x) =  - \cosx \\
			g'(x) = 1 & f'(x) = \sinx
	\end{cases}} = - x\cosx + \int \cosx \dx = - x\cosx + \sinx + C \]
	
	שיטת הרישום – udv. החוק: 
	\[ \int u\ d v = uv - \int v\ du \]
	ככה לרוב רושמים. 
	\begin{gather*}
		\int x^2e^x = \csb{\begin{cases}
				u = x^2, & v = e^x \\
				du = 2x \dx & dv = e6x \dx
		\end{cases}} = x^2e^x - \int 2xe^x \dx \\ 
		= \csb{\begin{cases}
				u = x & v = e^x \\
				du = \dx, & dv = e^x\dx
		\end{cases}} = (x^2 e^x) - 2\cl{xe^x - \int e^x \dx} = x^2e^x - 2xe^x +2e^x + C
	\end{gather*}
	דוגמה פשוטה בשביל להכניס עיזות לאינטגרל (הדבר הטוב ביותר בכל הזמנים). 
	\[ \int \ln x \dx = \int \ln x \cdot 1 \dx = \csb{\begin{cases}
			u = \ln x & v = x\\
			du =  \frac{\dx}{x} & dv = 1 \cdot \dx
	\end{cases}} = x \ln x - \int x \cdot \frac{\dx}{x} = x \ln x - x + C \]

תרגיל: 
\[ \int x^3 \ln x \dx = \csb{\begin{cases}
		u = \ln x & v = \frac{1}{4}x^4 \\
		du = \frac{\dx}{x} & dv = x^3 \cdot \dx
\end{cases}} = \frac{x^4\ln x}{4} - \int \frac{x^3}{4} = \frac{x^4\ln x}{4} - \frac{x^4}{16} + C \]
	וגם בעבור אינטגרל מסויים:
	\[ \int^b_a u \; dv = uv \Big|_a^b - \int^b_a v \; du \]
	הסבר לסימון: 
	\[ \int_1^3 x \dx = \frac{x^2}{2}\Big|^3_1 = \frac{3^2}{2}  - \frac{1^2}{2} \]
	דוגמה נוספת: (אותה ההצבה כמו קודם. אמור להסביר למה צריך להציב שם)
	\[ \int^5_2 \ln x \dx = \int^5_2 (x \ln x)' \dx - \int^5_2 x \cdot \frac{\dx}{x} = x \ln x \Big|^5_2 - x \big|^5_2 = 5 \ln5 - 5 - (2 \ln 2 - 2) \]
	
	ע"ש עצמי: 
	 \begin{align*}
	 	I &= \int e^x \sinx \dx = \csb{\begin{cases}
	 			u = \sinx & v = e^x \\
	 			du = \cosx \dx & dv = e^x\dx
	 	\end{cases}} = e^x\sinx - \int e^x \cos \dx \\
 	&= \csb{\begin{cases}
	 			u = \cosx & v = \dots \\
	 			du = \dots & dv = \dots
	 	\end{cases}} = e^x\sinx - \cl{e^x\cosx - \int - e^x\sinx \dx} \\
 		=e^x (\sinx - \cosx) - I
	 \end{align*}
	 נעביר אגפים. 
	 \[ 2I = e^x(\sinx - \cosx) \implies I = \frac{e^x(\sinx - \cosx)}{2} + C \]
	 אבל להציב ככה I זה טיפה בעייתי כי הם מוגדרים עד לכדי קבוע. אז כשמעבירים אגף יכול להיות שיש שני קבועים שונים. סה"כ הקבוע לא מענלם למרות שחיסרנו אותם. 
	 \section{\en{Partioal Fractions}}
	 "טריקים ושטיטיקים" שלא בהכרח תמיד עוזרים באינטגרלים אבל יכולים לעזור גם בהם. 
	 \[ \frac{7x - 7}{x^2 - x - 12} = \frac{7x - 7}{(x - 4)(x + 3)} \seq \frac{A}{x - 4} + \frac{B}{x + 3} = \frac{A(x + 3) + B(x - 4)}{x^2 - x - 12} = \frac{(A + B)x + (3A - 4B)}{x^2 - x - 12} \]
	 נמצא $A, B$ מתאימים. נקבל: 
	 \[ \begin{cases}
	 	A + B = 7 \\
	 	3A - 4B = -7
	 \end{cases} \implies A = 3, \ B = 4 \]
	 שזו מערכת משוואות שקל לפתור. סה"כ $\frac{7x -  7}{x^2 - x - 12}  = \frac{3}{x - 4} + \frac{4}{x + 3}$. הדבר הזה, עובד באופן כללי. באופן כללי, כל שבר נוכל לכתוב כסכום של שני שברים קטנים יותר. 
	 
	 נתבונן בשבר הבא: 
	 \[ \frac{x^2 + x - 2}{3x^3 - x^2 + 3x - 1} = \frac{x^2 + x - 2}{(x^2 + 1)(3x - 1)} \seq \frac{A}{3x - 1} + \frac{Bx + C}{x^2 + 1} = \frac{(A + 3B)x^2 (-B + 3C)x + (A - C)}{(x^2 - 1)(3x - 1)} \]
	 סה"כ: $A + 3B = 1, \ -B + 3C = 1, \ A - C = -2$. תשברו את הראש ותמצאו $A = -7/5, B = 4/5' C = 3/ 5$. קדימה, ננסח באופן כללי: 
	 
	 כל פונקציה רציונלית מהצורה להלן נוכל לכתוב כך: 
	 \[ \frac{P_{<m}(x)}{(a_1x + b_1) \cdot \dots \cdot (a_mx + b+m)} = \frac{A_1}{a_x + b_1} + \dots + \frac{A_m}{a_mx + b_m} \]
	 כאשר $P_i$ הוא פולינום ממעלה $i$. 
	 
	 הטבלה להלן תראה איזה איבר כל גורם שנוצר מהשורש למטה יביא: 
	 \begin{alignat}{9}
	 	&&&ax + b &&\to &&&\frac{A}{ax + b} \\
	 	&&&ax^2 + bx + c &&\to &&&\frac{Ax + B}{ax^2 + bx + c} \\
	 	&&&(ax + b)^k &&\to &&&\; \; \frac{A_1}{ax + b} + \frac{A_2}{(ax + b)^2} + \dots + \frac{A_k}{(ax + b)^k} \\
	 	&&&(ax^2 + bx + c)^k &&\to &&& \; \frac{A_1x + B_1}{ax^2 + bx + c} + \dots + \frac{A_kx + B_k}{(ax62 + bx + c)^k}
	 \end{alignat}
	 דוגמה: 
	 \[ \frac{2x + 4}{x^3 - 2x^2} = \frac{2x + 4}{x^2(x - 2)} = \seq = \frac{A}{x - 2} + \frac{B}{x} + \frac{C}{x^2} = \frac{(A + B)x^2 + (-2B + C)x - 2C}{x^3 - 2x^2} \implies \begin{cases}
	 	A + B = 0 & A = 2 \\
	 	-2B + C = 2 & B = -2 \\
	 	-2C = 4 & C = -2
	 \end{cases} \]
	 \textit{משפט. }כל פולינום ממשי אפשר לחלק לגורמים ממעלה ריבועית לכל היותר. זה למה עברי עצר בטבלה במעלה 2. 
	 
	 \textbf{"שיטת הכיסוי של הרדי או משהו כזה": }
	 \[ \frac{7x - 7}{x^2 - x - 12} = \frac{A}{x - 4} + \frac{B}{x + 3} \]
	 נכפול ב־$x - 4$: 
	 \[ \frac{7x - 7}{x + 3} = A + \frac{B(x - 4)}{x + 3} \overset{x = 4}{\longrightarrow} \frac{7 \cdot 5 - 7}{4 + 3} = A \]
	 אפשר גם לכפול ב־$(x - 3)$ ואז לקחת $x \to -3$. 
	 
	 זו סתם שיטה נוחה יותר לפתור את המשוואות האלו. 
	 \subsection{אז למה עשינו את זה}
	 כי אינטגרלים. 
	 עבור פונקציה $\frac{P(x)}{Q(x)}$, נעשה חילוק פולינומים ונקבל $p(x) + \frac{r(x)}{Q(x)}$. מעלת $r$ קטנה ממעלת $Q$, ונוכל לבצע חרא של שברים חלקיים ולקבל פולינום + שברים חלקיים. עכשיו נעשה אינטגרל. אין בעיה לעשות אינטגלים על פולינומים או על שברים חלקיים. 
	 
	 ש.ב.: תקראו על הצבת ויירשטראס (Wieirstrass). 
	 
	 \section{\en{L'Hôpital's rule (LH)}}
	 לופיטל קנה את הזכויות לכלל מברנולי. כאילו ליטרלי. הבעיה בגבולות מתחילה כאשר $f, g$ שואפים ל־0 ומנסים לחלק אותם אחד בשני. לכן, בינתיים נניח $\limxz f(x) = \limxz g(x) = 0$. 
	 \[ \limxz \frac{f(x)}{g(x)} = \limxz \frac{f(x) - f(x_0)}{g(x) - g(x_0)} = \limxz \frac{\frac{f(x) - f(x_0)}{x - x_0}}{\frac{g(x) - g(x_0)}{x - x_0}} \sqq \cl{\limxz \frac{f(x) - f(x_0)}{x - x_0}} \Bigg / \cl{\limxz \frac{g(x) - g(x_0)}{x - x_0}} = \frac{f'(x)}{g'(x)} \]
	 חוץ מההנחה שאפשר לפצל את הגבולות, "לא רימיתי בשום דבר" (עברי). 
	 
	 "הצורה הכי פורמלית שעדיין הגדרנו". יהיו $f, g$ פונקציות גזירות בסביבה בסביבה נקובה של $t$. נניח: 
	 \begin{enumerate}
	 	\item $\lim_{x \to t} f(x) = 0$ 
	 	\item $\lim_{x \to t} g(x) = 0$
	 	\item $g'(x) \neq 0$ בסביבה נקובה של $t$. 
	 	\item קיים הגבול $L = \lim_{x \to t} \frac{f'(x)}{g'(x)}$
	 \end{enumerate}
	 אם ארבעת התנאים להלן מתקיימים, אז קיים הגבול $L = \lim_{x \to t} \frac{f(x)}{g(x)}$ וגם $L = L'$. התנאי הרביעי מאוד חשוב. בד"כ שוכחים אותו. אפשר להשתמש בלופיטל כמה פעמים. תקף גם עבור קבולות חד־צדיים, ובפרט עבור $t = \pm \inf$. יש גם כלל לופיטל עבור $\frac{()}{\inf}$, כלומר כאשר אין תנאי 1, ותנאי 2 הופך ל־$\lim_{x \to t} g(x) = \pm \inf$. 
	 
	 דוגמאות: 
	 \[ \lim_{x \to 0} \frac{1 - \cosx}{\sinx^2} = \limz \frac{\sinx}{2x\cosx^2} = \frac{1}{2}\limz \frac{\sinx}{x} = \frac{1}{2} \]
	 יש לזה גבול אז השוויון בהתחלה היה תקין. 
	 
	 דוג': 
	 \begin{gather*}
	 	\limz = \frac{\sin(\sinx) - \sinx}{x^3} \overset{LH}{=} = \limz \frac{\cos(\sinx)\cosx - \cosx}{3x^2} = \frac{1}{3}\limz \frac{\cos(\sinx) -1 }{x^2} \slh \frac{1}{3}\limz \frac{-\sin(\sinx)\cosx}{2x} = \\ -\frac{1}{6}\limz \frac{(\sin(\sinx))}{x} \slh -\frac{1}{6} \limz\frac{\cos(\sinx)\cosx}{1} = -\frac{1}{6}
	 \end{gather*}
	 הבהרה: רק לאחר שראינו שהגבול קיים, בדיעבד נוכל לדעת שכל השוויונות שמבוססים על $LH$ הם תקינים. 
	 
	 "גבול שאתם צריכים לדעת אבל תנסו לחשב אותו עם לופיטל": 
	 \[ \limi \frac{\ln x}{x} \slh \limi \frac{1/x}{x} = 0 \]
	 שימו לב – \textbf{לא להשתמש בטיעונים מעגליים}. דוגמה: 
	 \[ \limxz = \frac{\sinx}{x} \slh \limz \frac{\cosx}{1} = 1 \]
	 בשביל לעשות את הגבול הזה ניאלצנו לגזור את $\sinx$, אך בשביל לעשות זאת ניאצלנו לחשב את הגבול. 
	 
	 
	 
	
\end{document}