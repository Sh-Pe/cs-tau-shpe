%! ~~~ Packages Setup ~~~ 
\documentclass[]{article}


% Math packages
\usepackage[usenames]{color}
\usepackage{forest}
\usepackage{ifxetex,ifluatex,amsmath,amssymb,mathrsfs,amsthm,witharrows,mathtools}
\WithArrowsOptions{displaystyle}
\renewcommand{\qedsymbol}{$\blacksquare$} % end proofs with \blacksquare. Overwrites the defualts. 
\usepackage{cancel,bm}
\usepackage[thinc]{esdiff}


% tikz
\usepackage{tikz}
\newcommand\sqw{1}
\newcommand\squ[4][1]{\fill[#4] (#2*\sqw,#3*\sqw) rectangle +(#1*\sqw,#1*\sqw);}


% code 
\usepackage{listings}
\usepackage{xcolor}

\definecolor{codegreen}{rgb}{0,0.35,0}
\definecolor{codegray}{rgb}{0.5,0.5,0.5}
\definecolor{codenumber}{rgb}{0.1,0.3,0.5}
\definecolor{codeblue}{rgb}{0,0,0.5}
\definecolor{codered}{rgb}{0.5,0.03,0.02}
\definecolor{codegray}{rgb}{0.96,0.96,0.96}

\lstdefinestyle{pythonstylesheet}{
	language=Python,
	emphstyle=\color{deepred},
	backgroundcolor=\color{codegray},
	keywordstyle=\color{deepblue}\bfseries\itshape,
	numberstyle=\scriptsize\color{codenumber},
	basicstyle=\ttfamily\footnotesize,
	commentstyle=\color{codegreen}\itshape,
	breakatwhitespace=false, 
	breaklines=true, 
	captionpos=b, 
	keepspaces=true, 
	numbers=left, 
	numbersep=5pt, 
	showspaces=false,                
	showstringspaces=false,
	showtabs=false, 
	tabsize=4, 
	morekeywords={as,assert,nonlocal,with,yield,self,True,False,None,AssertionError,ValueError,in,else},              % Add keywords here
	keywordstyle=\color{codeblue},
	emph={object,type,isinstance,copy,deepcopy,zip,enumerate,reversed,list,set,len,dict,tuple,print,range,xrange,append,execfile,real,imag,reduce,str,repr,__init__,__add__,__mul__,__div__,__sub__,__call__,__getitem__,__setitem__,__eq__,__ne__,__nonzero__,__rmul__,__radd__,__repr__,__str__,__get__,__truediv__,__pow__,__name__,__future__,__all__,},          % Custom highlighting
	emphstyle=\color{codered},
	stringstyle=\color{codegreen},
	showstringspaces=false,
	abovecaptionskip=0pt,belowcaptionskip =0pt,
	framextopmargin=-\topsep, 
}
\newcommand\pythonstyle{\lstset{pythonstylesheet}}
\newcommand\pyl[1]     {{\lstinline!#1!}}
\lstset{style=pythonstylesheet}

\usepackage[style=1,skipbelow=\topskip,skipabove=\topskip,framemethod=TikZ]{mdframed}
\definecolor{bggray}{rgb}{0.85, 0.85, 0.85}
\mdfsetup{leftmargin=0pt,rightmargin=0pt,innerleftmargin=15pt,backgroundcolor=codegray,middlelinewidth=0.5pt,skipabove=5pt,skipbelow=0pt,middlelinecolor=black,roundcorner=5}
\BeforeBeginEnvironment{lstlisting}{\begin{mdframed}\vspace{-0.4em}}
	\AfterEndEnvironment{lstlisting}{\vspace{-0.8em}\end{mdframed}}


% Deisgn
\usepackage[labelfont=bf]{caption}
\usepackage[margin=0.6in]{geometry}
\usepackage{multicol}
\usepackage[skip=4pt, indent=0pt]{parskip}
\usepackage[normalem]{ulem}
\forestset{default}
\renewcommand\labelitemi{$\bullet$}
\usepackage{titlesec}
\titleformat{\section}[block]
{\fontsize{15}{15}}
{\sen \dotfill (\thesection) \dotfill \she}
{0em}
{\MakeUppercase}
\usepackage{graphicx}
\graphicspath{ {./} }


% Hebrew initialzing
\usepackage[bidi=basic]{babel}
\PassOptionsToPackage{no-math}{fontspec}
\babelprovide[main, import, Alph=letters]{hebrew}
\babelprovide[import]{english}
\babelfont[hebrew]{rm}{David CLM}
\babelfont[hebrew]{sf}{David CLM}
\babelfont[english]{tt}{Monaspace Xenon}
\usepackage[shortlabels]{enumitem}
\newlist{hebenum}{enumerate}{1}

% Language Shortcuts
\newcommand\en[1] {\begin{otherlanguage}{english}#1\end{otherlanguage}}
\newcommand\sen   {\begin{otherlanguage}{english}}
	\newcommand\she   {\end{otherlanguage}}
\newcommand\del   {$ \!\! $}
\newcommand\ttt[1]{\en{\footnotesize\texttt{#1}\normalsize}}

\newcommand\npage {\vfil {\hfil \textbf{\textit{המשך בעמוד הבא}}} \hfil \vfil \pagebreak}
\newcommand\ndoc  {\dotfill \\ \vfil {\begin{center} {\textbf{\textit{שחר פרץ, 2024}} \\ \scriptsize \textit{אההה כותרת משנה}} \end{center}} \vfil	}

\newcommand{\rn}[1]{
	\textup{\uppercase\expandafter{\romannumeral#1}}
}

\makeatletter
\newcommand{\skipitems}[1]{
	\addtocounter{\@enumctr}{#1}
}
\makeatother

%! ~~~ Math shortcuts ~~~

% Letters shortcuts
\newcommand\N     {\mathbb{N}}
\newcommand\Z     {\mathbb{Z}}
\newcommand\R     {\mathbb{R}}
\newcommand\Q     {\mathbb{Q}}
\newcommand\C     {\mathbb{C}}

\newcommand\ml    {\ell}
\newcommand\mj    {\jmath}
\newcommand\mi    {\imath}

\newcommand\powerset {\mathcal{P}}
\newcommand\ps    {\mathcal{P}}
\newcommand\pc    {\mathcal{P}}
\newcommand\ac    {\mathcal{A}}
\newcommand\bc    {\mathcal{B}}
\newcommand\cc    {\mathcal{C}}
\newcommand\dc    {\mathcal{D}}
\newcommand\ec    {\mathcal{E}}
\newcommand\fc    {\mathcal{F}}
\newcommand\nc    {\mathcal{N}}
\newcommand\sca   {\mathcal{S}} % \sc is already definded
\newcommand\rca   {\mathcal{R}} % \rc is already definded

\newcommand\Si    {\Sigma}

% Logic & sets shorcuts
\newcommand\siff  {\longleftrightarrow}
\newcommand\ssiff {\leftrightarrow}
\newcommand\so    {\longrightarrow}
\newcommand\sso   {\rightarrow}

\newcommand\epsi  {\epsilon}
\newcommand\vepsi {\varepsilon}
\newcommand\vphi  {\varphi}
\newcommand\Neven {\N_{\mathrm{even}}}
\newcommand\Nodd  {\N_{\mathrm{odd }}}
\newcommand\Zeven {\Z_{\mathrm{even}}}
\newcommand\Zodd  {\Z_{\mathrm{odd }}}
\newcommand\Np    {\N_+}

% Text Shortcuts
\newcommand\open  {\big(}
\newcommand\qopen {\quad\big(}
\newcommand\close {\big)}
\newcommand\also  {\text{, }}
\newcommand\defi  {\text{ definition}}
\newcommand\defis {\text{ definitions}}
\newcommand\given {\text{given }}
\newcommand\case  {\text{if }}
\newcommand\syx   {\text{ syntax}}
\newcommand\rle   {\text{ rule}}
\newcommand\other {\text{else}}
\newcommand\set   {\ell et \text{ }}
\newcommand\ans   {\mathit{Ans.}}

% Set theory shortcuts
\newcommand\ra    {\rangle}
\newcommand\la    {\langle}

\newcommand\oto   {\leftarrow}

\newcommand\QED   {\quad\quad\mathscr{Q.E.D.}\;\;\blacksquare}
\newcommand\QEF   {\quad\quad\mathscr{Q.E.F.}}
\newcommand\eQED  {\mathscr{Q.E.D.}\;\;\blacksquare}
\newcommand\eQEF  {\mathscr{Q.E.F.}}
\newcommand\jQED  {\mathscr{Q.E.D.}}

\newcommand\dom   {\mathrm{dom}}
\newcommand\Img   {\mathrm{Im}}
\newcommand\range {\mathrm{range}}

\newcommand\trio  {\triangle}

\newcommand\rc    {\right\rceil}
\newcommand\lc    {\left\lceil}
\newcommand\rf    {\right\rfloor}
\newcommand\lf    {\left\lfloor}

\newcommand\lex   {<_{lex}}

\newcommand\az    {\aleph_0}
\newcommand\uaz   {^{\aleph_0}}
\newcommand\al    {\aleph}
\newcommand\ual   {^\aleph}
\newcommand\taz   {2^{\aleph_0}}
\newcommand\utaz  { ^{\left (2^{\aleph_0} \right )}}
\newcommand\tal   {2^{\aleph}}
\newcommand\utal  { ^{\left (2^{\aleph} \right )}}
\newcommand\ttaz  {2^{\left (2^{\aleph_0}\right )}}

\newcommand\n     {$n$־יה\ }

% Math A&B shortcuts
\newcommand\logn  {\log n}
\newcommand\logx  {\log x}
\newcommand\lnx   {\ln x}
\newcommand\cosx  {\cos x}
\newcommand\cost  {\cos \theta}
\newcommand\sinx  {\sin x}
\newcommand\sint  {\sin \theta}
\newcommand\tanx  {\tan x}
\newcommand\tant  {\tan \theta}
\newcommand\sex   {\sec x}
\newcommand\sect  {\sec^2}
\newcommand\cotx  {\cot x}
\newcommand\cscx  {\csc x}
\newcommand\sinhx {\sinh x}
\newcommand\coshx {\cosh x}
\newcommand\tanhx {\tanh x}

\newcommand\seq   {\overset{!}{=}}
\newcommand\slh   {\overset{LH}{=}}
\newcommand\sle   {\overset{!}{\le}}
\newcommand\sge   {\overset{!}{\ge}}
\newcommand\sll   {\overset{!}{<}}
\newcommand\sgg   {\overset{!}{>}}

\newcommand\h     {\hat}
\newcommand\ve    {\vec}
\newcommand\lv    {\overrightarrow}
\newcommand\ol    {\overline}

\newcommand\mlcm  {\mathrm{lcm}}

\DeclareMathOperator{\sech}   {sech}
\DeclareMathOperator{\csch}   {csch}
\DeclareMathOperator{\arcsec} {arcsec}
\DeclareMathOperator{\arccot} {arcCot}
\DeclareMathOperator{\arccsc} {arcCsc}
\DeclareMathOperator{\arccosh}{arccosh}
\DeclareMathOperator{\arcsinh}{arcsinh}
\DeclareMathOperator{\arctanh}{arctanh}
\DeclareMathOperator{\arcsech}{arcsech}
\DeclareMathOperator{\arccsch}{arccsch}
\DeclareMathOperator{\arccoth}{arccoth} 

\newcommand\dx    {\,\mathrm{d}x}
\newcommand\dt    {\,\mathrm{d}t}
\newcommand\dtt   {\,\mathrm{d}\theta}
\newcommand\du    {\,\mathrm{d}u}
\newcommand\dv    {\,\mathrm{d}v}
\newcommand\df    {\mathrm{d}f}
\newcommand\dfdx  {\diff{f}{x}}
\newcommand\dit   {\limhz \frac{f(x + h) - f(x)}{h}}

\newcommand\pu[3]{\csb{\begin{aligned}
			&u = #1 \quad u' = #2 \\
			&\du = #2 #3
	\end{aligned}}}
\newcommand\pus[2]{\csb{\begin{aligned}
			u &= #1 \\
			\du &= #2
\end{aligned}}}
\newcommand\ptt[3]{\csb{\begin{aligned}
			&\theta = #1 \quad \theta' = #2 \\
			&\dtt = #2 #3
\end{aligned}}}
\newcommand\ptts[2]{\csb{\begin{aligned}
\theta &= #1\\
\dtt &= #2
\end{aligned}}}
\newcommand\pt[3]{\csb{\begin{aligned}
			&t = #1 \quad t' = #2 \\
			&\dt = #2 \dx #3
\end{aligned}}}
\newcommand\pts[2]{\csb{\begin{aligned}
t &= #1 \quad \\
\dt &= #2
\end{aligned}}}
\newcommand\ptu[3]{\csb{\begin{aligned}
			&t = #1 \quad t' = #2 \\
			&\dt = #2 \du #3
\end{aligned}}}
\newcommand\px[3]{\csb{\begin{aligned}
&x = #1 \quad x' = #2 \\
&\dx = #2 #3
\end{aligned}}}
\newcommand\pxs[2]{\csb{\begin{aligned}
x &= #1\\
\dx &= #2
\end{aligned}}}

\newcommand\udv[4]{\csb{\begin{aligned}
			u &= #1  \ & v &= #3 \\
			du &= #2 \ & dv &= #4
\end{aligned}}}

\newcommand\nt[1] {\frac{#1}{#1}}

\newcommand\limz  {\lim_{x \to 0}}
\newcommand\limxz {\lim_{x \to x_0}}
\newcommand\limi  {\lim_{x \to \infty}}
\newcommand\limh  {\lim_{x \to 0}}
\newcommand\limni {\lim_{x \to - \infty}}
\newcommand\limpmi{\lim_{x \to \pm \infty}}

\newcommand\ta    {\theta}
\newcommand\ap    {\alpha}

\renewcommand\inf {\infty}
\newcommand  \ninf{-\inf}

% Combinatorics shortcuts
\newcommand\sumnk     {\sum_{k = 0}^{n}}
\newcommand\sumni     {\sum_{i = 0}^{n}}
\newcommand\sumnko    {\sum_{k = 1}^{n}}
\newcommand\sumnio    {\sum_{i = 1}^{n}}
\newcommand\sumai     {\sum_{i = 1}^{n} A_i}
\newcommand\nsum[2]   {\reflectbox{\displaystyle\sum_{\reflectbox{\scriptsize$#1$}}^{\reflectbox{\scriptsize$#2$}}}}

\newcommand\bink      {\binom{n}{k}}
\newcommand\setn      {\{a_i\}^{2n}_{i = 1}}
\newcommand\setc[1]   {\{a_i\}^{#1}_{i = 1}}

\newcommand\cupain    {\bigcup_{i = 1}^{n} A_i}
\newcommand\cupai[1]  {\bigcup_{i = 1}^{#1} A_i}
\newcommand\cupiiai   {\bigcup_{i \in I} A_i}
\newcommand\capain    {\bigcap_{i = 1}^{n} A_i}
\newcommand\capai[1]  {\bigcap_{i = 1}^{#1} A_i}
\newcommand\capiiai   {\bigcap_{i \in I} A_i}

\newcommand\xot       {x_{1, 2}}
\newcommand\ano       {a_{n - 1}}
\newcommand\ant       {a_{n - 2}}

% Other shortcuts
\newcommand\tl    {\tilde}
\newcommand\op    {^{-1}}

\newcommand\sof[1]    {\left | #1 \right |}
\newcommand\cl [1]    {\left ( #1 \right )}
\newcommand\csb[1]    {\left [ #1 \right ]}

\newcommand\bs    {\blacksquare}

%! ~~~ Document ~~~

\author{שחר פרץ}
\title{\textit{מתמטיקה B} $\sim$ \textit{תרגיל בית 4}}
\begin{document}
	\maketitle
	
	\section{}
	\begin{enumerate}
		\item[2.]
			\[ \int \cos^3x \sinx \dx = \pu{\cosx}{-\sinx}{\dx} = \int -u^3 = -\frac{1}{4}u^4 = \bm{-\frac{\cos^4x}{4} + C} \]
		\item[4.] 
			\[ \int \sqrt{\frac{\arcsin x}{1 - x^2}} \dx = \int \sqrt{\arcsin x} \arcsin' \dx = \ptt{\arcsin x}{\arcsin'}{\dx} = \int \sqrt\theta \dtt = \frac{2}{3}\theta^{1.5} = \bm{\frac{\arcsin^{1.5} x}{1.5} + C} \]
		\item[6.]
			\[ \int \frac{\ln^2x}{x} \dx = \pu{\lnx}{\tfrac{1}{x}}{\dx} = \int u^2 \du = \frac{1}{3}u^3 = \bm{\frac{\ln^3x}{3} + C} \]
		\item[8.]
			\begin{multline*}
				\int \frac{\dx}{\sqrt x + \sqrt[3]{x}} = \pu{x^{\frac{1}{6}}}{\tfrac{1}{6}x^{-\frac{5}{6}}}{\dx \quad \dx = 6u^5\du } = \int \frac{6u^5\du}{u^3 + u^2} = \int \frac{\cancel{u^2}6u^3\du}{\cancel{u^2}(1 + u)} = \ptu{u + 1}{u}{} \\
				 = \frac{6t^2\dt}{t} = 6 \int t \dt = 3t^2 = 3(u + 1)^2 = 3u^2 + 6u + 1 = \bm{3\sqrt[3]{x} + 6\sqrt[6]{x} + 1 + C}
			\end{multline*}
		\item[10.]
			\begin{multline*}
				\int x^3(3x^2 - 1)^{15}\dx = \px{\tfrac{1}{\sqrt3}\sint}{\tfrac{1}{\sqrt3\cos t}}{\dt} = \int \frac{1}{9\sqrt3}\sin^3 t \cdot (\sin^2 - 1)^{15} \frac{1}{\sqrt3}\cos t\dt = \int 27\op \sin^3 t \cos^{31}t \dt \\
				= \ptt{\sin t}{\cos t}{\dt} = \int 27\op \ta^3 \cos^{30}(\arcsin \ta)\dtt = \frac{1}{27} \int \ta^3(1 - \sin^2\arcsin\ta)^{15}\dtt = \frac{1}{27} \int \ta^3(1 - \ta^2)^{15}\dtt \\
				= \int \ta^5((1 - \ta^2))^5\dtt = \csb{\begin{aligned}
						u = 1 - \ta^2 & \quad x = \sqrt{1 - u} \\
						\du = 2\ta\dtt
				\end{aligned}} = \int u^5(1 - u)^2 \; 0.5\du  = \frac{1}{2}\int u^7 - \int u^6 + \frac{1}{2} \int u^5 \\
				= \frac{u^8}{14} - \frac{u^6}{6} + \frac{u^5}{10} + C = \frac{(1 - \ta^2)^8}{14} - \frac{(1 - \ta^2)^6}{6} + \frac{(1 - \ta^2)^5}{10} + C = \frac{\cos^{16}t}{14} - \frac{\cos^{12}t}{6} + \frac{\cos^{10}t}{10} +C \\
				= \bm{\frac{\cos^{16}\cl{3^{-0.5}x}}{14} - \frac{\cos^{12}\cl{3^{-0.5}x}}{6} + \frac{\cos^{10}\cl{3^{-0.5}x}}{10} + C}
			\end{multline*}
		\item[12.]
			\[ \int \frac{x}{(x + 3)^{\frac{1}{5}}}\dx = \pus{x + 3}{\dx} = \int \frac{u - 3}{\sqrt[5]{u}}\du = \int u^{\frac{4}{5}}\du - 3\int u^{-\frac{1}{5}}\du = \frac{5}{9}u^{1.8} - 3.75u^{\frac{4}{5}} = \bm{\frac{5}{9}(x + 3)^{1.8} - 3.75(x + 3)^{\frac{4}{5}} + C} \]
	\end{enumerate}
	\section{}
	\begin{enumerate}
		\item[2.]
		\[ \int_2^{23} \cos^3x \sinx = -\frac{\cos^4 2}{4} + \frac{\cos^4 23}{4} \approx  \]
		\item[6.]
		\[ \int_6^{19} \frac{\ln^2x}{x} \dx = \frac{\ln^3x}{3} \Bigg\lvert^{19}_{6} \approx 8.50915 - 1.9174 = 6.5917 \]
		\item[10.]
		\[ \int_{10}^{15} x^3(3x^2 - 1)^{15} = \frac{\cos^{16}\cl{3^{-0.5}x}}{14} - \frac{\cos^{12}\cl{3^{-0.5}x}}{6} + \frac{\cos^{10}\cl{3^{-0.5}x}}{10} \Bigg\lvert^{15}_{10} \approx 0.0009 - 0.0012 = -0.00029 \]
		\item[12.]
		\[ \int_{12}^{13} \frac{x}{(x + 3)^{\frac{1}{5}}}\dx = \frac{5}{9}(x + 3)^{1.8} - 3.75(x + 3)^{\frac{4}{5}} \Bigg\lvert_{12}^{13} \approx 47.2243 - 39.9995 = 7.2248 \]
	\end{enumerate}
	\section{}
	\begin{enumerate}[a.]
		\item 
			\begin{multline*}
				\int \frac{\sqrt{25\ta^2 - 4}}{\ta}\dx = \ptt{0.4\sinhx}{0.4\coshx}{\dtt} = \int \frac{\sqrt{4(6.25 \cdot 0.4^2\sinh^2x - 1)}}{0.4\sinhx} 0.4\coshx\dtt \\
				= \int \frac{\cancel{0.4}\sqrt2\sqrt{\sinh^2x - 1}}{\cancel{0.4}\sinhx}\coshx = \sqrt{2}\coshx \frac{\coshx}{\sinhx} = \bm{\sqrt2 \coshx \coth x}
			\end{multline*}
		\item למען הנוחות, נגדיר $a = 1 + \frac{3}{\sqrt2}$: 
			\begin{multline*}
				\int \frac{x}{\sqrt{2x^2 - 4x - 7}} \dx = \frac{x}{\sqrt{\cl{x - 1 - \frac{3}{\sqrt2}}\cl{x - 1 + \frac{3}{\sqrt2}}}} \dt = \pt{x - 1}{1}{} = \int \frac{t + 1}{t^2 + a^2}\dt \\
				= \int \frac{1}{t^2 + a^2} \dt + \int \frac{t}{t^2 + a^2} \dt = \frac{1}{a}\arctan\cl{\frac{x}{a}} + \int \frac{t}{t^2 + a^2}
			\end{multline*}
			נפתור את האינטגרל שנותרנו עימו בנפרד: 
			\[ \int \frac{t}{t^2 + a^2} \dt = \udv{t}{1}{\arctan t}{\tfrac{1}{t^2 + a^2}} = t \arctan t - \int \arctan t \dt \]
			כאשר האינטגרל של $\arctan$: 
			\begin{multline*}
				\int \arctan x \dx = \pxs{\tant\dx}{\tfrac{1}{\cos^2\ta}\dtt} = \int \arctan \tant \cdot \frac{\dtt}{\cos^2 \ta} = \int \frac{\ta \dtt}{\cos^2\ta} = \udv{\ta}{1}{\tan\ta}{\sec^2\ta} \\
				= \ta \tan \ta - \int \frac{\sin\ta}{\cos\ta}\dtt = \pts{\cos \ta}{-\sin \ta \dtt} = \ta \tan \ta  - \underbrace{\int -\frac{1}{t} \dt}_{-\ln |t|} = \ta \tan \ta + \ln |\cos \ta| + C \\
				= \arctan x \cdot (\tan \arctan x) + \ln (\cos(\arctan x)) + C = x \arctan x + \ln \cl{\tfrac{1}{\sqrt{1 + x^2}}} + C = x\arctan x - 0.5 \ln (1 + x^2) + C
			\end{multline*}
			זאת כי נזכר שהוכח בשיעורי בית 2 כי $\arctan' = \cos^2(\arctan) = \frac{1}{x^2 + 1}$, כלומר $\cos(\arctan x) = \frac{1}{\sqrt{x^2 + 1}}$. סה"כ, הראנו כי: 
			\[ \int \frac{t}{t^2 + a^2} = \cancel{+\ t \arctan t - t \arctan t} - 0.5\ln(1 + t^2) \]
			ניזכר למה עשינו את זה מלכתחילה, ונציב באינטגרל המקורי: 
			\begin{multline*}
				\cdots = a\op \arctan\cl{\frac{t}{a}} - 0.5\ln(1 + t^2) = \cl{1 + \frac{3}{\sqrt2}}\op \arctan \cl{\frac{(3 + \sqrt2)(t - 1)}{\sqrt2}} - 0.5 \ln (1 + (t - 1)^2) \\
				= \bm{\frac{\sqrt 2}{\sqrt 2 + 3} \arctan \cl{\frac{(1.2 + \frac{\sqrt8}{5})(\sinh\ta - 1))}{\sqrt2}} - 0.5\ln(0.16\sinh^2\ta - 0.8\sinh\ta + 2)}
			\end{multline*}
		\item 
			\begin{multline*}
				\int e^{4x}\sqrt{1 + e^{2x}} = \pts{e^{x}}{e^{x}\dx} = \int t^3\sqrt{1 + t^2} = \pts{\tant}{\sec^2\dt} = \int \tan^3 \cdot \overbrace{\sqrt{1 + \tan^2 \ta}}^{\sec\ta} \sec^2x \dtt = \int \tan^3\ta\sec^3\ta \dtt \\
				= \int (1 - \sec^2\ta) \sec^2\ta \tan\ta\sec\ta \dtt \pus{\sec \ta}{\sect\tant\dtt} = \int (1 - u^2)u^2 \du = \int u^2 - \int u^4 = \frac{u^3}{3} - \frac{u^5}{5} + C = \frac{\sec^3 \ta}{3} - \frac{\sec^5\ta}{5} +C \\
				= \frac{\sec^3(\arctan t)}{3} - \frac{\sec^5(\arctan t)}{5} + C = ((1 - t^2)^{3 / 2}3\op - (1 - t^2)^{5 / 2}5\op) + C = \bm{(1 - t^2)^{1.5}3\op + (1 - t^2)^{2.5}5\op + C}
			\end{multline*}
		\end{enumerate}
		
		\ndoc
\end{document}