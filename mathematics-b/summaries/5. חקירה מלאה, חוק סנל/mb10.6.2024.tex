\documentclass[]{article}

% Math packages
\usepackage[usenames]{color}
\usepackage{forest}
\usepackage{ifxetex,ifluatex,amsmath,amssymb,mathrsfs,amsthm,witharrows,mathtools}
\WithArrowsOptions{displaystyle}
\renewcommand{\qedsymbol}{$\blacksquare$} % end proofs with \blacksquare. Overwrites the defualts. 
\usepackage{cancel,bm}

% tikz
\usepackage{tikz}
\newcommand\sqw{1}
\newcommand\squ[4][1]{\fill[#4] (#2*\sqw,#3*\sqw) rectangle +(#1*\sqw,#1*\sqw);}


% code 
\usepackage{listings}
\usepackage{xcolor}

\definecolor{codegreen}{rgb}{0,0.35,0}
\definecolor{codegray}{rgb}{0.5,0.5,0.5}
\definecolor{codenumber}{rgb}{0.1,0.3,0.5}
\definecolor{deepblue}{rgb}{0,0,0.5}
\definecolor{deepred}{rgb}{0.5,0.03,0.02}

\lstdefinestyle{pythonstylesheet}{
	language=Python,
	morekeywords={}
	emphstyle=\color{deepred},
	backgroundcolor=\color{white},   
	commentstyle=\color{codegreen}\itshape,
	keywordstyle=\color{deepblue}\bfseries\itshape,
	numberstyle=\tiny\color{codenumber},
	basicstyle=\ttfamily\footnotesize,
	breakatwhitespace=false, 
	breaklines=true, 
	captionpos=b, 
	keepspaces=true, 
	numbers=left, 
	numbersep=5pt, 
	showspaces=false,                
	showstringspaces=false,
	showtabs=false, 
	tabsize=2, 
	morekeywords={object,type,isinstance,copy,deepcopy,zip,enumerate,reversed,list,set,len,dict,tuple,range,xrange,append,execfile,real,imag,reduce,str,repr},              % Add keywords here
	keywordstyle=\color{deepblue},
	emph={__init__,__add__,__mul__,__div__,__sub__,__call__,__getitem__,__setitem__,__eq__,__ne__,__nonzero__,__rmul__,__radd__,__repr__,__str__,__get__,__truediv__,__pow__,__name__,__future__,__all__,as,assert,nonlocal,with,yield,self,True,False,None},          % Custom highlighting
	emphstyle=\color{deepred},
	stringstyle=\color{deepgreen},
	showstringspaces=false
}
\newcommand\pythonstyle{\lstset{pythonstylesheet}}
\newcommand\pyl[1]     {{\pythonstyle\lstinline!#1!}}
\lstset{style=pythonstylesheet}


% Deisgn
\usepackage[labelfont=bf]{caption}
\usepackage[margin=0.6in]{geometry}
\usepackage{multicol}
\usepackage[skip=4pt, indent=0pt]{parskip}
\usepackage[normalem]{ulem}
\forestset{default}
\renewcommand\labelitemi{$\bullet$}
\usepackage{titlesec}
%\titleformat{\section}[block]
%	{\fontsize{15}{15}}
%	{\sen \dotfill \, \!\!\! \thesection \,\! \dotfill \she}
%	{1em}
%	{\MakeUppercase}

% Hebrew initialzing
\usepackage{polyglossia}
\setmainlanguage{hebrew}
\setotherlanguage{english}
\newfontfamily\hebrewfont[Script=Hebrew, Ligatures=TeX]{David CLM}
\usepackage[shortlabels]{enumitem}
\newlist{hebenum}{enumerate}{1}
\setlist[hebenum,1]{
	labelindent=\parindent,
	label={{\hebrewfont{\protect\hebrewnumeral{\value{hebenumi}}}}.}
}

% Language Shortcuts
\newcommand\en[1] {\selectlanguage{english}#1\selectlanguage{hebrew}}
\newcommand\sen   {\selectlanguage{english}}
\newcommand\she   {\selectlanguage{hebrew}}
\newcommand\del   {$ \!\! $}
\newcommand\ttt[1]{\en{\texttt{#1}}}

\newcommand\npage {\vfil {\hfil \textbf{\textit{המשך בעמוד הבא}}} \hfil \vfil}
\newcommand\ndoc  {\dotfill \\ \vfil \hfil \textbf{\textit{שחר פרץ, 2024}} \hfil \vfil}

\newcommand{\rn}[1]{
	\textup{\uppercase\expandafter{\romannumeral#1}}
}


%! ~~~ Math shortcuts ~~~

% Letters shortcuts
\newcommand\N     {\mathbb{N}}
\newcommand\Z     {\mathbb{Z}}
\newcommand\R     {\mathbb{R}}
\newcommand\Q     {\mathbb{Q}}
\newcommand\C     {\mathbb{C}}

\newcommand\ml    {\ell}
\newcommand\mj    {\jmath}
\newcommand\mi    {\imath}

\newcommand\powerset {\mathcal{P}}
\newcommand\ps    {\mathcal{P}}
\newcommand\pc    {\mathcal{P}}
\newcommand\ac    {\mathcal{A}}
\newcommand\bc    {\mathcal{B}}
\newcommand\cc    {\mathcal{C}}
\newcommand\dc    {\mathcal{D}}
\newcommand\ec    {\mathcal{E}}
\newcommand\fc    {\mathcal{F}}
\newcommand\nc    {\mathcal{N}}
\newcommand\kc    {\mathcal{K}}
\newcommand\sca   {\mathcal{S}} % \sc is already definded
\newcommand\rca   {\mathcal{R}} % \rc is already definded

\newcommand\Si    {\Sigma}

% Logic & sets shorcuts
\newcommand\siff  {\longleftrightarrow}
\newcommand\ssiff {\leftrightarrow}
\newcommand\so    {\longrightarrow}
\newcommand\sso   {\rightarrow}

\newcommand\epsi  {\epsilon}
\newcommand\vepsi {\varepsilon}
\newcommand\vphi  {\varphi}
\newcommand\Neven {\N_{\mathrm{even}}}
\newcommand\Nodd  {\N_{\mathrm{odd }}}
\newcommand\Zeven {\Z_{\mathrm{even}}}
\newcommand\Zodd  {\Z_{\mathrm{odd }}}
\newcommand\Np    {\N_+}

% Text Shortcuts
\newcommand\open  {\big(}
\newcommand\qopen {\quad\big(}
\newcommand\close {\big)}
\newcommand\also  {\text{, }}
\newcommand\defi  {\text{ definition}}
\newcommand\defis {\text{ definitions}}
\newcommand\given {\text{given }}
\newcommand\case  {\text{if }}
\newcommand\syx   {\text{ syntax}}
\newcommand\rle   {\text{ rule}}
\newcommand\other {\text{else}}
\newcommand\set   {\ell et \text{ }}
\newcommand\ans   {\mathit{Ans.}}

% Set theory shortcuts
\newcommand\ra    {\rangle}
\newcommand\la    {\langle}

\newcommand\oto   {\leftarrow}

\newcommand\QED   {\quad\quad\mathscr{Q.E.D.}\;\;\blacksquare}
\newcommand\QEF   {\quad\quad\mathscr{Q.E.F.}}
\newcommand\eQED  {\mathscr{Q.E.D.}\;\;\blacksquare}
\newcommand\eQEF  {\mathscr{Q.E.F.}}
\newcommand\jQED  {\mathscr{Q.E.D.}}

\newcommand\dom   {\text{dom}}
\newcommand\Img   {\text{Im}}
\newcommand\range {\text{range}}

\newcommand\trio  {\triangle}

\newcommand\rc    {\right\rceil}
\newcommand\lc    {\left\lceil}
\newcommand\rf    {\right\rfloor}
\newcommand\lf    {\left\lfloor}

\newcommand\lex   {<_{lex}}

\newcommand\az    {\aleph_0}
\newcommand\uaz   {^{\aleph_0}}
\newcommand\al    {\aleph}
\newcommand\ual   {^\aleph}
\newcommand\taz   {2^{\aleph_0}}
\newcommand\utaz  { ^{\left (2^{\aleph_0} \right )}}
\newcommand\tal   {2^{\aleph}}
\newcommand\utal  { ^{\left (2^{\aleph} \right )}}
\newcommand\ttaz  {2^{\left (2^{\aleph_0}\right )}}

\newcommand\n     {$n$־יה\ }

% Math A&B shortcuts
\newcommand\logn  {\log n}
\newcommand\cosx  {\cos x}
\newcommand\cost  {\cos \theta}
\newcommand\sinx  {\sin x}
\newcommand\sint  {\sin \theta}
\newcommand\tanx  {\tan x}
\newcommand\tant  {\tan \theta}
\newcommand\dx    {\,\mathrm{d}x}

\newcommand\seq   {\overset{!}{=}}
\newcommand\sle   {\overset{!}{\le}}
\newcommand\sge   {\overset{!}{\ge}}
\newcommand\sll   {\overset{!}{<}}
\newcommand\sgg   {\overset{!}{>}}

\newcommand\h     {\hat}
\newcommand\ve    {\vec}
\newcommand\lv    {\overrightarrow}
\newcommand\ol    {\overline}

\newcommand\mlcm  {\mathrm{lcm}}

\newcommand\limz  {\lim_{x \to 0}}
\newcommand\limxz {\lim_{x \to x_0}}
\newcommand\limi  {\lim_{x \to \infty}}
\newcommand\limni {\lim_{x \to - \infty}}
\newcommand\limpmi{\lim_{x \to \pm \infty}}

\newcommand\ta    {\theta}
\newcommand\ap    {\alpha}

\renewcommand\inf {\infty}
\newcommand  \ninf{-\inf}

% Combinatorics shortcuts
\newcommand\sumnk     {\sum_{k = 0}^{n}}
\newcommand\sumni     {\sum_{i = 0}^{n}}
\newcommand\sumnko    {\sum_{k = 1}^{n}}
\newcommand\sumnio    {\sum_{i = 1}^{n}}
\newcommand\sumai     {\sum_{i = 1}^{n} A_i}
\newcommand\nsum[2]   {\reflectbox{\displaystyle\sum_{\reflectbox{\scriptsize$#1$}}^{\reflectbox{\scriptsize$#2$}}}}

\newcommand\bink      {\binom{n}{k}}

\newcommand\cupain    {\bigcup_{i = 1}^{n} A_i}
\newcommand\cupai[1]  {\bigcup_{i = 1}^{#1} A_i}
\newcommand\cupiiai   {\bigcup_{i \in I} A_i}

\newcommand\sof[1]    {\left | #1 \right |}
\newcommand\cl [1]    {\left ( #1 \right )}

\newcommand\xot       {x_{1, 2}}
\newcommand\ano       {a_{n - 1}}
\newcommand\ant       {a_{n - 2}}

% Other shortcuts
\newcommand\tl    {\tilde}
\newcommand\op    {^{-1}}

\newcommand\bs    {\blacksquare}

%! ~~~ Document ~~~

\author{שחר פרץ}
\title{מתמטיקה B $\sim$ עברי נגר $\sim$ חקירה $\sim$ סיכום חלקי}
\date{10 ליוני 2024}

\begin{document}
	\maketitle
	
	\section{משהו קיצוני}
	\textbf{נק' קיצון (לוקאלית): }מינ'/מקס'
	\textbf{נק' סטציונרית: }$f'(x_0) = 0$
	יתקיים קיצון גורר סטציונרית, אך לא להפך. לדוגמה: 
	\[ f(x) = x^{3}, \ f'(0) \]
	עוד דרך לבדוק את סוג הקיצון, הוא לבדוק בנגזרת השנייה. אם $f'(t) = 0 \land f''(t) > 0$, אז ב־$t$ נוכל לדעת שהנגזרת השנייה עולה, וסה"כ הנגזרת תהפוך משלילית לחיובית – כלומר, הפונקציה תהיה בעל מינימום לוקאלי ב־$t$. באופן דומה, אם $f''(t) < 0$ וגם $f'(t) = 0$ נקבל מקסימום לוקראלי. אלו תנאיים מספיקים, אך לא הכרחיים. 
	
	נוכל להתבונן גם בנקודות סטצ' ולא קיצון, בהן יתקיים $f'(t) = 0 \land f''(t)$, אך גם זה לא אמ"מ. לדוגמה, $g(x) = x^{4}$ אך $g'(0) = 0 \land g''(0) = 0$ למרות שיש ריצון בנקודה. כלומר, גם התנאים האלו הכרחיים אך לא מספיקים או הפוך או משהו כזה. 
	
	נק' קיצון לוקאלית של $f'$ היא כזו שבה $f''$ משנה סימן $=$ נק' פיתול. הנגזרת השנייה מתארת כמה מהר הפונקציה "מתעקמת". באנגלית, פיתול זה inflection \del. 
	
	סטצ' + פיתול = עוקף – saddle \del. (לא כל נקודות הפיתול הן סטציונריות). 
	
	\subsection{סיכום}
	\begin{itemize}
		\item סטצ' $\iff$ $f'(t) = 0$
		\item קיצון לוקאלי $\iff$ סטצ' $\land$ הנגזרת משנה סימן (בהתאם לסוג הקיצון). $\implies$ לנגזרת השנייה סימן מתאים בנק'. 
		\item עוקף $\iff$ סטצ' $\land$ לא קיצון $\iff$ סטצ' $\land$ הנגזרת לא משנה סימן $\impliedby$ $f'' = 0$
	\end{itemize}
	\subsection{הערות נוספות}
	
	\begin{itemize}
		\item אם הפונקציה לא גזירה בנק' – צריך לבדוק ידנית. 
		לדוגמה: $f(x) = \sqrt{|x|}$, שלא גזירה בנק' המינימום הלוקאלי $f(0)$. 
		\item 	קצוות התחום – אוטומאטית קיצון. בדיקה ידנית גם־כן. לדוגמה: $g(x) = x, x \in [3, 5]$. יתקיים $f'(x) = 1$. כלומר אין מינ/מקס לוקאלי, אך קצוות התחום יהיו גם מינ/מקס לוקאלי. בעבור $g(x) = x, x \in (3, 5)$ אין שום מינ'/מקס – בכל נקודה שבה היא מוגדרת, קיים ערך אחר עבורו היא מוגדרת ויותר גדולה. 
		\item 	קיצון גלובאלי – להשוות בין הלוקאלים + קצוות התחום (כולל ב־$\pm \inf$ + אסימפטוטות)
	\end{itemize}
	
	\section{קופסא}
	<הכנס כאן ציור שצילמתי, כתבו לי אם אתם צריכים>. ננסה למצוא את הנפח המקסימלי. 
	נקבל: 
	\begin{align*}
		V(\ml) &= \ml \cdot (4 - 2\ml)(3 - 2)\\
		V'(\ell) &= 12\ell^2 - 28 \ml + 12 \\
		V(\ml) = 0 &\iff \ell = \frac{7 + \sqrt{13}}{6} =: a_\pm
	\end{align*}
	הנקודות הללו, עלולות להיות ריצון. תחום הפונקציה: $0 \le \ml \le \frac{3}{2}$ מתוך הגדרת התיבה. הוא גם ישלול את אחת מהתשובות. נתבונן בפרבולה מחייכת ונמצא כי $a_-$ מקס' לוקאלי. 
	
	בשביל למצוא מקס' גלובלי נתבונן בערכה בשני קצוותיה. נחשב ונמצא ששניה שווים ל־$0$ וגם מינימום, כלומר $\ell = a_-$ יהיה המינימום המוחלט. 
	\section{חקירה}
	\textbf{תזכורת: }חקירה, כוללת: 
	\begin{enumerate}
		\item תחום הגדרה
		\item נק' חיתוך עם הצירים 
		\item תחומי חיוביות/שליליות
		\item מציאת נקודות סטציונריות, וסיווגן, וקיצון בקצוות התחום
		\item תחומי עלייה וירידה
		\item אסימפטוטות וגבולות
		\item [-] נק' פיתול
		\item [-] עוקף
		\item [-] קעירות/קמירות
		\item [-] זוגיות
		\item [-] קיצון גלובאלי
		\item סרטוט
	\end{enumerate}
	כאשר מה שמסומן ב־[-] לא חובה בהכרח, ותלוי במה שמבקשים. 
	
	\textbf{המלצה לחקירה: }(ואולי קצת בכללי) לסמן באותיות דברים. לדוגמה, כאשר בודקים חיתוך עם הצירים, לבדוק עבור $f(t) = 0$ ולא $f(x) = 0$. 
	<עברי עשה דוגמה של חקירה אבל אני באמת לא חשוב שיש טעם להראות את זה>
	\section{חוק סנל}
	\textbf{אסור להתשמש בכלל לופיטל בש.ב.}
	
	נניח שהמרחב מחולק לשני תחומים. "פה זה חול ופה זה ים". בנקודה $B$, בים, יש מישוה שטובע. אנחנו נמצאים בנק' $A$ בחול. 
	המרחק בין $B$ לקו החוף הוא $\ml_2$, בינינו לבין החוף $\ml_1$ ובין שני האנכים לקו החוף $d$. אנחנו רוצים לשדוד לו את הכיסים לפני שהוא מת. אנחנו רצים ב־$v_1$ ורצים ב־$v_2$. המרחק ממנו נחתוך לים, יהיה $x$. לפי פיתגורס: (כאשר $L$ זה דרך ו־$T$ זמן)
	\[ L(x) = \sqrt{\ell_2 ^2 + x_2} + \sqrt{\ml_1^2 + (d - x)^2} \]
	\[ T(x) = \frac{\sqrt{\ml_2^2 + x^2}}{v_2} + \frac{\sqrt{\ml_1^2 + (d - x)^2}}{v_1} \]
	
	יהיה יותר נוח לפתור את זה עם טריגו. נצטרך לעשות את זה בשיעורי הבית. 
\end{document}