%! ~~~ Packages Setup ~~~ 
\documentclass[]{article}


% Math packages
\usepackage[usenames]{color}
\usepackage{forest}
\usepackage{ifxetex,ifluatex,amsmath,amssymb,mathrsfs,amsthm,witharrows,mathtools}
\WithArrowsOptions{displaystyle}
\renewcommand{\qedsymbol}{$\blacksquare$} % end proofs with \blacksquare. Overwrites the defualts. 
\usepackage{cancel,bm}
\usepackage[thinc]{esdiff}


% tikz
\usepackage{tikz}
\newcommand\sqw{1}
\newcommand\squ[4][1]{\fill[#4] (#2*\sqw,#3*\sqw) rectangle +(#1*\sqw,#1*\sqw);}


% code 
\usepackage{listings}
\usepackage{xcolor}

\definecolor{codegreen}{rgb}{0,0.35,0}
\definecolor{codegray}{rgb}{0.5,0.5,0.5}
\definecolor{codenumber}{rgb}{0.1,0.3,0.5}
\definecolor{codeblue}{rgb}{0,0,0.5}
\definecolor{codered}{rgb}{0.5,0.03,0.02}
\definecolor{codegray}{rgb}{0.96,0.96,0.96}

\lstdefinestyle{pythonstylesheet}{
	language=Python,
	emphstyle=\color{deepred},
	backgroundcolor=\color{codegray},
	keywordstyle=\color{deepblue}\bfseries\itshape,
	numberstyle=\scriptsize\color{codenumber},
	basicstyle=\ttfamily\footnotesize,
	commentstyle=\color{codegreen}\itshape,
	breakatwhitespace=false, 
	breaklines=true, 
	captionpos=b, 
	keepspaces=true, 
	numbers=left, 
	numbersep=5pt, 
	showspaces=false,                
	showstringspaces=false,
	showtabs=false, 
	tabsize=4, 
	morekeywords={as,assert,nonlocal,with,yield,self,True,False,None,AssertionError,ValueError,in,else},              % Add keywords here
	keywordstyle=\color{codeblue},
	emph={object,type,isinstance,copy,deepcopy,zip,enumerate,reversed,list,set,len,dict,tuple,print,range,xrange,append,execfile,real,imag,reduce,str,repr,__init__,__add__,__mul__,__div__,__sub__,__call__,__getitem__,__setitem__,__eq__,__ne__,__nonzero__,__rmul__,__radd__,__repr__,__str__,__get__,__truediv__,__pow__,__name__,__future__,__all__,},          % Custom highlighting
	emphstyle=\color{codered},
	stringstyle=\color{codegreen},
	showstringspaces=false,
	abovecaptionskip=0pt,belowcaptionskip =0pt,
	framextopmargin=-\topsep, 
}
\newcommand\pythonstyle{\lstset{pythonstylesheet}}
\newcommand\pyl[1]     {{\lstinline!#1!}}
\lstset{style=pythonstylesheet}

\usepackage[style=1,skipbelow=\topskip,skipabove=\topskip,framemethod=TikZ]{mdframed}
\definecolor{bggray}{rgb}{0.85, 0.85, 0.85}
\mdfsetup{leftmargin=0pt,rightmargin=0pt,innerleftmargin=15pt,backgroundcolor=codegray,middlelinewidth=0.5pt,skipabove=5pt,skipbelow=0pt,middlelinecolor=black,roundcorner=5}
\BeforeBeginEnvironment{lstlisting}{\begin{mdframed}\vspace{-0.4em}}
	\AfterEndEnvironment{lstlisting}{\vspace{-0.8em}\end{mdframed}}


% Deisgn
\usepackage[labelfont=bf]{caption}
\usepackage[margin=0.6in]{geometry}
\usepackage{multicol}
\usepackage[skip=4pt, indent=0pt]{parskip}
\usepackage[normalem]{ulem}
\forestset{default}
\renewcommand\labelitemi{$\bullet$}
\usepackage{titlesec}
\titleformat{\section}[block]
{\fontsize{15}{15}}
{\sen \dotfill (\thesection) \she}
{0em}
{\MakeUppercase}
\usepackage{graphicx}
\graphicspath{ {./} }


% Hebrew initialzing
\usepackage[bidi=basic]{babel}
\PassOptionsToPackage{no-math}{fontspec}
\babelprovide[main, import, Alph=letters]{hebrew}
\babelprovide[import]{english}
\babelfont[hebrew]{rm}{David CLM}
\babelfont[hebrew]{sf}{David CLM}
\babelfont[english]{tt}{Monaspace Xenon}
\usepackage[shortlabels]{enumitem}
\newlist{hebenum}{enumerate}{1}

% Language Shortcuts
\newcommand\en[1] {\begin{otherlanguage}{english}#1\end{otherlanguage}}
\newcommand\sen   {\begin{otherlanguage}{english}}
	\newcommand\she   {\end{otherlanguage}}
\newcommand\del   {$ \!\! $}
\newcommand\ttt[1]{\en{\footnotesize\texttt{#1}\normalsize}}

\newcommand\npage {\vfil {\hfil \textbf{\textit{המשך בעמוד הבא}}} \hfil \vfil \pagebreak}
\newcommand\ndoc  {\dotfill \\ \vfil {\begin{center} {\textbf{\textit{סוף מתמטיקה B, סייבר, 2024}} \\ \scriptsize \textit{עברי נגר}} \end{center}} \vfil	}

\newcommand{\rn}[1]{
	\textup{\uppercase\expandafter{\romannumeral#1}}
}

\makeatletter
\newcommand{\skipitems}[1]{
	\addtocounter{\@enumctr}{#1}
}
\makeatother

%! ~~~ Math shortcuts ~~~

% Letters shortcuts
\newcommand\N     {\mathbb{N}}
\newcommand\Z     {\mathbb{Z}}
\newcommand\R     {\mathbb{R}}
\newcommand\Q     {\mathbb{Q}}
\newcommand\C     {\mathbb{C}}

\newcommand\ml    {\ell}
\newcommand\mj    {\jmath}
\newcommand\mi    {\imath}

\newcommand\powerset {\mathcal{P}}
\newcommand\ps    {\mathcal{P}}
\newcommand\pc    {\mathcal{P}}
\newcommand\ac    {\mathcal{A}}
\newcommand\bc    {\mathcal{B}}
\newcommand\cc    {\mathcal{C}}
\newcommand\dc    {\mathcal{D}}
\newcommand\ec    {\mathcal{E}}
\newcommand\fc    {\mathcal{F}}
\newcommand\nc    {\mathcal{N}}
\newcommand\sca   {\mathcal{S}} % \sc is already definded
\newcommand\rca   {\mathcal{R}} % \rc is already definded

\newcommand\Si    {\Sigma}

% Logic & sets shorcuts
\newcommand\siff  {\longleftrightarrow}
\newcommand\ssiff {\leftrightarrow}
\newcommand\so    {\longrightarrow}
\newcommand\sso   {\rightarrow}

\newcommand\epsi  {\epsilon}
\newcommand\vepsi {\varepsilon}
\newcommand\vphi  {\varphi}
\newcommand\Neven {\N_{\mathrm{even}}}
\newcommand\Nodd  {\N_{\mathrm{odd }}}
\newcommand\Zeven {\Z_{\mathrm{even}}}
\newcommand\Zodd  {\Z_{\mathrm{odd }}}
\newcommand\Np    {\N_+}

% Text Shortcuts
\newcommand\open  {\big(}
\newcommand\qopen {\quad\big(}
\newcommand\close {\big)}
\newcommand\also  {\text{, }}
\newcommand\defi  {\text{ definition}}
\newcommand\defis {\text{ definitions}}
\newcommand\given {\text{given }}
\newcommand\case  {\text{if }}
\newcommand\syx   {\text{ syntax}}
\newcommand\rle   {\text{ rule}}
\newcommand\other {\text{else}}
\newcommand\set   {\ell et \text{ }}
\newcommand\ans   {\mathit{Ans.}}

% Set theory shortcuts
\newcommand\ra    {\rangle}
\newcommand\la    {\langle}

\newcommand\oto   {\leftarrow}

\newcommand\QED   {\quad\quad\mathscr{Q.E.D.}\;\;\blacksquare}
\newcommand\QEF   {\quad\quad\mathscr{Q.E.F.}}
\newcommand\eQED  {\mathscr{Q.E.D.}\;\;\blacksquare}
\newcommand\eQEF  {\mathscr{Q.E.F.}}
\newcommand\jQED  {\mathscr{Q.E.D.}}

\newcommand\dom   {\mathrm{dom}}
\newcommand\Img   {\mathrm{Im}}
\newcommand\range {\mathrm{range}}

\newcommand\trio  {\triangle}

\newcommand\rc    {\right\rceil}
\newcommand\lc    {\left\lceil}
\newcommand\rf    {\right\rfloor}
\newcommand\lf    {\left\lfloor}

\newcommand\lex   {<_{lex}}

\newcommand\az    {\aleph_0}
\newcommand\uaz   {^{\aleph_0}}
\newcommand\al    {\aleph}
\newcommand\ual   {^\aleph}
\newcommand\taz   {2^{\aleph_0}}
\newcommand\utaz  { ^{\left (2^{\aleph_0} \right )}}
\newcommand\tal   {2^{\aleph}}
\newcommand\utal  { ^{\left (2^{\aleph} \right )}}
\newcommand\ttaz  {2^{\left (2^{\aleph_0}\right )}}

\newcommand\n     {$n$־יה\ }

% Math A&B shortcuts
\newcommand\logn  {\log n}
\newcommand\logx  {\log x}
\newcommand\lnx   {\ln x}
\newcommand\cosx  {\cos x}
\newcommand\cost  {\cos \theta}
\newcommand\sinx  {\sin x}
\newcommand\sint  {\sin \theta}
\newcommand\tanx  {\tan x}
\newcommand\tant  {\tan \theta}
\newcommand\sex   {\sec x}
\newcommand\sect  {\sec^2}
\newcommand\cotx  {\cot x}
\newcommand\cscx  {\csc x}
\newcommand\sinhx {\sinh x}
\newcommand\coshx {\cosh x}
\newcommand\tanhx {\tanh x}

\newcommand\seq   {\overset{!}{=}}
\newcommand\slh   {\overset{LH}{=}}
\newcommand\sle   {\overset{!}{\le}}
\newcommand\sge   {\overset{!}{\ge}}
\newcommand\sll   {\overset{!}{<}}
\newcommand\sgg   {\overset{!}{>}}

\newcommand\h     {\hat}
\newcommand\ve    {\vec}
\newcommand\lv    {\overrightarrow}
\newcommand\ol    {\overline}

\newcommand\mlcm  {\mathrm{lcm}}

\DeclareMathOperator{\sech}   {sech}
\DeclareMathOperator{\csch}   {csch}
\DeclareMathOperator{\arcsec} {arcsec}
\DeclareMathOperator{\arccot} {arcCot}
\DeclareMathOperator{\arccsc} {arcCsc}
\DeclareMathOperator{\arccosh}{arccosh}
\DeclareMathOperator{\arcsinh}{arcsinh}
\DeclareMathOperator{\arctanh}{arctanh}
\DeclareMathOperator{\arcsech}{arcsech}
\DeclareMathOperator{\arccsch}{arccsch}
\DeclareMathOperator{\arccoth}{arccoth} 

\newcommand\dx    {\,\mathrm{d}x}
\newcommand\dt    {\,\mathrm{d}t}
\newcommand\dtt   {\,\mathrm{d}\theta}
\newcommand\du    {\,\mathrm{d}u}
\newcommand\dv    {\,\mathrm{d}v}
\newcommand\df    {\mathrm{d}f}
\newcommand\dfdx  {\diff{f}{x}}
\newcommand\dit   {\limhz \frac{f(x + h) - f(x)}{h}}

\newcommand\nt[1] {\frac{#1}{#1}}

\newcommand\limz  {\lim_{x \to 0}}
\newcommand\limxz {\lim_{x \to x_0}}
\newcommand\limi  {\lim_{x \to \infty}}
\newcommand\limh  {\lim_{x \to 0}}
\newcommand\limni {\lim_{x \to - \infty}}
\newcommand\limpmi{\lim_{x \to \pm \infty}}

\newcommand\ta    {\theta}
\newcommand\ap    {\alpha}

\renewcommand\inf {\infty}
\newcommand  \ninf{-\inf}

% Combinatorics shortcuts
\newcommand\sumnk     {\sum_{k = 0}^{n}}
\newcommand\sumni     {\sum_{i = 0}^{n}}
\newcommand\sumnko    {\sum_{k = 1}^{n}}
\newcommand\sumnio    {\sum_{i = 1}^{n}}
\newcommand\sumai     {\sum_{i = 1}^{n} A_i}
\newcommand\nsum[2]   {\reflectbox{\displaystyle\sum_{\reflectbox{\scriptsize$#1$}}^{\reflectbox{\scriptsize$#2$}}}}

\newcommand\bink      {\binom{n}{k}}
\newcommand\setn      {\{a_i\}^{2n}_{i = 1}}
\newcommand\setc[1]   {\{a_i\}^{#1}_{i = 1}}

\newcommand\cupain    {\bigcup_{i = 1}^{n} A_i}
\newcommand\cupai[1]  {\bigcup_{i = 1}^{#1} A_i}
\newcommand\cupiiai   {\bigcup_{i \in I} A_i}
\newcommand\capain    {\bigcap_{i = 1}^{n} A_i}
\newcommand\capai[1]  {\bigcap_{i = 1}^{#1} A_i}
\newcommand\capiiai   {\bigcap_{i \in I} A_i}

\newcommand\xot       {x_{1, 2}}
\newcommand\ano       {a_{n - 1}}
\newcommand\ant       {a_{n - 2}}

% Other shortcuts
\newcommand\tl    {\tilde}
\newcommand\op    {^{-1}}

\newcommand\sof[1]    {\left | #1 \right |}
\newcommand\cl [1]    {\left ( #1 \right )}
\newcommand\csb[1]    {\left [ #1 \right ]}

\newcommand\bs    {\blacksquare}

%! ~~~ Document ~~~

\author{שחר פרץ}
\title{מתמטיקה B $\sim$ עברי נגר $\sim$ גומרים}
\date{11 ליולי 2024}


\begin{document}
	\maketitle
	\section{\en{Swift UI}}
	תזכורות: 
	\begin{gather*}
		\sinx = \sum_{k = 0}^{\inf} \frac{(-1)^k}{(2k + 1)!}x^{2k + 1} = x - \frac{x^3}{6} + \frac{x^5}{120} + O(x^7) \\
		\cosx = \sum_{k = 0}^{\inf} \frac{(-1)^k}{(2k)!} = 1 - \frac{x^2}{2} + \frac{x^4}{24} + O(x^6)
	\end{gather*}
	תזכורת: $O(x^n)$ הוא בשאיפה ל־0 (כלומר, כל דברים שהולכים ל־0 יותר מהר מ־$x^n$, ובפרט $x^{n>}$). 
	
	"פעם הבאה תשיב מאמי" (עברי, על רתם)
	\[ \limz \underbrace{\frac{\sinx - x}{x^2(1 - \cos\sqrt x)}}_{f(x)} \]
	נקבל: 
	\[ f(x) = \frac{x - \frac{x^3}{6} + \frac{x^5}{120} + O(x^7)}{x^2\cl{1 - \cl{1 - \frac{x^2}{2} + \frac{x^4}{24} + O(x^6)}}}
	= \frac{-\frac{x^3}{6} + O(x^5)}{\frac{x^3}{2} + O(x^4)} \cdot \frac{\frac{1}{x^3}}{\frac{1}{x^3}}= \frac{-\frac{1}{6} + O(x^2)}{\frac{1}{2} + O(x)} \]
	וסה"כ: 
	\[ \limz f(x) = \frac{-1/6}{1 / 2} = -\frac{1}{3} \]
	יכולנו גם לפתור את הגבול עם לופיטל. בעצם, מה שעשינו זה כמו להפעיל לופיטל 3 פעמים. טור טיילור הוא "קיצור דרך" לכך. 
	
	\subsection{$\bm{e^{-\frac{1}{x^2}}}$}
	נגדיר $f(0) = 0$, כי זה גם הגבול ונרצה שהיא תהיה רציפה בנקודה הזו. 
	\[ f'(x) = e^{-1/x^2}\cl{2 \cdot \frac{1}{x^3}}, \ f''(x) = e^{-1/x^2}\cl{2 \cdot \frac{1}{x^3} 2\frac{1}{x^3} + \frac{6}{x^4}} \]
	וגם כאן נגדיר $f'(0) = 0$, והאסיפמטוטות יוותרו. באופן כללי, $f^{(k)} = e^{-1/x^2}Q\cl{\frac{1}{x}}$ ומשום שפולינום גדל לאט יותר ואקספוננט, אזי בגבול $f^{(k)}(0) = 0$. סה"כ, טור הטיילור שלה סביב $0$ יתאפס, כי כל הנגזרות סביב $0$ יתאפסו: 
	\[ g(x) = \sum_{k = 0}^{\inf}\frac{g^{(k)}(0)}{k!}(x - x_0)^k = 0 \]
	"הפונקציה רציפה אחושרמוטה" (עברי). לפחות תחת ההנחה שהגדרנו שזה הערך שלה ב־0, וזה לא ממש בעיה כי היא רציפה. 
	\section{\en{Euiler Formula}}
	נרצה להרחיב את הגדרת החזקה למספרים מרוכבים. שיעור שעבר ראינו את טור הטיילור של $e^x$: 
	\[ e^x = \sum_{k = 0}^{\inf}\frac{1}{k!}x^k =: e(x) \]
	נבחר, להגדיר את הפונקציה $e(x)$ כטור הטיילור להלן. נוכיח שההגדרה שלנו מקיימת חוקי חזקות: נוכיח את הטענה $e(x + y) = e(x) + e(y)$ (תחת ההנחה שזו הגדרה ל־$e^x$, ואי אפשר להשתמש בטענות קודמות על הפונקציה הזו). 
	\begin{alignat}{9}
		e^{(x + y)} &= \sum_{k = 0}^{\inf}\frac{1}{k!}(x + y)^k &&= \sum_{k = 0}^{\inf}\sum_{m = 0}^{k}\frac{1}{k!}\binom{k}{m}x^my^{k - m} \\
		&= \sum_{m = 0}^{\inf}\sum_{k = m}^{\inf}\frac{x^m}{m!} \cdot \frac{y^{k - m}}{m!} &&\overset{n = k - m}{=}\sum_{m = 0}^{\inf}\sum_{k = 0}^{\inf}\frac{x^m}{m!}\frac{y^n}{n!} = \cl{\sum_{m = 0}^{\inf}\frac{x^n}{n!}}\cl{\sum_{k = 0}^{\inf}\frac{y^n}{y!} = e(x)e(y)}
	\end{alignat}
	נגדיר: 
	\[ \forall x \in \C. e^x = e(x) = \sum_{k = 0}^{\inf}\frac{x^k}{k!} \]
	ההוכחה שלנו לחוק החזקות עובדת גם על מרוכבים, אז מצאנו סיבה לבצע את ההגדרה הזו – ההרחבה משמרת את חוקי החזקות. באופן דומה, נגדיר $\sinx, \cosx$ עבור $x \in \C$ לפי טורי הטיילור שלהם. 
	\[ e^{ix} = \sum_{k = 0}^{\inf}\frac{1}{k!}(ix)^k = \sum_{k = 0}^{\inf}\frac{1}{k!}x^ki^k = \sum_{k = 0}^{\inf} \frac{1}{(4m)!}x^{4m} + \sum_{k = 0}^{\inf}\frac{i}{(4m + 1)!}x^{4m + 1} + \sum_{k = 0}^{\inf}\frac{-1}{(4m + 2)}x^{4m + 2} + \sum_{k = 0}^{\inf}\frac{-i}{(4m + 3)!}x^{4m + 3} \]
	הסתמכנו על כך ש־: 
	\[ i^k = \begin{cases}
		1 & k \equiv 0  \\
		i & k \equiv 1 \\
		-1 & k \equiv 2 \\
		-i & k \equiv 3
	\end{cases}\bmod 4 \]
	נסדר את הטורים: 
	\[ \dots = \cl{\sum_{{n \in \Neven}}^{\inf} \frac{1}{(2n)!}x^{2n} + \sum_{n \in \Nodd}^{\inf}\frac{-1}{(2n)!}x^{2n}} + i\cl{\sum_{{n \in \Neven}}^{\inf} \frac{1}{(2n)!}x^{2n} + \sum_{n \in \Nodd}^{\inf}\frac{-1}{(2n)!}x^{2n}} \]
	ניעזר בעובדה ש־$(-1)^n = \begin{cases}
		1 & n \in \Neven \\
		-1 & n \in \Nodd
	\end{cases}$
	ולכן נוכל לאחד את הסכומים, ולקבל: 
	\[ \dots = \sum_{n = 0}^{\inf}\frac{(-1)^{n}}{(2n)!}x^{2n} + i \sum_{n = 0}^{\inf}\frac{(-1)^{n}}{(2n + 1)}x^{2n + 1} = e^{ix} = \cosx + i\sinx \]
	את הטורים האלו אנו מזהים כטורי הטיילור של החזקות. סה"כ קיבלנו את נוסחאת אויילר. 
	
	נשחק עם זה.
	\[ x = a + bi, \ a, b \in \R. e^{x} = e^{a + bi} = e^{a}e^{bi} = e^{a}(\cos b + i \sin b) \]
	כלומר, המספר $e^x$ הוא מספר שאורכו $e^a$ והזווית שלו היא $b$; $|e^x| = e^x, = e^{\Re(x)}$, באופן דומה על זווית. מהצד השני, אם נפרק מספר מרוכב $z \in \C$: 
	\[ z = r(\cost + i \sint) = e^{\ln r + i \ta} = re^{i\ta} \]
	בהתאם לחוקי החזקות, נוכל לקבל הבנה נוספת של מה משמעותו של הכפל:
	\[ z_1z_2 = (r_1e^{i\ta_1})(r_2e^{i\ta_2}) = r_1r_2e^{i(\ta_1\ta_2)} \]
	הכפלנו את האורך ב־$r$, וסובבנו ב־$\ta$. ולמי שזוכר את ההוכחה ההיא משיעורי הבית: 
	\[ z = a + bi, \ z^{n} = (re^{i\ta})^{n} = r^ne^{i\ta} = r^n(\sin\ta + i\cos\ta) \]
	שבאותה התקופה, ניאלצנו להיעזר באינדוקציה או באמצעים אחרים, מסובכים יותר. השתמשנו באותה הזהות בשביל למצוא את שורשי היחידה, המספרים שמקיימים $z^n = 1$. הוכחנו, $z = u_n^k$ כאשר $u_n = \cl{\cos \cl{\frac{2\pi}{n}} + i \sin \cl{\frac{2\pi}{n}}}$. הם, המספרים שייקימו את המשוואה: 
	\[ r^ne^{in\ta} = r^n(\cos n \ta + i \sin n \ta) \seq 1 \]
	עבורה בהכרח צריך להתקיים $n\ta = 2 \pi k$ כלומר $\ta = \frac{2\pi k}{n}$. נוכל להבין זאת גיאומטרית – נחלק את הזווית על מעגל היחידה ל־$n$ חלקים, ואז נרצה להסתובב עליהם $n$ פעמים עד לחזרה לנקודת ההתחלה. 
	
	\subsection{}
	נפתור את המשוואה 
	$z^2 = 1 + i \sqrt3$.
	לפי הייצוג הפולארי: 
	\[ \begin{cases}
		R = \sqrt{i^2 + \sqrt3^2} = 2 \\
		\tan\alpha = \frac{\sqrt3}{1} \implies \alpha = \tan\op\sqrt3 = \frac{\pi}{3}
	\end{cases} \]
	כלומר: 
	\[ z = re^\ta, , \ r = \sqrt 2, \ r^2e^{i2\ta} = 2 e^{i \frac{\pi}{3}} = r^2(\cos2\ta + i \sin2\ta) = 2(\cos\pi/3 + i \sin \pi/3) \]
	סה"כ: 
	\[ 2\ta = \frac{\pi}{3} + 2 \pi k \implies \ta = \frac{\pi}{6} + \pi k \]
	סה"כ $z = \sqrt2 e^{i\frac{\pi}{k} + i\pi k }$. סה"כ יש לנו 2 פתרונות, כי: 
	\begin{align}
		k = 0 & z = \sqrt2e^{i\pi/6} \\
		k = 1 & z = \sqrt2e^{i \frac{7\pi}{6}} \\
		k = 3 & z = \sqrt2e^{i \pi/6 + i 2\pi} = z_0
	\end{align}
	נוכל גם לקבל את זהות אוילר בצורותיה השונות: 
	\[ e^{2\pi i} = 1, \ e^{i\pi} = -1, \ e^{i\pi} + 1 = 0 \]
	תרגיל: 
	\begin{align}
		w^3 = 1 + i \sqrt3 &= 2e^{i\pi/3} \\
		r^3\cos3\ta + i \sin \ta &= 2\cl{\cos \pi/3 + i \sin \pi/3}
	\end{align}
	נקבל: 
	\[ w = re^{i\ta}, \implies r = \sqrt[3]{2} \ 3 \ta = \frac{\pi}{3} + 2 \pi k \implies \ta = \frac{\pi}{9} + \frac{2\pi k}{3} \]
	נקבל את הפתרונות: 
	\[ w_0 = \sqrt[3]{2}e^{i \pi/9}, \ w_1 = \sqrt[3]{2}e^{i7\pi/9}, \ w_2 = \sqrt[3]{2}e^{i 13 \pi /9} \]
	
	
	עשינו סיבוב $2\pi$. התחלנו ממרוכבים וחזרנו אליהם. 
	
	\ndoc
	
\end{document}