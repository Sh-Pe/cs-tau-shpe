%! ~~~ Packages Setup ~~~ 
\documentclass[]{article}
\usepackage{lipsum}
\usepackage{rotating}


% Math packages
\usepackage[usenames]{color}
\usepackage{forest}
\usepackage{ifxetex,ifluatex,amsmath,amssymb,mathrsfs,amsthm,witharrows,mathtools,mathdots}
\WithArrowsOptions{displaystyle}
\renewcommand{\qedsymbol}{$\blacksquare$} % end proofs with \blacksquare. Overwrites the defualts. 
\usepackage{cancel,bm}
\usepackage[thinc]{esdiff}


% tikz
\usepackage{tikz}
\usetikzlibrary{graphs}
\newcommand\sqw{1}
\newcommand\squ[4][1]{\fill[#4] (#2*\sqw,#3*\sqw) rectangle +(#1*\sqw,#1*\sqw);}


% code 
\usepackage{listings}
\usepackage{xcolor}

\definecolor{codegreen}{rgb}{0,0.35,0}
\definecolor{codegray}{rgb}{0.5,0.5,0.5}
\definecolor{codenumber}{rgb}{0.1,0.3,0.5}
\definecolor{codeblue}{rgb}{0,0,0.5}
\definecolor{codered}{rgb}{0.5,0.03,0.02}
\definecolor{codegray}{rgb}{0.96,0.96,0.96}

\lstdefinestyle{pythonstylesheet}{
	language=Java,
	emphstyle=\color{deepred},
	backgroundcolor=\color{codegray},
	keywordstyle=\color{deepblue}\bfseries\itshape,
	numberstyle=\scriptsize\color{codenumber},
	basicstyle=\ttfamily\footnotesize,
	commentstyle=\color{codegreen}\itshape,
	breakatwhitespace=false, 
	breaklines=true, 
	captionpos=b, 
	keepspaces=true, 
	numbers=left, 
	numbersep=5pt, 
	showspaces=false,                
	showstringspaces=false,
	showtabs=false, 
	tabsize=4, 
	morekeywords={as,assert,nonlocal,with,yield,self,True,False,None,AssertionError,ValueError,in,else},              % Add keywords here
	keywordstyle=\color{codeblue},
	emph={var, List, Iterable, Iterator},          % Custom highlighting
	emphstyle=\color{codered},
	stringstyle=\color{codegreen},
	showstringspaces=false,
	abovecaptionskip=0pt,belowcaptionskip =0pt,
	framextopmargin=-\topsep, 
}
\newcommand\pythonstyle{\lstset{pythonstylesheet}}
\newcommand\pyl[1]     {{\lstinline!#1!}}
\lstset{style=pythonstylesheet}

\usepackage[style=1,skipbelow=\topskip,skipabove=\topskip,framemethod=TikZ]{mdframed}
\definecolor{bggray}{rgb}{0.85, 0.85, 0.85}
\mdfsetup{leftmargin=0pt,rightmargin=0pt,innerleftmargin=15pt,backgroundcolor=codegray,middlelinewidth=0.5pt,skipabove=5pt,skipbelow=0pt,middlelinecolor=black,roundcorner=5}
\BeforeBeginEnvironment{lstlisting}{\begin{mdframed}\vspace{-0.4em}}
	\AfterEndEnvironment{lstlisting}{\vspace{-0.8em}\end{mdframed}}


% Deisgn
\usepackage[labelfont=bf]{caption}
\usepackage[margin=0.6in]{geometry}
\usepackage{multicol}
\usepackage[skip=4pt, indent=0pt]{parskip}
\usepackage[normalem]{ulem}
\forestset{default}
\renewcommand\labelitemi{$\bullet$}
\usepackage{titlesec}
\titleformat{\section}[block]
{\fontsize{15}{15}}
{\sen \dotfill (\thesection)\she}
{0em}
{\MakeUppercase}
\usepackage{graphicx}
\graphicspath{ {./} }


% Hebrew initialzing
\usepackage[bidi=basic]{babel}
\PassOptionsToPackage{no-math}{fontspec}
\babelprovide[main, import, Alph=letters]{hebrew}
\babelprovide[import]{english}
\babelfont[hebrew]{rm}{David CLM}
\babelfont[hebrew]{sf}{David CLM}
\babelfont[english]{tt}{Monaspace Xenon}
\usepackage[shortlabels]{enumitem}
\newlist{hebenum}{enumerate}{1}

% Language Shortcuts
\newcommand\en[1] {\begin{otherlanguage}{english}#1\end{otherlanguage}}
\newcommand\sen   {\begin{otherlanguage}{english}}
	\newcommand\she   {\end{otherlanguage}}
\newcommand\del   {$ \!\! $}

\newcommand\npage {\vfil {\hfil \textbf{\textit{המשך בעמוד הבא}}} \hfil \vfil \pagebreak}
\newcommand\ndoc  {\dotfill \\ \vfil {\begin{center} {\textbf{\textit{שחר פרץ, 2024}} \\ \scriptsize \textit{נוצר באמצעות תוכנה חופשית בלבד}} \end{center}} \vfil	}

\newcommand{\rn}[1]{
	\textup{\uppercase\expandafter{\romannumeral#1}}
}

\makeatletter
\newcommand{\skipitems}[1]{
	\addtocounter{\@enumctr}{#1}
}
\makeatother

%! ~~~ Math shortcuts ~~~

% Letters shortcuts
\newcommand\N     {\mathbb{N}}
\newcommand\Z     {\mathbb{Z}}
\newcommand\R     {\mathbb{R}}
\newcommand\Q     {\mathbb{Q}}
\newcommand\C     {\mathbb{C}}

\newcommand\ml    {\ell}
\newcommand\mj    {\jmath}
\newcommand\mi    {\imath}

\newcommand\powerset {\mathcal{P}}
\newcommand\ps    {\mathcal{P}}
\newcommand\pc    {\mathcal{P}}
\newcommand\ac    {\mathcal{A}}
\newcommand\bc    {\mathcal{B}}
\newcommand\cc    {\mathcal{C}}
\newcommand\dc    {\mathcal{D}}
\newcommand\ec    {\mathcal{E}}
\newcommand\fc    {\mathcal{F}}
\newcommand\nc    {\mathcal{N}}
\newcommand\sca   {\mathcal{S}} % \sc is already definded
\newcommand\rca   {\mathcal{R}} % \rc is already definded

\newcommand\Si    {\Sigma}

% Logic & sets shorcuts
\newcommand\siff  {\longleftrightarrow}
\newcommand\ssiff {\leftrightarrow}
\newcommand\so    {\longrightarrow}
\newcommand\sso   {\rightarrow}

\newcommand\epsi  {\epsilon}
\newcommand\vepsi {\varepsilon}
\newcommand\vphi  {\varphi}
\newcommand\Neven {\N_{\mathrm{even}}}
\newcommand\Nodd  {\N_{\mathrm{odd }}}
\newcommand\Zeven {\Z_{\mathrm{even}}}
\newcommand\Zodd  {\Z_{\mathrm{odd }}}
\newcommand\Np    {\N_+}

% Text Shortcuts
\newcommand\open  {\big(}
\newcommand\qopen {\quad\big(}
\newcommand\close {\big)}
\newcommand\also  {\text{, }}
\newcommand\defi  {\text{ definition}}
\newcommand\defis {\text{ definitions}}
\newcommand\given {\text{given }}
\newcommand\case  {\text{if }}
\newcommand\syx   {\text{ syntax}}
\newcommand\rle   {\text{ rule}}
\newcommand\other {\text{else}}
\newcommand\set   {\ell et \text{ }}
\newcommand\ans   {\mathscr{A}\!\mathit{nswer}}

% Set theory shortcuts
\newcommand\ra    {\rangle}
\newcommand\la    {\langle}

\newcommand\oto   {\leftarrow}

\newcommand\QED   {\quad\quad\mathscr{Q.E.D.}\;\;\blacksquare}
\newcommand\QEF   {\quad\quad\mathscr{Q.E.F.}}
\newcommand\eQED  {\mathscr{Q.E.D.}\;\;\blacksquare}
\newcommand\eQEF  {\mathscr{Q.E.F.}}
\newcommand\jQED  {\mathscr{Q.E.D.}}

\newcommand\dom   {\mathrm{dom}}
\newcommand\Img   {\mathrm{Im}}
\newcommand\range {\mathrm{range}}

\newcommand\trio  {\triangle}

\newcommand\rc    {\right\rceil}
\newcommand\lc    {\left\lceil}
\newcommand\rf    {\right\rfloor}
\newcommand\lf    {\left\lfloor}

\newcommand\lex   {<_{lex}}

\newcommand\az    {\aleph_0}
\newcommand\uaz   {^{\aleph_0}}
\newcommand\al    {\aleph}
\newcommand\ual   {^\aleph}
\newcommand\taz   {2^{\aleph_0}}
\newcommand\utaz  { ^{\left (2^{\aleph_0} \right )}}
\newcommand\tal   {2^{\aleph}}
\newcommand\utal  { ^{\left (2^{\aleph} \right )}}
\newcommand\ttaz  {2^{\left (2^{\aleph_0}\right )}}

\newcommand\n     {$n$־יה\ }

% Math A&B shortcuts
\newcommand\logn  {\log n}
\newcommand\logx  {\log x}
\newcommand\lnx   {\ln x}
\newcommand\cosx  {\cos x}
\newcommand\cost  {\cos \theta}
\newcommand\sinx  {\sin x}
\newcommand\sint  {\sin \theta}
\newcommand\tanx  {\tan x}
\newcommand\tant  {\tan \theta}
\newcommand\sex   {\sec x}
\newcommand\sect  {\sec^2}
\newcommand\cotx  {\cot x}
\newcommand\cscx  {\csc x}
\newcommand\sinhx {\sinh x}
\newcommand\coshx {\cosh x}
\newcommand\tanhx {\tanh x}

\newcommand\seq   {\overset{!}{=}}
\newcommand\slh   {\overset{LH}{=}}
\newcommand\sle   {\overset{!}{\le}}
\newcommand\sge   {\overset{!}{\ge}}
\newcommand\sll   {\overset{!}{<}}
\newcommand\sgg   {\overset{!}{>}}

\newcommand\h     {\hat}
\newcommand\ve    {\vec}
\newcommand\lv    {\overrightarrow}
\newcommand\ol    {\overline}

\newcommand\mlcm  {\mathrm{lcm}}

\DeclareMathOperator{\sech}   {sech}
\DeclareMathOperator{\csch}   {csch}
\DeclareMathOperator{\arcsec} {arcsec}
\DeclareMathOperator{\arccot} {arcCot}
\DeclareMathOperator{\arccsc} {arcCsc}
\DeclareMathOperator{\arccosh}{arccosh}
\DeclareMathOperator{\arcsinh}{arcsinh}
\DeclareMathOperator{\arctanh}{arctanh}
\DeclareMathOperator{\arcsech}{arcsech}
\DeclareMathOperator{\arccsch}{arccsch}
\DeclareMathOperator{\arccoth}{arccoth}
\DeclareMathOperator{\atant}  {atan2} 
\DeclareMathOperator{\spn}    {span}

\newcommand\dx    {\,\mathrm{d}x}
\newcommand\dt    {\,\mathrm{d}t}
\newcommand\dtt   {\,\mathrm{d}\theta}
\newcommand\du    {\,\mathrm{d}u}
\newcommand\dv    {\,\mathrm{d}v}
\newcommand\df    {\mathrm{d}f}
\newcommand\dfdx  {\diff{f}{x}}
\newcommand\dit   {\limhz \frac{f(x + h) - f(x)}{h}}

\newcommand\nt[1] {\frac{#1}{#1}}

\newcommand\limz  {\lim_{x \to 0}}
\newcommand\limxz {\lim_{x \to x_0}}
\newcommand\limi  {\lim_{x \to \infty}}
\newcommand\limh  {\lim_{x \to 0}}
\newcommand\limni {\lim_{x \to - \infty}}
\newcommand\limpmi{\lim_{x \to \pm \infty}}

\newcommand\ta    {\theta}
\newcommand\ap    {\alpha}

\renewcommand\inf {\infty}
\newcommand  \ninf{-\inf}

% Combinatorics shortcuts
\newcommand\sumnk     {\sum_{k = 0}^{n}}
\newcommand\sumni     {\sum_{i = 0}^{n}}
\newcommand\sumnko    {\sum_{k = 1}^{n}}
\newcommand\sumnio    {\sum_{i = 1}^{n}}
\newcommand\sumai     {\sum_{i = 1}^{n} A_i}
\newcommand\nsum[2]   {\reflectbox{\displaystyle\sum_{\reflectbox{\scriptsize$#1$}}^{\reflectbox{\scriptsize$#2$}}}}

\newcommand\bink      {\binom{n}{k}}
\newcommand\setn      {\{a_i\}^{2n}_{i = 1}}
\newcommand\setc[1]   {\{a_i\}^{#1}_{i = 1}}

\newcommand\cupain    {\bigcup_{i = 1}^{n} A_i}
\newcommand\cupai[1]  {\bigcup_{i = 1}^{#1} A_i}
\newcommand\cupiiai   {\bigcup_{i \in I} A_i}
\newcommand\capain    {\bigcap_{i = 1}^{n} A_i}
\newcommand\capai[1]  {\bigcap_{i = 1}^{#1} A_i}
\newcommand\capiiai   {\bigcap_{i \in I} A_i}

\newcommand\xot       {x_{1, 2}}
\newcommand\ano       {a_{n - 1}}
\newcommand\ant       {a_{n - 2}}

% Linear Algebra
\DeclareMathOperator{\chr}    {char}

\newcommand\lra       {\leftrightarrow}
\newcommand\chrf      {\chr(\F)}
\newcommand\F         {\mathbb{F}}
\newcommand\co        {\colon}
\newcommand\tmat[2]   {\cl{\begin{matrix}
			#1
		\end{matrix}\, \middle\vert\, \begin{matrix}
			#2
\end{matrix}}}

\makeatletter
\newcommand\rrr[1]    {\xxrightarrow{1}{#1}}
\newcommand\rrt[2]    {\xxrightarrow{1}[#1]{#2}}
\newcommand\mat[2]    {M_{#1\times#2}}
\newcommand\tomat     {\, \dequad \longrightarrow}

% someone's code from the internet: https://tex.stackexchange.com/questions/27545/custom-length-arrows-text-over-and-under
\makeatletter
\newlength\min@xx
\newcommand*\xxrightarrow[1]{\begingroup
	\settowidth\min@xx{$\m@th\scriptstyle#1$}
	\@xxrightarrow}
\newcommand*\@xxrightarrow[2][]{
	\sbox8{$\m@th\scriptstyle#1$}  % subscript
	\ifdim\wd8>\min@xx \min@xx=\wd8 \fi
	\sbox8{$\m@th\scriptstyle#2$} % superscript
	\ifdim\wd8>\min@xx \min@xx=\wd8 \fi
	\xrightarrow[{\mathmakebox[\min@xx]{\scriptstyle#1}}]
	{\mathmakebox[\min@xx]{\scriptstyle#2}}
	\endgroup}
\makeatother


% Greek Letters
\newcommand\ag        {\alpha}
\newcommand\bg        {\beta}
\newcommand\cg        {\gamma}
\newcommand\dg        {\delta}
\newcommand\eg        {\epsi}
\newcommand\zg        {\zeta}
\newcommand\hg        {\eta}
\newcommand\tg        {\theta}
\newcommand\ig        {\iota}
\newcommand\kg        {\keppa}
\renewcommand\lg      {\lambda}
\newcommand\og        {\omicron}
\newcommand\rg        {\rho}
\newcommand\sg        {\sigma}
\newcommand\yg        {\usilon}
\newcommand\wg        {\omega}

\newcommand\Ag        {\Alpha}
\newcommand\Bg        {\Beta}
\newcommand\Cg        {\Gamma}
\newcommand\Dg        {\Delta}
\newcommand\Eg        {\Epsi}
\newcommand\Zg        {\Zeta}
\newcommand\Hg        {\Eta}
\newcommand\Tg        {\Theta}
\newcommand\Ig        {\Iota}
\newcommand\Kg        {\Keppa}
\newcommand\Lg        {\Lambda}
\newcommand\Og        {\Omicron}
\newcommand\Rg        {\Rho}
\newcommand\Sg        {\Sigma}
\newcommand\Yg        {\Usilon}
\newcommand\Wg        {\Omega}

% Other shortcuts
\newcommand\tl    {\tilde}
\newcommand\op    {^{-1}}

\newcommand\sof[1]    {\left | #1 \right |}
\newcommand\cl [1]    {\left ( #1 \right )}
\newcommand\csb[1]    {\left [ #1 \right ]}

\newcommand\bs        {\blacksquare}
\newcommand\dequad    {\!\!\!\!\!\!}
\newcommand\dequadd   {\dequad\duquad}
\renewcommand\phi     {\varphi}

%! ~~~ Document ~~~

\author{שחר פרץ}
\title{הרצאה 5}
\begin{document}
	\maketitle
	הערה: $\mathrm{span}(\emptyset) = \{0\}$
	\section{\en{Linear maps}}
	\textbf{למה. }תהי $\phi \co V \to U$, ו־$F$ שדה. 
	\begin{enumerate}
		\item $\phi(0_V) = 0_V$
		\item $\Img \phi$ תמ"ו של $U$
		\item $\ker\phi$ תמ"ו של $V$
		\item $\phi$ על אמ"מ $\Img\phi = U$
		\item $\phi$ חח"ע אמ"מ $\ker \phi = 0$
		\item $\phi$ העתקת האפס אמ"מ $\Im\phi = \{0\}$ אמ"מ $\ker\phi = V$
	\end{enumerate}
	\begin{proof}\, 
		\begin{enumerate}
			\item \[ \phi(0_V) = \phi(0_F \cdot 0_V) = 0_F \cdot \phi(0_V) = 0 \]
			\item נראה: \[ \forall v_1, v_0 \in \Img\phi, \lambda \in F\co \lambda_1v_1 + \lambda_2v_2 \in \Img\phi  \]
			וגם $\Img\phi$ לא ריק כי $0 \in \Img\phi$ מסעיף קודם. 
			נוכיח את הטענה הראשונה: 
			\[ V_i = \chi(x_0) \implies \sum \lambda_1\phi(x_i) = \phi\cl{\sum \lambda_ix_i} \in \Img\phi \]
			מלינאריות והגדרה של תמונה. 
			\item $0 \in \ker\phi$ מסעיף קודם־קודם. נראה: $\sum\lambda_iv_i \in \ker\phi$ עבור $\lambda_i \in F$. ואכן: 
			\[ \phi\cl{\sum\lambda_iv_i} = \sum\lambda_i\phi(v_i) = \lambda_1 \cdot 0 + \lambda _2 \cdot 0 = 0 \]
			ולכן הקרנל. 
			\item 
			\[ \Img \phi = U \iff \forall y \in U. y \in \Img \phi \iff \exists x \co \phi(x) \ = y \iff \phi \, \text{על} \]
			\item 
			\[ \forall x, y \in V, t \in V.\phi(t) = 0 \iff \phi(x - y) = 0 \iff f(x) = f(y) \iff \phi \, \text{חח"ע} \]
			המעבר האחרון מליניאריות. 
			\item 
			\[ \phi \, \text{העתקת האפס} \iff (\forall x \in V. \phi(x) = 0) \iff \Img f = \{0\} \iff \ker(x) = V \]
			"זה פשוט ישירות. בוא נראה אם הם כתבו איזה משהו... אז אני פשוט אוסיף קצת מילים"
		\end{enumerate}
	\end{proof}
	
	\textbf{הגדרה. }$\phi \co V_1 \to V_2$ נאמר ש\textit{איזומורפיזם} (איזו') אם קיימת $\psi$ ליניארית כך ש־$\psi \co V_2 \to V_1. \psi \circ \phi = id_{V_1} \land \phi \circ \psi = id_{V_2}$. \textbf{סימון. }$\psi =: \phi\op$. 
	
	\textbf{למה. }$\psi \co V_1 \to V_2$ ליניארית $\impliedby$
	\begin{enumerate}
		\item $\phi$ איזו' $\iff$ $\phi$ חח"ע ועל
		\item אם $\phi$ איזו' $\impliedby$ הופכית יחידה. 
	\end{enumerate}
	
	אומרים שמשהו הוא "איזומורפי" אם יש איזומורפיזם ביניהם. 
	
	\textbf{טענה. }נתבונן ב־$\hom(V_1, V_2)$ מ"ו מעל $F$, בעבור הפעולות: 
	\[ (\phi + \psi)(v) := \phi(v) + \psi(v), \ (\lambda\phi) := \lambda\phi(v) \]
	
	זה פשוט לבדוק את כל התכונות. לא נוכיח את זה. 
	
	\textbf{טענה. }יהיו $\phi \co V_1 \to V_2, \ \psi \co V_2 \to V_3$, אז $\psi \circ \phi$ העתקה ליניארית. 
	
	"זה נפתח כזה כמו... נפתח כזה". 
	
	\textbf{טענה. }לטענה הזו, נסמן $\phi \cdot \psi = \phi \circ \psi$. יהיו $V$ מ"ו ו־$\hom(V) =: U$. $\impliedby$
	\begin{enumerate}
		\item קיום ניטרלי לכפל, שהוא $id_V$
		\item אסוציאטיביות: 
		\[ \forall \phi_1, \phi_2, \phi_3 \in U. (\phi_1 \phi_2)\phi_3 = \phi_1(\phi_2\phi_3) \]
		\item דירטביוטיביות משמאל: 
		\[ \forall \phi, \psi_1, \psi_2 \in U. \phi \cdot (\psi_1 + \psi_2) = \phi\psi_1 + \phi\psi_2 \]
		\item דיסטרביוטיביות מימין: 
		\[ \forall \phi, \psi_1, \psi_2 \in U. (\psi_1 + \psi_2)\phi = \psi_1 \cdot \phi + \psi_2 \cdot \phi \]
		\item תאימות עם כפל בסקלר: 
		\[ \forall \phi, \psi \in U, \ \lambda, \alpha \in F. (\lambda \phi)(\alpha\psi) = (\lambda\alpha)(\psi \cdot \phi) \]
	\end{enumerate}
	כלומר, זה כמעט־שדה – אין קומטטיביות. זו גם הסיבה שצריך להוכיח דיסטרביוטיביות משני הכיוונים. דוגמא למקרה בהו קומטטיביות לא עובדת: 
	
	\[ \set V = F^2, \ \phi \co (x, y) \mapsto (x, -y), \ \psi \co (x, y) \mapsto (-y, x). \phi\psi(1, 0) = \phi(0, 1) = (0, -1) \neq (0, 1) = \psi(1, 0) = \phi\psi(0, 1) \]
	
	\textbf{טענה. }יהיו $\phi \co V \to U$, $V_1, \dots V_s \in V$ ו־$\lambda_1 \dots \lambda_S\in F$, אז $\phi\cl{\sum\lambda_iv_i} = \sum \lambda_i\phi(v_i)$. 
	
	\textbf{מסקנה. }יהי $V$ מ"ו עם בסיס $B =(V_1, \dots, V_n)$ ותהי $\phi \co V \to U$ אז לכל $(u_1, \dots, u_n)$ c-$U$, מתקיים שקיימת ויחידה ההעתקה ליניארית $\forall i \in [n]. \phi(v_i) = u_i$ (המרצה הגיע ממצב ש־$u_i$ בסיס ב־$V$ למצב שזה לא בסיס ולא ב־$V$. לא חושב שהוא קרא את הסיכום שהוא גנב מהאינטרנט). 
	
	\begin{proof}
		\textbf{יחידות: }יהי $x \in V$. נראה ש־$\phi(x)$ קבוע. ואכן, $\phi(x) = \phi\cl{\sum\lambda_iv_i}$ עבור $\lambda_i$ המקדמים של הצירוף הליניארי של $x$ בבסיס. וזה: 
		\[ \cdots = \sum\lambda_i\phi(v_i) = \sum_{i = 1}^{n}\lambda_iu_i \]
		
		\textbf{קיום: }נראה שהפונקציה ליניארית (ספוילר: המורה יגדיר את הפוקנציה שנוכיח עליה רק למטה). יהי $\lambda_1, \lambda_2 \in F, x_1, x_2 \in V$ ונראה ש־$\phi(\sum\lambda_ix_i) = \sum\lambda_i\phi(x_i)$. 
		
		נסמן: 
		\[ x_1 = \sum_{i = 1}^{n}\alpha_iv_i, \ x_2 = \sum_{i = 1}^{n}\bg_iv_i \]
		פירוק לבסיס, ונראה שמתקיים: 
		[הערה: $\phi$ מוגדרת להיות $\phi\cl{\sum\lambda_iv_i} = \sum v_i\phi(v_i)$. למה לכל הרוחות אנחנו מוכחים משהו על $\phi$ שהמורה הגדיר אותה רק אחרי השוויון למטה]
		\begin{multline*}
			\phi(\lambda_1x_1 + \lambda_2x_2) = \phi\cl{\lambda_1 \sum \alpha_iv_i + \lambda_2 \sum \bg v_i} = \phi\cl{\sum(\lambda_1\ag_i + \lambda_2\bg_i)v_i} = \sum \cl{(\lambda_1\ag_i + \lambda_2\bg_i)}\phi(v_i) = \\ \lambda_1\sum\ag_i\phi(v_i) + \lambda_2\sum\bg_i\phi(v_i) = \lambda_1 \phi(x_1) + \lambda_2\phi(x_2)
		\end{multline*}
	\end{proof}
	
	\textbf{טענה. }תהי $\phi\co V \to U$ ליניארית ו־$B = (v_1 \dots v_s)$ וקטורים ב־$v$. נסמן $\phi(B) := (\phi(v_1) \dots \phi(v_s))$ (סדרת התמונות). $\impliedby$
	\begin{enumerate}
		\item אם $\phi(B)$ בת"ל $\impliedby$ $B$ בת"ל
		\item אם $B$ פורשת $\impliedby$ $\phi(B)$ פורשת את $\Img\phi$ [ובפרט אם $\phi$ על אז $\phi(B)$ פורשת את $U$]
		\item אם $\ker\phi = 0$ $\impliedby$ $\cl{f(B) \ \text{בת"ל} \iff B \ \text{בת"ל}}$. 
		\item אם $\phi$ איזו' אז: $B$ בת"ל/פורשת/בסיס גורר $\phi(B)$ בת"ל/פורשת/בסיס, בהתאמה. 
	\end{enumerate}
	
	\begin{proof}\, 
		\begin{enumerate}
			\item נניח $\phi(B)$ בת"ל. נראהְ ש־$B$ בת"ל. נניח $\sum\ag_iv_i = 0$ ונראה שגורר $\forall i \in [s]. \ag_i = 0$. 
			
			נסתכל על המשוואה לאחר הפעלת $\phi$: 
			\[ o = \phi(0) = \phi\cl{\sum \ag_iv_i) =  \sum\ag_i\phi(v_i)} \implies \ag_i = 0 \]
			קיבלנו צירוף ליניארי של $f(B)$ שווה ל־$0$. ובגלל ש־$\phi(B)$ בת"ל אז הצירוף הליניארי חייב להיות הטרוויאלי מהגדרת בת"ל. 
			\item יהי $y \in \Img\phi$. אז $\exists x \in V. y = \phi(x)$ (כי $B$ פורש). אזי $\exists \lambda_i \in F \co x = \sum\lambda_iv_i$ ולכן $y = \phi(\sum\lambda_iv_i)$ וסה"כ $y = \sum \lambda_i\phi(v_I)$ וסה"כ נפרש ע"י $\phi(B)$. 
			
			\item נניח $B$ בת"ל, ונראה $\phi(B)$ בת"ל. נסתכל על $\sum\ag_i\phi(v_i) = 0$  ונראה ש־$0 \ \ag_i$. מליניאריות $\phi(\sum\ag_iv_i) = 0$ ומהנתון $\ker\phi = \{0\}$, $\sum\ag_iv_i = 0$. סה"כ $\ag_i$ כי $B = \{v_i\}$ בת"ל. 
			
			\item 
			\begin{itemize}
				\item \textit{בת"ל: }אם $\phi$ איז' אז $\phi$ חח"ע ולכן $\ker\phi = \{0\}$ ערצוי מטענה קודמת. 
				\item \textit{פורשת: }נסתכל על $\phi\op$ כהעתקה (כי $\phi$ איזומורפיזם) ונסתכל על $C = \phi(B), \ \phi\op(C)$. אם $C$ פורש אז $\phi\op(C)$ פורש מטענה (2) ולכן $\phi(B)$ פורש גם$B$. 
			\end{itemize}
		\end{enumerate}
	\end{proof}
	
	\section{\en{Linear maps but now with dimmnesions}}
	\textbf{משפט. }בהינתן $\phi \co V \to U$ ו־$\dim V < \inf$ אז $\dim V= \dim\ker\phi + \dim \Img\phi$. 
	\begin{proof}
		נראה בסיס ל־$V$. 
		נסתכל על $\ker\phi$ ונסמן בסיס שלו $v_1 \dots v_s \in B_1$. נרחיב את $B_1$ לבסיס ל־$V$ עם $v_{s + 1} \dots v_{\dim V}$. נראה ש־$v_{s + 1} \dots v_[\dim V]$ בסיס למרחב שמימדו $\dim \Img\phi$. 
		
		נראה כי $\phi(v_{s + 1} \dots v_{\dim V})$ פורש את $\Img \phi$. 
		\[ \Img\phi = \phi(\spn(v_{s + 1} \dots v_{\dim V})) \cup \{0\} \]
		ההסבר לשוויון הוא שמתקיים (כאשר $v_i \in B$): 
		\[ \forall y \in \Img\phi \exists x \in V. y = \phi(x) \implies y = \phi(\sum\ag_iv_i) = \underbrace{\sum^s_{i = 1}\ag_i\phi(v_i)}_{0} + \sum_{i = s + 1}^{\dim V}\ag_i\phi(v_i) \implies y \in \spn(\phi(s + 1), \dots \phi(\dim V))) \]
		
		נראה שבת"ל. נניח $\sum_{i = s + 1}^{\dim V} \ag_i\phi(v_i) = 0$ ונראה $\ag_i  = 0$. מההנחה: 
		\[ \phi\cl{\sum_{i = s + 1}^{\dim V}\ag_iv_i} = 0 \]
		אבל, בגלל ש־$v_{s + 1} \dots v_{\dim V} \notin \ker\phi$  אז $\sum_{i = s + 1}^{\dim V} \ag_i v_i = 0$ ומבת"ליות $\ag_i = 0$. 
		 
	\end{proof}
	
	\textbf{מסקנה. }תהי $\phi \co V \to U$ ליניארית. אם $\dim V < \inf$ אז: 
	\begin{enumerate}
		\item אם $\phi$ שיכון, אז $\dim V \le \dim U$
		\item אם $\phi$ על אז $\dim U \le \dim V$
		\item אם $\phi$ איזו' אז $\dim V = \dim U$
		\item אם $\phi$ חח"ע או על, וגם $\dim V = \dim U$, אז $\phi$ איזו'. 
	\end{enumerate}
	
	\begin{proof}
		\begin{enumerate}
			\item $\phi$ שיכון, אז  $\dim V = \dim \ker \phi + \dim \Img \phi$ וידוע $\ker\phi = 0$ כלומר $\dim V \le 0 + \dim U = U$ 
			\item $\phi$ על, אז 
			\[ \dim V = \dim \ker \phi + \underbrace{\dim \Img \phi}_{\dim U} \implies \dim V -\dim \ker\phi = \dim U \]
			\item אם איזו', אז שיכון וגם על, ולכן $\dim U \le \dim V \le \dim U$ כדרוש. 
			\item אם $\phi$ וגם $\dim V = \dim U$: נראה שיכון, כלומר $\ker\phi = 0$. אכן: 
			\[ \dim V = \dim \ker\phi + \dim \underbrace{\Img\phi}_{\dim U} \]
			ולכן $\dim \ker \phi = 0$ וסה"כ $\ker\phi = \{0\}$. 
			אחרת, $\phi$ שיכון וגם $\dim U = \dim V$. נראה שעל. 
			\[ \dim V = \dim \ker\phi + \dim \Img\phi \implies \dim V = \dim \Img\phi \implies \Img\phi = U \]
			וגם: 
			\[ (\dim V)= \dim U = \dim (\Img\phi) \]
			הם אותו מרחב (ניקח בסיס של $\Img \phi$ והוא יהיה בסיס של $U$ בכלל שסדרה בת"ל באורך המימד). 
		\end{enumerate}
	\end{proof}
	
	\textbf{מסקנה. }יהיו $U, W \subseteq V$ תמ"ו. אז: (נובע מעקרון ההכלה וההדחה)
	\[ \dim(U + W) = \dim U + \dim W - \dim(U \cap W) \]
	"משפט המימדים". 
	\begin{proof}
		נתבונן ב־$\phi \co U + W \to V$
		אינטואיציה: נוכל להתסכל על $U \times W \to V$. הוא מעניין אותנו כי:
		\begin{enumerate}
			\item אם היינו יוצרים $\phi(u, w) = u + w$, אז הקרנל $\ker\phi = \{(u, -u) \mid u \in U\}$. הקרנל איזומורפי ל־$U$. 
			\item לכן, $\dim(U\times W) = \dim U + \dim \Img \phi$
		\end{enumerate}
		ואז: 
		\[ \ag \co U \times W \to U, \ag(u, w) = u, \ker \{(0, w) \mid w \in W\}, \ \dim U \times W = \dim U + \dim W \]
		
		נגמרה האינטואיציה. עכשיו ההוהחכה. 
		נגדיר: 
		\[ \alpha \co U \times W \to U, \ \ag(u, w) =u \]
		 נחשב ונקבל: 
		 ,\[ \dim(u \times W) = \dim \ker\ag = \underbrace{\dim \{(0, w) \mid w \in W\}}_{\dim W}+ \dim U \]
		 נראה ש־$\ker\ag$ אכן במימד כמו $W$ ע"י איזומורפיזם $\phi(0, w) \mapsto w$. וסיימנו ממשפט ממקודם מפיו איזו גורר אותו המימד. 
		 
		 נגדיר: 
		 \[ \phi \co U \times W \to V, \ \phi(u, w) = u + w \implies \{(u, -u) \mid u \in U \cap W\} = \ker\phi \ \text{איזומורפי ל־} \ U\cap W \]
		 אזי
		 \[ \dim U + \dim W = \dim U\times W = \dim U\cap W + \dim \Img\phi \]
		 ולכן
		 \[ \dim U + \dim W = \dim U \cap W + \dim \Img\phi \]
		 נשים לב ש-: 
		 \[ \Img\phi = U + W = \{(u + w) \mid u \in U, w \in W\} \]
		 וסה"כ
		 \[ \dim U + \dim W = \dim U \cap W + \dim (U + W) \]
		 כרצוי (איכשהו). 
	\end{proof}
	
	\section{\en{Linear Maps But It's in The Matrix}}
	\textbf{מסקנה. }יהיו $U, V$ מ"ו ממימד $n$, ו־$B = (v-1 \dots v_n)$ בסיס, אז ישנה התאמה חח"ע ועל בין $\phi \co V \to U$ איזו' לבין בסיס של $U$, והיא: עבור $\phi$ איזו' נתאים את $\phi(B)$ ועבור $C$ בסיס של $U$ נתאים את $\phi_C \co V \to U$ כך ש־$\forall i \in [n]. \phi_C(v_i) = u_i$
	
	יהי $V$ מ"ו ממימד $n$, ו־$B = (v_1 \dots v_n)$ בסיס םו־$v \in V$. אז: 
	\[ [v]_B = (\lambda_1 \dots \lambda_n) \in F^n, \ v = \sum \lambda_iv_i \]
	
	\textbf{משפט. }יהי $V$ מ"ו עם בסיס $B = (v_1, \dots v_n)$. אז $\phi = \phi_B \co F^n \to V$ כך ש־$\phi(\lambda_1 \dots \lambda_n) = \sum \lambda_iv_i$. אז: $\phi$ איזו וההופכית $\phi\op(v) = [v]_B$
	
	עכשיו נעשה דברים אקראיים ונעתיק את מה השמורה עושה. $\phi \co V \to U$, $B = (v_1 \dots v_n)$ בסיס, $C = (u_1 \dots u_n)$ בסיס של $U$, נרצה לבנות מטריצה כך ש־: 
	\[ \forall j \in [n]. \phi(v_j) = \sum_{i = 1}^{m}a_{ij}u_i \]
	את $\phi$ נרצה לייצא באמצעות תוצאות $\phi(B)$, ואותו לייצג באמצעות $([\phi(v_1)]_C \dots [\phi(v_n)]_C)$. נקרא למטריצה כזו \textit{מטריצה מייצגת}. 
	
	\textbf{הגדרה. }נגיד שיש לנו $\phi \co V \to U$ עם בסיסים $B$ בסיס של $V$ ו־$C$ בסיס של $U$. נסמן: 
	\[ [\phi]_C^B = (a_{ij})_{\begin{aligned}
			\scriptstyle i =1 \dots m \\
			\scriptstyle j = 1 \dots n
	\end{aligned}} = \begin{pmatrix}
	\vdots & & \vdots \\
	[\phi(V_1)]_C & \cdots &  [\phi(V_n)]_C \\
	\vdots &  & \vdots
\end{pmatrix} \]
זוהי המטריצה המייצגת של $\phi$ לפי בסיס $B$ של $V$ ו־$C$ של $U$. 
	נשים לב ש־$[\phi(V_i)]_C$ זו עמודה עם $m$ שורות. 
	
	\subsection{דוגמאות}
	\subsubsection{דוגמה ראשונה}
	יהי $\phi \co F^2 \to F^3$, המוגדרת לפי $\phi(x, y) = (x, x + y, x + 2y)$. נסמן ב־$e$ בסיס סטנדרטי של $F^2$ ו־$e'$ בסיס סטנדרטי של $F^3$. 
	\[ e =\{(1, 0), (0, 1)\} = (e_1, e_2), \ \phi(e_1) = (1, 1, 1) = e_1' + e_2' + e_3' \implies [\phi(e_1)]_{e'} = (1, 1, 1) \]
	נמשיך כך: 
	\begin{gather}
		\phi(0, 1) = (0, 1, 2) \implies [\phi(0, 1)]_{e'} = (0, 1, 2)
	\end{gather}
	וסה"כ: 
	\[ [\phi]_{e'}^e = \begin{pmatrix}
		1 & 0 \\ 1 & 1 \\ 1 & 2
	\end{pmatrix} \]
	
	\subsubsection{דוגמה נוספת}
	\[ C = ((1, 1, 1), (1, 0, 1), (0, 1, 1)) = (C_1, C_2, C_3) \]
	אזי: 
	\[ [\phi(e_1)]_C = [(1, 1, 1)]_C = (1, 0, 0) (\text{since} \ (1, 1, 1) = 1 \cdot c_1 + 0 \cdot c_2 + 0 \cdot c_3) \]
	וסה"כ: 
	\[ [\phi]_C^e = \begin{pmatrix}
		1 & -1 \\ 0 & 1 \\ 0 & 2
	\end{pmatrix} \]
	
	\subsection{ספוילר}
	יהיו $A, \phi $ מטריצה מייצגת של $\phi$. אז נגדיר
	\[ \phi(v) = Av \]
	כאשר כפל המטריצות: 
	\[ \begin{pmatrix}
		\vdots & \quad & \vdots \\
		c_1 & & c_n \\
		\vdots && \vdots
	\end{pmatrix}\begin{pmatrix}
		x_1 \\ \vdots \\ x_n
	\end{pmatrix} = \begin{pmatrix}
		i \mapsto \sum_{j = 1}^{n}a_{ij}x_j
	\end{pmatrix} \]
	
	
\end{document}