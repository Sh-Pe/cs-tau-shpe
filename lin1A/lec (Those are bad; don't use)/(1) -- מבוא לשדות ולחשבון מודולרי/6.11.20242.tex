%! ~~~ Packages Setup ~~~ 
\documentclass[]{article}


% Math packages
\usepackage[usenames]{color}
\usepackage{forest}
\usepackage{ifxetex,ifluatex,amsmath,amssymb,mathrsfs,amsthm,witharrows,mathtools}
\WithArrowsOptions{displaystyle}
\renewcommand{\qedsymbol}{$\blacksquare$} % end proofs with \blacksquare. Overwrites the defualts. 
\usepackage{cancel,bm}
\usepackage[thinc]{esdiff}


% tikz
\usepackage{tikz}
\newcommand\sqw{1}
\newcommand\squ[4][1]{\fill[#4] (#2*\sqw,#3*\sqw) rectangle +(#1*\sqw,#1*\sqw);}


% code 
\usepackage{listings}
\usepackage{xcolor}

\definecolor{codegreen}{rgb}{0,0.35,0}
\definecolor{codegray}{rgb}{0.5,0.5,0.5}
\definecolor{codenumber}{rgb}{0.1,0.3,0.5}
\definecolor{codeblue}{rgb}{0,0,0.5}
\definecolor{codered}{rgb}{0.5,0.03,0.02}
\definecolor{codegray}{rgb}{0.96,0.96,0.96}

\lstdefinestyle{pythonstylesheet}{
	language=Python,
	emphstyle=\color{deepred},
	backgroundcolor=\color{codegray},
	keywordstyle=\color{deepblue}\bfseries\itshape,
	numberstyle=\scriptsize\color{codenumber},
	basicstyle=\ttfamily\footnotesize,
	commentstyle=\color{codegreen}\itshape,
	breakatwhitespace=false, 
	breaklines=true, 
	captionpos=b, 
	keepspaces=true, 
	numbers=left, 
	numbersep=5pt, 
	showspaces=false,                
	showstringspaces=false,
	showtabs=false, 
	tabsize=4, 
	morekeywords={as,assert,nonlocal,with,yield,self,True,False,None,AssertionError,ValueError,in,else},              % Add keywords here
	keywordstyle=\color{codeblue},
	emph={object,type,isinstance,copy,deepcopy,zip,enumerate,reversed,list,set,len,dict,tuple,print,range,xrange,append,execfile,real,imag,reduce,str,repr,__init__,__add__,__mul__,__div__,__sub__,__call__,__getitem__,__setitem__,__eq__,__ne__,__nonzero__,__rmul__,__radd__,__repr__,__str__,__get__,__truediv__,__pow__,__name__,__future__,__all__,},          % Custom highlighting
	emphstyle=\color{codered},
	stringstyle=\color{codegreen},
	showstringspaces=false,
	abovecaptionskip=0pt,belowcaptionskip =0pt,
	framextopmargin=-\topsep, 
}
\newcommand\pythonstyle{\lstset{pythonstylesheet}}
\newcommand\pyl[1]     {{\lstinline!#1!}}
\lstset{style=pythonstylesheet}

\usepackage[style=1,skipbelow=\topskip,skipabove=\topskip,framemethod=TikZ]{mdframed}
\definecolor{bggray}{rgb}{0.85, 0.85, 0.85}
\mdfsetup{leftmargin=0pt,rightmargin=0pt,innerleftmargin=15pt,backgroundcolor=codegray,middlelinewidth=0.5pt,skipabove=5pt,skipbelow=0pt,middlelinecolor=black,roundcorner=5}
\BeforeBeginEnvironment{lstlisting}{\begin{mdframed}\vspace{-0.4em}}
	\AfterEndEnvironment{lstlisting}{\vspace{-0.8em}\end{mdframed}}


% Deisgn
\usepackage[labelfont=bf]{caption}
\usepackage[margin=0.6in]{geometry}
\usepackage{multicol}
\usepackage[skip=4pt, indent=0pt]{parskip}
\usepackage[normalem]{ulem}
\forestset{default}
\renewcommand\labelitemi{$\bullet$}
\usepackage{titlesec}
\titleformat{\section}[block]
{\fontsize{15}{15}}
{\sen \dotfill (\thesection) \she}
{0em}
{\MakeUppercase}
\usepackage{graphicx}
\graphicspath{ {./} }


% Hebrew initialzing
\usepackage[bidi=basic]{babel}
\PassOptionsToPackage{no-math}{fontspec}
\babelprovide[main, import, Alph=letters]{hebrew}
\babelprovide[import]{english}
\babelfont[hebrew]{rm}{David CLM}
\babelfont[hebrew]{sf}{David CLM}
\babelfont[english]{tt}{Monaspace Xenon}
\usepackage[shortlabels]{enumitem}
\newlist{hebenum}{enumerate}{1}

% Language Shortcuts
\newcommand\en[1] {\begin{otherlanguage}{english}#1\end{otherlanguage}}
\newcommand\sen   {\begin{otherlanguage}{english}}
	\newcommand\she   {\end{otherlanguage}}
\newcommand\del   {$ \!\! $}
\newcommand\ttt[1]{\en{\footnotesize\texttt{#1}\normalsize}}

\newcommand\npage {\vfil {\hfil \textbf{\textit{המשך בעמוד הבא}}} \hfil \vfil \pagebreak}
\newcommand\ndoc  {\dotfill \\ \vfil {\begin{center} {\textbf{\textit{שחר פרץ, 2024}} \\ \scriptsize \textit{נוצר באמצעות תוכנה חופשית בלבד}} \end{center}} \vfil	}

\newcommand{\rn}[1]{
	\textup{\uppercase\expandafter{\romannumeral#1}}
}

\makeatletter
\newcommand{\skipitems}[1]{
	\addtocounter{\@enumctr}{#1}
}
\makeatother

%! ~~~ Math shortcuts ~~~

% Letters shortcuts
\newcommand\N     {\mathbb{N}}
\newcommand\Z     {\mathbb{Z}}
\newcommand\R     {\mathbb{R}}
\newcommand\Q     {\mathbb{Q}}
\newcommand\C     {\mathbb{C}}

\newcommand\ml    {\ell}
\newcommand\mj    {\jmath}
\newcommand\mi    {\imath}

\newcommand\powerset {\mathcal{P}}
\newcommand\ps    {\mathcal{P}}
\newcommand\pc    {\mathcal{P}}
\newcommand\ac    {\mathcal{A}}
\newcommand\bc    {\mathcal{B}}
\newcommand\cc    {\mathcal{C}}
\newcommand\dc    {\mathcal{D}}
\newcommand\ec    {\mathcal{E}}
\newcommand\fc    {\mathcal{F}}
\newcommand\nc    {\mathcal{N}}
\newcommand\sca   {\mathcal{S}} % \sc is already definded
\newcommand\rca   {\mathcal{R}} % \rc is already definded

\newcommand\Si    {\Sigma}

% Logic & sets shorcuts
\newcommand\siff  {\longleftrightarrow}
\newcommand\ssiff {\leftrightarrow}
\newcommand\so    {\longrightarrow}
\newcommand\sso   {\rightarrow}

\newcommand\epsi  {\epsilon}
\newcommand\vepsi {\varepsilon}
\newcommand\vphi  {\varphi}
\newcommand\Neven {\N_{\mathrm{even}}}
\newcommand\Nodd  {\N_{\mathrm{odd }}}
\newcommand\Zeven {\Z_{\mathrm{even}}}
\newcommand\Zodd  {\Z_{\mathrm{odd }}}
\newcommand\Np    {\N_+}

% Text Shortcuts
\newcommand\open  {\big(}
\newcommand\qopen {\quad\big(}
\newcommand\close {\big)}
\newcommand\also  {\text{, }}
\newcommand\defi  {\text{ definition}}
\newcommand\defis {\text{ definitions}}
\newcommand\given {\text{given }}
\newcommand\case  {\text{if }}
\newcommand\syx   {\text{ syntax}}
\newcommand\rle   {\text{ rule}}
\newcommand\other {\text{else}}
\newcommand\set   {\ell et \text{ }}
\newcommand\ans   {\mathit{Ans.}}

% Set theory shortcuts
\newcommand\ra    {\rangle}
\newcommand\la    {\langle}

\newcommand\oto   {\leftarrow}

\newcommand\QED   {\quad\quad\mathscr{Q.E.D.}\;\;\blacksquare}
\newcommand\QEF   {\quad\quad\mathscr{Q.E.F.}}
\newcommand\eQED  {\mathscr{Q.E.D.}\;\;\blacksquare}
\newcommand\eQEF  {\mathscr{Q.E.F.}}
\newcommand\jQED  {\mathscr{Q.E.D.}}

\newcommand\dom   {\mathrm{dom}}
\newcommand\Img   {\mathrm{Im}}
\newcommand\range {\mathrm{range}}

\newcommand\trio  {\triangle}

\newcommand\rc    {\right\rceil}
\newcommand\lc    {\left\lceil}
\newcommand\rf    {\right\rfloor}
\newcommand\lf    {\left\lfloor}

\newcommand\lex   {<_{lex}}

\newcommand\az    {\aleph_0}
\newcommand\uaz   {^{\aleph_0}}
\newcommand\al    {\aleph}
\newcommand\ual   {^\aleph}
\newcommand\taz   {2^{\aleph_0}}
\newcommand\utaz  { ^{\left (2^{\aleph_0} \right )}}
\newcommand\tal   {2^{\aleph}}
\newcommand\utal  { ^{\left (2^{\aleph} \right )}}
\newcommand\ttaz  {2^{\left (2^{\aleph_0}\right )}}

\newcommand\n     {$n$־יה\ }

% Math A&B shortcuts
\newcommand\logn  {\log n}
\newcommand\logx  {\log x}
\newcommand\lnx   {\ln x}
\newcommand\cosx  {\cos x}
\newcommand\cost  {\cos \theta}
\newcommand\sinx  {\sin x}
\newcommand\sint  {\sin \theta}
\newcommand\tanx  {\tan x}
\newcommand\tant  {\tan \theta}
\newcommand\sex   {\sec x}
\newcommand\sect  {\sec^2}
\newcommand\cotx  {\cot x}
\newcommand\cscx  {\csc x}
\newcommand\sinhx {\sinh x}
\newcommand\coshx {\cosh x}
\newcommand\tanhx {\tanh x}

\newcommand\seq   {\overset{!}{=}}
\newcommand\slh   {\overset{LH}{=}}
\newcommand\sle   {\overset{!}{\le}}
\newcommand\sge   {\overset{!}{\ge}}
\newcommand\sll   {\overset{!}{<}}
\newcommand\sgg   {\overset{!}{>}}

\newcommand\h     {\hat}
\newcommand\ve    {\vec}
\newcommand\lv    {\overrightarrow}
\newcommand\ol    {\overline}

\newcommand\mlcm  {\mathrm{lcm}}

\DeclareMathOperator{\sech}   {sech}
\DeclareMathOperator{\csch}   {csch}
\DeclareMathOperator{\arcsec} {arcsec}
\DeclareMathOperator{\arccot} {arcCot}
\DeclareMathOperator{\arccsc} {arcCsc}
\DeclareMathOperator{\arccosh}{arccosh}
\DeclareMathOperator{\arcsinh}{arcsinh}
\DeclareMathOperator{\arctanh}{arctanh}
\DeclareMathOperator{\arcsech}{arcsech}
\DeclareMathOperator{\arccsch}{arccsch}
\DeclareMathOperator{\arccoth}{arccoth}
\DeclareMathOperator{\atant}  {atan2} 

\newcommand\dx    {\,\mathrm{d}x}
\newcommand\dt    {\,\mathrm{d}t}
\newcommand\dtt   {\,\mathrm{d}\theta}
\newcommand\du    {\,\mathrm{d}u}
\newcommand\dv    {\,\mathrm{d}v}
\newcommand\df    {\mathrm{d}f}
\newcommand\dfdx  {\diff{f}{x}}
\newcommand\dit   {\limhz \frac{f(x + h) - f(x)}{h}}

\newcommand\nt[1] {\frac{#1}{#1}}

\newcommand\limz  {\lim_{x \to 0}}
\newcommand\limxz {\lim_{x \to x_0}}
\newcommand\limi  {\lim_{x \to \infty}}
\newcommand\limh  {\lim_{x \to 0}}
\newcommand\limni {\lim_{x \to - \infty}}
\newcommand\limpmi{\lim_{x \to \pm \infty}}

\newcommand\ta    {\theta}
\newcommand\ap    {\alpha}

\renewcommand\inf {\infty}
\newcommand  \ninf{-\inf}

% Combinatorics shortcuts
\newcommand\sumnk     {\sum_{k = 0}^{n}}
\newcommand\sumni     {\sum_{i = 0}^{n}}
\newcommand\sumnko    {\sum_{k = 1}^{n}}
\newcommand\sumnio    {\sum_{i = 1}^{n}}
\newcommand\sumai     {\sum_{i = 1}^{n} A_i}
\newcommand\nsum[2]   {\reflectbox{\displaystyle\sum_{\reflectbox{\scriptsize$#1$}}^{\reflectbox{\scriptsize$#2$}}}}

\newcommand\bink      {\binom{n}{k}}
\newcommand\setn      {\{a_i\}^{2n}_{i = 1}}
\newcommand\setc[1]   {\{a_i\}^{#1}_{i = 1}}

\newcommand\cupain    {\bigcup_{i = 1}^{n} A_i}
\newcommand\cupai[1]  {\bigcup_{i = 1}^{#1} A_i}
\newcommand\cupiiai   {\bigcup_{i \in I} A_i}
\newcommand\capain    {\bigcap_{i = 1}^{n} A_i}
\newcommand\capai[1]  {\bigcap_{i = 1}^{#1} A_i}
\newcommand\capiiai   {\bigcap_{i \in I} A_i}

\newcommand\xot       {x_{1, 2}}
\newcommand\ano       {a_{n - 1}}
\newcommand\ant       {a_{n - 2}}

% Other shortcuts
\newcommand\tl    {\tilde}
\newcommand\op    {^{-1}}

\newcommand\sof[1]    {\left | #1 \right |}
\newcommand\cl [1]    {\left ( #1 \right )}
\newcommand\csb[1]    {\left [ #1 \right ]}

\newcommand\bs    {\blacksquare}

%! ~~~ Document ~~~

\author{שחר פרץ}
\title{ליניארית 1}
\begin{document}
	\maketitle
	\section{\en{ABOUT}}
	בן 29, בוגר ארזים. סטודנט לפסיגולוגיה. מייל: 	
	\en{kligman@mail.tau.ac.il}
	. עדיף לפנות למתרגל לשאלות. 
	
	\section{\en{intro}}
	הקורס מתעסק במשוואות ליניאריות, כמו שהוצגו בשיעור הקודם. משוואות ליניאריות הן כמעט המשוואות היחידות שכבני אדם נוכל לפתור באופן פשוט. 
	
	יש מגוון אובייקטים שניתנים לתיאור באמצעות אלגברה ליניארית. לדוגמה, מישור בטוך מרחב תלת ממדי. דדוקציה משאלות כאלו למשוואות ליניאריות היא כלי חזק. 
	
	ניתן לתאר מערכת משוואות ליניאריות באמצעות מטרציות (מעין טבלה). אפילו תיאור נגזרות של פונקציה ממרחב גבוהה מתוארות באמצעות אלגברה ליניארית. 
	
	תיאור כללי של הקורס: 
	\begin{enumerate}
		\item שדות (בערך שליש מהקרוס)
		\item מרחב וקטורי
		\item העתקות ליניאריות (בהרחבה)
		\item מטרציות, בהקשר למבנים שראינו קודם
		\item בקורסים מתדמים נראה אובייקטים נוספים
	\end{enumerate}
	
	\section{\en{linear euqtions systems}}
	נתבונן במשוואה הבאה: 
	\[ a_1x_1  + a_2x_2 = d_1 \]
	שימו לב: בתורת הקבוצות $a_1, a_2$ שייכים לקבוצה כלשהי. נבחר $a_1, a_2 \in A$. יהיה גם צורך להגדיר פעולת כפל בין $x_1$ ל־$a_1$. הוא יהיה פונקציה $f(a_1, x_2) \mapsto ?$. נצטרך פעולת חיבור גם. לאוסף הפעולות האלא נקרא שדה. 
	
	הרבה מהקושי הוא הפורמליות והדקדקנות. בניית דברים מאקסיומות יכולה להיות מתעתעת. זה רלוונטי בלמידה, בש"ב ובמבחנים. לפעמים, השאלה צריכה להיות שאלת פרמול בעקרה. לכן, מומלץ לפתור דברים באופן פורמלי – זה יעזור להצלחה בקורס, ועוד יותר יעזור בעתיד. 
	\section{\en{Fields}}
	\subsection{הגדרה}
	נגדיר באופן פרמלי מה זה שדה. 
	
	\textbf{הגדרה. }$F$ קבוצה, וקיימת $a \colon F \times F \to F$ פונקציה הקרויה חיבור, וקיימת $m \colon F \times F \to F$ פונקציה הקרויה כפל, נאמר ש־$F$ עם $a, m$ \textit{שדה} אם: 
	
	\textbf{סימון. }\hfill $a(x, y) = x + y, \ m(x, y) = x \cdot y$
	
	\begin{enumerate}
		\item איבר ניטרלי לחיבור: \hfill $\exists x \in F. \forall y \in F. y + x = y$ \\
		\textbf{סימון. }האיבר ב־$0$ או $0_F$. נקרא לו \textit{איבר האפס}. 
		\item אסוציאטיביות: \hfill $\forall x, y, z \in F: (x + y) + z = x + (y + z)$
		\item חילופיות של חיבור: \hfill $\forall x,  \in F: x + y = y + x$
		\item קיום איבר נגדי: \hfill $\forall x \in F: \exists y \in F. x + y = y + x = 0_F$ \\
		\textbf{סימון. }איבר נגדי של $x$ הוא $(-x)$. 
		\item ניטרלי לכפל: \hfill $\exists x \in F \colon \forall y \in F \colon x y = y x = y$
		\item אסוציאטיביות של כפל: \hfill $\forall x , y, z \in F \colon (xy) \cdot z = x \cdot (yz)$
		\item קיום הופכי: \hfill $\forall 0 \neq x \in F \exists y \in F \colon. xy = yx = 1$ \\
		\textbf{סימון. }הופכי של $x$ יהיה $x\op$. נסמן גם עם חילוק (כלומר $\frac{1}{x}$). 
		\item חילופיות כפל. \hfill $\forall x, y \in F \colon xy = yx$. 
		\item דיסטרביוטיביות: \hfill $\forall x, y, z \in F \colon x(y + z) = xy + xz$
		\item \hfil $1_F \neq 0_F$ \hfil
	\end{enumerate}
	\textbf{סימון. }איבר בשדה יקרא \textit{סקלר}. 
	
	\subsection{דוגמאות}
	
	\subsubsection{הרציונליים}
	נבתונן ב־$\Q$. הרציונלים, הם שדה. קבוצתם היא בערך: 
	\[ \Q = \left \{\frac{a}{b} \mid a, b \neq 0 \in \Z \right \} \]
	\textbf{טענה}. $\Q$ הוא השדה המינימלי המכיל את $\Z$. משהו בסגנון, ניקח $a \in \Q$, נסמנו $\frac{x}{y}$ ובהתאם השדה שלנו יעיל את $\frac{x}{y}$ כי ניקח את $x \in \Z$, $0 \neq y \in \Z$, ונייצר $x \cdot y\op$. שדה מינימלי לא הוגדר פורמלית בקורס הזה. 
	
	שימו לב – ב־$\Q$ ההצגה היא לא יחידה (ומשפיע על חוסר הפורמליות בנתון לעיל). הרי $\frac{2}{3}  = \frac{4}{6}$. 
	
	\subsubsection{הממשיים}
	גם $\R$ הם שדה. 	גם בשדה הזה הסימון לא יחיד. $0.99999... = 1$ (כי ככה מגדירים באנליזה). בנייה פורמלית בלי אקסיומת החסם העליון – ען=ם ארז. 
	
	\subsubsection{המרוכבים}
	גם $\C$ הם שדה. 	הם סגורים אלגברית, כלומר לכל משוואה יש פתרון. נזהה את $\R^2 = \C^2$. נסמן את $(1, 0)$ אם איבר היחידה, ונגדיר: 
	\[ (x, y) + (z, w) = (z + x, y + w), \ (x, z) \cdot (z, w) =  (xz - yw, xw + yz) \]
	נסמן $(0, 1) = i, \ (x, y) = x + yi$. נקבל $i^2 = -1$, ואם נסמן $\bar z = x - iy$ וגם נסמן $|z| = \sqrt{z \cdot \bar z}$ נקבל ש־$z\op = |z| = \frac{z}{|z|^2} - \frac{x - iy}{x^2 + y^2}$. למעשה, דרך להראות ש־$\C$ הוא שדה, בפרט נרצה למצוא את ההופכי. 
	
	\textbf{משפט. }המשפט היסודי של האלגברה (ראה עברי) – בהינתן פולינום מרוכב עם מקדמים מרוכבים, נסמן $f(x) = a_nx^n \dots a_0$, אז $\exists z \in \C \colon f(z) = 0$. 
	
	\subsubsection{תתי קבוצות של המרוכבים}
	מתברר, ש־$\Q \cup \{i\}$ הוא גם שדה. אם זאת $\R_{> 0}$ הוא אינו שדה. ניקח קבוצה $F = \{a, b\}$ ונגדיר טבלת כפל וחיבור:
	
	\begin{center}
		\begin{tabular}{|c|c|c|}
			\hline $+$ &  $a$ & $b$ \\
			\hline $a$ & $a$ & $b$ \\
			\hline $b $& $b$ & $a$ \\ \hline 
		\end{tabular} \quad \begin{tabular}{|c|c|c|}
		 	\hline $\cdot$ & $a$ & $b$ \\
		 	\hline $a$ & $a$ & $a$ \\
		 	\hline $b$ & $a$ & $b$ \\ \hline 
		\end{tabular}
	\end{center}
	ונגדיר $0_F = a, q_F = b$ אז $\mathbb{F} = \{0, 1\}$ שדה. 
	
	\subsection{טענות על שדות}
	\textbf{משפטים. }
	\begin{itemize}
		\item ניטרלי לחיבור הוא יחיד. 
		\item $\forall a \in F. 0 \cdot a = 0$ 
		\item ניטרלי לכפל הוא יחיד. 
		\item לכל $a \in F$ האיבר הנגדי יחיד וגם $-a = (-1) \cdot a$. 
		\item לכל $0 \neq a \in F$ הופכי יחיד. 
	\end{itemize}
	\begin{proof}[הוכחה (1)]נניח בשלילה $x, y \in F$ כך ששניהם ניטרליים לחיבור, ונראה $x = y$. אז: 
		\[ x = x + y = y \]
		כי $y$ ניטרלי לחיבור, וכי $x$ ניטרלי לחיבור. סתירה. 
	\end{proof}
	
	\begin{proof}[הוכחה (2)]
		יהי $a \in F$, נראה $0 \cdot a = 0$. ואכן: 
		\[ \begin{WithArrows}
			0 \cdot a = (0 + 0) \cdot a = 0 \cdot a + 0 \cdot a
		\end{WithArrows} \]
		נוסיף $-(0a)$ לשני הצדדים. ומכאן, $0 = 0a$. (חוקי מח"ע)
	\end{proof}
	
	\begin{proof}[הוכחה (3)]
		יהי $a \in F$ וניקח $b, c \in F$. 
		\[ b = b + 0 = b + (a + c) = (b + a) + c = 0 + c = c \]
	\end{proof}
	
	\begin{proof}[הוכחה (4)]
		נראה כי $(-1) \cdot a$ הוא נגדי ל־$a$, ולכן $-a = (-1) \cdot a$, כי זה הסימון. 
		\[ a + (-1) \cdot a = a \cdot (1 + (-1)) = a \cdot 0 = 0 \]
	\end{proof}
	
	\begin{proof}[הוכחה (5)]
		יהי $0 \neq a \in F$, וניקח $b, c \in F$ הופכיים. 
		\[ b = b \cdot (a \cdot c) = (b \cdot a) \cdot c = c \]
	\end{proof}
	
	\subsection{טענות נוספות}
	יהיו $a, b, c, d \in F$ כאשר $F$ שדה. 
	\begin{enumerate}
		\item \hfil $(b = 0 \lor a = 0) \iff ab = 0$ \hfil 
		\item \hfil $b = c \iff a + b = a + c$ \hfil 
		\item אם $a \neq 0 $ אז $b = c \iff ab = ac$
	\end{enumerate}
	
	\begin{proof}[הוכחה (1)] \ 
		\begin{itemize}
			\item[$\implies$] הוכחנו מקודם. 
			\item[$\impliedby$] $ab = 0$. אם $a = 0$, סיימנו. אחרת, $a \neq 0$, נראה כי $b = 0$: 
			\[ b = 1 \cdot b = (a\op \cdot a)b = a\op\cdot (a \cdot b) = a \op \cdot 0 = 0 \]
		\end{itemize}
	\end{proof}
	
	\begin{proof}[הוכחה (2)]  
		\begin{itemize}
			\item[$\implies$] אם $b = c$ אז נוסיף $a $ לשני האגפים ונקבל $a + b = a + c$. 
			\item[$\impliedby$] $a + b = c + d$, נוסיף $(-a)$ לשני האגפים, נקבל: $-a + a + b = -a + a + c$ ולכן $b = c$. 
			
			דרך אחרת לעשות זאת, היא: 
			\[ b = 0 + b = (-a + a) + b = -a + (a + b) = -a + (a + c) = c \]
		\end{itemize}
	\end{proof}
	
	\begin{proof}[הוכחה (3)]
		\begin{itemize}
			\item[$\implies$] $b = c$ אז $a - b = a - c$
			\item[$\impliedby$] באופן דומה: 
			\[ b = 1 \cdot b = (a \op a)b = a\op \cdot (ab) = a\op \cdot (ac) = (a\op \cdot a )c = 1 \cdot c = c \]
		\end{itemize}
	\end{proof}
	
	\section{\en{Finite Fileds}}
	\subsection{חשבון מודולרי}
	המטרה: לראות איך חשבון מודולרי מייצג שדות סופיים. 
	
	\textbf{הגדרה. }אם $n \ge 1$ טבעי, אז נגדיר יחס על זוגות שלמים $x, y \in \Z$ הוא: 
	\[ x \equiv y \bmod n \iff \exists k \in \N. x - y = nk \]
	זהו יחס שקילות, שיוגדר באמצעות $\{\la x, y \ra \in \Z^2 \mid \exists k \in \Z. x - y = nk\} = \equiv_n$. 
	
	\textbf{למה. }אם $n \ge 1$, אז $x \equiv y \mod n $ הוא שקילות (רפלקסיבי, סימטרי, טרנזיטיבי). הוכחנו עם נטלי. 
	
	\textbf{הגדרה. }$x \in \Z, \ \Z \ni n \ge 1$. נגדיר $[x]_n = \{y \in \Z \mid x \equiv y \mod n\}$ -- מחלקת השקילות של $x$. 
	
	\textbf{טענה. }$[x]_n = \{x + nk \mid k \in \Z\}$. לעת עתה לא נוכיח. 
	
	\subsubsection{דוגמאות}
	יהי $\N\ni n \ge 1$, אז: 
	\begin{gather}
		[0]_n = \{\dots, -2n, n, 0, n, 2n \dots \mid \mid n \in \N_+\} \\
		[1]_n = \{\dots, 1 - n, 1, 1 + n \dots \} \\
		[n]_1 = [0] \\
		[n + 1] = [1]_n
	\end{gather}
	
	\subsection{טענות נוספות}
	\begin{itemize}
		\item כל שתי מחלקות שקילו תשוות או זרות
		\item בכל מחלקת שקילות יש בדיוק אחד מבין $\{0, \dots n - 1\}$. 
	\end{itemize}
	
	\begin{proof}[הוכחה (1)]
		יהיו $[a], [b]$ מחלקות שקילות. אם $[a] \cap [b] = \varnothing$ סיימנו. אחרת, קיים $c \in [a] \cap [b]$. נראה ש־$[a] \subseteq [b]$ ומסימטריה נקבל שוויון. 
		
		יהי $x \in [a]$, נראה $x \in [b]$ ע"י $x \equiv b \bmod n$. ע"י $x \equiv a \bmod n, \ c \equiv a \bmod n, \ x = c \bmod n$ והחיתוך $c \equiv b \bmod n$ (רעיון $x = a = b = c$) ולכן $x =b \bmod n$ ומרטנזיטיביות. 
	\end{proof}
	\begin{proof}[הוכחה (2)]
		יהי $[a]_n$, $a \in \Z$. צ.ל. קיום $i \in \{0, \dots, n - 1\}$ כך ש־$i \in [a]_n$. נסתכל על $c = a - \lf \frac{a}{m}\rf \cdot n$. נראה ש־$c \in \{0, \dots, n - 1\}$. נשים לב ש־$\frac{a}{n} -1 > \lf \frac{a}{n}\rf \le \frac{a}{n}$. אזי: 
		\[ 0 = a - \frac{a}{n} \le c = a - \lf \frac{a}{n} \rf n< a - \cl{\frac{a}{n} - 1}n  = n \]

כדרוש. נראה יחידות. יהיו $r_1, r_2 \in \{0, \dots n - 1\}$ וגם $r_1, r_2 \in [a]_n$. נראה $r_1 = r_2$. מהשייכות ל־$[a]_n$ נקבל: 
\[ r_1 \equiv a \equiv r_2 \mod n \implies r_1 - r_2 \]
בה"כ $r_1 > r_2$, אי־שוויון וניצחנו. ולכן $0 \le r_1 - r_2 < n$ ואם שונים אז סתירה. 
	\end{proof}
	
	\subsection{חידה}
	תהי הקבוצה $\{[0], \dots, [n-1]\} = F$ והפעולות מודולו, צ.ל. ששדה $\iff$ $n$ ראשוני. 
	
\end{document}