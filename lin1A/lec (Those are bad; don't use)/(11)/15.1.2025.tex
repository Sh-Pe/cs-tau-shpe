%! ~~~ Packages Setup ~~~ 
\documentclass[]{article}
\usepackage{lipsum}
\usepackage{rotating}


% Math packages
\usepackage[usenames]{color}
\usepackage{forest}
\usepackage{ifxetex,ifluatex,amsmath,amssymb,mathrsfs,amsthm,witharrows,mathtools,mathdots}
\WithArrowsOptions{displaystyle}
\renewcommand{\qedsymbol}{$\blacksquare$} % end proofs with \blacksquare. Overwrites the defualts. 
\usepackage{cancel,bm}
\usepackage[thinc]{esdiff}


% tikz
\usepackage{tikz}
\usetikzlibrary{graphs}
\newcommand\sqw{1}
\newcommand\squ[4][1]{\fill[#4] (#2*\sqw,#3*\sqw) rectangle +(#1*\sqw,#1*\sqw);}


% code 
\usepackage{listings}
\usepackage{xcolor}

\definecolor{codegreen}{rgb}{0,0.35,0}
\definecolor{codegray}{rgb}{0.5,0.5,0.5}
\definecolor{codenumber}{rgb}{0.1,0.3,0.5}
\definecolor{codeblue}{rgb}{0,0,0.5}
\definecolor{codered}{rgb}{0.5,0.03,0.02}
\definecolor{codegray}{rgb}{0.96,0.96,0.96}

\lstdefinestyle{pythonstylesheet}{
	language=Java,
	emphstyle=\color{deepred},
	backgroundcolor=\color{codegray},
	keywordstyle=\color{deepblue}\bfseries\itshape,
	numberstyle=\scriptsize\color{codenumber},
	basicstyle=\ttfamily\footnotesize,
	commentstyle=\color{codegreen}\itshape,
	breakatwhitespace=false, 
	breaklines=true, 
	captionpos=b, 
	keepspaces=true, 
	numbers=left, 
	numbersep=5pt, 
	showspaces=false,                
	showstringspaces=false,
	showtabs=false, 
	tabsize=4, 
	morekeywords={as,assert,nonlocal,with,yield,self,True,False,None,AssertionError,ValueError,in,else},              % Add keywords here
	keywordstyle=\color{codeblue},
	emph={var, List, Iterable, Iterator},          % Custom highlighting
	emphstyle=\color{codered},
	stringstyle=\color{codegreen},
	showstringspaces=false,
	abovecaptionskip=0pt,belowcaptionskip =0pt,
	framextopmargin=-\topsep, 
}
\newcommand\pythonstyle{\lstset{pythonstylesheet}}
\newcommand\pyl[1]     {{\lstinline!#1!}}
\lstset{style=pythonstylesheet}

\usepackage[style=1,skipbelow=\topskip,skipabove=\topskip,framemethod=TikZ]{mdframed}
\definecolor{bggray}{rgb}{0.85, 0.85, 0.85}
\mdfsetup{leftmargin=0pt,rightmargin=0pt,innerleftmargin=15pt,backgroundcolor=codegray,middlelinewidth=0.5pt,skipabove=5pt,skipbelow=0pt,middlelinecolor=black,roundcorner=5}
\BeforeBeginEnvironment{lstlisting}{\begin{mdframed}\vspace{-0.4em}}
	\AfterEndEnvironment{lstlisting}{\vspace{-0.8em}\end{mdframed}}


% Deisgn
\usepackage[labelfont=bf]{caption}
\usepackage[margin=0.6in]{geometry}
\usepackage{multicol}
\usepackage[skip=4pt, indent=0pt]{parskip}
\usepackage[normalem]{ulem}
\forestset{default}
\renewcommand\labelitemi{$\bullet$}
\usepackage{titlesec}
\titleformat{\section}[block]
{\fontsize{15}{15}}
{\sen \dotfill (\thesection) \she}
{0em}
{\MakeUppercase}
\usepackage{graphicx}
\graphicspath{ {./} }


% Hebrew initialzing
\usepackage[bidi=basic]{babel}
\PassOptionsToPackage{no-math}{fontspec}
\babelprovide[main, import, Alph=letters]{hebrew}
\babelprovide[import]{english}
\babelfont[hebrew]{rm}{David CLM}
\babelfont[hebrew]{sf}{David CLM}
\babelfont[english]{tt}{Monaspace Xenon}
\usepackage[shortlabels]{enumitem}
\newlist{hebenum}{enumerate}{1}

% Language Shortcuts
\newcommand\en[1] {\begin{otherlanguage}{english}#1\end{otherlanguage}}
\newcommand\sen   {\begin{otherlanguage}{english}}
	\newcommand\she   {\end{otherlanguage}}
\newcommand\del   {$ \!\! $}

\newcommand\npage {\vfil {\hfil \textbf{\textit{המשך בעמוד הבא}}} \hfil \vfil \pagebreak}
\newcommand\ndoc  {\dotfill \\ \vfil {\begin{center} {\textbf{\textit{שחר פרץ, 2024}} \\ \scriptsize \textit{נוצר באמצעות תוכנה חופשית בלבד}} \end{center}} \vfil	}

\newcommand{\rn}[1]{
	\textup{\uppercase\expandafter{\romannumeral#1}}
}

\makeatletter
\newcommand{\skipitems}[1]{
	\addtocounter{\@enumctr}{#1}
}
\makeatother

%! ~~~ Math shortcuts ~~~

% Letters shortcuts
\newcommand\N     {\mathbb{N}}
\newcommand\Z     {\mathbb{Z}}
\newcommand\R     {\mathbb{R}}
\newcommand\Q     {\mathbb{Q}}
\newcommand\C     {\mathbb{C}}

\newcommand\ml    {\ell}
\newcommand\mj    {\jmath}
\newcommand\mi    {\imath}

\newcommand\powerset {\mathcal{P}}
\newcommand\ps    {\mathcal{P}}
\newcommand\pc    {\mathcal{P}}
\newcommand\ac    {\mathcal{A}}
\newcommand\bc    {\mathcal{B}}
\newcommand\cc    {\mathcal{C}}
\newcommand\dc    {\mathcal{D}}
\newcommand\ec    {\mathcal{E}}
\newcommand\fc    {\mathcal{F}}
\newcommand\nc    {\mathcal{N}}
\newcommand\vc    {\mathcal{V}} % Vance
\newcommand\sca   {\mathcal{S}} % \sc is already definded
\newcommand\rca   {\mathcal{R}} % \rc is already definded

\newcommand\prm   {\mathrm{p}}
\newcommand\arm   {\mathrm{a}} % x86
\newcommand\brm   {\mathrm{b}}
\newcommand\crm   {\mathrm{c}}
\newcommand\drm   {\mathrm{d}}
\newcommand\erm   {\mathrm{e}}
\newcommand\frm   {\mathrm{f}}
\newcommand\nrm   {\mathrm{n}}
\newcommand\vrm   {\mathrm{v}}
\newcommand\srm   {\mathrm{s}}
\newcommand\rrm   {\mathrm{r}}

\newcommand\Si    {\Sigma}

% Logic & sets shorcuts
\newcommand\siff  {\longleftrightarrow}
\newcommand\ssiff {\leftrightarrow}
\newcommand\so    {\longrightarrow}
\newcommand\sso   {\rightarrow}

\newcommand\epsi  {\epsilon}
\newcommand\vepsi {\varepsilon}
\newcommand\vphi  {\varphi}
\newcommand\Neven {\N_{\mathrm{even}}}
\newcommand\Nodd  {\N_{\mathrm{odd }}}
\newcommand\Zeven {\Z_{\mathrm{even}}}
\newcommand\Zodd  {\Z_{\mathrm{odd }}}
\newcommand\Np    {\N_+}

% Text Shortcuts
\newcommand\open  {\big(}
\newcommand\qopen {\quad\big(}
\newcommand\close {\big)}
\newcommand\also  {\text{\en{, }}}
\newcommand\defi  {\text{\en{ definition}}}
\newcommand\defis {\text{\en{ definitions}}}
\newcommand\given {\text{\en{given }}}
\newcommand\case  {\text{\en{if }}}
\newcommand\syx   {\text{\en{ syntax}}}
\newcommand\rle   {\text{\en{ rule}}}
\newcommand\other {\text{\en{else}}}
\newcommand\set   {\ell et \text{ }}
\newcommand\ans   {\mathscr{A}\!\mathit{nswer}}

% Set theory shortcuts
\newcommand\ra    {\rangle}
\newcommand\la    {\langle}

\newcommand\oto   {\leftarrow}

\newcommand\QED   {\quad\quad\mathscr{Q.E.D.}\;\;\blacksquare}
\newcommand\QEF   {\quad\quad\mathscr{Q.E.F.}}
\newcommand\eQED  {\mathscr{Q.E.D.}\;\;\blacksquare}
\newcommand\eQEF  {\mathscr{Q.E.F.}}
\newcommand\jQED  {\mathscr{Q.E.D.}}

\DeclareMathOperator\dom   {dom}
\DeclareMathOperator\Img   {Im}
\DeclareMathOperator\range {range}
\DeclareMathOperator\col   {Col}

\newcommand\trio  {\triangle}

\newcommand\rc    {\right\rceil}
\newcommand\lc    {\left\lceil}
\newcommand\rf    {\right\rfloor}
\newcommand\lf    {\left\lfloor}

\newcommand\lex   {<_{lex}}

\newcommand\az    {\aleph_0}
\newcommand\uaz   {^{\aleph_0}}
\newcommand\al    {\aleph}
\newcommand\ual   {^\aleph}
\newcommand\taz   {2^{\aleph_0}}
\newcommand\utaz  { ^{\left (2^{\aleph_0} \right )}}
\newcommand\tal   {2^{\aleph}}
\newcommand\utal  { ^{\left (2^{\aleph} \right )}}
\newcommand\ttaz  {2^{\left (2^{\aleph_0}\right )}}

\newcommand\n     {$n$־יה\ }

% Math A&B shortcuts
\newcommand\logn  {\log n}
\newcommand\logx  {\log x}
\newcommand\lnx   {\ln x}
\newcommand\cosx  {\cos x}
\newcommand\cost  {\cos \theta}
\newcommand\sinx  {\sin x}
\newcommand\sint  {\sin \theta}
\newcommand\tanx  {\tan x}
\newcommand\tant  {\tan \theta}
\newcommand\sex   {\sec x}
\newcommand\sect  {\sec^2}
\newcommand\cotx  {\cot x}
\newcommand\cscx  {\csc x}
\newcommand\sinhx {\sinh x}
\newcommand\coshx {\cosh x}
\newcommand\tanhx {\tanh x}

\newcommand\seq   {\overset{!}{=}}
\newcommand\slh   {\overset{LH}{=}}
\newcommand\sle   {\overset{!}{\le}}
\newcommand\sge   {\overset{!}{\ge}}
\newcommand\sll   {\overset{!}{<}}
\newcommand\sgg   {\overset{!}{>}}

\newcommand\h     {\hat}
\newcommand\ve    {\vec}
\newcommand\lv    {\overrightarrow}
\newcommand\ol    {\overline}

\newcommand\mlcm  {\mathrm{lcm}}

\DeclareMathOperator{\sech}   {sech}
\DeclareMathOperator{\csch}   {csch}
\DeclareMathOperator{\arcsec} {arcsec}
\DeclareMathOperator{\arccot} {arcCot}
\DeclareMathOperator{\arccsc} {arcCsc}
\DeclareMathOperator{\arccosh}{arccosh}
\DeclareMathOperator{\arcsinh}{arcsinh}
\DeclareMathOperator{\arctanh}{arctanh}
\DeclareMathOperator{\arcsech}{arcsech}
\DeclareMathOperator{\arccsch}{arccsch}
\DeclareMathOperator{\arccoth}{arccoth}
\DeclareMathOperator{\atant}  {atan2} 
\DeclareMathOperator{\Sp}     {span} 
\DeclareMathOperator{\rk}     {rk}
\DeclareMathOperator{\sgn}    {sgn} 

\newcommand\dx    {\,\mathrm{d}x}
\newcommand\dt    {\,\mathrm{d}t}
\newcommand\dtt   {\,\mathrm{d}\theta}
\newcommand\du    {\,\mathrm{d}u}
\newcommand\dv    {\,\mathrm{d}v}
\newcommand\df    {\mathrm{d}f}
\newcommand\dfdx  {\diff{f}{x}}
\newcommand\dit   {\limhz \frac{f(x + h) - f(x)}{h}}

\newcommand\nt[1] {\frac{#1}{#1}}

\newcommand\limz  {\lim_{x \to 0}}
\newcommand\limxz {\lim_{x \to x_0}}
\newcommand\limi  {\lim_{x \to \infty}}
\newcommand\limh  {\lim_{x \to 0}}
\newcommand\limni {\lim_{x \to - \infty}}
\newcommand\limpmi{\lim_{x \to \pm \infty}}

\newcommand\ta    {\theta}
\newcommand\ap    {\alpha}

\renewcommand\inf {\infty}
\newcommand  \ninf{-\inf}

% Combinatorics shortcuts
\newcommand\sumnk     {\sum_{k = 0}^{n}}
\newcommand\sumni     {\sum_{i = 0}^{n}}
\newcommand\sumnko    {\sum_{k = 1}^{n}}
\newcommand\sumnio    {\sum_{i = 1}^{n}}
\newcommand\sumai     {\sum_{i = 1}^{n} A_i}
\newcommand\nsum[2]   {\reflectbox{\displaystyle\sum_{\reflectbox{\scriptsize$#1$}}^{\reflectbox{\scriptsize$#2$}}}}

\newcommand\bink      {\binom{n}{k}}
\newcommand\setn      {\{a_i\}^{2n}_{i = 1}}
\newcommand\setc[1]   {\{a_i\}^{#1}_{i = 1}}

\newcommand\cupain    {\bigcup_{i = 1}^{n} A_i}
\newcommand\cupai[1]  {\bigcup_{i = 1}^{#1} A_i}
\newcommand\cupiiai   {\bigcup_{i \in I} A_i}
\newcommand\capain    {\bigcap_{i = 1}^{n} A_i}
\newcommand\capai[1]  {\bigcap_{i = 1}^{#1} A_i}
\newcommand\capiiai   {\bigcap_{i \in I} A_i}

\newcommand\xot       {x_{1, 2}}
\newcommand\ano       {a_{n - 1}}
\newcommand\ant       {a_{n - 2}}

% Linear Algebra
\DeclareMathOperator{\chr}    {char}

\newcommand\lra       {\leftrightarrow}
\newcommand\chrf      {\chr(\F)}
\newcommand\F         {\mathbb{F}}
\newcommand\co        {\colon}
\newcommand\tmat[2]   {\cl{\begin{matrix}
			#1
		\end{matrix}\, \middle\vert\, \begin{matrix}
			#2
\end{matrix}}}

\makeatletter
\newcommand\rrr[1]    {\xxrightarrow{1}{#1}}
\newcommand\rrt[2]    {\xxrightarrow{1}[#2]{#1}}
\newcommand\mat[2]    {M_{#1\times#2}}
\newcommand\tomat     {\, \dequad \longrightarrow}
\newcommand\pms[1]    {\begin{pmatrix}
		#1
\end{pmatrix}}
\newcommand\vms[1]    {\left \vert \begin{matrix}
		#1
\end{matrix}\right\vert}

% someone's code from the internet: https://tex.stackexchange.com/questions/27545/custom-length-arrows-text-over-and-under
\makeatletter
\newlength\min@xx
\newcommand*\xxrightarrow[1]{\begingroup
	\settowidth\min@xx{$\m@th\scriptstyle#1$}
	\@xxrightarrow}
\newcommand*\@xxrightarrow[2][]{
	\sbox8{$\m@th\scriptstyle#1$}  % subscript
	\ifdim\wd8>\min@xx \min@xx=\wd8 \fi
	\sbox8{$\m@th\scriptstyle#2$} % superscript
	\ifdim\wd8>\min@xx \min@xx=\wd8 \fi
	\xrightarrow[{\mathmakebox[\min@xx]{\scriptstyle#1}}]
	{\mathmakebox[\min@xx]{\scriptstyle#2}}
	\endgroup}
\makeatother


% Greek Letters
\newcommand\ag        {\alpha}
\newcommand\bg        {\beta}
\newcommand\cg        {\gamma}
\newcommand\dg        {\delta}
\newcommand\eg        {\epsi}
\newcommand\zg        {\zeta}
\newcommand\hg        {\eta}
\newcommand\tg        {\theta}
\newcommand\ig        {\iota}
\newcommand\kg        {\keppa}
\renewcommand\lg      {\lambda}
\newcommand\og        {\omicron}
\newcommand\rg        {\rho}
\newcommand\sg        {\sigma}
\newcommand\yg        {\usilon}
\newcommand\wg        {\omega}

\newcommand\Ag        {\Alpha}
\newcommand\Bg        {\Beta}
\newcommand\Cg        {\Gamma}
\newcommand\Dg        {\Delta}
\newcommand\Eg        {\Epsi}
\newcommand\Zg        {\Zeta}
\newcommand\Hg        {\Eta}
\newcommand\Tg        {\Theta}
\newcommand\Ig        {\Iota}
\newcommand\Kg        {\Keppa}
\newcommand\Lg        {\Lambda}
\newcommand\Og        {\Omicron}
\newcommand\Rg        {\Rho}
\newcommand\Sg        {\Sigma}
\newcommand\Yg        {\Usilon}
\newcommand\Wg        {\Omega}

% Other shortcuts
\newcommand\tl    {\tilde}
\newcommand\op    {^{-1}}

\newcommand\sof[1]    {\left | #1 \right |}
\newcommand\cl [1]    {\left ( #1 \right )}
\newcommand\csb[1]    {\left [ #1 \right ]}
\newcommand\ccb[1]    {\left \{ #1 \right \}}

\newcommand\bs        {\blacksquare}
\newcommand\dequad    {\!\!\!\!\!\!}
\newcommand\dequadd   {\dequad\duquad}
\newcommand\wmid      {\;\middle\vert\;}

\renewcommand\phi     {\varphi}
\newcommand\bcl[1]    {\big(#1\big)}
\renewcommand\tt      {\theta}

%! ~~~ Document ~~~

\author{שחר פרץ}
\title{\textit{ליניארית 11}}
\begin{document}
	\maketitle
	\section{\en{Changing Base}}
	\textbf{משפט. }יהי $B = \{\theta_ 1\dots \tt_n\}$ בסיס ל־$V$ וגם $B, = \{u_1 \dots u_n\}$ כך ש־$u_i = \sum\ag_{ji}\tt{j}$ לכל $i \in [n]$, אז 
	\[ \pms{\ag_{11} & \cdots & \ag_{in} \\ \vdots & \ddots & \vdots \\ a_{n1} & \cdots & \ag_{nn}} \]
	אז $M$ הפיכה אמ"מ $B'$ בסיס ל־$V$. 
	\begin{proof}\,
		
		\hfill $B'$ בסיס $\iff$ $B'$ בת"ל $\iff$ $\{[u_i]_B\}$ בת"ל $\iff$ העמודות בת"ל $\iff$ $\rk M = n$ $\iff$ $M$ הפיכה
	\end{proof}
	\textbf{הגדרה. }$B, B'$ בסיסים של $V$ מ"ו. אז $M = [id]^{B'}_B$ היא \textit{מטריצת המעבר מבסיס $B'$ ל־$B$}. 
	
	\textbf{דוגמה. }
	בעבור $P_2$ \textit{(סימון של המרצה ל־$\R_s[x]$)} אז בעבור הבסיסים $B = (1 + x, 1 -  x, 1 + x^2)$ ו־$B' = (1, x + x^2, 2x + 3x^2)$, נסמן $B = (b_1, b_2, b_3)$, אז: 
	\[ 1 = \frac{1}{2}b_1 + \frac{1}{2}b_3 + 0 b_3 \implies [1]_B = (0.5, 0.5, 0) \]
	ולכן השורה הראשונה: 
	\[ [M]_B^{B'} = \pms{0.5 & \cdots \\ 0.5 & \cdots \\ 0 & \cdots} \]
	
	אז איך נעביר בין בסיסים? כדי לעשות את המעבר $[T(x)]_B \to [T(x)]_{B'}$, נבצע כפולה מימין. 
	
	\textbf{משפט. }יהי $V$ מ"ו, כך ש־$\dim V = n$ ו־$B = \{\tt_1 \dots \tt_n\}$, $B' = \{u_1 \dots u_n\}$ בסיסים ל־$V$, אז מטריצת המעבר $M$ מ־$B'$ ל־$B$ תקיים $\forall \tt \in V \co [\tt]_B = M[\tt]_B$
	
	\begin{proof}
		ראינו עבור $U, V$ מ"וים עם $B, C$ בסיסים בהתאמה, ו־$\phi \co V \to U$, שמתקיים ש־$\forall v \in V \co [\phi]_C = [\phi]^B_C[v]_B$. לכן: 
		\[ M \cdot [\tt]_{B'} = [id]_B^{B'} [\tt]_{B'} = [id(\tt)]_B = [\tt]_B \]
	\end{proof}
	
	\textbf{סימון. }תהי $T \co V \to V$ ט"ל ו־$V$ מ"ו. נסמן $[T]^B_B =: [T]_B$. 
	
	\textbf{טענה. }תהי $T \co V \to W$ איזו', ו־$B$, $C$ בסיסים של $V$ ושל $W$. אז $[T\op]_B^C = ([T]_C^B)\op$
	
	\begin{proof}
		ראינו שעבור $\phi \co V \to U$ ו־$\psi \co U \to W$ עם $B_v, B_w$ בסיסים, מתקיים $[\psi \circ \phi]_{B_w}^{B_v} = [\psi]_{B_w}^{B_v}[\phi]_{B_U}^{B_V}$. 
		
		נראה ש־$[T\op]_B^C$ ההופכית ל־‏$[T]_C^B$. 
		\[ [T]^B_C [T\op]_B^C = [T\circ T\op]_C^C = [id_W]_C = [id_W]_C^C = I \]
		\textit{הערה: ומכיוון שהופכית הפיכה מצד אחד אמ"מ משני הצדדים והדבר שם ריבועי (הנחנו בסתר שוויון ממדים מתאים) אז זה תקין. }
	\end{proof}
	
	\textbf{משפט. }(השם של המורה: "הקשר בין מעבר בסיס למטריצה מייצגת"). יהיו $T \co V \to V$ ט"ך, $\dim V = n$, $B, B'$ בסיסים ל־$V$. אז: 
	\[ [T]_{B'} = M\op[T]_BM \]
	\begin{proof}
		\[ M\op[T]_BM = [id]_{B'}^B [T]_B^B [id]_B^{B'} = [id \circ T \circ id]_{B'}^{B'} = [T]_{B'} \]
	\end{proof}
	
	\textbf{הגדרה. }$A, B \in M_n(\F)$ מטריצות. אז $A, B$ \textit{דומות} אם $\exists M \in M_n(\F) \co A = M\op B M$. 
	
	\textbf{דוגמה. }
	\begin{enumerate}
		\item אם $A$ דומה ל־$I$, אז $A = M\op I  M = MM\op = I$
		\item אם $A$ דומה ל־$B$ אז $B$ דומה ל־$A$. 
	\end{enumerate}
	
	\textbf{משפט. }יהיו $A, B \in M_n(\F)$ מטריצות. אז $A, B$ דומות, אמ"מ קיים $V$ מ"ו, $T \co V \to V$ ו־$C, C'$ בסיסים כך ש־$A = [T]_C, \ B = [T]_{C'}$. 
	\begin{enumerate}
		\item[$\implies$] נניח $A, B $ כמתואר, אז $[T]_C = M\op[T]_CM$ עבור $M$ מט' מעבר בסיס, $A = M \op B M$. 
		
		טל: "יש פה בחור ישן". צימרמן: "אני כותב". טל: "אני מבין... תהיה איתנו". צימרמן: "אני כותב". טל: "בסדר פשוט... תכתוב". צימרמן: "אני כותב". 
		\item[$\impliedby$] יהיו $A, B$ דומות. אז $B = M\op A M$ ו־$V = \F^n$. נסמן ב־$E$ בסיס סטנדרטי ל־$\F^n$. נגדיר $T(v) = Av$ ונסמן ב־$(m_{ij})$ את האיברים ב־$M$. נסמן 
		\[ b_i' = \sum_{j = 1}^{n}m_{ji}e_j, \ B' = \{b'_i\} \] לכל $1 \le i \le n$. מהגדרת מטריצה מייצגת $M = [id]_E^{B'}$. $B'$ בסיס כי עבור $M$ הפיכה נקבל $\{b_i'\}$ בסיס. סה"כ קיבלנו $B', E, T, V$ כך ש־$A = [T]_E$ ו־$B = [T]_{B'}$. ולכן
		\[ [B] = M\op A M = [id]_{B'}^E \cdot [T]^E_E [id]_E^B = [T]_B \]
	\end{enumerate}
	
	\section{\en{determinants}}
	\textbf{משפט }\textit{(יחידות דיטרמיננטה)}\textbf{. }נניח שקיימות $d_1 \co M_n(\F) \to \F$ ו־$d_2$ שונות. תהי $A \in M_n(\F)$. אז $A = E_1 \cdots E_t \cdot B$ עבור $B$ מדורגת קאנונית ו־$E_i$ מט' אלמנטרית. 
	\begin{proof}
		אם נסמן $\phi1 \dots \phi_t$ פעולות דירוג, אז $B = (\phi_t( \cdots (\phi_1(A))))$ ולכן $\det B = \prod_{i = 1}^{t} \lg_i \det A$ (כך ש־$\lg_i$ הןא הקבוע שמתאים לפעולה)  ולכן $\det A = \lg \det B$ ($\lg$ קבוע). נפצל למקרים. 
		\begin{itemize}
			\item אם $B = I$, אז $\exists c \co \det A = c$ כדרוש. 
			\item אחרת, $B \neq I$. בגלל ש־$B \in M_n(\F)$, אם לא $I$ אז יש שורת אפסים: $\det B = 0 \implies \det A = 0 \implies d_1(A) = d_2(A)$. 
		\end{itemize}
	\end{proof}
	
	\textit{הערה: הגיון שלי להוכחה: דרטרמינטה היא 0 אם לא הופכי, ואחרת היא מכפלת הקבועים של הפעולות האלמנטריות שהיא צריכה לעשות כמו שהראו. }
	
	\textbf{למה. }תהי $A \in M_n(\F)$ עם שורת אפסים. אז $\det A = 0$. 
	\begin{proof}
		משהומשהו מולטילינאיריות אני לא מקליד את כל הבלגן הזה (דורש הרבה מטריצות כלליות כאלו עם המון 3 נקודות). השורה התחתונה שמפצלים ומקבלים $\det A = 2\det A$ כלומר $\det A = 0$. 
	\end{proof}
	
	\textbf{סימון. }$|A| := \det A$
	
	\textbf{משפט. }יהיו $A, B \in M_n(\F), \ \det \co M_n(\F) \to \F$ דט'. אז $\det(AB) = \det A \cdot \det B$
	\begin{proof}
		נסמן $A = E_1 \dots E_s \cdot A_1$ עבור $E_i$ מט' אלמנטריות, $A_1$ מדורגת קאנונית מתאימה ל־$A$. אז $|AB| = |E_1 \cdot A_1 B| = \prod_{i = 1}^{s}|E_i| \cdot |A_1 B|$ (נסתכל על $A, B$ בתור תוצאה של הפעולות האלמנטריות). אם $A_1 = I$, אז $|B| = |IB| = |A_1B|$, ולכן $|AB| = (\prod |E_I|) |B| = |A| |B|$. אחרת, $A_1$ בעלת שורת אפסים. 
		
		אחרת ל־$A_1$ שורת אפסים, כלומר $|AB| = \lg |A_1B| = 0$ כי $|A_1B| = 0$. סה"כ $|A| |B| = |0| |B|$. 
	\end{proof}
	
	\textbf{טענה. }תהי $A \in M_n(\F)$. אז $A$ הפיכה אמ"מ $|A| \neq 0$. 
	
	\begin{proof}
		נציב את $A$ באמצעות $A = \prod E_i \cdot A'$ עבור $A'$ קאנונית... 
		\[ |A| \neq 0 \iff |\prod E_i| |A'| \neq 0 \iff |A'| \neq 0 \iff A' = I \iff \text{הפיכה}\, A \]
	\end{proof}
	
	\textbf{משפט. }$|A| = |A^T|$. 
	\begin{proof}
		נפרק כרגיל ל־$A = \prod E_i \cdot A'$. אם $A$ לא הפיכה, אז $\det A = 0$ וגם $A^T$ לא הפיכה כלומר $\det A^T = 0$ ואכן $\det A = \det A^T$. 
		אחרת, $A' = I$. נראה ש־$E_i = E_i \cdot t$ לכל $i$. עבור הכפלת שורה בסקלר [רק פילוג למקרים שאין לי כוח להעתיק]. 
	\end{proof}
	
	\textbf{הגדרה. }יהי $A \in M_n(\F)$. נניח $1 \le i, j \le n$. אז ה\textit{מינור} $A_{ij}$ או $M_{ij}(A)$ היא המטעיצה המתקבלת מ־$A$ ע"י מחיקת השורה ה־$i$ והעמודה ה־$j$. 
	
	\textbf{משפט. }קיימת פאקינג דטרמיננטה. 
	
	\begin{proof}
		נוכיח ש־: 
		\[ |A| = \sum_{j = 1}^{n}\cl{(-1)^{n + j}\ag_{nj}} |A_{nj}| =: \mathrm{dn} \]
		(הערה: $A_{nj}$ הוא המינור ו־$\ag$ הקורדינאטות)
		נראה שאכן $\mathrm{dn}$ היא דיטרמיננה לכל $n$. נוכיח באינדוקציה. 
		\textbf{בסיס. }
		\begin{enumerate}
			\item $|I| = 1$. $a_{11} = 1 \iff ... AI \implies |A| = 1$
			\item מולטי־ליניאריות. אפשר לעשות על דף. 
			\item שתי שורות שוות גורר דט' הפיכה. לא קיים, באופן ריק. 
		\end{enumerate}
		\textbf{צעד. }
		\begin{enumerate}
			\item $|I| = 1$. נראה שאם $A = I_n$ אז $|A| = 1$
			\[ |A| = \sum_{j = 1}^{n}(-1)^{n + j}a_{nj} |A_{nj}| \]
			נשים לב שעבור $j \neq n$ נקבל שב־$A_{nj}$ יש שורת אפסים. זאת כי "העפנו" 2 איברי םפותחים אבל יש יותר שורות מאיברים פותחים. אזי יש שורת אפסים. לכן נקבל שהלעיל שווה ל־$(-1)^{n + n} a_{nn}|A_{nn}|$. בגלל ש־$a_{nn} = 1$ ובאינדוקציה $A_{nn} = 1$, וסה"כ נקבל שוויון ל־$1$. 
			\item מולטי־ליניאריות. ספוילר, עומד להיות כיף. 
			\[ \vms{\ag_{11} && \ag_{1n} \\
				\vdots && \vdots \\
				\ag \ag_{il} + \bg \bg_{i1} && \ag_{in} + \bg \bg_{in} \\
				\vdots && \vdots \\ \ag_{ni} & \quad \quad & \ag_{ni}}, \pms{\ag_{11} & } \]
		\end{enumerate}
		
	\end{proof}
	

\end{document}
