%! ~~~ Packages Setup ~~~ 
\documentclass[]{article}
\usepackage{lipsum}


% Math packages
\usepackage[usenames]{color}
\usepackage{forest}
\usepackage{ifxetex,ifluatex,amsmath,amssymb,mathrsfs,amsthm,witharrows,mathtools}
\WithArrowsOptions{displaystyle}
\renewcommand{\qedsymbol}{$\blacksquare$} % end proofs with \blacksquare. Overwrites the defualts. 
\usepackage{cancel,bm}
\usepackage[thinc]{esdiff}


% tikz
\usepackage{tikz}
\newcommand\sqw{1}
\newcommand\squ[4][1]{\fill[#4] (#2*\sqw,#3*\sqw) rectangle +(#1*\sqw,#1*\sqw);}


% code 
\usepackage{listings}
\usepackage{xcolor}

\definecolor{codegreen}{rgb}{0,0.35,0}
\definecolor{codegray}{rgb}{0.5,0.5,0.5}
\definecolor{codenumber}{rgb}{0.1,0.3,0.5}
\definecolor{codeblue}{rgb}{0,0,0.5}
\definecolor{codered}{rgb}{0.5,0.03,0.02}
\definecolor{codegray}{rgb}{0.96,0.96,0.96}

\lstdefinestyle{pythonstylesheet}{
	language=Python,
	emphstyle=\color{deepred},
	backgroundcolor=\color{codegray},
	keywordstyle=\color{deepblue}\bfseries\itshape,
	numberstyle=\scriptsize\color{codenumber},
	basicstyle=\ttfamily\footnotesize,
	commentstyle=\color{codegreen}\itshape,
	breakatwhitespace=false, 
	breaklines=true, 
	captionpos=b, 
	keepspaces=true, 
	numbers=left, 
	numbersep=5pt, 
	showspaces=false,                
	showstringspaces=false,
	showtabs=false, 
	tabsize=4, 
	morekeywords={as,assert,nonlocal,with,yield,self,True,False,None,AssertionError,ValueError,in,else},              % Add keywords here
	keywordstyle=\color{codeblue},
	emph={object,type,isinstance,copy,deepcopy,zip,enumerate,reversed,list,set,len,dict,tuple,print,range,xrange,append,execfile,real,imag,reduce,str,repr,__init__,__add__,__mul__,__div__,__sub__,__call__,__getitem__,__setitem__,__eq__,__ne__,__nonzero__,__rmul__,__radd__,__repr__,__str__,__get__,__truediv__,__pow__,__name__,__future__,__all__,},          % Custom highlighting
	emphstyle=\color{codered},
	stringstyle=\color{codegreen},
	showstringspaces=false,
	abovecaptionskip=0pt,belowcaptionskip =0pt,
	framextopmargin=-\topsep, 
}
\newcommand\pythonstyle{\lstset{pythonstylesheet}}
\newcommand\pyl[1]     {{\lstinline!#1!}}
\lstset{style=pythonstylesheet}

\usepackage[style=1,skipbelow=\topskip,skipabove=\topskip,framemethod=TikZ]{mdframed}
\definecolor{bggray}{rgb}{0.85, 0.85, 0.85}
\mdfsetup{leftmargin=0pt,rightmargin=0pt,innerleftmargin=15pt,backgroundcolor=codegray,middlelinewidth=0.5pt,skipabove=5pt,skipbelow=0pt,middlelinecolor=black,roundcorner=5}
\BeforeBeginEnvironment{lstlisting}{\begin{mdframed}\vspace{-0.4em}}
	\AfterEndEnvironment{lstlisting}{\vspace{-0.8em}\end{mdframed}}


% Deisgn
\usepackage[labelfont=bf]{caption}
\usepackage[margin=0.6in]{geometry}
\usepackage{multicol}
\usepackage[skip=4pt, indent=0pt]{parskip}
\usepackage[normalem]{ulem}
\forestset{default}
\renewcommand\labelitemi{$\bullet$}
\usepackage{titlesec}
\titleformat{\section}[block]
{\fontsize{15}{15}}
{\sen \dotfill (\thesection) \she}
{0em}
{\MakeUppercase}
\usepackage{graphicx}
\graphicspath{ {./} }


% Hebrew initialzing
\usepackage[bidi=basic]{babel}
\PassOptionsToPackage{no-math}{fontspec}
\babelprovide[main, import, Alph=letters]{hebrew}
\babelprovide[import]{english}
\babelfont[hebrew]{rm}{David CLM}
\babelfont[hebrew]{sf}{David CLM}
\babelfont[english]{tt}{Monaspace Xenon}
\usepackage[shortlabels]{enumitem}
\newlist{hebenum}{enumerate}{1}

% Language Shortcuts
\newcommand\en[1] {\begin{otherlanguage}{english}#1\end{otherlanguage}}
\newcommand\sen   {\begin{otherlanguage}{english}}
	\newcommand\she   {\end{otherlanguage}}
\newcommand\del   {$ \!\! $}
\newcommand\ttt[1]{\en{\footnotesize\texttt{#1}\normalsize}}

\newcommand\npage {\vfil {\hfil \textbf{\textit{המשך בעמוד הבא}}} \hfil \vfil \pagebreak}
\newcommand\ndoc  {\dotfill \\ \vfil {\begin{center} {\textbf{\textit{שחר פרץ, 2024}} \\ \scriptsize \textit{נוצר באמצעות תוכנה חופשית בלבד}} \end{center}} \vfil	}

\newcommand{\rn}[1]{
	\textup{\uppercase\expandafter{\romannumeral#1}}
}

\makeatletter
\newcommand{\skipitems}[1]{
	\addtocounter{\@enumctr}{#1}
}
\makeatother

%! ~~~ Math shortcuts ~~~

% Letters shortcuts
\newcommand\N     {\mathbb{N}}
\newcommand\Z     {\mathbb{Z}}
\newcommand\R     {\mathbb{R}}
\newcommand\Q     {\mathbb{Q}}
\newcommand\C     {\mathbb{C}}

\newcommand\ml    {\ell}
\newcommand\mj    {\jmath}
\newcommand\mi    {\imath}

\newcommand\powerset {\mathcal{P}}
\newcommand\ps    {\mathcal{P}}
\newcommand\pc    {\mathcal{P}}
\newcommand\ac    {\mathcal{A}}
\newcommand\bc    {\mathcal{B}}
\newcommand\cc    {\mathcal{C}}
\newcommand\dc    {\mathcal{D}}
\newcommand\ec    {\mathcal{E}}
\newcommand\fc    {\mathcal{F}}
\newcommand\nc    {\mathcal{N}}
\newcommand\sca   {\mathcal{S}} % \sc is already definded
\newcommand\rca   {\mathcal{R}} % \rc is already definded

\newcommand\Si    {\Sigma}

% Logic & sets shorcuts
\newcommand\siff  {\longleftrightarrow}
\newcommand\ssiff {\leftrightarrow}
\newcommand\so    {\longrightarrow}
\newcommand\sso   {\rightarrow}

\newcommand\epsi  {\epsilon}
\newcommand\vepsi {\varepsilon}
\newcommand\vphi  {\varphi}
\newcommand\Neven {\N_{\mathrm{even}}}
\newcommand\Nodd  {\N_{\mathrm{odd }}}
\newcommand\Zeven {\Z_{\mathrm{even}}}
\newcommand\Zodd  {\Z_{\mathrm{odd }}}
\newcommand\Np    {\N_+}

% Text Shortcuts
\newcommand\open  {\big(}
\newcommand\qopen {\quad\big(}
\newcommand\close {\big)}
\newcommand\also  {\text{, }}
\newcommand\defi  {\text{ definition}}
\newcommand\defis {\text{ definitions}}
\newcommand\given {\text{given }}
\newcommand\case  {\text{if }}
\newcommand\syx   {\text{ syntax}}
\newcommand\rle   {\text{ rule}}
\newcommand\other {\text{else}}
\newcommand\set   {\ell et \text{ }}
\newcommand\ans   {\mathit{Ans.}}

% Set theory shortcuts
\newcommand\ra    {\rangle}
\newcommand\la    {\langle}

\newcommand\oto   {\leftarrow}

\newcommand\QED   {\quad\quad\mathscr{Q.E.D.}\;\;\blacksquare}
\newcommand\QEF   {\quad\quad\mathscr{Q.E.F.}}
\newcommand\eQED  {\mathscr{Q.E.D.}\;\;\blacksquare}
\newcommand\eQEF  {\mathscr{Q.E.F.}}
\newcommand\jQED  {\mathscr{Q.E.D.}}

\newcommand\dom   {\mathrm{dom}}
\newcommand\Img   {\mathrm{Im}}
\newcommand\range {\mathrm{range}}

\newcommand\trio  {\triangle}

\newcommand\rc    {\right\rceil}
\newcommand\lc    {\left\lceil}
\newcommand\rf    {\right\rfloor}
\newcommand\lf    {\left\lfloor}

\newcommand\lex   {<_{lex}}

\newcommand\az    {\aleph_0}
\newcommand\uaz   {^{\aleph_0}}
\newcommand\al    {\aleph}
\newcommand\ual   {^\aleph}
\newcommand\taz   {2^{\aleph_0}}
\newcommand\utaz  { ^{\left (2^{\aleph_0} \right )}}
\newcommand\tal   {2^{\aleph}}
\newcommand\utal  { ^{\left (2^{\aleph} \right )}}
\newcommand\ttaz  {2^{\left (2^{\aleph_0}\right )}}

\newcommand\n     {$n$־יה\ }

% Math A&B shortcuts
\newcommand\logn  {\log n}
\newcommand\logx  {\log x}
\newcommand\lnx   {\ln x}
\newcommand\cosx  {\cos x}
\newcommand\cost  {\cos \theta}
\newcommand\sinx  {\sin x}
\newcommand\sint  {\sin \theta}
\newcommand\tanx  {\tan x}
\newcommand\tant  {\tan \theta}
\newcommand\sex   {\sec x}
\newcommand\sect  {\sec^2}
\newcommand\cotx  {\cot x}
\newcommand\cscx  {\csc x}
\newcommand\sinhx {\sinh x}
\newcommand\coshx {\cosh x}
\newcommand\tanhx {\tanh x}

\newcommand\seq   {\overset{!}{=}}
\newcommand\slh   {\overset{LH}{=}}
\newcommand\sle   {\overset{!}{\le}}
\newcommand\sge   {\overset{!}{\ge}}
\newcommand\sll   {\overset{!}{<}}
\newcommand\sgg   {\overset{!}{>}}

\newcommand\h     {\hat}
\newcommand\ve    {\vec}
\newcommand\lv    {\overrightarrow}
\newcommand\ol    {\overline}

\newcommand\mlcm  {\mathrm{lcm}}

\DeclareMathOperator{\sech}   {sech}
\DeclareMathOperator{\csch}   {csch}
\DeclareMathOperator{\arcsec} {arcsec}
\DeclareMathOperator{\arccot} {arcCot}
\DeclareMathOperator{\arccsc} {arcCsc}
\DeclareMathOperator{\arccosh}{arccosh}
\DeclareMathOperator{\arcsinh}{arcsinh}
\DeclareMathOperator{\arctanh}{arctanh}
\DeclareMathOperator{\arcsech}{arcsech}
\DeclareMathOperator{\arccsch}{arccsch}
\DeclareMathOperator{\arccoth}{arccoth}
\DeclareMathOperator{\atant}  {atan2} 


\newcommand\dx    {\,\mathrm{d}x}
\newcommand\dt    {\,\mathrm{d}t}
\newcommand\dtt   {\,\mathrm{d}\theta}
\newcommand\du    {\,\mathrm{d}u}
\newcommand\dv    {\,\mathrm{d}v}
\newcommand\df    {\mathrm{d}f}
\newcommand\dfdx  {\diff{f}{x}}
\newcommand\dit   {\limhz \frac{f(x + h) - f(x)}{h}}

\newcommand\nt[1] {\frac{#1}{#1}}

\newcommand\limz  {\lim_{x \to 0}}
\newcommand\limxz {\lim_{x \to x_0}}
\newcommand\limi  {\lim_{x \to \infty}}
\newcommand\limh  {\lim_{x \to 0}}
\newcommand\limni {\lim_{x \to - \infty}}
\newcommand\limpmi{\lim_{x \to \pm \infty}}

\newcommand\ta    {\theta}
\newcommand\ap    {\alpha}

\renewcommand\inf {\infty}
\newcommand  \ninf{-\inf}

% Combinatorics shortcuts
\newcommand\sumnk     {\sum_{k = 0}^{n}}
\newcommand\sumni     {\sum_{i = 0}^{n}}
\newcommand\sumnko    {\sum_{k = 1}^{n}}
\newcommand\sumnio    {\sum_{i = 1}^{n}}
\newcommand\sumai     {\sum_{i = 1}^{n} A_i}
\newcommand\nsum[2]   {\reflectbox{\displaystyle\sum_{\reflectbox{\scriptsize$#1$}}^{\reflectbox{\scriptsize$#2$}}}}

\newcommand\bink      {\binom{n}{k}}
\newcommand\setn      {\{a_i\}^{2n}_{i = 1}}
\newcommand\setc[1]   {\{a_i\}^{#1}_{i = 1}}

\newcommand\cupain    {\bigcup_{i = 1}^{n} A_i}
\newcommand\cupai[1]  {\bigcup_{i = 1}^{#1} A_i}
\newcommand\cupiiai   {\bigcup_{i \in I} A_i}
\newcommand\capain    {\bigcap_{i = 1}^{n} A_i}
\newcommand\capai[1]  {\bigcap_{i = 1}^{#1} A_i}
\newcommand\capiiai   {\bigcap_{i \in I} A_i}

\newcommand\xot       {x_{1, 2}}
\newcommand\ano       {a_{n - 1}}
\newcommand\ant       {a_{n - 2}}

% Other shortcuts
\newcommand\tl    {\tilde}
\newcommand\op    {^{-1}}

\newcommand\sof[1]    {\left | #1 \right |}
\newcommand\cl [1]    {\left ( #1 \right )}
\newcommand\csb[1]    {\left [ #1 \right ]}

\newcommand\bs    {\blacksquare}

%! ~~~ Document ~~~

\author{שחר פרץ}
\title{ליניארית 1א 5}
\begin{document}
	\maketitle
	\section{\en{Dimensions}}
	תזכורת: בהינתן $V$ מרחב וקטורי מעל $F$ שדה, ובהינתן $B_1, B_2$ בסיסים, אז $|B_1| = |B_2|$. ניזכר שבסיס הוא אוסף וקטורים כך שקיים ויחיד צירוף ליניארי מהוקטורים ב־$B$. כלומר בסיס $B$ במ"ו $V$ אם: 
	\[ \forall v \in V \exists! (\lambda _i)^{|B|}_{i =  1} \colon v = \{b_i\}_i \]
	
	\textbf{הגדרה. }יהיו $v_1, dots v_2 \in V$. השדה תלוי הליניארית אם קיימים $\lambda_1 \dots \lambda_s$ כך שאחד מהם שונה מ־$0$ כך ש־$\sum_{i = 1}^{s}\lambda_iv_i = 0$
	
	\textbf{סדרה בלתי־תלויה ליניארית} (בת"ל) אם היא לא תלויה ליניארית. שקול לקיום $\sum \lambda_iv_ = 0 \implies \forall 1 \le i \le s \colon \lambda _i = 0$
	
	לדוגמה, בהינתן $v_1 = (1, 2), v_2 = (-2, 3)$ הם יהיו תלויים ליניארית רק אם למשוואה למעלה (היא משוואה הומוגנית) יהיה פתרון לא טרוויאלי. במקרה הזה: 
	\[ \lambda_1v_2 + \lambda_2v_2 \iff \lambda_1(1, 2) + \lambda_2(-2, 3) = (0, 0) \iff \begin{cases}
		\lambda_1 - 2\lambda_2 \\
		2\lambda_1 + 3\lambda_2
	\end{cases} \iff \cl{\begin{matrix}
		1 & 2\\
		2&3
		\end{matrix}\middle\vert \begin{matrix}
		0 \\
		0
	\end{matrix}} \]
	\textbf{מסקנה. }בהינתן $v_1, \dots v_n \in F^n$, ו־$A$ מטריצת העמודות שלה, אז הסדרה בה"ל אמ"מ בצורה הקאנונית ששקולה ל־$A$ בכל שורה יש איבר פותח. 
	
	\textbf{טענה. }יהי $V \in F^n$, הסדרה $e_1, \dots e_n$ היא בת"ל והיא בסיס. 
	\begin{proof}
		נראה ש־$\{e_i\}^n_{i = 0}$ בהת"ל כלומר נניח $\{\lambda_i\}^n_{i = 1}$, $\lambda_i \in F$כך ש־$\sum \lambda_ie_i = 0$ ונראה ש־$\lambda_i = 0$. אכן נקבל: 
		\[ \cl{\begin{matrix}
				1 &\dots &0 \\
				0 &\ddots & 0\\
				0 &\dots  &1
		\end{matrix}\middle\vert \begin{matrix}
		0 \\ 0 \\ 0
	\end{matrix}} \]
ולכן עבור כל $1 \le i \le n$ נקבל $\lambda_i \cdot 1 = 0$ כרצוי, ולכן בת"ל. 

זהו גם בסיס: נראה שלכל $v \in V$ קיים ויחיד $\lambda_i, \dots, \lambda_n$ נבחר $\lambda_i = v_i$ ונקבל ייצוג. נראה שיחיד, כלומר $\lambda_i$ בהכרח שווה ל־$v_i$. בהכרח: 
\[ \sum \lambda_ie_i = v \]
(כי בכל קורדינרטה נקבל את השוואה הזאת). כלומר $\forall i. \lambda_i \cdot 1 = v_i$. 
	\end{proof}
	\textit{הערה. }תהי $U \subseteq V$ קבוצה. אז $U$ תמ"ו אמ"מ $\forall \lambda_1, \lambda_2 \in F, u, n \in U$  יתקיים $\lambda_i u + \lambda_i v \in U$ וגם $U$ לא ריקה. 
	
	\textbf{טענה. }בהינתן $U \subseteq V$ תמ"ו ובהינתן $u_1, \dots u_s \in U$, אז כל צירוף ליניארי שלהם ב־$U$. 
	\begin{proof}
		יהיו $\lambda_1, \dots \lambda_2 \in F$. נראה $\sum \lambda_i u_i \in U$ באינדוקציה על $s$. עבור $s = 1$, נקבל ששייך מתכונת הסגירות.  עבור $s < 1$, אז: 
		\[ \sum_{i = 1}^{s}\lambda_iu_i = \underbrace{\sum_{i = 1}^{s - 1}\lambda_iu_i}_{\mathclap{\in U (\text{ה.א.})}} + \lambda_su_s \]
		ובסכום של שניהם מסגירות כרצוי. 
	\end{proof}
	\textbf{הגדרה. }בהינתן $x = v_1, \dots v_2$ קבוצת וקטורים, אז $\{\sum_{i = 1}^{s}\lambda_iv_i \mid \lambda_i \i nF\} = \mathrm{span(X)}$. בפרט, נגדיר $\mathrm{span}(\emptyset) = \emptyset$ (המרצה לא בטוח לגבי מקרה הקצה הזה). 
	
	כדי למצוא מימד, יהיה איזשהו מתח – איך יוצרים את כל המרחב, כלומר, איך ה־$\mathrm{span}$ הוא כל המרחב, ומצד שני מתי יש לנו דברים מיותרים – כלומר מתי יש תלות ליניארית בין שני דברים. 
	
	\textbf{טענה. }יהי $V$ מ"ו. כן $x = (v_1, \dots v_22) \subseteq V$. אזי, $\mathrm{span}(x)$ הוא התמ"ו המינימלי שמכיל את $X$. 
	
	\begin{proof}
		נראה שזהו תמ"ו. נשתמש בהגדרה השקולה. נסמן את $\mathrm{span}(X) = T$. אז $T \neq \emptyset$ כי $v_1 \in T$. נראה סגירות. אכן, יהי $\sum \alpha_iv_i = u \in T$, $\sum\beta_iv_i = w \in T$, $\alpha, \beta \in F$. נראה ש־$\lambda_1 u + \lambda_2 w \in T$. ואכן: 
		\[ \lambda_1 v_i \sum \alpha_iv_i + \lambda_2 v_i = \sum(\lambda_1 \ap + \lambda_2\beta_i)v_) \] וזה ב־$T$ כי צירוף ליניארי של $X$. 
		
		נוכיח שמינימלי, נראה שכל $Y$ תמ"ו שמכיל את $X$ מכיל את $T$. ואכן יהי $t \in T$, נראה $t \in Y$. $t$ צירוף ליניארי של $X$ (מטענה קודמת) וגם $Y$ מ"ו שמכיל את $X$ ולכן $t \in Y$. 
	\end{proof}
	
	גרסה יותר חזקה של הטענה הזו: 
	\[ \mathrm{span}(X) =\bigcap_{\mathclap{X \subseteq  U \subseteq V, \ \text{U ת"מ}}} U \]
	
	\textbf{דוגמה. }נגדיר $V = (1, 0) \in \R^2$. עבורו, $\mathrm{span}(1, 0) = \{(x, 0) \mid x \in \R\}$ $ \mathrm{span}((1, 1)) = \{(\lambda, \lambda) \mid \lambda \in F\}$. 
	
	\textbf{הגדרה. }בהינתן $V$ מ"ו, $X \subseteq X$, נאמר ש־$X$ \textit{פורש} את $V$ אם $V = \mathrm{span}(X)$. לעיתים, נאמר ש־$X$ \textit{קבוצת יוצרים} של $V$. 
	
	\textbf{הגדרה. }בהינתן $V$ מ"ו. נאמר ש־$V$ \textit{נוצר סופית} אם קיימת $X \subseteq V$ סופית כך ש־$X$ פורשת את $V$. 
	
	\textit{הערה. }בסיס הוא גם פורש. 
	
	לדוגמה, $F^n$ נוצר סופית בעבור הבסיס הסטנדרטי, $e_1, \dots, e_n$, ו־$\mathrm{Func}(\R, \R)$ לא כי הוא מעוצמה $\R^\R$ אך: 
	\[ |\mathrm{span}(X)| = |\{\sum \lambda_i\phi_i \mid \lambda i \in \R\} = |\R^n| = |\R| = \taz \neq \tal = |\R^\R|\]
	
	\textbf{טענה. }$V$ מ"ו נוצר סופית, $X\subseteq V$ פורשת סופית. נגדיר $|X| = n \in \N\cup \{0\}$. כל סדרה בת"ל ב־$V$ גודלה לכל היותר. 
	
	\begin{proof}
		תהי $u_1, \dots, u_n, u_{n + }$1 סדרה, נראה שתלויה. 
		\[ u_i = \sum_{j = 1}^{n}\alpha_{i, j}v_{i, j}\ \] כי $\mathrm{span}(X) \ni u_i$ונראה קיימים $\lambda \dots \lambda_{n + 1}$ כך ש־$\sum_{i = 1}^{n + 1}\lambda_iu_i = 0 \land \exists \le n + 1 \colon \lambda_i \neq 0$. נמשיך לפתח: 
		\begin{align*}
			\sum_{i = 1}^{n + 1}\lambda_iu_i& = \sum_{i = 1}^{n + 1} \cdot (\sum_{j = 1}^{n}\alpha_{ij}v_j) \\
			&= \sum_{h = 1}^{n}v_j \cdot \sum_{i = 1}^{n + 1}\lambda_i \cdot \alpha_{ij}
		\end{align*}
		נראה כי קיימים $\{x_i\}_{i = 0}^{n + 1}$ לא טרוויאלים כך ש־$\sum_{i = 1}^{n + 1}\lambda r \cdot \ap_{ij}' i, j \in \N$. זוהי מערכת משוואות הומוגנית עם $n + 1$ נעלמים ו־$n$ משוואות. אכן, מדובר במערכת משוואות הומוגנית עם יורת נעלמים ממשוואוךת – יש יותר מפתרון אחד. סה"כ יש פתרון לא טרוויאלי. 
	\end{proof}
	
	\textbf{סימון. }בהינתן $X \subseteq V$ שדות וקטורים $(v_1, \dots v_n)$ ו־$u \in v$ וקטור. אז: 
	\[ (v_1, \dot , v_, u) = X X \cup \{u\} \]
	\textbf{למה. }$X$ בת"ל ב־$V$. נרצה לכויח $u \in V \setminus \mathrm{span}(X)$ גורר $X \cup\{u\}$ בת"ל. 
	\begin{proof}
		נניח בשלילה ש־$X \cup \{u\}$ צלויה. אז: 
		\[ \lambda_1, \dots, \lambda_s, \lambda \in F, \ v, \dots v_s, \in X \]
		לא כולם אפס;
		\[ \sum_{i = 1}^{s}\lambda_iv_i + \lambda_iu_i = 0 \]
	אם $\lambda = 0$, אז $\sum \lambda_iv_i = 0$ עבור $\{\lambda_i\}$ לא טרוויאלי. אזי סה"כ $X$ בת"ל. 
	
	אחרת, $\lambda \ne q0$. ואז $\sum \lambda_iv_i = -\lambda u$ ולכן $\sum \frac{\lambda_i}{-\lambda} \cdot v_i = u$. סתירה לכך ש־$u \in V \setminus \mathrm{span}(X)$. 
	\end{proof}
	
	\textbf{טענה. }$V$ מ"ו נוצר סופית, ו־$X$ סדרת יוצרים, $v_1, \dots, v_m$ בת"ל, ולכן קיימים $v_{m + 1} \dots v_n \in X$ כך ש־$v_1, \dots v_m, v_{m + 1}, \dots v_n$ פורשת ובהת"ל. 
	
	\begin{proof}
		נסמן $N = |x| - m$. לכן $N \ge 0$. נוכיח באינדוקציה על $N$. 
		\begin{enumerate}
			\item אם $N = 0$: אם $X \subseteq \mathrm{span}(v_1, \dots, v_m)$ אז $\mathrm{span}(X) \le \mathrm{span}(v_1, \dots, v_m)$ (מטענה קודמת על $X$ קבוצת וקטורים ומרחב וקטורי שמכיל אותה). לכן, $v_1, \dots, v_m$ בת"ל ופורשת. 
			 אחרת, $X \nsubseteq \mathrm{span}(v_1, \dots, v_m)$: אזי קיים $x \in X \colon x \notin \mathrm{span}v_1, \dots, v_m$ ולכן "אופס, משהו פה לא עובד" (המרצה). אז מצאנו סדרה בגודל $|X| + 1$ בת"ל למרות שיש סדרה בגודל $|X|$ שהיא פורשת, לפי טענה קודמת, כל סדרה בת"ל תהיה קטנה ממנה. 
			 
			 \item נראה עבור $N > 0$. אם $X \subseteq \mathrm{span}(v_1, \dots, v_m)$, אז מאנו סדרה רצויה כמו בבסיס. 
			 אחרת, $X \nsubseteq \mathrm{span}(v_1, \dots, v_m) := T$, וקיים $x \in X \setminus T$ ולכן $(v_1, \dots, v_m, x)$ בת"ל ומהנחת האינדוקציה ניתן להוסיף וקטורים מ־$X$ עד שתהיה בת"ל. 
		\end{enumerate}
	\end{proof}
	
	\textbf{משפט. }$B = (v_1, \dots, v_s) \subseteq V$. אז $B$ בסיס אמ"מ הוא פורש ובת"ל. 
	
	\begin{proof}\, 
		\begin{enumerate}
			\item[$\impliedby$] יהי $v \in V$. נראה שיש צירוף ליניארי של $B$ ששווה ל־$V$, ושהוא יחיד. 
			\begin{enumerate}
				\item[קיום:] נתון ש־$B$ פורש ולכן $\forall v \in V. v \in \mathrm{span}(B)$ (קיום צירוף ליניארי מהגדרת ה־$\mathrm{span}$). 
				\item[יחידות: ]נניח קיום בלה בלה כך ש־־: $\sum \alpha_i v_i = \sum \beta_i v_i$. אזי $\sum(\alpha_i - \beta_i)\v_i = 0$ ולכן $\{\ap_i, \beta_i\}$ סדרת סקלרים לא טרוויאלית ובפרט כל הערכים ב־$B$ בת"ל. 
			\end{enumerate}
			\item[$\implies$]\begin{enumerate}
				\item[פורש: ]נראה ש־$\mathrm{span}(B) = V$ ואכן יהי $v \in V$ נראה ש־$v \in \mathrm{span}(B)$ (בורר ש־$\mathrm{span} \subseteq V$). ל־$V$ יש צירוף ליניארי של $B$. בגלל ש־$B$ בסיס: 
				\[ \exists \{\lambda_i\}^{|B|} \sum \lambda_iv_i = v \implies v \in \mathrm{span}(B) \]
				\item[בת"ל: ]נניח $\sum^{|B|}_{i = 1}v_i = 0$. מיחידות – הצירוף הליניארי היחיד שייקיים את זה הוא $\forall 1 \le i \le |B|. \lambda_i = 0$. 
			\end{enumerate}
		\end{enumerate}
	\end{proof}
	
	\textbf{דוגמה. }בהינתן $V_1 = (1, 2, 3), \ v_2 = (3, 2, 1), V_3 = (3, 3, 4)$. נבדוק שהאם בסיס של $\Q^3$ מעל $\Q$. 
	נקבל את המטריצה: 
	\[ \begin{pmatrix}
		1 & 3 & 3 \\
		2 & 2 & 3 \\
		3 & 1 & 4
	\end{pmatrix} \to \begin{pmatrix}
	1 & 0 & 0 \\
	0 & 1 & 0\\
	0 & 0 & 1 
	\end{pmatrix} \]
	אם לא יהיה איבר פותח, זה לא יהיה בסיס. טריק למחשבה: בהינתן הוקטורים $V = (1, 0, 0), E = (0, 1, 0)$. זה קטן מדי – לכן לא בסיס. האם הוא תלוי ליניארית? 
	\[ \begin{pmatrix}
		1 & 0 \\
		0 & 1 \\
		0 & 0
	\end{pmatrix} \]
	למרות שהיא לכאורה לא כמו שאנחנו רגילים – עדיין הפתרון היחיד הוא $0$ והם בת"לים.
	
	\subsection{idk}
	בהינתן וקטורים $v_1, \dots v_m \in V = F^m$. נתבונן במטריצת השורות שלהם: 
	\[ \begin{pmatrix}
		\cdots & v_1 &\cdots \\
		\vdots &  &\vdots \\
		\cdots & v_m &\cdots
	\end{pmatrix} \]
	כל פעולה אלמנטרית משמרת את ה־$\mathrm{span}$ שלהם (=המרחב שנפרש ע"י השורות)
	\begin{proof}
		ניקח מטריצה $A$ ונעשה פעולה אלמנטרית ונקבל $B$ שורות. $B$ תהינה צירוף ליניארי של שורות $A$  – $ \mathrm{span}(B) \subseteq \mathrm{span}(A)$. נשים לב שעבור $A, B$ קיימת פעולה אלמנטרית שמעבירה מ־$B$ ל־$A$. לכן $\mathrm{span}(A) \subseteq \mathrm{span}(B)$, וסה"ר $\mathrm{span}(A) = \mathrm{span}(B)$. 
	\end{proof}
	
	\textbf{מסקנה. }$V$ מ"ו, $X$ סדרת יוצרים.
	\begin{enumerate}
		\item כל סדרה בת"ל ניתן להשלמים ע"י $X$. 
		\item בעבור $B_1, B_2$ בסיסים $|B_1| = |B_2|$
		\begin{proof}
			יהיו $B_1, B_2$ בסיסים. נניח בשלילה בה"כ $|B_1| < |B_2|$. אבל: $B_1$ בסיס ולכן פורש ובת"ל. כל סדרה ב‏$V$ שגדולה באורכה מ־$B_1$ לא בת"ל. סתירה לכך ש־$B_1$ בת"ל, ונסיק $|B_1| \not< |B_2$ וזו סתירה. 
		\end{proof}
	\end{enumerate}
	
	\textbf{הגדרה. }בהינתן $V$ מ"ו ו־$B$ בסיס סופי. אז $|B| := \dim V$ ("\textit{מימד}"). 
	
	\textbf{משפט. }בהינתן $V$ מ"ו ו־$v_1, \dots v_s$ סדרת יוצרים. אז ניתן לצמצם אותה לבסיס. 
	\begin{proof}
		"אני אעשה את ההוכחה הזו חצי פורמלית ואתן לכם להשלים את זה" (אני לא יודע לעשות אינדוקציה הפוכה) ~ המרצה. 
		אם $v_1, \dots v_s$ בת"ל – אז סיימנו. אחרת, מתקיים $\sum \lambda_i v_i = 0$ עבור $\{\lambda_i\}$ לא כולם $0$. נניח $\lambda_1 \neq 0$, אז $\sum_{i = 2}\lambda_iv_i = \lambda_1v_2$. ולכן, $ \mathrm{span}(v_2, \dots v_s) = \mathrm{span}(v_1, \dots, v_s)$. באינדוקציה אפשר לצמצם את זה. 
	\end{proof}
	
	\textbf{מסקנה. }$V$ מ"ו, ממימד $n$. אז: 
	\begin{enumerate}
		\item סדרה בת"ל מקסימלית היא בסיס. 
		\item סדרה פורשת מינימלית היא בסיס. 
		\item סדרה בת"ל/פורשת עם $\dim V$ איברים, היא בסיס. 
	\end{enumerate}
	
	\begin{proof}\,
		\begin{enumerate}
			\item נראה שפורשת – ואכן אחרת קיים $v \in V \setminus \mathrm{span}(B)$ ומכאן $B \cup \{v\}$ בת"ל בסתירה למקסימליות. 
			\item נראה שבת"ל – אחרת קיימים $\sum \lambda_iv_i = 0$ עבור $\{\lambda_i\}$ לא טרוויאלים. נניח $\lambda_1 = 0$, נגרר $\mathrm{span}(v_2, \dots, v_s) = V$ (כי $v_1 \in \mathrm{span}(v_2, \dots, v_s)$). ובפרט, $B$ לא מינימלית. 
			\item בת"ל בעל $\dim V$ איברים – זו מקסימלית מהטענה על סדרה יוצר עם $\dim V$ איברים (כל סדרה שגדול ממנה היא ???)
			פורשת עם $\dim V$ – נניח שלא מינימלי קיימת סדרה קטנה יותר, הצלחנו לצמצלם אותה לבסיס קטן יותר, מאנו בסיס $|B|$ שבגודל $|B| < \dim V$ וזו סתירה. 
		\end{enumerate}
	\end{proof}
	
	\textbf{משפט. }
	
		יהיו $V$ מ"ו, $U \subseteq V$ תמ"ו. אז: 
	\begin{enumerate}
		\item $\dim U \le \dim V$
		\item $\dim U = \dim V$ אמ"מ $U = V$
	\end{enumerate}
	
	\begin{proof}
		\begin{enumerate}
			\item יהי $B_U$ בסיס של $V$. נרחיב את $B_U$ לבסיס נקבל $B_V$. מתקיים $|B_V| \ge |B_U|$ ובהתאם המימדים. 
			\item \, 
			\begin{enumerate}
				\item[$\implies$] ברור
				\item[$\impliedby$] נסתכל על $B_V$ בסיס היא סדרה בת"ל והגודל $\dim V$ ולכן $B_U$ סכום ל־$V$ וסה"כ $\mathrm{span}(B_V) = V$
			\end{enumerate}
		\end{enumerate}
	\end{proof}
	
	\textbf{משפט. }יהי $V \subseteq F^n$ מרחב הפתרונות של משוואה הומוגנית. אזי $\dim V$ הוא מספר המשתנים החופשיים במטריצה הקאנונית המתאימה. \begin{proof}
		נסמן $= I$ אינדקסים של משתנים תלויים, ו־$= J$ אינדקסים של משתנים חופשיים. נשים לב שעבור $\{a_j\}_J \in F^J$ נותנת פתרןו יחיד. נסמן $V_j \in F^n$ להיות הפרון שמתאים ל־$a_j = 1$, ו־$a_{j'} = 0$ עבור $J \ni j' \neq j$. ולכן $(V_j)_J$ בסיס . אז $\{v_j\}_{j \in J}$ בסיס כי: 
		
		בת"ל כי נניח שלא, $\sum \lambda_i vj = 0$ (עבור האינדקסים $j \in J$ הערכים נראים כמו $e_i$) ולכן $\forall j \in J. \lambda_j = 0$, כרצוי. פורש כי לכל $v \in V$ פתרון בחירת $a_j$ מתאימה תתן אותו מתכונות המשוואה. כלומר, $x_i = - \sum_{j \in J} c_ia_j$. 
		
		סה"כ $|\{v_1, \dots, v_j\}| = \dim V$ (כי כזה בסיס) = מספר מהשתנים החופשיים (כי כזה הגדרה). 
	\end{proof}
	
	\section{\en{Linear Maps}}
	\textbf{הגדרה. }בהינתן $V_1, V_2$ מרחבים וקטוריים מעל $F$ שדה, וקיים $\varphi \colon V_1 \to V_2$. נגדיר $\varphi$\textit{העתקה ליניארית} אם: 
	\begin{enumerate}
		\item $\forall u, v \in v_1. \vphi(u + v) = \vphi(u) + \vphi(v)$
		\item $\forall \lambda \in F, v \in V. \vphi(\lambda v) = \lambda \vphi(v)$
		זה שקול לכך ש־: $\forall \lambda_1, \lambda_2 \in F, v_1, v-2 \in V. \vphi(\lambda_1v-2 + \lambda_2v_2)  = \lambda_1\vphi(v_1) + \lambda_2\vphi(v_2)$
	\end{enumerate}
	
	\textbf{הגדרה. }\textit{שיכון} אם חח"ע. 
	
	\textbf{סימון. }$\Img \vphi = \Img(\vphi) = \{\vphi(v) \mid v \in V_1\} \subseteq V_2$
	
	\textbf{סימון. }\textit{גרעין} (kernel) יהיה $\ker \vphi = \ker(\vphi) = \{v \in V_1 \mid \vphi(v) = 0\} \subseteq V_1$
	
	\textbf{סימון. }\textit{הומומורפיזם} $\hom_F(V_1, V_2) = \{\vphi \colon V_1 \to V_2 \mid \text{העתקה ליניארית} \, \vphi\}$
	
	\textbf{סימון. }$\hom(V) := \hom(V, V)$
	
	\subsection{דוגמאות}
	כל אלו העתקות ליניאריות: 
	
	נתבון ב־$\vphi \colon V_1 \to V_2\, \vphi(x) = 0$. לעיתים הוא יסומן ב־$0$ ("\textit{העתקת ה־0}"). 
	
	גם $\vphi \colon V \to V\, x \mapsto x$. לעיתים יסמון ב־$id, id_V, 1$. 
	
	גם סיבוב היא העתקה ליניארית. 
	
	"לא משנה עושים גיאומטריה בסדר", לדוגמה: $(x, y) \mapsto (y, -x)$ היא העתקה ליניארית (סוג של סיבוב). נוכיח – נסמן $u = (x_1, y_1), v = (x_2, y_2)$. ואין זמן למורה. 
	
\end{document}