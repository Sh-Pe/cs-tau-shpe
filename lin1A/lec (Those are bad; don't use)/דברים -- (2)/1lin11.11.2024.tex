%! ~~~ Packages Setup ~~~ 
\documentclass[]{article}


% Math packages
\usepackage[usenames]{color}
\usepackage{forest}
\usepackage{ifxetex,ifluatex,amsmath,amssymb,mathrsfs,amsthm,witharrows,mathtools}
\WithArrowsOptions{displaystyle}
\renewcommand{\qedsymbol}{$\blacksquare$} % end proofs with \blacksquare. Overwrites the defualts. 
\usepackage{cancel,bm}
\usepackage[thinc]{esdiff}


% tikz
\usepackage{tikz}
\newcommand\sqw{1}
\newcommand\squ[4][1]{\fill[#4] (#2*\sqw,#3*\sqw) rectangle +(#1*\sqw,#1*\sqw);}


% code 
\usepackage{listings}
\usepackage{xcolor}

\definecolor{codegreen}{rgb}{0,0.35,0}
\definecolor{codegray}{rgb}{0.5,0.5,0.5}
\definecolor{codenumber}{rgb}{0.1,0.3,0.5}
\definecolor{codeblue}{rgb}{0,0,0.5}
\definecolor{codered}{rgb}{0.5,0.03,0.02}
\definecolor{codegray}{rgb}{0.96,0.96,0.96}

\lstdefinestyle{pythonstylesheet}{
	language=Python,
	emphstyle=\color{deepred},
	backgroundcolor=\color{codegray},
	keywordstyle=\color{deepblue}\bfseries\itshape,
	numberstyle=\scriptsize\color{codenumber},
	basicstyle=\ttfamily\footnotesize,
	commentstyle=\color{codegreen}\itshape,
	breakatwhitespace=false, 
	breaklines=true, 
	captionpos=b, 
	keepspaces=true, 
	numbers=left, 
	numbersep=5pt, 
	showspaces=false,                
	showstringspaces=false,
	showtabs=false, 
	tabsize=4, 
	morekeywords={as,assert,nonlocal,with,yield,self,True,False,None,AssertionError,ValueError,in,else},              % Add keywords here
	keywordstyle=\color{codeblue},
	emph={object,type,isinstance,copy,deepcopy,zip,enumerate,reversed,list,set,len,dict,tuple,print,range,xrange,append,execfile,real,imag,reduce,str,repr,__init__,__add__,__mul__,__div__,__sub__,__call__,__getitem__,__setitem__,__eq__,__ne__,__nonzero__,__rmul__,__radd__,__repr__,__str__,__get__,__truediv__,__pow__,__name__,__future__,__all__,},          % Custom highlighting
	emphstyle=\color{codered},
	stringstyle=\color{codegreen},
	showstringspaces=false,
	abovecaptionskip=0pt,belowcaptionskip =0pt,
	framextopmargin=-\topsep, 
}
\newcommand\pythonstyle{\lstset{pythonstylesheet}}
\newcommand\pyl[1]     {{\lstinline!#1!}}
\lstset{style=pythonstylesheet}

\usepackage[style=1,skipbelow=\topskip,skipabove=\topskip,framemethod=TikZ]{mdframed}
\definecolor{bggray}{rgb}{0.85, 0.85, 0.85}
\mdfsetup{leftmargin=0pt,rightmargin=0pt,innerleftmargin=15pt,backgroundcolor=codegray,middlelinewidth=0.5pt,skipabove=5pt,skipbelow=0pt,middlelinecolor=black,roundcorner=5}
\BeforeBeginEnvironment{lstlisting}{\begin{mdframed}\vspace{-0.4em}}
	\AfterEndEnvironment{lstlisting}{\vspace{-0.8em}\end{mdframed}}


% Deisgn
\usepackage[labelfont=bf]{caption}
\usepackage[margin=0.6in]{geometry}
\usepackage{multicol}
\usepackage[skip=4pt, indent=0pt]{parskip}
\usepackage[normalem]{ulem}
\forestset{default}
\renewcommand\labelitemi{$\bullet$}
\usepackage{titlesec}
\titleformat{\section}[block]
{\fontsize{15}{15}}
{\sen \dotfill (\thesection) \she}
{0em}
{\MakeUppercase}
\usepackage{graphicx}
\graphicspath{ {./} }


% Hebrew initialzing
\usepackage[bidi=basic]{babel}
\PassOptionsToPackage{no-math}{fontspec}
\babelprovide[main, import, Alph=letters]{hebrew}
\babelprovide[import]{english}
\babelfont[hebrew]{rm}{David CLM}
\babelfont[hebrew]{sf}{David CLM}
\babelfont[english]{tt}{Monaspace Xenon}
\usepackage[shortlabels]{enumitem}
\newlist{hebenum}{enumerate}{1}

% Language Shortcuts
\newcommand\en[1] {\begin{otherlanguage}{english}#1\end{otherlanguage}}
\newcommand\sen   {\begin{otherlanguage}{english}}
	\newcommand\she   {\end{otherlanguage}}
\newcommand\del   {$ \!\! $}
\newcommand\ttt[1]{\en{\footnotesize\texttt{#1}\normalsize}}

\newcommand\npage {\vfil {\hfil \textbf{\textit{המשך בעמוד הבא}}} \hfil \vfil \pagebreak}
\newcommand\ndoc  {\dotfill \\ \vfil {\begin{center} {\textbf{\textit{שחר פרץ, 2024}} \\ \scriptsize \textit{נוצר באמצעות תוכנה חופשית בלבד}} \end{center}} \vfil	}

\newcommand{\rn}[1]{
	\textup{\uppercase\expandafter{\romannumeral#1}}
}

\makeatletter
\newcommand{\skipitems}[1]{
	\addtocounter{\@enumctr}{#1}
}
\makeatother

%! ~~~ Math shortcuts ~~~

% Letters shortcuts
\newcommand\N     {\mathbb{N}}
\newcommand\Z     {\mathbb{Z}}
\newcommand\R     {\mathbb{R}}
\newcommand\Q     {\mathbb{Q}}
\newcommand\C     {\mathbb{C}}
\newcommand\F     {\mathbb{F}}
\newcommand\co    {\colon}
\newcommand\zf    {0_\F}
\newcommand\of    {1_\F}
\newcommand\cd    {\cdot}

\newcommand\ml    {\ell}
\newcommand\mj    {\jmath}
\newcommand\mi    {\imath}

\newcommand\powerset {\mathcal{P}}
\newcommand\ps    {\mathcal{P}}
\newcommand\pc    {\mathcal{P}}
\newcommand\ac    {\mathcal{A}}
\newcommand\bc    {\mathcal{B}}
\newcommand\cc    {\mathcal{C}}
\newcommand\dc    {\mathcal{D}}
\newcommand\ec    {\mathcal{E}}
\newcommand\fc    {\mathcal{F}}
\newcommand\nc    {\mathcal{N}}
\newcommand\sca   {\mathcal{S}} % \sc is already definded
\newcommand\rca   {\mathcal{R}} % \rc is already definded

\newcommand\Si    {\Sigma}

% Logic & sets shorcuts
\newcommand\siff  {\longleftrightarrow}
\newcommand\ssiff {\leftrightarrow}
\newcommand\so    {\longrightarrow}
\newcommand\sso   {\rightarrow}

\newcommand\epsi  {\epsilon}
\newcommand\vepsi {\varepsilon}
\newcommand\vphi  {\varphi}
\newcommand\Neven {\N_{\mathrm{even}}}
\newcommand\Nodd  {\N_{\mathrm{odd }}}
\newcommand\Zeven {\Z_{\mathrm{even}}}
\newcommand\Zodd  {\Z_{\mathrm{odd }}}
\newcommand\Np    {\N_+}

% Text Shortcuts
\newcommand\open  {\big(}
\newcommand\qopen {\quad\big(}
\newcommand\close {\big)}
\newcommand\also  {\text{, }}
\newcommand\defi  {\text{ definition}}
\newcommand\defis {\text{ definitions}}
\newcommand\given {\text{given }}
\newcommand\case  {\text{if }}
\newcommand\syx   {\text{ syntax}}
\newcommand\rle   {\text{ rule}}
\newcommand\other {\text{else}}
\newcommand\set   {\ell et \text{ }}
\newcommand\ans   {\mathit{Ans.}}

% Set theory shortcuts
\newcommand\ra    {\rangle}
\newcommand\la    {\langle}

\newcommand\oto   {\leftarrow}

\newcommand\QED   {\quad\quad\mathscr{Q.E.D.}\;\;\blacksquare}
\newcommand\QEF   {\quad\quad\mathscr{Q.E.F.}}
\newcommand\eQED  {\mathscr{Q.E.D.}\;\;\blacksquare}
\newcommand\eQEF  {\mathscr{Q.E.F.}}
\newcommand\jQED  {\mathscr{Q.E.D.}}

\newcommand\dom   {\mathrm{dom}}
\newcommand\Img   {\mathrm{Im}}
\newcommand\range {\mathrm{range}}

\newcommand\trio  {\triangle}

\newcommand\rc    {\right\rceil}
\newcommand\lc    {\left\lceil}
\newcommand\rf    {\right\rfloor}
\newcommand\lf    {\left\lfloor}

\newcommand\lex   {<_{lex}}

\newcommand\az    {\aleph_0}
\newcommand\uaz   {^{\aleph_0}}
\newcommand\al    {\aleph}
\newcommand\ual   {^\aleph}
\newcommand\taz   {2^{\aleph_0}}
\newcommand\utaz  { ^{\left (2^{\aleph_0} \right )}}
\newcommand\tal   {2^{\aleph}}
\newcommand\utal  { ^{\left (2^{\aleph} \right )}}
\newcommand\ttaz  {2^{\left (2^{\aleph_0}\right )}}

\newcommand\n     {$n$־יה\ }

% Math A&B shortcuts
\newcommand\logn  {\log n}
\newcommand\logx  {\log x}
\newcommand\lnx   {\ln x}
\newcommand\cosx  {\cos x}
\newcommand\cost  {\cos \theta}
\newcommand\sinx  {\sin x}
\newcommand\sint  {\sin \theta}
\newcommand\tanx  {\tan x}
\newcommand\tant  {\tan \theta}
\newcommand\sex   {\sec x}
\newcommand\sect  {\sec^2}
\newcommand\cotx  {\cot x}
\newcommand\cscx  {\csc x}
\newcommand\sinhx {\sinh x}
\newcommand\coshx {\cosh x}
\newcommand\tanhx {\tanh x}

\newcommand\seq   {\overset{!}{=}}
\newcommand\slh   {\overset{LH}{=}}
\newcommand\sle   {\overset{!}{\le}}
\newcommand\sge   {\overset{!}{\ge}}
\newcommand\sll   {\overset{!}{<}}
\newcommand\sgg   {\overset{!}{>}}

\newcommand\h     {\hat}
\newcommand\ve    {\vec}
\newcommand\lv    {\overrightarrow}
\newcommand\ol    {\overline}

\newcommand\mlcm  {\mathrm{lcm}}

\DeclareMathOperator{\sech}   {sech}
\DeclareMathOperator{\csch}   {csch}
\DeclareMathOperator{\arcsec} {arcsec}
\DeclareMathOperator{\arccot} {arcCot}
\DeclareMathOperator{\arccsc} {arcCsc}
\DeclareMathOperator{\arccosh}{arccosh}
\DeclareMathOperator{\arcsinh}{arcsinh}
\DeclareMathOperator{\arctanh}{arctanh}
\DeclareMathOperator{\arcsech}{arcsech}
\DeclareMathOperator{\arccsch}{arccsch}
\DeclareMathOperator{\arccoth}{arccoth}
\DeclareMathOperator{\atant}  {atan2} 

\newcommand\dx    {\,\mathrm{d}x}
\newcommand\dt    {\,\mathrm{d}t}
\newcommand\dtt   {\,\mathrm{d}\theta}
\newcommand\du    {\,\mathrm{d}u}
\newcommand\dv    {\,\mathrm{d}v}
\newcommand\df    {\mathrm{d}f}
\newcommand\dfdx  {\diff{f}{x}}
\newcommand\dit   {\limhz \frac{f(x + h) - f(x)}{h}}

\newcommand\nt[1] {\frac{#1}{#1}}

\newcommand\limz  {\lim_{x \to 0}}
\newcommand\limxz {\lim_{x \to x_0}}
\newcommand\limi  {\lim_{x \to \infty}}
\newcommand\limh  {\lim_{x \to 0}}
\newcommand\limni {\lim_{x \to - \infty}}
\newcommand\limpmi{\lim_{x \to \pm \infty}}

\newcommand\ta    {\theta}
\newcommand\ap    {\alpha}

\renewcommand\inf {\infty}
\newcommand  \ninf{-\inf}

% Combinatorics shortcuts
\newcommand\sumnk     {\sum_{k = 0}^{n}}
\newcommand\sumni     {\sum_{i = 0}^{n}}
\newcommand\sumnko    {\sum_{k = 1}^{n}}
\newcommand\sumnio    {\sum_{i = 1}^{n}}
\newcommand\sumai     {\sum_{i = 1}^{n} A_i}
\newcommand\nsum[2]   {\reflectbox{\displaystyle\sum_{\reflectbox{\scriptsize$#1$}}^{\reflectbox{\scriptsize$#2$}}}}

\newcommand\bink      {\binom{n}{k}}
\newcommand\setn      {\{a_i\}^{2n}_{i = 1}}
\newcommand\setc[1]   {\{a_i\}^{#1}_{i = 1}}

\newcommand\cupain    {\bigcup_{i = 1}^{n} A_i}
\newcommand\cupai[1]  {\bigcup_{i = 1}^{#1} A_i}
\newcommand\cupiiai   {\bigcup_{i \in I} A_i}
\newcommand\capain    {\bigcap_{i = 1}^{n} A_i}
\newcommand\capai[1]  {\bigcap_{i = 1}^{#1} A_i}
\newcommand\capiiai   {\bigcap_{i \in I} A_i}

\newcommand\xot       {x_{1, 2}}
\newcommand\ano       {a_{n - 1}}
\newcommand\ant       {a_{n - 2}}

% Other shortcuts
\newcommand\tl    {\tilde}
\newcommand\op    {^{-1}}

\newcommand\sof[1]    {\left | #1 \right |}
\newcommand\cl [1]    {\left ( #1 \right )}
\newcommand\csb[1]    {\left [ #1 \right ]}

\newcommand\bs    {\blacksquare}

%! ~~~ Document ~~~

\author{שחר פרץ}
\title{תרגול ליניארית 2}
\begin{document}
	\maketitle
	תרגילי בית בימי שלישי. 
	\section{\en{Fields}}
	
	\textbf{הגדרה: }שדה היא קבוצה $\mathbb{F}$ ביחד עם פעולות חיבור וכפל, שנסמנם $+, \cdot$. 
	\begin{itemize}
		\item סגירות: \hfill $\forall a, b \in \F\colon a + b, a \cdot b \in \F$
		\item קיבוציות (אסוציאטיביות): \hfill $\forall a, b, c \in \F. a + (b + c) = (a + b) + c, \ a \cdot (b\cdot c) = (a \cdot b) \cdot c$
		\item חילופיות (קומטטיביות): \hfill $\forall a, b \in \F. a + b = b + a, \ a \cdot b = b \cdot a$
		\item קיום איבר ניטרלי: \hfill $\exists \of \neq \zf \in \F \co \forall a \in \F\co a \cdot 1 = a, \ a + 0 = a$
		\item קיום נגדיים והוכפיים: \hfill $\forall a \in (\F \exists b \in \F \co a + b = c) \land  (a \neq 0 \implies \exists c \in \F. a \cdot c = 1)$
		\item פילוג: \hfill $\forall a, b c \in \F. a (b + c) = a \cdot b +  a \cdot c$
	\end{itemize}
	
	דוגמאות: $\Z_p, \C, \R, \Q$ הם שדות. $\Z$ לא שדה כי אין הופכיים. גם $\Z_n$ לא שדה עבור $n$ לא ראשוני כי אין הופכי לכל איבר בשדה;
	
	אם אין $n$ לא ראשוני, אז יש $a, b > 1$ כך ש־$a \cdot b = n$. אז $a \cdot b = 0$ ב־$\Z_n$. אבל $a, b \neq 0$. 
	
	\subsection{תרגילים}
	\subsubsection{תרגיל 1}
	יהי $\F $ שדה. הוכיחו את התכונות הבאות: 
	\begin{enumerate}
		\item $\forall a \in \F. -(-a) = a$
		\item $\forall a, b \in \F \setminus \{0\} \co (ab)\op = a\op b\op$
		\item $\forall a, b \in \F\co ab \notin \{0, 1\}. )a - aba)\op = a\op + (b\op - a)\op$
		ובדקו ששני האגפים מוגדרים היטב
	\end{enumerate}
	\begin{proof}[הוכחה (1). ]
		ננסה: 
		\[ a + (-a) = 0 \to -(a + (-a)) = 0 \to (-1)(a + (-a)) = 0 \to (-1)a + (-1)(-a) = 0, \to -a + (-(-a)) = 0, \to -(-a) = a \]
		שיטה אחרת: מיחיודת של נגדי צ.ל. $(-a) + a = 0$. מחילופיות $a + (-a) =0$, שכבר ידוע לפי הגדרה. 
	\end{proof}
	
	\begin{proof}[הוכחה (2). ]
		לכל $a, b \in \F\setminus \{0\}$: 
		\[ (ab)\op \cdot (ab) = 1, \ a\op a = 1, \ b\op b = 1, \implies (ab)\op(ab) = a\op a b\op b = a\op b\op (ab) \rotatebox{60}{$\because$} \]
		
		תזכורת: לוודא שהכל מוגדר היטב. 
		
		פתרון אחד: מיחידות של הוכפי צ.ל. $(ab)(a\op b\op) = 1$. ניעזר בחילופיות ובקביוציות ונוכי חאת הדרוש. 
	\end{proof}
	
	\begin{proof}[הוכחה (3). ]צ.ל. $(a - aba)\op = a\op + (b\op -a)\op$
		
		$a - aba \neq 0$  כי אם $a - aba = 0$ אז $a = aba $ואז $a = ab$ וזאת סתירה. גם $b\op 0 a \neq 0$ כי אחרת $a = b\op$ ואז $ab = 1$ וזו סתירה. 
		מהיחידות של ההופכי צ.ל. $(a - aba)(a\op + (b\op - a)\op) = 1$. נדע: 
		\begin{multline*}
			\cdots = (a - aba)a\op = (a - aba)(b\op - a)\op = (1 - ab) + (abb\op - aba) (b\op - a) = (1 - ab) + ab(b\op - a)(b\op - a)\op \\
			= (1 - ab) + ab \cdot 1 = 1 - ab + ab = 1
		\end{multline*}
	\end{proof}
	
	\subsubsection{תרגיל 2}
	הוכיחו שהקבוצה $\Q(\sqrt2) = \{a + b \sqrt 2 \mid a, b \in \Q\}$ הוא שדה, ביחס לפעולות החיבור והכפל של הממשיים. 
	
	\begin{proof}
		כל הכונות כמו חילופיות, אסוציאטיביות, וכו' ארוזים עם העובדה שאנחנו משתמשים בפעולות על הממשיים. נמשיך מכאן. צריך לבדוק ש־$0, 1 \in \Q(\sqrt2)$, וסגירות לחיבור, כפל, נכדי והופכי. 
		
		בבירור $0, 1 \in \Q(\sqrt 2)$. סגירות לחיבור, כפל ונגדי נובעים מהסגירות של $\Q$ לפעולות האלה: 
		\begin{gather}
			(a + b\sqrt2)  + (c + d\sqrt2) = (a + c) + (b + d)\sqrt2 \in \Q(\sqrt 2) \\
			(a + b\sqrt2)(c + d\sqrt2) = ac + ad\sqrt2 + bc\sqrt 2 + bd = (ac + 2bd) + (ad + bc)\sqrt2 = (a + b\sqrt2) = (-a) + (-b)\sqrt2 \in \Q(2)
		\end{gather}
		בשביל קיום הופכי, נשים לב שלכל $a + b \sqrt 2 \in \Q(\sqrt 2)$: 
		\[ \frac{1}{a + b\sqrt2} = \frac{a - b\sqrt 2}{a^2 - 2b^2} = \cl{\frac{a}{a - 2b^2}} + \cl{\frac{b}{2b^2 - a^2}} \in \Q(\sqrt 2) \]
		
		נשים לב שזה מוגדר היטב, כי אם $a^2 - 2b^2 = 0$ אז $2 = \cl{\frac{a}{b}}^2$  $\sqrt = \frac{a}{b}$ וזו סתירה. 
	\end{proof}
	
	\subsubsection{תרגיל 3}
	\textbf{תרגיל}הראו שקיים שדה מגודל $4$. 
	
	\begin{proof}
		\begin{multicols}{2}
			
			חייבים להיות $\zf, \of$ בשדה, נסמן את שני האיברים שנשארו $a, b$. יאי, תוולעות חיבור וכפל: 
			
			{
				\hfil \begin{tabular}{|c|c|c|c|c|}
					\hline $+$ & $0$ & $1$ & $a$ & $b$ \\
					\hline $0$ & $0$ & $1$ & $a$ & $b$ \\
					\hline $1$ & $1$ & $0$ & $b$ & $a$ \\
					\hline $a$ & $a$ & $b$ & $0$ & $1$ \\
					\hline $b$ & $b$ & $a$ & $1$ & $0$ \\
					\hline 
				\end{tabular}
				\hfil
				\begin{tabular}{|c|c|c|c|c|}
					\hline $\cdot $ & $0$ & $1$ & $a$ & $b$ \\
					\hline $0$ & $0$ & $0$ & $0$ & $0$ \\
					\hline $1$ & $0$ & $1$ & $a$ & $b$ \\
					\hline $a$ & $0$ & $a$ & $b$ & $1$ \\
					\hline $b$ & $0$ & $b$ & $1$ & $a$ \\
					\hline
				\end{tabular}
				
				\hfil 
			}
			
			\textbf{שלב 1}: נמצא מה הוא $1 + 1$. 
			\begin{enumerate}
				\item אם $1 + 1 = 1$ אז $1 = 0$ וזו סתירה. 
				\item אם $1 + 1 = a$, נתובנן ב־$a + 1$
				\begin{enumerate}
					\item אם $1 + a = 1$ אז $a = 0$ וזו סתירה. 
					\item אם $a + 1 = a$ אז $1 = 0$ וזו גם סתירה. 
				\end{enumerate}
				\item אם $1 + a = 0$, נבדוק למה $b + 1$ שווה. 
				\begin{enumerate}
					\item אם $b + 1 = 0$, אז $a + 1 = 0 = b + 1$ ואז $a = b$ וזו סתירה. 
					\item אפ $b + 1 = 1$ אז $b = 0$ וזאת סתירה. 
					\item אם $b + 1 = b$ אז $1 = 0 $וזו סתירה
					\item  אם $b + 1 = a$ אז  אז $b +1  = 1 + 1$ ואז $b =1 $ וזו סתירה. 
				\end{enumerate}
				ולכן בלתי אפשרי ש־$1 + a = 0$. 
			\end{enumerate}
			
			לכן $1 + a = b$. מאותן סיבות מקבלים ש‏$b + 1 = 0$. 
			\[ a^2 = (1 + 1)^2(1 + 1)^2 = 1 + 1 + 1 + 1 = a + 1 + 1 = b + 1 = 0 \]
			וזאת סתירה ולכן שאנחנו בשדה. לכן $a _ a \neq a$ ואופן סימטרי $1 + 1 \neq b$. סך הכל, קיבלנו ש־$1 + 1 = 0$. 
			
			\textbf{שלב שני. }נרצה להבין מה זה $a + 1$
			\begin{enumerate}
				\item אם $a + 1 = 1 $ סתירה. 
				\item אם $a + 1 = a $ סתירה. 
				\item אם $a = 1 = 0$ אז $a = 1$ וזאת סתירה. 
			\end{enumerate}
			לכאן $a + 1 = b$. 
			
			\textbf{שלב 3. }מכאן אפשר להשלים את טבלת החיבור: 
			\begin{gather*}
				a + a  = a \cdot (1 +1 ) = a \cdot 0 = 0 \\ b + b = b (1 + 1) = b \cdot 0 = 0 \\
				a + b = a + (a + 1) - (a +a) + 1 = 0 + 1 = 1 \\ b + 1 = (a + 1) + 1 = a + (1 + 1) = a + 0 = a
			\end{gather*}
			
			\textbf{שלב 4. }נמצא את $a^2$. $a^2 \neq 0$ כי אנחנו בשדה. בנוסף, $a^a \neq a$ כי אחרת $a =1 $. בנוסף, $a^2 \neq 1$, כי אם $a^2 = 1$ אז $a^2 -1 = 0 \implies 0 = (a - 1)(a + 1) = (a + 1)(a + 1)$ וכך $(a + 1)^2 = 0$ כלומר $b^2 = 0$ וזה לא אפשרי. לכן $a^2 = b$
			
			\textbf{שלב 5. }
			\begin{gather}
				ab = a(a + 1) = a^2 + a = b + a = 1
			\end{gather}
			
					\textit{למעשה, לא הוכחנו שהשדה קיים, רק בנינו שדה בצורת שלילה לכל שדה אחר, אבל לא הוכחנו שהוא עובד. }
			
			\textit{השדה הלא ראשוני הקטן ביותר הבא הוא בגודל 9. תהנו בלבנות אותו. }
		\end{multicols}
		

	\end{proof}
	
	\subsection{טענה אקראית}
	קיים שדה ראשוני מגודל סופי אמ"מ גודלו היא חזקה של ראשוני. נוכיח כיוון אחד, הכיוון השני דורש כלים הרבה יותר מורכבים. אסור להשתמש באמ"מ אלא רק בכיוון אחד. 
	
\end{document}