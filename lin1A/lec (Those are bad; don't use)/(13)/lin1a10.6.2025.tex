%! ~~~ Packages Setup ~~~ 
\documentclass[]{article}
\usepackage{lipsum}
\usepackage{rotating}


% Math packages
\usepackage[usenames]{color}
\usepackage{forest}
\usepackage{ifxetex,ifluatex,amssymb,amsmath,mathrsfs,amsthm,witharrows,mathtools,mathdots}
\usepackage{amsmath}
\WithArrowsOptions{displaystyle}
\renewcommand{\qedsymbol}{$\blacksquare$} % end proofs with \blacksquare. Overwrites the defualts. 
\usepackage{cancel,bm}
\usepackage[thinc]{esdiff}


% tikz
\usepackage{tikz}
\usetikzlibrary{graphs}
\newcommand\sqw{1}
\newcommand\squ[4][1]{\fill[#4] (#2*\sqw,#3*\sqw) rectangle +(#1*\sqw,#1*\sqw);}


% code 
\usepackage{algorithm2e}
\usepackage{listings}
\usepackage{xcolor}

\definecolor{codegreen}{rgb}{0,0.35,0}
\definecolor{codegray}{rgb}{0.5,0.5,0.5}
\definecolor{codenumber}{rgb}{0.1,0.3,0.5}
\definecolor{codeblue}{rgb}{0,0,0.5}
\definecolor{codered}{rgb}{0.5,0.03,0.02}
\definecolor{codegray}{rgb}{0.96,0.96,0.96}

\lstdefinestyle{pythonstylesheet}{
    language=Java,
    emphstyle=\color{deepred},
    backgroundcolor=\color{codegray},
    keywordstyle=\color{deepblue}\bfseries\itshape,
    numberstyle=\scriptsize\color{codenumber},
    basicstyle=\ttfamily\footnotesize,
    commentstyle=\color{codegreen}\itshape,
    breakatwhitespace=false, 
    breaklines=true, 
    captionpos=b, 
    keepspaces=true, 
    numbers=left, 
    numbersep=5pt, 
    showspaces=false,                
    showstringspaces=false,
    showtabs=false, 
    tabsize=4, 
    morekeywords={as,assert,nonlocal,with,yield,self,True,False,None,AssertionError,ValueError,in,else},              % Add keywords here
    keywordstyle=\color{codeblue},
    emph={var, List, Iterable, Iterator},          % Custom highlighting
    emphstyle=\color{codered},
    stringstyle=\color{codegreen},
    showstringspaces=false,
    abovecaptionskip=0pt,belowcaptionskip =0pt,
    framextopmargin=-\topsep, 
}
\newcommand\pythonstyle{\lstset{pythonstylesheet}}
\newcommand\pyl[1]     {{\lstinline!#1!}}
\lstset{style=pythonstylesheet}

\usepackage[style=1,skipbelow=\topskip,skipabove=\topskip,framemethod=TikZ]{mdframed}
\definecolor{bggray}{rgb}{0.85, 0.85, 0.85}
\mdfsetup{leftmargin=0pt,rightmargin=0pt,innerleftmargin=15pt,backgroundcolor=codegray,middlelinewidth=0.5pt,skipabove=5pt,skipbelow=0pt,middlelinecolor=black,roundcorner=5}
\BeforeBeginEnvironment{lstlisting}{\begin{mdframed}\vspace{-0.4em}}
    \AfterEndEnvironment{lstlisting}{\vspace{-0.8em}\end{mdframed}}


% Design
\usepackage[labelfont=bf]{caption}
\usepackage[margin=0.6in]{geometry}
\usepackage{multicol}
\usepackage[skip=4pt, indent=0pt]{parskip}
\usepackage[normalem]{ulem}
\forestset{default}
\renewcommand\labelitemi{$\bullet$}
\usepackage{titlesec}
\titleformat{\section}[block]
{\fontsize{15}{15}}
{\sen \dotfill (\thesection)\dotfill\she}
{0em}
{\MakeUppercase}
\usepackage{graphicx}
\graphicspath{ {./} }

\usepackage[colorlinks]{hyperref}
\definecolor{mgreen}{RGB}{25, 160, 50}
\definecolor{mblue}{RGB}{30, 60, 200}
\usepackage{hyperref}
\hypersetup{
<<<<<<<< HEAD:lin1A/lec/(13)/lin1a10.6.2025.tex
    colorlinks=true,
    citecolor=mgreen,
    linkcolor=black,
    urlcolor=mblue,
    pdftitle={Document by Shahar Perets},
    %	pdfpagemode=FullScreen,
}
\usepackage{yfonts}
\def\gothstart#1{\noindent\smash{\lower3ex\hbox{\llap{\Huge\gothfamily#1}}}
    \parshape=3 3.1em \dimexpr\hsize-3.4em 3.4em \dimexpr\hsize-3.4em 0pt \hsize}
\def\frakstart#1{\noindent\smash{\lower3ex\hbox{\llap{\Huge\frakfamily#1}}}
    \parshape=3 1.5em \dimexpr\hsize-1.5em 2em \dimexpr\hsize-2em 0pt \hsize}
========
	colorlinks=true,
	citecolor=mgreen,
	linkcolor=black,
	urlcolor=mblue,
	pdftitle={Document by Shahar Perets},
	%	pdfpagemode=FullScreen,
}
\usepackage{yfonts}
\def\gothstart#1{\noindent\smash{\lower3ex\hbox{\llap{\Huge\gothfamily#1}}}
	\parshape=3 3.1em \dimexpr\hsize-3.4em 3.4em \dimexpr\hsize-3.4em 0pt \hsize}
\def\frakstart#1{\noindent\smash{\lower3ex\hbox{\llap{\Huge\frakfamily#1}}}
	\parshape=3 1.5em \dimexpr\hsize-1.5em 2em \dimexpr\hsize-2em 0pt \hsize}
>>>>>>>> master:lin2A/lec/(20)/lin2a22.6.2025.tex



% Hebrew initialzing
\usepackage[bidi=basic]{babel}
\PassOptionsToPackage{no-math}{fontspec}
\babelprovide[main, import, Alph=letters]{hebrew}
\babelprovide[import]{english}
\babelfont[hebrew]{rm}{David CLM}
\babelfont[hebrew]{sf}{David CLM}
%\babelfont[english]{tt}{Monaspace Xenon}
\usepackage[shortlabels]{enumitem}
\newlist{hebenum}{enumerate}{1}

% Language Shortcuts
\newcommand\en[1] {\begin{otherlanguage}{english}#1\end{otherlanguage}}
\newcommand\he[1] {\she#1\sen}
\newcommand\sen   {\begin{otherlanguage}{english}}
    \newcommand\she   {\end{otherlanguage}}
\newcommand\del   {$ \!\! $}

\newcommand\npage {\vfil {\hfil \textbf{\textit{המשך בעמוד הבא}}} \hfil \vfil \pagebreak}
\newcommand\ndoc  {\dotfill \\ \vfil {\begin{center}
            {\textbf{\textit{שחר פרץ, 2025}} \\
                \scriptsize \textit{קומפל ב־}\en{\LaTeX}\,\textit{ ונוצר באמצעות תוכנה חופשית בלבד}}
    \end{center}} \vfil	}

\newcommand{\rn}[1]{
    \textup{\uppercase\expandafter{\romannumeral#1}}
}

\makeatletter
\newcommand{\skipitems}[1]{
    \addtocounter{\@enumctr}{#1}
}
\makeatother

%! ~~~ Math shortcuts ~~~

% Letters shortcuts
\newcommand\N     {\mathbb{N}}
\newcommand\Z     {\mathbb{Z}}
\newcommand\R     {\mathbb{R}}
\newcommand\Q     {\mathbb{Q}}
\newcommand\C     {\mathbb{C}}
\newcommand\One   {\mathit{1}}

\newcommand\ml    {\ell}
\newcommand\mj    {\jmath}
\newcommand\mi    {\imath}

\newcommand\powerset {\mathcal{P}}
\newcommand\ps    {\mathcal{P}}
\newcommand\pc    {\mathcal{P}}
\newcommand\ac    {\mathcal{A}}
\newcommand\bc    {\mathcal{B}}
\newcommand\cc    {\mathcal{C}}
\newcommand\dc    {\mathcal{D}}
\newcommand\ec    {\mathcal{E}}
\newcommand\fc    {\mathcal{F}}
\newcommand\nc    {\mathcal{N}}
\newcommand\vc    {\mathcal{V}} % Vance
\newcommand\sca   {\mathcal{S}} % \sc is already definded
\newcommand\rca   {\mathcal{R}} % \rc is already definded
\newcommand\zc    {\mathcal{Z}}

\newcommand\prm   {\mathrm{p}}
\newcommand\arm   {\mathrm{a}} % x86
\newcommand\brm   {\mathrm{b}}
\newcommand\crm   {\mathrm{c}}
\newcommand\drm   {\mathrm{d}}
\newcommand\erm   {\mathrm{e}}
\newcommand\frm   {\mathrm{f}}
\newcommand\nrm   {\mathrm{n}}
\newcommand\vrm   {\mathrm{v}}
\newcommand\srm   {\mathrm{s}}
\newcommand\rrm   {\mathrm{r}}

\newcommand\Si    {\Sigma}

% Logic & sets shorcuts
\newcommand\siff  {\longleftrightarrow}
\newcommand\ssiff {\leftrightarrow}
\newcommand\so    {\longrightarrow}
\newcommand\sso   {\rightarrow}

\newcommand\epsi  {\epsilon}
\newcommand\vepsi {\varepsilon}
\newcommand\vphi  {\varphi}
\newcommand\Neven {\N_{\mathrm{even}}}
\newcommand\Nodd  {\N_{\mathrm{odd }}}
\newcommand\Zeven {\Z_{\mathrm{even}}}
\newcommand\Zodd  {\Z_{\mathrm{odd }}}
\newcommand\Np    {\N_+}

% Text Shortcuts
\newcommand\open  {\big(}
\newcommand\qopen {\quad\big(}
\newcommand\close {\big)}
\newcommand\also  {\mathrm{, }}
\newcommand\defis {\mathrm{ definitions}}
\newcommand\given {\mathrm{given }}
\newcommand\case  {\mathrm{if }}
\newcommand\syx   {\mathrm{ syntax}}
\newcommand\rle   {\mathrm{ rule}}
\newcommand\other {\mathrm{else}}
\newcommand\set   {\ell et \text{ }}
\newcommand\ans   {\mathscr{A}\!\mathit{nswer}}

% Set theory shortcuts
\newcommand\ra    {\rangle}
\newcommand\la    {\langle}

\newcommand\oto   {\leftarrow}

\newcommand\QED   {\quad\quad\mathscr{Q.E.D.}\;\;\blacksquare}
\newcommand\QEF   {\quad\quad\mathscr{Q.E.F.}}
\newcommand\eQED  {\mathscr{Q.E.D.}\;\;\blacksquare}
\newcommand\eQEF  {\mathscr{Q.E.F.}}
\newcommand\jQED  {\mathscr{Q.E.D.}}

\DeclareMathOperator\dom   {dom}
\DeclareMathOperator\Img   {Im}
\DeclareMathOperator\range {range}

\newcommand\trio  {\triangle}

\newcommand\rc    {\right\rceil}
\newcommand\lc    {\left\lceil}
\newcommand\rf    {\right\rfloor}
\newcommand\lf    {\left\lfloor}
\newcommand\ceil  [1] {\lc #1 \rc}
\newcommand\floor [1] {\lf #1 \rf}

\newcommand\lex   {<_{lex}}

\newcommand\az    {\aleph_0}
\newcommand\uaz   {^{\aleph_0}}
\newcommand\al    {\aleph}
\newcommand\ual   {^\aleph}
\newcommand\taz   {2^{\aleph_0}}
\newcommand\utaz  { ^{\left (2^{\aleph_0} \right )}}
\newcommand\tal   {2^{\aleph}}
\newcommand\utal  { ^{\left (2^{\aleph} \right )}}
\newcommand\ttaz  {2^{\left (2^{\aleph_0}\right )}}

\newcommand\n     {$n$־יה\ }

% Math A&B shortcuts
\newcommand\logn  {\log n}
\newcommand\logx  {\log x}
\newcommand\lnx   {\ln x}
\newcommand\cosx  {\cos x}
\newcommand\sinx  {\sin x}
\newcommand\sint  {\sin \theta}
\newcommand\tanx  {\tan x}
\newcommand\tant  {\tan \theta}
\newcommand\sex   {\sec x}
\newcommand\sect  {\sec^2}
\newcommand\cotx  {\cot x}
\newcommand\cscx  {\csc x}
\newcommand\sinhx {\sinh x}
\newcommand\coshx {\cosh x}
\newcommand\tanhx {\tanh x}

\newcommand\seq   {\overset{!}{=}}
\newcommand\slh   {\overset{LH}{=}}
\newcommand\sle   {\overset{!}{\le}}
\newcommand\sge   {\overset{!}{\ge}}
\newcommand\sll   {\overset{!}{<}}
\newcommand\sgg   {\overset{!}{>}}

\newcommand\h     {\hat}
\newcommand\ve    {\vec}
\newcommand\lv    {\overrightarrow}
\newcommand\ol    {\overline}

\newcommand\mlcm  {\mathrm{lcm}}

\DeclareMathOperator{\sech}   {sech}
\DeclareMathOperator{\csch}   {csch}
\DeclareMathOperator{\arcsec} {arcsec}
\DeclareMathOperator{\arccot} {arcCot}
\DeclareMathOperator{\arccsc} {arcCsc}
\DeclareMathOperator{\arccosh}{arccosh}
\DeclareMathOperator{\arcsinh}{arcsinh}
\DeclareMathOperator{\arctanh}{arctanh}
\DeclareMathOperator{\arcsech}{arcsech}
\DeclareMathOperator{\arccsch}{arccsch}
\DeclareMathOperator{\arccoth}{arccoth}
\DeclareMathOperator{\atant}  {atan2} 
\DeclareMathOperator{\Sp}     {span} 
\DeclareMathOperator{\sgn}    {sgn} 
\DeclareMathOperator{\row}    {Row} 
\DeclareMathOperator{\adj}    {adj} 
\DeclareMathOperator{\rk}     {rank} 
\DeclareMathOperator{\col}    {Col} 
\DeclareMathOperator{\tr}     {tr}

\newcommand\dx    {\,\mathrm{d}x}
\newcommand\dt    {\,\mathrm{d}t}
\newcommand\dtt   {\,\mathrm{d}\theta}
\newcommand\du    {\,\mathrm{d}u}
\newcommand\dv    {\,\mathrm{d}v}
\newcommand\df    {\mathrm{d}f}
\newcommand\dfdx  {\diff{f}{x}}
\newcommand\dit   {\limhz \frac{f(x + h) - f(x)}{h}}

\newcommand\nt[1] {\frac{#1}{#1}}

\newcommand\limz  {\lim_{x \to 0}}
\newcommand\limxz {\lim_{x \to x_0}}
\newcommand\limi  {\lim_{x \to \infty}}
\newcommand\limh  {\lim_{x \to 0}}
\newcommand\limni {\lim_{x \to - \infty}}
\newcommand\limpmi{\lim_{x \to \pm \infty}}

\newcommand\ta    {\theta}
\newcommand\ap    {\alpha}

\renewcommand\inf {\infty}
\newcommand  \ninf{-\inf}

% Combinatorics shortcuts
\newcommand\sumnk     {\sum_{k = 0}^{n}}
\newcommand\sumni     {\sum_{i = 0}^{n}}
\newcommand\sumnko    {\sum_{k = 1}^{n}}
\newcommand\sumnio    {\sum_{i = 1}^{n}}
\newcommand\sumai     {\sum_{i = 1}^{n} A_i}
\newcommand\nsum[2]   {\reflectbox{\displaystyle\sum_{\reflectbox{\scriptsize$#1$}}^{\reflectbox{\scriptsize$#2$}}}}

\newcommand\bink      {\binom{n}{k}}
\newcommand\setn      {\{a_i\}^{2n}_{i = 1}}
\newcommand\setc[1]   {\{a_i\}^{#1}_{i = 1}}

\newcommand\cupain    {\bigcup_{i = 1}^{n} A_i}
\newcommand\cupai[1]  {\bigcup_{i = 1}^{#1} A_i}
\newcommand\cupiiai   {\bigcup_{i \in I} A_i}
\newcommand\capain    {\bigcap_{i = 1}^{n} A_i}
\newcommand\capai[1]  {\bigcap_{i = 1}^{#1} A_i}
\newcommand\capiiai   {\bigcap_{i \in I} A_i}

\newcommand\xot       {x_{1, 2}}
\newcommand\ano       {a_{n - 1}}
\newcommand\ant       {a_{n - 2}}

% Linear Algebra
\DeclareMathOperator{\chr}     {char}
\DeclareMathOperator{\diag}    {diag}
\DeclareMathOperator{\Hom}     {Hom}
\DeclareMathOperator{\Sym}     {Sym}
\DeclareMathOperator{\Asym}    {ASym}

\newcommand\lra       {\leftrightarrow}
\newcommand\chrf      {\chr(\F)}
\newcommand\F         {\mathbb{F}}
\newcommand\co        {\colon}
\newcommand\tmat[2]   {\cl{\begin{matrix}
            #1
        \end{matrix}\, \middle\vert\, \begin{matrix}
            #2
\end{matrix}}}

\makeatletter
\newcommand\rrr[1]    {\xxrightarrow{1}{#1}}
\newcommand\rrt[2]    {\xxrightarrow{1}[#2]{#1}}
\newcommand\mat[2]    {M_{#1\times#2}}
\newcommand\gmat      {\mat{m}{n}(\F)}
\newcommand\tomat     {\, \dequad \longrightarrow}
\newcommand\pms[1]    {\begin{pmatrix}
        #1
\end{pmatrix}}

\newcommand\norm[1]   {\left \vert \left \vert #1 \right \vert \right \vert}
\newcommand\snorm     {\left \vert \left \vert \cdot \right \vert \right \vert}
\newcommand\smut      {\left \la \cdot \mid \cdot \right \ra}
\newcommand\mut[2]    {\left \la #1 \,\middle\vert\, #2 \right \ra}

% someone's code from the internet: https://tex.stackexchange.com/questions/27545/custom-length-arrows-text-over-and-under
\makeatletter
\newlength\min@xx
\newcommand*\xxrightarrow[1]{\begingroup
    \settowidth\min@xx{$\m@th\scriptstyle#1$}
    \@xxrightarrow}
\newcommand*\@xxrightarrow[2][]{
    \sbox8{$\m@th\scriptstyle#1$}  % subscript
    \ifdim\wd8>\min@xx \min@xx=\wd8 \fi
    \sbox8{$\m@th\scriptstyle#2$} % superscript
    \ifdim\wd8>\min@xx \min@xx=\wd8 \fi
    \xrightarrow[{\mathmakebox[\min@xx]{\scriptstyle#1}}]
    {\mathmakebox[\min@xx]{\scriptstyle#2}}
    \endgroup}
\makeatother


% Greek Letters
\newcommand\ag        {\alpha}
\newcommand\bg        {\beta}
\newcommand\cg        {\gamma}
\newcommand\dg        {\delta}
\newcommand\eg        {\epsi}
\newcommand\zg        {\zeta}
\newcommand\hg        {\eta}
\newcommand\tg        {\theta}
\newcommand\ig        {\iota}
\newcommand\kg        {\keppa}
\renewcommand\lg      {\lambda}
\newcommand\og        {\omicron}
\newcommand\rg        {\rho}
\newcommand\sg        {\sigma}
\newcommand\yg        {\usilon}
\newcommand\wg        {\omega}

\newcommand\Ag        {\Alpha}
\newcommand\Bg        {\Beta}
\newcommand\Cg        {\Gamma}
\newcommand\Dg        {\Delta}
\newcommand\Eg        {\Epsi}
\newcommand\Zg        {\Zeta}
\newcommand\Hg        {\Eta}
\newcommand\Tg        {\Theta}
\newcommand\Ig        {\Iota}
\newcommand\Kg        {\Keppa}
\newcommand\Lg        {\Lambda}
\newcommand\Og        {\Omicron}
\newcommand\Rg        {\Rho}
\newcommand\Sg        {\Sigma}
\newcommand\Yg        {\Usilon}
\newcommand\Wg        {\Omega}

% Other shortcuts
\newcommand\tl    {\tilde}
\newcommand\op    {^{-1}}

\newcommand\sof[1]    {\left | #1 \right |}
\newcommand\cl [1]    {\left ( #1 \right )}
\newcommand\csb[1]    {\left [ #1 \right ]}
\newcommand\ccb[1]    {\left \{ #1 \right \}}

\newcommand\bs        {\blacksquare}
\newcommand\dequad    {\!\!\!\!\!\!}
\newcommand\dequadd   {\dequad\duquad}

\renewcommand\phi     {\varphi}

\newtheorem{Theorem}{משפט}
\theoremstyle{definition}
\newtheorem{definition}{הגדרה}
\newtheorem{Lemma}{למה}
\newtheorem{Remark}{הערה}
\newtheorem{Notion}{סימון}


\newcommand\theo  [1] {\begin{Theorem}#1\end{Theorem}}
\newcommand\defi  [1] {\begin{definition}#1\end{definition}}
\newcommand\rmark [1] {\begin{Remark}#1\end{Remark}}
\newcommand\lem   [1] {\begin{Lemma}#1\end{Lemma}}
\newcommand\noti  [1] {\begin{Notion}#1\end{Notion}}

<<<<<<<< HEAD:lin1A/lec/(13)/lin1a10.6.2025.tex
========
% DS
\newcommand\limsi     {\limsup_{n \to \inf}}
\newcommand\limfi     {\liminf_{n \to \inf}}

\DeclareMathOperator\amort   {amort}
\DeclareMathOperator\worst   {worst}
\DeclareMathOperator\type    {type}
\DeclareMathOperator\cost    {cost}
\DeclareMathOperator\tim     {time}

\newcommand\dsList{
	\sFunc{List}
	\sFunc{Retrieve}
	\SetKwFunction{RetrieveFirst}{Retrieve-First}
	\SetKwFunction{RetrieveLast}{Retrieve-Last}
	\sFunc{Delete}
	\SetKwFunction{DeleteFirst}{Delete-First}
	\SetKwFunction{DeleteLast}{Delete-Last}
	\sFunc{Insert}
	\SetKwFunction{InsertFirst}{Insert-First}
	\SetKwFunction{InsertLast}{Insert-Last}
	\sFunc{Shift}
	\sFunc{Length}
	\sFunc{Concat}
	\sFunc{Plant}
	\sFunc{Split}
}
\newcommand\dsQueue{
	\sFunc{Queue}
	\sFunc{Enqueue}
	\sFunc{Head}
	\sFunc{Dequeue}
}
\newcommand\dsStack{
	\sFunc{Stack}
	\sFunc{Push}
	\sFunc{Top}
	\sFunc{Pop}
}
\newcommand\dsVector{
	\sFunc{Vector}
	\sFunc{Get}
	\sFunc{Set}
}
\newcommand\dsGraph{
	\sFunc{Graph}
	\sFunc{Edge}
	\SetKwFunction{AddEdge}{Add-Edge}
	\SetKwFunction{RemoveEdge}{Remove-Edge}
	\sFunc{InDeg} \sFunc{OutDeg}
}
\newcommand\importDs{
	\dsList
	\dsQueue
	\dsStack
	\dsVector
	\dsGraph
	\SetKwProg{Fn}{function}{ is}{end}
	\SetKwData{error}{\color{codered}error}
	\SetKwInOut{Input}{input}
	\SetKwInOut{Output}{output}
	\SetKwRepeat{Do}{do}{while}
	\SetKwData{Null}{\color{codegreen}null}
	\SetKwData{True}{\color{codeblue}true}
	\SetKwData{False}{\color{codeblue}false}
}


% Algorithems
\newcommand\sFunc [1] {\SetKwFunction{#1}{#1}}
\newcommand\sData [1] {\SetKwData{#1}{#1}}
\newcommand\sIO   [1] {\SetKwInOut{#1}{#1}}
\newcommand\ttt   [1] {\sen \texttt{#1} \she\,}
\newcommand\io    [2] {\Input{#1}\Output{#2}\BlankLine}
>>>>>>>> master:lin2A/lec/(20)/lin2a22.6.2025.tex

%! ~~~ Document ~~~

\author{שחר פרץ}
<<<<<<<< HEAD:lin1A/lec/(13)/lin1a10.6.2025.tex
\title{\textit{אלגברה לינארית 1א $\sim$ מרחבים דואליים}}
\begin{document}
    \maketitle
    
    
    \section{\en{Dual spaces}}
    \defi{יהי $V$ מ''ו מעל $\F$. אז: 
    \[ V^{*} := \{f \co V \to \F \co \mid f \, \text{\en{Linear}}\} \]
    הוא המ''ו הדואלי, ו־$f \in V^{*}$ פונקציונלי לינארי. }
    
    \textbf{דוגמאות. }
    \begin{itemize}
        \item בעבור $V = \R$ נראה שזה הישרים מהצורה $y = ax$. נסמן $y(1) = m$ ואז $y(x) = xy(1) = xm$ וסה''כ $y(x) = mx$ כדרוש. 
        \item בעבור $V = \R^{2}$: גם במקרה הזה נקבע את $f(e_1) = a, f(e_2) = b$ ואז $f(x, y) = f(e_1x + e_2y) = ax + by$
        \item במקרה הכללי יותר $V = \R^{n}$ ונסמן $f(e_i) = a_i$. סה''כ: 
        \[ f(x_1 \dots x_n) = f\cl{\sum_{i = 1}^{n}x_ie_i} = \sum_{i = 1}^{n}x_ia_i \]
    \end{itemize}
    
    \theo{המרחב הדואלי $(\R^{n})^{*}\cong \R^{n}$} \begin{proof}
        נגדיר: 
        \[ T \co \R^{n} \to (\R^{n})^{8}, \ T(a_1 \dots a_n) = f_{a_1 \dots a_n}, \ f_{a_1 \dots a_n}(x_1 \dots x_n) = \sum_{i = 1}^{n}a_xe_i \]
        נוכיח שהיא לינארית, חח''ע ועל. 
        \begin{itemize}
            \item $T$ לינארית: 
            \[ T(a + b) = f_{a + b} = \sum_{i = 1}^{n}(a_i + b_i)x_i = \sumni a_ix_i + \sumni b_ix_i = f_a + f_b = T(a) + T(b) \]
            עתה נבדוק כפל בסקלר: 
            \[ T_{\lg a} = f_{\lg a} = \sumni \lg a_i x_i = \lg \sumni a_ix_i = \lg f_{a}x \]
            \item $T$ חח''ע: נניח ש־$T(a_1 \dots a_n) = 0$. ידוע $f_a = \sumni x_ia_i$ ולכל $i$ מתקיים $f(e_i) = a_i = 0$ וסה''כ $a = 0$ כדרוש. 
        \end{itemize}
    \end{proof}
    
    באופן כללי, $\dim V = n \implies \dim V^* = n$. 
    
    בהינתן בסיס ל־$V$, ישנו בסיס דואלי $\{f_1 \dots f_n\}$ המוגדרות ע''י $\forall i, j \co f_j(b_i) = \dg_{ij}$. 
    
    \theo{זהו בסיס ל־$V^*$. }\begin{proof}\,
        \begin{itemize}
            \item \textbf{על: }
            \[ f(v) = f\cl{\sumni \ag_i b_i} = \sumni \ag_i f(b_i) \]
            \item \textbf{חח''ע: }
            \[ f = \sumni \ag_i f_i = 0 \implies f(b_i) = \ag_i = 0 \implies (\ag_i)_{i = 1}^{n} = 0 \quad \top \]
        \end{itemize}
    \end{proof}
    נבחין שב־$\R^{n}$: 
    \[ \ker f_a = \ccb{x \in \R^n \co \sumni a_ix_i = 0} \]
    זהו תמ''ו מממד $n - 1$ (זהו אילוץ אחד שמאבד דרגת חופש אחת). 
    
    \textbf{דוגמאות לפונקציונלים מעל מ''וים לא נוצרים סופית: }$V = \{f \co [0, 1] \to \R\}$, מתקיים ש־$\phi(f) = f(0)$ פוקציונל, וגרעינו $\ker \phi =\{f \co [0, 1] \to \R\co f(0) = 0\}$. אפשר להראות ש־$f = f - \underbrace{f(0)}_{\ker \phi} + \underbrace{f(0)}_{A}$ כאשר $A$ מ''ו הפונקציות הקבועות. למעשה, פירקנו לסכום ישר של הגרעין + משהו חד ממדי. בכך ניסחנו לכאורה את המספר של $n - 1$ גם כאשר אין באמת אפשרות לחסר ממדי. נקרא לזה קו־ממד אחד. 
    
    באופן כללי, אם $0 \neq f \in V^*$ יש $v \in V$ fl a-$f(v) \neq 0$. נניח $f(v) = 1$. אז $\forall x \in V \co x = f - f(x)v + f(x) v$ ולכן נוכל לפרק $V = \ker f \oplus \Sp v$. 
    
    \subsection*{המרחב $V^{**}$}
    ברור כי $V^{**}\cong V$. לכל $v \in V$ נוכל להתאים $\phi_V(f) = f(v)$. זה מגדיר העתקה לינארית מ־$V$ ל־$V^{**}$. נוכיח שהיא איזומורפיזם. חח''ע: הרעיון: להוכיח (לנוצרים סופית) $\forall f \in V^*, \phi_V(f) =  0 = f(v)$. נניח בשלילה ש־$v \neq 0$. אז נוכל להשלים אותו לבסיס $v, v_2 \dots v_n$, אפשר להגדיר $f$ כך ש־$f(v) = 1, \ f(v_i) = 0 \forall i \ge 2$. אבל אז $\phi_V(f) = f(v) = 2$ וסתירה. אזי זה חח''ע. 
    
    
    זהו איזומורפיזם קאנוני: הוא לא תלוי בבסיס או משהו של המרחב עצמו. האיזומורפיזם לא דורש שום בחירה לא טרוויאלית בתוך המרחב. נאמר שהם איזומורמים קאנונית. 
    
    \subsection*{המרחב המאפס}
    \defi{תהי $S \subseteq V$, אז המאפס שלה: 
    \[ S^{0} = \{f \in V^* \co f|_S = 0\}\subseteq V^* \]}
    \theo{\,
    \begin{itemize}
        \item \hfil $S^0 = (\Sp S)^0$
        \item $S^0$ מ''ו
        \item \hfil $S \subseteq T \implies S^0 \reflectbox{$\subseteq$} T^0$
    \end{itemize}}

    \theo{יהי $V$ מ''ו, $\dim V = n$, $U \subseteq V$ תמ''ו, נניח ש־$\dim U = r$ אז $\dim U^0 = n - r$}\begin{proof}
        ניקח בסיס ל־$U$, $e_1 \dots e_r$. ניקח $f_1 \dots f_r$ כך ש־$f_i(e_j) = \dg_{ij}(j)$. נשלים לבסיס של $V$: 
        \[ e_1 \dots e_r, \ e_{r + 1} \dots e_n \]
        ובאופן דומה ב־$V^{n}$: 
        \[ f_1 \dots f_r, f_{r + 1} \dots f_n \]
        זהו בסיס ל־$U^0$ (לכל)
        $j \ge r + 1$. מתקיים $f_i(e_j) = 0 \implies f_j|_U = 0$ לכל $i \ge r$. עתה נראה פרישה: 
        \[ \forall f \in U^0 \co f(v) = f\cl{\sumni \ag_i e_i} = f\cl{\sum_{i = 1}^{r}\ag_ie_i} + f\cl{\sum_{i = r + 1}^{n}\ag_i e_i} = f\cl{\sum_{i = r+ 1}^{n}\ag_i e_i} = \sum_{i = r + 1}^{n}\ag_i f(e_i) = \sum_{i= r + 1}^{n}f_i(\underbrace{\sum \ag_i e_i}_{v})f(e_i) \]
        בת''ל: 
        הם איברים בבסיס $V^*$ ובפרט בת''ל. 
    \end{proof}
    
    \ndoc
    
========
\title{\textit{לינארית 2א 20}}
\begin{document}
	\maketitle
	תזכורת: $A \in M_n(\C)$ נקראית סימטרית אממ $A = A^T$ והרמיטית אם $A = A^*$, ונורמלית אממ $AA^* = A^*A$. 
	
	\theo{תהי $T \co V \to V$ ט''ל, ו־$V$ ממ''פ מעל $\F \in \{\R, \C\}$, ויהי $B$ בסיס א''נ של $V$. אזי אם $A = [T]_B$: 
	\[ [T^*]_B = A = ([T]_B)^* \]}  
	\begin{proof}
		נזכר ש־: 
		\[ [T]_B = \pms{\vert &  & \vert \\ [Te_1]_B & \cdots & [Te_n]_B \\ \vert &  & \vert} \]
		נסמן $B = \{e_i\}_{i = 1}^{n}$ בסיס. נבחין ש־: 
		\[ Te_j = \sumni a_{ij}e_i, \ a_{ij} = \mut{Te_j}{e_i} \]
		נסמן ב־$C$ את המטריצה המייצגת $[T^*]_B$: 
		\[ c_{ij} = \mut{T^*e_j}{e_i} \]
		ונחשב: 
		\[ c_ij = \mut{T^*e_j}{e_i} = \mut{e_j}{Te_i} = \ol{\mut{Te_i}{e_j}} = a_{ij} \]
		
	\end{proof}
	
	\textbf{מסקנה: }אם $A$ נורמלית אז $T_A$ נורמלית מעל $\F^n$ אם הסטנדרטית. בפרט מתקיים עליה המשפט הספקטרלי. גם אם $A$ ממשית, הע''ע עלולים להמצא מעל $\C$ (אלא אם היא צמודה לעצמה, ואז הם מעל $\R$). 
	
	משהו על אינטרפולציות: 
	\theo{יהיו $x_1 \dots x_n, y_1 \dots y_n \in \R$. נניח $\forall i, j \in [n] \co i \neq j \implies x_i \neq x_j$. אז $\exists! p \in \R_{\le n - 1}[x] \co \forall i \in [n] \co p(x_i) = y_i$ עד לכדי חברות (באופן שקול: נניח $p$ מתוקן) }
	(הערה מהידע הכללי שלי: זהו פולינום לגראנג' והוא בונה אינטרפולציה די נחמדה). \begin{proof}
		ידוע שהפולינום מהצורה $p(x)  = \sum_{k = 0}^{n - 1} a_k x^k  = (1, x, x^2, \dots x^{n - 1}) (a_0 \dots a_{n - 1})^T$. למעשה, נקבל את מטריצת ונדרמונד: 
		\[ \pms{1 & x_1 & x_1^2 & \cdots & x_1^n \\ 1 & x_2 & x_2^2 & \cdots & x_2^{n - 1} \\ \vdots & \vdots & \vdots & \ddots & \vdots \\ 1 & x_n & x_n^2 & \cdots & x_n^{n - 1}}\pms{a_0 \\ a_1 \\ \vdots \\ a_{n - 1}} = \pms{y_1 \\ y_2 \\ \vdots \\ y_n} \]
		וידוע שהדטרמיננטה של ונדרמונד היא $\prod_{i < j} (x_i - x_j)$ וזה איכשהו אמור לגמור את ההוכחה. 
		
		אם $x_i = y_i$ בפולינום לעיל, אז $\forall a \in \C \co f(\bar a) = \ol{f(a)} \implies f \in \R[x]$. הוכחה: נניח בשלילה, אז $\exists \ag \in \C\setminus \R \co f(\ag) = 0$ כך ש־$f(\bar \ag) = 0$. אזי $0 \neq f(\bar \ag) = \ol{f(\ag)}  = 0$ וזו סתירה. 
	\end{proof}
	
	\theo{תהי $A \in M_n(\F)$ נורמלית, אז קיים פולינום $f(x) \in \R[x]$ כך ש־$A^* = f(A)$. }
	\textit{הערה: }באופן כללי \textbf{לא} נכון שאם $A, B$ מתחלפות אז $\exists f(x) \in \F[x] \co f(A) = B$. 
	\begin{proof}
		נוכיח מעל $\C$ ובפרט נכונות ל־$\R$. מהמשפט הספקטרלי קיימת $P$ הפיכה (מעבר לבסיס אורתונורמלי) כך ש־$P\op A P = \diag(\lg_1 \dots \lg_n)$ הוא הבסיס המלכסן. מהיות המלכסן א''נ נקבל: 
		\[ P\op A^* P = \diag(\lg_1 \dots \lg_n)^* = \diag(\bar\lg_1 \dots \bar \lg_n) \]
		ולכן מפולינום האינטרפולציה עבור הקבוצות $\{\lg_1 \dots \lg_n\}$ ו־$\{\bar \lg_1 \dots \bar \lg_n\}$ קיים ויחיד עד לכדי חברות $f(x) \in \R[x]$ כך ש־$f(\lg_i) = \bar \lg_i$. נותר להראות ש־$A^* = f(A)$, זאת כי: 
		\[ P\op A^* P = f(\diag(\lg_1 \dots \lg_n)) = \diag(\bar \lg_1 \dots \bar \lg_n) \implies f(P\op A P) = P\op f(A) P \]
		נכפול ב־$P$ מצד אחד וב־$P\op$ מהצד השני ונקבל את הדרוש
	\end{proof}
	
	
	\ndoc
>>>>>>>> master:lin2A/lec/(20)/lin2a22.6.2025.tex
\end{document}