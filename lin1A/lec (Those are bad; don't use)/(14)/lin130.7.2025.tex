%! ~~~ Packages Setup ~~~ 
\documentclass[]{article}
\usepackage{lipsum}
\usepackage{rotating}


% Math packages
\usepackage[usenames]{color}
\usepackage{forest}
\usepackage{ifxetex,ifluatex,amssymb,amsmath,mathrsfs,amsthm,witharrows,mathtools,mathdots}
\usepackage{amsmath}
\WithArrowsOptions{displaystyle}
\renewcommand{\qedsymbol}{$\blacksquare$} % end proofs with \blacksquare. Overwrites the defualts. 
\usepackage{cancel,bm}
\usepackage[thinc]{esdiff}


% tikz
\usepackage{tikz}
\usetikzlibrary{graphs}
\newcommand\sqw{1}
\newcommand\squ[4][1]{\fill[#4] (#2*\sqw,#3*\sqw) rectangle +(#1*\sqw,#1*\sqw);}


% code 
\usepackage{algorithm2e}
\usepackage{listings}
\usepackage{xcolor}

\definecolor{codegreen}{rgb}{0,0.35,0}
\definecolor{codegray}{rgb}{0.5,0.5,0.5}
\definecolor{codenumber}{rgb}{0.1,0.3,0.5}
\definecolor{codeblue}{rgb}{0,0,0.5}
\definecolor{codered}{rgb}{0.5,0.03,0.02}
\definecolor{codegray}{rgb}{0.96,0.96,0.96}

\lstdefinestyle{pythonstylesheet}{
	language=Java,
	emphstyle=\color{deepred},
	backgroundcolor=\color{codegray},
	keywordstyle=\color{deepblue}\bfseries\itshape,
	numberstyle=\scriptsize\color{codenumber},
	basicstyle=\ttfamily\footnotesize,
	commentstyle=\color{codegreen}\itshape,
	breakatwhitespace=false, 
	breaklines=true, 
	captionpos=b, 
	keepspaces=true, 
	numbers=left, 
	numbersep=5pt, 
	showspaces=false,                
	showstringspaces=false,
	showtabs=false, 
	tabsize=4, 
	morekeywords={as,assert,nonlocal,with,yield,self,True,False,None,AssertionError,ValueError,in,else},              % Add keywords here
	keywordstyle=\color{codeblue},
	emph={var, List, Iterable, Iterator},          % Custom highlighting
	emphstyle=\color{codered},
	stringstyle=\color{codegreen},
	showstringspaces=false,
	abovecaptionskip=0pt,belowcaptionskip =0pt,
	framextopmargin=-\topsep, 
}
\newcommand\pythonstyle{\lstset{pythonstylesheet}}
\newcommand\pyl[1]     {{\lstinline!#1!}}
\lstset{style=pythonstylesheet}

\usepackage[style=1,skipbelow=\topskip,skipabove=\topskip,framemethod=TikZ]{mdframed}
\definecolor{bggray}{rgb}{0.85, 0.85, 0.85}
\mdfsetup{leftmargin=0pt,rightmargin=0pt,innerleftmargin=15pt,backgroundcolor=codegray,middlelinewidth=0.5pt,skipabove=5pt,skipbelow=0pt,middlelinecolor=black,roundcorner=5}
\BeforeBeginEnvironment{lstlisting}{\begin{mdframed}\vspace{-0.4em}}
	\AfterEndEnvironment{lstlisting}{\vspace{-0.8em}\end{mdframed}}


% Design
\usepackage[labelfont=bf]{caption}
\usepackage[margin=0.6in]{geometry}
\usepackage{multicol}
\usepackage[skip=4pt, indent=0pt]{parskip}
\usepackage[normalem]{ulem}
\forestset{default}
\renewcommand\labelitemi{$\bullet$}
\usepackage{titlesec}
\titleformat{\section}[block]
{\fontsize{15}{15}}
{\sen \dotfill (\thesection)\dotfill\she}
{0em}
{\MakeUppercase}
\usepackage{graphicx}
\graphicspath{ {./} }

\usepackage[colorlinks]{hyperref}
\definecolor{mgreen}{RGB}{25, 160, 50}
\definecolor{mblue}{RGB}{30, 60, 200}
\usepackage{hyperref}
\hypersetup{
	colorlinks=true,
	citecolor=mgreen,
	linkcolor=black,
	urlcolor=mblue,
	pdftitle={Document by Shahar Perets},
	%	pdfpagemode=FullScreen,
}
\usepackage{yfonts}
\def\gothstart#1{\noindent\smash{\lower3ex\hbox{\llap{\Huge\gothfamily#1}}}
	\parshape=3 3.1em \dimexpr\hsize-3.4em 3.4em \dimexpr\hsize-3.4em 0pt \hsize}
\def\frakstart#1{\noindent\smash{\lower3ex\hbox{\llap{\Huge\frakfamily#1}}}
	\parshape=3 1.5em \dimexpr\hsize-1.5em 2em \dimexpr\hsize-2em 0pt \hsize}



% Hebrew initialzing
\usepackage[bidi=basic]{babel}
\PassOptionsToPackage{no-math}{fontspec}
\babelprovide[main, import, Alph=letters]{hebrew}
\babelprovide[import]{english}
\babelfont[hebrew]{rm}{David CLM}
\babelfont[hebrew]{sf}{David CLM}
%\babelfont[english]{tt}{Monaspace Xenon}
\usepackage[shortlabels]{enumitem}
\newlist{hebenum}{enumerate}{1}

% Language Shortcuts
\newcommand\en[1] {\begin{otherlanguage}{english}#1\end{otherlanguage}}
\newcommand\he[1] {\she#1\sen}
\newcommand\sen   {\begin{otherlanguage}{english}}
	\newcommand\she   {\end{otherlanguage}}
\newcommand\del   {$ \!\! $}

\newcommand\npage {\vfil {\hfil \textbf{\textit{המשך בעמוד הבא}}} \hfil \vfil \pagebreak}
\newcommand\ndoc  {\dotfill \\ \vfil {\begin{center}
			{\textbf{\textit{שחר פרץ, 2025}} \\
				\scriptsize \textit{קומפל ב־}\en{\LaTeX}\,\textit{ ונוצר באמצעות תוכנה חופשית בלבד}}
	\end{center}} \vfil	}

\newcommand{\rn}[1]{
	\textup{\uppercase\expandafter{\romannumeral#1}}
}

\makeatletter
\newcommand{\skipitems}[1]{
	\addtocounter{\@enumctr}{#1}
}
\makeatother

%! ~~~ Math shortcuts ~~~

% Letters shortcuts
\newcommand\N     {\mathbb{N}}
\newcommand\Z     {\mathbb{Z}}
\newcommand\R     {\mathbb{R}}
\newcommand\Q     {\mathbb{Q}}
\newcommand\C     {\mathbb{C}}
\newcommand\One   {\mathit{1}}

\newcommand\ml    {\ell}
\newcommand\mj    {\jmath}
\newcommand\mi    {\imath}

\newcommand\powerset {\mathcal{P}}
\newcommand\ps    {\mathcal{P}}
\newcommand\pc    {\mathcal{P}}
\newcommand\ac    {\mathcal{A}}
\newcommand\bc    {\mathcal{B}}
\newcommand\cc    {\mathcal{C}}
\newcommand\dc    {\mathcal{D}}
\newcommand\ec    {\mathcal{E}}
\newcommand\fc    {\mathcal{F}}
\newcommand\nc    {\mathcal{N}}
\newcommand\vc    {\mathcal{V}} % Vance
\newcommand\sca   {\mathcal{S}} % \sc is already definded
\newcommand\rca   {\mathcal{R}} % \rc is already definded
\newcommand\zc    {\mathcal{Z}}

\newcommand\prm   {\mathrm{p}}
\newcommand\arm   {\mathrm{a}} % x86
\newcommand\brm   {\mathrm{b}}
\newcommand\crm   {\mathrm{c}}
\newcommand\drm   {\mathrm{d}}
\newcommand\erm   {\mathrm{e}}
\newcommand\frm   {\mathrm{f}}
\newcommand\nrm   {\mathrm{n}}
\newcommand\vrm   {\mathrm{v}}
\newcommand\srm   {\mathrm{s}}
\newcommand\rrm   {\mathrm{r}}

\newcommand\Si    {\Sigma}

% Logic & sets shorcuts
\newcommand\siff  {\longleftrightarrow}
\newcommand\ssiff {\leftrightarrow}
\newcommand\so    {\longrightarrow}
\newcommand\sso   {\rightarrow}

\newcommand\epsi  {\epsilon}
\newcommand\vepsi {\varepsilon}
\newcommand\vphi  {\varphi}
\newcommand\Neven {\N_{\mathrm{even}}}
\newcommand\Nodd  {\N_{\mathrm{odd }}}
\newcommand\Zeven {\Z_{\mathrm{even}}}
\newcommand\Zodd  {\Z_{\mathrm{odd }}}
\newcommand\Np    {\N_+}

% Text Shortcuts
\newcommand\open  {\big(}
\newcommand\qopen {\quad\big(}
\newcommand\close {\big)}
\newcommand\also  {\mathrm{, }}
\newcommand\defis {\mathrm{ definitions}}
\newcommand\given {\mathrm{given }}
\newcommand\case  {\mathrm{if }}
\newcommand\syx   {\mathrm{ syntax}}
\newcommand\rle   {\mathrm{ rule}}
\newcommand\other {\mathrm{else}}
\newcommand\set   {\ell et \text{ }}
\newcommand\ans   {\mathscr{A}\!\mathit{nswer}}

% Set theory shortcuts
\newcommand\ra    {\rangle}
\newcommand\la    {\langle}

\newcommand\oto   {\leftarrow}

\newcommand\QED   {\quad\quad\mathscr{Q.E.D.}\;\;\blacksquare}
\newcommand\QEF   {\quad\quad\mathscr{Q.E.F.}}
\newcommand\eQED  {\mathscr{Q.E.D.}\;\;\blacksquare}
\newcommand\eQEF  {\mathscr{Q.E.F.}}
\newcommand\jQED  {\mathscr{Q.E.D.}}

\DeclareMathOperator\dom   {dom}
\DeclareMathOperator\Img   {Im}
\DeclareMathOperator\range {range}

\newcommand\trio  {\triangle}

\newcommand\rc    {\right\rceil}
\newcommand\lc    {\left\lceil}
\newcommand\rf    {\right\rfloor}
\newcommand\lf    {\left\lfloor}
\newcommand\ceil  [1] {\lc #1 \rc}
\newcommand\floor [1] {\lf #1 \rf}

\newcommand\lex   {<_{lex}}

\newcommand\az    {\aleph_0}
\newcommand\uaz   {^{\aleph_0}}
\newcommand\al    {\aleph}
\newcommand\ual   {^\aleph}
\newcommand\taz   {2^{\aleph_0}}
\newcommand\utaz  { ^{\left (2^{\aleph_0} \right )}}
\newcommand\tal   {2^{\aleph}}
\newcommand\utal  { ^{\left (2^{\aleph} \right )}}
\newcommand\ttaz  {2^{\left (2^{\aleph_0}\right )}}

\newcommand\n     {$n$־יה\ }

% Math A&B shortcuts
\newcommand\logn  {\log n}
\newcommand\logx  {\log x}
\newcommand\lnx   {\ln x}
\newcommand\cosx  {\cos x}
\newcommand\sinx  {\sin x}
\newcommand\sint  {\sin \theta}
\newcommand\tanx  {\tan x}
\newcommand\tant  {\tan \theta}
\newcommand\sex   {\sec x}
\newcommand\sect  {\sec^2}
\newcommand\cotx  {\cot x}
\newcommand\cscx  {\csc x}
\newcommand\sinhx {\sinh x}
\newcommand\coshx {\cosh x}
\newcommand\tanhx {\tanh x}

\newcommand\seq   {\overset{!}{=}}
\newcommand\slh   {\overset{LH}{=}}
\newcommand\sle   {\overset{!}{\le}}
\newcommand\sge   {\overset{!}{\ge}}
\newcommand\sll   {\overset{!}{<}}
\newcommand\sgg   {\overset{!}{>}}

\newcommand\h     {\hat}
\newcommand\ve    {\vec}
\newcommand\lv    {\overrightarrow}
\newcommand\ol    {\overline}

\newcommand\mlcm  {\mathrm{lcm}}

\DeclareMathOperator{\sech}   {sech}
\DeclareMathOperator{\csch}   {csch}
\DeclareMathOperator{\arcsec} {arcsec}
\DeclareMathOperator{\arccot} {arcCot}
\DeclareMathOperator{\arccsc} {arcCsc}
\DeclareMathOperator{\arccosh}{arccosh}
\DeclareMathOperator{\arcsinh}{arcsinh}
\DeclareMathOperator{\arctanh}{arctanh}
\DeclareMathOperator{\arcsech}{arcsech}
\DeclareMathOperator{\arccsch}{arccsch}
\DeclareMathOperator{\arccoth}{arccoth}
\DeclareMathOperator{\atant}  {atan2} 
\DeclareMathOperator{\Sp}     {span} 
\DeclareMathOperator{\sgn}    {sgn} 
\DeclareMathOperator{\row}    {Row} 
\DeclareMathOperator{\adj}    {adj} 
\DeclareMathOperator{\rk}     {rank} 
\DeclareMathOperator{\col}    {Col} 
\DeclareMathOperator{\tr}     {tr}

\newcommand\dx    {\,\mathrm{d}x}
\newcommand\dt    {\,\mathrm{d}t}
\newcommand\dtt   {\,\mathrm{d}\theta}
\newcommand\du    {\,\mathrm{d}u}
\newcommand\dv    {\,\mathrm{d}v}
\newcommand\df    {\mathrm{d}f}
\newcommand\dfdx  {\diff{f}{x}}
\newcommand\dit   {\limhz \frac{f(x + h) - f(x)}{h}}

\newcommand\nt[1] {\frac{#1}{#1}}

\newcommand\limz  {\lim_{x \to 0}}
\newcommand\limxz {\lim_{x \to x_0}}
\newcommand\limi  {\lim_{x \to \infty}}
\newcommand\limh  {\lim_{x \to 0}}
\newcommand\limni {\lim_{x \to - \infty}}
\newcommand\limpmi{\lim_{x \to \pm \infty}}

\newcommand\ta    {\theta}
\newcommand\ap    {\alpha}

\renewcommand\inf {\infty}
\newcommand  \ninf{-\inf}

% Combinatorics shortcuts
\newcommand\sumnk     {\sum_{k = 0}^{n}}
\newcommand\sumni     {\sum_{i = 0}^{n}}
\newcommand\sumnko    {\sum_{k = 1}^{n}}
\newcommand\sumnio    {\sum_{i = 1}^{n}}
\newcommand\sumai     {\sum_{i = 1}^{n} A_i}
\newcommand\nsum[2]   {\reflectbox{\displaystyle\sum_{\reflectbox{\scriptsize$#1$}}^{\reflectbox{\scriptsize$#2$}}}}

\newcommand\bink      {\binom{n}{k}}
\newcommand\setn      {\{a_i\}^{2n}_{i = 1}}
\newcommand\setc[1]   {\{a_i\}^{#1}_{i = 1}}

\newcommand\cupain    {\bigcup_{i = 1}^{n} A_i}
\newcommand\cupai[1]  {\bigcup_{i = 1}^{#1} A_i}
\newcommand\cupiiai   {\bigcup_{i \in I} A_i}
\newcommand\capain    {\bigcap_{i = 1}^{n} A_i}
\newcommand\capai[1]  {\bigcap_{i = 1}^{#1} A_i}
\newcommand\capiiai   {\bigcap_{i \in I} A_i}

\newcommand\xot       {x_{1, 2}}
\newcommand\ano       {a_{n - 1}}
\newcommand\ant       {a_{n - 2}}

% Linear Algebra
\DeclareMathOperator{\chr}     {char}
\DeclareMathOperator{\diag}    {diag}
\DeclareMathOperator{\Hom}     {Hom}
\DeclareMathOperator{\Sym}     {Sym}
\DeclareMathOperator{\Asym}    {ASym}

\newcommand\lra       {\leftrightarrow}
\newcommand\chrf      {\chr(\F)}
\newcommand\F         {\mathbb{F}}
\newcommand\co        {\colon}
\newcommand\tmat[2]   {\cl{\begin{matrix}
			#1
		\end{matrix}\, \middle\vert\, \begin{matrix}
			#2
\end{matrix}}}

\makeatletter
\newcommand\rrr[1]    {\xxrightarrow{1}{#1}}
\newcommand\rrt[2]    {\xxrightarrow{1}[#2]{#1}}
\newcommand\mat[2]    {M_{#1\times#2}}
\newcommand\gmat      {\mat{m}{n}(\F)}
\newcommand\tomat     {\, \dequad \longrightarrow}
\newcommand\pms[1]    {\begin{pmatrix}
		#1
\end{pmatrix}}

\newcommand\norm[1]   {\left \vert \left \vert #1 \right \vert \right \vert}
\newcommand\snorm     {\left \vert \left \vert \cdot \right \vert \right \vert}
\newcommand\smut      {\left \la \cdot \mid \cdot \right \ra}
\newcommand\mut[2]    {\left \la #1 \,\middle\vert\, #2 \right \ra}

% someone's code from the internet: https://tex.stackexchange.com/questions/27545/custom-length-arrows-text-over-and-under
\makeatletter
\newlength\min@xx
\newcommand*\xxrightarrow[1]{\begingroup
	\settowidth\min@xx{$\m@th\scriptstyle#1$}
	\@xxrightarrow}
\newcommand*\@xxrightarrow[2][]{
	\sbox8{$\m@th\scriptstyle#1$}  % subscript
	\ifdim\wd8>\min@xx \min@xx=\wd8 \fi
	\sbox8{$\m@th\scriptstyle#2$} % superscript
	\ifdim\wd8>\min@xx \min@xx=\wd8 \fi
	\xrightarrow[{\mathmakebox[\min@xx]{\scriptstyle#1}}]
	{\mathmakebox[\min@xx]{\scriptstyle#2}}
	\endgroup}
\makeatother


% Greek Letters
\newcommand\ag        {\alpha}
\newcommand\bg        {\beta}
\newcommand\cg        {\gamma}
\newcommand\dg        {\delta}
\newcommand\eg        {\epsi}
\newcommand\zg        {\zeta}
\newcommand\hg        {\eta}
\newcommand\tg        {\theta}
\newcommand\ig        {\iota}
\newcommand\kg        {\keppa}
\renewcommand\lg      {\lambda}
\newcommand\og        {\omicron}
\newcommand\rg        {\rho}
\newcommand\sg        {\sigma}
\newcommand\yg        {\usilon}
\newcommand\wg        {\omega}

\newcommand\Ag        {\Alpha}
\newcommand\Bg        {\Beta}
\newcommand\Cg        {\Gamma}
\newcommand\Dg        {\Delta}
\newcommand\Eg        {\Epsi}
\newcommand\Zg        {\Zeta}
\newcommand\Hg        {\Eta}
\newcommand\Tg        {\Theta}
\newcommand\Ig        {\Iota}
\newcommand\Kg        {\Keppa}
\newcommand\Lg        {\Lambda}
\newcommand\Og        {\Omicron}
\newcommand\Rg        {\Rho}
\newcommand\Sg        {\Sigma}
\newcommand\Yg        {\Usilon}
\newcommand\Wg        {\Omega}

% Other shortcuts
\newcommand\tl    {\tilde}
\newcommand\op    {^{-1}}

\newcommand\sof[1]    {\left | #1 \right |}
\newcommand\cl [1]    {\left ( #1 \right )}
\newcommand\csb[1]    {\left [ #1 \right ]}
\newcommand\ccb[1]    {\left \{ #1 \right \}}

\newcommand\bs        {\blacksquare}
\newcommand\dequad    {\!\!\!\!\!\!}
\newcommand\dequadd   {\dequad\duquad}

\renewcommand\phi     {\varphi}

\newtheorem{Theorem}{משפט}
\theoremstyle{definition}
\newtheorem{definition}{הגדרה}
\newtheorem{Lemma}{למה}
\newtheorem{Remark}{הערה}
\newtheorem{Notion}{סימון}


\newcommand\theo  [1] {\begin{Theorem}#1\end{Theorem}}
\newcommand\defi  [1] {\begin{definition}#1\end{definition}}
\newcommand\rmark [1] {\begin{Remark}#1\end{Remark}}
\newcommand\lem   [1] {\begin{Lemma}#1\end{Lemma}}
\newcommand\noti  [1] {\begin{Notion}#1\end{Notion}}


%! ~~~ Document ~~~

\author{שחר פרץ}
\title{אלגברה לינארית ביקום מקביל}
\begin{document}
	\maketitle
	זה השלב שבן מעביר עוד שיעור על תורת הקטגוריות במסווה של אגלברה לינארית. 
	
	ביקום מקביל, נלמד קוםד העתקות ואז מטריצות. כלומר: 
	\[ \F^{n} \overset{\wg_B}{\cong} V \overset{T}{\to} W \overset{\wg_c}{\cong} \F^{m} \overset{T_A}{\to} \F^n \]
	(תציירו את זה בריבוע)
	
	נוכיח שהדיאגרמה מתחלפת. 
	\theo{\[ \wg_C \circ T = T_a \circ \wg_B \]}
	\begin{proof}
		יהי $v \in V$. אז: 
		\[ (\wg_c \circ T)(v) = \wg_C(T(v)) = [Tv]_C = [T]^B_C[v]_B = T_A([v]_B) = (T_A \circ \wg_B)v \]
	\end{proof}
	\theo{$\wg_c(\Img T) = \Img T_A \land \wg_B(\ker T) = \ker T_A$}\begin{proof}
		נתחיל מהשוויון התמונות. 
		\begin{itemize}
			\item[$\subseteq$]יהי $y \in \wg_C(\Img T)$. אז: 
			\[ \exists y' \in \Img T\co y = \wg_C(y'), y' \in \Img T \implies \exists x' \in V \co T(x') = y' \]
			ידוע $\wg_C \circ T = T_A \circ \wg_C$. לכן: 
			\[ y = (\wg_C \circ T)(x) = (T_A \circ \wg_B)(x') = T_A(\wg_B(x')) \in \Img T_A \]
			\item[$\reflectbox{$\subseteq$}$] ידוע $y \in \Img T_A$. נוכיח $y \in \wg(\Img T)$: 
			\[ \exists c \in \F^{n} \co T_A(x) = y \implies \exists x' \in V \co \wg_B(x') = x \implies (\wg_C \circ T)(x') = (T_A \circ \wg_B)(x') = y \in \wg_C(\Img T) \]
		\end{itemize}
		עתה נוכיח את השוויון הקרנלים. 
		\begin{itemize}
			\item[$\subseteq$]יהי $x \in \wg_B(\ker T)$. לכן:
			\[ \exists x' \in \ker T \co \wg_B(x') = x \implies \underbrace{(\wg_C \circ T)(x')}_{\wg_c(0) = 0} = (T_A \circ \wg_B)(x') = T_A(x) \implies x\in \ker T_A \]
			\item[$\reflectbox{$\subseteq$}$]יהי $x \in \ker T_A$ (הערה: כאן כבר היה אפשר לסיים עם שיקולי ממדים מהסעיף הקודם). 
			$\exists x' \in V \co \wg_B(x') = x \implies (\wg_C \circ T)(x') = (T_A \circ \wg_B)(x') = T_A(x) = 0$
			רצינו להוכיח ש־$x \in \wg_B(\ker T)$, כלומר $\exists x' \in \ker T$ כך ש־$x = \wg_B(x')$. מהיות $(\wg_C \circ T)(x') = 0$ מהמשוואות לעיל, ומהיות $\wg_C$ איזו ובפרט חח''ע, נקבל $x' \in \ker T$ כדרוש. 
		\end{itemize}
	\end{proof}
	
	\dotfill
	
	\textbf{תרגיל: }תהי $T \co V \to V$, כך ש־$V = \R_{\le 2}[x]$ ו־$T(p(x)) = p(x) + p'(x) - p(0) \cdot \frac{x^{2}}{2}$. מצאו בסיסים לגרעין ולתמונה. 
	
	נצטרך לתרגם דברים למטריצות. 
	
	\textbf{פתרון. }נבחר בסיס ל־$B = C = (1, x, x^2)$. נבנה את: 
	\[ [T]^B_C = \pms{\vert &\vert& \vert \\ [T(1)]_B & [T(x)]_B & [T(x^2)]_B & \vert & \vert & \vert} \]
	נבחין ש־: 
	\begin{alignat*}{9}
		T(1) &= 1 + 0 - \frac{x^2}{2} &\implies [T(1)]_B = \pms{1 \\ 0 \\ -0.5} \\
		T(x) &= x + 1 - 0 &\implies [T(x)]_B = \pms{1 \\ 1 \\ 0} \\
		T(x^2) &= x^2 + 2x - 0 &\implies [T(x^2)]_B = \pms{0 \\ 2 \\1} 
	\end{alignat*}
	ולכן: 
	\[ [T]_B = \pms{1 & 1 & 0 \\ 0 &1 & 2 \\ -0.5 & 0 & 1} \]
	נמצא את המרחב המאפס: 
	\[ \nc [T]_B = \pms{1 & 1 & 0 \\ 0 &1 & 2 \\ -0.5 & 0 & 1} \rrr{} \pms{1 & 1 & 0 \\ 0 & 1 & 2 \\ 0 & 0 & 0} \rrr{} \pms{1 & 0 & -2 \\ 0 & 1 & 2 \\ 0 & 0 & 0} \]
	נבחין ש־: 
	\[ \Img [T]_A = \ccb{\pms{1 \\0 \\ -0.5}, \pms{1 \\ 1 \\ 0}} \implies \Img T = \Sp(1, -0.5x^2, 1 + x) \]
	\[ \nc[T]_A = \ccb{\pms{2t \\ -t \\ t} \mid t \in \R} = \Sp\pms{2 \\ -2 \\ 1} \implies \ker T = 2 - 2x + x^2 \]
	
	\dotfill 
	
	תהי $T \co V \to W$ ט''ל, ו־$V$ נ''ס. נגדיר $\rk T := \dim \Img T$. אם $A$ מייצגת את $T$, אז $\rk T = \rk A$. 
	
	\textbf{תזכורת: }$T \co V \to W$ איזו' אמ''מ $[T]_C^B$ איזו' \begin{proof}
		ידוע שאם $T$ איזו' אז $T\op$ איזו', ולכן: 
		\[ [T]_B^C \cdot [T\op]_C^B = [T \circ T\op]_B = [id]_B = I_n \]
		ולכן $[T\op]_C^B$ ההופכית של $[T]_B^C$. 
	\end{proof}
	
	\theo{תהי $A \in M_n(\F)$. אז $A$ הפיכה אמ''מ $\rk A = n$. }\begin{proof}
		כמו שאר הדברים בשיעור הזה, נרצה להוכיח זאת בניסוחים של העתקות לינאריות. נגדיר $T_A \co \F^n \to \F^n$ ע''י $T_A(v) = Av$. נגדיר את $\ec = (e_1 \dots e_n)$ הבסיס הסטנדרטי. אז $[T_A]_\ec = A$. נבחין ש־: 
		\[ n = \rk A = \rk [T_A]_\ec = \rk T_A = n \iff \dim \Img T_A = n \iff T_A \ \text{על} \iff T_A \ \text{איזו'} \iff A \text{הפיכה} \]
		כאשר הטענה האחרונה נובעת מהמשפט הקודם. 
	\end{proof}
	
	\dotfill
	
	\theo{יהי $U \overset{T_1}{V} \overset{T_2}{\to} W$. 
	\begin{enumerate}
		\item $\rk T_2 \circ T_1 \le \rk T_2$
		\item $\rk T_1 \circ T_2 \le \rk T_1$
		\item אם $T_1$ על אז $\rk (T_2 \circ T_1) = \rk T_2$
		\item אם $T_2$ חח''ע אז $\rk (T_2 \circ T_1) = \rk T_1$
	\end{enumerate}}
	\begin{proof}
		נתחיל מלהוכיח תמיד $\le$ (את 1 + 2)
		\begin{enumerate}
			\item יש הכלה $\Img T_2 \circ T_1 \subseteq T_2$. יהי $w \in \Img T_2 \circ T_1$. אז קיים $u \in U$ כך ש־$T_2 \circ T_1(u)ידוע = w$. לכן $T_1u =: v \in V$, כלומר $w = T_2 v \implies w \in \Img T_2$ כדרוש. 
			\item הפעם אין הכלה על התמונות (כי הם במכלל המ''וים שונים), אך יש הכלה על הגרעינים ויש לנו את משפט הממדים. 
			\[ \dim U =: n, n = \dim \ker T_1 + \underbrace{\dim \Img T_1}_{\rk T_1} = \dim \ker T_1 \circ T_2 + \underbrace{\dim \Img T_1 \circ T_2}_{\rk T_1 \circ T_2} \]
			לכן מספיק להוכיח ש־$\ker T_2 \circ T_1 \supseteq \ker T_1$. 
		\end{enumerate}
		עתה נותר להוכיח $\ge$, שלא יהיה נכון באופן כללי אלא תחת התנאים (על/חח''ע) בלבד. כלומר, נותר להוכיח $\Img T_2 \subseteq T_2 \circ T_1$ אם $T_1$ על, ו־$\ker T_2 \circ T_1 \subseteq \ker T_1$ אם $T_2$ חח,ע. 
		\begin{enumerate}
			\skipitems{2}
			\item יהי $w \in \Img T_2$. נרצה להוכיח $w \in \Img T_2 \circ T_1$. קיים $v \in V$ מקור כך ש־$T_2 v = w$, ידוע $T_1$ על ולכן קיים $u \in U$ כך ש־$T_1u = v$. סה''כ $(T_2 \circ T_1)v = w$ כלומר $w \in \Img T_2 \circ T_1 $ כדרוש. 
			\item יהי $u \in \ker T_2 \circ T_1$. אז ידוע $T_2(T_1u) = (T_2 \circ T_1)u = 0$. מהיות $T_2$ חח''ע נקבל $T_1u = 0$ כלומר $u \in \ker T_1$ כדרוש. 
		\end{enumerate}
	\end{proof}
	
	מהחלק הראשון של ההוכחה, הוכחנו למעשה ש־$\rk AB \le \rk A \rk B$. פרטים בקרוב. 
	\theo{יהיו $A \in M_{p \times n}(\F), A \in M_{m \times p}(\F)$. אז: 
	\begin{itemize}
		\item $\rk AB \le \rk A$
		\item $\rk AB \le \rk B$ 
		\item אם $B$ הפיכה אז $\rk AB = \rk A$
		\item אם $A$ הפיכה אז $\rk AB = \rk B$
	\end{itemize}}
	זה למעשה די זהה לחלק הראשון של ההוכחה הקודמת. 
	\begin{proof}[הוכחת 3. ]$B$ הפיכה, כלומר
		ידוע $p = n$, $\F^n \overset{T_B}{\to} \F^n \overset{T_A}{\to}\F^m$. נסמן $\ec_n$ סטנדרטי של $\F^n$ ו־$\ec_m$ סטנדרטי של $\F^m$. לכן: 
		\[ [T_A]^{\ec_m}_{\ec_n} = A, \ [T_B]^{\ec_n}_{\ec_m} = B \]
		נבחין ש־: 
		\[ T_{AB} \co \F^n \to \F^M \implies [T_{AB}]^{\ec_n}_{\ec_m} = AB \]
		סה''כ: 
		\[ \rk AB = \rk T_{AB} = \rk (T_A \circ T_B) = \cdots \]
		$B$ הפיכה אמ''מ $T_B$ איזו' אמ'',מ $T_B$ על, וסה''כ אנחנו בסיטואציה של 3 מהסעיף הקודם. לכן: 
		\[ \cdots = \rk T_A = \rk [T_A]^{\ec_n}_{\ec_m} = \rk A \]
		כדרוש. 
	\end{proof}
	
	\textbf{תרגיל. }להוכיח שאם $A$ הפיכה מימין ומשמאל, אז היא ריבועית. (לשם כך צריך להגדיר הפיכות ללא־ריבועיות). 
	
	\section{}
	יהיו $V, W$ תמ''וים מעל $\F$, ו־$T \co V \to W$ ט''ל. בעבור $U \subseteq W$ נגדיר $T\op(U) = \{v \in V \co Tv \in U\}$. 
	
	\begin{enumerate}
		\item נוכיח $T\op(U) \subseteq V$ תמ''ו \begin{proof}
			ע''מ להוכיח ש־$T\op(U)$ תמ''ו מספיק להוכיח את ההבאים: 
			\begin{itemize}
				\item $0_V \in T\op(U)$ כי $T(0_V) = 0_W \in U$ כדרוש. 
				\item $\forall v, u \in T\op(U), \forall \ag \in \F\co v \in T\op U \iff Tv \in U, \ T$ אז $v + \ag u \in T\op(U) Tv + \ag Tu \in U$ צריך לתת הסברים על לינאריות וסגירות של $U$. 
			\end{itemize}
		\end{proof}
		\item נגדיר $V = W = \R^3$, ו־: 
		\[ T\pms{x \\ y \\ z} = \pms{x + 2y + 3z \\ 4x + 5y + 6z \\ 7x + 8y  + 9z} \quad U = \Sp\ccb{\pms{1 \\ 4 \\ 7}, \pms{1 \\ 0 \\ 0}} \]
		נבחין ש־: 
		\[ T\op(U) \overset{\Sp}{=} \{v_1 \dots v_k\} \implies \underbrace{T(\Sp(v_1 \dots v_k))}_{\Sp(Tv_1 \dots Tv_k)} \subseteq U \]
		שימו לב! נמצאים שם לא רק המקורות של הוקטורים ב־$U$, אלא גם $\ker A$. זה יראה מוזר: נקבל שקבוצת הפתרונות של $Ax = (1, 4, 7)$, נקבל משהו כמו: 
		\[ v = \pms{1 \\ 0 \\ 0} + t\pms{1 \\ -2 \\ 1} \]
		אבל זה בסדר. כי למעשה גם כל הקרנל נמצא ב־$T\op(U)$ וכשמאחדים אותם זה מסתדר. 
		(צריך לכתוב פורמלית)
		\item נוכיח באופן כללי שאם $V$ נ''ס אז $\dim (T\op (U)) = \dim \ker T + \dim (\Img T \cap U)$. \begin{proof}
			נגדיר העתקה $S \co T\op(U) \to W$ כך ש־$\forall v \in T\op(U) \co Sv = Tv$ ונוכיח ש־$\dim(T\op(U)) = \dim \ker S + \dim \Img S$ (כי אפשר להוכיח ש־$\ker T = \ker S$). 
		\end{proof}
	\end{enumerate}
	
	\section{}
	נתונה מטריצה $A \in M_3(\R)$ כך ש־: 
	\[ A = \pms{1 & 2 & 3 \\ 2 & 4 & 5 \\ 3 & 5 & 5} \ U = \ccb{\pms{x \\ y \\ z} \in \R^3 \co x + y + z = 0} \]
	תהא $T \co U \to \R^3$ המקיימת $\forall u \in U \co Tu = AU$. עבור כל אחת מהמטריצות הבאות, תנו דוגמה לזוג ביסיסם $B, C$ של $U, \R^3$ כך ש־$[T]^B_C = M_i$. 
	\begin{enumerate}
		\item $M_1 = A$
		\item $M_2 = \pms{1 & 2 \\ 2 & 4 \\ 3 & 6}$
		\item $M_3 = \pms{1 & 0 \\ 0 & 1 \\ 0 & 0}$
	\end{enumerate}
	
	נבחין שב' נפסל כי הוא ת''ל ושיקולי ממדים לא מאפשרים זאת. ג' אפשרי באופן הבא: 
	\[ B = \pms{1 \\ - 1\\ 0}, \pms{1 \\ 0 \\ 1} \]
	ואז: 
	\[ \csb{T\pms{1 \\ - 1\\ 0}}_C = \pms{1 \\ 2 \\ 0}, \ \csb{T\pms{1 \\ 0 \\ 1}} = \pms{7 \\ 1 \\ 0} \]
	ואת הוקטור האחרון נבחר איך שבא לנו. 
	
	
	\ndoc
\end{document}