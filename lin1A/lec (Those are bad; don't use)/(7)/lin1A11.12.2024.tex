%! ~~~ Packages Setup ~~~ 
\documentclass[]{article}
\usepackage{lipsum}
\usepackage{rotating}


% Math packages
\usepackage[usenames]{color}
\usepackage{forest}
\usepackage{ifxetex,ifluatex,amsmath,amssymb,mathrsfs,amsthm,witharrows,mathtools,mathdots}
\WithArrowsOptions{displaystyle}
\renewcommand{\qedsymbol}{$\blacksquare$} % end proofs with \blacksquare. Overwrites the defualts. 
\usepackage{cancel,bm}
\usepackage[thinc]{esdiff}


% tikz
\usepackage{tikz}
\usetikzlibrary{graphs}
\newcommand\sqw{1}
\newcommand\squ[4][1]{\fill[#4] (#2*\sqw,#3*\sqw) rectangle +(#1*\sqw,#1*\sqw);}


% code 
\usepackage{listings}
\usepackage{xcolor}

\definecolor{codegreen}{rgb}{0,0.35,0}
\definecolor{codegray}{rgb}{0.5,0.5,0.5}
\definecolor{codenumber}{rgb}{0.1,0.3,0.5}
\definecolor{codeblue}{rgb}{0,0,0.5}
\definecolor{codered}{rgb}{0.5,0.03,0.02}
\definecolor{codegray}{rgb}{0.96,0.96,0.96}

\lstdefinestyle{pythonstylesheet}{
	language=Java,
	emphstyle=\color{deepred},
	backgroundcolor=\color{codegray},
	keywordstyle=\color{deepblue}\bfseries\itshape,
	numberstyle=\scriptsize\color{codenumber},
	basicstyle=\ttfamily\footnotesize,
	commentstyle=\color{codegreen}\itshape,
	breakatwhitespace=false, 
	breaklines=true, 
	captionpos=b, 
	keepspaces=true, 
	numbers=left, 
	numbersep=5pt, 
	showspaces=false,                
	showstringspaces=false,
	showtabs=false, 
	tabsize=4, 
	morekeywords={as,assert,nonlocal,with,yield,self,True,False,None,AssertionError,ValueError,in,else},              % Add keywords here
	keywordstyle=\color{codeblue},
	emph={var, List, Iterable, Iterator},          % Custom highlighting
	emphstyle=\color{codered},
	stringstyle=\color{codegreen},
	showstringspaces=false,
	abovecaptionskip=0pt,belowcaptionskip =0pt,
	framextopmargin=-\topsep, 
}
\newcommand\pythonstyle{\lstset{pythonstylesheet}}
\newcommand\pyl[1]     {{\lstinline!#1!}}
\lstset{style=pythonstylesheet}

\usepackage[style=1,skipbelow=\topskip,skipabove=\topskip,framemethod=TikZ]{mdframed}
\definecolor{bggray}{rgb}{0.85, 0.85, 0.85}
\mdfsetup{leftmargin=0pt,rightmargin=0pt,innerleftmargin=15pt,backgroundcolor=codegray,middlelinewidth=0.5pt,skipabove=5pt,skipbelow=0pt,middlelinecolor=black,roundcorner=5}
\BeforeBeginEnvironment{lstlisting}{\begin{mdframed}\vspace{-0.4em}}
	\AfterEndEnvironment{lstlisting}{\vspace{-0.8em}\end{mdframed}}


% Deisgn
\usepackage[labelfont=bf]{caption}
\usepackage[margin=0.6in]{geometry}
\usepackage{multicol}
\usepackage[skip=4pt, indent=0pt]{parskip}
\usepackage[normalem]{ulem}
\forestset{default}
\renewcommand\labelitemi{$\bullet$}
\usepackage{titlesec}
\titleformat{\section}[block]
{\fontsize{15}{15}}
{\sen \dotfill (\thesection)\she}
{0em}
{\MakeUppercase}
\usepackage{graphicx}
\graphicspath{ {./} }


% Hebrew initialzing
\usepackage[bidi=basic]{babel}
\PassOptionsToPackage{no-math}{fontspec}
\babelprovide[main, import, Alph=letters]{hebrew}
\babelprovide[import]{english}
\babelfont[hebrew]{rm}{David CLM}
\babelfont[hebrew]{sf}{David CLM}
\babelfont[english]{tt}{Monaspace Xenon}
\usepackage[shortlabels]{enumitem}
\newlist{hebenum}{enumerate}{1}

% Language Shortcuts
\newcommand\en[1] {\begin{otherlanguage}{english}#1\end{otherlanguage}}
\newcommand\sen   {\begin{otherlanguage}{english}}
	\newcommand\she   {\end{otherlanguage}}
\newcommand\del   {$ \!\! $}

\newcommand\npage {\vfil {\hfil \textbf{\textit{המשך בעמוד הבא}}} \hfil \vfil \pagebreak}
\newcommand\ndoc  {\dotfill \\ \vfil {\begin{center} {\textbf{\textit{שחר פרץ, 2024}} \\ \scriptsize \textit{נוצר באמצעות תוכנה חופשית בלבד}} \end{center}} \vfil	}

\newcommand{\rn}[1]{
	\textup{\uppercase\expandafter{\romannumeral#1}}
}

\makeatletter
\newcommand{\skipitems}[1]{
	\addtocounter{\@enumctr}{#1}
}
\makeatother

%! ~~~ Math shortcuts ~~~

% Letters shortcuts
\newcommand\N     {\mathbb{N}}
\newcommand\Z     {\mathbb{Z}}
\newcommand\R     {\mathbb{R}}
\newcommand\Q     {\mathbb{Q}}
\newcommand\C     {\mathbb{C}}

\newcommand\ml    {\ell}
\newcommand\mj    {\jmath}
\newcommand\mi    {\imath}

\newcommand\powerset {\mathcal{P}}
\newcommand\ps    {\mathcal{P}}
\newcommand\pc    {\mathcal{P}}
\newcommand\ac    {\mathcal{A}}
\newcommand\bc    {\mathcal{B}}
\newcommand\cc    {\mathcal{C}}
\newcommand\dc    {\mathcal{D}}
\newcommand\ec    {\mathcal{E}}
\newcommand\fc    {\mathcal{F}}
\newcommand\nc    {\mathcal{N}}
\newcommand\sca   {\mathcal{S}} % \sc is already definded
\newcommand\rca   {\mathcal{R}} % \rc is already definded

\newcommand\Si    {\Sigma}

% Logic & sets shorcuts
\newcommand\siff  {\longleftrightarrow}
\newcommand\ssiff {\leftrightarrow}
\newcommand\so    {\longrightarrow}
\newcommand\sso   {\rightarrow}

\newcommand\epsi  {\epsilon}
\newcommand\vepsi {\varepsilon}
\newcommand\vphi  {\varphi}
\newcommand\Neven {\N_{\mathrm{even}}}
\newcommand\Nodd  {\N_{\mathrm{odd }}}
\newcommand\Zeven {\Z_{\mathrm{even}}}
\newcommand\Zodd  {\Z_{\mathrm{odd }}}
\newcommand\Np    {\N_+}

% Text Shortcuts
\newcommand\open  {\big(}
\newcommand\qopen {\quad\big(}
\newcommand\close {\big)}
\newcommand\also  {\text{, }}
\newcommand\defi  {\text{ definition}}
\newcommand\defis {\text{ definitions}}
\newcommand\given {\text{given }}
\newcommand\case  {\text{if }}
\newcommand\syx   {\text{ syntax}}
\newcommand\rle   {\text{ rule}}
\newcommand\other {\text{else}}
\newcommand\set   {\ell et \text{ }}
\newcommand\ans   {\mathscr{A}\!\mathit{nswer}}

% Set theory shortcuts
\newcommand\ra    {\rangle}
\newcommand\la    {\langle}

\newcommand\oto   {\leftarrow}

\newcommand\QED   {\quad\quad\mathscr{Q.E.D.}\;\;\blacksquare}
\newcommand\QEF   {\quad\quad\mathscr{Q.E.F.}}
\newcommand\eQED  {\mathscr{Q.E.D.}\;\;\blacksquare}
\newcommand\eQEF  {\mathscr{Q.E.F.}}
\newcommand\jQED  {\mathscr{Q.E.D.}}

\newcommand\dom   {\mathrm{dom}}
\newcommand\Img   {\mathrm{Im}}
\newcommand\range {\mathrm{range}}

\newcommand\trio  {\triangle}

\newcommand\rc    {\right\rceil}
\newcommand\lc    {\left\lceil}
\newcommand\rf    {\right\rfloor}
\newcommand\lf    {\left\lfloor}

\newcommand\lex   {<_{lex}}

\newcommand\az    {\aleph_0}
\newcommand\uaz   {^{\aleph_0}}
\newcommand\al    {\aleph}
\newcommand\ual   {^\aleph}
\newcommand\taz   {2^{\aleph_0}}
\newcommand\utaz  { ^{\left (2^{\aleph_0} \right )}}
\newcommand\tal   {2^{\aleph}}
\newcommand\utal  { ^{\left (2^{\aleph} \right )}}
\newcommand\ttaz  {2^{\left (2^{\aleph_0}\right )}}

\newcommand\n     {$n$־יה\ }

% Math A&B shortcuts
\newcommand\logn  {\log n}
\newcommand\logx  {\log x}
\newcommand\lnx   {\ln x}
\newcommand\cosx  {\cos x}
\newcommand\cost  {\cos \theta}
\newcommand\sinx  {\sin x}
\newcommand\sint  {\sin \theta}
\newcommand\tanx  {\tan x}
\newcommand\tant  {\tan \theta}
\newcommand\sex   {\sec x}
\newcommand\sect  {\sec^2}
\newcommand\cotx  {\cot x}
\newcommand\cscx  {\csc x}
\newcommand\sinhx {\sinh x}
\newcommand\coshx {\cosh x}
\newcommand\tanhx {\tanh x}

\newcommand\seq   {\overset{!}{=}}
\newcommand\slh   {\overset{LH}{=}}
\newcommand\sle   {\overset{!}{\le}}
\newcommand\sge   {\overset{!}{\ge}}
\newcommand\sll   {\overset{!}{<}}
\newcommand\sgg   {\overset{!}{>}}

\newcommand\h     {\hat}
\newcommand\ve    {\vec}
\newcommand\lv    {\overrightarrow}
\newcommand\ol    {\overline}

\newcommand\mlcm  {\mathrm{lcm}}

\DeclareMathOperator{\sech}   {sech}
\DeclareMathOperator{\csch}   {csch}
\DeclareMathOperator{\arcsec} {arcsec}
\DeclareMathOperator{\arccot} {arcCot}
\DeclareMathOperator{\arccsc} {arcCsc}
\DeclareMathOperator{\arccosh}{arccosh}
\DeclareMathOperator{\arcsinh}{arcsinh}
\DeclareMathOperator{\arctanh}{arctanh}
\DeclareMathOperator{\arcsech}{arcsech}
\DeclareMathOperator{\arccsch}{arccsch}
\DeclareMathOperator{\arccoth}{arccoth}
\DeclareMathOperator{\atant}  {atan2} 

\newcommand\dx    {\,\mathrm{d}x}
\newcommand\dt    {\,\mathrm{d}t}
\newcommand\dtt   {\,\mathrm{d}\theta}
\newcommand\du    {\,\mathrm{d}u}
\newcommand\dv    {\,\mathrm{d}v}
\newcommand\df    {\mathrm{d}f}
\newcommand\dfdx  {\diff{f}{x}}
\newcommand\dit   {\limhz \frac{f(x + h) - f(x)}{h}}

\newcommand\nt[1] {\frac{#1}{#1}}

\newcommand\limz  {\lim_{x \to 0}}
\newcommand\limxz {\lim_{x \to x_0}}
\newcommand\limi  {\lim_{x \to \infty}}
\newcommand\limh  {\lim_{x \to 0}}
\newcommand\limni {\lim_{x \to - \infty}}
\newcommand\limpmi{\lim_{x \to \pm \infty}}

\newcommand\ta    {\theta}
\newcommand\ap    {\alpha}

\renewcommand\inf {\infty}
\newcommand  \ninf{-\inf}

% Combinatorics shortcuts
\newcommand\sumnk     {\sum_{k = 0}^{n}}
\newcommand\sumni     {\sum_{i = 0}^{n}}
\newcommand\sumnko    {\sum_{k = 1}^{n}}
\newcommand\sumnio    {\sum_{i = 1}^{n}}
\newcommand\sumai     {\sum_{i = 1}^{n} A_i}
\newcommand\nsum[2]   {\reflectbox{\displaystyle\sum_{\reflectbox{\scriptsize$#1$}}^{\reflectbox{\scriptsize$#2$}}}}

\newcommand\bink      {\binom{n}{k}}
\newcommand\setn      {\{a_i\}^{2n}_{i = 1}}
\newcommand\setc[1]   {\{a_i\}^{#1}_{i = 1}}

\newcommand\cupain    {\bigcup_{i = 1}^{n} A_i}
\newcommand\cupai[1]  {\bigcup_{i = 1}^{#1} A_i}
\newcommand\cupiiai   {\bigcup_{i \in I} A_i}
\newcommand\capain    {\bigcap_{i = 1}^{n} A_i}
\newcommand\capai[1]  {\bigcap_{i = 1}^{#1} A_i}
\newcommand\capiiai   {\bigcap_{i \in I} A_i}

\newcommand\xot       {x_{1, 2}}
\newcommand\ano       {a_{n - 1}}
\newcommand\ant       {a_{n - 2}}

% Linear Algebra
\DeclareMathOperator{\chr}    {char}

\newcommand\lra       {\leftrightarrow}
\newcommand\chrf      {\chr(\F)}
\newcommand\F         {\mathbb{F}}
\newcommand\co        {\colon}
\newcommand\tmat[2]   {\cl{\begin{matrix}
			#1
		\end{matrix}\, \middle\vert\, \begin{matrix}
			#2
\end{matrix}}}

\makeatletter
\newcommand\rrr[1]    {\xxrightarrow{1}{#1}}
\newcommand\rrt[2]    {\xxrightarrow{1}[#1]{#2}}
\newcommand\mat[2]    {M_{#1\times#2}}
\newcommand\tomat     {\, \dequad \longrightarrow}

% someone's code from the internet: https://tex.stackexchange.com/questions/27545/custom-length-arrows-text-over-and-under
\makeatletter
\newlength\min@xx
\newcommand*\xxrightarrow[1]{\begingroup
	\settowidth\min@xx{$\m@th\scriptstyle#1$}
	\@xxrightarrow}
\newcommand*\@xxrightarrow[2][]{
	\sbox8{$\m@th\scriptstyle#1$}  % subscript
	\ifdim\wd8>\min@xx \min@xx=\wd8 \fi
	\sbox8{$\m@th\scriptstyle#2$} % superscript
	\ifdim\wd8>\min@xx \min@xx=\wd8 \fi
	\xrightarrow[{\mathmakebox[\min@xx]{\scriptstyle#1}}]
	{\mathmakebox[\min@xx]{\scriptstyle#2}}
	\endgroup}
\makeatother


% Greek Letters
\newcommand\ag        {\alpha}
\newcommand\bg        {\beta}
\newcommand\cg        {\gamma}
\newcommand\dg        {\delta}
\newcommand\eg        {\epsi}
\newcommand\zg        {\zeta}
\newcommand\hg        {\eta}
\newcommand\tg        {\theta}
\newcommand\ig        {\iota}
\newcommand\kg        {\keppa}
\renewcommand\lg      {\lambda}
\newcommand\og        {\omicron}
\newcommand\rg        {\rho}
\newcommand\sg        {\sigma}
\newcommand\yg        {\usilon}
\newcommand\wg        {\omega}

\newcommand\Ag        {\Alpha}
\newcommand\Bg        {\Beta}
\newcommand\Cg        {\Gamma}
\newcommand\Dg        {\Delta}
\newcommand\Eg        {\Epsi}
\newcommand\Zg        {\Zeta}
\newcommand\Hg        {\Eta}
\newcommand\Tg        {\Theta}
\newcommand\Ig        {\Iota}
\newcommand\Kg        {\Keppa}
\newcommand\Lg        {\Lambda}
\newcommand\Og        {\Omicron}
\newcommand\Rg        {\Rho}
\newcommand\Sg        {\Sigma}
\newcommand\Yg        {\Usilon}
\newcommand\Wg        {\Omega}

% Other shortcuts
\newcommand\tl    {\tilde}
\newcommand\op    {^{-1}}

\newcommand\sof[1]    {\left | #1 \right |}
\newcommand\cl [1]    {\left ( #1 \right )}
\newcommand\csb[1]    {\left [ #1 \right ]}

\newcommand\bs        {\blacksquare}
\newcommand\dequad    {\!\!\!\!\!\!}
\newcommand\dequadd   {\dequad\duquad}
\renewcommand\phi     {\varphi}

%! ~~~ Document ~~~

\author{שחר פרץ}
\title{ליניארית 7}
\begin{document}
	\maketitle
	\section{\en{Just Reminders}}
	תזכורת: יהי $v \in V$ בעל $B = (v_1, \dots, v_n)$ בסיס, אז $v = \sum_{i = 1}^{|B|}\lg_iv_i$. הגדרנו $[v]_B = (\lg_1, \dots, \lg_n)$. הראינו ש־$[v]_B$ לכל $v \in B$ יכולים להגדיר באופן איזומטרי כל העתקה ליניארית. כלומר, $\phi(v_i) = u_i$ לכל $1 \le i \le n$ בהנחה שהיא ההעתקה ליניארית, אז היא יחידה (כאשר $\u_i \in U, v_i \in B$). פורמלית: 
	
	\textbf{טענה. }יהיו $V, U$ מ"ו עם בסיסים. $B$ בסיס של $V$. נסמן $B = (v_1, \dots, v_n)$ ויהיו $u_1, \dots, u_n \in U$. תהי $f \co V \to U$ כך ש־$\forall 1 \le i \le n \co \phi(v_i) = u_i$. נוכיח שקיימת יחידה $\phi$ ליניארית כך ש־$\phi(v_i) = u_i$. 
	\begin{proof}
		\textbf{קיום. }עבור $v \in V$ קיימים ויחידים $\{\lg_i\}_{i \in [n]}$ כך ש־$\sum \lg_iv_i = v$. (כי $B$ בסיס ולכן לכל וקטור קיימת ויחידה הצגה שלו כצירוף ליניארי). נסמן: 
		\[ f(v) = \sum\lg_if(v_i) = \sum \lg_i u_i \] 
		ובפרט יתקיים $\phi(v_i) = u_i$ כרצוי. 
		
		\textbf{ליניאריות. }נסמן $v = \sum\ag_iv_i, \ w = \sum \bg_iv_i$
		נראה ש־$\phi$ ליניארית. יהיו $v, w \in V$ ויהיו $\lg_1, \lg_2 \in F$ ונראה ש־$\phi(\lg_1v + \lg_2w) = \lg_1\phi(v) + \lg_2\phi(u)$. ביחד עם הגדרת $\phi$ נגרר מהתון: 
		\[ \phi(\lg_1 v + \lg_2 w) = \phi = \phi \cl{\sum v_i(\lg_1 \ag_i + \lg_2 \bg_i)} \]. לפי $\phi$ שהדרנו
		\[ \sum(\lg_1\ag_1 + \lg_2\bg_i)\phi(v_i) = \lg_1\sum\ag_iu_i + \lg_2\sum\bg_iu_i) \]
	\end{proof}
	
	\textbf{יחידות}. נתבונן ב$\lg - \sum 0\lg_i + v_u -=\sum \lg_if(v_i) = \sum \lg_i \psi (v_i)$
	
	\textbf{סימון. }יהי $V$ מ"ו כך ש־$\dim V =n$ ו־$B = (v_1, \dots, v_n)$ בסיס. נזמן $[v]_B$ להיות \textit{הקורדינאטות לפי הבסיס $B$} (או משהו דומה המרצה שכח) וההגדרה נמצאת בכל מקרה בפסקה הראשונ למעלה. 
	
	\textit{הערה. }מוגדר יכ לכל $u \in V$ קיימים ויחידים $(\lg_i)_{i \in [n]}$ כך ש־$\sum\lg_iv_i = w$. 
	
	\textbf{משפט. }יהי $V$ מ"ו ממימד $n$ ו־$B = \cdots$. אז $\phi = \phi_B \co F^n \to V$ שמוגדרת להיות $\phi(\la_1, \dots, v_n) = \sum_{i = 1}^{n}\lg_iv_i$ היא איזו' וגם ההופכית היא $\phi\op(v) = [v]_B$. 
	
	ההוכחה הוצגה בשיעור הקודם. 
	
	\textit{אינטואיציה. }בהינתן אתם יודעים מה, אם ידוע לכל $i$ מה הוא $\phi(v_i)$ אז ידוע $V$. 
	
	עבור $C$ בסיס של $[\phi(v_i)]_C$ לדוגמה $\R^3 \to \R^2$ העתקת השיכון (מוחק את הקורדינאטה האחרונה), יתקיים: 
	\[ (x, y, z) \mapsto (x, y) \implies \phi((1, 0, 0)) = (1, 0) \]
	ניתן את זה עכשיו כהגדרה מסודרת. 
	
	\textbf{הגדרה. }בעבור $\phi \co V \to U$, $V, U$ מ"ו, $B = (v_1, \dots, v_n), \ C = (u_1, \dots, u_n)$ בסיסים ל־$V, U$ בהתאמה, ה\textit{מטריצה המייצגת} לבסיסים של $B$ של $V$, $C$ של $U$: 
	\[ 
		[\phi]_C^B = \begin{pmatrix}
			\vdots && \vdots \\
			[\phi(v_1)]_C & \cdots & [\phi(v_n)]_C \\
			\vdots && \vdots
		\end{pmatrix} \in M_{m \times n}(F)
	 \]
	גובה $m$, רוחב $n$. 
	\textbf{דוגמה. }נתבונן ב־$\phi \co F^2 \to F^3$ כך ש־$\phi(x, y) \mapsto (x, x + y x + 2y)$. $e$ בסיס סטנדרטי ל־$F^2$, ו־$e'$ בסיס סטנדרטי ל־$F^3$. שאלה: $[\phi]_{e'}^{e} = \ ?$. נחשב כל עמודה בנפרד. נשים לב שבעבור הסימונים בהגדרה $B = e = ((1, 0), (0, 1)) := (v_1, v_2), \ C = e' = \cdots$. 
	\begin{gather*}
		[\phi(v_1)]_C = [(1, 1, 1)]_C = (\lg_1, \lg_2, \lg_3) \co \sum\lg_i e'_i = 1 \implies (\lg_1, \lg_2, \lg_3) = (1, 1, 1) \\
		[\phi(v_2)]_C = \phi((0, 1)) = (0, 1, 2) \implies [\phi(v_2)]_{e'} = (0, 1, 2)
	\end{gather*}
	נבחין כי $[v_1, \dots, v_n]_e = (v_1, \dots, v_n)$ בעבור $e$ בסיס סטנדרטי. אך לא כן לבסיסים אחרים. 
	
	\textbf{דוגמה. }בסיס בעבור $F^3$ להיות $C = ((1, 1, 1), (1, 0, 1), (0, 1, 1))$. נמצא את $[\phi]_C^e$. 
	\begin{gather*}
		\phi(v_1) = (1, 1, 1), \ [(1, 1, 1)]_C = (1, 0, 0), \ [\phi(v_2)]_C = [(0, 1, 2)]_C = (-1, 1, 2) \\
		[\phi]_C^e = \begin{pmatrix}
			1 & -1 \\ 0 & 1 \\ 0 & 2
		\end{pmatrix}
	\end{gather*}
	
	\section{\en{vec mul}}
	עבור $v = \sum_{j = 1}^{n}x_jv_j$ עבור $x_j \in F$ עבור $1 \le j \le n$ ($v$ איברי הבסיס). נראה מהו $\phi(v)$ כתלות במטריצה המייצגת. נסמן $[\phi]_C^B = (a_{ij})$ כך ש־$1 \le i \le m, 1 \le j \le n$. יתקיים: 
	\[ \phi(v) = \phi\cl{\sum x_jv_j} = \sum x_j \phi(v_j) = \sum_{j = 1}^{n} \cl{x_j \sum_{i = 1}^{m} u_i a_{ij}} = \sum_{i = 1}^{m}u_i \cl{\sum_{j = 1}^{n}x_ja_{ij}} \]
	
	\textbf{שאלה. }
	\[ [\phi(v)]_C = \cl{\sum_{j = 1}^{n}x_ja_{ij}, \dots, \sum_{j = 1}^{m}x_ja_{mj}} = x_1\cdot C_1 + \cdots + x_n\cdot C_n \]
	
	\textbf{הגדרה. }יהי $F$ שדה. \textit{וקטור עמודה} עם הרכיבים ב־$F$ הוא איבר ב־$M_{n \times 1}(F)$. לפעמים נזהה בינו לבין איבר ב־$F^n$. 
	
	\textbf{הגדרה. }יהיו $A = (a_{ij}) \in M_{m \times n}(F), \ V(x_j)_{j \in [n]}$ וקטור עמודה. אזי: 
	\[ Av := \begin{pmatrix}
		\sum_{j = 1}^n a_{1j}x_j \\ \vdots \\ \sum_{j = 1}^{n}a_{mj}x_j
	\end{pmatrix} = x_1C_1 + \cdots + x_nC_n \in M_{m \times 1} \]
	הסבר לשווין (לא לסימון): 
	\[ \sum_{i = 1}^{n} x_iC_i = \sum x_i \begin{pmatrix}
		a_{1i} \\ \vdots \\ a_{mi}
	\end{pmatrix} = \begin{pmatrix}
		\sum_{j = 1}^n a_{1j}x_j \\ \vdots \\ \sum_{j = 1}^{n}a_{mj}x_j
	\end{pmatrix} \]
	נשים לב שגובה הוקטור הוא ברוחב המטריצה. 
	\textbf{דוגמה. }$Ae_1 = C_1$
	
	\textbf{דוגמה. }
	\[ \begin{pmatrix}
		1 & 0 & 9 & 3 \\ 1 & -2 & 0 & 0 \\ 2 & -2 & 9 & 3
	\end{pmatrix}\begin{pmatrix}
		1 \\ 3\\ 1 \\ -1
	\end{pmatrix} = \begin{pmatrix}
		1 \cdot 1 + 0 \cdot 3 + 9 \cdot 1 + 3 \cdot (-1) \\
		1 \cdot 1 + -2 \cdot 3 + 1 \cdot 0 + 1 \cdot 0 \\
		1 \cdot 2 + -2 \cdot 3 + 9 \cdot 1  + 3 \cdot -1
	\end{pmatrix} = \begin{pmatrix}
		9 \\ -5 \\ 2
	\end{pmatrix} \]
	למעשה זוהי העתקה ליניארית מארבעה ממדים לשלושה ממדים. 
	
	\textbf{משפט. }יהיו $V, U$ מ"וים עם בסיסים $B = (v_1, \dots, v_n), C = (u_1, \dots, u_n)$ בהתאמה. תהי $\phi \co V \to U$ העתקה ליניארית. אזי $\forall v \in V_{[\phi(v)]_C} = [\phi(v)]_C^B \cdot [v]_B$. 
	
	\begin{proof}
		נסמן $v = \sum x_jv_j$. 
		לפי פיתוח אקראי מקודם בשיעור: 
		]\[ \phi(v) = \sum_{i = 1}^m u_i \cdot \cl{\sum_{i = 1}^{m} x_ja_{ij}} \implies [\phi]_C = \sum_{j = 1}^{n}x_jC_j \]
		עבור $C_j$ עמודות של $[\phi]_C^B$. וזוהי בדיוק הגדרת הכפל. 
	\end{proof}
	
	\textbf{טענה. }תהי $A \in M_{m \times n} (F)$. אם נגדיר $\phi_A(v) = Av, \ \phi \co F^n \to F^m$ אז $\phi$ העתקה ליניארית וגם $[\phi_A]$ לפי בסיסים סטדנרטיים של $F^m, F^n$ היא $A$
	
	\begin{proof}
		נגדיר פונ' $\psi$ מ־$F^n$ ל־$F^m$ כך ש־: 
		\[ [\psi]^B_C = A \]
		עבור $B, C$ סטנדרטיים. נראה ש־$\phi = \psi$. ואכן, תהי $v \in F^n$: 
		\[ \begin{WithArrows}
			&\psi(v) \Arrow{כי $c$ סטנדרטי} \\ 
			 =& [\psi(v)]_C \Arrow{מטענה קודמת}\\
			 = &[\psi]^B_C \cdot [v]_B \Arrow{כי $\psi$ מוגדרת כך, וכי $B$ סטנדרטי} \\
			 A \cdot v =& \phi_A(v)
		\end{WithArrows} \]
		קיבלנו $\phi_A = \psi$ בדיוק ולכן $\phi_A$ ליניארית וגם $[\phi_A]_T = [\psi]^B_C = A$ כאשר $T$ סטנדרטי. $B, C$ סטנדרטיים. 
	\end{proof}
	
	\textbf{טענה. }$U, V$ מ"ו מעל שדה $F$ ממימדים $n = \dim V, \ m = \dim U$ ו־$B = (v_1, \dots, v_n), C = (u_1, \dots, u_m)$ $\impliedby$ עבור $(a_{ij}) = A \in M_{m \times n}(F)$ (ממשיך בשורה הבאה)
		נגדיר $\phi \co V \to U$ כך ש־: 
		\[ \phi\cl{\sum_{j = 1}^{n}x_jv_j} = \sum_{\mathclap{(i, j) \in T}} x_ja_{ij}u_i \]
		כאשר: 
		\[ T = \left \{(x, y) \middle\vert \begin{matrix}
			x \in [m] \\ y \in [n]
		\end{matrix}\right \} \]
		
		נקבל ש־$\phi$ העתקה ליניארית וגם $[\phi]_C^B =A$. 
		
		\begin{proof}
			יהי $A \in M_{m \times n}(F) = (a_{ij})$. נגדיר כבטענה ונפשט את $\phi(v)$. עבור $v = \sum_{j = 1}^n$ הוגדר: 
			\[ \phi(v) = \sum_{(i, j) \in T}x_ja_{ij}u_i = \sum^m u_i \cl{\sum_{j = 1}^{n} x_ja_{ij}} \]
			
			והמרצה עשה טעות בהוכחה ונשלים את ההוכחה בשבוע הבא. 
		\end{proof}
	
	בזמן שנותר נוכיח כל מיני דברים על מטריצות. 
	\section{\en{Mat Mul}}
	\textbf{הגדרה. }$(a_{ij}) = A \in M_{m \times n}(F)$ ויהי $(b_{ij}) = B \in M_{m \times n}(F)$
	
	נגדיר חיבור מטריצות: 
	\[ A + B = (a_{ij} + b_{ij})_{i \in [m], j \in [n]} \in M_{m \times n}(F) \]
	ונגדיר כפל בסקלר:
	\[ \lambda A = (\lambda a_{ij})_{\cdots} \in M_{m \times n}(F) \]
	
	\textbf{למה. }יהיו $\phi, \psi \co V \to U$ ו־$B, C$ בסיסים של $U, V$ מ"ו, אז $[\phi + \psi]_{C}^B = [\phi]_C^B + [\psi]_C^B, \ [\lg \phi]_C^B = \lg [\phi]_C^B$
	
	\begin{proof}
		נסמן $T = \phi + \psi$. נחשב $T(v_i)$ ונראה שזה בדיוק $\phi(v_i) + \psi(v_i)$ עבור $v_i$ איבר בבסיס $B$. 
		\[ T(v_i) = (\phi + \psi)(v_i) = \phi(v_i) + \psi(v_i) \]
		כאשר המעברים מהגדרת החיבור. נסמן $L = [\phi]_C^B, R = [\psi]_C^B$. 
		\[  \]
	\end{proof}
	
	בשלב הזה המורה קלט שהכיתה מנותקת לחלוטין ואין לרוב האנשים שום מושג על מה הוא מדבר והוא התחיל לנסות להסביר איך בונים מטריצה מייצגת.
	
\end{document}