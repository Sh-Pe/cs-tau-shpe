%! ~~~ Packages Setup ~~~ 
\documentclass[]{article}
\usepackage{lipsum}
\usepackage{rotating}


% Math packages
\usepackage[usenames]{color}
\usepackage{forest}
\usepackage{ifxetex,ifluatex,amsmath,amssymb,mathrsfs,amsthm,witharrows,mathtools,mathdots}
\WithArrowsOptions{displaystyle}
\renewcommand{\qedsymbol}{$\blacksquare$} % end proofs with \blacksquare. Overwrites the defualts. 
\usepackage{cancel,bm}
\usepackage[thinc]{esdiff}


% tikz
\usepackage{tikz}
\usetikzlibrary{graphs}
\newcommand\sqw{1}
\newcommand\squ[4][1]{\fill[#4] (#2*\sqw,#3*\sqw) rectangle +(#1*\sqw,#1*\sqw);}


% code 
\usepackage{listings}
\usepackage{xcolor}

\definecolor{codegreen}{rgb}{0,0.35,0}
\definecolor{codegray}{rgb}{0.5,0.5,0.5}
\definecolor{codenumber}{rgb}{0.1,0.3,0.5}
\definecolor{codeblue}{rgb}{0,0,0.5}
\definecolor{codered}{rgb}{0.5,0.03,0.02}
\definecolor{codegray}{rgb}{0.96,0.96,0.96}

\lstdefinestyle{pythonstylesheet}{
	language=Java,
	emphstyle=\color{deepred},
	backgroundcolor=\color{codegray},
	keywordstyle=\color{deepblue}\bfseries\itshape,
	numberstyle=\scriptsize\color{codenumber},
	basicstyle=\ttfamily\footnotesize,
	commentstyle=\color{codegreen}\itshape,
	breakatwhitespace=false, 
	breaklines=true, 
	captionpos=b, 
	keepspaces=true, 
	numbers=left, 
	numbersep=5pt, 
	showspaces=false,                
	showstringspaces=false,
	showtabs=false, 
	tabsize=4, 
	morekeywords={as,assert,nonlocal,with,yield,self,True,False,None,AssertionError,ValueError,in,else},              % Add keywords here
	keywordstyle=\color{codeblue},
	emph={var, List, Iterable, Iterator},          % Custom highlighting
	emphstyle=\color{codered},
	stringstyle=\color{codegreen},
	showstringspaces=false,
	abovecaptionskip=0pt,belowcaptionskip =0pt,
	framextopmargin=-\topsep, 
}
\newcommand\pythonstyle{\lstset{pythonstylesheet}}
\newcommand\pyl[1]     {{\lstinline!#1!}}
\lstset{style=pythonstylesheet}

\usepackage[style=1,skipbelow=\topskip,skipabove=\topskip,framemethod=TikZ]{mdframed}
\definecolor{bggray}{rgb}{0.85, 0.85, 0.85}
\mdfsetup{leftmargin=0pt,rightmargin=0pt,innerleftmargin=15pt,backgroundcolor=codegray,middlelinewidth=0.5pt,skipabove=5pt,skipbelow=0pt,middlelinecolor=black,roundcorner=5}
\BeforeBeginEnvironment{lstlisting}{\begin{mdframed}\vspace{-0.4em}}
	\AfterEndEnvironment{lstlisting}{\vspace{-0.8em}\end{mdframed}}


% Deisgn
\usepackage[labelfont=bf]{caption}
\usepackage[margin=0.6in]{geometry}
\usepackage{multicol}
\usepackage[skip=4pt, indent=0pt]{parskip}
\usepackage[normalem]{ulem}
\forestset{default}
\renewcommand\labelitemi{$\bullet$}
\usepackage{titlesec}
\titleformat{\section}[block]
{\fontsize{15}{15}}
{\sen \dotfill (\thesection)\dotfill \she}
{0em}
{\MakeUppercase}
\usepackage{graphicx}
\graphicspath{ {./} }


% Hebrew initialzing
\usepackage[bidi=basic]{babel}
\PassOptionsToPackage{no-math}{fontspec}
\babelprovide[main, import, Alph=letters]{hebrew}
\babelprovide[import]{english}
\babelfont[hebrew]{rm}{David CLM}
\babelfont[hebrew]{sf}{David CLM}
\babelfont[english]{tt}{Monaspace Xenon}
\usepackage[shortlabels]{enumitem}
\newlist{hebenum}{enumerate}{1}

% Language Shortcuts
\newcommand\en[1] {\begin{otherlanguage}{english}#1\end{otherlanguage}}
\newcommand\sen   {\begin{otherlanguage}{english}}
	\newcommand\she   {\end{otherlanguage}}
\newcommand\del   {$ \!\! $}

\newcommand\npage {\vfil {\hfil \textbf{\textit{המשך בעמוד הבא}}} \hfil \vfil \pagebreak}
\newcommand\ndoc  {\dotfill \\ \vfil {\begin{center} {\textbf{\textit{שחר פרץ, 2024}} \\ \scriptsize \textit{נוצר באמצעות תוכנה חופשית בלבד}} \end{center}} \vfil	}

\newcommand{\rn}[1]{
	\textup{\uppercase\expandafter{\romannumeral#1}}
}

\makeatletter
\newcommand{\skipitems}[1]{
	\addtocounter{\@enumctr}{#1}
}
\makeatother

%! ~~~ Math shortcuts ~~~

% Letters shortcuts
\newcommand\N     {\mathbb{N}}
\newcommand\Z     {\mathbb{Z}}
\newcommand\R     {\mathbb{R}}
\newcommand\Q     {\mathbb{Q}}
\newcommand\C     {\mathbb{C}}

\newcommand\ml    {\ell}
\newcommand\mj    {\jmath}
\newcommand\mi    {\imath}

\newcommand\powerset {\mathcal{P}}
\newcommand\ps    {\mathcal{P}}
\newcommand\pc    {\mathcal{P}}
\newcommand\ac    {\mathcal{A}}
\newcommand\bc    {\mathcal{B}}
\newcommand\cc    {\mathcal{C}}
\newcommand\dc    {\mathcal{D}}
\newcommand\ec    {\mathcal{E}}
\newcommand\fc    {\mathcal{F}}
\newcommand\nc    {\mathcal{N}}
\newcommand\vc    {\mathcal{V}} % Vance
\newcommand\sca   {\mathcal{S}} % \sc is already definded
\newcommand\rca   {\mathcal{R}} % \rc is already definded

\newcommand\prm   {\mathrm{p}}
\newcommand\arm   {\mathrm{a}} % x86
\newcommand\brm   {\mathrm{b}}
\newcommand\crm   {\mathrm{c}}
\newcommand\drm   {\mathrm{d}}
\newcommand\erm   {\mathrm{e}}
\newcommand\frm   {\mathrm{f}}
\newcommand\nrm   {\mathrm{n}}
\newcommand\vrm   {\mathrm{v}}
\newcommand\srm   {\mathrm{s}}
\newcommand\rrm   {\mathrm{r}}

\newcommand\Si    {\Sigma}

% Logic & sets shorcuts
\newcommand\siff  {\longleftrightarrow}
\newcommand\ssiff {\leftrightarrow}
\newcommand\so    {\longrightarrow}
\newcommand\sso   {\rightarrow}

\newcommand\epsi  {\epsilon}
\newcommand\vepsi {\varepsilon}
\newcommand\vphi  {\varphi}
\newcommand\Neven {\N_{\mathrm{even}}}
\newcommand\Nodd  {\N_{\mathrm{odd }}}
\newcommand\Zeven {\Z_{\mathrm{even}}}
\newcommand\Zodd  {\Z_{\mathrm{odd }}}
\newcommand\Np    {\N_+}

% Text Shortcuts
\newcommand\open  {\big(}
\newcommand\qopen {\quad\big(}
\newcommand\close {\big)}
\newcommand\also  {\text{\en{, }}}
\newcommand\defi  {\text{\en{ definition}}}
\newcommand\defis {\text{\en{ definitions}}}
\newcommand\given {\text{\en{given }}}
\newcommand\case  {\text{\en{if }}}
\newcommand\syx   {\text{\en{ syntax}}}
\newcommand\rle   {\text{\en{ rule}}}
\newcommand\other {\text{\en{else}}}
\newcommand\set   {\ell et \text{ }}
\newcommand\ans   {\mathscr{A}\!\mathit{nswer}}

% Set theory shortcuts
\newcommand\ra    {\rangle}
\newcommand\la    {\langle}

\newcommand\oto   {\leftarrow}

\newcommand\QED   {\quad\quad\mathscr{Q.E.D.}\;\;\blacksquare}
\newcommand\QEF   {\quad\quad\mathscr{Q.E.F.}}
\newcommand\eQED  {\mathscr{Q.E.D.}\;\;\blacksquare}
\newcommand\eQEF  {\mathscr{Q.E.F.}}
\newcommand\jQED  {\mathscr{Q.E.D.}}

\DeclareMathOperator\dom   {dom}
\DeclareMathOperator\Img   {Im}
\DeclareMathOperator\range {range}
\DeclareMathOperator\col   {Col}

\newcommand\trio  {\triangle}

\newcommand\rc    {\right\rceil}
\newcommand\lc    {\left\lceil}
\newcommand\rf    {\right\rfloor}
\newcommand\lf    {\left\lfloor}

\newcommand\lex   {<_{lex}}

\newcommand\az    {\aleph_0}
\newcommand\uaz   {^{\aleph_0}}
\newcommand\al    {\aleph}
\newcommand\ual   {^\aleph}
\newcommand\taz   {2^{\aleph_0}}
\newcommand\utaz  { ^{\left (2^{\aleph_0} \right )}}
\newcommand\tal   {2^{\aleph}}
\newcommand\utal  { ^{\left (2^{\aleph} \right )}}
\newcommand\ttaz  {2^{\left (2^{\aleph_0}\right )}}

\newcommand\n     {$n$־יה\ }

% Math A&B shortcuts
\newcommand\logn  {\log n}
\newcommand\logx  {\log x}
\newcommand\lnx   {\ln x}
\newcommand\cosx  {\cos x}
\newcommand\cost  {\cos \theta}
\newcommand\sinx  {\sin x}
\newcommand\sint  {\sin \theta}
\newcommand\tanx  {\tan x}
\newcommand\tant  {\tan \theta}
\newcommand\sex   {\sec x}
\newcommand\sect  {\sec^2}
\newcommand\cotx  {\cot x}
\newcommand\cscx  {\csc x}
\newcommand\sinhx {\sinh x}
\newcommand\coshx {\cosh x}
\newcommand\tanhx {\tanh x}

\newcommand\seq   {\overset{!}{=}}
\newcommand\slh   {\overset{LH}{=}}
\newcommand\sle   {\overset{!}{\le}}
\newcommand\sge   {\overset{!}{\ge}}
\newcommand\sll   {\overset{!}{<}}
\newcommand\sgg   {\overset{!}{>}}

\newcommand\h     {\hat}
\newcommand\ve    {\vec}
\newcommand\lv    {\overrightarrow}
\newcommand\ol    {\overline}

\newcommand\mlcm  {\mathrm{lcm}}

\DeclareMathOperator{\sech}   {sech}
\DeclareMathOperator{\csch}   {csch}
\DeclareMathOperator{\arcsec} {arcsec}
\DeclareMathOperator{\arccot} {arcCot}
\DeclareMathOperator{\arccsc} {arcCsc}
\DeclareMathOperator{\arccosh}{arccosh}
\DeclareMathOperator{\arcsinh}{arcsinh}
\DeclareMathOperator{\arctanh}{arctanh}
\DeclareMathOperator{\arcsech}{arcsech}
\DeclareMathOperator{\arccsch}{arccsch}
\DeclareMathOperator{\arccoth}{arccoth}
\DeclareMathOperator{\atant}  {atan2} 
\DeclareMathOperator{\Sp}     {span} 
\DeclareMathOperator{\rk}     {rk}
\DeclareMathOperator{\sgn}    {sgn} 

\newcommand\dx    {\,\mathrm{d}x}
\newcommand\dt    {\,\mathrm{d}t}
\newcommand\dtt   {\,\mathrm{d}\theta}
\newcommand\du    {\,\mathrm{d}u}
\newcommand\dv    {\,\mathrm{d}v}
\newcommand\df    {\mathrm{d}f}
\newcommand\dfdx  {\diff{f}{x}}
\newcommand\dit   {\limhz \frac{f(x + h) - f(x)}{h}}

\newcommand\nt[1] {\frac{#1}{#1}}

\newcommand\limz  {\lim_{x \to 0}}
\newcommand\limxz {\lim_{x \to x_0}}
\newcommand\limi  {\lim_{x \to \infty}}
\newcommand\limh  {\lim_{x \to 0}}
\newcommand\limni {\lim_{x \to - \infty}}
\newcommand\limpmi{\lim_{x \to \pm \infty}}

\newcommand\ta    {\theta}
\newcommand\ap    {\alpha}

\renewcommand\inf {\infty}
\newcommand  \ninf{-\inf}

% Combinatorics shortcuts
\newcommand\sumnk     {\sum_{k = 0}^{n}}
\newcommand\sumni     {\sum_{i = 0}^{n}}
\newcommand\sumnko    {\sum_{k = 1}^{n}}
\newcommand\sumnio    {\sum_{i = 1}^{n}}
\newcommand\sumai     {\sum_{i = 1}^{n} A_i}
\newcommand\nsum[2]   {\reflectbox{\displaystyle\sum_{\reflectbox{\scriptsize$#1$}}^{\reflectbox{\scriptsize$#2$}}}}

\newcommand\bink      {\binom{n}{k}}
\newcommand\setn      {\{a_i\}^{2n}_{i = 1}}
\newcommand\setc[1]   {\{a_i\}^{#1}_{i = 1}}

\newcommand\cupain    {\bigcup_{i = 1}^{n} A_i}
\newcommand\cupai[1]  {\bigcup_{i = 1}^{#1} A_i}
\newcommand\cupiiai   {\bigcup_{i \in I} A_i}
\newcommand\capain    {\bigcap_{i = 1}^{n} A_i}
\newcommand\capai[1]  {\bigcap_{i = 1}^{#1} A_i}
\newcommand\capiiai   {\bigcap_{i \in I} A_i}

\newcommand\xot       {x_{1, 2}}
\newcommand\ano       {a_{n - 1}}
\newcommand\ant       {a_{n - 2}}

% Linear Algebra
\DeclareMathOperator{\chr}    {char}

\newcommand\lra       {\leftrightarrow}
\newcommand\chrf      {\chr(\F)}
\newcommand\F         {\mathbb{F}}
\newcommand\co        {\colon}
\newcommand\tmat[2]   {\cl{\begin{matrix}
			#1
		\end{matrix}\, \middle\vert\, \begin{matrix}
			#2
\end{matrix}}}

\makeatletter
\newcommand\rrr[1]    {\xxrightarrow{1}{#1}}
\newcommand\rrt[2]    {\xxrightarrow{1}[#2]{#1}}
\newcommand\mat[2]    {M_{#1\times#2}}
\newcommand\tomat     {\, \dequad \longrightarrow}
\newcommand\pms[1]    {\begin{pmatrix}
		#1
\end{pmatrix}}

% someone's code from the internet: https://tex.stackexchange.com/questions/27545/custom-length-arrows-text-over-and-under
\makeatletter
\newlength\min@xx
\newcommand*\xxrightarrow[1]{\begingroup
	\settowidth\min@xx{$\m@th\scriptstyle#1$}
	\@xxrightarrow}
\newcommand*\@xxrightarrow[2][]{
	\sbox8{$\m@th\scriptstyle#1$}  % subscript
	\ifdim\wd8>\min@xx \min@xx=\wd8 \fi
	\sbox8{$\m@th\scriptstyle#2$} % superscript
	\ifdim\wd8>\min@xx \min@xx=\wd8 \fi
	\xrightarrow[{\mathmakebox[\min@xx]{\scriptstyle#1}}]
	{\mathmakebox[\min@xx]{\scriptstyle#2}}
	\endgroup}
\makeatother


% Greek Letters
\newcommand\ag        {\alpha}
\newcommand\bg        {\beta}
\newcommand\cg        {\gamma}
\newcommand\dg        {\delta}
\newcommand\eg        {\epsi}
\newcommand\zg        {\zeta}
\newcommand\hg        {\eta}
\newcommand\tg        {\theta}
\newcommand\ig        {\iota}
\newcommand\kg        {\keppa}
\renewcommand\lg      {\lambda}
\newcommand\og        {\omicron}
\newcommand\rg        {\rho}
\newcommand\sg        {\sigma}
\newcommand\yg        {\usilon}
\newcommand\wg        {\omega}

\newcommand\Ag        {\Alpha}
\newcommand\Bg        {\Beta}
\newcommand\Cg        {\Gamma}
\newcommand\Dg        {\Delta}
\newcommand\Eg        {\Epsi}
\newcommand\Zg        {\Zeta}
\newcommand\Hg        {\Eta}
\newcommand\Tg        {\Theta}
\newcommand\Ig        {\Iota}
\newcommand\Kg        {\Keppa}
\newcommand\Lg        {\Lambda}
\newcommand\Og        {\Omicron}
\newcommand\Rg        {\Rho}
\newcommand\Sg        {\Sigma}
\newcommand\Yg        {\Usilon}
\newcommand\Wg        {\Omega}

% Other shortcuts
\newcommand\tl    {\tilde}
\newcommand\op    {^{-1}}

\newcommand\sof[1]    {\left | #1 \right |}
\newcommand\cl [1]    {\left ( #1 \right )}
\newcommand\csb[1]    {\left [ #1 \right ]}
\newcommand\ccb[1]    {\left \{ #1 \right \}}

\newcommand\bs        {\blacksquare}
\newcommand\dequad    {\!\!\!\!\!\!}
\newcommand\dequadd   {\dequad\duquad}
\newcommand\wmid      {\;\middle\vert\;}

\renewcommand\phi     {\varphi}
\newcommand\bcl[1]    {\big(#1\big)}

%! ~~~ Document ~~~

\author{שחר פרץ}
\title{\textit{ליניארית 1א $\sim$ תרגיל בית 7}}
\date{9 בינואר 2025}
\begin{document}
	\maketitle
	
	\section{}
	תהא $A$ מטריצה ריבועית, נניח $(A + 2I)^2 = 0$. צ.ל. $A + \lg I$ הפיכה אמ"מ $\lg \neq 2$.
	\begin{proof}\,
		\begin{enumerate}
			\item[$\implies$]נניח $\lg \neq 2$. אז נתון $(A + 2I)^2 = 0$, נסמן $B = A + 2I$ ונוכיח ש־$B$ לא הפיכה. נתבונן ב־$\phi = [B]^E_E$ כאשר $E$ הוא הבסיס הסטנדטי. וידוע $B^2 = 0$ ולכן $\phi \circ \phi = 0$. סה"כ: 
			\[ \forall v \in V \co (\phi \circ \phi)(v) = 0 \implies \phi(\phi(v)) = 0 \implies \phi(v) \in \ker \phi \]
			באופן שקול, $\Img\phi = \ker \phi$ משוויון קבוצות. נפלג למקרים: 
			\begin{enumerate}
				\item אם $\Img\phi = 0$ אז $\phi$ לא על (אלא אם המ"ו ממימד $0$ ואני מאוד מקווה שהוא לא) ובפרט לא איזומורפיזם. 
				\item אחרת $\Img\phi > 0$ כלומר $\ker\phi > 0$, ולכן $\ker \phi \neq 0$ באופן שקול $\phi$ לא חח"ע, ולכן $\phi$ איננה איזומורפיזם. 
			\end{enumerate}
			סה"כ $\phi$ אינה איזומורפיזם, ואם $B$ הייתה הפיכה אז היא מייצגת איזומורפיזמים ו־$\phi$ איזמורפיזם, סתירה. 
			\item[$\impliedby$]יהי $\lg \in \F \setminus \{2\}$ (כאשר $2 = 1_\F + 1_\F$, ו־$2 \neq 0_\F$ בהנחה שהשדה מגודל $3$ או יותר, הנחה שהמתבצעת בשאלה בעת השימוש ב־$2$). 
			\[ (A + \lg I)\frac{(A + 4I - \lg I)}{-(\lg - 2)^2} = \frac{A^2 + 4IA - \lg IA + \lg I A+ 4I\lg - \lg^2 I}{-(\lg - 2)^2} \overset{BI = B}{=} 
			\frac{\overbrace{A^2 + 4A + 4}^{=0} + \lg 4I - 4I - \lg^2 I}{-(\lg - 2)^2} = I\frac{-(\lg - 2)^2}{-(\lg - 2)^2} = I \]
			כאשר השוויון המצוין ל־$0$ מתקיים בגלל ש־: 
			\[ 0 = (A + 2I)^2 = A^2 + A2I + 2IA + 4I^2 = A^2 + 4A + 4I \]
			והחילוק מוגדר כי: 
			\[ -(\lg - 2)^2 \neq 0 \overset{(-1)}{=} (\lg - 2)^2 \neq 0 \overset{\sqrt{\, }}{=} \lg - 2 \neq 0 \overset{+2}{=} \lg \neq 2 \]
			שנתון. סה"כ מצאנו הופכית כדרוש. 
		\end{enumerate} 
	\end{proof}
	
	\section{}
	תהי $A \in M_n(\F)$ מטריצה הפיכה. צ.ל. קיו ם$p \in \F_{n^2}[x]$ כך ש־$A\op = p(A)$, כאשר $p(A) = \sum_{i = 0}^{m}a_iA^i$. 
	\begin{proof}
		תהי $A \in M_n(\F)$ מטריצה הפיכה. נבחר $\F_{n^2 - 1}(\F) \ni p = \lg_0x^0 + \cdots + \lg_{n^2}x^{n^2}$. 
		נתבונן ב־$p(A)$: 
		\[ p(A) = \lg_0I + \lg_1A + \cdots + \lg_{n^2 - 1}A^{n^2} \]
		נבחין שיש כאן חיבור של $n^2 + 1$ וקטורים, בעבור מ"ו ממימד $n^2$ ולכן, ממשפט, ת"ל, ולכן קיים צירוף ליניארי לא טרוויאלי בעבורו יתקיים $p(A) = 0$. נסמן את $\lg_i$ להיות האיבר שאינו אפס בעבור $i$ מינימלי, שקיים כי אם לא הצירוף הליניארי היה טרוויאלי. אז $\forall \lg_j < \lg_i \co \lg_j = 0$. לכן: 
		\[ \begin{WithArrows}
			0 &= \sum_{j = 0}^{n^2}\lg_jA^{j} \\
			&=\underbrace{\sum_{j = 0}^{i - 1}\lg_jA^{j}}_{=0} + \sum_{j = i}^{\mathclap{n^2}}\lg_jA^{j} \Arrow{$\cdot -(\frac{1}{\lg_i})$}\\
			&=-A_i - \sum_{j = i + 1}^{\mathclap{n^2}}\frac{\lg_j}{\lg_i}A^{j} \Arrow{$(\cdot A^i)\op$}\\
			&=-I + \sum_{j = i + 1}^{n^2}-\frac{\lg_j}{\lg_i}A^{j - i}
		\end{WithArrows} \]
		נחבר $I$ לשני האגפים ונקבל גרירה לכך שאגף ימין הוא הזהות. מכיוון ש־$j - i \in [n]$, סה"כ מצאנו פולינום ממעלה לכל היותר $n^2$ (ובפרט ממעלה $n^2$ בעבור מקדמים טרוויאלים לאחר מעלה לא $0$ מקסימלית), נסמנו ב־$p(x)$, שמקיים $p(A) = I$, כדרוש.		
	\end{proof}
	
	\section{}
	תהי $A$ מטריצה כך ש־$A^m = 0$. נוכיח $I + A$ ו־$I - A$ הפיכות. 
	
	\begin{proof}
		נתון קיום $m \in \N$ כך ש־$A^m = 0$. אז: 
		\begin{alignat*}{9}
			(I + A)\overbrace{\cl{\sum_{i = 1}^{m - 1}{(-1)^{i}A^{i}}}}^{\tl B} &= &&\sum_{i = 0}^{m - 1}\bcl{I(-1)^iA^{i}} + &&\sum_{i = 0}^{m - 1}\bcl{(-1)^iA^{i + 1}} &&= \sum_{i = 0}^{m - 1}\bcl{(-1)^{i}A^i} + \sum_{i = 1}^{m}\bcl{(-1)^{i - 1}A^{i}} \\
			 &= I \,+ &&\sum_{i = 1}^{m - 1}\bcl{(-1)^{i}A^i} + &&\sum_{i = 1}^{m - 1}\bcl{(-1)^{i - 1}A^{i}} + A^m
			 &&= I + A^m + \sum_{i = 1}^{m - 1}(-1)^{i}\underbrace{A^{i} + (-1)^{i - 1}A^{i}}_{=0} \\
			 &= I\,  + &&\, A^{m} = I
		\end{alignat*}
		\begin{alignat*}{9}
			(I - A)\overbrace{\cl{\sum_{i = 1}^{m - 1}{A^{i}}}}^{=: \bar B} &= &&\sum_{i = 0}^{m - 1}\bcl{IA^{i}} + &&\sum_{i = 0}^{m - 1}\bcl{-A^{i + 1}} &&= \sum_{i = 0}^{m - 1}\bcl{A^i} + \sum_{i = 1}^{m}\bcl{-A^{i}} \\
			&= I \,+ &&\sum_{i = 1}^{m - 1}\bcl{A^i} + &&\sum_{i = 1}^{m - 1}\bcl{-A^{i}} + A^m
			&&= I + A^m + \sum_{i = 1}^{m - 1}(-1)^{i}\underbrace{A^{i} -A^{i}}_{=0} \\
			&= I\,  + &&\, A^{m} = I
		\end{alignat*}
		וסה"כ $(I - A)\bar B = I = (I + A) \tl B$, ולכן הן הפיכות מימין, וממשפט הפיכות. 
	\end{proof}
	
	\section{}
	יהיו $A, B, C \in M_n(\F)$ מטריצות. נניח $AB$ הפיכה ו־$BC$ לא הפיכה. 
	\begin{enumerate}
		\item נוכיח $AC + BC$ לא הפיכה. 
		\begin{proof}
			ממשפט, $\rk(AC + BC) \le \rk BC$. ידוע $\rk BC = n$ אמ"מ $BC$ הפיכה אך היא איננה הפיכה, וגם ידוע $\rk BC  \le n$, וסה"כ $\rk BC < n$, ומטרנזיטיביות $\rk (AC + BC) < n$ ובפרט אי־שוויון ל־$n$, כלומר $AC + BC$ איננה הפיכה. 
		\end{proof}
		\item נוכיח שלא בהכרח $A + B$ הפיכה. 
		\begin{proof}
			נניח בשלילה לכל $A, B, C$ בתנאים לעיל, $A + B$ הפיכה. בפרט, בעבור $A = I, B = -I$ שתיהן הפיכות וכפל הפיכות הוא הופכי, כלומר $AB$ הפיכה, יתקיים $A + B = I - I = 0$ שהיא העתקה בעבורה $n \neq 0 = \Img 0 = \Img(A + B)$, לכן $A + B$ איננה הפיכה בסתירה לטענה. 
		\end{proof}
	\end{enumerate}
	
	\section{}
	hvhu $A, B \in M_n(\F)$ מטריצות, ונניח $A = I + AB$. אז: 
	\begin{enumerate}[A.]
		\item נוכיח $A$ הפיכה. 
		\begin{proof}
			מהנתון: 
			\[ A = I + AB \overset{\mathclap{-AB}}{\implies} A - AB = I \implies A(I - B) = I \]
				וסה"כ $A$ הפיכה מימין ע"י $I - B$ ולכן $A$ הפיכה. 
		\end{proof}
		\item נוכיח $A, B$ מתחלפות. 
		\begin{proof}
			\[ A(I - B) = I \overset{(*)}{\iff} (I - B)A = I \implies A - BA = I \implies A - AB = I = A - BA \overset{-A, \cdot (-1)}{\iff} BA = AB \quad \top \]
		\end{proof}
		
		\item נניח $A$ סימטרית, כלומר $A = A^T$. אז:
		\begin{multline*}
			A(I - B) = I \rrr{\text{\en{transpose}}} (I - B)^{T}A^{T} = I^{T} = I, A = A^{T} \overset{\cdot A\op}{\iff} I = A^{T}\underbrace{(I - B)}_{A\op} \overset{(*)}{\iff} (I - B)^T = (I - B) \\
			\iff I - B^T = I^T - B^T = I - B \overset{-I}{\iff} -B^T = -B \overset{\cdot (-1)}\iff B = B^{T}
		\end{multline*}
		ולכן $B$ סימטרית. $(*)$ נכון מיחידות ההופכית ל־$A^T$, ומטרניזטיביות. משקילות בפרט $B$ סימטרית גורר $A$ סימטרית, כדרוש. 
		\item 
		\[ 1 + B + B^2 = A \overset{1 + B + B^2}{\iff} (1 - B)(1 + B + B^2) = A(I - B) \iff 1 - B + B + B^2 - B^2 + B^3 = I \iff I + B^3 = I \iff B^3 = 0 \]
		(שקילות כי כפל במטריצה הופכית, והיא מטריצה הופכית כי הראינו בפרט שההופכית שלה היא $(I -B)$, וזה אינו טיעון מעגלי כי הגרירה ימינה נכונה גם כאשר אין היא הופכית)
	\end{enumerate}
	
	\section{}
	המטריצות האלמנטריות ב־$M_2(\Z_3)$: 
	\begin{gather*}
		\pms{1 & 1 \\ 0 & 1}, \ \pms{1 & 2 \\ 0 & 1}, \ \pms{1 & 0 \\ 1 & 1}, \ \pms{1 & 0 \\ 2 & 1}, \ \pms{0 & 1 \\ 1 & 0}
	\end{gather*}
	\section{}
	\begin{enumerate}[A.]
		\item נרצה למצוא $P$ הפיכה כך ש־: 
		\[ \underbrace{\pms{0 & 2 & 0 & 1 \\ 1 & 0 & 0 & 0 \\ 0 & 1 & 1 & 1}}_{:= B} = P \underbrace{\pms{1 & 2 & 0 & 1 \\ 3 & 2 & 0 & 1 \\ 1 & 1 & 1 & 1}}_{:= A} \]
		נדרג
		\begin{multline*}
			(A \mid I) = 
			\tmat{1 & 2 & 0 & 1 \\ 3 & 2 & 0 & 1 \\ 1 & 1 & 1 & 1}{1 & 0 & 0 \\ 0 & 1 & 0 \\ 0 & 0 & 1}
			\rrr{R_2 \to R_2 - R_1}
			\tmat{1 & 2 & 0 & 1 \\ 2 & 0 & 0 & 0 \\ 1 & 1 & 1 & 1}{1 & 0 & 0 \\ -1 & 1 & 0 \\ 0 & 0 & 1}
			\rrr{R_2 \to 0.5R_2}
			\tmat{1 & 2 & 0 & 1 \\ 1 & 0 & 0 & 0 \\ 1 & 1 & 1 & 1}{1 & 0 & 0 \\ -0.5 & 0.5 & 0 \\ 0 & 0 & 1} \\
			\rrt{R_1 \to R_1 - R_2}{R_3 \to R_3 - R_2}
			\tmat{0 & 2 & 0 & 1 \\ 1 & 0 & 0 & 0 \\ 0 & 1 & 1 & 1}{1.5 & -0.5 & 0 \\ -0.5 & 0.5 & 0 \\ 0.5 & -0.5 & 1} \overset{(*)}{=} (B \mid P)
		\end{multline*}
		$(*)$ מתקיים ממשפטים מההרצאה. $P$ אכן הפיכה כי היא הרכבה של מטריצות אלמנטריות. 
		\item נמצא מטריצה הפיכה $Q$ עבורה: 
		\[ \underbrace{\pms{1 & 3 & -1 \\ 2 & 3 & -1}}_{:= B} = \underbrace{\pms{1 & 2 & 1 \\ 2 & 1 & 1}}_{:= A} Q \]
		ידוע $AQ = B \iff Q^TA^T = B^T$. לכן, באופן שקול, יתקיים: 
		\[ \pms{1 & 2 \\ 3 & 3 \\ -1 & -1} = Q^T \pms{1 & 2 \\ 2 & 1 \\ 1 & 1} \]
		נדרג כדי למצוא סדרה של פעולות אלמנטריות שתביא אותנו למטריצה הדרושה, כמו בסעיף הקודם: 
		\[ (A^T \mid I) = \tmat{1 & 2 \\ 2 & 1 \\ 1 & 1}{1 & 0 & 0 \\ 0 & 1 & 0 \\ 0 & 0 & 1} \rrr{R_2 \to R_2 + R_1}
		\tmat{1 & 2 \\ 3 & 3 \\ 1 & 1}{1 & 0 & 0 \\ 1 & 1 & 0 \\ 0 & 0 & 1} \rrr{R_3 \to R_3 - \frac{2}{3}R_2} \tmat{1 & 2  \\ 3 & 3 \\ -1 & -1}{1 & 0 & 0 \\ 1 & 1 & 0 \\ -\frac{2}{3} & -\frac{2}{3} & 1} \overset{(*)}{=} (B^T \mid Q^T) \]
		$(*)$ מתקיים ממשפטים מההרצאה. $Q^T$ אכן הפיכה כי היא הרכבה של פעולות אלמנטריות, וכן $Q$ הפיכה כי $Q^T$ הפיכה (ע"פ משפט). סה"כ: 
		\[ Q = \pms{1 & 1 & -\frac{2}{3} \\ 0 & 1 & -\frac{2}{3} \\ 0 & 0 & 1} \]
	\end{enumerate}
	
	\ndoc
	
\end{document}