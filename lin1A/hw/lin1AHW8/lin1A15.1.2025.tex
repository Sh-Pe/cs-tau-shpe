%! ~~~ Packages Setup ~~~ 
\documentclass[]{article}
\usepackage{lipsum}
\usepackage{rotating}


% Math packages
\usepackage[usenames]{color}
\usepackage{forest}
\usepackage{ifxetex,ifluatex,amsmath,amssymb,mathrsfs,amsthm,witharrows,mathtools,mathdots}
\WithArrowsOptions{displaystyle}
\renewcommand{\qedsymbol}{$\blacksquare$} % end proofs with \blacksquare. Overwrites the defualts. 
\usepackage{cancel,bm}
\usepackage[thinc]{esdiff}


% tikz
\usepackage{tikz}
\usetikzlibrary{graphs}
\newcommand\sqw{1}
\newcommand\squ[4][1]{\fill[#4] (#2*\sqw,#3*\sqw) rectangle +(#1*\sqw,#1*\sqw);}


% code 
\usepackage{listings}
\usepackage{xcolor}

\definecolor{codegreen}{rgb}{0,0.35,0}
\definecolor{codegray}{rgb}{0.5,0.5,0.5}
\definecolor{codenumber}{rgb}{0.1,0.3,0.5}
\definecolor{codeblue}{rgb}{0,0,0.5}
\definecolor{codered}{rgb}{0.5,0.03,0.02}
\definecolor{codegray}{rgb}{0.96,0.96,0.96}

\lstdefinestyle{pythonstylesheet}{
	language=Java,
	emphstyle=\color{deepred},
	backgroundcolor=\color{codegray},
	keywordstyle=\color{deepblue}\bfseries\itshape,
	numberstyle=\scriptsize\color{codenumber},
	basicstyle=\ttfamily\footnotesize,
	commentstyle=\color{codegreen}\itshape,
	breakatwhitespace=false, 
	breaklines=true, 
	captionpos=b, 
	keepspaces=true, 
	numbers=left, 
	numbersep=5pt, 
	showspaces=false,                
	showstringspaces=false,
	showtabs=false, 
	tabsize=4, 
	morekeywords={as,assert,nonlocal,with,yield,self,True,False,None,AssertionError,ValueError,in,else},              % Add keywords here
	keywordstyle=\color{codeblue},
	emph={var, List, Iterable, Iterator},          % Custom highlighting
	emphstyle=\color{codered},
	stringstyle=\color{codegreen},
	showstringspaces=false,
	abovecaptionskip=0pt,belowcaptionskip =0pt,
	framextopmargin=-\topsep, 
}
\newcommand\pythonstyle{\lstset{pythonstylesheet}}
\newcommand\pyl[1]     {{\lstinline!#1!}}
\lstset{style=pythonstylesheet}

\usepackage[style=1,skipbelow=\topskip,skipabove=\topskip,framemethod=TikZ]{mdframed}
\definecolor{bggray}{rgb}{0.85, 0.85, 0.85}
\mdfsetup{leftmargin=0pt,rightmargin=0pt,innerleftmargin=15pt,backgroundcolor=codegray,middlelinewidth=0.5pt,skipabove=5pt,skipbelow=0pt,middlelinecolor=black,roundcorner=5}
\BeforeBeginEnvironment{lstlisting}{\begin{mdframed}\vspace{-0.4em}}
	\AfterEndEnvironment{lstlisting}{\vspace{-0.8em}\end{mdframed}}


% Deisgn
\usepackage[labelfont=bf]{caption}
\usepackage[margin=0.6in]{geometry}
\usepackage{multicol}
\usepackage[skip=4pt, indent=0pt]{parskip}
\usepackage[normalem]{ulem}
\forestset{default}
\renewcommand\labelitemi{$\bullet$}
\usepackage{titlesec}
\titleformat{\section}[block]
{\fontsize{15}{15}}
{\sen \dotfill (\thesection)\dotfill \she}
{0em}
{\MakeUppercase}
\usepackage{graphicx}
\graphicspath{ {./} }


% Hebrew initialzing
\usepackage[bidi=basic]{babel}
\PassOptionsToPackage{no-math}{fontspec}
\babelprovide[main, import, Alph=letters]{hebrew}
\babelprovide[import]{english}
\babelfont[hebrew]{rm}{David CLM}
\babelfont[hebrew]{sf}{David CLM}
\babelfont[english]{tt}{Monaspace Xenon}
\usepackage[shortlabels]{enumitem}
\newlist{hebenum}{enumerate}{1}

% Language Shortcuts
\newcommand\en[1] {\begin{otherlanguage}{english}#1\end{otherlanguage}}
\newcommand\sen   {\begin{otherlanguage}{english}}
	\newcommand\she   {\end{otherlanguage}}
\newcommand\del   {$ \!\! $}

\newcommand\npage {\vfil {\hfil \textbf{\textit{המשך בעמוד הבא}}} \hfil \vfil \pagebreak}
\newcommand\ndoc  {\dotfill \\ \vfil {\begin{center} {\textbf{\textit{שחר פרץ, 2025}} \\ \scriptsize \textit{נוצר באמצעות תוכנה חופשית בלבד}} \end{center}} \vfil	}

\newcommand{\rn}[1]{
	\textup{\uppercase\expandafter{\romannumeral#1}}
}

\makeatletter
\newcommand{\skipitems}[1]{
	\addtocounter{\@enumctr}{#1}
}
\makeatother

%! ~~~ Math shortcuts ~~~

% Letters shortcuts
\newcommand\N     {\mathbb{N}}
\newcommand\Z     {\mathbb{Z}}
\newcommand\R     {\mathbb{R}}
\newcommand\Q     {\mathbb{Q}}
\newcommand\C     {\mathbb{C}}

\newcommand\ml    {\ell}
\newcommand\mj    {\jmath}
\newcommand\mi    {\imath}

\newcommand\powerset {\mathcal{P}}
\newcommand\ps    {\mathcal{P}}
\newcommand\pc    {\mathcal{P}}
\newcommand\ac    {\mathcal{A}}
\newcommand\bc    {\mathcal{B}}
\newcommand\cc    {\mathcal{C}}
\newcommand\dc    {\mathcal{D}}
\newcommand\ec    {\mathcal{E}}
\newcommand\fc    {\mathcal{F}}
\newcommand\nc    {\mathcal{N}}
\newcommand\vc    {\mathcal{V}} % Vance
\newcommand\sca   {\mathcal{S}} % \sc is already definded
\newcommand\rca   {\mathcal{R}} % \rc is already definded

\newcommand\prm   {\mathrm{p}}
\newcommand\arm   {\mathrm{a}} % x86
\newcommand\brm   {\mathrm{b}}
\newcommand\crm   {\mathrm{c}}
\newcommand\drm   {\mathrm{d}}
\newcommand\erm   {\mathrm{e}}
\newcommand\frm   {\mathrm{f}}
\newcommand\nrm   {\mathrm{n}}
\newcommand\vrm   {\mathrm{v}}
\newcommand\srm   {\mathrm{s}}
\newcommand\rrm   {\mathrm{r}}

\newcommand\Si    {\Sigma}

% Logic & sets shorcuts
\newcommand\siff  {\longleftrightarrow}
\newcommand\ssiff {\leftrightarrow}
\newcommand\so    {\longrightarrow}
\newcommand\sso   {\rightarrow}

\newcommand\epsi  {\epsilon}
\newcommand\vepsi {\varepsilon}
\newcommand\vphi  {\varphi}
\newcommand\Neven {\N_{\mathrm{even}}}
\newcommand\Nodd  {\N_{\mathrm{odd }}}
\newcommand\Zeven {\Z_{\mathrm{even}}}
\newcommand\Zodd  {\Z_{\mathrm{odd }}}
\newcommand\Np    {\N_+}

% Text Shortcuts
\newcommand\open  {\big(}
\newcommand\qopen {\quad\big(}
\newcommand\close {\big)}
\newcommand\also  {\text{\en{, }}}
\newcommand\defi  {\text{\en{ definition}}}
\newcommand\defis {\text{\en{ definitions}}}
\newcommand\given {\text{\en{given }}}
\newcommand\case  {\text{\en{if }}}
\newcommand\syx   {\text{\en{ syntax}}}
\newcommand\rle   {\text{\en{ rule}}}
\newcommand\other {\text{\en{else}}}
\newcommand\set   {\ell et \text{ }}
\newcommand\ans   {\mathscr{A}\!\mathit{nswer}}

% Set theory shortcuts
\newcommand\ra    {\rangle}
\newcommand\la    {\langle}

\newcommand\oto   {\leftarrow}

\newcommand\QED   {\quad\quad\mathscr{Q.E.D.}\;\;\blacksquare}
\newcommand\QEF   {\quad\quad\mathscr{Q.E.F.}}
\newcommand\eQED  {\mathscr{Q.E.D.}\;\;\blacksquare}
\newcommand\eQEF  {\mathscr{Q.E.F.}}
\newcommand\jQED  {\mathscr{Q.E.D.}}

\DeclareMathOperator\dom   {dom}
\DeclareMathOperator\Img   {Im}
\DeclareMathOperator\range {range}
\DeclareMathOperator\col   {Col}

\newcommand\trio  {\triangle}

\newcommand\rc    {\right\rceil}
\newcommand\lc    {\left\lceil}
\newcommand\rf    {\right\rfloor}
\newcommand\lf    {\left\lfloor}

\newcommand\lex   {<_{lex}}

\newcommand\az    {\aleph_0}
\newcommand\uaz   {^{\aleph_0}}
\newcommand\al    {\aleph}
\newcommand\ual   {^\aleph}
\newcommand\taz   {2^{\aleph_0}}
\newcommand\utaz  { ^{\left (2^{\aleph_0} \right )}}
\newcommand\tal   {2^{\aleph}}
\newcommand\utal  { ^{\left (2^{\aleph} \right )}}
\newcommand\ttaz  {2^{\left (2^{\aleph_0}\right )}}

\newcommand\n     {$n$־יה\ }

% Math A&B shortcuts
\newcommand\logn  {\log n}
\newcommand\logx  {\log x}
\newcommand\lnx   {\ln x}
\newcommand\cosx  {\cos x}
\newcommand\cost  {\cos \theta}
\newcommand\sinx  {\sin x}
\newcommand\sint  {\sin \theta}
\newcommand\tanx  {\tan x}
\newcommand\tant  {\tan \theta}
\newcommand\sex   {\sec x}
\newcommand\sect  {\sec^2}
\newcommand\cotx  {\cot x}
\newcommand\cscx  {\csc x}
\newcommand\sinhx {\sinh x}
\newcommand\coshx {\cosh x}
\newcommand\tanhx {\tanh x}

\newcommand\seq   {\overset{!}{=}}
\newcommand\slh   {\overset{LH}{=}}
\newcommand\sle   {\overset{!}{\le}}
\newcommand\sge   {\overset{!}{\ge}}
\newcommand\sll   {\overset{!}{<}}
\newcommand\sgg   {\overset{!}{>}}

\newcommand\h     {\hat}
\newcommand\ve    {\vec}
\newcommand\lv    {\overrightarrow}
\newcommand\ol    {\overline}

\newcommand\mlcm  {\mathrm{lcm}}

\DeclareMathOperator{\sech}   {sech}
\DeclareMathOperator{\csch}   {csch}
\DeclareMathOperator{\arcsec} {arcsec}
\DeclareMathOperator{\arccot} {arcCot}
\DeclareMathOperator{\arccsc} {arcCsc}
\DeclareMathOperator{\arccosh}{arccosh}
\DeclareMathOperator{\arcsinh}{arcsinh}
\DeclareMathOperator{\arctanh}{arctanh}
\DeclareMathOperator{\arcsech}{arcsech}
\DeclareMathOperator{\arccsch}{arccsch}
\DeclareMathOperator{\arccoth}{arccoth}
\DeclareMathOperator{\atant}  {atan2} 
\DeclareMathOperator{\sgn}    {sgn} 

\newcommand\dx    {\,\mathrm{d}x}
\newcommand\dt    {\,\mathrm{d}t}
\newcommand\dtt   {\,\mathrm{d}\theta}
\newcommand\du    {\,\mathrm{d}u}
\newcommand\dv    {\,\mathrm{d}v}
\newcommand\df    {\mathrm{d}f}
\newcommand\dfdx  {\diff{f}{x}}
\newcommand\dit   {\limhz \frac{f(x + h) - f(x)}{h}}

\newcommand\nt[1] {\frac{#1}{#1}}

\newcommand\limz  {\lim_{x \to 0}}
\newcommand\limxz {\lim_{x \to x_0}}
\newcommand\limi  {\lim_{x \to \infty}}
\newcommand\limh  {\lim_{x \to 0}}
\newcommand\limni {\lim_{x \to - \infty}}
\newcommand\limpmi{\lim_{x \to \pm \infty}}

\newcommand\ta    {\theta}
\newcommand\ap    {\alpha}

\renewcommand\inf {\infty}
\newcommand  \ninf{-\inf}

% Combinatorics shortcuts
\newcommand\sumnk     {\sum_{k = 0}^{n}}
\newcommand\sumni     {\sum_{i = 0}^{n}}
\newcommand\sumnko    {\sum_{k = 1}^{n}}
\newcommand\sumnio    {\sum_{i = 1}^{n}}
\newcommand\sumai     {\sum_{i = 1}^{n} A_i}
\newcommand\nsum[2]   {\reflectbox{\displaystyle\sum_{\reflectbox{\scriptsize$#1$}}^{\reflectbox{\scriptsize$#2$}}}}

\newcommand\bink      {\binom{n}{k}}
\newcommand\setn      {\{a_i\}^{2n}_{i = 1}}
\newcommand\setc[1]   {\{a_i\}^{#1}_{i = 1}}

\newcommand\cupain    {\bigcup_{i = 1}^{n} A_i}
\newcommand\cupai[1]  {\bigcup_{i = 1}^{#1} A_i}
\newcommand\cupiiai   {\bigcup_{i \in I} A_i}
\newcommand\capain    {\bigcap_{i = 1}^{n} A_i}
\newcommand\capai[1]  {\bigcap_{i = 1}^{#1} A_i}
\newcommand\capiiai   {\bigcap_{i \in I} A_i}

\newcommand\xot       {x_{1, 2}}
\newcommand\ano       {a_{n - 1}}
\newcommand\ant       {a_{n - 2}}

% Linear Algebra
\DeclareMathOperator{\chr}    {char}

\newcommand\lra       {\leftrightarrow}
\newcommand\chrf      {\chr(\F)}
\newcommand\F         {\mathbb{F}}
\newcommand\co        {\colon}
\newcommand\tmat[2]   {\cl{\begin{matrix}
			#1
		\end{matrix}\, \middle\vert\, \begin{matrix}
			#2
\end{matrix}}}

\makeatletter
\newcommand\rrr[1]    {\xxrightarrow{1}{#1}}
\newcommand\rrt[2]    {\xxrightarrow{1}[#2]{#1}}
\newcommand\mat[2]    {M_{#1\times#2}}
\newcommand\tomat     {\, \dequad \longrightarrow}
\newcommand\pms[1]    {\begin{pmatrix}
		#1
\end{pmatrix}}
\newcommand\dms[1]    {\left\vert\begin{matrix}
		#1
\end{matrix}\right\vert}

% someone's code from the internet: https://tex.stackexchange.com/questions/27545/custom-length-arrows-text-over-and-under
\makeatletter
\newlength\min@xx
\newcommand*\xxrightarrow[1]{\begingroup
	\settowidth\min@xx{$\m@th\scriptstyle#1$}
	\@xxrightarrow}
\newcommand*\@xxrightarrow[2][]{
	\sbox8{$\m@th\scriptstyle#1$}  % subscript
	\ifdim\wd8>\min@xx \min@xx=\wd8 \fi
	\sbox8{$\m@th\scriptstyle#2$} % superscript
	\ifdim\wd8>\min@xx \min@xx=\wd8 \fi
	\xrightarrow[{\mathmakebox[\min@xx]{\scriptstyle#1}}]
	{\mathmakebox[\min@xx]{\scriptstyle#2}}
	\endgroup}
\makeatother

% someone's code form the internet: https://tex.stackexchange.com/questions/12910/how-do-i-typeset-vertical-and-horizontal-lines-inside-a-matrix
\newcommand*{\vertbar}{\rule[-.5ex]{0.5pt}{2.5ex}}
\newcommand*{\horzbar}{\rule[.5ex]{2.5ex}{0.5pt}}

\DeclareMathOperator{\Sp}     {span} 
\DeclareMathOperator{\rk}     {rank}
\DeclareMathOperator{\cols}   {Col}
\DeclareMathOperator{\rows}   {Row}


% Greek Letters
\newcommand\ag        {\alpha}
\newcommand\bg        {\beta}
\newcommand\cg        {\gamma}
\newcommand\dg        {\delta}
\newcommand\eg        {\epsi}
\newcommand\zg        {\zeta}
\newcommand\hg        {\eta}
\newcommand\tg        {\theta}
\newcommand\ig        {\iota}
\newcommand\kg        {\keppa}
\renewcommand\lg      {\lambda}
\newcommand\og        {\omicron}
\newcommand\rg        {\rho}
\newcommand\sg        {\sigma}
\newcommand\yg        {\usilon}
\newcommand\wg        {\omega}

\newcommand\Ag        {\Alpha}
\newcommand\Bg        {\Beta}
\newcommand\Cg        {\Gamma}
\newcommand\Dg        {\Delta}
\newcommand\Eg        {\Epsi}
\newcommand\Zg        {\Zeta}
\newcommand\Hg        {\Eta}
\newcommand\Tg        {\Theta}
\newcommand\Ig        {\Iota}
\newcommand\Kg        {\Keppa}
\newcommand\Lg        {\Lambda}
\newcommand\Og        {\Omicron}
\newcommand\Rg        {\Rho}
\newcommand\Sg        {\Sigma}
\newcommand\Yg        {\Usilon}
\newcommand\Wg        {\Omega}

% Other shortcuts
\newcommand\tl    {\tilde}
\newcommand\op    {^{-1}}

\newcommand\sof[1]    {\left | #1 \right |}
\newcommand\cl [1]    {\left ( #1 \right )}
\newcommand\csb[1]    {\left [ #1 \right ]}
\newcommand\ccb[1]    {\left \{ #1 \right \}}

\newcommand\bs        {\blacksquare}
\newcommand\dequad    {\!\!\!\!\!\!}
\newcommand\dequadd   {\dequad\duquad}
\newcommand\wmid      {\;\middle\vert\;}

\renewcommand\phi     {\varphi}
\newcommand\bcl[1]    {\big(#1\big)}

%! ~~~ Document ~~~

\author{שחר פרץ}
\title{\textit{ליניארית 8}}
\begin{document}
	\maketitle
	\section{}
	יהיו $A, B \in M_n(\F)$ מטריצות. נתבונן ב־: 
	\[ C = \pms{A & A \\ A & B}_{2n \times 2n} \]
	צ.ל. $\rk C \le 2\rk A + \rk B$. 
	\begin{proof}
			ידוע קיום $\tl A, \tl B$. ממשפט, לכל מטריצה $P$ קיימת צורה מדורגת קאנונית, נסמנה $\tl P$, וממשפט ידוע $\rk \tl P = \rk P$. נסמן ב־$\cols P$ את מרחב העמודות של $P$ וב־$\rows P$ את מרחב שורותיה. אם $P$ ריבועית, אז ממשפט $\rows P = \cols P = \rk P$. 
			
			
			
			\textbf{למה 1. }\textit{תהי $P = \binom{X}{Y}$ מטריצה, כאשר $X, Y$ ריבועיות. אז $\bar P = \binom{\tl X }{\tl Y}$ מקיים }
			\[\dim \rows P = \dim \rows \bar P = \dim \rows \tl P \le \rk X + \rk Y\]
			נתבונן ב־$\tl P$. 
			אז כמות שורות שאינן אפסים ב־$\tl P$ היא $\rk \tl X + \rk \tl Y = \rk X + \rk Y$ משום שבצורה מדורגת של מטריצה הדרגה כי כמות השורות שאינן אפסים, ממשפט. מכיוון שבדירוג $X, Y$ בנפרד, למעשה דירגנו את שורותיהם בלבד, ושורותיהם שורות $P$, ומשום שדירוג הוא הכפלה בהרכבת פעולות אלמנטריות בלבד ממשפט, אז $\bar P$ הכפלה בהרכבת שרשור הפעולות האלמנטריות על $X, Y$ (חוקי כי כל אחד בשורות אחרות ולכן הפעולות זרות, והסדר לא משנה), כלומר $\bar P \sim P$. ממשפט, $\rows \bar P = \rows P$, וכמות הוקטורים (שאינם אפסים, שהם ת"ל בינם לבין עצמם) חסם עליון לגודל מ"ו, כלומר $\dim \rows \bar P \le \rk X + \rk Y$. מטרנזיטיביות נקבל את הדרוש. 
			
			נתבונן ב־$D = (A, B)$. ניכר כי $D^T = \binom{A^T}{B^T}$. כמו כן ידוע $\cols A = \rows A = \cols A^T = \rk A$ ובאופן זהה בעבור $B$, ולכן $\rk A^T = \rk A$ וכן על $B$. מטענות שהוצגו $\rk \tl A^T = \rk A$ וכנ"ל בעבור $B$. נתבונן ב־$\bar D = \binom{\tl A^T}{\tl B^T}$. מלמה 1 ומטרנזיטיביות, $\dim \rows \bar D \ge \rk A + \rk B$. 
			
			נתבונן ב־$E = (A, A)$. ניכר כי $E^T = \binom{A^T}{A^T}$. אז באופן דומה לנעשה על $D$, $\rk A = \rk A^T$. מלמה 1 נקבל $\dim \rows E^T = \dim \rows \bar E$, כאשר $\bar E = \binom{\tl A^T}{\tl A^T}$. בגלל ש־$\tl A^T = \tl A^T$ אז $\tl A^T \sim \tl A^T$ ולכן שורותיהם תלויות ליניארית, וסה"כ $\dim \rows E^T = \dim \tl E^T = \rk A$. 
			
			סה"כ קיבלנו $\dim \col A \ge \rk A + \rk B \land \dim \col E = \rk A$. נסמן $C_1 = (A, A), C_2 = (A, B)$. אז $C = \binom{C_1}{C_2}$. 
			
			בגלל ש־$A = A^T$:
			\[ \dim \rows A = \dim \rows \pms{E \\ D} = \dim \cols A^T = \dim \cols \pms{E^T & D^T} = \dim \cols A = \dim \cols\pms{E & D} \]
			בגלל שהוקטורים ב־$\col E$ ו־$\col D$ יחדיו פורשים את מ"ו $\col \pms{E & D}$, ולכן כל אחד מהם תמ"ו של המרחב, ו־$\col D + \col E = \col \pms{D & E}$. לכן ממשפט: 
			\[ \dim \cols \pms{E & D} \le \dim \col E + \dim \col D \le \rk A + \rk B + \rk A = 2\rk A + \rk B \quad \top \]
	\end{proof}
	
	\npage
	\section{}
	נגדיר: 
	\[ v_1 = \pms{1 \\ 2 \\ 3}, \ v_2 = \pms{2 \\ 1 \\ 3}, \ v_3 = \pms{1 \\ 1 \\ 1} \]
	\begin{enumerate}[A)]
		\item נחשב את דרגת המטריצה $A = v_1v_1^T + v_2 + v_2^T + v_3v_3^T$. 
		ראשית כל, נוכיח שהוקטורים בת"לים. נעשה זאת ע"י דירוגם בשורות. 
		\[ \pms{1 & 1 & 1 \\ 1 & 2 & 3 \\ 2 & 1 & 3} \rrt{R_2 \to R_2 - R_1}{R_3 \to R_3 - 2R_1} = \pms{1 & 1 & 1  \\ 0 & 1 & 2 \\ 0 & -1 & 1} \rrr{R_3 \to R_3 + R_2} \pms{1 & 1 & 1 \\ 0 & 1 & 2 \\ 0 & 0 & 3} \]
		מדורג עם $3$ איברים פותחים ולכן פורש מרחב ממימד $3$, וישנם $3$ וקטורים שפרשו את המרחב ולכן הם בסיס ובפרט בת"ל. 
		
		אזי, מסעיף ב', שהוכח באופן בלתי תלוי מסעיף א', נקבל ש־$\rk A = 3$. 
		\item תהי $(v_1 \dots v_k)$ בת"ל. נמצא את דרגת $v_1v_1^T + \cdots + v_kv_k^T$. 
		\begin{proof}
			\textbf{למה 1. }\textit{לכל וקטור $v$, ולכל שורה $R$ ב־$vv^T$, יתקיים $\exists a \in \F \co R = av$. }
			יהי $v$ וקטור מאורך $n$, ו־$A = vv^T$. אז: 
			\[ (A)_{ij} = (v v^T)_{ij} = \sum_{j = 1}^{n = 1}v_{ij}v_{ji} = v_i \cdot v_j \implies (a_{ij})_{j = 0}^n = (v_i \cdot v_j)_{j = 0}^n = v_i \cdot (v_j)_{0}^n = v_i v \]
			סה"כ בעבור כל שורה ב־$A$ יתקיים קיום $a \in v_i$ כך ש־$\col = av$, כדרוש. 
			
			תהי $(v_1 \dots v_k) \in \R^k$ סדרה בת"ל. נסמן $A_k := v_kv_k^T$. נתבונן במטריצה ־$\Si := \sum_{i = 1}^{k} A_k$. נסמן ב־$B_i$ את השורה ה־$i$ במטריצה הכללית $B$. 
			\[ \exists (a_i)_{i = 1}^k \co (\Si)_i = \sum_{i = 1}^{k}(A_k)_i = \sum_{i = 1}^{k}a_i v_k \]
			כאשר טענה הקיום נובעת מלמה 1 (באינדוקציה). מצאנו קומבינציה ליניארית של $(v_i)$, היא $(a_i)$ לכל שורות $\Si$. הקומבינציה הליניארית הזו יחידה משום ש־$(v_i)$ בת"ל, ואם אינה הייתה יחידה, היו בפריסה של $(v_i)$ שני וקטורים עם ייצוג זהה, אך $(v_i)$ בת"ל ופורש את המרחב שהוא עצמו פורש ולכן בסיס, וייצוג איברים מתוך בסיס איזו', ובפרט חח"ע – סתירה. 
			
			בגלל שהוקטורים במרחב השורות של $\Si$ ניתנים לייצוג כקומבינציה ליניארית של שורות $\Si$ מהגדרת $\Sp$, ואלו מוגדרות באופן יחיד מ־$(v_i)$, אז הם ניתנים לייצוג באופן יחיד ע"י וקטורים ב־$(v_i)$. סה"כ, $(v_i)$ בסיס לפי הגדרה למרחב השורות של $\Si$, שממדו זהה למימד $\rk \Si$ לפי הגדרה. בגלל ש־$\Si = v_1v_1^T + \cdots + v_kv_k^T$ וגם $|(v_i)| = k$, אז דרגת המטריצה $v_1v_1^T + \cdots + v_kv_k^T$ היא $k$, כדרוש. 
		\end{proof}
	\end{enumerate}
	
	\section{}
	$A \in M_{m \times n}(\F)$ כך ש־$\rk A = r$. 
	\begin{enumerate}[A)]
		\item צ.ל. קיום $B_1 \dots B_r \in M_{m \times n}(\F)$ כך ש־$A = \sum B_i \land \forall 1 \le i \le r \co \rk B_i = 1$. 
		\begin{proof}
			תהי $A \in M_{m \times n}$ מטריצה, ונסמן $\rk A = r$. נסמן את קבוצת וקטורי העמודה ב־$A$ ב־$\cols A$. אז מרחב העמודות $\Sp \cols A$ נפרש ע"י $\cols A$. משום ש־$\dim \Sp \cols A = \rk A = r$, ובגלל של־$A$ יש $n$ עמודות (כלומר $|\cols A| = n$) אז ממשפט ישנו בסיס בגודל $r$. ממשפט לכל $n \ge i > r$, $i$. אם לא קיימים $r$ וקטורים בת"ל ב־$\cols A$ אז נפרש מרחב מגודל קטן ממש מ־$r$ על ידי הבסיס, ושאר הוקטורים ת"ל ולא משנים ממד, כלומר ישנם $r$ וקטורים ב־$\cols A$ בסיס של $\Sp\cols A$, נסמנם $B$. נסמן גם $\ol B = \cols \setminus B$. 
			
			משום ש־$|\cols A| = r$ אז קיימת פונ' חח"ע ועל $R_i \co [r] \to \cols A$ מהגדרת עוצמה. עבור ההופכית לה $f(x)$ נסמן $v_i = f(v) \in [r]$. בה"כ הסדר $R_i$ זהה לסדר השורות המטריצה, משמאל לימין. 
			אז, לכל $v \in \ol B$, אז $v \in \cols A$ ולכן ניתן לביטוי כקומבינציה ליניארית עם קבועים שנסמן $(\lg^{v_i}_{j})$ (עבור $j$ כך ש־$R_j \in B$). זאת כי: 
			\[ \begin{WithArrows}
				\exists (\tl \lg_i)_{i = 0}^{r + 1} \co 0 &= \tl\lg_{1}B_1 + \cdots + \tl\lg_rB_r + \lg_{r + 1}v \Arrow{ $-\tl\lg_{r +1}v$} \\
				- \lg_{r + 1}v &= \exists (\tl \lg_i)_{i = 0}^{r + 1} \co \tl\lg_{1}B_1 + \cdots + \tl\lg_rB_r \Arrow{$\cdot \frac{-1}{\tl \lg_{r + 1}}$} \\
				v &= -\frac{\tl\lg_1}{\tl\lg_{r + 1}}B_1 - \cdots -\frac{\tl\lg_r}{\tl\lg_{r + 1}}B_1 \\
				:&= \lg_1B_1 + \cdots + \lg_rB_r
			\end{WithArrows} \]
			
			קיים זיווג בין $[r]$ לבין $\{i \in [n] \co R_i \in B\}$ כי יש $r$ וקטורים ב־$B$, נסמנו $f(i)$. בעבור $M$ מט', נסמן ב־$M_i$ את הטור ה־$i$ ב־$M$. נבחין שתחת ההגדרה הזו $R_i = A_i$. 
			נתבונן בקבוצת המטריצות הבאה: 
			\[ (P_i)_j = \begin{cases}
				R_{f(i)} & f(i) = j \\
				\lg_{f(i)}^{j}R_{f(i)} & R_j \in \ol B \\
				0 & \other
			\end{cases} \]
			נוכיח ש־$\sum P_i = A \land \forall i \in [r] \co \rk P = 1$. 
			\begin{itemize}
				\item \textbf{דרגה. }נוכיח ש־$\rk P_i = 1$. ידוע ש־$\rk B = \dim \Sp \cols P_i$ (הדרגה היא ממד מרחב העמודות). נבחין ש־$\cols B$ הוא פרישה של וקטורים שהם אחד מההבאים: 
				\begin{itemize}
					\item וקטור ה־$0$, שת"ל ב־$R_{f(i)}$. 
					\item הוקטור $R_{f(i)}$, שתלוי ליניארית בעצמו. 
					\item וקטור כלשהו $\lg R_{f(i)}$, שבעבור הקבוע $-\frac{1}{\lg}$ סכומו עם $R_{f(i)}$ יהיה $0$ כלומר הוא ת"ל ב־$R_{f(i)}$. 
				\end{itemize}
				סה"כ כל הוקטורים שפורשים את המ"ו ת"ל בוקטור שאינו $0$ (ובפרט גודל המרחב אינו $0$), ולכן גודל המרחב לכל היותר $1$, וסה"כ גודל המ"ו $\cols B$ הוא $1$, כלומר $\rk P = 1$ כדרוש. 
				\item \textbf{סכום. }נוכיח ש־$\Si := \sum P_i = A$. נוכיח שוויון עמודות. נפלג למקרים בעבור העמודה$j$ ב־$A$. 
				\begin{itemize}
					\item נניח $j \in \{i \in [n] \co R_i \in B\} = \Img f$. אז עבור כל מטריצה $P_i$, אם $f(i) = j$, אז מהיות $f$ זיווג, $i$ האינדקס של $P$ הוא  גם קיים וגם היחיד שיקיים תכונה זו. כמו כן, $j \notin B$ ולכן בשאר המטריצות עמודה זו תהיה $0$. סה"כ קיבלנו $0 + \cdots + R_{f(i)} + \cdots + 0 = R_j = A_j$, כדרוש. 
					\item אחרת, $j \notin \Img f$. לכן בהכרח $R_j \in \ol B$ מהגדרת תמונת $f$. אז: 
					\[ \Si = \sum_{i = 1}^{r}\lg^{j}_{f(i)} R_{f(i)} \overset{(1)}{=} \sum_{i = 1}^{r}\lg^{j}_{f(i)}R_{f(i)} \overset{(2)}{=} A_j \]
					כדרוש. \textit{הערות: }
					\begin{enumerate}
						\item[$(1)$]\textit{נכון מהיות $f$ על + שינוי סדר סכימה. }
						\item[$(2)$]\textit{נכון מהגדרת $(\lg_i)$, ע"פ זהות זו בדיוק. }
					\end{enumerate}
				\end{itemize}
				סה"כ הוכחנו שאכן לסכום שוויון ל־$A$, וה־$\rk$ של כל מטריצה בסכום הוא $1$. 
			\end{itemize}
			משהוכח לעיל ובגלל ש־$|(P_i)_{i = 0}^r| = r$, הוכחנו את הדרוש, ו־$(P_i)$ הסדרה המבוקשת. 			
		\end{proof}
		\item צ.ל. אי קיום $B_1 \dots B_{r - 1} \in M_{m \times n}(\F)$ כך ש־$A = \sum B_i \land \forall 1 \le i \le r - 1 \co \rk B_i = 1$. 
		\begin{proof}
			תהי $A \in M_n(\F)$, נסמן $\rk A = r$. נניח בשלילה קיום $B_1 \cdots B_{r - 1} \in M_{m \times n}(\F)$, נסמן $B := (B_i)_{i = 0}^{r - 1}$, כך ש־$\sum_{i =1}^{r - 1}B_i = A$. 
	נסמן ב־$\rows P$ את קבוצת וקטורי השורה של המטריצה $P$. 
	
	בגלל ש־$\forall B_i \in B \co \rk B_i = 1 = \dim \Sp \rows B_i$, ומשום שקיימת שורה $R$ ב־$B_i$ כך ש־$R \neq 0$ כי אחרת $\rows B = \rows 0 = 0$ ו־$\rk B_i = 0$ סתירה, אז $R$ ב־$\Sp \rows B_i$ ואינו $0$ כלומר $\Sp\{R\}$ מממד $1$ גם כן, ותמ"ו ב־$\Sp\rows B_i$ כי $R \in B_i$, ממשפט תמ"ו של מ"ו זהה ממד הוא שווה מרחב, כלומר $\Sp \{R\} = \Sp \rows B_i$. משום שהממד הוא $1$ אז $\{R\}$ בסיס ל־$\rows B_i$. נסמן ב־$R^j_i$ את הוקטור ה־$j$ ב־$R_i$. אז משום ש־$R$ פורש את $\rows$ אז קיים $\lg$ כך ש־$\lg R = R^j_i$. נסמן אותו להיות $\lg_i^j$. נסמן את ה־$R$ המתאים לכל מטריצה $B_i$ ב־$R_i$. 
	
	נסמן ב־$A_i$ את השורה ה־$i$ ב־$A$. סה"כ: 
	\[ A_i = \cl{\sum_{j = 1}^{r - 1}B_j}_{\mathclap{i}} \!= \sum_{j = 1}^{r - 1}R^i_j = \sum_{j = 1}^{r - 1} \lg^{i}_jR_j \]
	לכל $v \in \Sp \rows A$ יתקיים קיום $(\ag_i)_{i = 1}^{n}$ כך ש־$v = \ag_1 A_1 + \cdots + \ag_nA_n$. אז: 
	\[ v = \sum_{i = 1}^{n}\ag_iA_i = \sum_{i = 1}^{n}\cl{\ag_i\sum_{j = 1}^{r - 1}\lg^{i}_jR_i}
	=\sum_{j = 1}^{r - 1}\Bigg[\underbrace{\cl{\sum_{i = 1}^{n}\ag_i\lg^i_j}}_{:= \bg_i}R_i \Bigg] = \sum_{i = 1}^{r - 1}\bg_iR_i
	 \]
	 סה"כ, $v$ ניתן לביטוי ע"י קומבינציה ליניארית עם מקדמים $(\bg_i)$ של $(R_i)_{i = 1}^{r - 1}$. סה"כ חסם עליון לבסיס בסיס שהוא $r - 1$ כי מצאנו פורש בגודל $r -1$ ל־$\Sp\rows A$, לכן $\dim \Sp\rows A \le r - 1$. בגלל ש־$\rk A = r$ אז $\dim \Sp \rows A =r$ ומטרנזיטיביות $r \le r - 1$. נחסיר $r$ משני האגפים ונכפיל ב־$(-1)$, נקבל $0 \ge 1$, סתירה. 
	 
	 הגענו לסתירה, והנחת השלילה הופרכה. הוכחנו כי אכן לא קיימים $(B_i)_{i = 1}^{r - 1}$ המקיימים את התנאים לעיל. 
		\end{proof}
	\end{enumerate}
	
	\section{}
	נתבונן בשלושת הוקטורים הבאים: 
	\[ u = \pms{u_1 \\ u_2 \\ u_3}, \ v = \pms{v_1 \\ v_2 \\ v_3}, \ w = \pms{w_1 \\ w_2 \\ w_3}, \ \det(u, v, w) = \det\pms{u_1 & v_1 & w_1 \\ u_2 & v_2 & w_2 \\ u_3 & v_3 & w_3} \]
	נביר את $\det(u + v, v, w)$ ואת $\det(u + v, v + w, w + u)$ באמצעות $\det(u, v, w)$. 
	
	\textbf{למה 1. }\textit{ליניאריות חיבור עמודות. }
	\begin{align*}
		\det(u + x, v, w) &= \det\underbrace{\pms{\vertbar & \vertbar & \vertbar \\ u + x & v & w \\ \vertbar & \vertbar & \vertbar}}_{:=A} = \det A = \det A^T =
		\dms{\horzbar & w & \horzbar \\ \horzbar & v & \horzbar \\ \horzbar & u + x & \horzbar}
		\overset{(1)}{=}
		\dms{\horzbar & w & \horzbar \\ \horzbar & v & \horzbar \\ \horzbar & u & \horzbar}
		+ \dms{\horzbar & w & \horzbar \\ \horzbar & v & \horzbar \\ \horzbar & x & \horzbar} \\
		&= \dms{\vertbar & \vertbar & \vertbar \\ w & v & u \\ \vertbar & \vertbar & \vertbar}
		+ \dms{\vertbar & \vertbar & \vertbar \\ w & v & x \\ \vertbar & \vertbar & \vertbar}
		 = \det(u, v, w) + \det(x, v, w)
	\end{align*}
	ניעזר בלמה זו תחת הסימון $[1]$. \textit{הערות: }
	\begin{enumerate}
		\item[$(1)$] מתקיים ממולטיליניאריות. 
		\item[$(2)$] מתקיים מהפעלת transpose, כי דיטרמיננטה משמרת transpose. 
	\end{enumerate}
	\textbf{למה 3. }\textit{חילוף עמודות הופך כיוון דיטרמיננטה. }תהי $A$ מטריצה בה עמודות $u, v$, נסמן ב־$A'$ את המטריצה לאחר החלפת השורות בהן נמצאים וקטורי השורה $u, v$ .
	\begin{align*}
		&\det A := \det(\dots, u, \dots v, \dots) = \det A = \det A^T \overset{(1)}{=} -\det A'^T = -\det A' 
	\end{align*}
	\textit{הערות: $(1)$ נכון בעבור החלפה של שתי השורות בהן $u^T, v^T$ וקטורי העמודה, וכי החלפת שורות מטריצה הופכת את כיוון הדיטרמיננטה שלה. }
	
	ניעזר בלמה זו תחת הסימון $[3]$. 
	
	\textbf{למה 2. }	\textit{הדיטרמיננטה של מטריצה עם עמודות זהות היא 0. }בעבור החלפת השורות הזהות: 
	\[ \det A \overset{[3]}{=} -\det A \implies 2\det A = 0 \implies \det A = 0 \]
	\textit{הערה: כחלק מהוכחת למה 2 ניעזרנו בלמה 3, אך בלמה 3 לא ניעזרנו בלמה 2, ולכן ההוכחה אינה מעגלית. }
	
	ניעזר בלמה זו תחת הסימון $[2]$. 
	
	\begin{enumerate}
		\item 
		\[ \det(u + v, v, w) \overset{[1]}{=} \det(u, v, w) + \det(v, v, w) \overset{[2]}{=} \bm{\det(u, v, w)} \]
		\item 
		\begin{align*}
			\det(u + v, v + w, w + u)
			\overset{[1]}{=} &\det(u, v + w, w + u) + \det(v, v + w, w + u) \\
			\overset{[1]}{=} &\det(u, v, w + u) + \det(u, w, w + u) + \det(v, v, w + u) + \det(v, w, w + u) \\
			\overset{[1]}{=} &\det(u, v, w) + \det(u, w, w) + \det(v, v, w) + \det(v, w, w) \\
			+&\det(u, v, u) + \det(u, w, u) + \det(v, v, u) + \det(v, w, u) \\
			\overset{[2]}{=} &\det(u, v, w) + \cancel{\det(u, w, w)} + \cancel{\det(v, v, w)} + \cancel{\det(v, w, w)} \\
			+&\cancel{\det(u, v, u)} + \cancel{\det(u, w, u)} + \cancel{\det(v, v, u)} + \det(v, w, u) \\
			=&\det(u, v, w) + \det(u, w, v) \\
			\overset{[3]}{=}&\det(u, v, w) - \det(u, v, w) = \bm{0}
		\end{align*}
	\end{enumerate}
	
	\ndoc
	
\end{document}