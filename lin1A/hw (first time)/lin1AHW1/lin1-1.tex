%! ~~~ Packages Setup ~~~ 
\documentclass[]{article}


% Math packages
\usepackage[usenames]{color}
\usepackage{forest}
\usepackage{ifxetex,ifluatex,amsmath,amssymb,mathrsfs,amsthm,witharrows,mathtools}
\WithArrowsOptions{displaystyle}
\renewcommand{\qedsymbol}{$\blacksquare$} % end proofs with \blacksquare. Overwrites the defualts. 
\usepackage{cancel,bm}
\usepackage[thinc]{esdiff}
\usepackage{lineno}


% tikz
\usepackage{tikz}
\newcommand\sqw{1}
\newcommand\squ[4][1]{\fill[#4] (#2*\sqw,#3*\sqw) rectangle +(#1*\sqw,#1*\sqw);}


% code 
\usepackage{listings}
\usepackage{xcolor}

\definecolor{codegreen}{rgb}{0,0.35,0}
\definecolor{codegray}{rgb}{0.5,0.5,0.5}
\definecolor{codenumber}{rgb}{0.1,0.3,0.5}
\definecolor{codeblue}{rgb}{0,0,0.5}
\definecolor{codered}{rgb}{0.5,0.03,0.02}
\definecolor{codegray}{rgb}{0.96,0.96,0.96}

\lstdefinestyle{pythonstylesheet}{
	language=Python,
	emphstyle=\color{deepred},
	backgroundcolor=\color{codegray},
	keywordstyle=\color{deepblue}\bfseries\itshape,
	numberstyle=\scriptsize\color{codenumber},
	basicstyle=\ttfamily\footnotesize,
	commentstyle=\color{codegreen}\itshape,
	breakatwhitespace=false, 
	breaklines=true, 
	captionpos=b, 
	keepspaces=true, 
	numbers=left, 
	numbersep=5pt, 
	showspaces=false,                
	showstringspaces=false,
	showtabs=false, 
	tabsize=4, 
	morekeywords={as,assert,nonlocal,with,yield,self,True,False,None,AssertionError,ValueError,in,else},              % Add keywords here
	keywordstyle=\color{codeblue},
	emph={object,type,isinstance,copy,deepcopy,zip,enumerate,reversed,list,set,len,dict,tuple,print,range,xrange,append,execfile,real,imag,reduce,str,repr,__init__,__add__,__mul__,__div__,__sub__,__call__,__getitem__,__setitem__,__eq__,__ne__,__nonzero__,__rmul__,__radd__,__repr__,__str__,__get__,__truediv__,__pow__,__name__,__future__,__all__,},          % Custom highlighting
	emphstyle=\color{codered},
	stringstyle=\color{codegreen},
	showstringspaces=false,
	abovecaptionskip=0pt,belowcaptionskip =0pt,
	framextopmargin=-\topsep, 
}
\newcommand\pythonstyle{\lstset{pythonstylesheet}}
\newcommand\pyl[1]     {{\lstinline!#1!}}
\lstset{style=pythonstylesheet}

\usepackage[style=1,skipbelow=\topskip,skipabove=\topskip,framemethod=TikZ]{mdframed}
\definecolor{bggray}{rgb}{0.85, 0.85, 0.85}
\mdfsetup{leftmargin=0pt,rightmargin=0pt,innerleftmargin=15pt,backgroundcolor=codegray,middlelinewidth=0.5pt,skipabove=5pt,skipbelow=0pt,middlelinecolor=black,roundcorner=5}
\BeforeBeginEnvironment{lstlisting}{\begin{mdframed}\vspace{-0.4em}}
	\AfterEndEnvironment{lstlisting}{\vspace{-0.8em}\end{mdframed}}


% Deisgn
\usepackage[labelfont=bf]{caption}
\usepackage[margin=0.6in]{geometry}
\usepackage{multicol}
\usepackage[skip=4pt, indent=0pt]{parskip}
\usepackage[normalem]{ulem}
\forestset{default}
\renewcommand\labelitemi{$\bullet$}
\usepackage{titlesec}
\titleformat{\section}[block]
{\fontsize{15}{15}}
{\sen \dotfill (\thesection) \dotfill \she}
{0em}
{\MakeUppercase}
\usepackage{graphicx}
\graphicspath{ {./} }


% Hebrew initialzing
\usepackage[bidi=basic]{babel}
\PassOptionsToPackage{no-math}{fontspec}
\babelprovide[main, import, Alph=letters]{hebrew}
\babelprovide[import]{english}
\babelfont[hebrew]{rm}{David CLM}
\babelfont[hebrew]{sf}{David CLM}
\babelfont[english]{tt}{Monaspace Xenon}
\usepackage[shortlabels]{enumitem}
\newlist{hebenum}{enumerate}{1}

% Language Shortcuts
\newcommand\en[1] {\begin{otherlanguage}{english}#1\end{otherlanguage}}
\newcommand\sen   {\begin{otherlanguage}{english}}
	\newcommand\she   {\end{otherlanguage}}
\newcommand\del   {$ \!\! $}
\newcommand\ttt[1]{\en{\footnotesize\texttt{#1}\normalsize}}

\newcommand\npage {\vfil {\hfil \textbf{\textit{המשך בעמוד הבא}}} \hfil \vfil \pagebreak}
\newcommand\ndoc  {\dotfill \\ \vfil {\begin{center} {\textbf{\textit{שחר פרץ, 2024}} \\ \scriptsize \textit{נוצר באמצעות תוכנה חופשית בלבד}} \end{center}} \vfil	}

\newcommand{\rn}[1]{
	\textup{\uppercase\expandafter{\romannumeral#1}}
}

\makeatletter
\newcommand{\skipitems}[1]{
	\addtocounter{\@enumctr}{#1}
}
\makeatother

%! ~~~ Math shortcuts ~~~

% Letters shortcuts
\newcommand\N     {\mathbb{N}}
\newcommand\Z     {\mathbb{Z}}
\newcommand\R     {\mathbb{R}}
\newcommand\Q     {\mathbb{Q}}
\newcommand\C     {\mathbb{C}}

\newcommand\ml    {\ell}
\newcommand\mj    {\jmath}
\newcommand\mi    {\imath}

\newcommand\powerset {\mathcal{P}}
\newcommand\ps    {\mathcal{P}}
\newcommand\pc    {\mathcal{P}}
\newcommand\ac    {\mathcal{A}}
\newcommand\bc    {\mathcal{B}}
\newcommand\cc    {\mathcal{C}}
\newcommand\dc    {\mathcal{D}}
\newcommand\ec    {\mathcal{E}}
\newcommand\fc    {\mathcal{F}}
\newcommand\nc    {\mathcal{N}}
\newcommand\sca   {\mathcal{S}} % \sc is already definded
\newcommand\rca   {\mathcal{R}} % \rc is already definded

\newcommand\Si    {\Sigma}

% Logic & sets shorcuts
\newcommand\siff  {\longleftrightarrow}
\newcommand\ssiff {\leftrightarrow}
\newcommand\so    {\longrightarrow}
\newcommand\sso   {\rightarrow}

\newcommand\epsi  {\epsilon}
\newcommand\vepsi {\varepsilon}
\newcommand\vphi  {\varphi}
\newcommand\Neven {\N_{\mathrm{even}}}
\newcommand\Nodd  {\N_{\mathrm{odd }}}
\newcommand\Zeven {\Z_{\mathrm{even}}}
\newcommand\Zodd  {\Z_{\mathrm{odd }}}
\newcommand\Np    {\N_+}

% Text Shortcuts
\newcommand\open  {\big(}
\newcommand\qopen {\quad\big(}
\newcommand\close {\big)}
\newcommand\also  {\text{, }}
\newcommand\defi  {\text{ definition}}
\newcommand\defis {\text{ definitions}}
\newcommand\given {\text{given }}
\newcommand\case  {\text{if }}
\newcommand\syx   {\text{ syntax}}
\newcommand\rle   {\text{ rule}}
\newcommand\other {\text{else}}
\newcommand\set   {\ell et \text{ }}
\newcommand\ans   {\mathit{Ans.}}

% Set theory shortcuts
\newcommand\ra    {\rangle}
\newcommand\la    {\langle}

\newcommand\oto   {\leftarrow}

\newcommand\QED   {\quad\quad\mathscr{Q.E.D.}\;\;\blacksquare}
\newcommand\QEF   {\quad\quad\mathscr{Q.E.F.}}
\newcommand\eQED  {\mathscr{Q.E.D.}\;\;\blacksquare}
\newcommand\eQEF  {\mathscr{Q.E.F.}}
\newcommand\jQED  {\mathscr{Q.E.D.}}

\newcommand\dom   {\mathrm{dom}}
\newcommand\Img   {\mathrm{Im}}
\newcommand\range {\mathrm{range}}

\newcommand\trio  {\triangle}

\newcommand\rc    {\right\rceil}
\newcommand\lc    {\left\lceil}
\newcommand\rf    {\right\rfloor}
\newcommand\lf    {\left\lfloor}

\newcommand\lex   {<_{lex}}

\newcommand\az    {\aleph_0}
\newcommand\uaz   {^{\aleph_0}}
\newcommand\al    {\aleph}
\newcommand\ual   {^\aleph}
\newcommand\taz   {2^{\aleph_0}}
\newcommand\utaz  { ^{\left (2^{\aleph_0} \right )}}
\newcommand\tal   {2^{\aleph}}
\newcommand\utal  { ^{\left (2^{\aleph} \right )}}
\newcommand\ttaz  {2^{\left (2^{\aleph_0}\right )}}

\newcommand\n     {$n$־יה\ }

% Math A&B shortcuts
\newcommand\logn  {\log n}
\newcommand\logx  {\log x}
\newcommand\lnx   {\ln x}
\newcommand\cosx  {\cos x}
\newcommand\cost  {\cos \theta}
\newcommand\sinx  {\sin x}
\newcommand\sint  {\sin \theta}
\newcommand\tanx  {\tan x}
\newcommand\tant  {\tan \theta}
\newcommand\sex   {\sec x}
\newcommand\sect  {\sec^2}
\newcommand\cotx  {\cot x}
\newcommand\cscx  {\csc x}
\newcommand\sinhx {\sinh x}
\newcommand\coshx {\cosh x}
\newcommand\tanhx {\tanh x}

\newcommand\seq   {\overset{!}{=}}
\newcommand\slh   {\overset{LH}{=}}
\newcommand\sle   {\overset{!}{\le}}
\newcommand\sge   {\overset{!}{\ge}}
\newcommand\sll   {\overset{!}{<}}
\newcommand\sgg   {\overset{!}{>}}

\newcommand\h     {\hat}
\newcommand\ve    {\vec}
\newcommand\lv    {\overrightarrow}
\newcommand\ol    {\overline}

\newcommand\mlcm  {\mathrm{lcm}}

\DeclareMathOperator{\sech}   {sech}
\DeclareMathOperator{\csch}   {csch}
\DeclareMathOperator{\arcsec} {arcsec}
\DeclareMathOperator{\arccot} {arcCot}
\DeclareMathOperator{\arccsc} {arcCsc}
\DeclareMathOperator{\arccosh}{arccosh}
\DeclareMathOperator{\arcsinh}{arcsinh}
\DeclareMathOperator{\arctanh}{arctanh}
\DeclareMathOperator{\arcsech}{arcsech}
\DeclareMathOperator{\arccsch}{arccsch}
\DeclareMathOperator{\arccoth}{arccoth}
\DeclareMathOperator{\atant}  {atan2} 

\newcommand\dx    {\,\mathrm{d}x}
\newcommand\dt    {\,\mathrm{d}t}
\newcommand\dtt   {\,\mathrm{d}\theta}
\newcommand\du    {\,\mathrm{d}u}
\newcommand\dv    {\,\mathrm{d}v}
\newcommand\df    {\mathrm{d}f}
\newcommand\dfdx  {\diff{f}{x}}
\newcommand\dit   {\limhz \frac{f(x + h) - f(x)}{h}}

\newcommand\nt[1] {\frac{#1}{#1}}

\newcommand\limz  {\lim_{x \to 0}}
\newcommand\limxz {\lim_{x \to x_0}}
\newcommand\limi  {\lim_{x \to \infty}}
\newcommand\limh  {\lim_{x \to 0}}
\newcommand\limni {\lim_{x \to - \infty}}
\newcommand\limpmi{\lim_{x \to \pm \infty}}

\newcommand\ta    {\theta}
\newcommand\ap    {\alpha}

\renewcommand\inf {\infty}
\newcommand  \ninf{-\inf}

% Combinatorics shortcuts
\newcommand\sumnk     {\sum_{k = 0}^{n}}
\newcommand\sumni     {\sum_{i = 0}^{n}}
\newcommand\sumnko    {\sum_{k = 1}^{n}}
\newcommand\sumnio    {\sum_{i = 1}^{n}}
\newcommand\sumai     {\sum_{i = 1}^{n} A_i}
\newcommand\nsum[2]   {\reflectbox{\displaystyle\sum_{\reflectbox{\scriptsize$#1$}}^{\reflectbox{\scriptsize$#2$}}}}

\newcommand\bink      {\binom{n}{k}}
\newcommand\setn      {\{a_i\}^{2n}_{i = 1}}
\newcommand\setc[1]   {\{a_i\}^{#1}_{i = 1}}

\newcommand\cupain    {\bigcup_{i = 1}^{n} A_i}
\newcommand\cupai[1]  {\bigcup_{i = 1}^{#1} A_i}
\newcommand\cupiiai   {\bigcup_{i \in I} A_i}
\newcommand\capain    {\bigcap_{i = 1}^{n} A_i}
\newcommand\capai[1]  {\bigcap_{i = 1}^{#1} A_i}
\newcommand\capiiai   {\bigcap_{i \in I} A_i}

\newcommand\xot       {x_{1, 2}}
\newcommand\ano       {a_{n - 1}}
\newcommand\ant       {a_{n - 2}}

% Other shortcuts
\newcommand\tl    {\tilde}
\newcommand\op    {^{-1}}

\newcommand\sof[1]    {\left | #1 \right |}
\newcommand\cl [1]    {\left ( #1 \right )}
\newcommand\csb[1]    {\left [ #1 \right ]}

\newcommand\bs    {\blacksquare}

%! ~~~ Document ~~~

\author{שחר פרץ}
\title{\textit{תרגיל בית 1 -- אלגברה ליניארית  1א'}}
\begin{document}
	\maketitle
	\section{}
	נחשב את הביטויים הבאים על $\C$: 
	\en{\begin{gather}
		(1 + i)(2 - 5i) + (-1 + i) \cdot (4 + 3i) = 2 - 5i + 2i + 5 -4 -3i + 4i - 3 = \bm{-2i} \\
		\frac{1 - 7i}{4 - 5i} + (3 - i) = \frac{(1 - 7i)(4 + 5i)}{4^2 + 5^2} + 3 - i = \frac{4 + 5i -28i +25}{41} + 3 - i = \frac{29}{41} - \frac{23}{41}i + 3 - i = \bm{\frac{152}{41} - \frac{64}{41}i} \\
		\frac{(\ol{-3 + i})(2 + 4i)}{1 + 8i} = \frac{(-3 - i)(2 + 4i)(1 - 8i)}{1 + 8^2} = 65\op\cl{(-2 - 14i)(1 - 8i)} = 65\op(-114 + 2i) = \bm{-1\frac{49}{65} + \frac{2}{65}i}
	\end{gather}}
	
	\section{}
	נמיר את המספרים הבאים להצגה פולארית (כאשר $\ta$ הזווית של $z$): 
	\setcounter{equation}{0}
	\en{\begin{gather}
			z = 1 + i, \ |z| = \sqrt{1 + 1} = \sqrt{2}, \ \ta = \arctan\cl{\frac{1}{1}} = \frac{\pi}{4}, \ \bm{z = \sqrt2e^{i \cdot \frac{\pi}{4}}} \\
			z = \sqrt{2}i, \, \text{which is a line, hence: }|z| = \sqrt{2}, \ \ta = 90^\circ = \frac{\pi}{2}, \ z = \sqrt{2}e^{\frac{\pi}{2}i} \\
			z =  -7i, \, \text{which is a line too, hence: }|z| = -7, \ \ta = \frac{\pi}{2}, \ z = -7e^{\frac{\pi}{2}i} \\
			z = -1 + \sqrt{3}i, \ |z| = \cl{1^2 + \sqrt{3}^2}^{0.5} = 2, \ \ta = \pi - \arctan\left(\sqrt3\right)\ = \frac{2}{3}\pi, \ \bm{z = 2e^{\frac{2}{3}\pi i}}
	\end{gather}}

	\section{}
	נמצא את ההצגה הקרטזית של המספרים הבאים (נסמן ב־$m$ את היחס $y / x$): 
	\setcounter{equation}{0}
	\sen
	\begin{gather}
		z = 2e^{i\frac{\pi}{2}}, \ \tan\cl{\frac{\pi}{2}} = \inf \implies y / x = \inf \implies x = 0 \implies y = \sqrt{y^2 + 0} = |z| = 2, \ \bm{z = 2i} \\
		6e^{-i\frac{\pi}{6}} = 6\cl{\cos\cl{-\frac{\pi}{6}} + i \sin \cl{-\frac{\pi}{6}}} = 6\cl{\frac{\sqrt3}{2} + -0.5i} = \bm{3\sqrt3 + -3i}
	\end{gather}
	\she
	
	\section{}
	נמצא את כל הפתתרונות למשוואות הבאות: 
	\begin{enumerate}[A.]
		\item בעבור $z^8 = -1$. נסמן $z = re^{i\ta}$. כלומר: 
		\[ r^8e^{i8\ta} = z^8 = -1 = e^{i \pi } \implies r^8 = 1 \to r = 1, \ 8\ta = \pi + 2\pi k \to \ta \pi + \frac{k}{4}\pi \]
		כאשר מחזור של $2\pi k$, כלומר די בפתרונות $k \in [8]$. 
		וסה"כ קבוצת הפתרונות תהיה $\bm{\{e^{0.25k \pi} \mid k \in [8]\}}$
		
		\item בעבור $z^3 = 8$. נסמן $z = re^{i \ta}$. 
		\[ (re^{i 3\ra}) = z^3 = 8 = 8e^{0} \implies r = 8, \ \ta = 0 + 2\pi k \]
		מכיוון שבעבור כל $k$ נמצא סיבוב שלם, סה"כ קבוצת הפתרונות תהיה $\bm{\{8\}}$
		\item בעבור $z^2 + z + 1 = 0$, נתבונן בנוסחאת השורשים: 
		\[ z = \frac{-1\pm\sqrt{1^2 - 4 \cdot 1 \cdot 1}}{2 \cdot 1} = -\frac{1}{2} \pm \sqrt{-\frac{3}{4}} = -0.5 \pm 0.5\sqrt3i \]
		וקבוצת הפתרונות בהתאם. 
	\end{enumerate}
	
	\section{}
	נמצא את הנגדי וההופכי של כל איבר ב־$\Z_7$: 
	\begin{center}
		\begin{tabular}{|c|c|c|c|c|c|c|c|}
			\hline $\bm{n}$ & $0$ & $1$ & $2$ & $3$ & $4$ & $5$ & $6$ \\
			\hline $\bm{-n}$ & $0$ & $6$ & $5$ & $4$ & $3$ & $2$ & $1$ \\
			\hline $\bm{n\op}$ & $\varnothing$ & $1$ & $4$ & $5$ & $2$ & $3$ & $6$ \\
			\hline
		\end{tabular}
	\end{center}
	
	\section{}
	ידועות הטענות: 
	\begin{gather*}
		(a + b) \bmod n = (a \bmod n + b \bmod n) \bmod n \\
		(a \cdot b) \bmod n = ((a \bmod n) \cdot (b \bmod n)) \bmod n \\
		(a^b)\bmod n = (a \bmod n)^b \bmod n
	\end{gather*}
	ולכן: 
	\begin{multline*}
		(24^{100} - 4^{100})\bmod 13 = ((24)^{100}\bmod 13 - 16^{50}\bmod 13)\bmod 13 = (11^{100}\bmod 13 - 3^{50}\bmod 13)\bmod 13 \\
		= ((11^2 \bmod 13)^{50} - (3^5\bmod 13)^{10})\bmod 13 = (16^{25}\bmod 13 - 9^{10}\bmod 13) \bmod 13 = (3^{25}\bmod - 81^{5}\bmod 13) \bmod 13 \\
		= (9^{5}\bmod 13 - 3^{5}\bmod 13)\bmod 13 = (3 - 9)\bmod 13 = \bm{7}
	\end{multline*}
	
	
	\section{}
	יהי $\mathbb{F}$ שדה. צ.ל. $\forall a, b \in \mathbb{F}. (-a)b = a(-b) = ab$
	\begin{proof}
		יהיו $a, b \in \mathbb{F}$. ניזכר בטענה מההרצאה לפיה $(-1) \cdot c = -c$, לכל $c$ מספר בשדה, ובפרט עבור $c = a, b, ab$. לכן, מאסוציאטיביות: 
		\[ (-a)b = ((-1)a)b = (-1)(ab) = -ab \]
		ומסימטריה גם $a(-b)$ יקיים את הדרוש. 
	\end{proof}
	
	\section{}
	נתבונן בפעולות החיבור והכפל $\oplus, \otimes$ בהחלפה על $\R$. ננמק אילו מתכונות השדה מתקיימות או לא מתקיימות בעבור כל אחת מהן. יהיו $x, y, z, a \in \R$ (בקונטסט הזה, אלא אם צויין אחרת);
	\begin{enumerate}[A.]
		\item נגדיר $\oplus(x, y) = x + y + 5, \otimes(x, y) = 2xy$. 
		\begin{itemize}
			\item תתקיים קומטטביות חיבור: \hfill $x \oplus y = x + y + 5 = y + x + 5 = y \oplus x$
			\item תתקיים קומטטביות כפל: \hfill $x \otimes y = 2xy = 2yx = y \otimes x$
			\item תתקיים אסוציאטיביות חיבור: \hfill $(x \oplus y)\oplus z = (x + y + 5) + z + 5 = x + y + z + 10 = x + (y + z + 5) + 5 = x \oplus (y \oplus z)$
			\item תתקיים אסוציאטיביות כפל: \hfill $(x \otimes y) \otimes z = 2(2xy)z = 4xyz = 2x(2yz) = x \otimes (y \otimes z)$
			\item ימצא איבר הופכי לחיבור: \hfill $x + (-x - 5) = x - x - 5 + 5 = 0_\R$
			\item ימצא איבר הופכי לכפל: \hfill $x \cdot (0.5x\op) = 2 \cdot 0.5 x x\op = 1_\R$
			\item לא תתקיים דיסטרבוטיביות: בעבור $z = x = y = 1$, \ \hfill $(x \oplus y) \otimes z = (x + y + 5) \cdot 2z = 2zx + 2zy + 10z = x \otimes z + y \otimes z + 10 \otimes z \bot$
		\end{itemize}
		ומשום שאין דיסטרבוטיביות זהו אינו שדה. 
		
		\item נגדיר $\oplus = +, \ \otimes = \cdot \circ (\lambda x \in \R. 2x)$. 
		באופן זהה לסעיף א' בעבור הכפל שמוגדר אותו הדבר. החיבור מוגדר באופן זהה לחיבור ב־$\R$ ולכן קומטטיבי, אסוציאטיבי ובעל איבר נגדי. בעבור דיסטרבוטיביות: 
		\[ (x \oplus y) \otimes z = (x + y) \cdot 2z = 2zx + 2zy = x \otimes z + y \otimes z = (x \otimes z) \oplus (y \otimes z) \]
		ולכן זהו שדה. 
		\item נגדיר $x \oplus y = x, \ x \times y = x^2$. אזי: 
		\begin{itemize}
			\item תתקיים קומטטיביות חיבור: \hfill $x \oplus y = x = y \oplus x$
			\item תתקיים קומטטיביות כפל: \hfill $x \otimes y = x = y \otimes x$
			\item לא תתקיים אסוציאטיביות חיבור: בעבור $z = 1, x = 2$ \hfill $(x \oplus y) \oplus z = x \neq z =  z \oplus (y \oplus x)$
			\item לא תתקיים אסוציאטיביות כפל: בעבור $z = 1, x = 2$ \hfill $(x \otimes y) \otimes z = (x^2)^2 =x^4 \neq z^4 = (z^2)z^2 =  z \otimes (y \otimes x)$
			\item לא ימצא נגדי חיבור: נניח בשלילה שקיים כזה  $0_O$, \hfill $\forall y \in \R \setminus \{0_O\}. x \oplus y = x \neq 0_O \bot$
			\item לא ימצא נגדי לכפל: נניח בשלילה שקיים כזה $1_O$ \hfill $\forall y \in \R. x \otimes y = x^2 \neq 1_O \bot$
			
				(כי לא ריבוע כל מספר הוא $1_O$, כי $1^2 = 1 = 1_O = 4 = 2^2$ אך $1 \neq 4$)
			\item תתקיים דיסטרבוטיביות: \hfill $(x \oplus y) \otimes z = x \otimes z = x^2 = x^2 \oplus a = (x \otimes z) \oplus (y \otimes z)$
		\end{itemize}
		ומשום שאין אסוציאטיביות זהו אינו שדה. 
	\end{enumerate}
	
	\ndoc
	
	
\end{document}