%! ~~~ Packages Setup ~~~ 
\documentclass[]{article}
\usepackage{lipsum}
\usepackage{rotating}


% Math packages
\usepackage[usenames]{color}
\usepackage{forest}
\usepackage{ifxetex,ifluatex,amssymb,amsmath,mathrsfs,amsthm,witharrows,mathtools,mathdots}
\usepackage{amsmath}
\WithArrowsOptions{displaystyle}
\renewcommand{\qedsymbol}{$\blacksquare$} % end proofs with \blacksquare. Overwrites the defualts. 
\usepackage{cancel,bm}
\usepackage[thinc]{esdiff}


% tikz
\usepackage{tikz}
\usetikzlibrary{graphs}
\newcommand\sqw{1}
\newcommand\squ[4][1]{\fill[#4] (#2*\sqw,#3*\sqw) rectangle +(#1*\sqw,#1*\sqw);}


% code 
\usepackage{algorithm2e}
\usepackage{listings}
\usepackage{xcolor}

\definecolor{codegreen}{rgb}{0,0.35,0}
\definecolor{codegray}{rgb}{0.5,0.5,0.5}
\definecolor{codenumber}{rgb}{0.1,0.3,0.5}
\definecolor{codeblue}{rgb}{0,0,0.5}
\definecolor{codered}{rgb}{0.5,0.03,0.02}
\definecolor{codegray}{rgb}{0.96,0.96,0.96}

\lstdefinestyle{pythonstylesheet}{
	language=Java,
	emphstyle=\color{deepred},
	backgroundcolor=\color{codegray},
	keywordstyle=\color{deepblue}\bfseries\itshape,
	numberstyle=\scriptsize\color{codenumber},
	basicstyle=\ttfamily\footnotesize,
	commentstyle=\color{codegreen}\itshape,
	breakatwhitespace=false, 
	breaklines=true, 
	captionpos=b, 
	keepspaces=true, 
	numbers=left, 
	numbersep=5pt, 
	showspaces=false,                
	showstringspaces=false,
	showtabs=false, 
	tabsize=4, 
	morekeywords={as,assert,nonlocal,with,yield,self,True,False,None,AssertionError,ValueError,in,else},              % Add keywords here
	keywordstyle=\color{codeblue},
	emph={var, List, Iterable, Iterator},          % Custom highlighting
	emphstyle=\color{codered},
	stringstyle=\color{codegreen},
	showstringspaces=false,
	abovecaptionskip=0pt,belowcaptionskip =0pt,
	framextopmargin=-\topsep, 
}
\newcommand\pythonstyle{\lstset{pythonstylesheet}}
\newcommand\pyl[1]     {{\lstinline!#1!}}
\lstset{style=pythonstylesheet}

\usepackage[style=1,skipbelow=\topskip,skipabove=\topskip,framemethod=TikZ]{mdframed}
\definecolor{bggray}{rgb}{0.85, 0.85, 0.85}
\mdfsetup{leftmargin=0pt,rightmargin=0pt,innerleftmargin=15pt,backgroundcolor=codegray,middlelinewidth=0.5pt,skipabove=5pt,skipbelow=0pt,middlelinecolor=black,roundcorner=5}
\BeforeBeginEnvironment{lstlisting}{\begin{mdframed}\vspace{-0.4em}}
	\AfterEndEnvironment{lstlisting}{\vspace{-0.8em}\end{mdframed}}


% Design
\usepackage[labelfont=bf]{caption}
\usepackage[margin=0.6in]{geometry}
\usepackage{multicol}
\usepackage[skip=4pt, indent=0pt]{parskip}
\usepackage[normalem]{ulem}
\forestset{default}
\renewcommand\labelitemi{$\bullet$}
\usepackage{titlesec}
\titleformat{\section}[block]
{\fontsize{15}{15}}
{\sen \dotfill (\thesection)\dotfill\she}
{0em}
{\MakeUppercase}
\usepackage{graphicx}
\graphicspath{ {./} }

\usepackage[colorlinks]{hyperref}
\definecolor{mgreen}{RGB}{25, 160, 50}
\definecolor{mblue}{RGB}{30, 60, 200}
\usepackage{hyperref}
\hypersetup{
	colorlinks=true,
	citecolor=mgreen,
	linkcolor=black,
	urlcolor=mblue,
	pdftitle={Document by Shahar Perets},
	%	pdfpagemode=FullScreen,
}
\usepackage{yfonts}
\def\gothstart#1{\noindent\smash{\lower3ex\hbox{\llap{\Huge\gothfamily#1}}}
	\parshape=3 3.1em \dimexpr\hsize-3.4em 3.4em \dimexpr\hsize-3.4em 0pt \hsize}
\def\frakstart#1{\noindent\smash{\lower3ex\hbox{\llap{\Huge\frakfamily#1}}}
	\parshape=3 1.5em \dimexpr\hsize-1.5em 2em \dimexpr\hsize-2em 0pt \hsize}



% Hebrew initialzing
\usepackage[bidi=basic]{babel}
\PassOptionsToPackage{no-math}{fontspec}
\babelprovide[main, import, Alph=letters]{hebrew}
\babelprovide[import]{english}
\babelfont[hebrew]{rm}{David CLM}
\babelfont[hebrew]{sf}{David CLM}
%\babelfont[english]{tt}{Monaspace Xenon}
\usepackage[shortlabels]{enumitem}
\newlist{hebenum}{enumerate}{1}

% Language Shortcuts
\newcommand\en[1] {\begin{otherlanguage}{english}#1\end{otherlanguage}}
\newcommand\he[1] {\she#1\sen}
\newcommand\sen   {\begin{otherlanguage}{english}}
	\newcommand\she   {\end{otherlanguage}}
\newcommand\del   {$ \!\! $}

\newcommand\npage {\vfil {\hfil \textbf{\textit{המשך בעמוד הבא}}} \hfil \vfil \pagebreak}
\newcommand\ndoc  {\dotfill \\ \vfil {\begin{center}
			{\textbf{\textit{שחר פרץ, 2025}} \\
				\scriptsize \textit{קומפל ב־}\en{\LaTeX}\,\textit{ ונוצר באמצעות תוכנה חופשית בלבד}}
	\end{center}} \vfil	}

\newcommand{\rn}[1]{
	\textup{\uppercase\expandafter{\romannumeral#1}}
}

\makeatletter
\newcommand{\skipitems}[1]{
	\addtocounter{\@enumctr}{#1}
}
\makeatother

%! ~~~ Math shortcuts ~~~

% Letters shortcuts
\newcommand\N     {\mathbb{N}}
\newcommand\Z     {\mathbb{Z}}
\newcommand\R     {\mathbb{R}}
\newcommand\Q     {\mathbb{Q}}
\newcommand\C     {\mathbb{C}}
\newcommand\One   {\mathit{1}}

\newcommand\ml    {\ell}
\newcommand\mj    {\jmath}
\newcommand\mi    {\imath}

\newcommand\powerset {\mathcal{P}}
\newcommand\ps    {\mathcal{P}}
\newcommand\pc    {\mathcal{P}}
\newcommand\ac    {\mathcal{A}}
\newcommand\bc    {\mathcal{B}}
\newcommand\cc    {\mathcal{C}}
\newcommand\dc    {\mathcal{D}}
\newcommand\ec    {\mathcal{E}}
\newcommand\fc    {\mathcal{F}}
\newcommand\nc    {\mathcal{N}}
\newcommand\vc    {\mathcal{V}} % Vance
\newcommand\sca   {\mathcal{S}} % \sc is already definded
\newcommand\rca   {\mathcal{R}} % \rc is already definded
\newcommand\zc    {\mathcal{Z}}

\newcommand\prm   {\mathrm{p}}
\newcommand\arm   {\mathrm{a}} % x86
\newcommand\brm   {\mathrm{b}}
\newcommand\crm   {\mathrm{c}}
\newcommand\drm   {\mathrm{d}}
\newcommand\erm   {\mathrm{e}}
\newcommand\frm   {\mathrm{f}}
\newcommand\nrm   {\mathrm{n}}
\newcommand\vrm   {\mathrm{v}}
\newcommand\srm   {\mathrm{s}}
\newcommand\rrm   {\mathrm{r}}

\newcommand\Si    {\Sigma}

% Logic & sets shorcuts
\newcommand\siff  {\longleftrightarrow}
\newcommand\ssiff {\leftrightarrow}
\newcommand\so    {\longrightarrow}
\newcommand\sso   {\rightarrow}

\newcommand\epsi  {\epsilon}
\newcommand\vepsi {\varepsilon}
\newcommand\vphi  {\varphi}
\newcommand\Neven {\N_{\mathrm{even}}}
\newcommand\Nodd  {\N_{\mathrm{odd }}}
\newcommand\Zeven {\Z_{\mathrm{even}}}
\newcommand\Zodd  {\Z_{\mathrm{odd }}}
\newcommand\Np    {\N_+}

% Text Shortcuts
\newcommand\open  {\big(}
\newcommand\qopen {\quad\big(}
\newcommand\close {\big)}
\newcommand\also  {\mathrm{, }}
\newcommand\defis {\mathrm{ definitions}}
\newcommand\given {\mathrm{given }}
\newcommand\case  {\mathrm{if }}
\newcommand\syx   {\mathrm{ syntax}}
\newcommand\rle   {\mathrm{ rule}}
\newcommand\other {\mathrm{else}}
\newcommand\set   {\ell et \text{ }}
\newcommand\ans   {\mathscr{A}\!\mathit{nswer}}

% Set theory shortcuts
\newcommand\ra    {\rangle}
\newcommand\la    {\langle}

\newcommand\oto   {\leftarrow}

\newcommand\QED   {\quad\quad\mathscr{Q.E.D.}\;\;\blacksquare}
\newcommand\QEF   {\quad\quad\mathscr{Q.E.F.}}
\newcommand\eQED  {\mathscr{Q.E.D.}\;\;\blacksquare}
\newcommand\eQEF  {\mathscr{Q.E.F.}}
\newcommand\jQED  {\mathscr{Q.E.D.}}

\DeclareMathOperator\dom   {dom}
\DeclareMathOperator\Img   {Im}
\DeclareMathOperator\range {range}

\newcommand\trio  {\triangle}

\newcommand\rc    {\right\rceil}
\newcommand\lc    {\left\lceil}
\newcommand\rf    {\right\rfloor}
\newcommand\lf    {\left\lfloor}
\newcommand\ceil  [1] {\lc #1 \rc}
\newcommand\floor [1] {\lf #1 \rf}

\newcommand\lex   {<_{lex}}

\newcommand\az    {\aleph_0}
\newcommand\uaz   {^{\aleph_0}}
\newcommand\al    {\aleph}
\newcommand\ual   {^\aleph}
\newcommand\taz   {2^{\aleph_0}}
\newcommand\utaz  { ^{\left (2^{\aleph_0} \right )}}
\newcommand\tal   {2^{\aleph}}
\newcommand\utal  { ^{\left (2^{\aleph} \right )}}
\newcommand\ttaz  {2^{\left (2^{\aleph_0}\right )}}

\newcommand\n     {$n$־יה\ }

% Math A&B shortcuts
\newcommand\logn  {\log n}
\newcommand\logx  {\log x}
\newcommand\lnx   {\ln x}
\newcommand\cosx  {\cos x}
\newcommand\sinx  {\sin x}
\newcommand\sint  {\sin \theta}
\newcommand\tanx  {\tan x}
\newcommand\tant  {\tan \theta}
\newcommand\sex   {\sec x}
\newcommand\sect  {\sec^2}
\newcommand\cotx  {\cot x}
\newcommand\cscx  {\csc x}
\newcommand\sinhx {\sinh x}
\newcommand\coshx {\cosh x}
\newcommand\tanhx {\tanh x}

\newcommand\seq   {\overset{!}{=}}
\newcommand\slh   {\overset{LH}{=}}
\newcommand\sle   {\overset{!}{\le}}
\newcommand\sge   {\overset{!}{\ge}}
\newcommand\sll   {\overset{!}{<}}
\newcommand\sgg   {\overset{!}{>}}

\newcommand\h     {\hat}
\newcommand\ve    {\vec}
\newcommand\lv    {\overrightarrow}
\newcommand\ol    {\overline}

\newcommand\mlcm  {\mathrm{lcm}}

\DeclareMathOperator{\sech}   {sech}
\DeclareMathOperator{\csch}   {csch}
\DeclareMathOperator{\arcsec} {arcsec}
\DeclareMathOperator{\arccot} {arcCot}
\DeclareMathOperator{\arccsc} {arcCsc}
\DeclareMathOperator{\arccosh}{arccosh}
\DeclareMathOperator{\arcsinh}{arcsinh}
\DeclareMathOperator{\arctanh}{arctanh}
\DeclareMathOperator{\arcsech}{arcsech}
\DeclareMathOperator{\arccsch}{arccsch}
\DeclareMathOperator{\arccoth}{arccoth}
\DeclareMathOperator{\atant}  {atan2} 
\DeclareMathOperator{\Sp}     {span} 
\DeclareMathOperator{\sgn}    {sgn} 
\DeclareMathOperator{\row}    {Row} 
\DeclareMathOperator{\adj}    {adj} 
\DeclareMathOperator{\rk}     {rank} 
\DeclareMathOperator{\col}    {Col} 
\DeclareMathOperator{\tr}     {tr}

\newcommand\dx    {\,\mathrm{d}x}
\newcommand\dt    {\,\mathrm{d}t}
\newcommand\dtt   {\,\mathrm{d}\theta}
\newcommand\du    {\,\mathrm{d}u}
\newcommand\dv    {\,\mathrm{d}v}
\newcommand\df    {\mathrm{d}f}
\newcommand\dfdx  {\diff{f}{x}}
\newcommand\dit   {\limhz \frac{f(x + h) - f(x)}{h}}

\newcommand\nt[1] {\frac{#1}{#1}}

\newcommand\limz  {\lim_{x \to 0}}
\newcommand\limxz {\lim_{x \to x_0}}
\newcommand\limi  {\lim_{x \to \infty}}
\newcommand\limh  {\lim_{x \to 0}}
\newcommand\limni {\lim_{x \to - \infty}}
\newcommand\limpmi{\lim_{x \to \pm \infty}}

\newcommand\ta    {\theta}
\newcommand\ap    {\alpha}

\renewcommand\inf {\infty}
\newcommand  \ninf{-\inf}

% Combinatorics shortcuts
\newcommand\sumnk     {\sum_{k = 0}^{n}}
\newcommand\sumni     {\sum_{i = 0}^{n}}
\newcommand\sumnko    {\sum_{k = 1}^{n}}
\newcommand\sumnio    {\sum_{i = 1}^{n}}
\newcommand\sumai     {\sum_{i = 1}^{n} A_i}
\newcommand\nsum[2]   {\reflectbox{\displaystyle\sum_{\reflectbox{\scriptsize$#1$}}^{\reflectbox{\scriptsize$#2$}}}}

\newcommand\bink      {\binom{n}{k}}
\newcommand\setn      {\{a_i\}^{2n}_{i = 1}}
\newcommand\setc[1]   {\{a_i\}^{#1}_{i = 1}}

\newcommand\cupain    {\bigcup_{i = 1}^{n} A_i}
\newcommand\cupai[1]  {\bigcup_{i = 1}^{#1} A_i}
\newcommand\cupiiai   {\bigcup_{i \in I} A_i}
\newcommand\capain    {\bigcap_{i = 1}^{n} A_i}
\newcommand\capai[1]  {\bigcap_{i = 1}^{#1} A_i}
\newcommand\capiiai   {\bigcap_{i \in I} A_i}

\newcommand\xot       {x_{1, 2}}
\newcommand\ano       {a_{n - 1}}
\newcommand\ant       {a_{n - 2}}

% Linear Algebra
\DeclareMathOperator{\chr}     {char}
\DeclareMathOperator{\diag}    {diag}
\DeclareMathOperator{\Hom}     {Hom}
\DeclareMathOperator{\Sym}     {Sym}
\DeclareMathOperator{\Asym}    {ASym}

\newcommand\lra       {\leftrightarrow}
\newcommand\chrf      {\chr(\F)}
\newcommand\F         {\mathbb{F}}
\newcommand\co        {\colon}
\newcommand\tmat[2]   {\cl{\begin{matrix}
			#1
		\end{matrix}\, \middle\vert\, \begin{matrix}
			#2
\end{matrix}}}

\makeatletter
\newcommand\rrr[1]    {\xxrightarrow{1}{#1}}
\newcommand\rrt[2]    {\xxrightarrow{1}[#2]{#1}}
\newcommand\mat[2]    {M_{#1\times#2}}
\newcommand\gmat      {\mat{m}{n}(\F)}
\newcommand\tomat     {\, \dequad \longrightarrow}
\newcommand\pms[1]    {\begin{pmatrix}
		#1
\end{pmatrix}}

\newcommand\norm[1]   {\left \vert \left \vert #1 \right \vert \right \vert}
\newcommand\snorm     {\left \vert \left \vert \cdot \right \vert \right \vert}
\newcommand\smut      {\left \la \cdot \mid \cdot \right \ra}
\newcommand\mut[2]    {\left \la #1 \,\middle\vert\, #2 \right \ra}

% someone's code from the internet: https://tex.stackexchange.com/questions/27545/custom-length-arrows-text-over-and-under
\makeatletter
\newlength\min@xx
\newcommand*\xxrightarrow[1]{\begingroup
	\settowidth\min@xx{$\m@th\scriptstyle#1$}
	\@xxrightarrow}
\newcommand*\@xxrightarrow[2][]{
	\sbox8{$\m@th\scriptstyle#1$}  % subscript
	\ifdim\wd8>\min@xx \min@xx=\wd8 \fi
	\sbox8{$\m@th\scriptstyle#2$} % superscript
	\ifdim\wd8>\min@xx \min@xx=\wd8 \fi
	\xrightarrow[{\mathmakebox[\min@xx]{\scriptstyle#1}}]
	{\mathmakebox[\min@xx]{\scriptstyle#2}}
	\endgroup}
\makeatother


% Greek Letters
\newcommand\ag        {\alpha}
\newcommand\bg        {\beta}
\newcommand\cg        {\gamma}
\newcommand\dg        {\delta}
\newcommand\eg        {\epsi}
\newcommand\zg        {\zeta}
\newcommand\hg        {\eta}
\newcommand\tg        {\theta}
\newcommand\ig        {\iota}
\newcommand\kg        {\keppa}
\renewcommand\lg      {\lambda}
\newcommand\og        {\omicron}
\newcommand\rg        {\rho}
\newcommand\sg        {\sigma}
\newcommand\yg        {\usilon}
\newcommand\wg        {\omega}

\newcommand\Ag        {\Alpha}
\newcommand\Bg        {\Beta}
\newcommand\Cg        {\Gamma}
\newcommand\Dg        {\Delta}
\newcommand\Eg        {\Epsi}
\newcommand\Zg        {\Zeta}
\newcommand\Hg        {\Eta}
\newcommand\Tg        {\Theta}
\newcommand\Ig        {\Iota}
\newcommand\Kg        {\Keppa}
\newcommand\Lg        {\Lambda}
\newcommand\Og        {\Omicron}
\newcommand\Rg        {\Rho}
\newcommand\Sg        {\Sigma}
\newcommand\Yg        {\Usilon}
\newcommand\Wg        {\Omega}

% Other shortcuts
\newcommand\tl    {\tilde}
\newcommand\op    {^{-1}}

\newcommand\sof[1]    {\left | #1 \right |}
\newcommand\cl [1]    {\left ( #1 \right )}
\newcommand\csb[1]    {\left [ #1 \right ]}
\newcommand\ccb[1]    {\left \{ #1 \right \}}

\newcommand\bs        {\blacksquare}
\newcommand\dequad    {\!\!\!\!\!\!}
\newcommand\dequadd   {\dequad\duquad}

\renewcommand\phi     {\varphi}

\newtheorem{Theorem}{משפט}
\theoremstyle{definition}
\newtheorem{definition}{הגדרה}
\newtheorem{Lemma}{למה}
\newtheorem{Remark}{הערה}
\newtheorem{Notion}{סימון}


\newcommand\theo  [1] {\begin{Theorem}#1\end{Theorem}}
\newcommand\defi  [1] {\begin{definition}#1\end{definition}}
\newcommand\rmark [1] {\begin{Remark}#1\end{Remark}}
\newcommand\lem   [1] {\begin{Lemma}#1\end{Lemma}}
\newcommand\noti  [1] {\begin{Notion}#1\end{Notion}}

% DS
\newcommand\limsi     {\limsup_{n \to \inf}}
\newcommand\limfi     {\liminf_{n \to \inf}}

\DeclareMathOperator\amort   {amort}
\DeclareMathOperator\worst   {worst}
\DeclareMathOperator\type    {type}
\DeclareMathOperator\cost    {cost}
\DeclareMathOperator\tim     {time}

\newcommand\dsList{
	\sFunc{List}
	\sFunc{Retrieve}
	\SetKwFunction{RetrieveFirst}{Retrieve-First}
	\SetKwFunction{RetrieveLast}{Retrieve-Last}
	\sFunc{Delete}
	\SetKwFunction{DeleteFirst}{Delete-First}
	\SetKwFunction{DeleteLast}{Delete-Last}
	\sFunc{Insert}
	\SetKwFunction{InsertFirst}{Insert-First}
	\SetKwFunction{InsertLast}{Insert-Last}
	\sFunc{Shift}
	\sFunc{Length}
	\sFunc{Concat}
	\sFunc{Plant}
	\sFunc{Split}
}
\newcommand\dsQueue{
	\sFunc{Queue}
	\sFunc{Enqueue}
	\sFunc{Head}
	\sFunc{Dequeue}
}
\newcommand\dsStack{
	\sFunc{Stack}
	\sFunc{Push}
	\sFunc{Top}
	\sFunc{Pop}
}
\newcommand\dsVector{
	\sFunc{Vector}
	\sFunc{Get}
	\sFunc{Set}
}
\newcommand\dsGraph{
	\sFunc{Graph}
	\sFunc{Edge}
	\SetKwFunction{AddEdge}{Add-Edge}
	\SetKwFunction{RemoveEdge}{Remove-Edge}
	\sFunc{InDeg} \sFunc{OutDeg}
}
\newcommand\importDs{
	\dsList
	\dsQueue
	\dsStack
	\dsVector
	\dsGraph
	\SetKwProg{Fn}{function}{ is}{end}
	\SetKwData{error}{\color{codered}error}
	\SetKwInOut{Input}{input}
	\SetKwInOut{Output}{output}
	\SetKwRepeat{Do}{do}{while}
	\SetKwData{Null}{\color{codegreen}null}
	\SetKwData{True}{\color{codeblue}true}
	\SetKwData{False}{\color{codeblue}false}
}


% Algorithems
\newcommand\sFunc [1] {\SetKwFunction{#1}{#1}}
\newcommand\sData [1] {\SetKwData{#1}{#1}}
\newcommand\sIO   [1] {\SetKwInOut{#1}{#1}}
\newcommand\ttt   [1] {\sen \texttt{#1} \she\,}
\newcommand\io    [2] {\Input{#1}\Output{#2}\BlankLine}

%! ~~~ Document ~~~

\usepackage{etoolbox}
\AtBeginEnvironment{align}{\setcounter{equation}{0}} 
\newcommand\lilyarr {\hookleftarrow}

\author{שחר פרץ}
\title{\textit{תרגול אחרון}}
\begin{document}
	\maketitle
	\textbf{מתרגלת: }לילי
	
	\section{}
	2022AA(1)
	
	יהי $\F$ שדה ותהי $T \co \F^3 \to \F^2$ המוגדרת ע''י $T(x, y, z) = (x + zx, x + y + z)$. מצאו את כל הבסיסים מהצורה: 
	\[ B = \{b_1, b_2, b_3\} := \ccb{\pms{a \\ b \\c}  , \pms{1 \\ 1 \\ 1}, \pms{0 \\ 0 \\1}}, \ C = \{c_1, c_2\} := \ccb{\pms{n_1 \\ m_1}, \pms{n_2 \\ m_2}} \]
	כך ש־: 
	\[ [T]^{B}_C = \pms{1 & 3 & 0 \\ 1 & 0 & 1} \]
	
	\textbf{פתרון. }
	\begin{itemize}
		\item נמצא תנאים כך ש־$[T]^{B}_C = \cdots$
		\item נמצא תנאים כך ש־$B, C$ בסיסים
	\end{itemize}
	מהגדרת מטריצה מייצגת: 
	\sen\begin{align}
		T(b_1) =& c_1 + c_2 \\ T(b_2) =& 3c_1 \\ T(b_3) =& c_2
	\end{align}\she
	
	נציב $b_i, c_i$	ונכתוב את זה כמערכת משוואות: 
	\sen\begin{align}
		\text{useless} \\ 
		(1 + 2 \cdot 1, 1 + 1 + 1) = T(b_2) = 3c_1 = (3n_1, 3m_1) \\
		(2, 1) = (n_2, m_2)
	\end{align}\she
	\[ \implies n_1 = 1, \ m_1 = 1, \ n_2 = 2, \ m_2 = 1 \]
	נבחין ונוודא ש־$C$ אכן בסיס. מכאן: 
	\[ (a + 2c, a + b + c) = (1, 1) + (2, 1) = (3, 2) \]
	זה מותיר אותנו עם מערכת משוואות ממנה אפשר למצוא את $a, b, c$. כמובן שיהיה לפחות משתנה חופשי אחד. 
	\[ \begin{cases}
		1a + 0b + 2c = 3 \\
		1a + 1b + 1c = 2
	\end{cases} \dequad\implies \{(\underbrace{3 - 2c}_a, \underbrace{c - 1}_{b}, \underbrace{c}_{c}) \mid c \in \F\} \]
	\textit{הערה: }אנחנו בשדה שרירותי. לכן $3:= 1 + 1 + 1, \ 2 = 1 + 1$ וכו'. הזהרו מחלוקה בהם כי יכול להיות ש־$2 = 0$. 
	נעבור לתנאי 2 ונדרוש ש־$B$ בסיס. משום שאלו $3$ וקטורים די להראות שזו קבוצה פורשת. 
	יש אלגו' סטנדרטי לבדוק את זה. נשים את $b_1 \dots b_3$ בשורות מטריצה: 
	\[ \pms{3 - 2c & c - 1 & c \\ 0  & 0 & 1 \\ 1 & 1 & 1} \to \pms{1 &1 &1 \\ 0 & 3c - 4 & 3c - 3 & 0 & 0 &1} \]
	אם $3c - 4  =0$ אז הקבוצה ת''ל ואחרת המטריצה מדורגת. 
	מסקנה: 
	\[ B = \{(3 - 2c, c - 1, c), (1, 1, 1), (0, 0, 1)\} \ C = \{(1, 1), (2, 1)\} \]
	מקיימים את התנאי םלהלן לכל $c \neq \frac{4}{3}$ (צריך להניח משהו על המציין של השדה כדי שיהיה אפשר לחלק ב־$3$, אז צריך גם לפצל למקרים בהתאם למקדם). 
	
	\section{}
	יהיו $0 \le a, b, c, d \in \R$ כך שבדיוק אחד מהם הוא $0$. הוכיחו: 
	\[ \det\pms{a & -1 & -1 & -1 \\ -1 & b & -1 & -1 \\ -1 & -1 & c & -1 \\ -1 & -1 & -1 & d} < 0 \]
	\textbf{פתרון. }
	בה''כ $a  = 0$ והשאר חיוביים. אפשר להניח את זה כי אפשר להחליף שורה ואז עמודה ולהגיע לאותו המקום (וזה פעמיים כפל ב־$-1$ שמוביל אותנו לאותה הדטרמיננטה בכל מקרה). לדוגמה, אם $c = 0$ נחליף $R_1 \lra R_3$ ו־$C_1 \lra C_3$, נקבל שני פקטורים של $-1$ ולכן ה־$\det$ לא ישתנה. נוריד מכל העמודות את $c_1$ ונקבל: 
	\[ \pms{0 & -1 & -1 & -1 \\ -1 & b + 1 & 0 & 0 \\ -1 & 0 & c + 1 & 0 \\ -1 & 0 & 0 & d + 1} \]
	ראינו כבר בתרגול על תמורות איך נראית דטרמיננטה של דבר כזה. אם לא זוכרים את הנוסחה, אפשר להוריד מהעמודה הראשונה את $\frac{1}{b + 1}c_2$, $\frac{-1}{c + 1}c_3$, $\frac{-c_4}{d + 1}$. מותר לעשות זאת כי $0 \le b, c, d$. נקבל: 
	\[ \pms{-\frac{1}{b + 1} - \frac{1}{c + 1} - \frac{1}{d + 1} & - 1 & -1 & -1 \\ 0 & b + 1 & 0 & 0 \\ 0 & 0 & c + 1 & 0 \\ 0 & 0 & 0 & d + 1} \]
	סה''כ מכפלת איברי האלכסון של המשולשית היא: 
	\[ \det = -(b + 1)(c + 1)(d + 1)\cl{(b + 1)\op + (c + 1)\op + (d + 1)\op} \]
	כדרוש. 
	
	\section{}
	יהיו $V, W$ מ''ו נ''ס מעל $\F$. נניח כי $T \co V \to W$ העתקה על המקיימת לכל $K \subseteq V$ סופית, אם $\Sp(T(k)) = W$ אז $V = \Sp(k)$. הוכיחו ש־$T$ חח''ע. 
	
	\textbf{פתרון. }נניח בשלילה שהיא איננה חח''ע. אז ישנו $0 \neq v_1 \in \ker T$. נשלים אותו לבסיס $B$ של $V$. הבסיס הזה הוא קבוצה סופית. ממשפט מההרצאה מתאיים: 
	\[ \underbrace{\Img T}_W = \Sp(T(B)) = \Sp(\underbrace{Tv_1}_{0} \dots Tv_n) \implies \Sp(B \setminus \{b_1\}) = V \]
	בסתירה לכך ש־$B$ בסיס, כלומר פורש מינימלי. הגרירה נכונה מההנחה של השאלה. 
	
	\section{}
	יהי $\F$ שדה, וכן $n \ge 2$ ותהי $A \in M_n(\F)$. נגדיר העתקה לינארית $T \co M_n(\F) \lilyarr$ לפי $T(B) = A^TB + B^TA$. הראו ש־$T$ לא חח''ע. 
	
	\textbf{פתרון. }נבחין ש־$T(B)$ סימטרית: 
	\[ (T(B))^{T} = (A^TB + B^TA)^{T} = B^T(A^T)^T + A^T(B^T)^T = B^TA + A^TB = T(B) \]
	לכן $\Img T \subsetneq \Sym(M_n(\F)) \subseteq M_n(\F)$. לכן  $\dim M_n(\F) < \dim \Img T$ כי $\dim M_n(\F) = n^2, \ \dim \Sym_n = \frac{n^2 + n}{2}$. סה''כ: 
	\[ \underbrace{\dim M_n(\F)}_{n^2} = \underbrace{\dim \Img T}_{< n^2} + \underbrace{\dim \ker T}_{k } \]
	לכן $\dim \ker T > 0$ כדרוש. 
	
	אלטרנטיבית אפשר לומר ש־$T$ לא על והעתקה ממרחב לעצמו היא על אמ''מ היא חח''ע. 
	
	\section{}
	יהי $V$ נ''ס מעל $\C$ ויהיו $T, S \co V \lilyarr$ כך שמתקיים $5T^4 - 2T^2 + 3TS + T - I = 0$. 
	
	לול אני מכיר את השאלה הזו. 
	
	\textbf{פתרון. }אם $S$ היא פולינום ב־$T, T\op$ אז הטענה תתקיים. לדוגמה $ ST = (T^3 + T\op - 2T)T = T(T^3 + T\op - 2T)$. נרצה אם כך לבודד את $S$ בצד אחד של המשוואה. נניח $T$ הפיכה: 
	\[ S = -\frac{1}{3}T\op (5T^4  - 2T^2 + T - I) \]
	ואז $TS = ST$. זאת כי: 
	\[ T\cl{\sumnio \ag_i T^{i}} = \cl{\sumnio \ag_i T^i}T \]
	 עתה נוכיח ש־$T$ הפיכה. נעביר את $I$ אגף: 
	\[ 5T^4 - 2T^2 + 3TS + T = I \implies T(5T^3 - 2T + 3S) = I \]
	
	\section{}
	תהי $A \in M_5(\Z)$ (לא שדה, אך עדיין אפשר להתבונן במטריצות מעליהם) כך שכל שורה ועמודה שלהם מכילה בדיוק את המספרים $1, 2 \dots 5$. למשל: 
	\[ \pms{1 & 2 & 3 & 4 & 5 \\ 5 & 1 &2 &3 &4 \\ 4 &5 & 1 & 2 &3 \\ 3 & 4 & 5 & 1 & 2\\ 2 & 3 & 4 &5 & 1} \]
	עתה צריך להראות ש־$\det A$ מתחלקת ב־$75$. 
	
	\textbf{פתרון. }נוסיף לשורה הראשונה את כל השאר: 
	\[ \pms{15 & 15 & \cdots & 15 & 15 \\ * & * & * & * & * \\ * \\ * \\ *} = 15 \det\pms{1 & 1 & 1 & 1 & 1\\ * & * & * & * & * \\ * \\ * \\ *} \]
	עתה נוסיף לעמודה הראשונה את כל השאר: 
	\[ 15\det\pms{5 & 1 & 1 & 1 & 1 \\ 15 & * & * & * & * \\ \vdots & \vdots & \ddots & \ddots & \vdots \\ \vdots & \vdots & \ddots & \ddots & \vdots \\ 15 & * & * & * & *} = 15 \cdot 5 \det \det\pms{1 & 1 & 1 & 1 & 1 \\ 3 & * & * & * & * \\ \vdots & \vdots & \ddots & \ddots & \vdots \\ \vdots & \vdots & \ddots & \ddots & \vdots \\ 3 & * & * & * & *} \]
	ואכן $15 \cdot 5 = 75$ מחלק את הנדרש. 
	
	\npage
	\section{}
	יהי $V$ מ''ו נ''ס מעל $\C$ ויהיו $T, S \co V \lilyarr$ ט''ל כך שמתקיים: 
	\[ V = \Img T + \Img S = \ker T + \ker S \]
	הראו שהסכומים להלן ישרים. 
	
	\textbf{פתרון. }יהיו הכי קל לעבוד עם ממדים במקום לעבוד ישירות ולהראות שהחיתוך אפס. 
	\[ n = \dim (\Img T + \Img S) = \dim (\ker T + \ker S) \]
	\[ \dim \Img T + \dim \Img S = n + \dim(\Img T \cap \Img S) \]
	\[ \dim V = \dim \ker T + \dim \ker S = n + \dim (\ker T + \ker S) \]
	ממשפט הממדים להעתקות: 
	\[ \dim \Img T + \dim \ker T = \dim \Img S + \dim \ker S = n \]
	נסכום את הכל ונקבל: 
	\[ \underbrace{(\dim \Img T + \dim \ker T)}_n + \underbrace{(\dim \Img S + \dim \ker S)}_n - \dim (\Img T \cap \Img S) - \dim (\ker T \cap \ker S) = n + n \]
	סה''כ: 
	\[ -\underbrace{\dim(\Img T \cap \Img S)}_{\ge 0} - \underbrace{\dim(\ker T \cap \ker S)}_{\ge 0} = 0 \]
	לכן: 
	\[ \dim (\Img T \cap \Img S) = 0 \land \dim (\ker T + \ker S) = 0 \]
	כדרוש. 
	
	\textbf{דרך אלטרנטיבית להוכחה. }
	\[ n = \dim V = \underbrace{\Img T}_k + \underbrace{\Img S}_{\ml} = \underbrace{\ker T}_{n - k} + \underbrace{\ker S}_{n- \ml} \]
	ואז לקבל: 
	\[ n \le k + \ml, \ n \le n- l + n - \ml = 2n - (l + \ml) \implies n \ge k + \ml \]
	וברגע שסכום של שני מרחבים נותן משהו ממד $n$, בהכרח הסכום סכום ישר (הערה שלי: זו טענה ידועה אבל אפשר להוכיח את זה באמצעות פיתוח בסיסים). זו הוכחה שקולה. 
	
	\section{}
	יהי $V$ מ''ו נ''ס ויהיו $U_1, U_2, U_3$ תת''מ של $V$. הוכיחו: 
	\[ \dim(U_1 \cap U_2 \cap U_3) \ge \underbrace{\dim U_1}_{n_1} + \underbrace{\dim U_2}_{n_2} + \underbrace{\dim U_3}_{n_3} -2\underbrace{\dim V}_{n} \]
	\textbf{פתרון. }אנו יודעים שעבור זוג מרחבים: 
	\[ \dim(U_1 \cap U_2) = \dim U_1 + \dim U_2 - \dim (U_1 \cap U_2) \ge n_1 + n_2 - n \]
	\[ \dim ((U_1 \cap U_2) \cap U_3) \ge \dim (U_1 \cap U_2) + n_3 - n \ge n_1 + n_2 + n_3 - 2n \]
	כדרוש. 
	
	\section{}
	יהי $V$ מ''ו נ''ס מעל $\F$ ויהיו $T, S \lilyarr$ ט''ל. נוכיח: 
	\[ T(\ker(S \circ T)) = \Img T \cap \ker S \]
	
	\textbf{פתרון. }
	
	\begin{itemize}
		\item[$\subseteq$]מכיוון אחד, ברור ש־$T(\ker(S \circ T)) \subseteq \Img T$. יהי $v \in V$ אז ישנו $w \in \ker S \circ T$ כך ש־$v = T(w)$. 
		ולכן: 
		\[ S(v) = S(T(w)) =  \]
		אוקי היא העבירה דף ולא הספקתי להעתיק את השוויון האחרון. 
		\item[\reflectbox{$\subseteq$}]יהי $v \in \Img T \cap \ker S$. מהגדרת $\Img T$ ישנו $w \in V$ כך ש־$v = T(w)$. מהגדרת $\ker S$ מתקיים $0 = S(v) = S(T(w)) \implies w \in \ker (S \circ T)$. נפעיל את $T$ על המשוואה ונקבל $T(w) \in T(\ker (S \circ T))$. 
	\end{itemize}
	סיימנו את ההכלה הדו־כיוונית כדרוש. 
	
	
	
	
	
	
	
	\ndoc
\end{document}