%! ~~~ Packages Setup ~~~ 
\documentclass[]{article}
\usepackage{lipsum}
\usepackage{rotating}


% Math packages
\usepackage[usenames]{color}
\usepackage{forest}
\usepackage{ifxetex,ifluatex,amssymb,amsmath,mathrsfs,amsthm,witharrows,mathtools,mathdots}
\usepackage{amsmath}
\WithArrowsOptions{displaystyle}
\renewcommand{\qedsymbol}{$\blacksquare$} % end proofs with \blacksquare. Overwrites the defualts. 
\usepackage{cancel,bm}
\usepackage[thinc]{esdiff}


% tikz
\usepackage{tikz}
\usetikzlibrary{graphs}
\newcommand\sqw{1}
\newcommand\squ[4][1]{\fill[#4] (#2*\sqw,#3*\sqw) rectangle +(#1*\sqw,#1*\sqw);}


% code 
\usepackage{algorithm2e}
\usepackage{listings}
\usepackage{xcolor}

\definecolor{codegreen}{rgb}{0,0.35,0}
\definecolor{codegray}{rgb}{0.5,0.5,0.5}
\definecolor{codenumber}{rgb}{0.1,0.3,0.5}
\definecolor{codeblue}{rgb}{0,0,0.5}
\definecolor{codered}{rgb}{0.5,0.03,0.02}
\definecolor{codegray}{rgb}{0.96,0.96,0.96}

\lstdefinestyle{pythonstylesheet}{
	language=Java,
	emphstyle=\color{deepred},
	backgroundcolor=\color{codegray},
	keywordstyle=\color{deepblue}\bfseries\itshape,
	numberstyle=\scriptsize\color{codenumber},
	basicstyle=\ttfamily\footnotesize,
	commentstyle=\color{codegreen}\itshape,
	breakatwhitespace=false, 
	breaklines=true, 
	captionpos=b, 
	keepspaces=true, 
	numbers=left, 
	numbersep=5pt, 
	showspaces=false,                
	showstringspaces=false,
	showtabs=false, 
	tabsize=4, 
	morekeywords={as,assert,nonlocal,with,yield,self,True,False,None,AssertionError,ValueError,in,else},              % Add keywords here
	keywordstyle=\color{codeblue},
	emph={var, List, Iterable, Iterator},          % Custom highlighting
	emphstyle=\color{codered},
	stringstyle=\color{codegreen},
	showstringspaces=false,
	abovecaptionskip=0pt,belowcaptionskip =0pt,
	framextopmargin=-\topsep, 
}
\newcommand\pythonstyle{\lstset{pythonstylesheet}}
\newcommand\pyl[1]     {{\lstinline!#1!}}
\lstset{style=pythonstylesheet}

\usepackage[style=1,skipbelow=\topskip,skipabove=\topskip,framemethod=TikZ]{mdframed}
\definecolor{bggray}{rgb}{0.85, 0.85, 0.85}
\mdfsetup{leftmargin=0pt,rightmargin=0pt,innerleftmargin=15pt,backgroundcolor=codegray,middlelinewidth=0.5pt,skipabove=5pt,skipbelow=0pt,middlelinecolor=black,roundcorner=5}
\BeforeBeginEnvironment{lstlisting}{\begin{mdframed}\vspace{-0.4em}}
	\AfterEndEnvironment{lstlisting}{\vspace{-0.8em}\end{mdframed}}


% Design
\usepackage[labelfont=bf]{caption}
\usepackage[margin=0.6in]{geometry}
\usepackage{multicol}
\usepackage[skip=4pt, indent=0pt]{parskip}
\usepackage[normalem]{ulem}
\forestset{default}
\renewcommand\labelitemi{$\bullet$}
\usepackage{titlesec}
\titleformat{\section}[block]
{\fontsize{15}{15}}
{\sen \dotfill (\thesection)\dotfill\she}
{0em}
{\MakeUppercase}
\usepackage{graphicx}
\graphicspath{ {./} }

\usepackage[colorlinks]{hyperref}
\definecolor{mgreen}{RGB}{25, 160, 50}
\definecolor{mblue}{RGB}{30, 60, 200}
\usepackage{hyperref}
\hypersetup{
	colorlinks=true,
	citecolor=mgreen,
	linkcolor=black,
	urlcolor=mblue,
	pdftitle={Document by Shahar Perets},
	%	pdfpagemode=FullScreen,
}
\usepackage{yfonts}
\def\gothstart#1{\noindent\smash{\lower3ex\hbox{\llap{\Huge\gothfamily#1}}}
	\parshape=3 3.1em \dimexpr\hsize-3.4em 3.4em \dimexpr\hsize-3.4em 0pt \hsize}
\def\frakstart#1{\noindent\smash{\lower3ex\hbox{\llap{\Huge\frakfamily#1}}}
	\parshape=3 1.5em \dimexpr\hsize-1.5em 2em \dimexpr\hsize-2em 0pt \hsize}



% Hebrew initialzing
\usepackage[bidi=basic]{babel}
\PassOptionsToPackage{no-math}{fontspec}
\babelprovide[main, import, Alph=letters]{hebrew}
\babelprovide[import]{english}
\babelfont[hebrew]{rm}{David CLM}
\babelfont[hebrew]{sf}{David CLM}
%\babelfont[english]{tt}{Monaspace Xenon}
\usepackage[shortlabels]{enumitem}
\newlist{hebenum}{enumerate}{1}

% Language Shortcuts
\newcommand\en[1] {\begin{otherlanguage}{english}#1\end{otherlanguage}}
\newcommand\he[1] {\she#1\sen}
\newcommand\sen   {\begin{otherlanguage}{english}}
	\newcommand\she   {\end{otherlanguage}}
\newcommand\del   {$ \!\! $}

\newcommand\npage {\vfil {\hfil \textbf{\textit{המשך בעמוד הבא}}} \hfil \vfil \pagebreak}
\newcommand\ndoc  {\dotfill \\ \vfil {\begin{center}
			{\textbf{\textit{שחר פרץ, 2025}} \\
				\scriptsize \textit{קומפל ב־}\en{\LaTeX}\,\textit{ ונוצר באמצעות תוכנה חופשית בלבד}}
	\end{center}} \vfil	}

\newcommand{\rn}[1]{
	\textup{\uppercase\expandafter{\romannumeral#1}}
}

\makeatletter
\newcommand{\skipitems}[1]{
	\addtocounter{\@enumctr}{#1}
}
\makeatother

%! ~~~ Math shortcuts ~~~

% Letters shortcuts
\newcommand\N     {\mathbb{N}}
\newcommand\Z     {\mathbb{Z}}
\newcommand\R     {\mathbb{R}}
\newcommand\Q     {\mathbb{Q}}
\newcommand\C     {\mathbb{C}}
\newcommand\One   {\mathit{1}}

\newcommand\ml    {\ell}
\newcommand\mj    {\jmath}
\newcommand\mi    {\imath}

\newcommand\powerset {\mathcal{P}}
\newcommand\ps    {\mathcal{P}}
\newcommand\pc    {\mathcal{P}}
\newcommand\ac    {\mathcal{A}}
\newcommand\bc    {\mathcal{B}}
\newcommand\cc    {\mathcal{C}}
\newcommand\dc    {\mathcal{D}}
\newcommand\ec    {\mathcal{E}}
\newcommand\fc    {\mathcal{F}}
\newcommand\nc    {\mathcal{N}}
\newcommand\vc    {\mathcal{V}} % Vance
\newcommand\sca   {\mathcal{S}} % \sc is already definded
\newcommand\rca   {\mathcal{R}} % \rc is already definded
\newcommand\zc    {\mathcal{Z}}

\newcommand\prm   {\mathrm{p}}
\newcommand\arm   {\mathrm{a}} % x86
\newcommand\brm   {\mathrm{b}}
\newcommand\crm   {\mathrm{c}}
\newcommand\drm   {\mathrm{d}}
\newcommand\erm   {\mathrm{e}}
\newcommand\frm   {\mathrm{f}}
\newcommand\nrm   {\mathrm{n}}
\newcommand\vrm   {\mathrm{v}}
\newcommand\srm   {\mathrm{s}}
\newcommand\rrm   {\mathrm{r}}

\newcommand\Si    {\Sigma}

% Logic & sets shorcuts
\newcommand\siff  {\longleftrightarrow}
\newcommand\ssiff {\leftrightarrow}
\newcommand\so    {\longrightarrow}
\newcommand\sso   {\rightarrow}

\newcommand\epsi  {\epsilon}
\newcommand\vepsi {\varepsilon}
\newcommand\vphi  {\varphi}
\newcommand\Neven {\N_{\mathrm{even}}}
\newcommand\Nodd  {\N_{\mathrm{odd }}}
\newcommand\Zeven {\Z_{\mathrm{even}}}
\newcommand\Zodd  {\Z_{\mathrm{odd }}}
\newcommand\Np    {\N_+}

% Text Shortcuts
\newcommand\open  {\big(}
\newcommand\qopen {\quad\big(}
\newcommand\close {\big)}
\newcommand\also  {\mathrm{, }}
\newcommand\defis {\mathrm{ definitions}}
\newcommand\given {\mathrm{given }}
\newcommand\case  {\mathrm{if }}
\newcommand\syx   {\mathrm{ syntax}}
\newcommand\rle   {\mathrm{ rule}}
\newcommand\other {\mathrm{else}}
\newcommand\set   {\ell et \text{ }}
\newcommand\ans   {\mathscr{A}\!\mathit{nswer}}

% Set theory shortcuts
\newcommand\ra    {\rangle}
\newcommand\la    {\langle}

\newcommand\oto   {\leftarrow}

\newcommand\QED   {\quad\quad\mathscr{Q.E.D.}\;\;\blacksquare}
\newcommand\QEF   {\quad\quad\mathscr{Q.E.F.}}
\newcommand\eQED  {\mathscr{Q.E.D.}\;\;\blacksquare}
\newcommand\eQEF  {\mathscr{Q.E.F.}}
\newcommand\jQED  {\mathscr{Q.E.D.}}

\DeclareMathOperator\dom   {dom}
\DeclareMathOperator\Img   {Im}
\DeclareMathOperator\range {range}

\newcommand\trio  {\triangle}

\newcommand\rc    {\right\rceil}
\newcommand\lc    {\left\lceil}
\newcommand\rf    {\right\rfloor}
\newcommand\lf    {\left\lfloor}
\newcommand\ceil  [1] {\lc #1 \rc}
\newcommand\floor [1] {\lf #1 \rf}

\newcommand\lex   {<_{lex}}

\newcommand\az    {\aleph_0}
\newcommand\uaz   {^{\aleph_0}}
\newcommand\al    {\aleph}
\newcommand\ual   {^\aleph}
\newcommand\taz   {2^{\aleph_0}}
\newcommand\utaz  { ^{\left (2^{\aleph_0} \right )}}
\newcommand\tal   {2^{\aleph}}
\newcommand\utal  { ^{\left (2^{\aleph} \right )}}
\newcommand\ttaz  {2^{\left (2^{\aleph_0}\right )}}

\newcommand\n     {$n$־יה\ }

% Math A&B shortcuts
\newcommand\logn  {\log n}
\newcommand\logx  {\log x}
\newcommand\lnx   {\ln x}
\newcommand\cosx  {\cos x}
\newcommand\sinx  {\sin x}
\newcommand\sint  {\sin \theta}
\newcommand\tanx  {\tan x}
\newcommand\tant  {\tan \theta}
\newcommand\sex   {\sec x}
\newcommand\sect  {\sec^2}
\newcommand\cotx  {\cot x}
\newcommand\cscx  {\csc x}
\newcommand\sinhx {\sinh x}
\newcommand\coshx {\cosh x}
\newcommand\tanhx {\tanh x}

\newcommand\seq   {\overset{!}{=}}
\newcommand\slh   {\overset{LH}{=}}
\newcommand\sle   {\overset{!}{\le}}
\newcommand\sge   {\overset{!}{\ge}}
\newcommand\sll   {\overset{!}{<}}
\newcommand\sgg   {\overset{!}{>}}

\newcommand\h     {\hat}
\newcommand\ve    {\vec}
\newcommand\lv    {\overrightarrow}
\newcommand\ol    {\overline}

\newcommand\mlcm  {\mathrm{lcm}}

\DeclareMathOperator{\sech}   {sech}
\DeclareMathOperator{\csch}   {csch}
\DeclareMathOperator{\arcsec} {arcsec}
\DeclareMathOperator{\arccot} {arcCot}
\DeclareMathOperator{\arccsc} {arcCsc}
\DeclareMathOperator{\arccosh}{arccosh}
\DeclareMathOperator{\arcsinh}{arcsinh}
\DeclareMathOperator{\arctanh}{arctanh}
\DeclareMathOperator{\arcsech}{arcsech}
\DeclareMathOperator{\arccsch}{arccsch}
\DeclareMathOperator{\arccoth}{arccoth}
\DeclareMathOperator{\atant}  {atan2} 
\DeclareMathOperator{\Sp}     {span} 
\DeclareMathOperator{\sgn}    {sgn} 
\DeclareMathOperator{\row}    {Row} 
\DeclareMathOperator{\adj}    {adj} 
\DeclareMathOperator{\rk}     {rank} 
\DeclareMathOperator{\col}    {Col} 
\DeclareMathOperator{\tr}     {tr}

\newcommand\dx    {\,\mathrm{d}x}
\newcommand\dt    {\,\mathrm{d}t}
\newcommand\dtt   {\,\mathrm{d}\theta}
\newcommand\du    {\,\mathrm{d}u}
\newcommand\dv    {\,\mathrm{d}v}
\newcommand\df    {\mathrm{d}f}
\newcommand\dfdx  {\diff{f}{x}}
\newcommand\dit   {\limhz \frac{f(x + h) - f(x)}{h}}

\newcommand\nt[1] {\frac{#1}{#1}}

\newcommand\limz  {\lim_{x \to 0}}
\newcommand\limxz {\lim_{x \to x_0}}
\newcommand\limi  {\lim_{x \to \infty}}
\newcommand\limh  {\lim_{x \to 0}}
\newcommand\limni {\lim_{x \to - \infty}}
\newcommand\limpmi{\lim_{x \to \pm \infty}}

\newcommand\ta    {\theta}
\newcommand\ap    {\alpha}

\renewcommand\inf {\infty}
\newcommand  \ninf{-\inf}

% Combinatorics shortcuts
\newcommand\sumnk     {\sum_{k = 0}^{n}}
\newcommand\sumni     {\sum_{i = 0}^{n}}
\newcommand\sumnko    {\sum_{k = 1}^{n}}
\newcommand\sumnio    {\sum_{i = 1}^{n}}
\newcommand\sumai     {\sum_{i = 1}^{n} A_i}
\newcommand\nsum[2]   {\reflectbox{\displaystyle\sum_{\reflectbox{\scriptsize$#1$}}^{\reflectbox{\scriptsize$#2$}}}}

\newcommand\bink      {\binom{n}{k}}
\newcommand\setn      {\{a_i\}^{2n}_{i = 1}}
\newcommand\setc[1]   {\{a_i\}^{#1}_{i = 1}}

\newcommand\cupain    {\bigcup_{i = 1}^{n} A_i}
\newcommand\cupai[1]  {\bigcup_{i = 1}^{#1} A_i}
\newcommand\cupiiai   {\bigcup_{i \in I} A_i}
\newcommand\capain    {\bigcap_{i = 1}^{n} A_i}
\newcommand\capai[1]  {\bigcap_{i = 1}^{#1} A_i}
\newcommand\capiiai   {\bigcap_{i \in I} A_i}

\newcommand\xot       {x_{1, 2}}
\newcommand\ano       {a_{n - 1}}
\newcommand\ant       {a_{n - 2}}

% Linear Algebra
\DeclareMathOperator{\chr}     {char}
\DeclareMathOperator{\diag}    {diag}
\DeclareMathOperator{\Hom}     {Hom}
\DeclareMathOperator{\Sym}     {Sym}
\DeclareMathOperator{\Asym}    {ASym}

\newcommand\lra       {\leftrightarrow}
\newcommand\chrf      {\chr(\F)}
\newcommand\F         {\mathbb{F}}
\newcommand\co        {\colon}
\newcommand\tmat[2]   {\cl{\begin{matrix}
			#1
		\end{matrix}\, \middle\vert\, \begin{matrix}
			#2
\end{matrix}}}

\makeatletter
\newcommand\rrr[1]    {\xxrightarrow{1}{#1}}
\newcommand\rrt[2]    {\xxrightarrow{1}[#2]{#1}}
\newcommand\mat[2]    {M_{#1\times#2}}
\newcommand\gmat      {\mat{m}{n}(\F)}
\newcommand\tomat     {\, \dequad \longrightarrow}
\newcommand\pms[1]    {\begin{pmatrix}
		#1
\end{pmatrix}}

\newcommand\norm[1]   {\left \vert \left \vert #1 \right \vert \right \vert}
\newcommand\snorm     {\left \vert \left \vert \cdot \right \vert \right \vert}
\newcommand\smut      {\left \la \cdot \mid \cdot \right \ra}
\newcommand\mut[2]    {\left \la #1 \,\middle\vert\, #2 \right \ra}

% someone's code from the internet: https://tex.stackexchange.com/questions/27545/custom-length-arrows-text-over-and-under
\makeatletter
\newlength\min@xx
\newcommand*\xxrightarrow[1]{\begingroup
	\settowidth\min@xx{$\m@th\scriptstyle#1$}
	\@xxrightarrow}
\newcommand*\@xxrightarrow[2][]{
	\sbox8{$\m@th\scriptstyle#1$}  % subscript
	\ifdim\wd8>\min@xx \min@xx=\wd8 \fi
	\sbox8{$\m@th\scriptstyle#2$} % superscript
	\ifdim\wd8>\min@xx \min@xx=\wd8 \fi
	\xrightarrow[{\mathmakebox[\min@xx]{\scriptstyle#1}}]
	{\mathmakebox[\min@xx]{\scriptstyle#2}}
	\endgroup}
\makeatother


% Greek Letters
\newcommand\ag        {\alpha}
\newcommand\bg        {\beta}
\newcommand\cg        {\gamma}
\newcommand\dg        {\delta}
\newcommand\eg        {\epsi}
\newcommand\zg        {\zeta}
\newcommand\hg        {\eta}
\newcommand\tg        {\theta}
\newcommand\ig        {\iota}
\newcommand\kg        {\keppa}
\renewcommand\lg      {\lambda}
\newcommand\og        {\omicron}
\newcommand\rg        {\rho}
\newcommand\sg        {\sigma}
\newcommand\yg        {\usilon}
\newcommand\wg        {\omega}

\newcommand\Ag        {\Alpha}
\newcommand\Bg        {\Beta}
\newcommand\Cg        {\Gamma}
\newcommand\Dg        {\Delta}
\newcommand\Eg        {\Epsi}
\newcommand\Zg        {\Zeta}
\newcommand\Hg        {\Eta}
\newcommand\Tg        {\Theta}
\newcommand\Ig        {\Iota}
\newcommand\Kg        {\Keppa}
\newcommand\Lg        {\Lambda}
\newcommand\Og        {\Omicron}
\newcommand\Rg        {\Rho}
\newcommand\Sg        {\Sigma}
\newcommand\Yg        {\Usilon}
\newcommand\Wg        {\Omega}

% Other shortcuts
\newcommand\tl    {\tilde}
\newcommand\op    {^{-1}}

\newcommand\sof[1]    {\left | #1 \right |}
\newcommand\cl [1]    {\left ( #1 \right )}
\newcommand\csb[1]    {\left [ #1 \right ]}
\newcommand\ccb[1]    {\left \{ #1 \right \}}

\newcommand\bs        {\blacksquare}
\newcommand\dequad    {\!\!\!\!\!\!}
\newcommand\dequadd   {\dequad\duquad}

\renewcommand\phi     {\varphi}

\newtheorem{Theorem}{משפט}
\theoremstyle{definition}
\newtheorem{definition}{הגדרה}
\newtheorem{Lemma}{למה}
\newtheorem{Remark}{הערה}
\newtheorem{Notion}{סימון}


\newcommand\theo  [1] {\begin{Theorem}#1\end{Theorem}}
\newcommand\defi  [1] {\begin{definition}#1\end{definition}}
\newcommand\rmark [1] {\begin{Remark}#1\end{Remark}}
\newcommand\lem   [1] {\begin{Lemma}#1\end{Lemma}}
\newcommand\noti  [1] {\begin{Notion}#1\end{Notion}}

% DS
\newcommand\limsi     {\limsup_{n \to \inf}}
\newcommand\limfi     {\liminf_{n \to \inf}}

\DeclareMathOperator\amort   {amort}
\DeclareMathOperator\worst   {worst}
\DeclareMathOperator\type    {type}
\DeclareMathOperator\cost    {cost}
\DeclareMathOperator\tim     {time}

\newcommand\dsList{
	\sFunc{List}
	\sFunc{Retrieve}
	\SetKwFunction{RetrieveFirst}{Retrieve-First}
	\SetKwFunction{RetrieveLast}{Retrieve-Last}
	\sFunc{Delete}
	\SetKwFunction{DeleteFirst}{Delete-First}
	\SetKwFunction{DeleteLast}{Delete-Last}
	\sFunc{Insert}
	\SetKwFunction{InsertFirst}{Insert-First}
	\SetKwFunction{InsertLast}{Insert-Last}
	\sFunc{Shift}
	\sFunc{Length}
	\sFunc{Concat}
	\sFunc{Plant}
	\sFunc{Split}
}
\newcommand\dsQueue{
	\sFunc{Queue}
	\sFunc{Enqueue}
	\sFunc{Head}
	\sFunc{Dequeue}
}
\newcommand\dsStack{
	\sFunc{Stack}
	\sFunc{Push}
	\sFunc{Top}
	\sFunc{Pop}
}
\newcommand\dsVector{
	\sFunc{Vector}
	\sFunc{Get}
	\sFunc{Set}
}
\newcommand\dsGraph{
	\sFunc{Graph}
	\sFunc{Edge}
	\SetKwFunction{AddEdge}{Add-Edge}
	\SetKwFunction{RemoveEdge}{Remove-Edge}
	\sFunc{InDeg} \sFunc{OutDeg}
}
\newcommand\importDs{
	\dsList
	\dsQueue
	\dsStack
	\dsVector
	\dsGraph
	\SetKwProg{Fn}{function}{ is}{end}
	\SetKwData{error}{\color{codered}error}
	\SetKwInOut{Input}{input}
	\SetKwInOut{Output}{output}
	\SetKwRepeat{Do}{do}{while}
	\SetKwData{Null}{\color{codegreen}null}
	\SetKwData{True}{\color{codeblue}true}
	\SetKwData{False}{\color{codeblue}false}
}


% Algorithems
\newcommand\sFunc [1] {\SetKwFunction{#1}{#1}}
\newcommand\sData [1] {\SetKwData{#1}{#1}}
\newcommand\sIO   [1] {\SetKwInOut{#1}{#1}}
\newcommand\ttt   [1] {\sen \texttt{#1} \she\,}
\newcommand\io    [2] {\Input{#1}\Output{#2}\BlankLine}

%! ~~~ Document ~~~

\author{שחר פרץ}
\title{\textit{תרגול 1}}
\date{28 ביולי  2025}
\begin{document}
	\maketitle
	\section{}
	יהי $V$ מ''ו מעל $\R$. נניח $v_1, v_2, v_3, v_4$ בת''ל. נגדיר: $U = \Sp\{v_1, \dots v_4\}$. $W = \Sp \{v_1 - v_2, v_2 - v_3, v_3 - v_4, v_4 - v_1\}$. האם $U = W$? 
	
	\textbf{פתרון: }ברור ש־$\dim U = 4$. משום ש־$\sum v_{i} - v_{i - 1 \mod 4} = 0$, אז הוקטורים שיוצרים את $W$ ת''ל ולכן $\dim W\le 3$. סה''כ $\dim W \neq \dim U$ ולכן $U \neq W$ וסיימנו. 
	
	\textit{הערה: }למעשה מתקיים $\dim W = 3$. \textbf{נימוק: }ע''פ משפט מהכיתה ניתן לכתוב $W = \Sp\{v_1 - v-2, v_2 - v_3, v_3 - v_4\}$ (נכון כי בצירוף הלינארי אין אפסים). יהיו $\lg_1 \dots \lg_3$ ואז $\lg_1(v_1 - v_2) + \lg_2(v_2 - v_3) + \lg_3(v_3 - v_4) = 0$ ונוכיח שהוא טרוויאלי. נפתח את הסוגריים ונקבל $\lg_1 v_2 + (\lg_2 - \lg_1)v_2 + (\lg_3 - \lg_2)v_3 - \lg_3 - \lg_3v_4$ צירוף לינארי של וקטורים בת''ל ונקבל $\lg_1 = 0, \ \lg_2 - \lg_1 = 0, \ \lg_3 - \lg_2 = 0, \ \lg_3 = 0$ וסיימנו. 
	
	\textit{הערה: }אם נוסיף לקבוצה מקודם את $v_4$ במקום את $v_4 - v_1$, נקבל קבוצה מממד $4$. 
	
	\section{}
	יהי $\F$ שדה סופי בעל $q$ איברים. כמה איזומורפיזמים $\F^4 \to \F^4$ יש? 
	\textbf{פתרון: }יהי בסיס $B = (v_1 \dots v_4)$ כלשהו. נוכל להגדיר את ההעתקה בהינתן $Tv_1 \dots Tv_4$. נבחין ש־$\sof{\F^4} = q^4$ ולכן יש $q^4$ וקטורים ב־$\F^4$. 
	\begin{enumerate}
		\item בחירת הוקטור הראשון, כל וקטור שאינו אפס $q^4 - 1$ אפשרויות. 
		\item בחירת הוקטור השני, כל דבר שלא ת''ל בוטקור הראשון כלומר כל דבר שהוא לא כפל בסקלר בוקטור הראשון – $q^4 - q$ אפשרויות. 
		\item בחירת הוקטור השני – כל דבר שאינו צירוף לינארי (שני סקלרים) ונקבל $q^4 - q^2$. 
		\item כנ''ל $q^4 - q^3$
	\end{enumerate}
	סה''כ: 
	\[ (q^4 - 1)(q^4 - q)(q^4 - q^2)(q^4 - q^3) = q^6(q^4 - 1)(q^3 - 1)(q^2 - 1)(q - 1) \]
	
	\section{}
	תהי $A$ מטריצה ריבועית מעל שדה $\F$ כך ש־$\rho(A^2) \ge \rho(A)$ הוכיחו שמרחב הפתרונות של $A\bar x = 0$ שווה למרחב הפתרונות של $A^2\bar x = 0$. מסתבר ש־$\rho = \rk$. 
	
	\begin{proof}
		יהי $\bar x \in \nc A$, כלומר $A\bar x = 0$ מהגדרה. לכן $A^2x = (A \cdot A)x = A(Ax) = A0 = 0 \implies \bar x \in \nc A^2$ ולכן $\nc A \subseteq \nc A^2$. 
		
		נתון $\rk A^2 \ge \rk A$ ומשום שידוע ממשפט הדרגה והאפסות:
		\[ n - \dim \nc A  = \rk A\ge \rk A^2 = n - \dim \nc A^2 \]
		ומשום שנתון $\rk A = \rk A^2$ מקבלים שוויון. סה''כ: 
		\[ n - \dim \nc A = n - \dim \nc A^2 \implies \dim \nc A = \dim \nc A^2 \]
		וראינו הכלה, לכן $\nc A = \nc A^2$ וסיימנו. 
	\end{proof}
	\textit{הערה: }יותר קל להראות הכלה ושוויון ממדים מהכלה דו־כיוונית, ברוב הפעמים. 
	
	\section{}
	תהי $T \co M_3(\R) \to M_3(\R)$ ההעתקה הלינארית $T(A) = A^T$. מצאו $\det T$, כלומר את $\det([T]^B_B)$ כאשר $B$ בסיס של $M_3(\R)$. 
	
	\textit{הערה: }זה מוגדר היטב כי דטרמיננטה לא תלויה בבסיס מהסיבה הבאה: 
	\[ \det([T]_C^C) = \det([id]_C^B[T]^B_B[id]_B^C) = \det(M_C^B)\det([T]^B_B)\det(M_B^C) = \cancel{\det(M_C^B)\det((M_C^B))\op }\det([T]_B^B) = \det ([T]^B_B) \]
	
	עתה נפנה להוכיח את השאלה. משום שאנו יכולים לבחור כל בסיס, נבחר את הבסיס הסטנדרטי. בהינתן: 
	\[ B = \{e_1 \dots e_9\}, \ T(B) = \{e_1, e_4, e_7, e_2, e_5, e_8, e_3, e_6, e_9\} \]
	הערה: $e_i$ מוגדר באופן הבא: 
	\[ e_i = \pms{\dg_{i1} & \dg_{i2} & \dg_{i3} \\ \dg_{i4} & \dg_{i5} & \dg_{i6} \\ \dg_{i7} & \dg_{i8} & \dg_{i9}} \]
	קיבלנו מטריצה $9 \times 9$ שנוכל למצוא את הדטרמיננטה שלה באמצעות מינורים ושטויות, אבל אפשר להסתכל על זה גם כעל $3$ החלפות עמודות ביחס ל־$I$ כלומר $\det[T]^B_B = (-1)^{3}\det I = -1$. 
	
	\section{}
	יהי $V$ מ''ו מעל $\F$. תהי $T \co V \to V$ ט''ל המקיימת $T \circ T = T$. הוכיחו ש־$V = \ker T \oplus \Img T$. 
	
	\begin{proof}
		נתחיל מלהוכיח חיתוך ריק. יהי $v \in \ker T \cap \Img T$. אז $Tv = 0 \land \exists w \in V \co Tw = v$. אזי: 
		\[ T^2 = T \implies 0 = Tv = T(T(w)) = T^2w = Tw = v \implies v = 0 \]
		לכן $\Img T \cap \ker T = \{0\}$. מהגדרה, $\ker T, \Img T \subseteq V$ ולכן בפרט $\Img T \oplus \ker T \subseteq V$. ברם, ממשפט ממדים כלשהו $\dim(\ker T \oplus \dim \Img T) = \dim \ker T + \dim \Img T = \dim V$. דהיינו $\ker T \oplus \Img T = V$ כדרוש. 
	
	\end{proof}
	
	\section{}
	יהי $V$ מ''ו המוגדר מעל $\R$ יהיו. תהיינה $T, S \co V \to \R$ העתקות לינאריות שאינן העתקת האפס. נניח שמתקיים לכל $v \in V$ כך שאם $T(v) \ge 0 \implies S(v) \ge 0$. הוכיחו כי קיים סקלר $\ag > 0$ ממשי המקיים $T = \ag S$. 
	
	\begin{proof}[טיוטה. ]
		הבחנה: $\dim \Img T = 1$ (כי לא העתקת האפס, ולכן $\dim \Img T \neq 0$). באופן דומה $\dim \Img S = 1$. ממשפט הממדים $\dim \ker T = \dim  \ker S = n - 1$. לכן נוכל לבחור בסיס $\{v_1 \dots v_n\}$ כך ש־$T(v) = \lg_nT(v_n)$ (כי קיים $v_n$ כזה ואז אפשר להרחיב בסיס). נרצה להגיד אותו הדבר על $S$, כך לא ברור שזה אותו הבסיס. נבחר בסיס כך ש־$S(u) = \ag_nT(u_n)$. הרעיון הכלילי הוא להראות שהן מתאפסות ביחד, ואז למעשה הוכחנו שוויון קרנלים מה שמוכיח את הטענה. למעשה רוצים להוכיח $T(v) = 0 \implies S(v) = 0$. אבל זה לא קשה: נניח $S(v) = 0$ אז $Sv \ge 0$. לכן $T(-v) = 0 \implies S(-v) \ge 0$ וסה''כ הראנו $S(v) = 0$. 
	\end{proof}
	
	\begin{proof}
		יהי $v \in \ker T$, כלומר $Tv = 0$. ע''פ הנתון $Sv \ge 0$. ,תשתקו, $T$ ט''ל ולכן $T(-v) = 0$ משמע $S(-v) = -S(v) \ge 0$. מכאן נסיק שבהכרח $S(v) = 0$ כלומר $v \in \ker S$ ועל כן $\ker T \subseteq \ker S$. 
		
		כיוון ו־$T, S$ אינן העתקות האפס והטווח שלהן הוא $\R$ שממדו $1$, אז: 
		\[ \dim \Img T = \dim \Img S = 1 \]
		ע''פ משפט הממדים, נגרר: 
		\[ \dim \ker T = \dim \ker S = n - 1 \]
		יחדיו עם ההכלה ממקודם נסיק $\ker T = \ker S$. יהי בסיס $\{v_1 \dots v_n\}$ בסיס ל־$\ker T$ ונשלימו לבסיס $\{v_1 \dots v_n\}$ של $V$. יהי $v \in V$ נתון על־ידי צירוף לינארי $v = \sumni \lg_i v_i$. נקבל $T(v) = \sumni \lg_i T(v_i) = \lg_n T(v_n)$ ובאופן דומה $S(v) = \lg_n S(v_n)$. בהגדרה $v_n \notin \ker T = \ker S$, לכן $T(v_n), S(v_n) \neq 0$. זה מאפשר לחלק ולקבל: 
		\[ \lg_n = \frac{T(v_n)}{S(v_n)}T(v_n) = \frac{T(v_n)}{S(v_n)} S(v) \]
		נגדיר $\ag = \frac{T(v_n)}{S(v_n)}$. נקבל $\forall v \in V\co T(v) = \ag S(v) \implies T = \ag S$. 
		
		מותר להניח בה''כ $T(v_n) > 0$ (אחרת פשוט נשלים לבסיס עם $-v_n$ במקום) ואז ע''פ הנתון $S(v_n) \ge 0$ ולכן $\ag > 0$ מש''ל. 
	\end{proof}
	
	אינטואציה שלי: צריך למעשה שקבוצת המקורות של התמונה תהיה אותה הדבר. תנסו להבין למה זה נכון, זה המפתח לפתרון השאלה. 
	
	\section{}
	תההינה $A, B \in M_n(\F)$. נניח $\rk A, \rk B \neq n - 1$. הוכיחו כי $\adj(AB) = \adj B \adj A$. 
	
		הרמז להפרדה למקרים: אי־השוויון ל־$n - 1$. במקרה של $n -2$ אי אפשר להשתמש בקטע של ההפיכות ולכן חייבים להשתמש בהגדרה. 
	\begin{proof}
		נחלק למקרים. 
		\begin{itemize}
			\item אם $n = \rk A, \rk B$ במקרה הזה שתיהן הפיכות ובפרט $AB$ הפיכה. אז: 
			\[ \adj(AB) = \det(AB) \cdot (AB)\op = (\det(B) B\op)(\det A A\op) = \adj B \adj A \]
			\item נניח שלפחות אחת מהמטריצות בה''כ $A$ מדרגה לכל היותר $n - 2$. טענת עזר: אם $\rk C \le n - 2$ אז $\adj C = 0$. לכן אגף ימין $0$ וסיימנו. ידוע $\rk AB \le \min\{\rk A, \rk B\} \le n - 2$ ולכן גם $\adj AB = 0$, וסה''כ $0 = 0$. 
			
			טענת העזר נכונה ישירות מההגדרה של $\adj$, כי המינור מתאפס. 
		\end{itemize}
	\end{proof}
	הערה שלי: מי שמתקשה שיפתור את 23BA שאלה 1. ו־25AB שאלה 1. יש שם גם הרחבה על $\rk A = n - 2$. 
	
	עוד הערה שלי: תפתרו שאלות מס' 3-4 ממבחנים של גינסבורג. אלו השאלות הכי קשות שתמצאו. 
	
	``קוראים לזה הג'וינט ואני לא יכול לומר לך'' $\sim$ גינסוברג למיכאל ששאל אותו מה זה
	
	
	
	
	\ndoc
\end{document}