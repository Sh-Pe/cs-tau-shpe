%! ~~~ Packages Setup ~~~ 
\documentclass[]{article}
\usepackage{lipsum}
\usepackage{rotating}


% Math packages
\usepackage[usenames]{color}
\usepackage{forest}
\usepackage{ifxetex,ifluatex,amssymb,amsmath,mathrsfs,amsthm,witharrows,mathtools,mathdots}
\usepackage{amsmath}
\WithArrowsOptions{displaystyle}
\renewcommand{\qedsymbol}{$\blacksquare$} % end proofs with \blacksquare. Overwrites the defualts. 
\usepackage{cancel,bm}
\usepackage[thinc]{esdiff}


% tikz
\usepackage{tikz}
\usetikzlibrary{graphs}
\newcommand\sqw{1}
\newcommand\squ[4][1]{\fill[#4] (#2*\sqw,#3*\sqw) rectangle +(#1*\sqw,#1*\sqw);}


% code 
\usepackage{listings}
\usepackage{xcolor}

\definecolor{codegreen}{rgb}{0,0.35,0}
\definecolor{codegray}{rgb}{0.5,0.5,0.5}
\definecolor{codenumber}{rgb}{0.1,0.3,0.5}
\definecolor{codeblue}{rgb}{0,0,0.5}
\definecolor{codered}{rgb}{0.5,0.03,0.02}
\definecolor{codegray}{rgb}{0.96,0.96,0.96}

\lstdefinestyle{pythonstylesheet}{
	language=Java,
	emphstyle=\color{deepred},
	backgroundcolor=\color{codegray},
	keywordstyle=\color{deepblue}\bfseries\itshape,
	numberstyle=\scriptsize\color{codenumber},
	basicstyle=\ttfamily\footnotesize,
	commentstyle=\color{codegreen}\itshape,
	breakatwhitespace=false, 
	breaklines=true, 
	captionpos=b, 
	keepspaces=true, 
	numbers=left, 
	numbersep=5pt, 
	showspaces=false,                
	showstringspaces=false,
	showtabs=false, 
	tabsize=4, 
	morekeywords={as,assert,nonlocal,with,yield,self,True,False,None,AssertionError,ValueError,in,else},              % Add keywords here
	keywordstyle=\color{codeblue},
	emph={var, List, Iterable, Iterator},          % Custom highlighting
	emphstyle=\color{codered},
	stringstyle=\color{codegreen},
	showstringspaces=false,
	abovecaptionskip=0pt,belowcaptionskip =0pt,
	framextopmargin=-\topsep, 
}
\newcommand\pythonstyle{\lstset{pythonstylesheet}}
\newcommand\pyl[1]     {{\lstinline!#1!}}
\lstset{style=pythonstylesheet}

\usepackage[style=1,skipbelow=\topskip,skipabove=\topskip,framemethod=TikZ]{mdframed}
\definecolor{bggray}{rgb}{0.85, 0.85, 0.85}
\mdfsetup{leftmargin=0pt,rightmargin=0pt,innerleftmargin=15pt,backgroundcolor=codegray,middlelinewidth=0.5pt,skipabove=5pt,skipbelow=0pt,middlelinecolor=black,roundcorner=5}
\BeforeBeginEnvironment{lstlisting}{\begin{mdframed}\vspace{-0.4em}}
	\AfterEndEnvironment{lstlisting}{\vspace{-0.8em}\end{mdframed}}


% Deisgn
\usepackage[labelfont=bf]{caption}
\usepackage[margin=0.6in]{geometry}
\usepackage{multicol}
\usepackage[skip=4pt, indent=0pt]{parskip}
\usepackage[normalem]{ulem}
\forestset{default}
\renewcommand\labelitemi{$\bullet$}
\usepackage{titlesec}
\titleformat{\section}[block]
{\fontsize{15}{15}}
{\sen \dotfill (\thesection)\dotfill\she}
{0em}
{\MakeUppercase}
\usepackage{graphicx}
\graphicspath{ {./} }


% Hebrew initialzing
\usepackage[bidi=basic]{babel}
\PassOptionsToPackage{no-math}{fontspec}
\babelprovide[main, import, Alph=letters]{hebrew}
\babelprovide[import]{english}
\babelfont[hebrew]{rm}{David CLM}
\babelfont[hebrew]{sf}{David CLM}
\babelfont[english]{tt}{Monaspace Xenon}
\usepackage[shortlabels]{enumitem}
\newlist{hebenum}{enumerate}{1}

% Language Shortcuts
\newcommand\en[1] {\begin{otherlanguage}{english}#1\end{otherlanguage}}
\newcommand\sen   {\begin{otherlanguage}{english}}
	\newcommand\she   {\end{otherlanguage}}
\newcommand\del   {$ \!\! $}

\newcommand\npage {\vfil {\hfil \textbf{\textit{המשך בעמוד הבא}}} \hfil \vfil \pagebreak}
\newcommand\ndoc  {\dotfill \\ \vfil {\begin{center}
			{\textbf{\textit{שחר פרץ, 2025}} \\
				\scriptsize \textit{קומפל ב־}\en{\LaTeX}\,\textit{ ונוצר באמצעות תוכנה חופשית בלבד}}
	\end{center}} \vfil	}

\newcommand{\rn}[1]{
	\textup{\uppercase\expandafter{\romannumeral#1}}
}

\makeatletter
\newcommand{\skipitems}[1]{
	\addtocounter{\@enumctr}{#1}
}
\makeatother

%! ~~~ Math shortcuts ~~~

% Letters shortcuts
\newcommand\N     {\mathbb{N}}
\newcommand\Z     {\mathbb{Z}}
\newcommand\R     {\mathbb{R}}
\newcommand\Q     {\mathbb{Q}}
\newcommand\C     {\mathbb{C}}
\newcommand\One   {\mathit{1}}

\newcommand\ml    {\ell}
\newcommand\mj    {\jmath}
\newcommand\mi    {\imath}

\newcommand\powerset {\mathcal{P}}
\newcommand\ps    {\mathcal{P}}
\newcommand\pc    {\mathcal{P}}
\newcommand\ac    {\mathcal{A}}
\newcommand\bc    {\mathcal{B}}
\newcommand\cc    {\mathcal{C}}
\newcommand\dc    {\mathcal{D}}
\newcommand\ec    {\mathcal{E}}
\newcommand\fc    {\mathcal{F}}
\newcommand\nc    {\mathcal{N}}
\newcommand\vc    {\mathcal{V}} % Vance
\newcommand\sca   {\mathcal{S}} % \sc is already definded
\newcommand\rca   {\mathcal{R}} % \rc is already definded

\newcommand\prm   {\mathrm{p}}
\newcommand\arm   {\mathrm{a}} % x86
\newcommand\brm   {\mathrm{b}}
\newcommand\crm   {\mathrm{c}}
\newcommand\drm   {\mathrm{d}}
\newcommand\erm   {\mathrm{e}}
\newcommand\frm   {\mathrm{f}}
\newcommand\nrm   {\mathrm{n}}
\newcommand\vrm   {\mathrm{v}}
\newcommand\srm   {\mathrm{s}}
\newcommand\rrm   {\mathrm{r}}

\newcommand\Si    {\Sigma}

% Logic & sets shorcuts
\newcommand\siff  {\longleftrightarrow}
\newcommand\ssiff {\leftrightarrow}
\newcommand\so    {\longrightarrow}
\newcommand\sso   {\rightarrow}

\newcommand\epsi  {\epsilon}
\newcommand\vepsi {\varepsilon}
\newcommand\vphi  {\varphi}
\newcommand\Neven {\N_{\mathrm{even}}}
\newcommand\Nodd  {\N_{\mathrm{odd }}}
\newcommand\Zeven {\Z_{\mathrm{even}}}
\newcommand\Zodd  {\Z_{\mathrm{odd }}}
\newcommand\Np    {\N_+}

% Text Shortcuts
\newcommand\open  {\big(}
\newcommand\qopen {\quad\big(}
\newcommand\close {\big)}
\newcommand\also  {\text{, }}
\newcommand\defis {\text{ definitions}}
\newcommand\given {\text{given }}
\newcommand\case  {\text{if }}
\newcommand\syx   {\text{ syntax}}
\newcommand\rle   {\text{ rule}}
\newcommand\other {\text{else}}
\newcommand\set   {\ell et \text{ }}
\newcommand\ans   {\mathscr{A}\!\mathit{nswer}}

% Set theory shortcuts
\newcommand\ra    {\rangle}
\newcommand\la    {\langle}

\newcommand\oto   {\leftarrow}

\newcommand\QED   {\quad\quad\mathscr{Q.E.D.}\;\;\blacksquare}
\newcommand\QEF   {\quad\quad\mathscr{Q.E.F.}}
\newcommand\eQED  {\mathscr{Q.E.D.}\;\;\blacksquare}
\newcommand\eQEF  {\mathscr{Q.E.F.}}
\newcommand\jQED  {\mathscr{Q.E.D.}}

\DeclareMathOperator\dom   {dom}
\DeclareMathOperator\Img   {Im}
\DeclareMathOperator\range {range}

\newcommand\trio  {\triangle}

\newcommand\rc    {\right\rceil}
\newcommand\lc    {\left\lceil}
\newcommand\rf    {\right\rfloor}
\newcommand\lf    {\left\lfloor}

\newcommand\lex   {<_{lex}}

\newcommand\az    {\aleph_0}
\newcommand\uaz   {^{\aleph_0}}
\newcommand\al    {\aleph}
\newcommand\ual   {^\aleph}
\newcommand\taz   {2^{\aleph_0}}
\newcommand\utaz  { ^{\left (2^{\aleph_0} \right )}}
\newcommand\tal   {2^{\aleph}}
\newcommand\utal  { ^{\left (2^{\aleph} \right )}}
\newcommand\ttaz  {2^{\left (2^{\aleph_0}\right )}}

\newcommand\n     {$n$־יה\ }

% Math A&B shortcuts
\newcommand\logn  {\log n}
\newcommand\logx  {\log x}
\newcommand\lnx   {\ln x}
\newcommand\cosx  {\cos x}
\newcommand\sinx  {\sin x}
\newcommand\sint  {\sin \theta}
\newcommand\tanx  {\tan x}
\newcommand\tant  {\tan \theta}
\newcommand\sex   {\sec x}
\newcommand\sect  {\sec^2}
\newcommand\cotx  {\cot x}
\newcommand\cscx  {\csc x}
\newcommand\sinhx {\sinh x}
\newcommand\coshx {\cosh x}
\newcommand\tanhx {\tanh x}

\newcommand\seq   {\overset{!}{=}}
\newcommand\slh   {\overset{LH}{=}}
\newcommand\sle   {\overset{!}{\le}}
\newcommand\sge   {\overset{!}{\ge}}
\newcommand\sll   {\overset{!}{<}}
\newcommand\sgg   {\overset{!}{>}}

\newcommand\h     {\hat}
\newcommand\ve    {\vec}
\newcommand\lv    {\overrightarrow}
\newcommand\ol    {\overline}

\newcommand\mlcm  {\mathrm{lcm}}

\DeclareMathOperator{\sech}   {sech}
\DeclareMathOperator{\csch}   {csch}
\DeclareMathOperator{\arcsec} {arcsec}
\DeclareMathOperator{\arccot} {arcCot}
\DeclareMathOperator{\arccsc} {arcCsc}
\DeclareMathOperator{\arccosh}{arccosh}
\DeclareMathOperator{\arcsinh}{arcsinh}
\DeclareMathOperator{\arctanh}{arctanh}
\DeclareMathOperator{\arcsech}{arcsech}
\DeclareMathOperator{\arccsch}{arccsch}
\DeclareMathOperator{\arccoth}{arccoth}
\DeclareMathOperator{\atant}  {atan2} 
\DeclareMathOperator{\Sp}     {span} 
\DeclareMathOperator{\sgn}    {sgn} 
\DeclareMathOperator{\row}    {Row} 
\DeclareMathOperator{\adj}    {adj} 
\DeclareMathOperator{\rk}     {rank} 
\DeclareMathOperator{\col}    {Col} 
\DeclareMathOperator{\tr}     {tr}

\newcommand\dx    {\,\mathrm{d}x}
\newcommand\dt    {\,\mathrm{d}t}
\newcommand\dtt   {\,\mathrm{d}\theta}
\newcommand\du    {\,\mathrm{d}u}
\newcommand\dv    {\,\mathrm{d}v}
\newcommand\df    {\mathrm{d}f}
\newcommand\dfdx  {\diff{f}{x}}
\newcommand\dit   {\limhz \frac{f(x + h) - f(x)}{h}}

\newcommand\nt[1] {\frac{#1}{#1}}

\newcommand\limz  {\lim_{x \to 0}}
\newcommand\limxz {\lim_{x \to x_0}}
\newcommand\limi  {\lim_{x \to \infty}}
\newcommand\limh  {\lim_{x \to 0}}
\newcommand\limni {\lim_{x \to - \infty}}
\newcommand\limpmi{\lim_{x \to \pm \infty}}

\newcommand\ta    {\theta}
\newcommand\ap    {\alpha}

\renewcommand\inf {\infty}
\newcommand  \ninf{-\inf}

% Combinatorics shortcuts
\newcommand\sumnk     {\sum_{k = 0}^{n}}
\newcommand\sumni     {\sum_{i = 0}^{n}}
\newcommand\sumnko    {\sum_{k = 1}^{n}}
\newcommand\sumnio    {\sum_{i = 1}^{n}}
\newcommand\sumai     {\sum_{i = 1}^{n} A_i}
\newcommand\nsum[2]   {\reflectbox{\displaystyle\sum_{\reflectbox{\scriptsize$#1$}}^{\reflectbox{\scriptsize$#2$}}}}

\newcommand\bink      {\binom{n}{k}}
\newcommand\setn      {\{a_i\}^{2n}_{i = 1}}
\newcommand\setc[1]   {\{a_i\}^{#1}_{i = 1}}

\newcommand\cupain    {\bigcup_{i = 1}^{n} A_i}
\newcommand\cupai[1]  {\bigcup_{i = 1}^{#1} A_i}
\newcommand\cupiiai   {\bigcup_{i \in I} A_i}
\newcommand\capain    {\bigcap_{i = 1}^{n} A_i}
\newcommand\capai[1]  {\bigcap_{i = 1}^{#1} A_i}
\newcommand\capiiai   {\bigcap_{i \in I} A_i}

\newcommand\xot       {x_{1, 2}}
\newcommand\ano       {a_{n - 1}}
\newcommand\ant       {a_{n - 2}}

% Linear Algebra
\DeclareMathOperator{\chr}     {char}
\DeclareMathOperator{\diag}    {diag}
\DeclareMathOperator{\Hom}     {Hom}

\newcommand\lra       {\leftrightarrow}
\newcommand\chrf      {\chr(\F)}
\newcommand\F         {\mathbb{F}}
\newcommand\co        {\colon}
\newcommand\tmat[2]   {\cl{\begin{matrix}
			#1
		\end{matrix}\, \middle\vert\, \begin{matrix}
			#2
\end{matrix}}}

\makeatletter
\newcommand\rrr[1]    {\xxrightarrow{1}{#1}}
\newcommand\rrt[2]    {\xxrightarrow{1}[#2]{#1}}
\newcommand\mat[2]    {M_{#1\times#2}}
\newcommand\gmat      {\mat{m}{n}(\F)}
\newcommand\tomat     {\longrightarrow}
\newcommand\pms[1]    {\begin{pmatrix}
		#1
\end{pmatrix}}

% someone's code from the internet: https://tex.stackexchange.com/questions/27545/custom-length-arrows-text-over-and-under
\makeatletter
\newlength\min@xx
\newcommand*\xxrightarrow[1]{\begingroup
	\settowidth\min@xx{$\m@th\scriptstyle#1$}
	\@xxrightarrow}
\newcommand*\@xxrightarrow[2][]{
	\sbox8{$\m@th\scriptstyle#1$}  % subscript
	\ifdim\wd8>\min@xx \min@xx=\wd8 \fi
	\sbox8{$\m@th\scriptstyle#2$} % superscript
	\ifdim\wd8>\min@xx \min@xx=\wd8 \fi
	\xrightarrow[{\mathmakebox[\min@xx]{\scriptstyle#1}}]
	{\mathmakebox[\min@xx]{\scriptstyle#2}}
	\endgroup}
\makeatother


% Greek Letters
\newcommand\ag        {\alpha}
\newcommand\bg        {\beta}
\newcommand\cg        {\gamma}
\newcommand\dg        {\delta}
\newcommand\eg        {\epsi}
\newcommand\zg        {\zeta}
\newcommand\hg        {\eta}
\newcommand\tg        {\theta}
\newcommand\ig        {\iota}
\newcommand\kg        {\keppa}
\renewcommand\lg      {\lambda}
\newcommand\og        {\omicron}
\newcommand\rg        {\rho}
\newcommand\sg        {\sigma}
\newcommand\yg        {\usilon}
\newcommand\wg        {\omega}

\newcommand\Ag        {\Alpha}
\newcommand\Bg        {\Beta}
\newcommand\Cg        {\Gamma}
\newcommand\Dg        {\Delta}
\newcommand\Eg        {\Epsi}
\newcommand\Zg        {\Zeta}
\newcommand\Hg        {\Eta}
\newcommand\Tg        {\Theta}
\newcommand\Ig        {\Iota}
\newcommand\Kg        {\Keppa}
\newcommand\Lg        {\Lambda}
\newcommand\Og        {\Omicron}
\newcommand\Rg        {\Rho}
\newcommand\Sg        {\Sigma}
\newcommand\Yg        {\Usilon}
\newcommand\Wg        {\Omega}

% Other shortcuts
\newcommand\tl    {\tilde}
\newcommand\op    {^{-1}}

\newcommand\sof[1]    {\left | #1 \right |}
\newcommand\cl [1]    {\left ( #1 \right )}
\newcommand\csb[1]    {\left [ #1 \right ]}
\newcommand\ccb[1]    {\left \{ #1 \right \}}

\newcommand\bs        {\blacksquare}
\newcommand\dequad    {\!\!\!\!\!\!}
\newcommand\dequadd   {\dequad\duquad}

\renewcommand\phi     {\varphi}

\newtheorem{Theorem}{משפט}
\theoremstyle{definition}
\newtheorem{definition}{הגדרה}
\newtheorem{Lemma}{למה}
\newtheorem{Remark}{הערה}
\newtheorem{Notion}{סימון}

\newcommand\theo  [1] {\begin{Theorem}#1\end{Theorem}}
\newcommand\defi  [1] {\begin{definition}#1\end{definition}}
\newcommand\rmark [1] {\begin{Remark}#1\end{Remark}}
\newcommand\lem   [1] {\begin{Lemma}#1\end{Lemma}}
\newcommand\noti  [1] {\begin{Notion}#1\end{Notion}}

% DS
\DeclareMathOperator\amort   {amort}
\DeclareMathOperator\worst   {worst}
\DeclareMathOperator\type    {type}
\DeclareMathOperator\cost    {cost}

%! ~~~ Document ~~~

\author{שחר פרץ}
\title{\textit{ליניארית 1א $\sim$ תרגיל בית 2}}
\begin{document}
	\maketitle
	
	\textbf{הערה: }אלא אם מצוין אחרת, בתרגיל בית זה $\equiv_p$ מייצג את יחס השיקלות מעל השלמים בעבורו $a \equiv_p b$ אמ"מ קיימים $a \bmod p = b \bmod p$. לעיתים יופיע היחס מעל הטבעיים. 
	
	\section{}
	\begin{enumerate}[(A)]
		\item יהיו $a, b \in \F$. נוכיח $a^2 = b^2 \implies a = -b \lor a = b$. \begin{proof}\textit{(הערה: אשתמש בסעיף הבא כמו והוכחתי אותו לפני הסעיף הזה)}
			נתון השוויון $a^2 = b^2$. נפצל למקרים: 
			\begin{itemize}
				\item אם $a \neq 0$אז ידוע קיו םאיבר $a\op$ הופכי ל־$a$. אז נכפול את המשוואה הנתונה ב־$(a\op)^2 = a^{-2}$, נקבל $aaa\op a\op = b^2a^{-2}$. אז מקומטטיביות והגדרת הופכי $1 = (ba\op)^{2}$. 
				אחרת, קיים $ba\op =: c \neq 1, -1$ כך ש־$c^2 = 1$. אזי $c$ הופכי של $c$, כלומר $c^2 - 1 = 0$. אזי: 
				\[ 0 = c^2 - 1 = c^2 - c + c - 1 = (c - 1)(c + 1) \implies c = 1 \lor c = -1 \]
				קיבלנו $ba\op = 1 \lor ba\op = -1$. נכפול את המשוואות חזרה ב־$a$, נקבל $b = a \lor b = -a$ כדרוש. 	
				\item אחרת, $a = 0$ ואז: 
				\[ b^2 = a^2 = 0^2 = 0 \implies b = 0, \ (0 = 0 \land 0 = -0) \implies (a = -b \land a = b) \]
			\end{itemize}
		\end{proof}
		\item צ.ל. $(-a)(-b) = a \cdot b$ \begin{proof}
			יהיו $a, b \in \F$. אז: \\
			\[ (-a)(-b) = (-1)a \cdot (-1)b = (-1)^2ab = 1 \cdot ab = ab \]
			נותר להוכיח את השוויון $(-1)^2 = 1$. זאת כי: 
			\[ 0 = (-2) \cdot 0 = (-1 -1)(-1 + 1) = (-1)^2 - 1^2 = (-1)^2 - 1 \]
			נחבר $1$ לשני האגפים, נקבל $(-1)^2 = 1$ כדרוש. 
		\end{proof}
		\item יהיו $a, b, c, d \in \F$, נניח $b, d \neq 0$. צ.ל. $ab\op + cd\op = (ad + bc) \cdot (bd)\op$
		\begin{proof}מדיסטרבוטיביות: 
			\[ (ad + bc)(bd)\op = (ad + bc)(b\op d\op) = adb\op d\op + bcb\op d\op = ab\op\underbrace{(dd\op)}_{=1} + cd\op\underbrace{(bb\op)}_{=1} = ab\op + cd\op \quad \top \]
			נבחין כי הביטוי עצמו מוגדר היטב אמ"מ $b, d \neq 0$ ולכן ניגרשה הנחה זו (שכן, אחרת $bd = 0$ כי כפל ב־$0$ יוביל ל־$0$, ועל כן $(bd)\op$ יהיה לא מוגדר). 
		\end{proof}
	\end{enumerate}
	
	\section{}
	
	\begin{enumerate}[(A)]
		\item  \begin{proof}למה. בשדה $\F_p$ הפונ' $f(a) \to a\op$ חח"ע. נוכיח את הלמה.
			יהיו $a, b \in \F_p$ ונניח $f(a) = f(b)$. נראה $a = b$. מההנחה $a\op = b\op$. נכפול ב־$a\op b$ ונקבל $a\op a b = b\op b a$ מקומטטיביות. מאיבר הופכי נקבל $1b = 1a$ כלומר $a = b$ כדרוש. 
			
			משום ש־$f \co \F_p \to \F_p$ שדה סופי אז $f$ על. סה"כ $f$ זיווג. משום ש־$f(f(a)) = f(a\op) = (a\op)\op = a$, אז לכל $a$ קיים ויחיד $b = a\op$. אם $b = a$ אז $a = a\op$ ואז $aa\op = 1$ ומטענה שראינו בסעיף 1א נקבל $a = 1, -1$. משום ש־$-1  \equiv_p p -1$ בשדה, אז $a = 1, p - 1$. עבור כל $a$ אחר קיים ויחיד ב־$\F_p$ איזשהו $b = a\op \neq a$ אחר ובאופן דומה $a = b\op$, כלומר נוכל לסדר את השוויון להלן באופן הבא: 
			\[ (p-1)! = 1 \cdot 2 \cdots (p - 1) = (p - 1) \cdot 1 \cdot \underbrace{(2 \cdot 2\op)}_{=1}\underbrace{(3 \cdot 3\op)}_{=1} \cdots = 1(p - 1)\prod_{j = 2}^{p}1 = (p - 1) = -1 \]
			(בה"כ $2 \neq 3\op$, אחרת סימונים אחרים). השוויונות לעיל בתוך השדה $\F_p$ (כלומר, אלו שוויונות של מחלקות שקילות ולא של מספרים). 
			
			הפירוק לזוגות כאלו חוקי משום שהראינו ש־$f$, הפונקציה שהופכת את איברי השדה, חח"ע, על, וגם $(f \circ f) = id$. 
		\end{proof}
		\item \begin{proof}
			יהי $p$ ראשוני. אזי מהסעיף הקודם $(p - 1)! \equiv_p -1 \equiv_p p - 1$. נוסיף $1$ לשני האגפים ונסיק $(p - 1)! + 1 \equiv_p p - 1 + 1 = p$. סה"כ נקבל לפי הגדרת $\equiv_p$ ש־$\exists a \in \Z \co ap = (p - 1)!$, ובמילים אחרות, $1 + (p - 1)!$ כפולה שלמה של $p$, כדרוש. 
		\end{proof}
	\end{enumerate}
	
	\section{}
	שאלת בונוס: נראה כי $\Z_n$ שדה אמ"מ $n$ ראשוני. 
	\begin{proof}
		יהי $n \in \N$. אז: 
		\begin{itemize}
			\item[$\implies$] נניח $\Z_n$ שדה, נראה $n$ ראשוני. נניח בשלילה $n$ אינו ראשוני, ומההגדרה $\exists \N \ni b, c \notin \{1, n\} \co bc = n$. אזי $0_\F = (bc)_\F = (b)_\F (c)_\F$ וסה"כ ממשפט $b \equiv 0 \lor c \equiv 0$, בה"כ $b = 0$ ואז $n = bc = 0c = 0$ וסה"כ קיבלנו שדה מגודל $n = 0$ ולכן לא קיים שום איבר ובפרט לא קיים איבר יחידה, וסתירה להנחת השלילה כדרוש. 
			\item[$\impliedby$]יהי $n$ ראשוני, נראה $\Z_n$ שדה. ראינו כי $\Z_n$ מקיים את כל תכונות השדה פרט לקיום מספר הופכי ללא תלות בהיות $n$ ראשוני. נותר להראות קיום איבר הופכי. יהי $a \in \Z_n$, נראה קיום $a\op$ כך ש־$a\op a \equiv_n 1$. ממשפט שמצ"ב בש.ב. קיימים $\lg_1, \lg_2$ כך ש־$\lg_1 a + \lg_2 p = \gcd(a, p)$. בגלל ש־$p$ ראשוני, אז $\gcd(a, p) = 1$ בעבור כל $a \in \N$ ובפרט $a \in \Z_n$. לכן $\lg_1 a + \lg_2 p = 1$, נעביר אגפים ונקבל $\lg_1 a = 1 - \lg_2p \equiv_p 1$ כדרוש. 
		\end{itemize}
		סה"כ הראינו גרירה משני הכיוונים ובכך הטענה הוכחה. 
	\end{proof}
	
	\section{}
	\begin{enumerate}[(A)]
		\item 
		\begin{align*}
			\begin{cases}
				3x_2 + 2x_2 - 3x_ + x_4\! &\!= 7 \\
				2x_1 + x_2 - 3x_3 - x_4\! &\!= 11
			\end{cases} \tomat&\pms{3 & 2 & -1 & 1 & 7 \\ 
				2 & 1 & -3 & -1 & 11 \\ 
			} \rrr{R_1 \to \frac{1}{3}R_1}\pms{1 & \frac{2}{3} & -\frac{1}{3} & \frac{1}{3} & \frac{7}{3} \\ 
				2 & 1 & -3 & -1 & 11 \\ 
			} \\\rrr{R_2 \to R_2 - 2 R_1}&\pms{1 & \frac{2}{3} & -\frac{1}{3} & \frac{1}{3} & \frac{7}{3} \\ 
				0 & -\frac{1}{3} & -\frac{7}{3} & -\frac{5}{3} & \frac{19}{3} \\ 
			} \rrr{R_2 \to -3R_2}\pms{1 & \frac{2}{3} & -\frac{1}{3} & \frac{1}{3} & \frac{7}{3} \\ 
				0 & 1 & 7 & 5 & -19 \\ 
			} \\\rrr{R_1 \to R_1 - \frac{2}{3} R_2}&\pms{1 & 0 & -5 & -3 & 15 \\ 
				0 & 1 & 7 & 5 & -19 \\ 
			}
		\end{align*}
		נסיק: 
		\[ \mathrm{sols} = \ccb{\pms{15 + 5x_3 + 3x_4 \\ -19 - 7x_3 - 5x_4 \\ x_3 \\ x_4} \mid x_3, x_4 \in \R} \]
		ובפרט $x_3, x_4$ חופשיים. 
		\npage
		\item 
		\begin{gather*}\tomat \pms{3 & 2 & 2 & 2 & 2 \\ 
				2 & 3 & 2 & 5 & 3 \\ 
				9 & 1 & 4 & -5 & 1 \\ 
				2 & 2 & 3 & 4 & 5 \\ 
				7 & 1 & 6 & -4 & 7 \\ 
			} \rrr{R_1 \to R_1 \cdot \frac{1}{3}} \pms{1 & \frac{2}{3} & \frac{2}{3} & \frac{2}{3} & \frac{2}{3} \\ 
				2 & 3 & 2 & 5 & 3 \\ 
				9 & 1 & 4 & -5 & 1 \\ 
				2 & 2 & 3 & 4 & 5 \\ 
				7 & 1 & 6 & -4 & 7 \\ 
			} \rrt{R_2 \to R_2 - 2 \cdot R_1}{R_3 \to R_3 - 9 \cdot R_1}\rrt{R_4 \to R_4 - 2 \cdot R_1}{R_5 \to R_5 - 7 \cdot R_1} \pms{1 & \frac{2}{3} & \frac{2}{3} & \frac{2}{3} & \frac{2}{3} \\ 
				0 & \frac{5}{3} & \frac{2}{3} & \frac{11}{3} & \frac{5}{3} \\ 
				0 & -5 & -2 & -11 & -5 \\ 
				0 & \frac{2}{3} & \frac{5}{3} & \frac{8}{3} & \frac{11}{3} \\ 
				0 & -\frac{11}{3} & \frac{4}{3} & -\frac{26}{3} & \frac{7}{3} \\ 
			} \\\rrr{R_2 \to R_2 \cdot \frac{3}{5}} \pms{1 & \frac{2}{3} & \frac{2}{3} & \frac{2}{3} & \frac{2}{3} \\ 
				0 & 1 & \frac{2}{5} & \frac{11}{5} & 1 \\ 
				0 & -5 & -2 & -11 & -5 \\ 
				0 & \frac{2}{3} & \frac{5}{3} & \frac{8}{3} & \frac{11}{3} \\ 
				0 & -\frac{11}{3} & \frac{4}{3} & -\frac{26}{3} & \frac{7}{3} \\ 
			} \rrt{R_4 \to R_4 - \frac{2}{3} \cdot R_2}{\overset{R_5 \to R_5 + \frac{11}{3} \cdot R_2}{R_3 \to R_3 - (-5) \cdot R_2}} \pms{1 & \frac{2}{3} & \frac{2}{3} & \frac{2}{3} & \frac{2}{3} \\ 
				0 & 1 & \frac{2}{5} & \frac{11}{5} & 1 \\ 
				0 & 0 & 0 & 0 & 0 \\ 
				0 & 0 & \frac{7}{5} & \frac{6}{5} & 3 \\ 
				0 & 0 & \frac{14}{5} & -\frac{3}{5} & 6 \\ 
			} \rrr{R_3 \siff R_5} \pms{1 & \frac{2}{3} & \frac{2}{3} & \frac{2}{3} & \frac{2}{3} \\ 
				0 & 1 & \frac{2}{5} & \frac{11}{5} & 1 \\ 
				0 & 0 & \frac{14}{5} & -\frac{3}{5} & 6 \\ 
				0 & 0 & \frac{7}{5} & \frac{6}{5} & 3 \\ 
				0 & 0 & 0 & 0 & 0 \\ 
			} \\\rrr{R_3 \to R_3 \cdot \frac{5}{14}} \pms{1 & \frac{2}{3} & \frac{2}{3} & \frac{2}{3} & \frac{2}{3} \\ 
				0 & 1 & \frac{2}{5} & \frac{11}{5} & 1 \\ 
				0 & 0 & 1 & -\frac{3}{14} & \frac{15}{7} \\ 
				0 & 0 & \frac{7}{5} & \frac{6}{5} & 3 \\ 
				0 & 0 & 0 & 0 & 0 \\ 
			} \rrr{R_4 \to R_4 - \frac{7}{5} \cdot R_3} \pms{1 & \frac{2}{3} & \frac{2}{3} & \frac{2}{3} & \frac{2}{3} \\ 
				0 & 1 & \frac{2}{5} & \frac{11}{5} & 1 \\ 
				0 & 0 & 1 & -\frac{3}{14} & \frac{15}{7} \\ 
				0 & 0 & 0 & \frac{3}{2} & 0 \\ 
				0 & 0 & 0 & 0 & 0 \\ 
			} \rrr{R_4 \to R_4 \cdot \frac{2}{3}} \pms{1 & \frac{2}{3} & \frac{2}{3} & \frac{2}{3} & \frac{2}{3} \\ 
				0 & 1 & \frac{2}{5} & \frac{11}{5} & 1 \\ 
				0 & 0 & 1 & -\frac{3}{14} & \frac{15}{7} \\ 
				0 & 0 & 0 & 1 & 0 \\ 
				0 & 0 & 0 & 0 & 0 \\ 
			} \\\rrr{R_5 \siff R_5} \pms{1 & \frac{2}{3} & \frac{2}{3} & \frac{2}{3} & \frac{2}{3} \\ 
				0 & 1 & \frac{2}{5} & \frac{11}{5} & 1 \\ 
				0 & 0 & 1 & \frac{3}{-14} & \frac{15}{7} \\ 
				0 & 0 & 0 & 1 & 0 \\ 
				0 & 0 & 0 & 0 & 0 \\ 
			} \rrt{R_2 \to R_2 - \frac{11}{5} R_4}{\overset{R_1 \to R_1 - \frac{2}{3} R_4}{R_3 \to R_3 + \frac{3}{14} R_4}} \pms{1 & \frac{2}{3} & \frac{2}{3} & 0 & \frac{2}{3} \\ 
				0 & 1 & \frac{2}{5} & 0 & 1 \\ 
				0 & 0 & 1 & 0 & \frac{15}{7} \\ 
				0 & 0 & 0 & 1 & 0 \\ 
				0 & 0 & 0 & 0 & 0 \\ 
			} \rrt{R_2 \to R_2 - \frac{2}{5} R_3}{R_1 \to R_1 - \frac{2}{3} R_3} \pms{1 & \frac{2}{3} & 0 & 0 & -\frac{16}{21} \\ 
				0 & 1 & 0 & 0 & \frac{1}{7} \\ 
				0 & 0 & 1 & 0 & \frac{15}{7} \\ 
				0 & 0 & 0 & 1 & 0 \\ 
				0 & 0 & 0 & 0 & 0 \\ 
			} \\\rrr{R_1 \to R_1 - \frac{2}{3} R_2} \pms{1 & 0 & 0 & 0 & -\frac{6}{7} \\ 
				0 & 1 & 0 & 0 & \frac{1}{7} \\ 
				0 & 0 & 1 & 0 & \frac{15}{7} \\ 
				0 & 0 & 0 & 1 & 0 \\ 
				0 & 0 & 0 & 0 & 0 \\ 
			} \end{gather*}
		לכן אין משתנים חופשיים, והפתרון היחיד הוא: 
		\[ \pms{x_1 \\ x_2 \\ x_3 \\ x_4} = \pms{-\frac{6}{7} \\ \frac{1}{7} \\ \frac{15}{7} \\ 0} \]
		\item 
		\begin{gather*}\begin{cases}
				2x_1 + 3x_2 + 4x_3 &\!\!= 3 \\
				3x_1 - 2x-2 + 5x_3 &\!\!= 10 \\
				7x_1 + 9x_2 + 3x_3 &\!\!= 1
			\end{cases}\tomat\pms{2 & 3 & 4 & 3 \\ 
				3 & -2 & 5 & 10 \\ 
				7 & 9 & 3 & 1 \\ 
			} \rrr{R_1 \to 0.5R_1}\pms{1 & \frac{3}{2} & 2 & \frac{3}{2} \\ 
				3 & -2 & 5 & 10 \\ 
				7 & 9 & 3 & 1 \\ 
			} \\\rrt{R_2 \to R_2 - 3 R_1}{R_3 \to R_3 - 7 R_1}\pms{1 & \frac{3}{2} & 2 & \frac{3}{2} \\ 
				0 & -\frac{13}{2} & -1 & \frac{11}{2} \\ 
				0 & -\frac{3}{2} & -11 & -\frac{19}{2} \\ 
			} \rrr{R_2 \to \frac{2}{13}R_2}\pms{1 & \frac{3}{2} & 2 & \frac{3}{2} \\ 
				0 & 1 & \frac{2}{13} & -\frac{11}{13} \\ 
				0 & -\frac{3}{2} & -11 & -\frac{19}{2} \\ 
			} \rrr{R_3 \to R_3 + \frac{3}{2}R_2}\pms{1 & \frac{3}{2} & 2 & \frac{3}{2} \\ 
				0 & 1 & \frac{2}{13} & -\frac{11}{13} \\ 
				0 & 0 & -\frac{140}{13} & -\frac{140}{13} \\ 
			} \\\rrr{R_3 \to -\frac{13}{140}R_3}\pms{1 & \frac{3}{2} & 2 & \frac{3}{2} \\ 
				0 & 1 & \frac{2}{13} & -\frac{11}{13} \\ 
				0 & 0 & 1 & 1 \\ 
			} \rrt{R_2 \to R_2 - \frac{2}{13} R_3}{R_1 \to R_1 - 2 \cdot R_3}\pms{1 & \frac{3}{2} & 0 & \frac{1}{-2} \\ 
				0 & 1 & 0 & -1 \\ 
				0 & 0 & 1 & 1 \\ 
			} \rrr{R_1 \to R_1 - \frac{3}{2} R_2}\pms{1 & 0 & 0 & 1 \\ 
				0 & 1 & 0 & -1 \\ 
				0 & 0 & 1 & 1 \\ 
			} \end{gather*}
		נסיק ש־$(x_1, x_2, x_3) = (1, -1, 1)$, וגם במקרה הזה אין משתנים חופשיים. 
	\end{enumerate}
	\section{}
	נמצא את המשתנים החופשיים במערכות המשוואות להלן. 
	\begin{enumerate}[(A)]
		\item מעל $\F_5$
		\begin{gather*}\begin{cases}
				3x_1 + 2x_2 + 4x_3 + 1x_4\!\! &= 2 \\
				2x_1 + x_2 + 2x_3 + 4x_4\!\! &= 1
			\end{cases} \tomat \pms{3 & 2 & 4 & 1 & 2 \\ 
				2 & 1 & 2 & 4 & 1 \\ 
			} \rrr{R_1 \to 2R_1} \pms{1 & 4 & 3 & 2 & 4 \\ 
				2 & 1 & 2 & 4 & 1 \\ 
			} \rrr{R_2 \to R_2 - 2R_1} \pms{1 & 4 & 3 & 2 & 4 \\ 
				0 & 3 & 1 & 0 & 3 \\ 
			} \\\rrr{R_2 \to 2R_2} \pms{1 & 4 & 3 & 2 & 4 \\ 
				0 & 1 & 2 & 0 & 1 \\ 
			} \rrr{R_1 \to R_1 - 4 R_2} \pms{1 & 0 & 0 & 2 & 0 \\ 
				0 & 1 & 2 & 0 & 1 \\ 
			} \end{gather*}
		קבוצת הפתרונות: 
		\[ \ccb{\pms{- 2x_4 \\ 1 - 2x_3 \\ x_3 \\ x_4}, \ x_3, x_4 \in \F_5} \]
		סה"כ $x_3, x_4$ משתנים חופשיים. 
		\item מעל $\F_7$
		\begin{gather*}\begin{cases}
				3x_1 + 2x_2 + 4x_3 + x_4\!\! &= 2 \\
				2x_1 + x-2 + 2x_3 + 4x_4\!\! &= 1
			\end{cases}\tomat \pms{3 & 2 & 4 & 1 & 2 \\ 
				2 & 1 & 2 & 4 & 1 \\ 
			} \rrr{R_1 \to 5R_1} \pms{1 & 3 & 6 & 5 & 3 \\ 
				2 & 1 & 2 & 4 & 1 \\ 
			} \rrr{R_2 \to R_2 - 2 R_1} \pms{1 & 3 & 6 & 5 & 3 \\ 
				0 & 2 & 4 & 1 & 2 \\ 
			} \\\rrr{R_2 \to 4R_2} \pms{1 & 3 & 6 & 5 & 3 \\ 
				0 & 1 & 2 & 4 & 1 \\ 
			} \rrr{R_1 \to R_1 - 3 R_2} \pms{1 & 0 & 0 & 0 & 0 \\ 
				0 & 1 & 2 & 4 & 1 \\ 
			} \end{gather*}
		קבוצת הפתרונות: 
		\[ \mathrm{sols} = \ccb{\pms{0 \\ 1 - 2x_3 - 4x_4\\ x_3 \\ x_4} \mid x_3, x_4 \in \F_7} \]
		אזי $x_3, x_4$ משתנים חופשיים. 
		\item מעל $\C$
		\begin{gather*}
			\begin{cases}
				x_1 + 2x_2 + 3x_3 + 4x_4\!\! &= 0 \\
				x_1 + ix_2 - x_3 - ix_4\!\! &= 1
			\end{cases}\tomat \pms{1 & 2 & 3 & 4 & 0 \\ 1 & i & -1 & -i & 1} \rrr{R_2 \to R_2 - R_1} \pms{1 & 2 & 3 & 4 & 0 \\ 0 & i - 2 & -4 & -i -4 & 1} \\
			\rrr{R_2 \to (-\frac{2}{5} - \frac{1}{5}i)R_2} \pms{1 & 2 & 3 & 4 & 0 \\ 0 & 1 & \frac{8}{5} + \frac{4}{5}i & \frac{7}{5} + \frac{6}{5}i & -\frac{2}{5} - \frac{1}{5}i} \rrr{R_1 \to R_1 - 2R_2} \pms{1 & 0 & -\frac{1}{5} - \frac{8}{5}i & \frac{6}{5} - \frac{12}{5}i & \frac{4}{5} + \frac{2}{5}i \\ 0 & 1 & \frac{8}{5} + \frac{4}{5}i & \frac{7}{5} + \frac{6}{5}i & -\frac{2}{5} - \frac{1}{5}i}
		\end{gather*}
		קבוצת הפתרונות:
		\[ \mathrm{sols} = \ccb{\pms{\frac{4}{5} + \frac{2}{5}i + (\frac{1}{5} + \frac{1}{8})x_3 + (-\frac{6}{5} + \frac{12}{5}i)x_4 \\ -\frac{2}{5} - \frac{1}{5}i - (\frac{8}{5} + \frac{4}{5}i)x_3 - (\frac{7}{5} + \frac{6}{5}i)x_4 \\ x_3 \\ x_4} \mid x_3, x_4 \in \C} \]
		סה"כ $x_3, x_4$ משתנים חופשיים. 
		\item מעל $\F_2 = \{0, 1\}$
		\begin{gather*}\begin{cases}
				x_1 + x_2 + x_4 + x_5\!\!\! &= 1 \\
				x_1 + x_3 + x_4 &= 0 \\
				x_2 + x_4 + x_5 &= 1
			\end{cases}\tomat \pms{1 & 1 & 0 & 1 & 1 & 1 \\ 
				1 & 0 & 1 & 1 & 0 & 1 \\ 
				0 & 1 & 0 & 1 & 1 & 1 \\ 
			} \rrr{R_2 \to R_2 - R_1} \pms{1 & 1 & 0 & 1 & 1 & 1 \\ 
				0 & 1 & 1 & 0 & 1 & 0 \\ 
				0 & 1 & 0 & 1 & 1 & 1 \\ 
			} \\\rrr{R_3 \to R_3 - R_2} \pms{1 & 1 & 0 & 1 & 1 & 1 \\ 
				0 & 1 & 1 & 0 & 1 & 0 \\ 
				0 & 0 & 1 & 1 & 0 & 1 \\ 
			} \rrr{R_2 \to R_2 - R_3} \pms{1 & 1 & 0 & 1 & 1 & 1 \\ 
				0 & 1 & 0 & 1 & 1 & 1 \\ 
				0 & 0 & 1 & 1 & 0 & 1 \\ 
			} \rrr{R_1 \to R_1 - R_2} \pms{1 & 0 & 0 & 0 & 0 & 0 \\ 
				0 & 1 & 0 & 1 & 1 & 1 \\ 
				0 & 0 & 1 & 1 & 0 & 1 \\ 
			} \end{gather*}
		קבוצת הפתרונות: 
		\[ \mathrm{sols} = \ccb{\pms{0 \\ 1 - x_4 - x_5 \\ 1 - x_4 \\ x_4 \\ x_5} \mid x_4, x_5 \in \{0, 1\}} \]
		נבחין כי $x_4, x_5$ משתנים חופשיים. 
	\end{enumerate}
	\section{}
	\begin{enumerate}[(A)]
		\item להלן כל המטריצות $2 \times 2$ המדורגות קאנונית מעל $\F$ כלשהו: 
		\[ \ccb{\pms{1 & 0 \\ 0 & 1}, \pms{0 & 1 \\ 0 & 0}, \pms{0 & 0\\ 0& 0}} \uplus \ccb{\pms{1 & c \\ 0 & 0} \mid c \in \F} \]
		\item להלן כל המטריצות $3 \times 3$ המדורגות קאנונית מעל $\F$ כלשהו: 
		\begin{multline*}
			\ccb{I_{3}, 0_{3}, \pms{0 & 1 & 0 \\ 0 & 0 & 1 \\ 0 & 0 & 0}, \pms{0 & 0 & 1 \\ 0 & 0 & 0 \\ 0& 0& 0}} \uplus \ccb{\pms{1 & 0 & a \\ 0 & 1 & b \\ 0 & 0 & 0}\middle\vert\, a, b \in \F} \uplus \ccb{\pms{1 & a & b \\ 0 & 0 & 0 \\ 0 & 0 & 0} \middle\vert\, a, b \in \F} \\ \uplus \ccb{\pms{1 & a & 0 \\ 0 & 0 & 1 \\ 0 & 0 & 0} \middle\vert\,  a \in \F} \uplus \ccb{\pms{0 & 1 & a  \\ 0 & 0 & 0 \\ 0& 0 & 0} \middle\vert\,  a \in \F}
		\end{multline*}
	\end{enumerate}
	
	\ndoc
\end{document}