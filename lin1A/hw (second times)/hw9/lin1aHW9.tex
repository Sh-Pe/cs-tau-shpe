%! ~~~ Packages Setup ~~~ 
\documentclass[]{article}
\usepackage{lipsum}
\usepackage{rotating}


% Math packages
\usepackage[usenames]{color}
\usepackage{forest}
\usepackage{ifxetex,ifluatex,amssymb,amsmath,mathrsfs,amsthm,witharrows,mathtools,mathdots}
\usepackage{amsmath}
\WithArrowsOptions{displaystyle}
\renewcommand{\qedsymbol}{$\blacksquare$} % end proofs with \blacksquare. Overwrites the defualts. 
\usepackage{cancel,bm}
\usepackage[thinc]{esdiff}


% tikz
\usepackage{tikz}
\usetikzlibrary{graphs}
\newcommand\sqw{1}
\newcommand\squ[4][1]{\fill[#4] (#2*\sqw,#3*\sqw) rectangle +(#1*\sqw,#1*\sqw);}


% code 
\usepackage{algorithm2e}
\usepackage{listings}
\usepackage{xcolor}

\definecolor{codegreen}{rgb}{0,0.35,0}
\definecolor{codegray}{rgb}{0.5,0.5,0.5}
\definecolor{codenumber}{rgb}{0.1,0.3,0.5}
\definecolor{codeblue}{rgb}{0,0,0.5}
\definecolor{codered}{rgb}{0.5,0.03,0.02}
\definecolor{codegray}{rgb}{0.96,0.96,0.96}

\lstdefinestyle{pythonstylesheet}{
    language=Java,
    emphstyle=\color{deepred},
    backgroundcolor=\color{codegray},
    keywordstyle=\color{deepblue}\bfseries\itshape,
    numberstyle=\scriptsize\color{codenumber},
    basicstyle=\ttfamily\footnotesize,
    commentstyle=\color{codegreen}\itshape,
    breakatwhitespace=false, 
    breaklines=true, 
    captionpos=b, 
    keepspaces=true, 
    numbers=left, 
    numbersep=5pt, 
    showspaces=false,                
    showstringspaces=false,
    showtabs=false, 
    tabsize=4, 
    morekeywords={as,assert,nonlocal,with,yield,self,True,False,None,AssertionError,ValueError,in,else},              % Add keywords here
    keywordstyle=\color{codeblue},
    emph={var, List, Iterable, Iterator},          % Custom highlighting
    emphstyle=\color{codered},
    stringstyle=\color{codegreen},
    showstringspaces=false,
    abovecaptionskip=0pt,belowcaptionskip =0pt,
    framextopmargin=-\topsep, 
}
\newcommand\pythonstyle{\lstset{pythonstylesheet}}
\newcommand\pyl[1]     {{\lstinline!#1!}}
\lstset{style=pythonstylesheet}

\usepackage[style=1,skipbelow=\topskip,skipabove=\topskip,framemethod=TikZ]{mdframed}
\definecolor{bggray}{rgb}{0.85, 0.85, 0.85}
\mdfsetup{leftmargin=0pt,rightmargin=0pt,innerleftmargin=15pt,backgroundcolor=codegray,middlelinewidth=0.5pt,skipabove=5pt,skipbelow=0pt,middlelinecolor=black,roundcorner=5}
\BeforeBeginEnvironment{lstlisting}{\begin{mdframed}\vspace{-0.4em}}
    \AfterEndEnvironment{lstlisting}{\vspace{-0.8em}\end{mdframed}}


% Design
\usepackage[labelfont=bf]{caption}
\usepackage[margin=0.6in]{geometry}
\usepackage{multicol}
\usepackage[skip=4pt, indent=0pt]{parskip}
\usepackage[normalem]{ulem}
\forestset{default}
\renewcommand\labelitemi{$\bullet$}
\usepackage{titlesec}
\titleformat{\section}[block]
{\fontsize{15}{15}}
{\sen \dotfill (\thesection)\dotfill\she}
{0em}
{\MakeUppercase}
\usepackage{graphicx}
\graphicspath{ {./} }

\usepackage[colorlinks]{hyperref}
\definecolor{mgreen}{RGB}{25, 160, 50}
\definecolor{mblue}{RGB}{30, 60, 200}
\usepackage{hyperref}
\hypersetup{
    colorlinks=true,
    citecolor=mgreen,
    linkcolor=black,
    urlcolor=mblue,
    pdftitle={Document by Shahar Perets},
    %	pdfpagemode=FullScreen,
}
\usepackage{yfonts}
\def\gothstart#1{\noindent\smash{\lower3ex\hbox{\llap{\Huge\gothfamily#1}}}
    \parshape=3 3.1em \dimexpr\hsize-3.4em 3.4em \dimexpr\hsize-3.4em 0pt \hsize}
\def\frakstart#1{\noindent\smash{\lower3ex\hbox{\llap{\Huge\frakfamily#1}}}
    \parshape=3 1.5em \dimexpr\hsize-1.5em 2em \dimexpr\hsize-2em 0pt \hsize}



% Hebrew initialzing
\usepackage[bidi=basic]{babel}
\PassOptionsToPackage{no-math}{fontspec}
\babelprovide[main, import, Alph=letters]{hebrew}
\babelprovide[import]{english}
\babelfont[hebrew]{rm}{David CLM}
\babelfont[hebrew]{sf}{David CLM}
%\babelfont[english]{tt}{Monaspace Xenon}
\usepackage[shortlabels]{enumitem}
\newlist{hebenum}{enumerate}{1}

% Language Shortcuts
\newcommand\en[1] {\begin{otherlanguage}{english}#1\end{otherlanguage}}
\newcommand\he[1] {\she#1\sen}
\newcommand\sen   {\begin{otherlanguage}{english}}
    \newcommand\she   {\end{otherlanguage}}
\newcommand\del   {$ \!\! $}

\newcommand\npage {\vfil {\hfil \textbf{\textit{המשך בעמוד הבא}}} \hfil \vfil \pagebreak}
\newcommand\ndoc  {\dotfill \\ \vfil {\begin{center}
            {\textbf{\textit{שחר פרץ, 2025}} \\
                \scriptsize \textit{קומפל ב־}\en{\LaTeX}\,\textit{ ונוצר באמצעות תוכנה חופשית בלבד}}
    \end{center}} \vfil	}

\newcommand{\rn}[1]{
    \textup{\uppercase\expandafter{\romannumeral#1}}
}

\makeatletter
\newcommand{\skipitems}[1]{
    \addtocounter{\@enumctr}{#1}
}
\makeatother

%! ~~~ Math shortcuts ~~~

% Letters shortcuts
\newcommand\N     {\mathbb{N}}
\newcommand\Z     {\mathbb{Z}}
\newcommand\R     {\mathbb{R}}
\newcommand\Q     {\mathbb{Q}}
\newcommand\C     {\mathbb{C}}
\newcommand\One   {\mathit{1}}

\newcommand\ml    {\ell}
\newcommand\mj    {\jmath}
\newcommand\mi    {\imath}

\newcommand\powerset {\mathcal{P}}
\newcommand\ps    {\mathcal{P}}
\newcommand\pc    {\mathcal{P}}
\newcommand\ac    {\mathcal{A}}
\newcommand\bc    {\mathcal{B}}
\newcommand\cc    {\mathcal{C}}
\newcommand\dc    {\mathcal{D}}
\newcommand\ec    {\mathcal{E}}
\newcommand\fc    {\mathcal{F}}
\newcommand\nc    {\mathcal{N}}
\newcommand\vc    {\mathcal{V}} % Vance
\newcommand\sca   {\mathcal{S}} % \sc is already definded
\newcommand\rca   {\mathcal{R}} % \rc is already definded

\newcommand\prm   {\mathrm{p}}
\newcommand\arm   {\mathrm{a}} % x86
\newcommand\brm   {\mathrm{b}}
\newcommand\crm   {\mathrm{c}}
\newcommand\drm   {\mathrm{d}}
\newcommand\erm   {\mathrm{e}}
\newcommand\frm   {\mathrm{f}}
\newcommand\nrm   {\mathrm{n}}
\newcommand\vrm   {\mathrm{v}}
\newcommand\srm   {\mathrm{s}}
\newcommand\rrm   {\mathrm{r}}

\newcommand\Si    {\Sigma}

% Logic & sets shorcuts
\newcommand\siff  {\longleftrightarrow}
\newcommand\ssiff {\leftrightarrow}
\newcommand\so    {\longrightarrow}
\newcommand\sso   {\rightarrow}

\newcommand\epsi  {\epsilon}
\newcommand\vepsi {\varepsilon}
\newcommand\vphi  {\varphi}
\newcommand\Neven {\N_{\mathrm{even}}}
\newcommand\Nodd  {\N_{\mathrm{odd }}}
\newcommand\Zeven {\Z_{\mathrm{even}}}
\newcommand\Zodd  {\Z_{\mathrm{odd }}}
\newcommand\Np    {\N_+}

% Text Shortcuts
\newcommand\open  {\big(}
\newcommand\qopen {\quad\big(}
\newcommand\close {\big)}
\newcommand\also  {\mathrm{, }}
\newcommand\defis {\mathrm{ definitions}}
\newcommand\given {\mathrm{given }}
\newcommand\case  {\mathrm{if }}
\newcommand\syx   {\mathrm{ syntax}}
\newcommand\rle   {\mathrm{ rule}}
\newcommand\other {\mathrm{else}}
\newcommand\set   {\ell et \text{ }}
\newcommand\ans   {\mathscr{A}\!\mathit{nswer}}

% Set theory shortcuts
\newcommand\ra    {\rangle}
\newcommand\la    {\langle}

\newcommand\oto   {\leftarrow}

\newcommand\QED   {\quad\quad\mathscr{Q.E.D.}\;\;\blacksquare}
\newcommand\QEF   {\quad\quad\mathscr{Q.E.F.}}
\newcommand\eQED  {\mathscr{Q.E.D.}\;\;\blacksquare}
\newcommand\eQEF  {\mathscr{Q.E.F.}}
\newcommand\jQED  {\mathscr{Q.E.D.}}

\DeclareMathOperator\dom   {dom}
\DeclareMathOperator\Img   {Im}
\DeclareMathOperator\range {range}

\newcommand\trio  {\triangle}

\newcommand\rc    {\right\rceil}
\newcommand\lc    {\left\lceil}
\newcommand\rf    {\right\rfloor}
\newcommand\lf    {\left\lfloor}
\newcommand\ceil  [1] {\lc #1 \rc}
\newcommand\floor [1] {\lf #1 \rf}

\newcommand\lex   {<_{lex}}

\newcommand\az    {\aleph_0}
\newcommand\uaz   {^{\aleph_0}}
\newcommand\al    {\aleph}
\newcommand\ual   {^\aleph}
\newcommand\taz   {2^{\aleph_0}}
\newcommand\utaz  { ^{\left (2^{\aleph_0} \right )}}
\newcommand\tal   {2^{\aleph}}
\newcommand\utal  { ^{\left (2^{\aleph} \right )}}
\newcommand\ttaz  {2^{\left (2^{\aleph_0}\right )}}

\newcommand\n     {$n$־יה\ }

% Math A&B shortcuts
\newcommand\logn  {\log n}
\newcommand\logx  {\log x}
\newcommand\lnx   {\ln x}
\newcommand\cosx  {\cos x}
\newcommand\sinx  {\sin x}
\newcommand\sint  {\sin \theta}
\newcommand\tanx  {\tan x}
\newcommand\tant  {\tan \theta}
\newcommand\sex   {\sec x}
\newcommand\sect  {\sec^2}
\newcommand\cotx  {\cot x}
\newcommand\cscx  {\csc x}
\newcommand\sinhx {\sinh x}
\newcommand\coshx {\cosh x}
\newcommand\tanhx {\tanh x}

\newcommand\seq   {\overset{!}{=}}
\newcommand\slh   {\overset{LH}{=}}
\newcommand\sle   {\overset{!}{\le}}
\newcommand\sge   {\overset{!}{\ge}}
\newcommand\sll   {\overset{!}{<}}
\newcommand\sgg   {\overset{!}{>}}

\newcommand\h     {\hat}
\newcommand\ve    {\vec}
\newcommand\lv    {\overrightarrow}
\newcommand\ol    {\overline}

\newcommand\mlcm  {\mathrm{lcm}}

\DeclareMathOperator{\sech}   {sech}
\DeclareMathOperator{\csch}   {csch}
\DeclareMathOperator{\arcsec} {arcsec}
\DeclareMathOperator{\arccot} {arcCot}
\DeclareMathOperator{\arccsc} {arcCsc}
\DeclareMathOperator{\arccosh}{arccosh}
\DeclareMathOperator{\arcsinh}{arcsinh}
\DeclareMathOperator{\arctanh}{arctanh}
\DeclareMathOperator{\arcsech}{arcsech}
\DeclareMathOperator{\arccsch}{arccsch}
\DeclareMathOperator{\arccoth}{arccoth}
\DeclareMathOperator{\atant}  {atan2} 
\DeclareMathOperator{\Sp}     {span} 
\DeclareMathOperator{\sgn}    {sgn} 
\DeclareMathOperator{\row}    {Row} 
\DeclareMathOperator{\adj}    {adj} 
\DeclareMathOperator{\rk}     {rank} 
\DeclareMathOperator{\col}    {Col} 
\DeclareMathOperator{\tr}     {tr}

\newcommand\dx    {\,\mathrm{d}x}
\newcommand\dt    {\,\mathrm{d}t}
\newcommand\dtt   {\,\mathrm{d}\theta}
\newcommand\du    {\,\mathrm{d}u}
\newcommand\dv    {\,\mathrm{d}v}
\newcommand\df    {\mathrm{d}f}
\newcommand\dfdx  {\diff{f}{x}}
\newcommand\dit   {\limhz \frac{f(x + h) - f(x)}{h}}

\newcommand\nt[1] {\frac{#1}{#1}}

\newcommand\limz  {\lim_{x \to 0}}
\newcommand\limxz {\lim_{x \to x_0}}
\newcommand\limi  {\lim_{x \to \infty}}
\newcommand\limh  {\lim_{x \to 0}}
\newcommand\limni {\lim_{x \to - \infty}}
\newcommand\limpmi{\lim_{x \to \pm \infty}}

\newcommand\ta    {\theta}
\newcommand\ap    {\alpha}

\renewcommand\inf {\infty}
\newcommand  \ninf{-\inf}

% Combinatorics shortcuts
\newcommand\sumnk     {\sum_{k = 0}^{n}}
\newcommand\sumni     {\sum_{i = 0}^{n}}
\newcommand\sumnko    {\sum_{k = 1}^{n}}
\newcommand\sumnio    {\sum_{i = 1}^{n}}
\newcommand\sumai     {\sum_{i = 1}^{n} A_i}
\newcommand\nsum[2]   {\reflectbox{\displaystyle\sum_{\reflectbox{\scriptsize$#1$}}^{\reflectbox{\scriptsize$#2$}}}}

\newcommand\bink      {\binom{n}{k}}
\newcommand\setn      {\{a_i\}^{2n}_{i = 1}}
\newcommand\setc[1]   {\{a_i\}^{#1}_{i = 1}}

\newcommand\cupain    {\bigcup_{i = 1}^{n} A_i}
\newcommand\cupai[1]  {\bigcup_{i = 1}^{#1} A_i}
\newcommand\cupiiai   {\bigcup_{i \in I} A_i}
\newcommand\capain    {\bigcap_{i = 1}^{n} A_i}
\newcommand\capai[1]  {\bigcap_{i = 1}^{#1} A_i}
\newcommand\capiiai   {\bigcap_{i \in I} A_i}

\newcommand\xot       {x_{1, 2}}
\newcommand\ano       {a_{n - 1}}
\newcommand\ant       {a_{n - 2}}

% Linear Algebra
\DeclareMathOperator{\chr}     {char}
\DeclareMathOperator{\diag}    {diag}
\DeclareMathOperator{\Hom}     {Hom}
\DeclareMathOperator{\Sym}     {Sym}
\DeclareMathOperator{\Asym}    {ASym}

\newcommand\lra       {\leftrightarrow}
\newcommand\chrf      {\chr(\F)}
\newcommand\F         {\mathbb{F}}
\newcommand\co        {\colon}
\newcommand\tmat[2]   {\cl{\begin{matrix}
            #1
        \end{matrix}\, \middle\vert\, \begin{matrix}
            #2
\end{matrix}}}

\makeatletter
\newcommand\rrr[1]    {\xxrightarrow{1}{#1}}
\newcommand\rrt[2]    {\xxrightarrow{1}[#2]{#1}}
\newcommand\mat[2]    {M_{#1\times#2}}
\newcommand\gmat      {\mat{m}{n}(\F)}
\newcommand\tomat     {\, \dequad \longrightarrow}
\newcommand\pms[1]    {\begin{pmatrix}
        #1
\end{pmatrix}}
\newcommand\detms[1]   {\sof{\begin{matrix}
            #1
\end{matrix}}}

% someone's code from the internet: https://tex.stackexchange.com/questions/27545/custom-length-arrows-text-over-and-under
\makeatletter
\newlength\min@xx
\newcommand*\xxrightarrow[1]{\begingroup
    \settowidth\min@xx{$\m@th\scriptstyle#1$}
    \@xxrightarrow}
\newcommand*\@xxrightarrow[2][]{
    \sbox8{$\m@th\scriptstyle#1$}  % subscript
    \ifdim\wd8>\min@xx \min@xx=\wd8 \fi
    \sbox8{$\m@th\scriptstyle#2$} % superscript
    \ifdim\wd8>\min@xx \min@xx=\wd8 \fi
    \xrightarrow[{\mathmakebox[\min@xx]{\scriptstyle#1}}]
    {\mathmakebox[\min@xx]{\scriptstyle#2}}
    \endgroup}
\makeatother


% Greek Letters
\newcommand\ag        {\alpha}
\newcommand\bg        {\beta}
\newcommand\cg        {\gamma}
\newcommand\dg        {\delta}
\newcommand\eg        {\epsi}
\newcommand\zg        {\zeta}
\newcommand\hg        {\eta}
\newcommand\tg        {\theta}
\newcommand\ig        {\iota}
\newcommand\kg        {\keppa}
\renewcommand\lg      {\lambda}
\newcommand\og        {\omicron}
\newcommand\rg        {\rho}
\newcommand\sg        {\sigma}
\newcommand\yg        {\usilon}
\newcommand\wg        {\omega}

\newcommand\Ag        {\Alpha}
\newcommand\Bg        {\Beta}
\newcommand\Cg        {\Gamma}
\newcommand\Dg        {\Delta}
\newcommand\Eg        {\Epsi}
\newcommand\Zg        {\Zeta}
\newcommand\Hg        {\Eta}
\newcommand\Tg        {\Theta}
\newcommand\Ig        {\Iota}
\newcommand\Kg        {\Keppa}
\newcommand\Lg        {\Lambda}
\newcommand\Og        {\Omicron}
\newcommand\Rg        {\Rho}
\newcommand\Sg        {\Sigma}
\newcommand\Yg        {\Usilon}
\newcommand\Wg        {\Omega}

% Other shortcuts
\newcommand\tl    {\tilde}
\newcommand\op    {^{-1}}

\newcommand\sof[1]    {\left | #1 \right |}
\newcommand\cl [1]    {\left ( #1 \right )}
\newcommand\csb[1]    {\left [ #1 \right ]}
\newcommand\ccb[1]    {\left \{ #1 \right \}}

\newcommand\bs        {\blacksquare}
\newcommand\dequad    {\!\!\!\!\!\!}
\newcommand\dequadd   {\dequad\duquad}


\newtheorem{Theorem}{משפט}
\theoremstyle{definition}
\newtheorem{definition}{הגדרה}
\newtheorem{Lemma}{למה}
\newtheorem{Remark}{הערה}
\newtheorem{Notion}{סימון}


\newcommand\theo  [1] {\begin{Theorem}#1\end{Theorem}}
\newcommand\defi  [1] {\begin{definition}#1\end{definition}}
\newcommand\rmark [1] {\begin{Remark}#1\end{Remark}}
\newcommand\lem   [1] {\begin{Lemma}#1\end{Lemma}}
\newcommand\noti  [1] {\begin{Notion}#1\end{Notion}}

% DS
\newcommand\limsi     {\limsup_{n \to \inf}}
\newcommand\limfi     {\liminf_{n \to \inf}}

\DeclareMathOperator\amort   {amort}
\DeclareMathOperator\worst   {worst}
\DeclareMathOperator\type    {type}
\DeclareMathOperator\cost    {cost}
\DeclareMathOperator\tim     {time}

\newcommand\dsList{
    \sFunc{List}
    \sFunc{Retrieve}
    \SetKwFunction{RetrieveFirst}{Retrieve-First}
    \SetKwFunction{RetrieveLast}{Retrieve-Last}
    \sFunc{Delete}
    \SetKwFunction{DeleteFirst}{Delete-First}
    \SetKwFunction{DeleteLast}{Delete-Last}
    \sFunc{Insert}
    \SetKwFunction{InsertFirst}{Insert-First}
    \SetKwFunction{InsertLast}{Insert-Last}
    \sFunc{Shift}
    \sFunc{Length}
    \sFunc{Concat}
    \sFunc{Plant}
    \sFunc{Split}
}
\newcommand\dsQueue{
    \sFunc{Queue}
    \sFunc{Enqueue}
    \sFunc{Head}
    \sFunc{Dequeue}
}
\newcommand\dsStack{
    \sFunc{Stack}
    \sFunc{Push}
    \sFunc{Top}
    \sFunc{Pop}
}
\newcommand\dsVector{
    \sFunc{Vector}
    \sFunc{Get}
    \sFunc{Set}
}
\newcommand\dsGraph{
    \sFunc{Graph}
    \sFunc{Edge}
    \SetKwFunction{AddEdge}{Add-Edge}
    \SetKwFunction{RemoveEdge}{Remove-Edge}
    \sFunc{InDeg} \sFunc{OutDeg}
}
\newcommand\importDs{
    \dsList
    \dsQueue
    \dsStack
    \dsVector
    \dsGraph
    \SetKwData{error}{\color{codered}error}
    \SetKwInOut{Input}{input}
    \SetKwInOut{Output}{output}
    \SetKwRepeat{Do}{do}{while}
    \SetKwData{Null}{\color{codeblue}null}
}


% Algorithems
\newcommand\sFunc [1] {\SetKwFunction{#1}{#1}}
\newcommand\sData [1] {\SetKwData{#1}{#1}}
\newcommand\sIO   [1] {\SetKwInOut{#1}{#1}}
\newcommand\ttt   [1] {\sen \texttt{#1} \she\,}
\newcommand\io    [2] {\Input{#1}\Output{#2}\BlankLine}

%! ~~~ Document ~~~

\author{שחר פרץ}
\title{\textit{אלגברה לינארית 1א $\sim$ תרגיל בית 9 $\sim$ סמסטר ב' 2025}}
\begin{document}
    \maketitle
    \section{}
    
    נגדיר: 
    \[ B = \cl{\pms{1 & 1 \\ 1 & 1}, \ \pms{-1 & 1 \\ 1 & -1 }, \pms{1 & 1  \\ 1 & -1}, \pms{1 & 1  \\ -1 & 1}}, \ C = \cl{\pms{1 & 1 \\ 1 & 1}, \pms{-1 & 1  \\ 1 & -1 }, \pms{1 & 1 \\ 1 & -1 }} \]
    כאשר כל החישובים בסעיף זה יתבצעו תחת האיזומורפיזם הבא: 
    \[ \vphi \co M_{2 \times 2}(\F) \to \R^{4} = \vphi\pms{a & b \\ c & d} = \pms{a \\ b \\ c \\d} \]
    נמצא קרנל ותמונה להעתקה הבאה: 
    \[ T \co M_{2 \times 2} \to \Sym_{2}(\R), \quad [T]^{B}_{C} = \pms{1 & - 2 & 2 & 1 \\ 2 & -4 & 3 & 1 \\ -1 & 2 & 4 & 5} \]
    נמצא קרנל למטריצה באמצעות דירוג מערכת משוואות הומוגנית
    \begin{gather*}\pms{1 & -2 & 2 & 1 \\ 
            2 & -4 & 3 & 1 \\ 
            -1 & 2 & 4 & 5 \\ 
        } \rrt{R_2 \to R_2 - 2 \cdot R_1}{R_3 \to R_3 + R_1} \pms{1 & -2 & 2 & 1 \\ 
            0 & 0 & -1 & -1 \\ 
            0 & 0 & 6 & 6 \\ 
        } \rrr{R_3 \to R_3 + 6R_1} \pms{
            1 & -2 & 2 & 1 \\ 
            0 & 0 & -1 & -1 \\
            0 & 0 & 0 & 0
        } \rrt{R_1 \to R_1 + 2 R_3}{R_2 \to -R_2} \pms{
            1 & -1 & 0 & -1 \\
            0 & 0 & 1 & 1 \\
            0 & 0 & 0 & 0
        }
     \end{gather*}
     סה''כ: 
     \[ \nc[T]^{B}_C = \Sp\cl{\pms{1 \\ -1  \\ 0 \\ -1}, \pms{0 \\ 0  \\ 1 \\ 1}} \]
     נעביר ייצוג מבסיס $B$ לבסיס סטנדרטי כדי למצוא קרנל: 
     \[ \pms{1 \\ 1 \\ 0 \\ -1}[id]^{B}_E = 1\pms{1 & 1 \\ 1 & 1} -1\pms{-1 & 1 \\ 1 & -1} -1\pms{1 & 1 \\ -1 &1} = \pms{1 & - 1 \\ 1 & 1} \]
     \[ \pms{0 \\ 0 \\ 1 \\ 1}[id]^{B}_E = \pms{1 & 1 \\ 1 & -1} + \pms{1 & 1 \\ -1 & 1} = \pms{0 & 2 \\ 0 & 0} \]
     ולכן סה''כ: 
     \[ \ker T = \Sp\cl{\pms{1 & -1 \\ 1 & 1} , \pms{0 & 2 \\ 0 & 0}} \]
     עתה ניגש למצוא תמונה. לשם כך, נמצא בסיס למרחב העמודות של המטריצה המייצגת, ע''י כך שנדרג שורות ה־transpose שלה (דירוג שורות לא משנה מרחב שורות): 
     \begin{gather*}\tomat \pms{1 & 2 & -1 \\ 
             -2 & -4 & 2 \\ 
             2 & 3 & 4 \\ 
             1 & 1 & 5 \\ 
         } \rrt{\overset{R_2 \to R_2 + 2 R_1}{R_3 \to R_3 - 2 R_1}}{R_4 \to R_4 - R_1} \pms{1 & 2 & -1 \\ 
             0 & 0 & 0 \\ 
             0 & -1 & 6 \\ 
             0 & -1 & 6 \\ 
         } \rrr{R_2 \siff R_4} \pms{1 & 2 & -1 \\ 
             0 & -1 & 6 \\ 
             0 & -1 & 6 \\ 
             0 & 0 & 0 \\ 
         } \rrr{R_2 \to -R_2} \pms{1 & 2 & -1 \\ 
             0 & 1 & -6 \\ 
             0 & -1 & 6 \\ 
             0 & 0 & 0 \\ 
         } \\\rrr{R_3 \to R_3 + R_2} \pms{1 & 2 & -1 \\ 
             0 & 1 & -6 \\ 
             0 & 0 & 0 \\ 
             0 & 0 & 0 \\ 
         } \rrr{R_1 \to R_1 - 2 R_2} \pms{1 & 0 & 11 \\ 
             0 & 1 & -6 \\ 
             0 & 0 & 0 \\ 
             0 & 0 & 0 \\ 
         } \end{gather*}
     סה''כ, מרחב העמודות: 
     \[ \col [T]^{B}_C = \Sp\cl{\pms{1 & 0 & 11}, \pms{0 & 1 & -6}} \]
     זהו בייצוג $C$, נעביר חזרה לייצוג סטנדרטי: 
     \[ \pms{1 \\ 0 \\ 11}[id]^{C}_{E} = \pms{1 & 1 \\ 1 & 1} + 11\pms{1 & 1 \\ 1 & -1} = \pms{12 & 12 \\ 12 & -10} \]
     \[ \pms{0 \\ 1 \\ -6}[id]^{C}_E = \pms{-1 & 1 \\ 1 & -1} - 6\pms{1 & 1 \\ 1 & -1} = \pms{-7 & -5 \\ -5 & 5} \]
     וסה''כ: 
     \[ \Img T = \Sp\cl{\pms{12 & 12 \\ 12 & -10}, \pms{-7 & -5 \\ -5 & 5}} \]
     עתה ניגש למצוא את $[T]^{B'}_{C'}$ עבור הבסיסים הבאים:  
     \[ B' = \ccb{\pms{1 & 1 \\ -1 & 1}  ,\pms{1 & 1 \\ 1 & -1} ,\pms{-1 & 1 \\ 1 & -1},\pms{1 & 1 \\ 1 & 1}}, \quad C' = \cl{\pms{1 & 2 \\ 2 & 1}, \pms{2 & 3 \\ 3 & 1}, \pms{2 &1 \\ 1 & 2}} \]
     נבחין ש־: 
     \[ [T]^{B'}_{C} = [id]^{B'}_{B}[T]^{B}_{C} = \pms{1&2&-2&1 \\ 1&3&-4&2 \\ 5&4&2&-1} \]
     כי זהו רק שינוי סדר שורות. עתה נעביר את המטריצה לייצוג ב־$C$. נמצא את מטריצת המעבר מ־$C$ ל־$C'$:
     \[ [id]^{C}_{C'} = \cl{\csb{\pms{1 & 1 \\ 1 & 1}}_{C'}, \csb{\pms{-1 & 1 \\ 1 & -1}_{C'}, \csb{\pms{1 & 1 \\ 1 & -1}}_{C'}}} = \cl{\pms{\frac{1}{3} \\ 0 \\ \frac{1}{3}}, \pms{1 \\ 0 \\ -1}, \pms{-\frac{7}{3} \\ 2 \\ -\frac{1}{3}}} = \pms{\frac{1}{3} & -1 & -\frac{7}{3} \\ 0 & 0& 2 \\ \frac{1}{3} & -1 & -\frac{1}{3}} \]
     נכפול: 
     \[ [T]^{B'}_{C'} = [T]^{B'}_C[id]^{C}_{C'} = \pms{-12\frac{1}{3} & -11\frac{2}{3} & -1\frac{1}{3} & \frac{2}{3} \\ 10 & 8 & 4 & -2 \\ -2\frac{1}{3} & -3\frac{2}{3}& 2\frac{2}{3} & -1\frac{1}{3}} \]
     כדרוש. 
     
    
    \section{}
    \begin{enumerate}[(a)]
        \item יהיו מטריצות $A_1, A_2 \in M_{m \times n}(\F)$. נוכיח ש־$A_1, A_2$ שקולות שורה אם ורק אם קיימת $T \co \F^n \to \F^m$ ובסיס $B$ של $\F^n$ וכן בסיסים $C_1, C_2$ של $\F^m$ כך ש־$A_1 = [T]^B_{C_1}, \ A_2 = [T]^B_{C_2}$. 
        
        \begin{proof}\,
            \begin{enumerate} 
                \item[$\implies$]נניח $\row A_1 = \row A_2$, נוכיח קיום בסיסים $B, C_1, C_2$ וכן העתקה $T \co \F^{n} \to \F^{m}$ כך ש־$A_1 = [T]^{B}_{C_1}, \ A_2 = [T]^{B}_{C_2}$. נתבונן בבסיס $B$ כלשהו למרחב השורות $\row A_1 = \row A_2 =: \row A$. אזי ניתן לדרג את המטריצה המתקבלת מהבסיס $B$ כשורותיה, נסמנה $A$, אל $A_1$ וכן אל $A_2$. נוכל להתסכל על $A$ כעל מייצגת העתקה כלשהי בבסיס $B$ ל־$C$ כלשהו היא $[T]^{B}_C$, ועל הדירוג כעל כפל של מטריצות מעבר שורה -- דהיינו, מטריצה הפיכה, וכל מטריצה הפיכה היא מטריצת שינוי בסיס מבסיס נתון לבסיס כלשהו אחר, בה''כ $[id]^{C}_{C_1}$ וכן $[id]^{C}_{C_2}$ יהיו אותן המטריצות. סה''כ: 
                \begin{alignat*}{4}
                    A_1 & = A[id]^{C}_{C_1} &= [T]^{B}_{C}[id]^{C}_{C_1} &= [T]^{B}_{C_1} \\
                    A_2 & = A[id]^{C}_{C_2} &= [T]^{B}_{C}[id]^{C}_{C_2} &= [T]^{B}_{C_2}
                \end{alignat*}
                \item[$\impliedby$]נניח ששתי מטריצות $A_1, A_2$ מקיימות $A_1 = [T]^{B}_{C_1}, A_2 = [T]^{B}_{C_2}$ עבור $T \co \F^{n} \to \F^{m}$ העתקה כלשהי וכן $B, C_1, C_2$ בסיסים כלשהם. נוכיח $\row A_1 = \row A_2$. 
                
                יהי $C$ בסיס. נתבונן במטריצה $[id]^{C}_{C_1}$ מטריצת מעבר בסיס, היא מטריצה הפיכה. אזי היא כפל של מטריצות אלמנטריות, דהיינו שקולה לדירוג שורות. ידוע שכפל במטריצה הפיכה לא משנה שורות מטריצה, ולכן: 
                \[ \row A_1 = \row [T]^{B}_{C_1} = \row [T]^{B}_C[id]^{C}_{C_1} = \row [T]^{B}_C \]
                נפעיל טיעונים דומים בעבור $[id]^{C}_{C_2}$ ונקבל: 
                \[ \row A_2 = \row [T]^{B}_{C_2} = \row [T]^{B}_C[id]^{C}_{C_2} = \row [T]^{B}_C \]
                סה''כ מטרנזטיביות: 
                \[ \row A_1 = \row A_1 \quad \top \]
                
            \end{enumerate}
        \end{proof}
        
        \item יהיו $S, T \co V \to U$ העתקות ליניאריות. נוכיח ש־$\dim(\Img S) = \dim (\Img T)$ אמ''מ $\exists B_1, B_2$ בסיסים של $V$ וכן $C_1, C_2$ בסיסים של $U$ כך ש־$[S]^{B_2}_{C_2} = [T]^{B_1}_{C_1}$. \begin{proof}
            ראשית כל, נראה שעבור העתקה $\vphi \co V \to U$ ו־$B, C$ בסיסים ל־$V, U$ בהתאמה, מתקיים $\col [\vphi]^B_C \cong \Img \vphi$. ידוע: 
            \[ [\vphi(v)]_C = [\vphi]_C^B \cdot [v]_B \overset{(1)}{\in} \col [\vphi]_C^B \implies \col[\vphi]^B_C = \{[\vphi v]_C \mid v \in V\} \overset{(2)}{\cong} \{\vphi c \mid v \in V\} = \Img \vphi \]
            כאשר $(1)$ נכון מכפל מטריצות ו־$(2)$ נכון כי $[\, \cdot \,]_C \co U \to \F^m$ איזו'. סה''כ הראינו את הדרוש כלומר קיים איזו' בן $\col[\vphi]_C^B$ לבין $\Img \vphi$ הוא מעבר לייצוג בבסיס $C$, כלומר $\dim \col[\vphi]^B_C = \dim \Img \vphi$. 
            
            נחזור לטענה עצמה. 
            \begin{itemize}
                \item[$\implies$] אם $\exists B_1, B_2 \subseteq V, C_1, C_2 \subseteq U \co [S]^{B_2}_{C_2} = [T]^{B_1}_{C_1}$ אזי: 
                \[ \dim \Img S = \dim \row [\vphi]^{C_2}_{B_2} = \dim \row [\vphi]_{B_1}^{C_1} = \dim \Img T \]
                כדרוש. 
                \item[$\impliedby$] אם $\dim \Img S = \dim \Img T$, אז קיימים בסיסים כלשהם כך ש־$[T]^{\tl B_1}_{\tl C_1}, [S]_{C_2}^{B_2}$ המטריצות המייצגות. בשיעורי בית קודמים ראינו שהמטריצות להלן מתאימות אמ''מ $\rk [T]^{\tl B_1}_{\tl C_1} = \rk [S]_{B_2}^{C_2}$ מה שאכן מתקיים שכן: 
                \[ \rk [T]^{\tl B_1}_{\tl C_1} = \dim \row [T]^{\tl B_1}_{\tl C_1} = \dim \Img T = \dim \Img T = \dim \row [S]_{C_2}^{B_2} = \rk [S]_{C_2}^{B_2} \]
                ומשום שהן מתאימות קיימים קיימות $P, Q$ הפיכות כך ש־$Q\op [T]^{\tl B_1}_{\tl C_1} P$ ומשום שהן הפיכות הן בפרט מטריצות העברת בסיס כלשהן מ־$\tl B_1$ ל־$B_1$ כלשהו וכן מ־$\tl C_1$ ל־$C_1$ כלשהו, ועל כן קיימים $B_1, C_1$ כך ש־$[T]^{\tl B_1}_{\tl C_1} = Q\op [T]^{B_1}_{C_1}P$. סה''כ הראינו קיום $C_1, B_1, C_2, B_2$ כך ש־$[T]^{B_1}_{C_1} = [S]_{B_2}^{C_2}$ כדרוש. 
            \end{itemize}
        \end{proof}
        \item תהי $T \co V \to V$ העתקה לינארית, נוכיח ש־$T$ איזו' אמ''מ קיימים $B, C$ כך ש־$[T]^{B}_C = I$. \begin{proof}נסמן $\dim V = n$. 
            \begin{itemize}
                \item[$\implies$]נניח $T$ איזו', נוכיח קיומים בסיסים $B, C$ כך ש־$[T]^{B}_C = I$. מהיותה איזו', דרגתה $n$. אזי קיים רצף דירוג שורות $E_1 \dots E_k$ כך ש־$[T]^{B}_{\tl C}$ ייצוג לפי בסיסים כלשהו מקיים $[T]^{B}_{\tl C}\prod_{i = 1}^{k}E_k = I$. אזי $\prod_{i = 1}^{n}E_k =: \trio$ מטריצה הפיכה שכן היא כפל של אלמנטריות ובפרט היא מטריצה מעבר שורה מ־$\tl C$ ל־$C$ כלשהי, ונקבל $\trio = [id]^{C}_{C_1}$. סה''כ: 
                \[ [T]^{B}_{\tl C}[id]^{\tl C}_{C} = I \implies [T]^{B}_{C} = I \quad \top \]
                \item[$\impliedby$] אם $T$ מתאימה ל־$I$ אז $\rk [T]^{B}_C = n$ ומשום ש־$T \co V \to V$ ממדים מגודל זהה $n$, אזי היא הפיכה כדרוש. 
            \end{itemize}
        \end{proof}
    \end{enumerate}
    
    \section{}
    נמצא את החיתוך של $U, W \subseteq \R^4$ הבאים: 
    \[ U = \Sp\ccb{\pms{0 \\ 2 \\ 0 \\ 0}, \ \pms{1 \\ 0 \\ 0 \\ 0}, \ \pms{2 \\1 \\ 3 \\ 7}}, \ W = \Sp\ccb{\pms{1 \\ 0 \\3 \\ 0}, \ \pms{0 \\1 \\ -3 \\ 7}} \]
    ידוע שדירוג מטריצה לא משנה את מרחב השורות שלה:
    \[ \pms{0 & 3 & 0 & 1 \\ 7 & -3 & 1 & 0} \rrr{R_2 \to R_2 + R_1} \pms{0 & 3 & 0 & 1 \\ 7 & 0 & 0 & 1} \rrt{R_1 \to \frac{1}{3}R_1}{R_2 \to \frac{1}{7}R_2} \pms{0 & 1 & 0 & \frac{1}{3} \\ 1 & 0 & 0 & \frac{1}{7}} \]
    וכן: 
    \[ \pms{0 & 0 & 2 & 0 \\ 0 & 0 & 0 & 1 \\ 7 & 3 & 1 & 2} \rrt{R_3 \to R_3 - R_2}{R_3 \to R_3 - \frac{1}{2}R_2} \pms{0 & 0 & 2 & 0 \\ 0 & 0 & 0& 1 \\ 7 & 3 & 0 & 0} \rrr{R_1 \to \frac{1}{2}R_1 \\ R_3 \to \frac{1}{7}R_3} \pms{0 & 0& 1 & 0 \\ 0 & 0 & 0 & 1 \\ 1 & \frac{3}{7}& 0 &0} \]
    סה''כ: 
    \[ U = \Sp\ccb{\pms{0 \\ 1 \\ 0 \\ 0}, \pms{1 \\ 0 \\ 0 \\ 0}, \pms{0 \\ 0 \\ \frac{3}{7} \\ 1}} =: \Sp\{u_1, u_2, u_3\}, \ W = \Sp\ccb{\pms{0 \\ 1 \\ 0 \\ \frac{1}{3}}, \pms{1 \\ 0 \\ 0 \\ \frac{3}{7}}} =: \Sp\{w_1, w_2\} \]
    נחפש אילו וקטורים מהבסיס של $W$ נמצאים ב־$U$: 
    \[ \tmat{U}{w_1, w_2} = \tmat{0 & 1 & 0 \\ 1 & 0 & 0 \\ 0 & 0 & \frac{3}{7} \\ 0 & 0 & 1}{0 & 1 \\ 1 & 0 \\ 0 & 0 \\ \frac{1}{3} & \frac{3}{7}} \rrt{R_1 \lra R_2}{R_3 \lra R_4} \tmat{1 & 0 & 0 \\ 0 & 1 & 0 \\ 0 & 0 & 1 \\ 0 & 0 & \frac{3}{7}}{1 & 0 \\ 0 & 1 \\ \frac{1}{3} & \frac{3}{7} \\ 0 & 0} \]
    משום שהשוויון $\frac{3}{7}\ag = 0$ גורר $\ag = 0$ החלק של $u_3$ בקומבינציה הליניארית הוא $0$, אך זה גורר $\frac{1}{3} = 0$ אם $w_1 \in U \cap W$ וזו סתירה, וכן $\frac{3}{7} = 0$ אם $w_2 \in W \cap U$ וזו סתירה, וסה''כ $U \cap W = \{0\}$ ובסיס $\{\}$. 
    
    עתה נמצא בסיס ל־$U + W$: 
    \[ U + W = \Sp U + \Sp W = \Sp \ccb{\pms{0 \\ 1 \\ 0 \\ 0}, \pms{1 \\ 0 \\ 0 \\ 0}, \pms{0 \\ 0 \\ \frac{3}{7} \\ 1}, \pms{0 \\ 1 \\ 0 \\ \frac{1}{3}}, \pms{1 \\ 0 \\ 0 \\ \frac{3}{7}}} \]
    נצמצם לבסיס; ידוע שמרחב השורות של המטריצה הבאה הוא ה־$\Sp$ של הוקטורים לעיל, וכן דירוגו לא משנה את המ''ו: 
    \[ \pms{0 & 0 & 1 & 0 \\ 0 & 0 & 0 & 1 \\ 1  & \frac{3}{7} & 0 & 0 \\ \frac{1}{3} & 0 & 1 & 0 \\ \frac{3}{7} & 0 & 0 & 1} \rrt{R_4 \to R_4 - R_1}{R_5 \to R_5 - R_2} 
    \pms{0 & 0 & 1 & 0 \\ 0 & 0 & 0 & 1 \\ 1  & \frac{3}{7} & 0 & 0 \\ \frac{1}{3} & 0 & 0 & 0 \\ \frac{3}{7} & 0 & 0 & 0}
    \rrr{R_4 \to 4 R_4} \rrt{R_3 \to R_3 - R_4}{R_ 5\to R_5 - \frac{3}{7}R_4}
    \pms{0 & 0 & 1 & 0 \\ 0 & 0 & 0 & 1 \\ 0  & \frac{3}{7} & 0 & 0 \\ 1 & 0 & 0 & 0 \\ 0 & 0 & 0 & 0} \rrr{\quad\quad}
    \pms{1 & 0 & 0 & 0 \\ 0& \frac{3}{7} & 0 & 0 \\ 0 & 0 & 1 & 0 \\ 0& 0 & 0 & 1 \\ 0 & 0 & 0 & 0} \rrr{R_2 \to \frac{7}{3}R_2} \pms{I_4 \\ 0_{1 \times 4}} \] 
    סה''כ הבסיס הסטנדרטי הוא בסיס ל־$U + W = \R^4$. 
    
    \section{}
    נתבונן במרחבים הבאים: 
    \[ U = \Sp\ccb{\pms{0 \\1  \\ 1 \\ 2} , \pms{0 \\1 \\ 1 \\ 3} ,\pms{0 \\1 \\2 \\4}}, W = \Sp\ccb{\pms{1 \\ 0 \\ 1 \\ 0}, \pms{5 \\ 5 \\3 \\2}} \]
    נחפש את $U + W$. ידוע ש־: 
    \[ A_U = \pms{0 & 1 & 1 & 2 \\ 0 & 1 & 1 & 3 \\ 0 & 1 &2 &4}, \ A_W = \pms{1 & 0 & 1 & 0 \\ 5 & 5 & 3 & 2} \]
    מקיימות $\row A_W = W, \ \row A_U = U$. אזי $\row\binom{A_U}{A_W} = U + w$, חיבור בסיסים. נמצא בסיס למרחב הבא באמצעות דירוג המטריצה: 
    \begin{gather*}\pms{A_U \\ A_W} = \pms{0 & 1 & 1 & 2 \\ 0 & 1 & 1 & 3 \\ 0 & 1 &2 &4 \\ 1 & 0 & 1 & 0 \\ 5 & 5 & 3 & 2}\rrr{R_1 \siff R_5} \pms{5 & 5 & 3 & 2 \\ 
            0 & 1 & 1 & 3 \\ 
            0 & 1 & 2 & 4 \\ 
            1 & 0 & 1 & 0 \\ 
            0 & 1 & 1 & 2 \\ 
        } \rrr{R_1 \to \frac{1}{5}R_1} \pms{1 & 1 & \frac{3}{5} & \frac{2}{5} \\ 
            0 & 1 & 1 & 3 \\ 
            0 & 1 & 2 & 4 \\ 
            1 & 0 & 1 & 0 \\ 
            0 & 1 & 1 & 2 \\ 
        } \rrr{R_4 \to R_4 - \cdot R_1} \pms{1 & 1 & \frac{3}{5} & \frac{2}{5} \\ 
            0 & 1 & 1 & 3 \\ 
            0 & 1 & 2 & 4 \\ 
            0 & -1 & \frac{2}{5} & \frac{2}{-5} \\ 
            0 & 1 & 1 & 2 \\ 
        } \\\rrt{\overset{R_3 \to R_3 - R_2}{R_4 \to R_4 +1R_2}}{R_5 \to R_5 - 1 \cdot R_2} \pms{1 & 1 & \frac{3}{5} & \frac{2}{5} \\ 
            0 & 1 & 1 & 3 \\ 
            0 & 0 & 1 & 1 \\ 
            0 & 0 & \frac{7}{5} & \frac{13}{5} \\ 
            0 & 0 & 0 & -1 \\ 
        } \rrr{R_4 \to R_4 - \frac{7}{5} R_3} \pms{1 & 1 & \frac{3}{5} & \frac{2}{5} \\ 
            0 & 1 & 1 & 3 \\ 
            0 & 0 & 1 & 1 \\ 
            0 & 0 & 0 & \frac{6}{5} \\ 
            0 & 0 & 0 & -1 \\ 
        } \rrr{R_4 \to \frac{5}{6}R_4} \pms{1 & 1 & \frac{3}{5} & \frac{2}{5} \\ 
            0 & 1 & 1 & 3 \\ 
            0 & 0 & 1 & 1 \\ 
            0 & 0 & 0 & 1 \\ 
            0 & 0 & 0 & -1 \\ 
        } \rrr{R_5 \to R_5 +R_4} \pms{1 & 1 & \frac{3}{5} & \frac{2}{5} \\ 
            0 & 1 & 1 & 3 \\ 
            0 & 0 & 1 & 1 \\ 
            0 & 0 & 0 & 1 \\ 
            0 & 0 & 0 & 0 \\ 
        } \\\rrt{\overset{R_3 \to R_3 - 1 R_4}{R_2 \to R_2 - 3 R_4}}{R_1 \to R_1 - \frac{2}{5} R_4} \pms{1 & 1 & \frac{3}{5} & 0 \\ 
            0 & 1 & 1 & 0 \\ 
            0 & 0 & 1 & 0 \\ 
            0 & 0 & 0 & 1 \\ 
            0 & 0 & 0 & 0 \\ 
        } \rrt{R_2 \to R_2 - R_3}{R_1 \to R_1 - \frac{3}{5} R_3} \pms{1 & 1 & 0 & 0 \\ 
            0 & 1 & 0 & 0 \\ 
            0 & 0 & 1 & 0 \\ 
            0 & 0 & 0 & 1 \\ 
            0 & 0 & 0 & 0 \\ 
        } \rrr{R_1 \to R_1 - R_2} \pms{1 & 0 & 0 & 0 \\ 
            0 & 1 & 0 & 0 \\ 
            0 & 0 & 1 & 0 \\ 
            0 & 0 & 0 & 1 \\ 
            0 & 0 & 0 & 0 \\ 
        } \end{gather*}
        סה''כ הבסיס הסטנדרטי בסיס ל־$U + W$. עתה נמצא בסיס ל־$U \cap W$. יהי $v \in U \cap W$. אזי: 
        \[ v \in U \implies \exists a, b, c \co v = a\pms{0 \\ 1 \\ 1 \\ 2} + b\pms{0 \\ 1 \\ 1 \\ 3} + c\pms{0 \\1 \\ 2 \\ 4} = \pms{0 \\ a+ b + c \\ a + b + 2c \\ 2a + 3b + 4c} \]
        \[ v \in W \implies \exists d, e \co v = d\pms{1 \\ 0 \\ 1 \\ 0} + e\pms{5 \\ 5 \\ 3 \\ 2} = \pms{d + 5e \\ 5e \\ d + 3e \\ 2e} \]
        סה''כ קיבלנו: 
        \[ \pms{d + 5e \\ 5e \\ d + 3e \\ 2e} = v = \pms{0 \\ a+ b + c \\ a + b + 2c \\ 2a + 3b + 4c} \implies \pms{-d - 5e \\ a + b + c - 5e \\ a+ b + 2c - d - 3e \\ 2a + 3b + 4c - 2e} = \pms{0 \\ 0 \\ 0 \\0} \]
        נעביר למערכת משוואות הומוגנית: 
        \begin{gather*} \pms{0 & 0 & 0 & -1 & -5 \\ 1 & 1 & 1 & 0 & -5 \\ 1 & 1 & 1 & -1 & -3 \\ 2 & 3 & 4 & 0 & -2} \rrr{R_1 \siff R_4} \pms{2 & 3 & 4 & 0 & -2 \\ 
            1 & 1 & 1 & 0 & -5 \\ 
            1 & 1 & 1 & -1 & -3 \\ 
            0 & 0 & 0 & -1 & -5 \\ 
        } \rrr{R_1 \to 0.5R_1} \pms{1 & \frac{3}{2} & 2 & 0 & -1 \\ 
            1 & 1 & 1 & 0 & -5 \\ 
            1 & 1 & 1 & -1 & -3 \\ 
            0 & 0 & 0 & -1 & -5 \\ 
        } \rrt{R_2 \to R_2 - R_1}{R_3 \to R_3 - R_1} \pms{1 & \frac{3}{2} & 2 & 0 & -1 \\ 
            0 & \frac{1}{-2} & -1 & 0 & -4 \\ 
            0 & \frac{1}{-2} & -1 & -1 & -2 \\ 
            0 & 0 & 0 & -1 & -5 \\ 
        } \\\rrr{R_2 \to -2R_2} \pms{1 & \frac{3}{2} & 2 & 0 & -1 \\ 
            0 & 1 & 2 & 0 & 8 \\ 
            0 & \frac{1}{-2} & -1 & -1 & -2 \\ 
            0 & 0 & 0 & -1 & -5 \\ 
        } \rrr{R_3 \to R_3 + 0.5R_2} \pms{1 & \frac{3}{2} & 2 & 0 & -1 \\ 
            0 & 1 & 2 & 0 & 8 \\ 
            0 & 0 & 0 & -1 & 2 \\ 
            0 & 0 & 0 & -1 & -5 \\ 
        } \rrr{R_3 \to -R_3} \pms{1 & \frac{3}{2} & 2 & 0 & -1 \\ 
            0 & 1 & 2 & 0 & 8 \\ 
            0 & 0 & 0 & 1 & -2 \\
            0 & 0 & 0 & -1 & -5 \\  
        } \\\rrr{R_4 \to R_4 +R_3} \pms{1 & \frac{3}{2} & 2 & 0 & -1 \\ 
            0 & 1 & 2 & 0 & 8 \\ 
            0 & 0 & 0 & 1 & -2 \\
            0 & 0 & 0 & 0 & -7 \\  
        } \rrt{R_1 \to R_1 - 1\frac{1}{2}R_2}{R_4 \to -\frac{1}{7}R_4} \pms{1 & 0 & -1 & 0 & -13 \\ 
            0 & 1 & 2 & 0 & 8 \\ 
            0 & 0 & 0 & 1 & -2 \\
            0 & 0 & 0 & 0 & 1 \\  
        } \rrt{\overset{R_3 \to R_3 + 2R_4}{R_2 \to R_2 - 8R_4}}{R_1 \to R_1 + 13 R_4} \pms{1 & 0 & -1 & 0 & 0 \\ 
        0 & 1 & 2 & 0 & 0 \\ 
        0 & 0 & 0 & 1 & 0 \\
        0 & 0 & 0 & 0 & 1 \\  
        } 
    \end{gather*}
    סה''כ מצאנו $d = e = 0$ אך גם $c \in \R$ וכן $a = c, \ b = -2c$. סה''כ: 
    \[ v \in \ccb{\pms{0 \\ c -2c + c \\ c -2c + 2c \\ 2c - 6c + 4c} \mid c \in \R} = \ccb{\pms{0 \\ 0 \\ 1 \\ 0}c \mid c \in \R} = \Sp\pms{0 \\\ 0 \\ 1 \\ 0} = U \cap W \]
    
    
    
    
    
    \section{}
    נראה שהכלה והדחה לא עובר על מממדי מרחבים עבור $n = 3$. נתבונן בשלושת התמ''וים הבאים של $\R^2$: 
    \[ U_1 = \Sp \{e_1\}, \ U_2 = \Sp\{e_2\}, \ U_3 = \Sp\{(e_2 + e_1)\} = \Sp\cl{\pms{1 \\ 1}} \]
    אזי: 
    \[ \dim(U_1 + U_2 + U_3) = 2 \neq 1 = \underbrace{\dim U_1}_{=1} + \underbrace{\dim U_2}_{=1} + \underbrace{\dim U_3}_{=1} - \underbrace{\dim(U_1 \cap U_2)}_{=0} - \underbrace{\dim(U_1 \cap U_3)}_{=1} - \underbrace{\dim(U_2 \cap U_3)}_{=1} + \underbrace{\dim(U_1 \cap U_2 \cap U_3)}_{=0} \]
    כדרוש. 
    
    \section{}
    לפני שנחשב דטרמיננטות, נפתח את הנוסחה לדטרממיננטה למטריצה $\in M_{3 \times 3}(\F)$: 
    \begin{align*}
        \detms{a & b & c \\ d & e & f \\ g & h & i} &= a\detms{e & f \\ h & i} - b\detms{d & f \\ g & i} + c\detms{d & e \\ g & h} \\
        &= a(ei - hf) - b(di + gf) + c(dh - eg) \\
        &= aei - ahf - bdi + bgf + cdh - ceg \\
        &= aei + bgf + cdh - ahf - bdi  - ceg
    \end{align*}
    
    \begin{enumerate}[(a)]
        \item מהנוסחה שהוכחנו: 
        \[ \detms{1 &2 & 3 \\ 4 & 5 & 6 \\ 7 & 8 & 9} = 1 \cdot 5 \cdot 9 + 2 \cdot 6 \cdot 7 + 3 \cdot 4 \cdot 8 - 1 \cdot 6 \cdot 8 - 2 \cdot 4 \cdot 9 - 3 \cdot 5 \cdot 7 = \bm{0} \]
        \item 
        \[ \detms{1 & 2 & 3 \\ 
            2 & 4 & 5 \\ 
            -1 & -2 & -4 \\ 
        }= 1 \cdot 4 \cdot -4 + 2 \cdot 5 \cdot -1 + 3 \cdot 2 \cdot -2 - 1 \cdot 5 \cdot -2 - 2 \cdot 2 \cdot -4 - 3 \cdot 4 \cdot -1 = 0 \]
        \item נפתח לפי השורה האחרונה: 
        \[ \detms{0 & 0 & 4 & 3 \\ 4 & 0 & 7 & 5 \\ -1 & 4 & 3 & 1 \\ 0 & 0 & 0 & 4} = 4(-1)^{4+4} \detms{0 & 4 & 3 \\ 
            0 & 7 & 5 \\ 
            4 & 3 & 1 \\ 
        }= 4(0 \cdot 7 \cdot 1 + 4 \cdot 5 \cdot 4 + 3 \cdot 0 \cdot 3 - 0 \cdot 5 \cdot 3 - 4 \cdot 0 \cdot 1 - 3 \cdot 7 \cdot 4) = 4 \cdot (-4) = -16 \]
        \item גם כאן, נפתח לפי השורה האחרונה. נמצא את המינורים בנפרד: 
        \begin{align*}
            \detms{2 & 3 & 4 \\ 
                2 & 3 & 4 \\ 
                3 & 3 & 4 \\ 
            } &= 2 \cdot 3 \cdot 4 + 3 \cdot 4 \cdot 3 + 4 \cdot 2 \cdot 3 - 2 \cdot 4 \cdot 3 - 3 \cdot 2 \cdot 4 - 4 \cdot 3 \cdot 3 = 0 \\
            \detms{1 & 3 & 4 \\ 
                    2 & 3 & 4 \\ 
                    3 & 3 & 4 \\ 
                } &= 1 \cdot 3 \cdot 4 + 3 \cdot 4 \cdot 3 + 4 \cdot 2 \cdot 3 - 1 \cdot 4 \cdot 3 - 3 \cdot 2 \cdot 4 - 4 \cdot 3 \cdot 3 = 0 \\
            \detms{1 & 2 & 4 \\ 
                2 & 2 & 4 \\ 
                3 & 3 & 4 \\ 
            } &= 1 \cdot 2 \cdot 4 + 2 \cdot 4 \cdot 3 + 4 \cdot 2 \cdot 3 - 1 \cdot 4 \cdot 3 - 2 \cdot 2 \cdot 4 - 4 \cdot 2 \cdot 3 = 4 \\
            \detms{1 & 2 & 3 \\ 
                2 & 2 & 3 \\ 
                3 & 3 & 3 \\ 
            } &= 1 \cdot 2 \cdot 3 + 2 \cdot 3 \cdot 3 + 3 \cdot 2 \cdot 3 - 1 \cdot 3 \cdot 3 - 2 \cdot 2 \cdot 3 - 3 \cdot 2 \cdot 3 = 3
        \end{align*}
        ונמצא: 
        \[ \detms{1 & 2 & 3 & 4 \\ 2 & 2 & 3 & 4 \\ 3 & 3 & 3 & 4 \\ 4 & 4 & 4 &4} = 4\detms{2 & 3 & 4 \\ 
            2 & 3 & 4 \\ 
            3 & 3 & 4 \\ 
        } -4 \detms{1 & 3 & 4 \\ 
            2 & 3 & 4 \\ 
            3 & 3 & 4 \\ 
        } + 4\detms{1 & 2 & 4 \\ 
            2 & 2 & 4 \\ 
            3 & 3 & 4 \\ 
        } - 4\detms{1 & 2 & 3 \\ 
            2 & 2 & 3 \\ 
            3 & 3 & 3 \\ 
        } = 4 \cdot 0 - 4 \cdot 0 + 4 \cdot 4 - 4 \cdot 3 = 16 - 12 = \bm{4} \]
        \item נתבונן בדטרמיננטה הבאה: 
        \[ \detms{A} := \detms{1 & 2 & 3 & 4 &5 \\ -78 & 13 & -43 & 111 & 11 \\ 1 & 1 & 0 & 0 & 0 \\ 235 & 14 & 0 & 0 & 0 \\ 1 & 89 & 0 & 0 &0} \seq 0 \]
        נבחין ששלושת השורות האחרונות ת''ל אמ''מ הוקטורים $(1, 1), \ (235, 11), \ (1, 89)$ ת''ל שכן שאר האפסים ת''ל. אלו $3$ וקטורים במרחב $\R^{2}$, כלומר הם בהכרח תלויים לינארית. לכן שלושת השורות האחרונות של המטריצה ת''ל, ודרגתה איננה $5$ גודל המ''ו $\R^{5}$. סה''כ $\detms{A} = \bm{0}$. 
    \end{enumerate}
    
    \section{}
    נמצא את הדטרמיננטה של המטריצות הבאות: 
    
    \begin{enumerate}[(a)]
        \item ראשית כל, נמצא את הדטרמיננטה של: 
        \[ A_n =: \detms{\ag_1 & 0 & 0 & \cdots & 0 \\ \bg_1 & \ag_2 & 0 & \cdots & 0 \\ 0 & \ddots & \ddots & \ddots & \vdots \\ \vdots & \ddots & \bg_{n - 2} & \ag_{n - 1} & 0 \\ 0 & \cdots & 0 & \bg_{n - 1} & \ag_n}
        = (-1)^{2n}\ag_n \detms{\ag_1 & 0 & \cdots & 0 \\ \bg_1 & \ag_2 & 0 & \vdots \\ 0 & \ddots & \ddots & 0 \\ \vdots & \ddots & \bg_{n - 2} & \ag_{n - 1}} = \ag_n A_{n - 1} \]
        ובסיס: 
        \[ A_1 = \detms{\ag_1} = \sof{\ag_1 I_{1 \times 1}} = \ag_1 \]
        ולכן: 
        \[ \begin{cases}
            A_n = \ag_n A_{n - 1} \\
            A_1 = \ag_1
        \end{cases} \dequad\implies A_n = \prod_{i = 1}^{n}\ag_i \]
        נחזור לשאלה מהקורית. נרצה לחשב את הדטרמיננטה הבאה באמצעות פיתוח לפי העמודה האחרונה: 
        \[ \detms{a_1 & 0 & \cdots & 0 & b_1 \\ b_2 & a_2 & 0 & \cdots & 0 \\ 0 & b_3 & a_3 & \ddots & \vdots \\ \vdots & \ddots & \ddots & \ddots & 0 \\ 0& \cdots & 0& b_n & a_n} = (-1)^{n + n}a_n\detms{a_1 & 0 & 0 & \cdots & 0 \\ b_2 & a_2 & 0 & \cdots & 0 \\ 0 & \ddots & \ddots & \ddots & \vdots \\ \vdots & \ddots & b_{n - 2} & a_{n - 2} & 0 \\ 0 & \cdots & 0 & b_{n - 1} & a_{n - 1}} + (-1)^{n}
        + b_1\detms{a_2 & 0 & 0 & \cdots & 0 \\ b_3 & a_3 & 0 & \cdots & 0 \\ 0 & \ddots & \ddots & \ddots & \vdots \\ \vdots & \ddots & b_{n - 1} & a_{n - 1} & 0 \\ 0 & \cdots & 0 & b_{n} & a_n} \]
        לפי הדטרמיננטה שפיתחנו קודם לכן: 
        \[ = ((-1)^{2})^{n}a_n \prod_{i = 1}^{n - 1}a_{i} + (-1)^{n}b_1\prod_{i = 2}^{n}a_i = a_1\prod_{i = 2}^{n}a_i + (-1)^{n}b_1\prod_{i = 2}^{n} = \bm{\prod_{i = 2}^{n}\cl{\ag_1 + (-1)^{n}b_1}} \]
        
        \item נמצא את הדטרמיננטה של המטריצה הבאה: 
        \[ A = \pms{a^2 & (a + 1)^{2} & (a + 2)^{2} & (a + 3)^{2} \\ b^2 & (b + 1)^{2} & (b + 2)^{2} & (b + 3)^{2} \\ c^2 & (c + 1)^{2} & (c + 2)^{2} & (c + 3)^{2} \\ d^2 & (d + 1)^{2} & (d + 2)^{2} & (d + 2)^{3}} \]
        אם $\det A \neq 0$, אזי $A$ הפיכה כלומר $\rk A = 4$, אך $\row A \subseteq \R_3[x]$ וסה''כ: 
        \[ 4 = \rk A = \dim \row A \le \dim \R_3[x] = 3 \implies 4 \le 3 \quad \bot \]
        סתירה. לכן $\det A = 0$ כדרוש. 
        
    \end{enumerate}
    
    
    
    \ndoc
\end{document}