%! ~~~ Packages Setup ~~~ 
\documentclass[]{article}
\usepackage{lipsum}
\usepackage{rotating}


% Math packages
\usepackage[usenames]{color}
\usepackage{forest}
\usepackage{ifxetex,ifluatex,amssymb,amsmath,mathrsfs,amsthm,witharrows,mathtools,mathdots}
\usepackage{amsmath}
\WithArrowsOptions{displaystyle}
\renewcommand{\qedsymbol}{$\blacksquare$} % end proofs with \blacksquare. Overwrites the defualts. 
\usepackage{cancel,bm}
\usepackage[thinc]{esdiff}


% tikz
\usepackage{tikz}
\usetikzlibrary{graphs}
\newcommand\sqw{1}
\newcommand\squ[4][1]{\fill[#4] (#2*\sqw,#3*\sqw) rectangle +(#1*\sqw,#1*\sqw);}


% code 
\usepackage{algorithm2e}
\usepackage{listings}
\usepackage{xcolor}

\definecolor{codegreen}{rgb}{0,0.35,0}
\definecolor{codegray}{rgb}{0.5,0.5,0.5}
\definecolor{codenumber}{rgb}{0.1,0.3,0.5}
\definecolor{codeblue}{rgb}{0,0,0.5}
\definecolor{codered}{rgb}{0.5,0.03,0.02}
\definecolor{codegray}{rgb}{0.96,0.96,0.96}

\lstdefinestyle{pythonstylesheet}{
    language=Java,
    emphstyle=\color{deepred},
    backgroundcolor=\color{codegray},
    keywordstyle=\color{deepblue}\bfseries\itshape,
    numberstyle=\scriptsize\color{codenumber},
    basicstyle=\ttfamily\footnotesize,
    commentstyle=\color{codegreen}\itshape,
    breakatwhitespace=false, 
    breaklines=true, 
    captionpos=b, 
    keepspaces=true, 
    numbers=left, 
    numbersep=5pt, 
    showspaces=false,                
    showstringspaces=false,
    showtabs=false, 
    tabsize=4, 
    morekeywords={as,assert,nonlocal,with,yield,self,True,False,None,AssertionError,ValueError,in,else},              % Add keywords here
    keywordstyle=\color{codeblue},
    emph={var, List, Iterable, Iterator},          % Custom highlighting
    emphstyle=\color{codered},
    stringstyle=\color{codegreen},
    showstringspaces=false,
    abovecaptionskip=0pt,belowcaptionskip =0pt,
    framextopmargin=-\topsep, 
}
\newcommand\pythonstyle{\lstset{pythonstylesheet}}
\newcommand\pyl[1]     {{\lstinline!#1!}}
\lstset{style=pythonstylesheet}

\usepackage[style=1,skipbelow=\topskip,skipabove=\topskip,framemethod=TikZ]{mdframed}
\definecolor{bggray}{rgb}{0.85, 0.85, 0.85}
\mdfsetup{leftmargin=0pt,rightmargin=0pt,innerleftmargin=15pt,backgroundcolor=codegray,middlelinewidth=0.5pt,skipabove=5pt,skipbelow=0pt,middlelinecolor=black,roundcorner=5}
\BeforeBeginEnvironment{lstlisting}{\begin{mdframed}\vspace{-0.4em}}
    \AfterEndEnvironment{lstlisting}{\vspace{-0.8em}\end{mdframed}}


% Deisgn
\usepackage[labelfont=bf]{caption}
\usepackage[margin=0.6in]{geometry}
\usepackage{multicol}
\usepackage[skip=4pt, indent=0pt]{parskip}
\usepackage[normalem]{ulem}
\forestset{default}
\renewcommand\labelitemi{$\bullet$}
\usepackage{titlesec}
\titleformat{\section}[block]
{\fontsize{15}{15}}
{\sen \dotfill (\thesection)\dotfill\she}
{0em}
{\MakeUppercase}
\usepackage{graphicx}
\graphicspath{ {./} }

\usepackage[colorlinks]{hyperref}
\definecolor{mgreen}{RGB}{25, 160, 50}
\definecolor{mblue}{RGB}{30, 60, 200}
\usepackage{hyperref}
\hypersetup{
    colorlinks=true,
    citecolor=mgreen,
    linkcolor=black,
    urlcolor=mblue,
    pdftitle={Document by Shahar Perets}
}


% Hebrew initialzing
\usepackage[bidi=basic]{babel}
\PassOptionsToPackage{no-math}{fontspec}
\babelprovide[main, import, Alph=letters]{hebrew}
\babelprovide[import]{english}
\babelfont[hebrew]{rm}{David CLM}
\babelfont[hebrew]{sf}{David CLM}
%\babelfont[english]{tt}{Monaspace Xenon}
\usepackage[shortlabels]{enumitem}
\newlist{hebenum}{enumerate}{1}

% Language Shortcuts
\newcommand\en[1] {\begin{otherlanguage}{english}#1\end{otherlanguage}}
\newcommand\he[1] {\she#1\sen}
\newcommand\sen   {\begin{otherlanguage}{english}}
    \newcommand\she   {\end{otherlanguage}}
\newcommand\del   {$ \!\! $}

\newcommand\npage {\vfil {\hfil \textbf{\textit{המשך בעמוד הבא}}} \hfil \vfil \pagebreak}
\newcommand\ndoc  {\dotfill \\ \vfil {\begin{center}
            {\textbf{\textit{שחר פרץ, 2025}} \\
                \scriptsize \textit{קומפל ב־}\en{\LaTeX}\,\textit{ ונוצר באמצעות תוכנה חופשית בלבד}}
    \end{center}} \vfil	}

\newcommand{\rn}[1]{
    \textup{\uppercase\expandafter{\romannumeral#1}}
}

\makeatletter
\newcommand{\skipitems}[1]{
    \addtocounter{\@enumctr}{#1}
}
\makeatother

%! ~~~ Math shortcuts ~~~

% Letters shortcuts
\newcommand\N     {\mathbb{N}}
\newcommand\Z     {\mathbb{Z}}
\newcommand\R     {\mathbb{R}}
\newcommand\Q     {\mathbb{Q}}
\newcommand\C     {\mathbb{C}}
\newcommand\One   {\mathit{1}}

\newcommand\ml    {\ell}
\newcommand\mj    {\jmath}
\newcommand\mi    {\imath}

\newcommand\powerset {\mathcal{P}}
\newcommand\ps    {\mathcal{P}}
\newcommand\pc    {\mathcal{P}}
\newcommand\ac    {\mathcal{A}}
\newcommand\bc    {\mathcal{B}}
\newcommand\cc    {\mathcal{C}}
\newcommand\dc    {\mathcal{D}}
\newcommand\ec    {\mathcal{E}}
\newcommand\fc    {\mathcal{F}}
\newcommand\nc    {\mathcal{N}}
\newcommand\vc    {\mathcal{V}} % Vance
\newcommand\sca   {\mathcal{S}} % \sc is already definded
\newcommand\rca   {\mathcal{R}} % \rc is already definded
\newcommand\zc    {\mathcal{Z}}

\newcommand\prm   {\mathrm{p}}
\newcommand\arm   {\mathrm{a}} % x86
\newcommand\brm   {\mathrm{b}}
\newcommand\crm   {\mathrm{c}}
\newcommand\drm   {\mathrm{d}}
\newcommand\erm   {\mathrm{e}}
\newcommand\frm   {\mathrm{f}}
\newcommand\nrm   {\mathrm{n}}
\newcommand\vrm   {\mathrm{v}}
\newcommand\srm   {\mathrm{s}}
\newcommand\rrm   {\mathrm{r}}

\newcommand\Si    {\Sigma}

% Logic & sets shorcuts
\newcommand\siff  {\longleftrightarrow}
\newcommand\ssiff {\leftrightarrow}
\newcommand\so    {\longrightarrow}
\newcommand\sso   {\rightarrow}

\newcommand\epsi  {\epsilon}
\newcommand\vepsi {\varepsilon}
\newcommand\vphi  {\varphi}
\newcommand\Neven {\N_{\mathrm{even}}}
\newcommand\Nodd  {\N_{\mathrm{odd }}}
\newcommand\Zeven {\Z_{\mathrm{even}}}
\newcommand\Zodd  {\Z_{\mathrm{odd }}}
\newcommand\Np    {\N_+}

% Text Shortcuts
\newcommand\open  {\big(}
\newcommand\qopen {\quad\big(}
\newcommand\close {\big)}
\newcommand\also  {\text{, }}
\newcommand\defis {\text{ definitions}}
\newcommand\given {\text{given }}
\newcommand\case  {\text{if }}
\newcommand\syx   {\text{ syntax}}
\newcommand\rle   {\text{ rule}}
\newcommand\other {\mathrm{else}}
\newcommand\set   {\ell et \text{ }}
\newcommand\ans   {\mathscr{A}\!\mathit{nswer}}

% Set theory shortcuts
\newcommand\ra    {\rangle}
\newcommand\la    {\langle}

\newcommand\oto   {\leftarrow}

\newcommand\QED   {\quad\quad\mathscr{Q.E.D.}\;\;\blacksquare}
\newcommand\QEF   {\quad\quad\mathscr{Q.E.F.}}
\newcommand\eQED  {\mathscr{Q.E.D.}\;\;\blacksquare}
\newcommand\eQEF  {\mathscr{Q.E.F.}}
\newcommand\jQED  {\mathscr{Q.E.D.}}

\DeclareMathOperator\dom   {dom}
\DeclareMathOperator\Img   {Im}
\DeclareMathOperator\range {range}

\newcommand\trio  {\triangle}

\newcommand\rc    {\right\rceil}
\newcommand\lc    {\left\lceil}
\newcommand\rf    {\right\rfloor}
\newcommand\lf    {\left\lfloor}
\newcommand\ceil  [1] {\lc #1 \rc}
\newcommand\floor [1] {\lf #1 \rf}

\newcommand\lex   {<_{lex}}

\newcommand\az    {\aleph_0}
\newcommand\uaz   {^{\aleph_0}}
\newcommand\al    {\aleph}
\newcommand\ual   {^\aleph}
\newcommand\taz   {2^{\aleph_0}}
\newcommand\utaz  { ^{\left (2^{\aleph_0} \right )}}
\newcommand\tal   {2^{\aleph}}
\newcommand\utal  { ^{\left (2^{\aleph} \right )}}
\newcommand\ttaz  {2^{\left (2^{\aleph_0}\right )}}

\newcommand\n     {$n$־יה\ }

% Math A&B shortcuts
\newcommand\logn  {\log n}
\newcommand\logx  {\log x}
\newcommand\lnx   {\ln x}
\newcommand\cosx  {\cos x}
\newcommand\sinx  {\sin x}
\newcommand\sint  {\sin \theta}
\newcommand\tanx  {\tan x}
\newcommand\tant  {\tan \theta}
\newcommand\sex   {\sec x}
\newcommand\sect  {\sec^2}
\newcommand\cotx  {\cot x}
\newcommand\cscx  {\csc x}
\newcommand\sinhx {\sinh x}
\newcommand\coshx {\cosh x}
\newcommand\tanhx {\tanh x}

\newcommand\seq   {\overset{!}{=}}
\newcommand\slh   {\overset{LH}{=}}
\newcommand\sle   {\overset{!}{\le}}
\newcommand\sge   {\overset{!}{\ge}}
\newcommand\sll   {\overset{!}{<}}
\newcommand\sgg   {\overset{!}{>}}

\newcommand\h     {\hat}
\newcommand\ve    {\vec}
\newcommand\lv    {\overrightarrow}
\newcommand\ol    {\overline}

\newcommand\mlcm  {\mathrm{lcm}}

\DeclareMathOperator{\sech}   {sech}
\DeclareMathOperator{\csch}   {csch}
\DeclareMathOperator{\arcsec} {arcsec}
\DeclareMathOperator{\arccot} {arcCot}
\DeclareMathOperator{\arccsc} {arcCsc}
\DeclareMathOperator{\arccosh}{arccosh}
\DeclareMathOperator{\arcsinh}{arcsinh}
\DeclareMathOperator{\arctanh}{arctanh}
\DeclareMathOperator{\arcsech}{arcsech}
\DeclareMathOperator{\arccsch}{arccsch}
\DeclareMathOperator{\arccoth}{arccoth}
\DeclareMathOperator{\atant}  {atan2} 
\DeclareMathOperator{\Sp}     {span} 
\DeclareMathOperator{\sgn}    {sgn} 
\DeclareMathOperator{\row}    {Row} 
\DeclareMathOperator{\adj}    {adj} 
\DeclareMathOperator{\rk}     {rank} 
\DeclareMathOperator{\col}    {Col} 
\DeclareMathOperator{\tr}     {tr}

\newcommand\dx    {\,\mathrm{d}x}
\newcommand\dt    {\,\mathrm{d}t}
\newcommand\dtt   {\,\mathrm{d}\theta}
\newcommand\du    {\,\mathrm{d}u}
\newcommand\dv    {\,\mathrm{d}v}
\newcommand\df    {\mathrm{d}f}
\newcommand\dfdx  {\diff{f}{x}}
\newcommand\dit   {\limhz \frac{f(x + h) - f(x)}{h}}

\newcommand\nt[1] {\frac{#1}{#1}}

\newcommand\limz  {\lim_{x \to 0}}
\newcommand\limxz {\lim_{x \to x_0}}
\newcommand\limi  {\lim_{x \to \infty}}
\newcommand\limh  {\lim_{x \to 0}}
\newcommand\limni {\lim_{x \to - \infty}}
\newcommand\limpmi{\lim_{x \to \pm \infty}}

\newcommand\ta    {\theta}
\newcommand\ap    {\alpha}

\renewcommand\inf {\infty}
\newcommand  \ninf{-\inf}

% Combinatorics shortcuts
\newcommand\sumnk     {\sum_{k = 0}^{n}}
\newcommand\sumni     {\sum_{i = 0}^{n}}
\newcommand\sumnko    {\sum_{k = 1}^{n}}
\newcommand\sumnio    {\sum_{i = 1}^{n}}
\newcommand\sumai     {\sum_{i = 1}^{n} A_i}
\newcommand\nsum[2]   {\reflectbox{\displaystyle\sum_{\reflectbox{\scriptsize$#1$}}^{\reflectbox{\scriptsize$#2$}}}}

\newcommand\bink      {\binom{n}{k}}
\newcommand\setn      {\{a_i\}^{2n}_{i = 1}}
\newcommand\setc[1]   {\{a_i\}^{#1}_{i = 1}}

\newcommand\cupain    {\bigcup_{i = 1}^{n} A_i}
\newcommand\cupai[1]  {\bigcup_{i = 1}^{#1} A_i}
\newcommand\cupiiai   {\bigcup_{i \in I} A_i}
\newcommand\capain    {\bigcap_{i = 1}^{n} A_i}
\newcommand\capai[1]  {\bigcap_{i = 1}^{#1} A_i}
\newcommand\capiiai   {\bigcap_{i \in I} A_i}

\newcommand\xot       {x_{1, 2}}
\newcommand\ano       {a_{n - 1}}
\newcommand\ant       {a_{n - 2}}

% Linear Algebra
\DeclareMathOperator{\chr}     {char}
\DeclareMathOperator{\diag}    {diag}
\DeclareMathOperator{\Hom}     {Hom}
\DeclareMathOperator{\Sym}     {Sym}
\DeclareMathOperator{\Asym}    {ASym}

\newcommand\lra       {\leftrightarrow}
\newcommand\chrf      {\chr(\F)}
\newcommand\F         {\mathbb{F}}
\newcommand\K         {\mathbb{K}}
\newcommand\co        {\colon}
\newcommand\tmat[2]   {\cl{\begin{matrix}
            #1
        \end{matrix}\, \middle\vert\, \begin{matrix}
            #2
\end{matrix}}}

\makeatletter
\newcommand\rrr[1]    {\xxrightarrow{1}{#1}}
\newcommand\rrt[2]    {\xxrightarrow{1}[#2]{#1}}
\newcommand\mat[2]    {M_{#1\times#2}}
\newcommand\gmat      {\mat{m}{n}(\F)}
\newcommand\tomat     {\, \dequad \longrightarrow}
\newcommand\pms[1]    {\begin{pmatrix}
        #1
\end{pmatrix}}

% someone's code from the internet: https://tex.stackexchange.com/questions/27545/custom-length-arrows-text-over-and-under
\makeatletter
\newlength\min@xx
\newcommand*\xxrightarrow[1]{\begingroup
    \settowidth\min@xx{$\m@th\scriptstyle#1$}
    \@xxrightarrow}
\newcommand*\@xxrightarrow[2][]{
    \sbox8{$\m@th\scriptstyle#1$}  % subscript
    \ifdim\wd8>\min@xx \min@xx=\wd8 \fi
    \sbox8{$\m@th\scriptstyle#2$} % superscript
    \ifdim\wd8>\min@xx \min@xx=\wd8 \fi
    \xrightarrow[{\mathmakebox[\min@xx]{\scriptstyle#1}}]
    {\mathmakebox[\min@xx]{\scriptstyle#2}}
    \endgroup}
\makeatother


% Greek Letters
\newcommand\ag        {\alpha}
\newcommand\bg        {\beta}
\newcommand\cg        {\gamma}
\newcommand\dg        {\delta}
\newcommand\eg        {\epsi}
\newcommand\zg        {\zeta}
\newcommand\hg        {\eta}
\newcommand\tg        {\theta}
\newcommand\ig        {\iota}
\newcommand\kg        {\keppa}
\renewcommand\lg      {\lambda}
\newcommand\og        {\omicron}
\newcommand\rg        {\rho}
\newcommand\sg        {\sigma}
\newcommand\yg        {\usilon}
\newcommand\wg        {\omega}

\newcommand\Ag        {\Alpha}
\newcommand\Bg        {\Beta}
\newcommand\Cg        {\Gamma}
\newcommand\Dg        {\Delta}
\newcommand\Eg        {\Epsi}
\newcommand\Zg        {\Zeta}
\newcommand\Hg        {\Eta}
\newcommand\Tg        {\Theta}
\newcommand\Ig        {\Iota}
\newcommand\Kg        {\Keppa}
\newcommand\Lg        {\Lambda}
\newcommand\Og        {\Omicron}
\newcommand\Rg        {\Rho}
\newcommand\Sg        {\Sigma}
\newcommand\Yg        {\Usilon}
\newcommand\Wg        {\Omega}

% Other shortcuts
\newcommand\tl    {\tilde}
\newcommand\op    {^{-1}}

\newcommand\sof[1]    {\left | #1 \right |}
\newcommand\cl [1]    {\left ( #1 \right )}
\newcommand\csb[1]    {\left [ #1 \right ]}
\newcommand\ccb[1]    {\left \{ #1 \right \}}

\newcommand\bs        {\blacksquare}
\newcommand\dequad    {\!\!\!\!\!\!}
\newcommand\dequadd   {\dequad\duquad}

\renewcommand\phi     {\varphi}

\newtheorem{Theorem}{משפט}
\theoremstyle{definition}
\newtheorem{definition}{הגדרה}
\newtheorem{Lemma}{למה}
\newtheorem{Remark}{הערה}
\newtheorem{Notion}{סימון}

\newcommand\theo  [1] {\begin{Theorem}#1\end{Theorem}}
\newcommand\defi  [1] {\begin{definition}#1\end{definition}}
\newcommand\rmark [1] {\begin{Remark}#1\end{Remark}}
\newcommand\lem   [1] {\begin{Lemma}#1\end{Lemma}}
\newcommand\noti  [1] {\begin{Notion}#1\end{Notion}}

% DS
\newcommand\limsi     {\limsup_{n \to \inf}}
\newcommand\limfi     {\liminf_{n \to \inf}}

\DeclareMathOperator\amort   {amort}
\DeclareMathOperator\worst   {worst}
\DeclareMathOperator\type    {type}
\DeclareMathOperator\cost    {cost}
\DeclareMathOperator\tim     {time}

\newcommand\dsList{
    \sFunc{List}
    \sFunc{Retrieve}
    \SetKwFunction{RetrieveFirst}{Retrieve-First}
    \SetKwFunction{RetrieveLast}{Retrieve-Last}
    \sFunc{Delete}
    \SetKwFunction{DeleteFirst}{Delete-First}
    \SetKwFunction{DeleteLast}{Delete-Last}
    \sFunc{Insert}
    \SetKwFunction{InsertFirst}{Insert-First}
    \SetKwFunction{InsertLast}{Insert-Last}
    \sFunc{Shift}
    \sFunc{Length}
    \sFunc{Concat}
    \sFunc{Plant}
    \sFunc{Split}
}
\newcommand\dsQueue{
    \sFunc{Queue}
    \sFunc{Enqueue}
    \sFunc{Head}
    \sFunc{Dequeue}
}
\newcommand\dsStack{
    \sFunc{Stack}
    \sFunc{Push}
    \sFunc{Top}
    \sFunc{Pop}
}
\newcommand\dsVector{
    \sFunc{Vector}
    \sFunc{Get}
    \sFunc{Set}
}
\newcommand\dsGraph{
    \sFunc{Graph}
    \sFunc{Edge}
    \SetKwFunction{AddEdge}{Add-Edge}
    \SetKwFunction{RemoveEdge}{Remove-Edge}
    \sFunc{InDeg} \sFunc{OutDeg}
}
\newcommand\importDs{
    \dsList
    \dsQueue
    \dsStack
    \dsVector
    \dsGraph
    \SetKwData{error}{\color{codered}error}
    \SetKwInOut{Input}{input}
    \SetKwInOut{Output}{output}
    \SetKwRepeat{Do}{do}{while}
    \SetKwData{Null}{\color{codeblue}null}
}


% Algorithems
\newcommand\sFunc [1] {\SetKwFunction{#1}{#1}}
\newcommand\sData [1] {\SetKwData{#1}{#1}}
\newcommand\sIO   [1] {\SetKwInOut{#1}{#1}}
\newcommand\ttt   [1] {\sen \texttt{#1} \she\,}
\newcommand\io    [2] {\Input{#1}\Output{#2}\BlankLine}

\newcommand\ASym      {\Asym}

%! ~~~ Document ~~~


\author{שחר פרץ}
\title{\textit{לינארית 1א $\sim$ תרגיל בית 6}}
\begin{document}
    \maketitle
    
    \section{}
    נוכיח ש־$\dim \Asym_n(\R) = \frac{n^2 - n}{2}$ ו־$\dim \Sym_n(\R) = \frac{n^2 + n}{2}$. \begin{proof}
        הוכחנו בתרגיל בית קודם ש־$\Sym_n(\R) \oplus \Asym_n(\R) = V$, ובפרט הם תמ''וים [ההוכחה מתרגיל הבית הקודם מצורפת \textit{בסוף} שיעורי בית אלו ע''מ שלא להכביד על ההוכחה כאן]. נוכיח שהקבוצה להלן בסיס ל־$\Asym_n(\R)$: 
        \[ B := \cl{A_{ij} = \begin{cases}
                1 & i = \tl i, \ j = \tl j \\
                -1 & i = - \tl i,  \ j = -\tl j \\
                0 & \other
        \end{cases} \,\Bigg\vert\, i, j \in \N \land  n > j > i} \]
    נבחין שכמות המספרים $i, j \in [n]$ כך ש־$i < j$ היא $\frac{n^2 - n}{2}$. על כן $|B| = \frac{n^2 - n}{2}$ (כי כל זוג מטריצות בבסיס שונה). עתה נוכיח ש־$B$ אכן בסיס: 
        נתבונן ברצף $\ag_i \dots \ag_k \in \R$סקלרים, וקודם לכן הראינו $\exists n \co \frac{n^2 - n}{2} = k$. אזי, קל לראות ש־:
        \[ \sum_{i = 1}^{k}\ag_i B_i = \ag_1B_1 + \dots + \ag_kB_k = \pms{
               0 & \ag_1 & \ag_2 & \cdots & \ag_{n - 1} \\
               -\ag_1 & 0 &  \ag_n & \cdots & \ag_{2n - 3} \\ 
               -\ag_2 & -\ag_n & 0 & \ddots & \vdots \\ 
               \vdots & \vdots & \ddots & \ddots & -\ag_k \\ 
               -\ag_{n - 1} & -\ag_{2n - 3} & \cdots & -\ag_k & 0 
            } \]
    מטריצה להלן המוגדרת ביחידות ע''י כל אחד מהסקלרים $\ag_1 \dots \ag_k$ ולכן $B$ בסיס למרחב המטריצות מהצורה לעיל (זהו מרחב משום שזהו $\Sp B$ תמ''ו של $M_n(\R)$). עתה נראה ש־$\Sp B = \Asym_n(\R)$. קל לראות $\Sp_B \subseteq \Asym_n(\R)$. נראה הכלה מהכיוון השני. יהי $M \in \Asym_n(\R)$, נוכיח $M \in \Sp B$. נניח בשלילה $M \notin \Sp B$, אז: 
    \begin{itemize}
        \item (בדיקה עבור האלכסון האמצעי): אם $\exists i \co M_{ii} \neq 0$, אז בפרט משום ש־$M \in \Asym_n(\R)$ נקבל $M_{ii} = -M_{ii}$, האי־השוויון ל־$0$ מותר לחלק את אגפי המשוואה ($\R$ תחום שלמות ללא מחלקי אפס) ולקבל $-1 = 1$ ובאופן שקול $0 = 2$, בסתירה לכך שאיננו עובדים ב־$\Z_2$. 
        \item (עבור המשולש התחתון): אם $\exists i, j \co M_{ij} \neq -M_{ji}$ בסתירה להיות $M \in \Asym_n(\R)$. 
        \item (עבור המשולש העליון): אין ב־$\Asym_n(\R)$ שום הגבלה בעבור סקלר במשולש העליון בהינתן שהתחתון הוגבל. 
    \end{itemize}
    סה''כ הראינו שבהכרח $M \in \Sp B$, ו־$B$ בסיס ל־$\Sp B$, וסה''כ $B$ בסיס ל־$\Asym_n(\R)$ והראינו $\dim \Asym_n(\R) = \frac{n^2 - n}{2}$. מהיות $\Asym_n(\R) \oplus \Sym_n(\R) =n$, יתקיים: 
        \[ \dim \Asym_n(\R) + \dim\Sym_n(\R) = n \implies \frac{n^2 - n}{2} + \dim \Sym_n(\R) = n \implies \dim \Sym_n(\R) = \frac{n^2 + n}{2} \]
        סה''כ: 
        \[ \bm{\dim \Asym_n(\R) = \frac{n^2 - n}{2}, \ \dim \Sym_n(\R) = \frac{n^2 + n}{2}} \]
    \end{proof}
    
    \section{}
    יהיו $v_1 \dots v_k \in \F^n$ וקטורים בת''ל ותהי $A$ המטריצה ששורותיה $v_1 \dots v_k$. נוכיח $\dim\{x \in \F^n \co Ax = 0\} = n - k$. \begin{proof}
        ל־$\F^n$ קיים בסיס סטנדרטי $E$ כלשהו ובאופן דומה $\tl E$ בסיס סטנדרטי ל־$\F^k$. נראה כי $\col A = \F^{k}$. הכלה $\F^k \subseteq \col A$ מתקיימת כי $A \in M_{k \times n}(\F)$. נראה ש־$\rk A \ge k$, זאת בגלל שמתקיים בעבור בסיס סטנדרטי $E$ ש־: 
        \[ \col A^T \, \reflectbox{$\subseteq$}\, \{A^Te \mid e \in E\} \overset{(1)}{=} \Sp \{v_1 \dots v_k\} =: B \implies \dim \col \le \dim B \overset{(2)}{=} k \]
        $(2)$ מתקיים מבת''ליות $B$ ו־$(2)$ מתקיים מהגדרת כפל מטריצה בוקטור. מתקיים $\col A^T = \row A = \rk A$ ולכן $\rk A \ge k$. 
        נבחין כי $\rk A \le k$ כי $\rk A = \dim \row A \le k$ בגלל שיש רק $k$ וקטורים בשורות $A$ (לא יכול להיות שהקבוצה $\row A$ מממד יותר גדול מהקבוצה הפורשת שלה שורות $A$). סה''כ $\rk A = k$. ממשפט הדרגה והאפסות: 
        \[ \rk A + \dim \nc(A) = \dim \F^{n} \implies k + \nc(A) = n \implies \{x \in \F^n \mid Ax = 0\} = \nc(A) = n - k \]
        כדרוש. 
    \end{proof}
    
    \section{}
    תהי $A \in M_n(R)$ מטריצה נילפטוטנטית כך ש־$A^k = 0$. צ.ל. $\{(A + I)^0 \dots (A + I)^{k}\}$ ת''ל. 
    \begin{proof}
        נוכיח בעזרת הבינום של ניוטון: 
        \[ (A + I)^k = \sum_{i = 1}^{k}\binom{k}{n}A^iI^{n - i} = \cancel{A^k} + \sum_{i = 1}^{k-1}\binom{k}{n}A^i \in \Sp(A^{0} \dots A^{k - 1}) \]
        לכן, $ (A + i)^k \in \Sp(A^{k - 1} \dots A, I)$ (כי צירוף ליניארי של הבינומים). בפרט, לכל $j < k$: 
        \[ (A + I)^n = \sum_{i = 1}^{j}\binom{i}{n}A^jI^i = \sum_{i = 1}^{j}\binom{i}{n}A^i \in \Sp (A^j, \cdots, A^1, I) \overset{\mathclap{j < k}}{\subseteq} \Sp (A^{k - 1}, \cdots, A^1, I) \]
        במילים אחרות: 
        \[ \dim \Sp\{(A + I)^0 \cdots (A + I)^{k}\} \le \dim \Sp\{A^0 \dots A^{k - 1}\} \le k \]
        אך בסדרה $(A + I)^{0}, \cdots, (A + I)^{{k}}$ ישנם $k + 1$ וקטורים, במ''ו מממד $k$, ולכן ממשפט היא תלויה לינארית כדרוש. 
    \end{proof}
    
    \section{}
    נגדיר: 
    \[ A = \pms{2 & -1 & 0 \\ 1 & -1 & 3}, \ B = \pms{-1 & 1 & 1 \\ -1 & -2 & -3}, \ C = \pms{2 & 0 & 1 & 1 \\ 5 & -1 & 2 & 3 \\ 0 & 0 & 1 & 2}, \ D = \pms{2 \\ -1 \\ 1} \]
    אז: 
    \sen\begin{gather}
        3A - 4B = \pms{6 & -3 & 0 \\ 3 & -3 & 9}
        - \pms{-4 & 4 & 4 \\ -4 & -8 & -12}
        = \pms{10 & -7 & -4 \\  7 & 5 & 21} \\
        CD \in \varnothing, \  \mathrm{undefined} \\
        B^T A = \pms{
            2 & -1 & 0 \\ 
            1 & -1 & 3
        } \pms{
            -1 & -1 \\ 
            1 & -2 \\ 
            1 & -3
        } = \pms{
            -3 & 2 & -3 \\ 
            0 & 1 & -6 \\ 
            -1 & 2 & -9
        } \\
        D D^T = \pms{2 \\ -1 \\ 1}\pms{2 & -1 & 1}
        = \pms{
            4 & -2 & 2 \\ 
            -2 & 1 & -1 \\ 
            2 & -1 & 1
        } \\ 
        AC = \pms{
            -1 & 1 & 0 & -1 \\ 
            -3 & 1 & 2 & 4
        } \\
        BD = \pms{-2 \\ -3}
    \end{gather}\she
    
    נבחין כי $CD$ אינו מוגדר, כי $D \in M_{3 \times 1}(\R)$ ו־$C \in M_{3 \times 4}$, אך $4 \neq 3$. 
    
    \npage
    
    \section{}
    תהי $A \in M_n(\F)$ ונגדיר $\tr(A)$ להיות סכום איברי האלכסון הראשי. נוכיח את הטענות הבאות: 
    \begin{enumerate}[(A)]
        \item נוכיח $\forall A< B \in M_n(\F) \co \tr(A + B) = \tr A + \tr B$ \begin{proof}
            יהיו $A, B \in M_n(\F)$ מטריצות. אז: 
            \[ \tr(A + B) = \sum_{i = 1}^{n}(A + B)_{ii} = \sum_{i = 1}^{n}(A)_{ii} + (B)_{ii} = \sum_{i = 1}^{n}(A)_ii + \sum_{i = 1}^{n}(B)_{ii} = \tr A + \tr B \quad \top \]
        \end{proof}
        \item נוכיח שלכל $C \in M_{m \times n}(\F), \ D \in M_{n \times m}(\F)$ מתקיים $\tr(CD) = \tr(DC)$. \begin{proof}
            יהיו $C \in M_{m \times n}(\F),\ D \in M_{n \times m}(\F)$ מטריצות. אז: 
            \[ \sum_{i = 1}^{n}(CD)_{ii} = \sum_{i = 1}^{n}\sum_{j = 1}^{m}(C)_{ji}(D)_{ij} \overset{(1)}{=} \sum_{j = 1}^{m}\sum_{i = 1}^{n}(D)_{ij}(C)_{ji} = \sum_{i = 1}^{m}(DC)_{ii} = \tr(DC) \]
            כאשר $(1)$ נובע מהחלפת סדר סכימה (חוקי מקומטטיביות סכום) ומקומטטיביות כפל. 
        \end{proof}
    \end{enumerate}
    
    \section{}
    יהיו $A, B \in M_n(\F)$ הפיכות, ותהי $C \in M_n(\F)$. נוכיח או נפריך את הטענות הבאות. 
    \begin{enumerate}[(A)]
        \item נפריך $A + B$ הפיכה. \begin{proof}[הפרכה.]
            נראה דוגמה נגדית. נבחר $A = I, \ B = -I$ הפיכות. אז $A + B = 0$ שאיננה הפיכה, בסתירה לטענה. 
        \end{proof}
        \item נפריך $AC$ הפיכה. \begin{proof}[הפרכה.]
            נראה דוגמה נגדית. עבור $A = I$ הפיכה, ו־$C = 0$ מטריצה $C \in M_n(\F)$, מתקיים $AC = I \cdot 0 = 0$ שאיננה הפיכה. 
        \end{proof}
        \item נוכיח $ABA$ הפיכה. \begin{proof}
            \textit{למה. }לכל זוג מטריצות $\bar A, \bar B \in M_n(\F)$ הפיכות, $AB$ הפיכה. נוכיח את הלמה: מהיות $\bar A, \bar B$ הפיכות, הן מייצגות בבסיס סטנדרטי $E$ פונקציה איזו', ונסמן $T_A(x) := \bar Ax, \ T_B := \bar Bx$ ובפרט $[T_A]_E = A, \ [T_B]_E = B$. נסיק $AB = [T_A]_E \cdot [T_B]_E = [T_A \circ T_B]_E$. הרכבת איזו' היא איזו' ולכן $T_A \circ T_B$ איזו', כלומר $[T_A \circ T_B]_E$ הפיכה כדרוש. 
            
            הלמה הוכחה. ממנה, $AB$ הפיכה כי $A, B$ הפיכות. לכן גם $ABA$ הפיכה כי $AB$ הפיכה ו־$A$ הפיכה, כדרוש. 
        \end{proof}
        \item נוכיח שאם $\F \subseteq \K$ שדה המרחיב את $\F$, אז $A$ הפיכה ב־$M_n(\K)$. \begin{proof}
            ידוע $M_n(\F) \subseteq M_n(\K)$, ותהי $M \in M_n(\F)$ הפיכה ב־$M_n(\F)$. נסיק $M \in M_n(\K)$, ונראה שגם בחוג הזה היא הפיכה. מהיות $M$ הפיכה מעל $M_n(\F)$, ידוע קיום מטריצות אלמנטריות $E_1 \dots E_k$ כך ש־$M \cdot \prod_{i = 1}^{n} E_i = I$ (כי היא ניתנת לדירוג לכדי $I$ מהגדרה). אזי $E_i \in M_n(\K)$ מהכלה. בפרט לכל זוג מטריצות מעל $A, B \in M_n(\F)$, נבחין כי $AB$ נשאר זהה מעל $M_n(\K)$ כי כפל מטריצות מוגדר באמצעות פעולות חיבור וחיסור על השדה, שנותרו זהות במקרה הזה (אחרת צמצום הפעולות מ־$\K$ ל־$\F$ שונה מהפעולות על $\F$, וזו סתירה לנתון ש־$\K$ מרחיב את $\F$). לכן גם נשאר השוויון $M \cdot \prod_{i = 1}^{n} E_i = I$ מעל $M_n(\K)$. באופן שקול $M$ הפיכה מעל $M_n(\K)$ כדרוש. 
        \end{proof}
        \item נפריך את הטענה $AC = BC \implies A = B$ בקשירה לעיל. \begin{proof}[הפרכה.]
            נבחר $C = 0, \ A = I, \ B = -I$, ונבחין כי $A, B$ הפיכות. נבחין כי $AC = A \cdot 0 = 0 = B \cdot 0 = BC$ וכן $A = I \neq -I = B$, כלומר $AC = BC \land A \neq B$ בסתירה לטענה. 
        \end{proof}
    \end{enumerate}
    
    \section{}
    \begin{enumerate}[(A)]
        \item מעל $\R$:
        \[ A: =\pms{1 & 2 & 3 \\ 4 & 5 & 6 \\ 7 & 8 & 9} \]
        נוכיח שהמטריצה לא הפיכה באמצעות כך שנראה ש־$\rk A < 3$ באמצעות הוכחת ת''ליות עמודותיה. 
        \[ 1 \cdot \pms{1 \\ 2 \\ 3} -2\pms{4 \\ 5 \\ 6} + 1 \cdot \pms{7 \\ 8 \\ 9} = \pms{1 -8 + 7 \\ 2 - 10 + 8 \\ 3 - 12 + 9} = \pms{0 \\ 0 \\ 0} \]
        \item מעל $\R$:
        \[ \pms{0 & 1 & 2 \\ 2 & 5 & 1 \\ 1 & -4 & 3} \]
        נוכיח שהיא הפיכה ונמצא את ההפיכה שלה: 
        \begin{gather*}
            \tmat{0 & 1 & 2 \\ 2 & 5 & 1 \\ 1 & -4 & 3}{I} \rrr{R_1 \lra R_3} \tmat{1 & -4 & 3 \\ 2 & 5 & 1 \\ 0 & 1 & 2}{0 & 0 & 1 \\ 0 & 1 & 0 \\ 1 & 0 & 0} \rrr{R_2 \to R_2 - 2R_1}
            \tmat{1 & -4 & 3 \\ 0 & 13 & -5 \\ 0 & 1 & 2}{0 & 0 & 1 \\ 0 & 1 & -2 \\ 1 & 0 & 0} \\
            \rrr{R_2 \to R_2 \lra R_3}
            \tmat{1 & -4 & 3 \\ 0 & 1 & 2 \\ 0 & 13 & -5}{0 & 0 & 1 \\ 1 & 0 & 0 \\ 0 & 1 &2} 
            \rrr{R_3 \to R_3 - 13R_2} 
            \tmat{1 & -4 & 3 \\ 0 & 1 & 2 \\ 0 & 0 & -31}{0 & 0 & 1 \\ 1 & 0 & 0 \\ -13 & 1 & 2}
            \rrr{R_1 \to \frac{1}{-31}R_1}
            \tmat{1 & -4 & 3 \\ 0 & 1 & 2 \\ 0 & 0 & 1}{0 & 0 & 1 \\ 1 & 0 & 0 \\ -\frac{13}{31} & \frac{1}{31} & \frac{2}{31}} \\
            \rrt{R_2 \to 2R_3}{R_1 \to R_1 - R_3} 
            \tmat{1 & -4 & 0 \\ 0 & 1 & 0 \\ 0 & 0 & 1}{0 & 0 & \frac{29}{31} \\ \frac{18}{31} & 0 & 0 \\ -\frac{13}{31} & \frac{1}{31} & \frac{2}{31}}
            \rrr{R_1 \to R_1 + 4R_2}
            \tmat{1 & 0 & 0 \\ 0 & 1 & 0 \\ 0 & 0 & 1}{\frac{71}{31} & 0 & \frac{29}{31} \\ \frac{18}{31} & 0 & 0 \\ -\frac{13}{31} & \frac{1}{31} & \frac{2}{31}} \implies A \op = \pms{\frac{71}{31} & 0 & \frac{29}{31} \\ \frac{18}{31} & 0 & 0 \\ -\frac{13}{31} & \frac{1}{31} & \frac{2}{31}} \\
        \end{gather*}
        \item מעל $\Z_7$
        \begin{gather*}
            \tmat{0 & 1 & 2 \\ 2 & 5 & 1 \\ 1 & -4 & 3}{I}
            \rrr{R_1 \lra R_3} \tmat{1 & 3 & 3 \\ 2 & 5 & 1 \\ 0 & 1 & 2}{0 & 0 & 1 \\ 0 & 1 & 0 \\ 1 & 0 & 0}
            \rrr{R_2 \to R_2 + 5R_1} \tmat{1 & 3 & 3 \\ 0 & 6 & 2 \\ 0 & 1 & 2}{0 & 0 & 1 \\ 0 & 1 & 5 \\ 1 & 0 & 0}
            \rrr{R_2 \to R_2 + R_3} \tmat{1 & 3 & 3 \\ 0 & 0 & 4 \\ 0 & 1 & 2}{0 & 0 & 1 \\ 1 & 1 & 5 \\ 1 & 0 & 0} \\
            \rrt{R_3 \to 2R_3}{R_2 \lra R_3} \tmat{1 & 3 & 3 \\ 0 & 1 & 2 \\ 0 & 0 & 1}{0 & 0 & 1 \\ 1 & 0 & 0 \\ 2 & 2 & 3} 
            \rrt{R_2 \to R_2 + 5R_3}{R_1 \to R_1 + 4R_1} \tmat{1 & 3 & 0 \\ 0 & 1 & 0 \\ 0 & 0 & 1}{2 & 1 & 5 \\ 4 & 3 & 1 \\ 2 & 2 & 3}
            \rrr{R_1 \to R_1 + 4R_2} \tmat{1 & 0 & 0 \\ 0 & 1 & 0 \\ 0 & 0 & 1}{4 & 6 & 2 \\ 4 & 3 & 1 \\ 2 & 2 & 3} \implies A \op = \pms{4 & 6 & 2 \\ 4 & 3 & 1 \\ 2 & 2 & 3}
        \end{gather*}
        \item מעל $\R$:
        \begin{gather*}\tomat \tmat{1 & -1 & 1 & 1 \\ 
                1 & 1 & 2 & -1 \\ 
                2 & 1 & 2 & 1 \\ 
                -1 & 1 & 1 & 1 \\ 
            }{I} \rrt{\overset{R_2 \to R_2 - R_1}{R_3 \to R_3 - 2 R_1}}{R_4 \to R_4 + R_1} \tmat{1 & -1 & 1 & 1 \\ 
                0 & 2 & 1 & -2 \\ 
                0 & 3 & 0 & -1 \\ 
                0 & 0 & 2 & 2 \\ 
            }{1 & 0 & 0 & 0 \\ -1 & 1 & 0 & 0 \\ 0 & -2 & 1 & 0 \\ 0 & 0 & 1 & 1} \rrr{R_2 \to 0.5R_2} \tmat{1 & -1 & 1 & 1 \\ 
                0 & 1 & \frac{1}{2} & -1 \\ 
                0 & 3 & 0 & -1 \\ 
                0 & 0 & 2 & 2 \\ 
            }{1 & 0 & 0 & 0 \\ -0.5 & 0.5 & 0 & 0 \\ 0 & -2 & 1 & 0 \\ 0 & 0 & 1 & 1} \\\rrr{R_3 \to R_3 - 3 R_2} \tmat{1 & -1 & 1 & 1 \\ 
                0 & 1 & \frac{1}{2} & -1 \\ 
                0 & 0 & \frac{3}{-2} & 2 \\ 
                0 & 0 & 2 & 2 \\ 
            }{1 & 0 & 0 & 0 \\ -0.5 & 0.5 & 0 & 0 \\ 1.5 & -3.5 & 1 & 0 \\ 0 & 0 & 1 & 1} 
            \rrr{R_3 \to R_3 \cdot \frac{-2}{3}} \tmat{1 & -1 & 1 & 1 \\ 
                0 & 1 & \frac{1}{2} & -1 \\ 
                0 & 0 & 1 & \frac{4}{-3} \\ 
                0 & 0 & 2 & 2 \\ 
            }{1 & 0 & 0 & 0 \\ -0.5 & 0.5 & 0 & 0 \\ -1 & -\frac{7}{3} & -\frac{2}{3} & 0 \\ 0 & 0 & 1 & 1}
            \\\rrr{R_4 \to R_4 - 2 R_3} \tmat{1 & -1 & 1 & 1 \\ 
                0 & 1 & \frac{1}{2} & -1 \\ 
                0 & 0 & 1 & \frac{4}{-3} \\ 
                0 & 0 & 0 & \frac{14}{3} \\ 
            }{1 & 0 & 0 & 0 \\ -0.5 & 0.5 & 0 & 0 \\ -1 & -\frac{7}{3} & -\frac{2}{3} & 0 \\ 2 & \frac{14}{3} & \frac{7}{3} & 1}
            \rrr{R_4 \to \frac{3}{14}R_4 } \tmat{1 & -1 & 1 & 1 \\ 
                0 & 1 & \frac{1}{2} & -1 \\ 
                0 & 0 & 1 & \frac{4}{-3} \\ 
                0 & 0 & 0 & 1 \\ 
            }{1 & 0 & 0 & 0 \\ -0.5 & 0.5 & 0 & 0 \\ -1 & -\frac{7}{3} & -\frac{2}{3} & 0 \\ \frac{3}{7} & 1  & \frac{3}{8}  & \frac{3}{14}}
            \\\rrt{\overset{R_3 \to R_3 + \frac{4}{3} R_4}{R_2 \to R_2 + R_4}}{R_1 \to R_1 - 1 R_4} \tmat{1 & -1 & 1 & 0 \\ 
                0 & 1 & \frac{1}{2} & 0 \\ 
                0 & 0 & 1 & 0 \\ 
                0 & 0 & 0 & 1 \\ 
            }{\frac{4}{7} & -1 & -\frac{3}{8} & -\frac{3}{14} \\
            -\frac{1}{14} & 1 & \frac{3}{8} & \frac{3}{14} \\
            -\frac{3}{7} & -1 & -\frac{1}{6} & \frac{2}{7} \\
            \frac{3}{7} & 1  & \frac{3}{8}  & \frac{3}{14}}
            \rrt{R_2 \to R_2 - \frac{1}{2} R_3}{R_1 \to R_1 - 1 R_3} \tmat{1 & -1 & 0 & 0 \\ 
                0 & 1 & 0 & 0 \\ 
                0 & 0 & 1 & 0 \\ 
                0 & 0 & 0 & 1 \\ 
            }{1 & 0 & -\frac{5}{24} & -\frac{1}{2} \\
            \frac{1}{7} & \frac{3}{2} & \frac{11}{24} & \frac{3}{98} \\
            -\frac{3}{7} & -1 & -\frac{1}{6} & \frac{2}{7} \\
            \frac{3}{7} & 1  & \frac{3}{8}  & \frac{3}{14}} \\\rrr{R_1 \to R_1 + R_2} \tmat{1 & 0 & 0 & 0 \\ 
                0 & 1 & 0 & 0 \\ 
                0 & 0 & 1 & 0 \\ 
                0 & 0 & 0 & 1 \\ 
            }{\frac{8}{7} & \frac{3}{2} & \frac{1}{4} & -\frac{23}{49} \\
            \frac{1}{7} & \frac{3}{2} & \frac{11}{24} & \frac{3}{98} \\
            -\frac{3}{7} & -1 & -\frac{1}{6} & \frac{2}{7} \\
            \frac{3}{7} & 1  & \frac{3}{8}  & \frac{3}{14}} \implies A \op = \pms{\frac{8}{7} & \frac{3}{2} & \frac{1}{4} & -\frac{23}{49} \\
            \frac{1}{7} & \frac{3}{2} & \frac{11}{24} & \frac{3}{98} \\
            -\frac{3}{7} & -1 & -\frac{1}{6} & \frac{2}{7} \\
            \frac{3}{7} & 1  & \frac{3}{8}  & \frac{3}{14}}
        \end{gather*}
        
        \item מעל $\R$, נמצא את ההופכית למטריצה הבאה (אם יש כזו, כתלות בערכי $\lg \in \R$): 
        \begin{multline*}
            A:= \tmat{1 & 0 & 1 \\ 0 & 1 & \lg \\ 0 & \lg & 1}{ I } \rrt{R_3 \to R_3 - \lg R_2}{(1)}
            \tmat{1 & 0 & 1 \\ 0 & 1 & \lg \\ 0 & 0 & 1 - \lg^2}{1 & 0 & 0 \\ 0 & 1 & 0 \\ 0 & -\lg & 1} \rrt{R_3 \to \frac{1}{1 - \lg^2}R_3}{(2)}
            \tmat{1 & 0 & 1 \\ 0 & 1 & \lg \\ 0 & 0 & 1}{1 & 0 & 1 \\ 0 & 1 & 0 \\ 0 & \frac{-\lg}{1 - \lg^2} & \frac{1}{1 - \lg^2}} \\ \rrt{R_2 \to R_2 - \lg R_1}{R_1 \to R_1 - R_3}
            \tmat{I}{1 & 0 & 1 - \frac{1}{1 - \lg^2} \\ 0 & 1 & -\frac{\lg}{1 - \lg^2} \\ 0 & \frac{-\lg}{1 - \lg^2} & \frac{1}{1 - \lg^2}}
            \implies A\op = \pms{1 & 0 & -\frac{\lg^2}{1 - \lg^2} \\ 0 & 1 & 0 \\ 0 & -\frac{\lg}{1 - \lg^2} & \frac{1}{1 - \lg^2}}
        \end{multline*}
        \begin{itemize}
            \item במקרה והנחה $(1)$ לא נכונה ו־$\lg = 0$: 
            \[ \tmat{1 & 0 & 1 \\ 0 & 1 & 0 \\ 0& 0 & 1}{I} \rrr{R_1 \to R_1 - R_3} \tmat{I}{1 & 0 & -1 \\ 0 & 1 & 0 \\ 0 & 0 & 1} \]
            \item במידה והנחה $(2)$ לא נכונה ו־$1 - \lg^2 = 0$: 
            \[ \pms{1 & 0 & 1 \\ 0 & 1 & \lg \\ 0 & 0 & 0} \]
            מטריצה לא הפיכה כי יש שורת אפסים ולכן שורותיה ת''ל. 
        \end{itemize}
        נסכם: 
        \[ A\op = \begin{cases}
            \pms{1 & 0 & -1 \\ 0 & 1 & 0 \\ 0 & 0 & 1} & \lg = 0 \\
            \text{\en{\quad ininevitable}} & \lg = \pm1 \\
            \pms{1 & 0 & -\frac{\lg^2}{1 - \lg^2} \\ 0 & 1 & 0 \\ 0 & -\frac{\lg}{1 - \lg^2} & \frac{1}{1 - \lg^2}} & \other
        \end{cases} \]
    \end{enumerate}
    
    \section{}
    תהיינה $A, B$ ריבועיות המקיימות $A^2 = A \land B^k = 0$. נוכיח $I - B, I + A$ הפיכות. \begin{proof}\, 
        \begin{itemize}
            \item נוכיח ש־$I + A$ הפיכה: 
            \[ I = A + I - A = A + I - \underbrace{\frac{1}{2}A}_{=\frac{1}{2}A^2} + \frac{1}{2} A = -\frac{1}{2}A^2 + A - \frac{1}{2}A + I = (A + I)\cl{A - \frac{1}{2}} \implies A\op = A - \frac{1}{2}I \]
            \item נוכיח ש־$I - B$ הפיכה. לכפול בצורה כזו שכל איבר ``יבטל'' את קודמו. כלומר: 
            \[ (B - I)(B^{k - 1} + \cdots + I) = B^{k} + B^{k - 1} - B^{k - 1} + \cdots + I = I \]
            פורמלית: 
            \[ (B - I)\cl{\sum_{i = 1}^{k - 1}B^i} = \sum_{i = 2}^{k}B^i + \sum_{i = 1}^{k - 1}B^i = I + B^k + \sum_{i = 2}^{k - 1} \cancel{(B^i - B^i)} = I + B^k = I \]
            כי $B^k = 0$, כדרוש. 
        \end{itemize}
    \end{proof}
    
    
    \dotfill
    
    \textbf{העתק הדבק מהתרגיל בית 4, שאלה שעוזרת לתרגיל 1: }
    
    יהיו: 
    \[ \Sym_n(\F) = \{A \in M_n(\F) \mid \forall i, j \co A_{ij} = A_{ji}\}, \ \ASym_n(\F) = \{A \in M_n(\F) \mid \forall i, j \co A_{ij} = -A_{ji}\} \]
    נראה ש־$\ASym_n(\F) + \Sym(\F) = M_n(\F)$ (סכום ישר) נסמן $\Sym := \Sym_n(\F)$ ו־$\ASym := \ASym_n(\F)$. 
    \begin{proof}\,
    \begin{itemize}
        \item נבחין ש־$\Sym_n(\F), \ASym_n(\F) \subseteq M_n(\F)$ מעקרון ההפרדה. עתה נראה בשניהם סגירות לחיבור ולכפל: 
        \begin{itemize}
            \item \textbf{סגירות לכפל: }
            \begin{itemize}
                \item עבור $\Sym$, יהיו $\lg \in \F, \ M \in \Sym$, ונבחין שלכל $i, j \in [n]$ עדיין $\lg M$ מקיים $(\lg M)_{ij} = \lg (M)_{\ij} = \lg (M)_{ji} = (\lg M)_{ji}$. 
                \item עבור $\ASym$, לכל $\lg \in \F , \ M \in \ASym$ עדיין מתקיים $\forall i, j \in [n]\co (\lg M)_{ij} = \lg (M)_{ij} = -\lg (M)_{ji} = -(\lg M)_{ji}$. 
            \end{itemize}
            \item \textbf{סגירות לחיבור: }
            \begin{itemize}
                \item עבור $\Sym$, יהיו $M, P \in \Sym$ ואכן $(M + P)_{ij} = (M)_{ij} + (P)_{ij} = (M)_{ji} + (P)_{ji} = (M + P)_{ji}$. 
                \item עבור $\ASym$, יהיו $M, P \in \ASym$ ואכן $(M + P)_{ij} = (M)_{ij} + (P)_{ij} = -(M)_{ji} - (P)_{ji} = -(M + P)_{ji}$. 
            \end{itemize}
            \item \textbf{קיום אפס: }
            \begin{itemize}
                \item עבור $\Sym$ נבחין ש־
                $(0_M)_{ij} = 0_\F = (0_M)_{ji}$
                \item עבור $\ASym$ נבחין ש־: 
                $(0_M)_{ij} = 0_\F = -0_\F = -(0_M)_{ji}$
            \end{itemize}
        \end{itemize}
        סה"כ הראינו ש־$\Sym, \ \ASym$ תמ"וים. נראה ש־$\Sym \cap \ASym = \{0_M\}$. יהי $A \in \Sym \cap \ASym$. אזי: 
        \[ -(A)_{ij} \!\overset{\Sym}{=}\! -(A)_{ji} \!\overset{\ASym}{=}\! (A)_{ij} \]
        אם $(A)_{ij} \neq 0$, אז נוכל לחלק בו ולקבל $-1 = 1$ וזו סתירה. סה"כ $\forall i, j \in [n] \co (A)_{ij} = 0$, ולכן $A = 0_M$, כלומר אכן $\Sym \cap \ASym = \{0_M\}$ כדרוש. 
        \item עתה נראה שלכל $A \in M_n(\F)$ קיימות שתי מטריצות $A_s, \ A_{as} \in \ASym$ כך ש־$A = A_s + A_{as}$. 
            תהי $A \in M_n(\F)$, נסמן $A_s = \frac{A + A^T}{2}$ ו־$A_{as} = \frac{A - A^T}{2}$. אזי: 
            \begin{itemize}
                \item $\bm{A_s \in \Sym}$\textbf{: }יהיו $i, j \in [n]$, אז: 
                \[ (A_s)_{ij} = \cl{\frac{A + A^T}{2}}_{ij} = \frac{\overbrace{(A)_{ij}}^{(A^T)_{ji}} + \overbrace{(A^T)_{ij}}^{(A)_{ji}}}{2} = \frac{(A)_{ji} + (A^T)_{ji}}{2} = (A_s)_{ji} \quad \top \]
                \item $\bm{A_{as} \in \Sym}$\textbf{: }יהיו $i, j \in [n]$, אז: 
                \[ (A_{as})_{ij} = \cl{\frac{A - A^T}{2}}_{ij} = \frac{\overbrace{(A)_{ij}}^{(A^T)_{ji}} - \overbrace{(A^T)_{ij}}^{(A)_{ji}}}{2} = \frac{(A^T)_{ji} - (A)_{ji}}{2} = -\frac{(A)_{ji} + (A^T)_{ji}}{2} = -(A_s)_{ji} \quad \top \]
                \item $\bm{A_s + A_{as} = A}$\textbf{: }
                \[ A_s + A_{as} = \frac{A + A^T}{2} + \frac{A - A^T}{2} = \frac{A + A + \cancel{A^T - A^T}}{2} = \frac{2A}{2} = A \quad \top \]
            \end{itemize}
        
    \end{itemize}
    סה"כ הראינו ש־$\ASym_n(\F) \oplus \Sym_n(\F) = M_n(\F)$ סכום ישר, כדרוש. 
    \end{proof}
    
    
    \ndoc
\end{document}