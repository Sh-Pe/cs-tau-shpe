%! ~~~ Packages Setup ~~~ 
\documentclass[]{article}
\usepackage{lipsum}
\usepackage{rotating}


% Math packages
\usepackage[usenames]{color}
\usepackage{forest}
\usepackage{ifxetex,ifluatex,amssymb,amsmath,mathrsfs,amsthm,witharrows,mathtools,mathdots}
\usepackage{amsmath}
\WithArrowsOptions{displaystyle}
\renewcommand{\qedsymbol}{$\blacksquare$} % end proofs with \blacksquare. Overwrites the defualts. 
\usepackage{cancel,bm}
\usepackage[thinc]{esdiff}


% tikz
\usepackage{tikz}
\usetikzlibrary{graphs}
\newcommand\sqw{1}
\newcommand\squ[4][1]{\fill[#4] (#2*\sqw,#3*\sqw) rectangle +(#1*\sqw,#1*\sqw);}


% code 
\usepackage{listings}
\usepackage{xcolor}

\definecolor{codegreen}{rgb}{0,0.35,0}
\definecolor{codegray}{rgb}{0.5,0.5,0.5}
\definecolor{codenumber}{rgb}{0.1,0.3,0.5}
\definecolor{codeblue}{rgb}{0,0,0.5}
\definecolor{codered}{rgb}{0.5,0.03,0.02}
\definecolor{codegray}{rgb}{0.96,0.96,0.96}

\lstdefinestyle{pythonstylesheet}{
	language=Java,
	emphstyle=\color{deepred},
	backgroundcolor=\color{codegray},
	keywordstyle=\color{deepblue}\bfseries\itshape,
	numberstyle=\scriptsize\color{codenumber},
	basicstyle=\ttfamily\footnotesize,
	commentstyle=\color{codegreen}\itshape,
	breakatwhitespace=false, 
	breaklines=true, 
	captionpos=b, 
	keepspaces=true, 
	numbers=left, 
	numbersep=5pt, 
	showspaces=false,                
	showstringspaces=false,
	showtabs=false, 
	tabsize=4, 
	morekeywords={as,assert,nonlocal,with,yield,self,True,False,None,AssertionError,ValueError,in,else},              % Add keywords here
	keywordstyle=\color{codeblue},
	emph={var, List, Iterable, Iterator},          % Custom highlighting
	emphstyle=\color{codered},
	stringstyle=\color{codegreen},
	showstringspaces=false,
	abovecaptionskip=0pt,belowcaptionskip =0pt,
	framextopmargin=-\topsep, 
}
\newcommand\pythonstyle{\lstset{pythonstylesheet}}
\newcommand\pyl[1]     {{\lstinline!#1!}}
\lstset{style=pythonstylesheet}

\usepackage[style=1,skipbelow=\topskip,skipabove=\topskip,framemethod=TikZ]{mdframed}
\definecolor{bggray}{rgb}{0.85, 0.85, 0.85}
\mdfsetup{leftmargin=0pt,rightmargin=0pt,innerleftmargin=15pt,backgroundcolor=codegray,middlelinewidth=0.5pt,skipabove=5pt,skipbelow=0pt,middlelinecolor=black,roundcorner=5}
\BeforeBeginEnvironment{lstlisting}{\begin{mdframed}\vspace{-0.4em}}
	\AfterEndEnvironment{lstlisting}{\vspace{-0.8em}\end{mdframed}}


% Deisgn
\usepackage[labelfont=bf]{caption}
\usepackage[margin=0.6in]{geometry}
\usepackage{multicol}
\usepackage[skip=4pt, indent=0pt]{parskip}
\usepackage[normalem]{ulem}
\forestset{default}
\renewcommand\labelitemi{$\bullet$}
\usepackage{titlesec}
\titleformat{\section}[block]
{\fontsize{15}{15}}
{\sen \dotfill (\thesection)\dotfill\she}
{0em}
{\MakeUppercase}
\usepackage{graphicx}
\graphicspath{ {./} }


% Hebrew initialzing
\usepackage[bidi=basic]{babel}
\PassOptionsToPackage{no-math}{fontspec}
\babelprovide[main, import, Alph=letters]{hebrew}
\babelprovide[import]{english}
\babelfont[hebrew]{rm}{David CLM}
\babelfont[hebrew]{sf}{David CLM}
\babelfont[english]{tt}{Monaspace Xenon}
\usepackage[shortlabels]{enumitem}
\newlist{hebenum}{enumerate}{1}

% Language Shortcuts
\newcommand\en[1] {\begin{otherlanguage}{english}#1\end{otherlanguage}}
\newcommand\sen   {\begin{otherlanguage}{english}}
	\newcommand\she   {\end{otherlanguage}}
\newcommand\del   {$ \!\! $}

\newcommand\npage {\vfil {\hfil \textbf{\textit{המשך בעמוד הבא}}} \hfil \vfil \pagebreak}
\newcommand\ndoc  {\dotfill \\ \vfil {\begin{center}
			{\textbf{\textit{שחר פרץ, 2025}} \\
				\scriptsize \textit{קומפל ב־}\en{\LaTeX}\,\textit{ ונוצר באמצעות תוכנה חופשית בלבד}}
	\end{center}} \vfil	}

\newcommand{\rn}[1]{
	\textup{\uppercase\expandafter{\romannumeral#1}}
}

\makeatletter
\newcommand{\skipitems}[1]{
	\addtocounter{\@enumctr}{#1}
}
\makeatother

%! ~~~ Math shortcuts ~~~

% Letters shortcuts
\newcommand\N     {\mathbb{N}}
\newcommand\Z     {\mathbb{Z}}
\newcommand\R     {\mathbb{R}}
\newcommand\Q     {\mathbb{Q}}
\newcommand\C     {\mathbb{C}}
\newcommand\One   {\mathit{1}}

\newcommand\ml    {\ell}
\newcommand\mj    {\jmath}
\newcommand\mi    {\imath}

\newcommand\powerset {\mathcal{P}}
\newcommand\ps    {\mathcal{P}}
\newcommand\pc    {\mathcal{P}}
\newcommand\ac    {\mathcal{A}}
\newcommand\bc    {\mathcal{B}}
\newcommand\cc    {\mathcal{C}}
\newcommand\dc    {\mathcal{D}}
\newcommand\ec    {\mathcal{E}}
\newcommand\fc    {\mathcal{F}}
\newcommand\nc    {\mathcal{N}}
\newcommand\vc    {\mathcal{V}} % Vance
\newcommand\sca   {\mathcal{S}} % \sc is already definded
\newcommand\rca   {\mathcal{R}} % \rc is already definded

\newcommand\prm   {\mathrm{p}}
\newcommand\arm   {\mathrm{a}} % x86
\newcommand\brm   {\mathrm{b}}
\newcommand\crm   {\mathrm{c}}
\newcommand\drm   {\mathrm{d}}
\newcommand\erm   {\mathrm{e}}
\newcommand\frm   {\mathrm{f}}
\newcommand\nrm   {\mathrm{n}}
\newcommand\vrm   {\mathrm{v}}
\newcommand\srm   {\mathrm{s}}
\newcommand\rrm   {\mathrm{r}}

\newcommand\Si    {\Sigma}

% Logic & sets shorcuts
\newcommand\siff  {\longleftrightarrow}
\newcommand\ssiff {\leftrightarrow}
\newcommand\so    {\longrightarrow}
\newcommand\sso   {\rightarrow}

\newcommand\epsi  {\epsilon}
\newcommand\vepsi {\varepsilon}
\newcommand\vphi  {\varphi}
\newcommand\Neven {\N_{\mathrm{even}}}
\newcommand\Nodd  {\N_{\mathrm{odd }}}
\newcommand\Zeven {\Z_{\mathrm{even}}}
\newcommand\Zodd  {\Z_{\mathrm{odd }}}
\newcommand\Np    {\N_+}

% Text Shortcuts
\newcommand\open  {\big(}
\newcommand\qopen {\quad\big(}
\newcommand\close {\big)}
\newcommand\also  {\text{, }}
\newcommand\defis {\text{ definitions}}
\newcommand\given {\text{given }}
\newcommand\case  {\text{if }}
\newcommand\syx   {\text{ syntax}}
\newcommand\rle   {\text{ rule}}
\newcommand\other {\text{else}}
\newcommand\set   {\ell et \text{ }}
\newcommand\ans   {\mathscr{A}\!\mathit{nswer}}

% Set theory shortcuts
\newcommand\ra    {\rangle}
\newcommand\la    {\langle}

\newcommand\oto   {\leftarrow}

\newcommand\QED   {\quad\quad\mathscr{Q.E.D.}\;\;\blacksquare}
\newcommand\QEF   {\quad\quad\mathscr{Q.E.F.}}
\newcommand\eQED  {\mathscr{Q.E.D.}\;\;\blacksquare}
\newcommand\eQEF  {\mathscr{Q.E.F.}}
\newcommand\jQED  {\mathscr{Q.E.D.}}

\DeclareMathOperator\dom   {dom}
\DeclareMathOperator\Img   {Im}
\DeclareMathOperator\range {range}

\newcommand\trio  {\triangle}

\newcommand\rc    {\right\rceil}
\newcommand\lc    {\left\lceil}
\newcommand\rf    {\right\rfloor}
\newcommand\lf    {\left\lfloor}

\newcommand\lex   {<_{lex}}

\newcommand\az    {\aleph_0}
\newcommand\uaz   {^{\aleph_0}}
\newcommand\al    {\aleph}
\newcommand\ual   {^\aleph}
\newcommand\taz   {2^{\aleph_0}}
\newcommand\utaz  { ^{\left (2^{\aleph_0} \right )}}
\newcommand\tal   {2^{\aleph}}
\newcommand\utal  { ^{\left (2^{\aleph} \right )}}
\newcommand\ttaz  {2^{\left (2^{\aleph_0}\right )}}

\newcommand\n     {$n$־יה\ }

% Math A&B shortcuts
\newcommand\logn  {\log n}
\newcommand\logx  {\log x}
\newcommand\lnx   {\ln x}
\newcommand\cosx  {\cos x}
\newcommand\sinx  {\sin x}
\newcommand\sint  {\sin \theta}
\newcommand\tanx  {\tan x}
\newcommand\tant  {\tan \theta}
\newcommand\sex   {\sec x}
\newcommand\sect  {\sec^2}
\newcommand\cotx  {\cot x}
\newcommand\cscx  {\csc x}
\newcommand\sinhx {\sinh x}
\newcommand\coshx {\cosh x}
\newcommand\tanhx {\tanh x}

\newcommand\seq   {\overset{!}{=}}
\newcommand\slh   {\overset{LH}{=}}
\newcommand\sle   {\overset{!}{\le}}
\newcommand\sge   {\overset{!}{\ge}}
\newcommand\sll   {\overset{!}{<}}
\newcommand\sgg   {\overset{!}{>}}

\newcommand\h     {\hat}
\newcommand\ve    {\vec}
\newcommand\lv    {\overrightarrow}
\newcommand\ol    {\overline}

\newcommand\mlcm  {\mathrm{lcm}}

\DeclareMathOperator{\sech}   {sech}
\DeclareMathOperator{\csch}   {csch}
\DeclareMathOperator{\arcsec} {arcsec}
\DeclareMathOperator{\arccot} {arcCot}
\DeclareMathOperator{\arccsc} {arcCsc}
\DeclareMathOperator{\arccosh}{arccosh}
\DeclareMathOperator{\arcsinh}{arcsinh}
\DeclareMathOperator{\arctanh}{arctanh}
\DeclareMathOperator{\arcsech}{arcsech}
\DeclareMathOperator{\arccsch}{arccsch}
\DeclareMathOperator{\arccoth}{arccoth}
\DeclareMathOperator{\atant}  {atan2} 
\DeclareMathOperator{\Sp}     {span} 
\DeclareMathOperator{\sgn}    {sgn} 
\DeclareMathOperator{\row}    {Row} 
\DeclareMathOperator{\adj}    {adj} 
\DeclareMathOperator{\rk}     {rank} 
\DeclareMathOperator{\col}    {Col} 
\DeclareMathOperator{\tr}     {tr}

\newcommand\dx    {\,\mathrm{d}x}
\newcommand\dt    {\,\mathrm{d}t}
\newcommand\dtt   {\,\mathrm{d}\theta}
\newcommand\du    {\,\mathrm{d}u}
\newcommand\dv    {\,\mathrm{d}v}
\newcommand\df    {\mathrm{d}f}
\newcommand\dfdx  {\diff{f}{x}}
\newcommand\dit   {\limhz \frac{f(x + h) - f(x)}{h}}

\newcommand\nt[1] {\frac{#1}{#1}}

\newcommand\limz  {\lim_{x \to 0}}
\newcommand\limxz {\lim_{x \to x_0}}
\newcommand\limi  {\lim_{x \to \infty}}
\newcommand\limh  {\lim_{x \to 0}}
\newcommand\limni {\lim_{x \to - \infty}}
\newcommand\limpmi{\lim_{x \to \pm \infty}}

\newcommand\ta    {\theta}
\newcommand\ap    {\alpha}

\renewcommand\inf {\infty}
\newcommand  \ninf{-\inf}

% Combinatorics shortcuts
\newcommand\sumnk     {\sum_{k = 0}^{n}}
\newcommand\sumni     {\sum_{i = 0}^{n}}
\newcommand\sumnko    {\sum_{k = 1}^{n}}
\newcommand\sumnio    {\sum_{i = 1}^{n}}
\newcommand\sumai     {\sum_{i = 1}^{n} A_i}
\newcommand\nsum[2]   {\reflectbox{\displaystyle\sum_{\reflectbox{\scriptsize$#1$}}^{\reflectbox{\scriptsize$#2$}}}}

\newcommand\bink      {\binom{n}{k}}
\newcommand\setn      {\{a_i\}^{2n}_{i = 1}}
\newcommand\setc[1]   {\{a_i\}^{#1}_{i = 1}}

\newcommand\cupain    {\bigcup_{i = 1}^{n} A_i}
\newcommand\cupai[1]  {\bigcup_{i = 1}^{#1} A_i}
\newcommand\cupiiai   {\bigcup_{i \in I} A_i}
\newcommand\capain    {\bigcap_{i = 1}^{n} A_i}
\newcommand\capai[1]  {\bigcap_{i = 1}^{#1} A_i}
\newcommand\capiiai   {\bigcap_{i \in I} A_i}

\newcommand\xot       {x_{1, 2}}
\newcommand\ano       {a_{n - 1}}
\newcommand\ant       {a_{n - 2}}

% Linear Algebra
\DeclareMathOperator{\chr}     {char}
\DeclareMathOperator{\diag}    {diag}
\DeclareMathOperator{\Hom}     {Hom}

\newcommand\lra       {\leftrightarrow}
\newcommand\chrf      {\chr(\F)}
\newcommand\F         {\mathbb{F}}
\newcommand\co        {\colon}
\newcommand\tmat[2]   {\cl{\begin{matrix}
			#1
		\end{matrix}\, \middle\vert\, \begin{matrix}
			#2
\end{matrix}}}

\makeatletter
\newcommand\rrr[1]    {\xxrightarrow{1}{#1}}
\newcommand\rrt[2]    {\xxrightarrow{1}[#2]{#1}}
\newcommand\mat[2]    {M_{#1\times#2}}
\newcommand\gmat      {\mat{m}{n}(\F)}
\newcommand\tomat     {\, \dequad \longrightarrow}
\newcommand\pms[1]    {\begin{pmatrix}
		#1
\end{pmatrix}}

% someone's code from the internet: https://tex.stackexchange.com/questions/27545/custom-length-arrows-text-over-and-under
\makeatletter
\newlength\min@xx
\newcommand*\xxrightarrow[1]{\begingroup
	\settowidth\min@xx{$\m@th\scriptstyle#1$}
	\@xxrightarrow}
\newcommand*\@xxrightarrow[2][]{
	\sbox8{$\m@th\scriptstyle#1$}  % subscript
	\ifdim\wd8>\min@xx \min@xx=\wd8 \fi
	\sbox8{$\m@th\scriptstyle#2$} % superscript
	\ifdim\wd8>\min@xx \min@xx=\wd8 \fi
	\xrightarrow[{\mathmakebox[\min@xx]{\scriptstyle#1}}]
	{\mathmakebox[\min@xx]{\scriptstyle#2}}
	\endgroup}
\makeatother


% Greek Letters
\newcommand\ag        {\alpha}
\newcommand\bg        {\beta}
\newcommand\cg        {\gamma}
\newcommand\dg        {\delta}
\newcommand\eg        {\epsi}
\newcommand\zg        {\zeta}
\newcommand\hg        {\eta}
\newcommand\tg        {\theta}
\newcommand\ig        {\iota}
\newcommand\kg        {\keppa}
\renewcommand\lg      {\lambda}
\newcommand\og        {\omicron}
\newcommand\rg        {\rho}
\newcommand\sg        {\sigma}
\newcommand\yg        {\usilon}
\newcommand\wg        {\omega}

\newcommand\Ag        {\Alpha}
\newcommand\Bg        {\Beta}
\newcommand\Cg        {\Gamma}
\newcommand\Dg        {\Delta}
\newcommand\Eg        {\Epsi}
\newcommand\Zg        {\Zeta}
\newcommand\Hg        {\Eta}
\newcommand\Tg        {\Theta}
\newcommand\Ig        {\Iota}
\newcommand\Kg        {\Keppa}
\newcommand\Lg        {\Lambda}
\newcommand\Og        {\Omicron}
\newcommand\Rg        {\Rho}
\newcommand\Sg        {\Sigma}
\newcommand\Yg        {\Usilon}
\newcommand\Wg        {\Omega}

% Other shortcuts
\newcommand\tl    {\tilde}
\newcommand\op    {^{-1}}

\newcommand\sof[1]    {\left | #1 \right |}
\newcommand\cl [1]    {\left ( #1 \right )}
\newcommand\csb[1]    {\left [ #1 \right ]}
\newcommand\ccb[1]    {\left \{ #1 \right \}}

\newcommand\bs        {\blacksquare}
\newcommand\dequad    {\!\!\!\!\!\!}
\newcommand\dequadd   {\dequad\duquad}

\renewcommand\phi     {\varphi}

\newtheorem{Theorem}{משפט}
\theoremstyle{definition}
\newtheorem{definition}{הגדרה}
\newtheorem{Lemma}{למה}
\newtheorem{Remark}{הערה}
\newtheorem{Notion}{סימון}

\newcommand\theo  [1] {\begin{Theorem}#1\end{Theorem}}
\newcommand\defi  [1] {\begin{definition}#1\end{definition}}
\newcommand\rmark [1] {\begin{Remark}#1\end{Remark}}
\newcommand\lem   [1] {\begin{Lemma}#1\end{Lemma}}
\newcommand\noti  [1] {\begin{Notion}#1\end{Notion}}

% DS
\DeclareMathOperator\amort   {amort}
\DeclareMathOperator\worst   {worst}
\DeclareMathOperator\type    {type}
\DeclareMathOperator\cost    {cost}

%! ~~~ Document ~~~
\newcommand\unquad           {\!\!\!\!}

\author{שחר פרץ}
\title{\textit{תרגיל בית 1 $\sim$ אלגברה ליניארית 1א}}
\begin{document}
	\maketitle
	\section{}
	יהיו $z, w \in \C$ מרוכבים. אזי מהיותם מרוכבים, קיימים ויחידים $a, b, c, d \in \R$ כך ש־$z = a + bi, \ w = c + di$. 
	נוכיח את הטענות להלן. 
	\begin{enumerate}[(A)]
		\item צ.ל. $\ol{z \cdot w} = \bar z \cdot \bar w$ \begin{proof}
			\[ \ol{z \cdot w} = \ol{(a + bi)(c + di)} = \ol{ac + bci + adi - bd} = \ol{(ac - bd) + (bc + ad)i} = ac - bd - bci - adi = (a - bi)(c - di) = \bar z \cdot \bar w \]
		\end{proof}
		\item צ.ל. $z = \bar z \iff z \in \R$ \begin{proof}
			\[ z = \bar z \iff (a + bi) = (a - bi) \iff bi = -bi \]
			\begin{itemize}
				\item[$\impliedby$]נניח $z \in \R$, נראה $bi = -bi$. משום ש־$b = \Im z = 0$ אז $0 \cdot i = - 0 \cdot i = 0$ כדרוש. 
				\item[$\implies$]נניח $bi = -ib$, נראה $z \in \R$. נניח בשלילה $z \notin \R$, אז $z \in \C \setminus \R$ ולכן $b = \Im z \neq 0$ וסה"כ נוכל לחלק ב־$0$ את שני האגפים ולשמור על שקילות, כלומר $i = -i$ וסה"כ סתירה. אזי $z \in \R$ כדרוש. 
			\end{itemize}
		\end{proof}
		\item צ.ל. $\Re z = \frac{1}{2}(z + \bar z)$ \begin{proof}
			\[ \Re z = a = \frac{1}{2}(2a) = \frac{1}{2}(a + bi + a - bi) = \frac{1}{2}(z + \bar z) \]
		\end{proof}
		\item צ.ל. $\Im z = \frac{1}{2i}(z - \bar z)$ \begin{proof}
			\[ \Im z = b = \frac{1}{2i}(2bi) = \frac{1}{2i}(a + bi - a + bi) = \frac{1}{2i}(z - \bar z) \]
		\end{proof}
	\end{enumerate}
	
	\section{}
	יהיו $a_0, d \in \C$ ונגדיר את האוסף $(a_i)_{i = 0}^{\inf}$ לפי כלל הנסיגה $a_{n + 1} = a_n + d$. נראה כי $\forall n \in \N \co \sum_{i = 0}^{n} a_i = \frac{(n + 1)(a_0 + a_n)}{2}$. \begin{proof}
		נוכיח באינדוקציה על $n$ את הטענה. 
		\begin{itemize}
			\item \textit{בסיס. }$n = 0$ ולכן: 
			\[ \sum_{i = 0}^{0}a_i = a_0 = \frac{2a_0}{2} = \frac{(0 + 1) \cdot (a_0 + a_0)}{2} \quad \top \]
			\item \textit{צעד. }נניח באינדוקציה את נכונות הטענה בעבור $n$ ונוכיח בעבור $n + 1$: 
			\[ \begin{WithArrows}
				\sum_{i = 0}^{n + 1}a_i &= a_{n + 1} + \sum_{i = 0}^{n}a_i \overset{\text{ה.א.}}{=} a_{n + 1} + \frac{(n + 1) \cdot (a_0 + a_n)}{2}
			\end{WithArrows} \]
			לשם ההמשך, נוכיח באינדוקציה למת עזר: לפיה, $a_n = a_0 + nd$. 
			\begin{itemize}
				\item \textit{בסיס. }$n = 0$ ואכן מתקיים $a_0 = a_0 + 0d$ מהיות הזהות יחס רפלקסיבי. 
				\item \textit{צעד. }נניח את נכונות הטענה בעבור $n$ ונוכיחה בעבור $n + 1$. אז מכלל הנסיגה: 
				\[ a_{n + 1} = a_n + d \overset{\text{ה.א.}}{=}a_0 + nd + d = a_0 + (n + 1)d \quad \top \]
			\end{itemize}
			נחזור להוכחה. נקבל את השוויון: 
			\begin{alignat*}{9}
				\sum_{i = 0}^{n + 1}a_i &= a_0 + (n + 1)d + \frac{(n + 1)(a_0 + a_n)}{2}
				= \frac{2a_0 + 2(n + 1)d + (n + 1)(a_0 + a_0nd)}{2} \\
				&= \frac{2a_0 + 2nd + 2d + na_0 + a_0n^2d + a_0 + a_0nd}{2}
				= \frac{na_0 + n(a_0 + (n + 1)d) + a_0 + a_0 + (n + 1)d}{2} \\
				&= \frac{na_0 + na_{n + 1} + a_0 + a_{n + 1}}{2} = \frac{(n + 1)(a_0 + a_{n + 1})}{2}
			\end{alignat*}
			כדרוש. 
		\end{itemize}
		בכך, השלמנו את בסיס וצעד האינדוקציה וטענת האינדוקציה הוכחה. 
	\end{proof}
	
	\section{}
	נמצא את הפתרונות המרוכבים של מהשוואות הבאות. בכל שאלה, יהי $z \in \C$, אזי קיימים $a := \Re z, \ b := \Im z$ כך ש־$z = a + bi$. 
	\begin{enumerate}[(A)]
		\item 
		\[ \begin{aligned}
			(1 + i)z &= 2 + i \\
			(1+ i)(a + bi) &= 2 + i \\
			(a - b) + (a + b)i &= 2 + i \\
		\end{aligned} \implies \begin{cases}
			a - b = 2 \\
			a + b = 1
		\end{cases} \unquad\!\!\implies \begin{cases}
			a = b + 2\\
			2b + 2 = 1
		\end{cases} \unquad\!\!\implies \begin{cases}
			a = \frac{1}{2}\\
			b = -\frac{1}{2}
		\end{cases} \]
		לכן $z = \bm{\frac{1}{2} - \frac{1}{2}i}$. 
		\item 
		\[ \begin{aligned}
			(\sqrt 3 + 2i)z &= (\sqrt 2 - 1)i \\
			(\sqrt 3 + 2i)(a + bi) &= (\sqrt 2 - 1)i + 0 \\
			(\sqrt 3 a - 2b) + (2a + \sqrt 3 b)i &= (\sqrt 2 - 1)i + 0
		\end{aligned} \implies \begin{cases}
			\sqrt 3 a - 2b = 0 \\
			2a + \sqrt 3 b = \sqrt 2 - 1
		\end{cases} \unquad\!\!\implies \begin{cases}
			b = \frac{\sqrt 3}{2}a \\
			(2 + \frac{3}{2})a = \sqrt 2 - 1
		\end{cases} \unquad\!\!\implies \begin{cases}
			b \approx 0.1025 \\
			a \approx 0.1183
		\end{cases} \]
		לכן $z \approx \bm{0.1183 + 0.1025i}$. 
		\item 
		\[ \begin{aligned}
			(3 - 2i)(5 + i)z &= 1 + 2i \\
			(17 - 7i)(a + bi) &= 1 + 2i \\
			(17a + 7b) + (-7a + 17b)i &= 1 + 2i
		\end{aligned} \implies \begin{cases}
			17a + 7b = 1 \\
			-7a + 17b = 2
		\end{cases} \unquad\!\!\implies \begin{cases}
			b = \frac{1 - 17a}{7} \\
			(-7 + \frac{-17^2}{7})a + \frac{17}{7} = 2
		\end{cases} \unquad\!\!\implies \begin{cases}
			b = \frac{41}{338} \\
			a = \frac{-\frac{3}{7} }{-7 -\frac{17^2}{7}} = \frac{3}{388}
		\end{cases}\]
		וסה"כ $z = \bm{\frac{3}{388} + \frac{41}{338}i}$. 
	\end{enumerate}
	\section{}
	נמצא את הייצוגים הפולאריים של המספרים הבאים: 
	\begin{enumerate}[(A)]
		\item נתבונן ב־$z := 1 + i$. אז $|z| = \sqrt{1^2 + 1^2} = \sqrt 2$. נמצא את הזווית: $\tg = \arctan\cl{\frac{1}{1}} = \frac{\pi}{4}$. סה"כ $z = |z|e^{\tg i} = \bm{\sqrt 2 e^{\frac{\pi}{4}i}}$. 
		\item נתבונן ב־$z = \frac{1}{\sqrt 2} - \frac{1}{\sqrt 2}i$. אז $|z| = \sqrt{(1/\sqrt{0.5})^{2} + (-1/\sqrt{0.5})^2} = 1$. וכן $\tg = \arctan\cl{\frac{1/\sqrt2}{-1\sqrt 2}} = - \frac{\pi}{4}$. וסה"כ $z = e^{-\frac{\pi}{4}i}$. 
		\item נתבונן ב־$z = -1 + 2i$. אז $|z| = \sqrt{(-1)^2 + 2^2} = \sqrt{5}$. אז $\tg = \pi + \atant\cl{-1, 2} \approx -\arctan(0.5) + \pi \approx -0.4636$. סה"כ $z \approx \sqrt 5 e^{(-0.4636 + \pi)i}$. 
	\end{enumerate}
	\section{}
	נמצא את הייצוג האלגברי של המספרים הבאים (ניעזר בנוסחאת אוילר): 
	\begin{enumerate}[(A)]
		\item 
		\[ 5e^{\frac{\pi}{3}i} = 5\cl{\cos\cl{\frac{\pi}{3}} + i\sin\cl{\frac{\pi}{3}}} = 5\cl{0.5 + \frac{\sqrt 3}{2}} = \bm{2.5 + 2.5\sqrt 3 i} \]
		\item 
		\[ e^{-\pi i}/ 2 = \frac{1}{2}\cl{\cos\cl{-\pi} + i \sin\cl{-\pi}} = \frac{1}{2}(-1 + 0i) = \bm{-\frac{1}{2}} \]
	\end{enumerate}
	\section{}
	יהי $(a_i)_{i = 0}^{n} \in \R$ אוסף של $n + 1$ סקלרים ממשיים. נגדיר לפיהם $p(z) = \sum_{k = 0}^{n}a_kz^k$ פולינום ממשי. נניח ש־$z_0 \in C$ שורש של $p$. צ.ל. $\bar z_0$ שורש. 
	\begin{proof}
		למען הנוחות, נסמן $z = z_0$ ונראה $p(\bar z) = 0$. נראה באינדוקציה על $k$ ש־$(\bar z)^{k} = \ol{z^k}$ לכל $k \in \N$. בסיס: ידוע $z^0 = 1 = (\bar z)^0$. צעד: נניח נכונות בעבור $k$, נוכיח בעבור $k + 1$. אז: 
		\[ \ol{(z^{k + 1})} = \ol{z \cdot z^k} \quad \overset{\mathclap{\text{שאלה 1א'}}}{=} \quad \bar z \cdot \ol{(z^k)} \overset{\text{ה.א.}}{=} \bar z \cdot (\bar z)^k = (\bar z)^{k + 1} \quad \top \]
		ובכך הלמה הוכחה. ראינו כי $\forall w, x \in \C \co \ol{w + x} = \bar w + \bar x$ (ישירות מדסטריבוטיביות). על כן: 
		\[ p(\bar z) = \sum_{k = 0}^{n}a_k(\bar z)^k = \sum_{k = 0}^{n}\ol {a_k z^k} = \ol{\sum_{k = 0}^{n}a_kz^k} = \ol{p(z)} = \bar 0 = 0 \]
		ומהגדרה $\bar z$ שורש של הפולינום, ולכן $\bar z_0$ שורש של הפולינום כדרוש. 
	\end{proof}
	
	\section{}
	יהיו $a_0, d \in \C$, ונגדיר את $(a_i)_{i = 0}^\inf$ לפי כלל הנסיגה $a_{n + 1} = a_n \cdot d$ לכל $n \in \N$. נראה כי $\forall n \in \N \co \sum_{i = 0}^{n}a_i = a_0 \cdot \frac{d^{n + 1} - 1}{d - 1}$. 
	\begin{proof}ראשית כל נוכיח את הזהות $a_i = a_0 d^{i}$ באינדוקציה על $i \in \N$. בסיס $i = 0$ כך ש־$a_0 = a_0 \cdot d^{0}$. בעבור הצעד נניח באינדוקציה בעבור $i$ ונוכיח על $i + 1$, ואכן מכלל הנסיגה ומה.א. $a_{i + 1} = a_i \cdot d = d \cdot a_0d^{i} = a_0d^{i + 1}$ כדרוש. 
		
		נוכיח את הטענה המקורית באינדוקציה על $n$. 
		\begin{itemize}
			\item \textit{בסיס: }אז $n = 0$, כלומר: 
			\[ \sum_{i = 0}^{0}a_i = a_0 = a_0 \cdot 1 = a_0 \cdot \frac{d^{0 + 1} - 1}{d - 1} \]
			\item \textit{צעד: }נניח באינדוקציה על $n$ ונוכיח על $n + 1$: 
			\begin{gather*}
				\sum_{i = 0}^{n + 1}a_i = a_{n + 1} + \sum_{i = 0}^{n}a_i = a_0 \cdot d^{n} + a_0\frac{d^{n} - 1}{d - 1} = a_0\frac{(d^{n})(d - 1) + d^{n} - 1}{d - 1} = a_0\frac{d^{n + 1} - d^{n} + d^{n} - 1}{d - 1} = a_0\frac{d^{n + 1} - 1}{d - 1}
			\end{gather*}
			כדרוש. 
		\end{itemize}
		סה"כ הראינו את נכונות הצעד והבסיס ולכן האינדוקציה הושלמה. 
	\end{proof}
	
	\ndoc
	
\end{document}