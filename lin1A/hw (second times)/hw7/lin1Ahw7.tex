%! ~~~ Packages Setup ~~~ 
\documentclass[]{article}
\usepackage{lipsum}
\usepackage{rotating}


% Math packages
\usepackage[usenames]{color}
\usepackage{forest}
\usepackage{ifxetex,ifluatex,amssymb,amsmath,mathrsfs,amsthm,witharrows,mathtools,mathdots}
\usepackage{amsmath}
\WithArrowsOptions{displaystyle}
\renewcommand{\qedsymbol}{$\blacksquare$} % end proofs with \blacksquare. Overwrites the defualts. 
\usepackage{cancel,bm}
\usepackage[thinc]{esdiff}


% tikz
\usepackage{tikz}
\usetikzlibrary{graphs}
\newcommand\sqw{1}
\newcommand\squ[4][1]{\fill[#4] (#2*\sqw,#3*\sqw) rectangle +(#1*\sqw,#1*\sqw);}


% code 
\usepackage{algorithm2e}
\usepackage{listings}
\usepackage{xcolor}

\definecolor{codegreen}{rgb}{0,0.35,0}
\definecolor{codegray}{rgb}{0.5,0.5,0.5}
\definecolor{codenumber}{rgb}{0.1,0.3,0.5}
\definecolor{codeblue}{rgb}{0,0,0.5}
\definecolor{codered}{rgb}{0.5,0.03,0.02}
\definecolor{codegray}{rgb}{0.96,0.96,0.96}

\lstdefinestyle{pythonstylesheet}{
    language=Java,
    emphstyle=\color{deepred},
    backgroundcolor=\color{codegray},
    keywordstyle=\color{deepblue}\bfseries\itshape,
    numberstyle=\scriptsize\color{codenumber},
    basicstyle=\ttfamily\footnotesize,
    commentstyle=\color{codegreen}\itshape,
    breakatwhitespace=false, 
    breaklines=true, 
    captionpos=b, 
    keepspaces=true, 
    numbers=left, 
    numbersep=5pt, 
    showspaces=false,                
    showstringspaces=false,
    showtabs=false, 
    tabsize=4, 
    morekeywords={as,assert,nonlocal,with,yield,self,True,False,None,AssertionError,ValueError,in,else},              % Add keywords here
    keywordstyle=\color{codeblue},
    emph={var, List, Iterable, Iterator},          % Custom highlighting
    emphstyle=\color{codered},
    stringstyle=\color{codegreen},
    showstringspaces=false,
    abovecaptionskip=0pt,belowcaptionskip =0pt,
    framextopmargin=-\topsep, 
}
\newcommand\pythonstyle{\lstset{pythonstylesheet}}
\newcommand\pyl[1]     {{\lstinline!#1!}}
\lstset{style=pythonstylesheet}

\usepackage[style=1,skipbelow=\topskip,skipabove=\topskip,framemethod=TikZ]{mdframed}
\definecolor{bggray}{rgb}{0.85, 0.85, 0.85}
\mdfsetup{leftmargin=0pt,rightmargin=0pt,innerleftmargin=15pt,backgroundcolor=codegray,middlelinewidth=0.5pt,skipabove=5pt,skipbelow=0pt,middlelinecolor=black,roundcorner=5}
\BeforeBeginEnvironment{lstlisting}{\begin{mdframed}\vspace{-0.4em}}
    \AfterEndEnvironment{lstlisting}{\vspace{-0.8em}\end{mdframed}}


% Deisgn
\usepackage[labelfont=bf]{caption}
\usepackage[margin=0.6in]{geometry}
\usepackage{multicol}
\usepackage[skip=4pt, indent=0pt]{parskip}
\usepackage[normalem]{ulem}
\forestset{default}
\renewcommand\labelitemi{$\bullet$}
\usepackage{titlesec}
\titleformat{\section}[block]
{\fontsize{15}{15}}
{\sen \dotfill (\thesection)\dotfill\she}
{0em}
{\MakeUppercase}
\usepackage{graphicx}
\graphicspath{ {./} }

\usepackage[colorlinks]{hyperref}
\definecolor{mgreen}{RGB}{25, 160, 50}
\definecolor{mblue}{RGB}{30, 60, 200}
\usepackage{hyperref}
\hypersetup{
    colorlinks=true,
    citecolor=mgreen,
    linkcolor=black,
    urlcolor=mblue,
    pdftitle={Document by Shahar Perets},
    %	pdfpagemode=FullScreen,
}


% Hebrew initialzing
\usepackage[bidi=basic]{babel}
\PassOptionsToPackage{no-math}{fontspec}
\babelprovide[main, import, Alph=letters]{hebrew}
\babelprovide[import]{english}
\babelfont[hebrew]{rm}{David CLM}
\babelfont[hebrew]{sf}{David CLM}
%\babelfont[english]{tt}{Monaspace Xenon}
\usepackage[shortlabels]{enumitem}
\newlist{hebenum}{enumerate}{1}

% Language Shortcuts
\newcommand\en[1] {\begin{otherlanguage}{english}#1\end{otherlanguage}}
\newcommand\he[1] {\she#1\sen}
\newcommand\sen   {\begin{otherlanguage}{english}}
    \newcommand\she   {\end{otherlanguage}}
\newcommand\del   {$ \!\! $}

\newcommand\npage {\vfil {\hfil \textbf{\textit{המשך בעמוד הבא}}} \hfil \vfil \pagebreak}
\newcommand\ndoc  {\dotfill \\ \vfil {\begin{center}
            {\textbf{\textit{שחר פרץ, 2025}} \\
                \scriptsize \textit{קומפל ב־}\en{\LaTeX}\,\textit{ ונוצר באמצעות תוכנה חופשית בלבד}}
    \end{center}} \vfil	}

\newcommand{\rn}[1]{
    \textup{\uppercase\expandafter{\romannumeral#1}}
}

\makeatletter
\newcommand{\skipitems}[1]{
    \addtocounter{\@enumctr}{#1}
}
\makeatother

%! ~~~ Math shortcuts ~~~

% Letters shortcuts
\newcommand\N     {\mathbb{N}}
\newcommand\Z     {\mathbb{Z}}
\newcommand\R     {\mathbb{R}}
\newcommand\Q     {\mathbb{Q}}
\newcommand\C     {\mathbb{C}}
\newcommand\One   {\mathit{1}}

\newcommand\ml    {\ell}
\newcommand\mj    {\jmath}
\newcommand\mi    {\imath}

\newcommand\powerset {\mathcal{P}}
\newcommand\ps    {\mathcal{P}}
\newcommand\pc    {\mathcal{P}}
\newcommand\ac    {\mathcal{A}}
\newcommand\bc    {\mathcal{B}}
\newcommand\cc    {\mathcal{C}}
\newcommand\dc    {\mathcal{D}}
\newcommand\ec    {\mathcal{E}}
\newcommand\fc    {\mathcal{F}}
\newcommand\nc    {\mathcal{N}}
\newcommand\vc    {\mathcal{V}} % Vance
\newcommand\sca   {\mathcal{S}} % \sc is already definded
\newcommand\rca   {\mathcal{R}} % \rc is already definded

\newcommand\prm   {\mathrm{p}}
\newcommand\arm   {\mathrm{a}} % x86
\newcommand\brm   {\mathrm{b}}
\newcommand\crm   {\mathrm{c}}
\newcommand\drm   {\mathrm{d}}
\newcommand\erm   {\mathrm{e}}
\newcommand\frm   {\mathrm{f}}
\newcommand\nrm   {\mathrm{n}}
\newcommand\vrm   {\mathrm{v}}
\newcommand\srm   {\mathrm{s}}
\newcommand\rrm   {\mathrm{r}}

\newcommand\Si    {\Sigma}

% Logic & sets shorcuts
\newcommand\siff  {\longleftrightarrow}
\newcommand\ssiff {\leftrightarrow}
\newcommand\so    {\longrightarrow}
\newcommand\sso   {\rightarrow}

\newcommand\epsi  {\epsilon}
\newcommand\vepsi {\varepsilon}
\newcommand\vphi  {\varphi}
\newcommand\Neven {\N_{\mathrm{even}}}
\newcommand\Nodd  {\N_{\mathrm{odd }}}
\newcommand\Zeven {\Z_{\mathrm{even}}}
\newcommand\Zodd  {\Z_{\mathrm{odd }}}
\newcommand\Np    {\N_+}

% Text Shortcuts
\newcommand\open  {\big(}
\newcommand\qopen {\quad\big(}
\newcommand\close {\big)}
\newcommand\also  {\mathrm{, }}
\newcommand\defis {\mathrm{ definitions}}
\newcommand\given {\mathrm{given }}
\newcommand\case  {\mathrm{if }}
\newcommand\syx   {\mathrm{ syntax}}
\newcommand\rle   {\mathrm{ rule}}
\newcommand\other {\mathrm{else}}
\newcommand\set   {\ell et \text{ }}
\newcommand\ans   {\mathscr{A}\!\mathit{nswer}}

% Set theory shortcuts
\newcommand\ra    {\rangle}
\newcommand\la    {\langle}

\newcommand\oto   {\leftarrow}

\newcommand\QED   {\quad\quad\mathscr{Q.E.D.}\;\;\blacksquare}
\newcommand\QEF   {\quad\quad\mathscr{Q.E.F.}}
\newcommand\eQED  {\mathscr{Q.E.D.}\;\;\blacksquare}
\newcommand\eQEF  {\mathscr{Q.E.F.}}
\newcommand\jQED  {\mathscr{Q.E.D.}}

\DeclareMathOperator\dom   {dom}
\DeclareMathOperator\Img   {Im}
\DeclareMathOperator\range {range}

\newcommand\trio  {\triangle}

\newcommand\rc    {\right\rceil}
\newcommand\lc    {\left\lceil}
\newcommand\rf    {\right\rfloor}
\newcommand\lf    {\left\lfloor}
\newcommand\ceil  [1] {\lc #1 \rc}
\newcommand\floor [1] {\lf #1 \rf}

\newcommand\lex   {<_{lex}}

\newcommand\az    {\aleph_0}
\newcommand\uaz   {^{\aleph_0}}
\newcommand\al    {\aleph}
\newcommand\ual   {^\aleph}
\newcommand\taz   {2^{\aleph_0}}
\newcommand\utaz  { ^{\left (2^{\aleph_0} \right )}}
\newcommand\tal   {2^{\aleph}}
\newcommand\utal  { ^{\left (2^{\aleph} \right )}}
\newcommand\ttaz  {2^{\left (2^{\aleph_0}\right )}}

\newcommand\n     {$n$־יה\ }

% Math A&B shortcuts
\newcommand\logn  {\log n}
\newcommand\logx  {\log x}
\newcommand\lnx   {\ln x}
\newcommand\cosx  {\cos x}
\newcommand\sinx  {\sin x}
\newcommand\sint  {\sin \theta}
\newcommand\tanx  {\tan x}
\newcommand\tant  {\tan \theta}
\newcommand\sex   {\sec x}
\newcommand\sect  {\sec^2}
\newcommand\cotx  {\cot x}
\newcommand\cscx  {\csc x}
\newcommand\sinhx {\sinh x}
\newcommand\coshx {\cosh x}
\newcommand\tanhx {\tanh x}

\newcommand\seq   {\overset{!}{=}}
\newcommand\slh   {\overset{LH}{=}}
\newcommand\sle   {\overset{!}{\le}}
\newcommand\sge   {\overset{!}{\ge}}
\newcommand\sll   {\overset{!}{<}}
\newcommand\sgg   {\overset{!}{>}}

\newcommand\h     {\hat}
\newcommand\ve    {\vec}
\newcommand\lv    {\overrightarrow}
\newcommand\ol    {\overline}

\newcommand\mlcm  {\mathrm{lcm}}

\DeclareMathOperator{\sech}   {sech}
\DeclareMathOperator{\csch}   {csch}
\DeclareMathOperator{\arcsec} {arcsec}
\DeclareMathOperator{\arccot} {arcCot}
\DeclareMathOperator{\arccsc} {arcCsc}
\DeclareMathOperator{\arccosh}{arccosh}
\DeclareMathOperator{\arcsinh}{arcsinh}
\DeclareMathOperator{\arctanh}{arctanh}
\DeclareMathOperator{\arcsech}{arcsech}
\DeclareMathOperator{\arccsch}{arccsch}
\DeclareMathOperator{\arccoth}{arccoth}
\DeclareMathOperator{\atant}  {atan2} 
\DeclareMathOperator{\Sp}     {span} 
\DeclareMathOperator{\sgn}    {sgn} 
\DeclareMathOperator{\row}    {Row} 
\DeclareMathOperator{\adj}    {adj} 
\DeclareMathOperator{\rk}     {rank} 
\DeclareMathOperator{\col}    {Col} 
\DeclareMathOperator{\tr}     {tr}

\newcommand\dx    {\,\mathrm{d}x}
\newcommand\dt    {\,\mathrm{d}t}
\newcommand\dtt   {\,\mathrm{d}\theta}
\newcommand\du    {\,\mathrm{d}u}
\newcommand\dv    {\,\mathrm{d}v}
\newcommand\df    {\mathrm{d}f}
\newcommand\dfdx  {\diff{f}{x}}
\newcommand\dit   {\limhz \frac{f(x + h) - f(x)}{h}}

\newcommand\nt[1] {\frac{#1}{#1}}

\newcommand\limz  {\lim_{x \to 0}}
\newcommand\limxz {\lim_{x \to x_0}}
\newcommand\limi  {\lim_{x \to \infty}}
\newcommand\limh  {\lim_{x \to 0}}
\newcommand\limni {\lim_{x \to - \infty}}
\newcommand\limpmi{\lim_{x \to \pm \infty}}

\newcommand\ta    {\theta}
\newcommand\ap    {\alpha}

\renewcommand\inf {\infty}
\newcommand  \ninf{-\inf}

% Combinatorics shortcuts
\newcommand\sumnk     {\sum_{k = 0}^{n}}
\newcommand\sumni     {\sum_{i = 0}^{n}}
\newcommand\sumnko    {\sum_{k = 1}^{n}}
\newcommand\sumnio    {\sum_{i = 1}^{n}}
\newcommand\sumai     {\sum_{i = 1}^{n} A_i}
\newcommand\nsum[2]   {\reflectbox{\displaystyle\sum_{\reflectbox{\scriptsize$#1$}}^{\reflectbox{\scriptsize$#2$}}}}

\newcommand\bink      {\binom{n}{k}}
\newcommand\setn      {\{a_i\}^{2n}_{i = 1}}
\newcommand\setc[1]   {\{a_i\}^{#1}_{i = 1}}

\newcommand\cupain    {\bigcup_{i = 1}^{n} A_i}
\newcommand\cupai[1]  {\bigcup_{i = 1}^{#1} A_i}
\newcommand\cupiiai   {\bigcup_{i \in I} A_i}
\newcommand\capain    {\bigcap_{i = 1}^{n} A_i}
\newcommand\capai[1]  {\bigcap_{i = 1}^{#1} A_i}
\newcommand\capiiai   {\bigcap_{i \in I} A_i}

\newcommand\xot       {x_{1, 2}}
\newcommand\ano       {a_{n - 1}}
\newcommand\ant       {a_{n - 2}}

% Linear Algebra
\DeclareMathOperator{\chr}     {char}
\DeclareMathOperator{\diag}    {diag}
\DeclareMathOperator{\Hom}     {Hom}
\DeclareMathOperator{\Sym}     {Sym}
\DeclareMathOperator{\Asym}    {ASym}

\newcommand\lra       {\leftrightarrow}
\newcommand\chrf      {\chr(\F)}
\newcommand\F         {\mathbb{F}}
\newcommand\co        {\colon}
\newcommand\tmat[2]   {\cl{\begin{matrix}
            #1
        \end{matrix}\, \middle\vert\, \begin{matrix}
            #2
\end{matrix}}}

\makeatletter
\newcommand\rrr[1]    {\xxrightarrow{1}{#1}}
\newcommand\rrt[2]    {\xxrightarrow{1}[#2]{#1}}
\newcommand\mat[2]    {M_{#1\times#2}}
\newcommand\gmat      {\mat{m}{n}(\F)}
\newcommand\tomat     {\, \dequad \longrightarrow}
\newcommand\pms[1]    {\begin{pmatrix}
        #1
\end{pmatrix}}

% someone's code from the internet: https://tex.stackexchange.com/questions/27545/custom-length-arrows-text-over-and-under
\makeatletter
\newlength\min@xx
\newcommand*\xxrightarrow[1]{\begingroup
    \settowidth\min@xx{$\m@th\scriptstyle#1$}
    \@xxrightarrow}
\newcommand*\@xxrightarrow[2][]{
    \sbox8{$\m@th\scriptstyle#1$}  % subscript
    \ifdim\wd8>\min@xx \min@xx=\wd8 \fi
    \sbox8{$\m@th\scriptstyle#2$} % superscript
    \ifdim\wd8>\min@xx \min@xx=\wd8 \fi
    \xrightarrow[{\mathmakebox[\min@xx]{\scriptstyle#1}}]
    {\mathmakebox[\min@xx]{\scriptstyle#2}}
    \endgroup}
\makeatother


% Greek Letters
\newcommand\ag        {\alpha}
\newcommand\bg        {\beta}
\newcommand\cg        {\gamma}
\newcommand\dg        {\delta}
\newcommand\eg        {\epsi}
\newcommand\zg        {\zeta}
\newcommand\hg        {\eta}
\newcommand\tg        {\theta}
\newcommand\ig        {\iota}
\newcommand\kg        {\keppa}
\renewcommand\lg      {\lambda}
\newcommand\og        {\omicron}
\newcommand\rg        {\rho}
\newcommand\sg        {\sigma}
\newcommand\yg        {\usilon}
\newcommand\wg        {\omega}

\newcommand\Ag        {\Alpha}
\newcommand\Bg        {\Beta}
\newcommand\Cg        {\Gamma}
\newcommand\Dg        {\Delta}
\newcommand\Eg        {\Epsi}
\newcommand\Zg        {\Zeta}
\newcommand\Hg        {\Eta}
\newcommand\Tg        {\Theta}
\newcommand\Ig        {\Iota}
\newcommand\Kg        {\Keppa}
\newcommand\Lg        {\Lambda}
\newcommand\Og        {\Omicron}
\newcommand\Rg        {\Rho}
\newcommand\Sg        {\Sigma}
\newcommand\Yg        {\Usilon}
\newcommand\Wg        {\Omega}

% Other shortcuts
\newcommand\tl    {\tilde}
\newcommand\op    {^{-1}}

\newcommand\sof[1]    {\left | #1 \right |}
\newcommand\cl [1]    {\left ( #1 \right )}
\newcommand\csb[1]    {\left [ #1 \right ]}
\newcommand\ccb[1]    {\left \{ #1 \right \}}

\newcommand\bs        {\blacksquare}
\newcommand\dequad    {\!\!\!\!\!\!}
\newcommand\dequadd   {\dequad\duquad}

\renewcommand\phi     {\varphi}

\newtheorem{Theorem}{משפט}
\theoremstyle{definition}
\newtheorem{definition}{הגדרה}
\newtheorem{Lemma}{למה}
\newtheorem{Remark}{הערה}
\newtheorem{Notion}{סימון}

\newcommand\theo  [1] {\begin{Theorem}#1\end{Theorem}}
\newcommand\defi  [1] {\begin{definition}#1\end{definition}}
\newcommand\rmark [1] {\begin{Remark}#1\end{Remark}}
\newcommand\lem   [1] {\begin{Lemma}#1\end{Lemma}}
\newcommand\noti  [1] {\begin{Notion}#1\end{Notion}}

% DS
\newcommand\limsi     {\limsup_{n \to \inf}}
\newcommand\limfi     {\liminf_{n \to \inf}}

\DeclareMathOperator\amort   {amort}
\DeclareMathOperator\worst   {worst}
\DeclareMathOperator\type    {type}
\DeclareMathOperator\cost    {cost}
\DeclareMathOperator\tim     {time}

\newcommand\dsList{
    \sFunc{List}
    \sFunc{Retrieve}
    \SetKwFunction{RetrieveFirst}{Retrieve-First}
    \SetKwFunction{RetrieveLast}{Retrieve-Last}
    \sFunc{Delete}
    \SetKwFunction{DeleteFirst}{Delete-First}
    \SetKwFunction{DeleteLast}{Delete-Last}
    \sFunc{Insert}
    \SetKwFunction{InsertFirst}{Insert-First}
    \SetKwFunction{InsertLast}{Insert-Last}
    \sFunc{Shift}
    \sFunc{Length}
    \sFunc{Concat}
    \sFunc{Plant}
    \sFunc{Split}
}
\newcommand\dsQueue{
    \sFunc{Queue}
    \sFunc{Enqueue}
    \sFunc{Head}
    \sFunc{Dequeue}
}
\newcommand\dsStack{
    \sFunc{Stack}
    \sFunc{Push}
    \sFunc{Top}
    \sFunc{Pop}
}
\newcommand\dsVector{
    \sFunc{Vector}
    \sFunc{Get}
    \sFunc{Set}
}
\newcommand\dsGraph{
    \sFunc{Graph}
    \sFunc{Edge}
    \SetKwFunction{AddEdge}{Add-Edge}
    \SetKwFunction{RemoveEdge}{Remove-Edge}
    \sFunc{InDeg} \sFunc{OutDeg}
}
\newcommand\importDs{
    \dsList
    \dsQueue
    \dsStack
    \dsVector
    \dsGraph
    \SetKwData{error}{\color{codered}error}
    \SetKwInOut{Input}{input}
    \SetKwInOut{Output}{output}
    \SetKwRepeat{Do}{do}{while}
    \SetKwData{Null}{\color{codeblue}null}
}


% Algorithems
\newcommand\sFunc [1] {\SetKwFunction{#1}{#1}}
\newcommand\sData [1] {\SetKwData{#1}{#1}}
\newcommand\sIO   [1] {\SetKwInOut{#1}{#1}}
\newcommand\ttt   [1] {\sen \texttt{#1} \she\,}
\newcommand\io    [2] {\Input{#1}\Output{#2}\BlankLine}

%! ~~~ Document ~~~

\author{שחר פרץ}
\title{\textit{לינארית 1א $\sim$ תרגיל בית 7 $\sim$ סמסטר ב' 2025}}
\begin{document}
    \maketitle
    \section{}
    יהיו $A, B \in M_n(\F)$. נוכיח $\rk(A + B + AB) \le \rk A + \rk B$. 
    \begin{proof}
        יהיו $A, B \in M_n(F)$. נוכיח שתי למות. 
        \begin{itemize}
            \item \textbf{למה 1: $\bm{\rk (A + B) \le \rk A + \rk B}$. }נוכיח את הלמה. נבחין ש־$v \in \col(A + B)$ אמ''מ $\exists x \co (A + B)x = v$ כלומר $Ax + Bx = v$, ובפרט $v \in \col A + \col B$. לכן $\col(A + B) \subseteq \col A + \col B$. ממשפט הממדים: 
            \begin{align*}
                \rk (A + B) &= \dim \col (A + B) \le \dim (\col A + \col B) \\
                &= \dim \col A + \dim\col B  - \dim\col (A \cap B) \\
                &\le \dim \col A + \dim \col B \\
                &= \rk A + \rk B
            \end{align*}
            כדרוש. 
            \item \textbf{למה 2: $\bm{\rk(A B) \le \min(\rk A, \rk B)}$.} לשם כך נוכיח קודם כל $\rk(AB) \le \rk A$. ידוע $\rk AB = \dim \col (AB)$, ולכן קיים $B$ בסיס מגודל $\dim \col(AB)$ כך ש־$\Sp B = \col (AB)$. נסמן $B = (v_i)_{i = 1}^{k}$, אזי $(AB v_i)_{i = 1}^{k}$ בת''ל באופן שקול. נניח בשלילה $(A v_i)_{i = 1}^{k}$ איננו בת''ל, אזי קיימת קומבינציה לינארית $(\lg_i)_{i =1}^{k}$ כך ש־$\sum \ag_i A v_i = 0$. נכפיל ב־$B$ ונקבל מדיסטרבוטיביות $\sum \ag_i AB v_i = 0 = 0B$, כלומר $(ABv_i)_{i= 1}^{k}$ ת''ל וזו סתירה. סה''כ $(Av_i)_{i = 1}^{k}$ בת''ל, ומהגדרת כפל מטריצה בוקטור זהו בסיס ל־$\col A$ ולכן $\dim \col A \ge \dim \col (AB)$. סה''כ מהגדרת $\rk$ נסיק $\rk A \ge \rk (AB)$. מהמשפט שעתה הוכחנו: 
            \[ \rk(AB) = \rk((AB)^{T}) = \rk(B^TA^T) \le \rk(B^T) = \rk B \]
            ולכן $\rk(AB) \le \rk B \land \rk (AB) \le \rk A$, כלומר $\rk(AB) \le \min(\rk A, \rk B)$ כדרוש. 
        \end{itemize}
        למען האמת אני לא בטוח שאני צריך את שתי הלמות אבל כבר הוכחתי אותן. נבחין ש־: 
        \[ B =: \pms{\vert & \cdots & \vert \\ v_1 & \cdots & v_n \\ \vert & \cdots & \vert}, \ AB = \pms{\vert & \cdots & \vert \\ Av_1 & \cdots & Av_n \\ \vert & \cdots & \vert} \implies AB + A = \pms{\vert & \cdots & \vert \\ A(v_1 + 1) & \cdots & A(v_n + 1) \\ \vert & \cdots & \vert} \]
        משום ש־$\forall v \co Av \in \col A$ אז $\col(AB + A) = \col(A)$. 
        אזי: 
        \[ \rk(A + B + AB) = \dim \col(A + B + AB)) \le \dim \col(A + AB) + \dim \col B = \dim \col A + \dim \col B = \rk A + \rk B \]
        (בסוף השתמשתי רק בלמה 1) כדרוש. 
    \end{proof}
    \section{}
    נמצא בסיס למרחב השורות ולמרחב העמודות של המטריצות הנתונת מעל הממשיים: 
    \begin{enumerate}
        \item 
        \[ \pms{1 & 1 & 2 \\ 0 & 1 & 2} \]
        \begin{itemize}
            \item \textbf{מרחב שורות. }נרצה למצוא בסיס: 
            \[ \Sp\ccb{\pms{2 \\ 1 \\ 1}, \pms{2 \\ 1 \\ 0}} = \row\pms{1 & 1 & 2 \\ 0 & 1 & 2} \rrr{R_1 \to R_1 - R_2} \row\pms{1 & 0 & 0 \\ 0 & 1 & 2} = \Sp\ccb{\pms{0 \\ 0 \\ 1}, \pms{2 \\ 1 \\ 0}} \]                סה''כ מצאנו בסיס $\{(0, 1, 2), (0, 0, 1)\}$. 
                \item \textbf{מרחב העמודות. }נבחין כי מרחב העמודות חסום בגודל $2$, וכן שהוקטורים $\ccb{\binom{1}{0}, \binom{1}{1}}$ בת''ל, הם בסיס ל־$\Sp\ccb{\binom{1}{0}, \binom{1}{1}} = \Sp \ccb{\binom{1}{0}, \binom{1}{1}, \binom{2}{2}} = \col \binom{1\,1\,2}{0\,1\,2}$ (כי $\binom{1}{1}$ ו־$\binom{2}{2}$ ת''ל עבור פקטור של 2), כדרוש. 
        \end{itemize}
        \item 
        \[ \pms{1 & 1 & 3 \\ 1 & -2 & 0 \\ 1 & -1 & \lg} \]
        בזמן דירוג מרחב השורות לא ישתנה. 
        \begin{itemize}
            \item \textbf{מרחב השורות: }
            \[ \pms{1 & 1 & 3 \\ 1 & -2 & 0 \\ 1 & -1 & \lg} \rrt{R_2 \to R_2 - R_1}{R_3 \to R_3 - R_1}
            \pms{1 & 1 & 3 \\ 0 & -3 & -3 \\ 0 & -2 & \lg - 3}
            \rrr{R_2 \to \frac{R_2}{-3}}
            \pms{1 & 1 & 3 \\ 0 & 1 & 1 \\ 0 & -2 & \lg - 3}
            \rrr{R_3 \to R_3 + 2R_2}
            \pms{1 & 1 & 3 \\ 0 & 1 & 1 \\ 0 & 0 & \lg - 1}
             \]
             נפריד למקרים. 
             \begin{itemize}
                 \item אם $\lg - 1 = 0$, אז זוהי צורה מדורגת ולכן: 
                 \[ \ccb{\pms{1 \\ 0 \\ 0}, \pms{1 \\ 1 \\ 0}, \pms{3 \\ 1 \\ \lg - 1}} \]
                 \item אחרת, $\lg - 1 = 0$, כלומר: 
                 \[ \pms{ 1 & 1 & 3 \\ 0 & 1 & 1 \\ 0 & 0 & 0} \rrr{R_1 \to R_1 - R_3} \pms{1 & 0 & 2 \\ 0 & 1 & 1 \\ 0 & 0 & 0} \]
                 בגלל ש־$\binom{2}{1}$ תלוי לינארית ב־$\binom{1}{0}, \binom{0}{1}$, אז: 
                 \[ \ccb{\pms{0 \\ 1 \\ 0}, \pms{1 \\0 \\ 0}} \]
                 פורש את המרחב. 
             \end{itemize}
             \item \textbf{מרחב העמודות: }באופן דומה, נדרג את הטרנספוז: 
             \[ \pms{1 & 1 & 1 \\ 1 & -2 & -1 \\ 3 & 0 & \lg} \rrt{R_2 \to R_2 - R_1}{R_3 \to R_3 - 3R_1}
             \pms{1 & 1 & 1 \\  0 & -3 & -2  \\ 0 & -3 & \lg - 3}
             \rrr{R_2 \to \frac{R_2}{-3}}
             \pms{1 & 1 & 1 \\  0 & 1 & \frac{2}{3}  \\ 0 & -3 & \lg - 3}
             \rrr{R_3 \to R_3 + 3R_2}
             \pms{1 & 1 & 1 \\  0 & 1 & \frac{2}{3}  \\ 0 & 0 & \lg - 1} \]
             כמו פעם קודמת, נפריד למקרים. 
             \begin{itemize}
                 \item אם $\lg - 1 = 0$, אז המטריצה מדורגת ולכן הקבוצה הבאה פורשת את מרחב השורות: 
                 \[ \ccb{\pms{1 \\ 0 \\ 0}, \pms{1 \\ 1 \\ 0}, \pms{1 \\ \frac{2}{3} \\ \lg - 1}} \]
                 \item אחרת, הקבוצה לעיל תלויה לינארית, ולכן ממד מרחב השורות הוא $2$. בגלל ששני הוקטורים הבאים נמצאים בו, והם בת''לים: 
                 \[ \ccb{\ccb{1 \\ 0 \\ 0},\ \pms{1 \\ 1 \\ 0}} \]
                 הם פורשים את המרחב, כדרוש. 
             \end{itemize}
        \end{itemize}
    \end{enumerate}
    \npage
    \section{}
        נניח ש־$v_1 \dots v_n \in \F^n$ וקטורי עמודה בת''ל. נוכיח $A := \sum_{i = 1}^{n}v_i \cdot v_i^{T}$ מקיימת $\rk A = n$. 
    \begin{proof}
        תהי $(v_1 \dots v_k)$ בת"ל. נמצא את דרגת $A = v_1v_1^T + \cdots + v_kv_k^TA$. נסמן ב־$B_i$ להיות השורה ה־$i$ במטריצה כלשהי $B$. 
        
        \textbf{למה 1. }\textit{$\forall i \in [n]\,\forall k \in [n] \, \exists a_{ik} \co (v_iv_i^T)_k = a_i v_i $}. נוכיחה: 
        \[ (v_iv_i^T)_{ik} = \sum_{j = 1}^{n}v_{ik}v_{ji} \implies 
        (v_iv_i^T)_k = (v_k v_i)_{i = 1}^n = v_i \cdot \underbrace{(v_i)_k}_{:= a_{ik}} = v_i a_{ik} \]
        ניעזר בסימון $\ag_i = \sum_{k = 1}^{n}\ag_{ki}$. נבחין 
        
        נניח בשלילה ששורות $A$ ת''ל. אזי קיימת קומבינציה לינארית $\lg_1 \dots \lg_n$ כך ש־: 
        \[ 0 = \sum_{i = 1}^{n}\lg_i A_i = \sum_{i = 1}^{n}\cl{\lg_i \cl{\sum_{k = 1}^{n}v_iv_i^T}_i} = \sum_{i = 1}^{n}\cl{\lg_i \cl{\sum_{k = 1}^{n}v_ia_{ki}}_i} = \sum_{i = 1}^{n}\cl{\lg_i v_i \sum_{k = 1}^{n}\ag_{ki}} = \sum_{i = 1}^{n}\underbrace{\lg_i\ag_i}_{:= m_i} v_i = \sum_{i = 1}^{n}m_iv_i  \]
        זוהי קומבינציה לינארית לא טרוויאלית של וקטורים בת''לים ולכן אינה שווה ל־$0$, וסתירה כדרוש. 
    \end{proof}
    
    \section{}
    נסמן: 
    \[ A = \pms{m & 0 & 1 \\ 1 & m - 1 & 0 \\ m & 0 & 2} \in M_3(\R) \]
    נמצא עבור אילו ערכים $v := (1, 1, m)\in \col A$. נבחין ש־$\exists x \co Ax = v$ אמ''מ $v \in \col A$. לכן, יהיה $x \in \R^3$, נמצא מתי $Ax = v$: 
    \begin{multline*}
        Ax = v \rightarrow \tmat{m & 0 & 1 \\ 1 & m - 1 & 0 \\ m & 0 & 2}{1 \\ 1 \\ m} \rrr{R_1 \lra R_2}
        \tmat{1 & m - 1 & 0 \\ m & 0 & 1 \\ m & 0 & 2}{1 \\ 1 \\ m} \rrt{R_2 \to R_2 - mR_1}{R_3 \to R_3 - mR_1}
        \tmat{1 & m - 1 & 0 \\ 0 & -m^2 + m & 1 \\ 0 & -m^2 + m & 2}{1 \\ 1 - m \\ 0} \\
        \rrr{R_3 \to R_3 - R_2} 
        \tmat{1 & m - 1 & 0 \\ 0 & -m^2 + m & 1 \\ 0 & 0 & 1}{1 \\ 1 - m \\ m - 1}
        \rrr{R_2 \to \frac{R_2}{-m^2 + m}}
        \tmat{1 & m - 1 & 0 \\ 0 & 1 & \frac{1}{-m^2 + m} \\ 0 & 0 & 1}{1 \\ \frac{1 - m}{-m^2 + m} \\ m - 1}
    \end{multline*}
    וזוהי מטריצה מדורגת, כלומר קיימת צורה מדורגת קאנונית כך שמצאנו ערכים מתאימים ל־$x$. אך, הנחנו הנחות כדי להגיע לכך. נראה מה קורה אם הנחות אילו לא מתקיימות: 
    \begin{itemize}
        \item אם $m = 0$, אז: 
        \[ \tmat{1 & -1 & 0 \\ 0 & 0 & 1 \\ 0 & 0 & 2}{1 \\ 1 \\ 0} \]
        כלומר $\exists \ag \co 1 \ag = 1 \land 2 \ag = 0$. מהמשוואה הראשונה $\ag = 1$ ולכן $2 = 2\ag = 0$. נחלק ב־$2$ ונקבל $0 = 1$ וזו סתירה. 
        \item אם $-m^2 + m \neq 0$, או באופן שקול $m = 0 \lor m = 1$ (כבר הנחנו $m = 0$, לכן נבדוק מקרה בו $m = 1$): 
        \[ \tmat{1 & 0 & 0 \\ 1 & 0 & 1 \\ 1 & 0 & 2}{1 \\ 1 \\ 1} \rrt{R_2 \to R_2 - R_1}{R_3 \to R_3 - R_1} \tmat{1 & 0 & 0 \\ 0 & 0 & 1 \\ 0 & 0 & 2}{1 \\ 0 \\ 0} \]
        מערכת משוואות עם הפתרון $x = (1, \ag, 0)$.
    \end{itemize}
    סה''כ מצאנו ש־$v \in \col A$ אמ''מ $m \neq 0$. 
    
    \npage
    \section{}
    בכל סעיף נקבע האם ההעתקה המתתוארת היא לינארית. 
    \begin{enumerate}[(1)]
        \item נוכיח ש־$T \co \R^2 \to \R^2$ המתוארת ע''י $T\binom{x}{y} = \binom{x}{|y|}$ איננה לינארית. נוכיח סתירה להומוגניות חיבור. אם היא הייתה לינארית, אז: 
        \[ \pms{2 \\ 2} = \pms{1 \\ |1|} + \pms{1 \\ |-1|} = T\pms{1 \\ -1} + T\pms{1 \\ 1} = T\pms{1 + 1 \\-1 + 1} = T\pms{2 \\ 0} = \pms{2 \\ \sof{0}} = \pms{2 \\ 0} \]
        מהמשפט היסודי של זוג סדור $2 = 0$, נחלק ב־$2$ ונקבל $1 = 0$ וזו סתירה. 
        \item נוכיח שההעתקה הבאה לינארית: 
        \[ T \co \R^n \to \R^n, \ T\pms{x_1 \\ \vdots \\ x_{n - 1} \\ x_n} = \pms{x_1 + x_2 \\ \vdots \\ x_{n - 1} + x_n \\ x_n + x_1} \]
        נוכיח זאת באמצעות כך שנראה שהיא מיוצגת ע''י מטריצה. נסמן ב־$E$ את הבסיס הסטנדרטי ל־$\R^n$. אזי: 
        \[ A := \pms{1 & 1 & 0 & \cdots & 0 \\ 0 & 1 & 1 & \ddots & 0 \\ \vdots & \ddots & \ddots & \ddots & \vdots \\ 0 & \cdots & 0 & 1 & 1 \\ 1 & 0 & \cdots & 0 & 1} \]
        מקיימת מהגדרת כפל מטריצה בוקטור ש־$A = [T]^E_E$ (כלומר, שלכל $x$ מתקיים $Ax = T(x)$, וכך נדע ש־$T$ לינארית – כי חיבור וכפל מטריצות הוא לינארי). נראה שהמטריצה הפיכה: 
        \[ \det A \rrt{\forall i \in \{n \dots 2\}}{R_{i -1} \to R_{i - 1} - R_i} \det \pms{2 & 0 & 0 & \cdots & 0 \\ -1 & 1 & 0 & \ddots & 0 \\ \vdots & \vdots & \ddots & \ddots & \vdots \\ -1 & 0 & \cdots & 1 & 0 \\ 1 & 0 & \cdots & 0 & 1} = 2(-1)^{1 + 1}\det \pms{1 & 0 & \cdots & 0 \\ 0 & 1& \ddots & \vdots \\ \vdots & \ddots & \ddots & 0 \\ 0 & \cdots & 0 & 1} = 2 \cdot \det I = 2 \]
        סה''כ $\det A \neq 0$ ולכן המטריצה הפיכה, כלומר $\rk A = n$. קל לראות ש־$\rk A = \dim \col A = \dim \col \Img T$ כי $\col A = \Im T$, ולכן ממשפט הממדים להעתקות לינארית:
        \[ \dim \Img T + \dim \ker T = n \implies n + \dim \ker T = n \implies \dim \ker T = 0 \]
        וסה''כ $\ker T = \{0\}$ מרחב האפס, ו־$\Img T = V = \R^n$ המרחב. 
        \item נוכיח שההעתקה $T \co \R_{\le n}[x] \to \R_{\le n}[x]$ המוגדרת ע''פ $T = p(x) + x^2 - x$ איננה העתקה לינארית. נסתור הומוגניות לכפל בסקלר, בעבור $n = 3$. נניח בשלילה שהיא לינארית: 
        \[ 2x^3 + x^2 - x = T(2x^3) = T(2 \cdot x^3) = 2 T(x^3) = 2(x^3 + x^2 - x) = 2x^2 + 2x^2 - 2x \]
        נחסר אגפים ונקבל $x^2 - x = 0$, אך זהו איננו פולינום האפס וסתירה. 
        \item נוכיח שההעתקה $T \co A + A^T, \ T \co M_n(\R) \to M_n(\R)$ לינארית: 
        \begin{itemize}
            \item \textbf{הומוגניות כפל בסקלר. }$\forall \lg \in \R\, \forall A \in M_n(\R)$, מתקיים: \hfill $\lg A + \lg A^T = T(\lg A) = \lg T(A) = \lg(A + A^T) = \lg A + \lg A^T$ 
            \item \textbf{הומגניות חיבור. }$\forall A, B \in M_n(\R)$, מתקיים: \hfill $A + A^T + B + B^T = A + B + A^T + B^T = (A + B) + (A + B)^T = T(A + B) = T(A) + T(B) = A + A^T + B + B^T$
        \end{itemize}
        עתה נראה ש־$\ker T = \Asym_n(\R)$ ו־$\Img T = \Sym_n(\R)$. לכל $M \in \Asym(\R)$ מתקיים קיום $A \in M_n(\R)$ כך ש־$M$ החלק האנטי־סימטרי של $A$, כלומר: 
        \[ T(M) = T\cl{\frac{A - A^T}{2}} = \frac{A - A^T}{2} + \cl{\frac{A - A^T}{2}}^T = \frac{A - A^T - A + A^T}{2} = 0 \]
        כלומר $\Asym_n(\R) \subseteq \ker T$. עתה נראה $\Sym_n(\R) \subseteq \Img T$. לכל $M \in M_n(\R)$, מתקיים: 
        \[ T\cl{\frac{M}{2}} = \frac{M}{2} + \cl{\frac{M}{2}}^T = \frac{M + M^2}{2} \in \Sym_n(\R) \]
        נניח בשלילה $\dim \ker T \neq \frac{n^2 - n}{2} \land \dim \Img T \neq \frac{n^2 + n}{2}$: ידוע $\Asym_n(\R) \subseteq \ker T$ ולכן $\frac{n^2 - n}{2}  = \dim \Asym_n(\R) < \dim \ker T$ (אי־שוויון חזק מהנחת השלילה), ובאופן דומה $\frac{n^2 + n}{2} < \Img T$: 
        \[ n = \dim \R^n = \dim \ker T + \dim \Img T < \frac{n^2 - n}{2} + \frac{n^2 + n}{2} = n \implies n < n \implies 0 < 0 \]
        וזו סתירה. לכן: 
        \begin{itemize}
            \item אם $\dim \ker T = \frac{n^2 - n}{2}$, אז $\dim \Img T + \dim \ker T = n \implies \dim \Img T = n - \frac{n^2 - n}{2} = \frac{n^2 + n}{2} = \dim \Sym_n(\R)$. 
            \item אם $\dim \Img T = \frac{n^2 + n}{2}$, אז $\dim \ker T + \dim \Img T = n \implies \dim \ker T = n - \frac{n^2 + n}{2} = \frac{n^2 - n}{2} = \dim \Asym_n(\R)$. 
        \end{itemize}
        לכן סה''כ: 
        \begin{alignat*}{9}
            \dim \ker T &= \dim \Asym_n(\R) &&\land \Asym_n(\R) \subseteq \ker T &&\implies \bm{\ker T = \Asym_n(\R)} \\
            \dim \Img T &= \dim \Sym_n(\R) && \land \Sym_n(\R) \subseteq \Img T &&\implies \bm{\Img T = \Sym_n(\R)}
        \end{alignat*}
        כדרוש. 
    \end{enumerate}
    
    \section{}
    יהיו $U, V$ מ''וים מעל $\F$, ותהי $T \co V \to U$ לינארית. נוכיח מספר טענות. 
    \begin{itemize}
        \item אם $(v_i)_{i = 1}^{n} \subset V$ קבוצת וקטורים ו־$(T(v_i))_{i = 1}^{n}$ בת''ל, נראה $(v_i)_{i = 1}^{n}$ בת''ל. \begin{proof}
            יהיו וקטורים והעתקה כמתואר לעיל. נניח בשלילה $(v_i)_{i = 1}^{n}$ ת''ל, אזי קיימת קומבינציה לינארית $(\lg_i)_{i = 1}^{n}$ כך ש־$\sum \lg_i v_i = 0$ לא טרוויאלית. לכן, (חוקי משום ששני המרחבים מעל אותו השדה, אז הכפל בסקלר מוגדר היטב) מלינאריות: 
            \[ \sum_{i = 1}^{n}\lg_iT(v_i) = T\cl{\sum_{i = 1}^{n}\lg_i v_i} = T(0) = 0 \]
            ובכך מצאנו קומבינציה לינארית לא טרוויאלית של $(T(v_i))_{i = 1}^{n}$ כך ש־$\sumni \lg_i v_i = 0$, לכן $(T(v_i))_{i = 1}^{n}$ ת''ל, אך נתון שהיא בת''ל וסתירה. לכן $(v_i)_{i = 1}^{n}$ בת''ל כדרוש. 
        \end{proof}
        \item נוכיח שאם $v_1 \dots v_n$ בת''ל ו־$T$ חח''ע, אז $Tv_1 \dots Tv_n$ בת''ל. \begin{proof}
            נניח בשלילה $Tv_1 \dots Tv_n$ ת''ל. אזי קיימת קומבינציה לינארית $\lg_1 \dots \lg_n$ כך ש־$\sumnio \lg_i T(v_i) = 0$. לכן, מלינאריות: 
            \[ \ker T = \{0\} \land \sumnio \lg_i Tv_i = 0 \implies T\cl{\sumnio \lg_i v_i} = 0 \implies \sumnio \lg_i v_i = 0 \]
            כאשר $\ker T = \{0\}$ כי $T$ חח''ע. משמע, סה''כ מצאנו קומבינציה לינארית $\lg_1 \dots \lg_n$ ל־$v_1 \dots v_n$. דהיינו $v_1 \dots v_n$ ת''ל וכן נתון להיות בת''ל כלומר סתירה, ו־$Tv_1 \dots Tv_n$ בת''ל כדרוש. 
        \end{proof}
    \end{itemize}
    
    \section{}
    תהי $A \in \mat{m}{n}(\F)$ ונסמן $\rk A = r$. נוכיח את הטענות הבאות: 
    \begin{itemize}
        \item נוכיח קיום מטריצות $A_1 \dots A_r \in \mat{m}{n}(\R)$ כך ש־$\rk A_i = 1$ לכל $i$ ומתקיים $A = \sum_{i = 1}^{r}A_i$. \begin{proof}
            ידוע $\dim \row A = r$, ולכן קיימת קבוצה של $r$ שורות כך ש־$\tl A_1 \dots \tl A_r$ יוצרות את $\row A$ (כאשר $A_i$ מטריצה בעלת שורה של $A$ כלשהי, והשאר אפסים, ותקרא ``\textit{מטריצה שורה}''), ונבחין כי $\tl A_1 \dots \tl A_r$ בסיס ל־$\row A$ כך ש־$\rk A_i = 1$ (כי ישנה רק שורה אחת בכל מטריצה שאיננה אפסים). בפרט, כל מטריצת שורה אחרת כלשהי $\tl A_{i > r}$ תקיים $\tl A_{i > r} \in \row A$ ולכן נוצרת ע''י $\tl A_1 \dots \tl A_r$, מהקומבינציה הלינארית $\lg_{i1} \dots \lg_{in}$. נסמן $\forall i \in [r] \co A_i = (\sumnko \lg_{ik})\tl A_i$. אזי: 
            \[ \sum_{i = 1}^r A_i = \sum_{i = 1}^{r}\cl{\tl A_i \sumnko\lg_{ik}} = \sumnko\underbrace{\sum_{i = 1}^{r}\lg_{ik}\tl A_i}_{\tl A_i} = \sumnko \tl A_i = A \]
            (כאשר $\sumnko \tl A_i = A$ כי אלו כל $n$ שורות $A$)וכן בכלל ש־$\rk \tl A_i = 1$ ו־$A_i$ תלויה לינארית בעבור הקבוע $\sumnko \lg_{ik}$ ב־־$\tl A_i$, אז $\rk A_i = 1$. סה''כ מצאנו רצף של $A_1 \dots A_r$ מטריצות כך ש־$\sum_{i = 1}^{r} A_i = A$ וכן $\rk A_i = 1$ כדרוש. 
        \end{proof}
        \item נניח $k < r$ ונוכיח שלא קיימות מטריצות $A_1 \dots A_k \in M_{m \times n}(\R)$ כך ש־$\rk A_i = 1 \land \sum_{i = 1}^{k}A_i = A$. \begin{proof}
            נניח בשלילה שקיימות מטריצות כאלו. עבור מטריצה כלשהי $A_i$ משום ש־$\rk A_i = 1$, קיים וקטור $v_i \in \R^m$ כך ש־$\row A_i = \Sp v_i$. ובפרט, כל וקטור שורה $(A_i)_j = \lg_{ij}v_i$ השורה ה־$j$ ב־$A_i$ שכן $(A_i)_j \in \row A_i$. 
            
            עתה נראה $\row A \subseteq \Sp\{v_i \dots v_k\}$: לכל $w \in \row A$, מתקיים:
            \[ w \in \Sp \ccb{{(A_{i})_j} \mid i, j \in [k] \times [n]} = \Sp\{\lg_{vi} v_i \mid i, j \in [k] \times [n]\} = \Sp\{v_i \mid i \in [k]\} \]
            כדרוש. מכלה $\Sp\{v_i \mid i \in [k]\} \subseteq \row A$ ולכן $k \le \dim\row A$, אך $k \le r \land r < k$ סתירה כדרוש. סה''כ לא קיימות מטריצות כמתוארות לעיל כדרוש. 
        \end{proof}
    \end{itemize}
    
    
    
    
    
    \ndoc
\end{document}