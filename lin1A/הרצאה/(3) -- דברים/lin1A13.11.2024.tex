%! ~~~ Packages Setup ~~~ 
\documentclass[]{article}


% Math packages
\usepackage[usenames]{color}
\usepackage{forest}
\usepackage{ifxetex,ifluatex,amsmath,amssymb,mathrsfs,amsthm,witharrows,mathtools}
\WithArrowsOptions{displaystyle}
\renewcommand{\qedsymbol}{$\blacksquare$} % end proofs with \blacksquare. Overwrites the defualts. 
\usepackage{cancel,bm}
\usepackage[thinc]{esdiff}


% tikz
\usepackage{tikz}
\newcommand\sqw{1}
\newcommand\squ[4][1]{\fill[#4] (#2*\sqw,#3*\sqw) rectangle +(#1*\sqw,#1*\sqw);}


% code 
\usepackage{listings}
\usepackage{xcolor}

\definecolor{codegreen}{rgb}{0,0.35,0}
\definecolor{codegray}{rgb}{0.5,0.5,0.5}
\definecolor{codenumber}{rgb}{0.1,0.3,0.5}
\definecolor{codeblue}{rgb}{0,0,0.5}
\definecolor{codered}{rgb}{0.5,0.03,0.02}
\definecolor{codegray}{rgb}{0.96,0.96,0.96}

\lstdefinestyle{pythonstylesheet}{
	language=Python,
	emphstyle=\color{deepred},
	backgroundcolor=\color{codegray},
	keywordstyle=\color{deepblue}\bfseries\itshape,
	numberstyle=\scriptsize\color{codenumber},
	basicstyle=\ttfamily\footnotesize,
	commentstyle=\color{codegreen}\itshape,
	breakatwhitespace=false, 
	breaklines=true, 
	captionpos=b, 
	keepspaces=true, 
	numbers=left, 
	numbersep=5pt, 
	showspaces=false,                
	showstringspaces=false,
	showtabs=false, 
	tabsize=4, 
	morekeywords={as,assert,nonlocal,with,yield,self,True,False,None,AssertionError,ValueError,in,else},              % Add keywords here
	keywordstyle=\color{codeblue},
	emph={object,type,isinstance,copy,deepcopy,zip,enumerate,reversed,list,set,len,dict,tuple,print,range,xrange,append,execfile,real,imag,reduce,str,repr,__init__,__add__,__mul__,__div__,__sub__,__call__,__getitem__,__setitem__,__eq__,__ne__,__nonzero__,__rmul__,__radd__,__repr__,__str__,__get__,__truediv__,__pow__,__name__,__future__,__all__,},          % Custom highlighting
	emphstyle=\color{codered},
	stringstyle=\color{codegreen},
	showstringspaces=false,
	abovecaptionskip=0pt,belowcaptionskip =0pt,
	framextopmargin=-\topsep, 
}
\newcommand\pythonstyle{\lstset{pythonstylesheet}}
\newcommand\pyl[1]     {{\lstinline!#1!}}
\lstset{style=pythonstylesheet}

\usepackage[style=1,skipbelow=\topskip,skipabove=\topskip,framemethod=TikZ]{mdframed}
\definecolor{bggray}{rgb}{0.85, 0.85, 0.85}
\mdfsetup{leftmargin=0pt,rightmargin=0pt,innerleftmargin=15pt,backgroundcolor=codegray,middlelinewidth=0.5pt,skipabove=5pt,skipbelow=0pt,middlelinecolor=black,roundcorner=5}
\BeforeBeginEnvironment{lstlisting}{\begin{mdframed}\vspace{-0.4em}}
	\AfterEndEnvironment{lstlisting}{\vspace{-0.8em}\end{mdframed}}


% Deisgn
\usepackage[labelfont=bf]{caption}
\usepackage[margin=0.6in]{geometry}
\usepackage{multicol}
\usepackage[skip=4pt, indent=0pt]{parskip}
\usepackage[normalem]{ulem}
\forestset{default}
\renewcommand\labelitemi{$\bullet$}
\usepackage{titlesec}
\titleformat{\section}[block]
{\fontsize{15}{15}}
{\sen \dotfill (\thesection) \she}
{0em}
{\MakeUppercase}
\usepackage{graphicx}
\graphicspath{ {./} }


% Hebrew initialzing
\usepackage[bidi=basic]{babel}
\PassOptionsToPackage{no-math}{fontspec}
\babelprovide[main, import, Alph=letters]{hebrew}
\babelprovide[import]{english}
\babelfont[hebrew]{rm}{David CLM}
\babelfont[hebrew]{sf}{David CLM}
\babelfont[english]{tt}{Monaspace Xenon}
\usepackage[shortlabels]{enumitem}
\newlist{hebenum}{enumerate}{1}

% Language Shortcuts
\newcommand\en[1] {\begin{otherlanguage}{english}#1\end{otherlanguage}}
\newcommand\sen   {\begin{otherlanguage}{english}}
	\newcommand\she   {\end{otherlanguage}}
\newcommand\del   {$ \!\! $}
\newcommand\ttt[1]{\en{\footnotesize\texttt{#1}\normalsize}}

\newcommand\npage {\vfil {\hfil \textbf{\textit{המשך בעמוד הבא}}} \hfil \vfil \pagebreak}
\newcommand\ndoc  {\dotfill \\ \vfil {\begin{center} {\textbf{\textit{שחר פרץ, 2024}} \\ \scriptsize \textit{נוצר באמצעות תוכנה חופשית בלבד}} \end{center}} \vfil	}

\newcommand{\rn}[1]{
	\textup{\uppercase\expandafter{\romannumeral#1}}
}

\makeatletter
\newcommand{\skipitems}[1]{
	\addtocounter{\@enumctr}{#1}
}
\makeatother

%! ~~~ Math shortcuts ~~~

% Letters shortcuts
\newcommand\N     {\mathbb{N}}
\newcommand\Z     {\mathbb{Z}}
\newcommand\R     {\mathbb{R}}
\newcommand\Q     {\mathbb{Q}}
\newcommand\C     {\mathbb{C}}

\newcommand\ml    {\ell}
\newcommand\mj    {\jmath}
\newcommand\mi    {\imath}

\newcommand\powerset {\mathcal{P}}
\newcommand\ps    {\mathcal{P}}
\newcommand\pc    {\mathcal{P}}
\newcommand\ac    {\mathcal{A}}
\newcommand\bc    {\mathcal{B}}
\newcommand\cc    {\mathcal{C}}
\newcommand\dc    {\mathcal{D}}
\newcommand\ec    {\mathcal{E}}
\newcommand\fc    {\mathcal{F}}
\newcommand\nc    {\mathcal{N}}
\newcommand\sca   {\mathcal{S}} % \sc is already definded
\newcommand\rca   {\mathcal{R}} % \rc is already definded

\newcommand\Si    {\Sigma}

% Logic & sets shorcuts
\newcommand\siff  {\longleftrightarrow}
\newcommand\ssiff {\leftrightarrow}
\newcommand\so    {\longrightarrow}
\newcommand\sso   {\rightarrow}

\newcommand\epsi  {\epsilon}
\newcommand\vepsi {\varepsilon}
\newcommand\vphi  {\varphi}
\newcommand\Neven {\N_{\mathrm{even}}}
\newcommand\Nodd  {\N_{\mathrm{odd }}}
\newcommand\Zeven {\Z_{\mathrm{even}}}
\newcommand\Zodd  {\Z_{\mathrm{odd }}}
\newcommand\Np    {\N_+}

% Text Shortcuts
\newcommand\open  {\big(}
\newcommand\qopen {\quad\big(}
\newcommand\close {\big)}
\newcommand\also  {\text{, }}
\newcommand\defi  {\text{ definition}}
\newcommand\defis {\text{ definitions}}
\newcommand\given {\text{given }}
\newcommand\case  {\text{if }}
\newcommand\syx   {\text{ syntax}}
\newcommand\rle   {\text{ rule}}
\newcommand\other {\text{else}}
\newcommand\set   {\ell et \text{ }}
\newcommand\ans   {\mathit{Ans.}}

% Set theory shortcuts
\newcommand\ra    {\rangle}
\newcommand\la    {\langle}

\newcommand\oto   {\leftarrow}

\newcommand\QED   {\quad\quad\mathscr{Q.E.D.}\;\;\blacksquare}
\newcommand\QEF   {\quad\quad\mathscr{Q.E.F.}}
\newcommand\eQED  {\mathscr{Q.E.D.}\;\;\blacksquare}
\newcommand\eQEF  {\mathscr{Q.E.F.}}
\newcommand\jQED  {\mathscr{Q.E.D.}}

\newcommand\dom   {\mathrm{dom}}
\newcommand\Img   {\mathrm{Im}}
\newcommand\range {\mathrm{range}}

\newcommand\trio  {\triangle}

\newcommand\rc    {\right\rceil}
\newcommand\lc    {\left\lceil}
\newcommand\rf    {\right\rfloor}
\newcommand\lf    {\left\lfloor}

\newcommand\lex   {<_{lex}}

\newcommand\az    {\aleph_0}
\newcommand\uaz   {^{\aleph_0}}
\newcommand\al    {\aleph}
\newcommand\ual   {^\aleph}
\newcommand\taz   {2^{\aleph_0}}
\newcommand\utaz  { ^{\left (2^{\aleph_0} \right )}}
\newcommand\tal   {2^{\aleph}}
\newcommand\utal  { ^{\left (2^{\aleph} \right )}}
\newcommand\ttaz  {2^{\left (2^{\aleph_0}\right )}}

\newcommand\n     {$n$־יה\ }

% Math A&B shortcuts
\newcommand\logn  {\log n}
\newcommand\logx  {\log x}
\newcommand\lnx   {\ln x}
\newcommand\cosx  {\cos x}
\newcommand\cost  {\cos \theta}
\newcommand\sinx  {\sin x}
\newcommand\sint  {\sin \theta}
\newcommand\tanx  {\tan x}
\newcommand\tant  {\tan \theta}
\newcommand\sex   {\sec x}
\newcommand\sect  {\sec^2}
\newcommand\cotx  {\cot x}
\newcommand\cscx  {\csc x}
\newcommand\sinhx {\sinh x}
\newcommand\coshx {\cosh x}
\newcommand\tanhx {\tanh x}

\newcommand\seq   {\overset{!}{=}}
\newcommand\slh   {\overset{LH}{=}}
\newcommand\sle   {\overset{!}{\le}}
\newcommand\sge   {\overset{!}{\ge}}
\newcommand\sll   {\overset{!}{<}}
\newcommand\sgg   {\overset{!}{>}}

\newcommand\h     {\hat}
\newcommand\ve    {\vec}
\newcommand\lv    {\overrightarrow}
\newcommand\ol    {\overline}

\newcommand\mlcm  {\mathrm{lcm}}

\DeclareMathOperator{\sech}   {sech}
\DeclareMathOperator{\csch}   {csch}
\DeclareMathOperator{\arcsec} {arcsec}
\DeclareMathOperator{\arccot} {arcCot}
\DeclareMathOperator{\arccsc} {arcCsc}
\DeclareMathOperator{\arccosh}{arccosh}
\DeclareMathOperator{\arcsinh}{arcsinh}
\DeclareMathOperator{\arctanh}{arctanh}
\DeclareMathOperator{\arcsech}{arcsech}
\DeclareMathOperator{\arccsch}{arccsch}
\DeclareMathOperator{\arccoth}{arccoth}
\DeclareMathOperator{\atant}  {atan2} 

\newcommand\dx    {\,\mathrm{d}x}
\newcommand\dt    {\,\mathrm{d}t}
\newcommand\dtt   {\,\mathrm{d}\theta}
\newcommand\du    {\,\mathrm{d}u}
\newcommand\dv    {\,\mathrm{d}v}
\newcommand\df    {\mathrm{d}f}
\newcommand\dfdx  {\diff{f}{x}}
\newcommand\dit   {\limhz \frac{f(x + h) - f(x)}{h}}

\newcommand\nt[1] {\frac{#1}{#1}}

\newcommand\limz  {\lim_{x \to 0}}
\newcommand\limxz {\lim_{x \to x_0}}
\newcommand\limi  {\lim_{x \to \infty}}
\newcommand\limh  {\lim_{x \to 0}}
\newcommand\limni {\lim_{x \to - \infty}}
\newcommand\limpmi{\lim_{x \to \pm \infty}}

\newcommand\ta    {\theta}
\newcommand\ap    {\alpha}
\newcommand\ag    {\alpha}
\newcommand\bg    {\beta}
\newcommand\cg   {\gamma}
\newcommand\F     {\mathbb{F}}

\renewcommand\inf {\infty}
\newcommand  \ninf{-\inf}

% Combinatorics shortcuts
\newcommand\sumnk     {\sum_{k = 0}^{n}}
\newcommand\sumni     {\sum_{i = 0}^{n}}
\newcommand\sumnko    {\sum_{k = 1}^{n}}
\newcommand\sumnio    {\sum_{i = 1}^{n}}
\newcommand\sumai     {\sum_{i = 1}^{n} A_i}
\newcommand\nsum[2]   {\reflectbox{\displaystyle\sum_{\reflectbox{\scriptsize$#1$}}^{\reflectbox{\scriptsize$#2$}}}}

\newcommand\bink      {\binom{n}{k}}
\newcommand\setn      {\{a_i\}^{2n}_{i = 1}}
\newcommand\setc[1]   {\{a_i\}^{#1}_{i = 1}}

\newcommand\cupain    {\bigcup_{i = 1}^{n} A_i}
\newcommand\cupai[1]  {\bigcup_{i = 1}^{#1} A_i}
\newcommand\cupiiai   {\bigcup_{i \in I} A_i}
\newcommand\capain    {\bigcap_{i = 1}^{n} A_i}
\newcommand\capai[1]  {\bigcap_{i = 1}^{#1} A_i}
\newcommand\capiiai   {\bigcap_{i \in I} A_i}

\newcommand\xot       {x_{1, 2}}
\newcommand\ano       {a_{n - 1}}
\newcommand\ant       {a_{n - 2}}

% Other shortcuts
\newcommand\tl    {\tilde}
\newcommand\op    {^{-1}}

\newcommand\sof[1]    {\left | #1 \right |}
\newcommand\cl [1]    {\left ( #1 \right )}
\newcommand\csb[1]    {\left [ #1 \right ]}

\newcommand\bs    {\blacksquare}

%! ~~~ Document ~~~

\author{שחר פרץ}
\title{ליניארית 1א 2}
\begin{document}
	\maketitle
	\section{\en{Reminders}}
	שייעור שעבר דיברנו על שגות, ועל מחלקת השקילות $\bmod$. 
	
	\section{\en{Modular Field}}
	\textbf{הגדרה. }$\Z / _{n\Z} = \{[x]_n \mid x \in \Z\}$ ("מודולו $n$z")
	
	נגדיר פעולות להיות: 
	\begin{gather*}
		[x]_n + [y]_n = [x + y]_n \\
		[x]_n \cdot [y]_n = [x \cdot y]_n
	\end{gather*}
	בשביל ח"ע, נדרוש שבפרט: 
	\[ \Z_5 \implies [1]_5 = [6]_5 = [11]_5, \ [1]_5 + [2]_5 = [3]_5 \seq [8]_5 = [6]_5 + [2]_5 \]
	
	\textbf{למה. }חיבור וכפל ב־$\Z/_{n\Z}$ מוגדרים היטב ואינם תלויים בבחירת הנציגים. 
	\begin{proof}
		יהי $A, B \in \Z$ מחלקות שקילות. יהיו $a_1, a_2, a \in A, b_1. b_2, b \in B$. כלומר $[a_1] + [a_2] = A, \ [b_1] + [b_2] = B$. נראה כי $[a_1 + b_1] = [a_2 + b_2]$ וגם $[a_1 \cdot b_1] = [a_2 \cdot b_2]$. ואכן: 
		\[ a_2 = a_1 + na, \ b_2 =b_1 + nb \]
		אזי
		\[ a_2 + b_2 = a_1 + b_1 + b(a + b) \equiv a_1 + b_1 \mod n \]
		ובעבור כפל: 
		\[ a_2b_2 = (a_1 + na) (b + nb) = \cdots = a_1b_1 \mod n \]
	\end{proof}
	
	נרצה לחקור מתי הדבר הזה הוא שדה, ומתי הוא לא. 
	
	\textbf{טענה. }לכל $n> 1$ הקבוצה $\Z/_{n\Z}$ עם $[0]$ בתור איבר ה־$0$ ו־$[1] $ בתור איבר היחידה, מקיימת את כל התכונות של שדה פרט להופכי. 
	
	\subsection{דוגמאות}
	\begin{align}
		\Z_2 &= \{0, 1\}, \ 1 \cdot 1 \equiv 1 \mod 2 \\
		\Z_3 &= \{1, 2, 3\}, \ 1 \cdot 1 \equiv \mod 3, \ 2 \cdot 2 = 4 \equiv 1 \mod 3 \\
		\Z_4 &= \{1, 2, 3, 4\}, \ 2 \cdot 2 \equiv 4 \equiv 0 \\
		\Z_5, \ 2 \cdot 3 = 6 \equiv 0 \mod 6
	\end{align}
	השניים האחרונים סתירה כי לא ייתכנו שני איברים שכפלם הוא 0. 
	
	\textbf{טענה. }$\Z/_{n\Z}$ שדה אמ"מ $n$ ראשוני. 
	\textbf{תכונה של ראשוניים. }$p$ ראשוני וגם $p \mid n = ab$ ן־$a, b \in \Z$, אז $p \mid a \lor p \mid b$. 
	\begin{proof}
		\begin{enumerate}
			\item[$\impliedby$] אם $n$ לא ראשוני, אז $\exists a, b \in \N. 1 \le ab < n, \ n = ab$. אזי $ab \not\equiv 0 \mod n$. אבל $ab \equiv - \mod n$, ולכן $\Z_n$ לא שדה. 
			\item[$\implies$]נניח $p$ ראשוני. יהיה $x \in \Z_{n\Z}$ כך ש־$x \not\equiv 0 \mod p$. נראה כי $f \colon \Z_p \to \Z_p$ כאשר $f([y]) = [x][y]$ היא הפיכה. נראה שהיא חח"ע. יהיו $y_1, y_2 \in \Z_n$ נבקש $f(y_1) = f(y)2$ כלומר $xy_1 \equiv xy_2 \mod n$ וסה"כ $n \mid x(y_1 - y_2)$. אזי $n$ ($p$ שלכם) ראשוני ולכן: 
			$p \mid x \lor p \mid (y_1 - y_2)$. לא ייתכן $p \mid x$ כי $x \equiv 0 \mod n$. סה"כ $p \mid (y_1 - y_2)[y_1] = [y_2]$. ולכן $f$ חחע. וכם על מקור ותמונה סופית וזהים בגודלם, ולכן $f$ על ולכן $\exists y \in \Z_n. f(y) = xy = 1$. ל־$x$ יש הופכי. 
		\end{enumerate}
	\end{proof}
	
	\section{\en{idk the name in english}}
	\textbf{הגדרה. }יהי $F$ שדה, $a \in F, \Z \ni  n \ge 0$. נגדיר: 
	\begin{gather}
		n \cdot a := \underbrace{a + \dots + a}_{\times n} \\
		(-n) \cdot a := -(na)
	\end{gather}
	נאמר שה\textit{מציין} של השדה הוא אפס אם $\forall n > 0. n \times 1_f \neq 0$. אחרת, המקדם של השדה יהיה: 
	\[ \mathrm{char}(F) = \min\{n \in \N \mid n \cdot 1_f = 0\} \]
	\textbf{משפט. }$F$ שדה. יהי $p \ge 0$ מציין של $F$. נגרר: 
	\begin{enumerate}
		\item $p = 0 \lor p \text{ ראשוני}$
		
		\begin{proof}
			אם $p = 0$ אז $F$ מכיל עותק של $ 2$. אחרת, $F$ מכיל עותק שלם של $\Z$. 
			אם אכן $n \in \N$ נבקש $n \cdot 1_F = 0 $ כלומר $\mathrm{char}(F) = 0$. 
			
			נזהה עם $\Q$. לכל $n \in \N$ נזהה את $n \cdot 1_F$ עם $n$ ולכן $F$ "מכיל" את הטבעיים. נזהה את $-(n \cdot 1_F)$ עם $-n$ כלומר $F$ "מכיל" את $\Z$. עכשיו לכל $m, n \in \Z$ מזהה את $m \cdot n\op$ עם $\frac{n}{m} \in \Q$. ולכן קיבלנו "עותק" של $\Q$ (למעשה, צריך פורמלית להראות קיםו איזומורפיזם). 
			
			במקרה השני, נניח $n \cdot 1_F = 0$. יהיה $p$ הטבעי המינימלי שמקיים $p = \mathrm{char(F0)}$. נניח בשלילה ש־$p$ לא ראשוני, אזי $a \le a, b, < p$ עבורם $p = ab$ קיימים. מכיוון ש־$p$ מינימלי עבור $p \cdot 1 = 0$. לכן: 
			\begin{gather}
				b \cdot 1 \neq 0, \ a \cdot 1 \neq 0, \ (ab) \cdot 1 = 0 \implies (a \cdot 1_F) \cdot (b \cdot 1_F) = 0
			\end{gather}
			בסתירה כי מצאנו $a, b \neq 0$ כך ש־$ab = 0$ וגם $a, b \in F$. 
			
			יהי $a \in \Z_p$. נזהה עם $a$ עם $1_F \cdot a$. נשים לב ש־$a \equiv \bmod \R \iff a = b + kp$. לכן: 
			\[ a \cdot 1_F = (b + pk) \cdot 1 = b \cdot 1_F  + k(pk) = b1_F \]
		\end{proof}
		\item המציין של שדה סופי הוא חיובי. 
		\begin{proof}
				יש אינסוף טבעיים, אך $|F|$ סופית. לכן $n \cdot 1_F = m \cdot 1_F$ (שובך היונים). בה"כ $m > n$. לכן: 
				\[ m \cdot 1_F -n \cdot 1_F = (m - n) \cdot 1_F = 0 \]
				ובפרט $(m - n) \in \N$. משהו לגבי מינימום שלא הספקתי כי התעסקתי עם השלט של השם. 
		\end{proof}
		
		
	\end{enumerate}
	
	\section{\en{The Matrix}}
	הפאנץ': בהינתן מערכת משוואות כמו: 
	\[ \begin{cases}
		3x + y = 7 \\
		8 = 2x - y
	\end{cases} \sim \begin{pmatrix}
		3 & 1 \\
		2 & -1
	\end{pmatrix} \begin{matrix}
	 7 \\
	 8
	\end{matrix} \]
	נוכל לעשות משחקים על המטריצות כמו על משוואות רגילות, כמו לחלק ולחסר אגפים. 
	
	\subsection{מערכת משוואות ליניאריות}
	\textbf{הגדרה. }\textit{משוואה ליניארית} מעל שדה $F$ ב־$n$ נעלמים $x_1, \dots, x_n$ עם $a_1, \dots a_n$ מקדמים היא משוואה מהצורה: 
	\[ a_1x_1 + \dots + a_n x_n = b \]
	(זהו הייצוג הסטנדטי)
	לדוגמה $3x - 7 = 0$ ליניארי אך לא סטנדרטי, בעוד $y^2 + 7 = x$ כלל לא ליניארי. 
	
	\textbf{הגדרה. }\textit{מערכת של $m$ משוואות ב־$n$ נעלמים} מעל שדה $F$ הוא אוסף של $m$ משוואות מעל $F$ ב־$n$ נעלמים. צורת ריסום סטנדרטית: 
	\[ \begin{cases}
		a_{11}x_{1} + a_{12}x_{2} + \dots + a_{1n} = b_1 \\
		a_{21}x_{1} + a_{22}x_{2} + \dots + a_{2n} = b_2 \\
		\vdots \\
		a_{n1}x_{n1} + \dots + a_{nn} = b_n
	\end{cases} \]
	ל־$b_1, \dots b_n \in $F  נקרא \textit{מקדמים חופשיים}. לדוגמה: 
	\[ \begin{cases}
		x_1 + 3x_2 = 2 \\
		7x_1 - 6x_2 = 1
	\end{cases} \]
	יתקיים $a_{12} = 3, b_1 = 1$ וכו'. 
	
	\textbf{הגדרה. }$a_{ij} \in F$ \textit{מקדמים} $i \in \{1, \dots m\}, \ j \in \{1, \dots ,n\}$. 
	
	\textbf{הגדרה. }$A$ קבוצה לא ריקה, $n \in \N$. יהיו $a_1, \dots a_n$. נסמן את ה־\textit{\n שאיבריה לפי הסדר} בתור $(a_1, \dots a_n) \in A^n$. 
	"שתי $n$־יות שוות אם שוות בכל $n$-מקום" (פרמול בבדידה). 
	
	\textbf{הגדרה. }\textit{פתרון למערכת משוואות} זה $(x_1 \dots x_n) \in F^n$ כך שכל המשוואות מתקיימת לאחר הצבה. 
	
	\textbf{הגדרה. }שתי מערכות משוואות נקראות \textit{שקולות} אם יש להן את אותה קבוצת הפתרונות. 
	
	\textbf{חידה. }בהינתן שדה $F = \Z_{17}$. הוכיחו, שאין מערכת משוואות עם בדיוק $16$ פתרונות. 
	
	דוגמה: בעבור $x + y = 0$, קבוצת הפתרונות היא $\{(\alpha, -\alpha) \mid \alpha \in \mathbb{F}\}$. 
	ל־$x \in \F^n$ נקרא וקטור. $c \in \F$ יקרא סקלר. בעבור מערכת משוואות, נוכל לחסר משהו מהמשוואות, להפכיל אותן, וכו', ולשמר את קבוצת הפתרונות. 
	
	\textbf{הגדרה. }תהי מערכת משוואות. פעולה אלמנטרית היא אחת מבין: 
	\begin{enumerate}
		\item החלפת מיקום של שתי משוואות. 
		\item הכפלה של משוואה אחת בסקלר שונה מ־$0$. 
		\item הוספה לאחת המשוואות משוואה אחרת מוכפלת בסקלר. 
	\end{enumerate}
	\textbf{משפט. }פעולה אלמנטרית אל מערכת משוואות מעבירה למערכת שקולה. 
	\begin{proof} \ 
		
		\textbf{החלפת סדר} לא משפיע על האם $x \in \F^n$ הוא פתרון. 
		
		נסתכל על מרעכת משוואות \textbf{מוכפלת בסקלר} $\lambda \neq 0$. 
		\[ \begin{cases}
			\sum_{i = 0}^{n}a_{1i}x_i = b_1 \\
			\vdots \\
			\lambda \sum_{i = 0}^{n}a_{ti}x_i = \lambda b_t \\
			\vdots \\
			\sum_{i = 0}^{n}a_{ni}x_i = b_n
		\end{cases} \]
		
		יהי $(\ag_1, \dots ag_n) \in \F^n$ שפותר את המערכת המקורית. נראה שגם פותר את החדשה: 
		\[ \lambda \underbrace{\sum x_{tj}x_j}_{bt} = \lambda b_t \]
		כדרוש. 
		נראה כיוון הפוך (כדי להראות שלא הוספנו פתרונות). יהי $\ag \in F^n$ פתרון של החדשה. נסתכל על מערכת משוואות חדשה מאוד, מוכפלת ב־$\frac{1}{\lambda}$. מההוכחה שלנו, $\alpha$ פתרון שלה, וזו בדיוק המקורית. 
		
		\textbf{הכפלה בסקלר וחיסור}. יהי $\ag \in F^n$ פתרון של המקורית. נראה שהוא של החדשה. 
		\textit{לא פורמלי, תוכלו לפרמל בצעמכם. }הפעולה שעשינו היא על שורה $t$. חיבור $c \cdot \binom{\text{שורה}}{p}$. נקבל: 
		\[ \underbrace{\sum x_j a_{tj}}_{bt} + \underbrace{\sum x_j a_{pj}}_{bp} = b_t + cb_p \]
		כיוון הפוך אפשר לעשות באופן דומה ע"י יצירת משוואה חדשה מאוד. 
		\textit{סוף קטע לא פורמלי. }
		
	\end{proof}
	
	\subsection{\en{Blue Pill, Red Pill}}
	יהיו $m, n \in \N$. \textbf{מטריצה} מסדר $m\times n$ אוסף $mn$ סקלרים מסוגרים במבלן $a_{ij}$. יתקיים: 
	\begin{gather}
		i \in \{1 \dots m\}, \ j \in \{1 \dots n\} \\
		A = (a_{ij})_{\begin{matrix}
				i = 1\dots m \\
				\small  j = 1 \dots n
		\end{matrix}} = \begin{pmatrix}
		a_{11} & a_{12} & \dots & a{1n} \\
		a_{21} & a_{22} & \dots & a_{2n} \\
		\vdots & \ & \ & \vdots \\
		a_{m1} & a_{m2} & \dots & a_{mn}
	\end{pmatrix}
	\end{gather}
	כאשר $R_i := (a_{1i}, \dots, a_{in}) \in \F^1$ יקרא \textit{וקטור השורה}. 
	
	$c_j := (a_1k, \dots, a_{mj}) \in \F^1$ יקרא \textit{וקטור עמודה}/ 
	
	\[ A =  \begin{pmatrix}
		R_1 \\
		R_2 \\
		\vdots \\
		R_n
	\end{pmatrix} = \begin{pmatrix}
	C_1 \dots C_n
	\end{pmatrix} \]
	
	$:= M_{mn}(F)$ כל המטרציות מסדר $m\times n$ מעל שדה $\F$. 
	
	$:= M_{n}(F)$ כל המטריצות מסדר $n \times n$  מעל שדה $\F$ (\textit{מטריצות ריבועיות}). 
	
	לדוגמה: 
	\[ (4) \in M_1(F), \ (1 \ 2 \ 3) \in M_{1 \times 3}(F), \ \begin{pmatrix}
		1 \\
		2 \\
		3
	\end{pmatrix} \in M_{3 \times 1}, \ \begin{pmatrix}
	4 & -1 & 7\\
	7 & -2 & 4
	\end{pmatrix} \in M_{2 \times 3}(F) \]
	
	\textit{מטריצה של מערכת משוואות: }
	\[ A = \begin{pmatrix}
		a_{11} &\dots & a_n & b_1 \\
		\vdots & \ & \ & \vdots \\
		a_{m1} & \dots & a_{mn} & b_n
	\end{pmatrix} \]
	\textit{מטריצה מצומצמת} היא מטריצה בלי העמודה ה־$m + 1$. 
	
	\textbf{הגדרה. }פעולות אלמנטריות על מטריצה: 
	\begin{enumerate}
		\item החלפת מיקום שורות $R_i \siff R_j$\\
		\item הכפלה של שורה בסקלר שונה מאפס: $R_i \to \lambda R_i$
		\item הוספה לשורה אחרת מוכפלת בסקלר: $R_i \to R_i + C \cdot R_j, \ \neq 0 c \in \F$
	\end{enumerate}
	
	דוגמה: 
	\[ \begin{cases}
		x + y + z = 1 \\
		x + 2y + 3z = 4\\
		2x + 0 + z = -1
	\end{cases} \implies \left( 
	\begin{matrix}
		1  & 1 & 1\\
		1 & 2 & 3 \\
		2 & 0 & 1
	\end{matrix}
	\middle\vert
	\begin{matrix}
		1 \\
		4 \\
		-1
	\end{matrix}
	\right) \overset{R_2 \to R_2 - R_1}{\to} \cl{\begin{matrix}
			1 & 1 &1 \\
			0 & 1 & 2 \\
			2 & 0 & 1
	\end{matrix} \middle\vert \begin{matrix}
	1 \\ 3 \\ -1
	\end{matrix}} \overset{R_3 \to R_3 - 2R_1}{\to} \cl{\begin{matrix}
				1 &1 & 1 \\
				0 & 1 & 2\\
				0 & -2 & -1
	\end{matrix}\middle\vert\begin{matrix}
	1 \\ 3 \\ -3
	\end{matrix}} \]
	
	\[ \overset{R_1 \to R_1 - R_2, R_3 \to R_3 + 2R_2}{\to} \cl{\begin{matrix}
			1 & 0 & -1 \\
			0 & 1 & 2 \\
			0 & 0 & 3
	\end{matrix}\middle\vert\begin{matrix}
	-2 \\ 3 \\ 3
	\end{matrix}} \overset{J}{\to} \cl{\begin{matrix}
	1 & 0 & 0 \\
	0 & 1 & 0\\
	0 & 0 &1
	\end{matrix}\middle\vert\begin{matrix}
	-1 \\ 1\\1
	\end{matrix}}
	 \]
	 כאשר $J$ אומר $R_3 \to 1 / 3R_3$ וגם $R_1 \to R_1 + R-3, \ R_2 \to R_2 - 2R_3$ . 
	 
	 \textbf{הגדרה. }$A, B \in M_{n, m}$ מטריצות. נאמר ש־$A, B$ \textit{שקולות} אם ניתן לקבל מ־$B$ את $A$ ע"י מספר סופי של פעולות אלמנטריות. נסמן $A \sim B$. 
	 
	 \textbf{טענה. }יחס זה הוא שקילות. 
	\begin{proof}\ 
		\begin{itemize}
			\item $A \sim A$: ברור, כי 0 פעולות. 
			\item $A \sim B, \ B \sim C$: נסמן בתור $E$ את רצף הפעולות מ־$A$ ל־$B$ וב־$E'$ את רצף הפעולות מ־$B$ ל־$C$. בהתאם, $E, E'$ יהיה מ־$A$ ל־$C$. 
			\item $A \sim B$ ונראה $B \sim A$. נסמן את $E_1, \dots E_b$ שדה של פעולות עלמנטריות מ־$A$ ל־$B$ ונמצא $E\op$ כך שסדרה מ־$B$ ל־$A$. נסתכל על $E_x$ ונמצא הפוכה. 
			\begin{itemize}
				\item החלפת שורות: $Ex = E\op x$
				\item מכפלה בסלקר: $E\op x = \frac{1}{\lambda}R_{\text{שורה}}$
				\item הוספת שורה כפולה:
				$E\op x = R_{\text{שורה}} \to \lambda R_{\text{שורה אחרת}}$
			\end{itemize}
			ונסמן $E\op = E\op_t, E\op_{t - 1}, \dots E\op_1$. סה"כ מצאנו הוכפי. 
		\end{itemize}
	\end{proof}
	
	\textbf{הגדרה. }\textit{שורת האפסים} אם כל הרכיבים  $0$
	
	\textit{שורה שאיננה אפסים. }שאיננה שורת אפסים. 
	
	\textit{איבר פותח. }האיבר הכי שמאלי שאינו אפס. 
	
	\textit{מטריצה מדורגת} אם: 
	\begin{enumerate}
		\item כל שורות האפסים מתחת לשורות שאינן אפסים
		\item האיבר פותח של שורה נמצא מימין לאיבר הפותח של השורה מעליה (מימין, אך לא בהכרח בעמודה אחת). 
	\end{enumerate}
	
	\textbf{הגדרה. }$A$ מטריצה. $A$ \textit{מדורגת קאנונית} אם כל איבר פותח הוא $1$ וגם שאר האיברים בעמודה הם $0$, וגם שאר האיברים בעמודה הם $0$, וגם $A$ מדורגת. 
	
	\textbf{הגדרה. }מערכת משוואות אשר מיוצגת במטריצה ששקולת שורה למטריצה מדורגת כלשהי. 
	
	משתנה \textit{קשור} (תלוי) אם הוא מיוצג בעמודה שבה אם יש איבר פותח, המטריצה מדורגת. 
	
	\textbf{משפט. }כל מטריצה שקולה שורות למטריצה מגורדת קאנונית יחידה ("שיטת האלימינציה של גאוס"). 
	\begin{proof}
		(לא נוכיח יחידות). אלג' בצעדים אשר יגיע ליעד. 
		\textbf{שלב 1. }$1 \le j \le n$ מספר הכי קטן של עמודה ששונה מ־$0$ $c_j$. נסמן $1 \le i \le n$ רכיב הכי קטן ב־$c_j$ שאיננו אפס. $a = a_{ij}$. (כלומר, מינימלי כך שמעל ומימין ל־$a_{ij}$ יש אפסים בלבד)
		נבצע פעולות אלמנטריות: $R_1 \siff R_i, \ R_1 \to \frac{1}{a}R_1$. ונקבל: 
		\[ \begin{pmatrix}
			0... & 0 & ?? \\
			0... & 1 & ??\\
			0... & ?? & ?? \\
			\vdots & \vdots & \vdots
		\end{pmatrix} \]
		\textbf{צעד 2. }עבור $i = 2\dots n$ נבצע $R_1 \to R_i - \ag_iR_1$ כאשר $\ag_i$ הוא הרכיב ה־$ij$ של $A_1$. בכך נפנה את כל מה שמתחת ל־1. 
		
		\textbf{צעד 3. }נחזור על שלבים $1, 2$ על העמודה ה־$2$ של $A_2$ ושורה $j + 1$. 
		
		\textbf{שלב 4}. נחזור על צעד 3 $n$ פעמים או עד שהמטריצה שנמצאת בפינה הימנית התחתונה תהיה אפסים ונקבל מטריצה מדורגת עם איבר פותח 1. 
		קיבלנו מטריצה מדורגת, כמעט קאנונית. 
		
		\textbf{שלב 5. }עבור $i = 2 \dots n$, נעשה $R-1 \to R_1 - \ag_1R_i$ עבור $=\alpha$ הרכיב של השורה הראשונה שנמצא מעל האיבר הפותח של השורה ה־$i$. נעשה $R_2 \to R_2 - \beta R_i$ עבור $i = 3 \dots n$. 
		
		כדי לכנס את כל האיברים והעמודות עם איבר פותח פרט לאיבר המוביל. בסוף נקבל קאנונית. 
	\end{proof}
	\section{\en{Solutions to Linear Equation Systems}}
	בעבור מערכת כמו $(0 \ \dots \ 0 \ 1)$ אין פתרון. גם בעבור: 
	\[ \cl{\begin{matrix}
			1 & 0 & 3 \\
			0 & 1 & 2 \\
			0 & 0 & 0 
		\end{matrix} \middle\vert \begin{matrix}
			4 \\
			7 \\
			0
	\end{matrix}} \implies \cl{\begin{matrix}
	1 & 0 \\
	0 & 1
\end{matrix}\middle\vert \begin{matrix}
* \\
*
\end{matrix}} \]
	ולכל $x_3$ נקבל פתרון: $\{(4 - 3t, 7 - 2t) \mid t \in \F\}$. מסקנה. 
	\textbf{מסקנה. }מערכת משוואות עם $\text{מספר נעלמים} > \text{משפר משוואות}$ גורר (1) אין פתרונות, או (2) מספר הפתרונות לפחות $|F|$. \textit{(אם $\F$ אין־סופי, כך גם כמות הפתרונות)}. 

\end{document}