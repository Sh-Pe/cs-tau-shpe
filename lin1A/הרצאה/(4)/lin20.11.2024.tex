%! ~~~ Packages Setup ~~~ 
\documentclass[]{article}


% Math packages
\usepackage[usenames]{color}
\usepackage{forest}
\usepackage{ifxetex,ifluatex,amsmath,amssymb,mathrsfs,amsthm,witharrows,mathtools}
\WithArrowsOptions{displaystyle}
\renewcommand{\qedsymbol}{$\blacksquare$} % end proofs with \blacksquare. Overwrites the defualts. 
\usepackage{cancel,bm}
\usepackage[thinc]{esdiff}


% tikz
\usepackage{tikz}
\newcommand\sqw{1}
\newcommand\squ[4][1]{\fill[#4] (#2*\sqw,#3*\sqw) rectangle +(#1*\sqw,#1*\sqw);}


% code 
\usepackage{listings}
\usepackage{xcolor}

\definecolor{codegreen}{rgb}{0,0.35,0}
\definecolor{codegray}{rgb}{0.5,0.5,0.5}
\definecolor{codenumber}{rgb}{0.1,0.3,0.5}
\definecolor{codeblue}{rgb}{0,0,0.5}
\definecolor{codered}{rgb}{0.5,0.03,0.02}
\definecolor{codegray}{rgb}{0.96,0.96,0.96}

\lstdefinestyle{pythonstylesheet}{
	language=Python,
	emphstyle=\color{deepred},
	backgroundcolor=\color{codegray},
	keywordstyle=\color{deepblue}\bfseries\itshape,
	numberstyle=\scriptsize\color{codenumber},
	basicstyle=\ttfamily\footnotesize,
	commentstyle=\color{codegreen}\itshape,
	breakatwhitespace=false, 
	breaklines=true, 
	captionpos=b, 
	keepspaces=true, 
	numbers=left, 
	numbersep=5pt, 
	showspaces=false,                
	showstringspaces=false,
	showtabs=false, 
	tabsize=4, 
	morekeywords={as,assert,nonlocal,with,yield,self,True,False,None,AssertionError,ValueError,in,else},              % Add keywords here
	keywordstyle=\color{codeblue},
	emph={object,type,isinstance,copy,deepcopy,zip,enumerate,reversed,list,set,len,dict,tuple,print,range,xrange,append,execfile,real,imag,reduce,str,repr,__init__,__add__,__mul__,__div__,__sub__,__call__,__getitem__,__setitem__,__eq__,__ne__,__nonzero__,__rmul__,__radd__,__repr__,__str__,__get__,__truediv__,__pow__,__name__,__future__,__all__,},          % Custom highlighting
	emphstyle=\color{codered},
	stringstyle=\color{codegreen},
	showstringspaces=false,
	abovecaptionskip=0pt,belowcaptionskip =0pt,
	framextopmargin=-\topsep, 
}
\newcommand\pythonstyle{\lstset{pythonstylesheet}}
\newcommand\pyl[1]     {{\lstinline!#1!}}
\lstset{style=pythonstylesheet}

\usepackage[style=1,skipbelow=\topskip,skipabove=\topskip,framemethod=TikZ]{mdframed}
\definecolor{bggray}{rgb}{0.85, 0.85, 0.85}
\mdfsetup{leftmargin=0pt,rightmargin=0pt,innerleftmargin=15pt,backgroundcolor=codegray,middlelinewidth=0.5pt,skipabove=5pt,skipbelow=0pt,middlelinecolor=black,roundcorner=5}
\BeforeBeginEnvironment{lstlisting}{\begin{mdframed}\vspace{-0.4em}}
	\AfterEndEnvironment{lstlisting}{\vspace{-0.8em}\end{mdframed}}


% Deisgn
\usepackage[labelfont=bf]{caption}
\usepackage[margin=0.6in]{geometry}
\usepackage{multicol}
\usepackage[skip=4pt, indent=0pt]{parskip}
\usepackage[normalem]{ulem}
\forestset{default}
\renewcommand\labelitemi{$\bullet$}
\usepackage{titlesec}
\titleformat{\section}[block]
{\fontsize{15}{15}}
{\sen \dotfill (\thesection) \she}
{0em}
{\MakeUppercase}
\usepackage{graphicx}
\graphicspath{ {./} }


% Hebrew initialzing
\usepackage[bidi=basic]{babel}
\PassOptionsToPackage{no-math}{fontspec}
\babelprovide[main, import, Alph=letters]{hebrew}
\babelprovide[import]{english}
\babelfont[hebrew]{rm}{David CLM}
\babelfont[hebrew]{sf}{David CLM}
\babelfont[english]{tt}{Monaspace Xenon}
\usepackage[shortlabels]{enumitem}
\newlist{hebenum}{enumerate}{1}

% Language Shortcuts
\newcommand\en[1] {\begin{otherlanguage}{english}#1\end{otherlanguage}}
\newcommand\sen   {\begin{otherlanguage}{english}}
	\newcommand\she   {\end{otherlanguage}}
\newcommand\del   {$ \!\! $}
\newcommand\ttt[1]{\en{\footnotesize\texttt{#1}\normalsize}}

\newcommand\npage {\vfil {\hfil \textbf{\textit{המשך בעמוד הבא}}} \hfil \vfil \pagebreak}
\newcommand\ndoc  {\dotfill \\ \vfil {\begin{center} {\textbf{\textit{שחר פרץ, 2024}} \\ \scriptsize \textit{נוצר באמצעות תוכנה חופשית בלבד}} \end{center}} \vfil	}

\newcommand{\rn}[1]{
	\textup{\uppercase\expandafter{\romannumeral#1}}
}

\makeatletter
\newcommand{\skipitems}[1]{
	\addtocounter{\@enumctr}{#1}
}
\makeatother

%! ~~~ Math shortcuts ~~~

% Letters shortcuts
\newcommand\N     {\mathbb{N}}
\newcommand\F     {\mathbb{F}}
\newcommand\Z     {\mathbb{Z}}
\newcommand\R     {\mathbb{R}}
\newcommand\Q     {\mathbb{Q}}
\newcommand\C     {\mathbb{C}}

\newcommand\ml    {\ell}
\newcommand\mj    {\jmath}
\newcommand\mi    {\imath}

\newcommand\powerset {\mathcal{P}}
\newcommand\ps    {\mathcal{P}}
\newcommand\pc    {\mathcal{P}}
\newcommand\ac    {\mathcal{A}}
\newcommand\bc    {\mathcal{B}}
\newcommand\cc    {\mathcal{C}}
\newcommand\dc    {\mathcal{D}}
\newcommand\ec    {\mathcal{E}}
\newcommand\fc    {\mathcal{F}}
\newcommand\nc    {\mathcal{N}}
\newcommand\sca   {\mathcal{S}} % \sc is already definded
\newcommand\rca   {\mathcal{R}} % \rc is already definded

\newcommand\Si    {\Sigma}

% Logic & sets shorcuts
\newcommand\siff  {\longleftrightarrow}
\newcommand\ssiff {\leftrightarrow}
\newcommand\so    {\longrightarrow}
\newcommand\sso   {\rightarrow}

\newcommand\epsi  {\epsilon}
\newcommand\vepsi {\varepsilon}
\newcommand\vphi  {\varphi}
\newcommand\Neven {\N_{\mathrm{even}}}
\newcommand\Nodd  {\N_{\mathrm{odd }}}
\newcommand\Zeven {\Z_{\mathrm{even}}}
\newcommand\Zodd  {\Z_{\mathrm{odd }}}
\newcommand\Np    {\N_+}

% Text Shortcuts
\newcommand\open  {\big(}
\newcommand\qopen {\quad\big(}
\newcommand\close {\big)}
\newcommand\also  {\text{, }}
\newcommand\defi  {\text{ definition}}
\newcommand\defis {\text{ definitions}}
\newcommand\given {\text{given }}
\newcommand\case  {\text{if }}
\newcommand\syx   {\text{ syntax}}
\newcommand\rle   {\text{ rule}}
\newcommand\other {\text{else}}
\newcommand\set   {\ell et \text{ }}
\newcommand\ans   {\mathit{Ans.}}

% Set theory shortcuts
\newcommand\ra    {\rangle}
\newcommand\la    {\langle}

\newcommand\oto   {\leftarrow}

\newcommand\QED   {\quad\quad\mathscr{Q.E.D.}\;\;\blacksquare}
\newcommand\QEF   {\quad\quad\mathscr{Q.E.F.}}
\newcommand\eQED  {\mathscr{Q.E.D.}\;\;\blacksquare}
\newcommand\eQEF  {\mathscr{Q.E.F.}}
\newcommand\jQED  {\mathscr{Q.E.D.}}

\newcommand\dom   {\mathrm{dom}}
\newcommand\Img   {\mathrm{Im}}
\newcommand\range {\mathrm{range}}

\newcommand\trio  {\triangle}

\newcommand\rc    {\right\rceil}
\newcommand\lc    {\left\lceil}
\newcommand\rf    {\right\rfloor}
\newcommand\lf    {\left\lfloor}

\newcommand\lex   {<_{lex}}

\newcommand\az    {\aleph_0}
\newcommand\uaz   {^{\aleph_0}}
\newcommand\al    {\aleph}
\newcommand\ual   {^\aleph}
\newcommand\taz   {2^{\aleph_0}}
\newcommand\utaz  { ^{\left (2^{\aleph_0} \right )}}
\newcommand\tal   {2^{\aleph}}
\newcommand\utal  { ^{\left (2^{\aleph} \right )}}
\newcommand\ttaz  {2^{\left (2^{\aleph_0}\right )}}

\newcommand\n     {$n$־יה\ }

% Math A&B shortcuts
\newcommand\logn  {\log n}
\newcommand\logx  {\log x}
\newcommand\lnx   {\ln x}
\newcommand\cosx  {\cos x}
\newcommand\cost  {\cos \theta}
\newcommand\sinx  {\sin x}
\newcommand\sint  {\sin \theta}
\newcommand\tanx  {\tan x}
\newcommand\tant  {\tan \theta}
\newcommand\sex   {\sec x}
\newcommand\sect  {\sec^2}
\newcommand\cotx  {\cot x}
\newcommand\cscx  {\csc x}
\newcommand\sinhx {\sinh x}
\newcommand\coshx {\cosh x}
\newcommand\tanhx {\tanh x}

\newcommand\seq   {\overset{!}{=}}
\newcommand\slh   {\overset{LH}{=}}
\newcommand\sle   {\overset{!}{\le}}
\newcommand\sge   {\overset{!}{\ge}}
\newcommand\sll   {\overset{!}{<}}
\newcommand\sgg   {\overset{!}{>}}

\newcommand\h     {\hat}
\newcommand\ve    {\vec}
\newcommand\lv    {\overrightarrow}
\newcommand\ol    {\overline}

\newcommand\mlcm  {\mathrm{lcm}}

\DeclareMathOperator{\sech}   {sech}
\DeclareMathOperator{\csch}   {csch}
\DeclareMathOperator{\arcsec} {arcsec}
\DeclareMathOperator{\arccot} {arcCot}
\DeclareMathOperator{\arccsc} {arcCsc}
\DeclareMathOperator{\arccosh}{arccosh}
\DeclareMathOperator{\arcsinh}{arcsinh}
\DeclareMathOperator{\arctanh}{arctanh}
\DeclareMathOperator{\arcsech}{arcsech}
\DeclareMathOperator{\arccsch}{arccsch}
\DeclareMathOperator{\arccoth}{arccoth}
\DeclareMathOperator{\atant}  {atan2} 

\newcommand\dx    {\,\mathrm{d}x}
\newcommand\dt    {\,\mathrm{d}t}
\newcommand\dtt   {\,\mathrm{d}\theta}
\newcommand\du    {\,\mathrm{d}u}
\newcommand\dv    {\,\mathrm{d}v}
\newcommand\df    {\mathrm{d}f}
\newcommand\dfdx  {\diff{f}{x}}
\newcommand\dit   {\limhz \frac{f(x + h) - f(x)}{h}}

\newcommand\nt[1] {\frac{#1}{#1}}

\newcommand\limz  {\lim_{x \to 0}}
\newcommand\limxz {\lim_{x \to x_0}}
\newcommand\limi  {\lim_{x \to \infty}}
\newcommand\limh  {\lim_{x \to 0}}
\newcommand\limni {\lim_{x \to - \infty}}
\newcommand\limpmi{\lim_{x \to \pm \infty}}

\newcommand\ta    {\theta}
\newcommand\ap    {\alpha}

\renewcommand\inf {\infty}
\newcommand  \ninf{-\inf}

% Combinatorics shortcuts
\newcommand\sumnk     {\sum_{k = 0}^{n}}
\newcommand\sumni     {\sum_{i = 0}^{n}}
\newcommand\sumnko    {\sum_{k = 1}^{n}}
\newcommand\sumnio    {\sum_{i = 1}^{n}}
\newcommand\sumai     {\sum_{i = 1}^{n} A_i}
\newcommand\nsum[2]   {\reflectbox{\displaystyle\sum_{\reflectbox{\scriptsize$#1$}}^{\reflectbox{\scriptsize$#2$}}}}

\newcommand\bink      {\binom{n}{k}}
\newcommand\setn      {\{a_i\}^{2n}_{i = 1}}
\newcommand\setc[1]   {\{a_i\}^{#1}_{i = 1}}

\newcommand\cupain    {\bigcup_{i = 1}^{n} A_i}
\newcommand\cupai[1]  {\bigcup_{i = 1}^{#1} A_i}
\newcommand\cupiiai   {\bigcup_{i \in I} A_i}
\newcommand\capain    {\bigcap_{i = 1}^{n} A_i}
\newcommand\capai[1]  {\bigcap_{i = 1}^{#1} A_i}
\newcommand\capiiai   {\bigcap_{i \in I} A_i}

\newcommand\xot       {x_{1, 2}}
\newcommand\ano       {a_{n - 1}}
\newcommand\ant       {a_{n - 2}}

% Other shortcuts
\newcommand\tl    {\tilde}
\newcommand\op    {^{-1}}

\newcommand\sof[1]    {\left | #1 \right |}
\newcommand\cl [1]    {\left ( #1 \right )}
\newcommand\csb[1]    {\left [ #1 \right ]}

\newcommand\bs    {\blacksquare}

%! ~~~ Document ~~~

\author{שחר פרץ}
\title{ליניארית 1א 4}
\begin{document}
	\maketitle
	\section{\en{Meaning of many solutions}}
	בשיעורים הקודמים דיברנו על שדה, מטריצות, ואז על משוואות ליניאריות. שמנו לב שכל מטריצה היא שקולה למטריצה מדורגת קאנונית. 
	היום נתבונן במערכות משוואות בהן קבוצת הפתרונות יכולה להיראות כמו $\{(4 - 3t, t)\}$, או $\{4, x, y\}$. נוכל להסתכל על זה כקבוצה על $\R^3$, שנצייר אותה כמישור על המרחב התלת ממדי. במקרה הזה, מישור שמקביל למישור של $y, z$. 
	ניזכר במשפט מהשיעור הקודם.
	\textbf{משפט. }בהינתן מערכות משוואות שיותר נעלמים ממשואות אז: 
	\begin{enumerate}
		\item אין פתרונות, או: 
		\item מספר הפתרונות לפחות $|\F|$. 
	\end{enumerate}
	
	\begin{proof}
		נחלק למקרים. תהי $A$ המטריצה המתאימה למערכת המשוואות. אזי קיימת $B$ שקולת שורות קאנונית ֱב־$A$ בעלת אותה מרחב פתרונות, ולכן נראה עבור $B$. אם שורה אחרונה שאיננה אפסים היא $0 \dots 1$, אז אין פתרונות (כי $0 \neq 1$). אחרת, קיימת שורה עם שני משתנים (משובך היונים – יש יותר משתנים ממשואות). כלומר, קיים משתנה חופשי. כעת, נראה $|\F|$ פתרונות. לכל $x \in \F$\ נבחן עבור המשתנה החופשי את הערך $x$ ונפתור – נוכל לפתור בכלל שכל משוואה היא מהצורה $x + \sum\alpha_ix_i = b$, סה"כ $||\F|$ פתרונות.
		
	\end{proof}
	
	\textbf{מסקנה. }בהינתן מערכת משוואות, אחד מהבאים מתקיים: 
	\begin{enumerate}
		\item אין פתרונות
		\item יש בדיוק פתרון אחד
		\item יש לפחות $|\F|$ פתרונות. 
	\end{enumerate}
	\begin{proof}
		נסתכל על $A$ קאנונית שקולה. אם יש שורה $(0, 0, \dots, 1)$ $\impliedby$ אין פתרון. 
		
		אחרת, אם יש משתנה חופשי $\impliedby$ לפחות $|\F|$ פתרונות. 
		
		אחרת, אין משתנה חופשי, והמטריצה מהצורה: 
		\[ \begin{pmatrix}
			1 & 0 & \cdots & 0 & b_1 \\
			0 & 1 & \cdots & 0 & b_2 \\
			\vdots && \ddots && \vdots \\
			0 & 0 & \cdots & 1 & b_n
		\end{pmatrix} \]
		והמשוואות הן $x_m = b_m$ ולכן פתרון יחיד. 
	\end{proof}
	
	\textbf{הגדרה. }מערכת משוואות שכל מקדמיה החופשיים הם $0$ היא \textit{מערכת הומוגנית}. [כלומר, ערכי ה־$b$ הן $0$]
	 
	 \textbf{הגדרה. }הפתרון $x_1 \dots x_n = 0$ הוא \textit{הפתרון הטרוויאלי}. [למערכת משוואות הומוגנית]
	
	\textbf{מסקנה(ות). }
	\begin{itemize}
		\item למערכת הומוגנית [המורה מסמן ב"הומו'"] שבה מספר נעלמים גדול מהמשוואות, יש $|\F|$ $>$ פתרונות. 
		\item למערכת משוואות הומו' יש רק פתרון טרוויאלי או $|\F|$ לפחות. 
	\end{itemize}
	
	 \section{\en{Intro. to Vectoric Fields}}
	 \textbf{הגדרה. }בהינתן $\F$ שדה, \textit{מרחב וקטורי} (לעיתים קרוי \textit{מרחב ליניארי}) הוא $\la V, \, a \colon V \times V \to V, \, m \colon \F \times V \to V \ra $ כאשר $a$ נקראן \textit{חיבור} ו־$m$ \textit{כפל בסקלר}, המקיים תכונות: 
	 \begin{enumerate}
	 	\item חילופיות חיבור 
	 	\item אסוציאטיביות חיבור
	 	\item קיום איבר אפס ניטרלי לחיבור (יסומן ב־$0_V$ ולפעמים בתור $0$)
	 	\item קיום נגדי לחיבור (נסמן אותו ה־$-v$ לכל $v \in V$). 
	 	\item דיטרבוטיביות 1: \hfill $\forall \lambda \in \F. u, v \in V \colon \lambda (u + v) = \lambda u + \lambda v$
	 	\item דיסטרבוטיביות 2: \hfill $\forall \lambda, \mu \in \F, v \in V \colon (\lambda + \mu) \cdot v = \lambda v + \mu v$
	 	\item אסוציאטיביות של כפל: \hfill $\forall \lambda, \mu \in \F, v \in V. (\lambda \mu)v = \lambda(uv)$
	 	\item לאיבר יחידה: \hfill $\forall v \in V. 1 \cdot v = v$
	 \end{enumerate}
	 \textbf{סימון. }$\lambda v = \lambda \cdot v = m(\lambda, v)$
	 
	 \subsection{דוגמאות}
	 \subsubsection{פתיחה}
	 נסתכל על מרחב ה־$\n$־יות מעל $\F^n$. נגדיר: 
	 \begin{align*}
	 	g, x \in \F^n \colon \, &x + y = (x_1 + y_1, \dots, x_n + y_n) \\
	 	& \lambda \cdot x = (\lambda x, \dots, \lambda x_N) \\
	 	&0_r = (0 \dots 0)
	 \end{align*}
	 
	 זה לא שדה כי אין הופכי. נאמר שכל $v \in V$ הוא \textit{וקטור}, ו־$c \in \F$ סקלר. 
	 \subsubsection{דוג' 2}
	 גם $M_{n \times n}(F)$ הוא מרחב וקטורי (הדבר הזה של המטריצות). 
	 \subsection{הרחבות}
	 \textbf{הגדרה. }בהינתן $w \subseteq V$, ו־$V$ מ"ו (מרחב וקטורי), אז $W$ \textit{תת מרחב וקטורי} ביחס לפעולות המושרות מ־$V$ אם: 
	 \begin{enumerate}[(A)]
	 	\item $0 \in W$
	 	\item $W$ שגורה לחיבור
	 	\item $W$ סגורה לכפל
	 \end{enumerate}
	 
	 \textbf{טענה. }תת מרחב וקטורי הוא מרחב וקטורי. 
	 
	 ניזכר בקבוצת הפתרונות $\{(x, y, 0)\}$ שראינו קודם. או ב־$\{(x, x, 0)\}$. הם כולם תת־מרחבים וקטורים של $\R^3$. ברור כי כל אחד מהצירים הוא גם מרחב וקטורי. 
	 
	 \textbf{משפט. }קבוצת הפתרונות של מערכת משוואות הומוגנית היא תת מרחב וקטורי ב־$\F^n$
	 
	 \textit{אינטואיציה: }נתבונן המערכת הבאה: 
	 \[ \begin{pmatrix}
	 	a & b & 0 \\
	 	d & c & 0
	 \end{pmatrix} \]
	 מרחב הפתרונות הוא $\sum x_i \alpha a_i = 0 $ לכל שורה. הוא סגור לחיבור; $ax_1 + by_1, ax_2, by_2$ נקודות $(x_1, y_1)$ וכו', נקבל $a(x_1 + x_2) + b(y_1 + y_2)$. 
	 
	 \subsection{המשך דוגמאות}
	 \subsubsection{דוג' 5}
	 תהי $A \neq 0$ קבוצה. נגדיר $Funct(a, f) = \{f \colon A \to f\}$. נסמנה ב־$T$. $T$ היא מרחב וקטורי מעל $F$ עם הפעולות: 
	 \[ (f_1 = f_2)(a) := f_1(a) + f_2(a), \ (\lambda) f)(a) := \lambda f(a), \ 0_V(a) = 0 \]
	 
	 \subsubsection{דוג' 6}
	 פולינומים במשתנה אחד היא: 
	 $F[x] = \{a_nx^n + \cdots a_1x_1 + a_0 \mid a_i \in \F\}$. גם הוא מרחב וקטורי. 
	 
	 הערה לגבי פולינומים: הוא לא תת קבוצה של $Funct(F, F)$ כי $x^p - x \equiv 0 \mod p$, הוא
	 $0_{Funct(\Z_p, \Z_p)}$
	 
	 (כלומר, זה לא עובד על $F$ סופי). 
	 
	 \textbf{טענה. }יהי $V$ מ"ו. 
	 \begin{enumerate}
	 	\item אם $\forall u, v, w \in V. u  + v = u + w \implies v = w$
	 	\item איבר האפס יחיד.
	 	\item לכל $v \in V$, נגדי יחיד, ונסמן בתור $(-v)$. 
	 \end{enumerate}
	 
	 \begin{proof}[הוכחה (1).]
	 	\[ v = v + 0 = v + w - w = u + w - w = u \]
	 	מותר להשתמש ב־$-w$ כי קיים הופכי, גם אם לא הוכחנו עדיין יחידות. 
	 \end{proof}
	 
	 \begin{proof}[הוכחה (2).]
	 	נוכיח שאיבר האפס יחיד. יהיו $w, w'$ איברי אפס. בכלל ש־$w'$ נטירלי, ו־$w$ ניטרלי, אז:
	 	\[ w = w + w' = w' \]
	 \end{proof}
	 
	 \begin{proof}[הוכחה (3).]
	 	יהי $V \ni v$. יהיו $m, w$ נגדיים. ידוע $m = w$ מטענה 1. 
	 	\[ v + m = 0 = v + w \]
	 \end{proof}
	 
	 \textbf{טענה. }יהי $V$ מרחב וקטורי:
	 \begin{enumerate}
	 	\item \hfil $\forall \lambda \in F \quad \lambda \cdot 0 v = 0_v$ \hfil 
	 	\item \hfil $\forall v \in V \quad 0 \cdot v = 0 $\hfil 
	 	\item \hfil $\lambda v = 0 \implies \lambda = 0 \lor v = 0_V$ \hfil
	 	\item \hfil $\forall v \in V \quad -v = (-1)v$ \hfil
	 \end{enumerate}
	 \begin{proof}
	 	\begin{enumerate}
	 		\item יהי $\lambda \in F$. יתקיים: 
	 		\[ (\lambda \cdot 0_V = \lambda (0_V + 0_V) = \lambda 0_v + \lambda 0_v) \implies (0 = \lambda 0_V) \]
	 		\item \[ 0_F v = (0_F + 0_F)v = 0_Fv + 0_Fv \implies (0 = 0_Fv) \]
	 		\item נניח $\lambda v = 0$. אם $\lambda = 0$, סיימנו. אחרת \[ v = \lambda\op \cdot \lambda 0 = \lambda\op 0 = 0 \]
	 		\item נראה $(-1)v + v = 0$. ואכן, 
	 		\[ -1 \cdot v + v = (-1 + 1) \cdot v = 0 \cdot v = 0 \]
	 		מתוך דיסטריבוטיביות. 
	 	\end{enumerate}
	 \end{proof}
	 
	 \section{\en{Developing on vectoric fields}}
	 \textbf{משפט. }יהי $V$ מ"ו מעל שדה $F$, ויהיו $U, W \subseteq V$ תמ"ו של $V$. אזי, $U \cap W$ תמ"ו, ו־$U \cup W$ תמ"ו אמ"מ $U \subseteq W \lor W \subseteq U$. 
	 \begin{proof}\, 
	 	\begin{enumerate}
	 		\item נראה ש־$T:= U \cap W$ מקיים תחונות תמ"ו. ואכן $0 \in W, 0 \in U \implies 0 \in W \cap U$. בנוסף $\forall x, y \in Y. x  + y \in V \land x + y \in W \implies x + y \in W \cap U$, כי $x, y \in W, x, y \in V$
	 		
	 		יהי $\lambda \in F$, $x \in T$. אזי: 
	 		\[ \lambda x \neq 0, \lambda x \in w \implies \lambda x \in U \cap W = T \]
	 		\item מכיוון אחד. בה"כ $U \subseteq W$, אז $T:= V \cup W = W$ שהוא תמ"ו. מהכיוון השני, נניח $U \cup W$ תמ"ו. אם $U \subseteq W$, סיימנו. אחרת, $U \nsubseteq W$. אזי $\exists u \in U. u \neq w$. יהי $w \in W$. מסגירות, $u + w \in T$. נראה כי $u + w \notin W$. נניח בשלילה זאת, נקבל $u + w - w \in W$ כי $W$ סגור לחיבור. סה"כ $u \in W$, וזו סתירה. לכן, $u + w \in U$. נגרר $u + w - u \in U$, מסגירות. לכן $w \in U$ ולכן $U \subseteq W$ כדרוש. 
	 	\end{enumerate}
	 \end{proof}
	 
	 \textbf{הגדרה. }$V$ מ"ו מעל $F$. $Vת W \in V$ תמ"ו. נגדיר $U + W = \{u + w \mid u \in V, w \in W\}$, אם $\{0\} = U \cap W$, נאמר שסכום זה הוא סכום ישר ונסמן $V \oplus W$. 
	 
	 \textbf{משפט. }$V$ מו מעל שדה $F$, $U, W \subseteq V$ תמ"ו. נגרר $ U + W $ תמ"ו של $V$. 
	 \begin{proof}
	 	נדע $0 \in V  + W$ כי: 
	 	\[ 0_V + 0_W = 0 \]
	 	סגירות מתקיימת: קיימים $u_1, u_2 \in U$ ו־$w_1, w_2 \in W$ מתאימים כך ש־:
	 	\[ \forall x, y \in U + W. x = u_1 + w_1, y = u_2 + w_2 \implies x + y = \underbrace{(u_1 + u_2)}_{\in U} + \underbrace{(w_1 + w_2)}_{\in W} \]
	 	
	 	וגם: 
	 	\[ \exists u '\in U, w' \in W \quad x + y = u' + w' \]סגורות לסלקר: 
	 	\[ \lambda \in F, u + w = x \in U + w \implies \lambda x = \lambda (u + w) = \underbrace{\lambda u}_{\in U} + \underbrace{\lambda w}_{\in W} \]
	 \end{proof}
	 
	 \textbf{טענה. }$V$ מ"ו מעל $F$, $U, W \subseteq V$. אזי $V + W$ סכום ישר אמ"מ כל וקטור בסכום ניתן להגדיר בצורה יחידה ע"י וקטור מ־$U$ ווקטור מ־$W$. 
	 
	 \begin{proof}\, 
		\begin{enumerate}
			\item [$\implies$] יהי $v \in V + W$. נניח $u' + w ' = v, \ u + w = v$ כך ש־$u', u \in U, \ w, w' \in U$. מההנחה, 
			\[ u' + w' = v = u + w \implies U \ni u - u' = w - w' \in W \]				לכן: 
				\[ u - u' = w - w' \in W \cap U = \{0\} \]
				ולכן $u = u', \ w = w'$. 
				
			\item [$\impliedby$] יהי $x \in U \cap W$, נוכיח $x = 0$. נשים לב: 
			\begin{align*}
				x = x + 0, \ x \in U, & \ 0 \in W \\
				x = 0 + x, \ x \in W, & \ 0 \in U
			\end{align*}
			
			מהיחידות נובע ש־$x = 0 \land 0 = x \implies x = 0$. 
		\end{enumerate}
	\end{proof}
	
	\section{\en{Base and dimenssion}}
	\textbf{הגדרה. }$\Z \ni S \ge 0$, וקטורים $v_1 \dots v_s \in V$  וסקלרים $\lambda_1 \dots \lambda_s \in \F$. \textit{הצירוף הליניארי} של $v_1 \dots v_s$ עם המקדמים $\lambda_1 \dots \lambda_s$ וה הוקטור: 
	\[ \sum_{i = 0}^{s}\lambda_iv_i = \lambda_1v_1 + \cdots + \lambda_sv_s \]
	אם $\lambda_i = 0$ לכל $1 \le i \le s$ אז זהו \textit{צירוף טרוויאלי}. אם $s = 0$ הצירוף הוא $0$. 
	
	\textbf{הגדרה. }יהי $B = (v_1 \dots v_s) \in V^s$. $V$ מ"ו. אז $B$ \textit{בסיס} אם לכל $v \in V$ קיים ויחיד צירוף ליניארי מהוקטורים ב־$B$. כלומר: 
	\[ \exists! \lambda _1 \dots \lambda_s \in F \colon v = \sum x_i\lambda_i \]
	אפשר גם להגדיר למרחבים אינסופיים. נראה הרחבה כזו בהמשך הקורס. 
	
	\textbf{הגדרה. }$e_i \in F^n$ מוגדר להיות $e_i \equiv (0, \dots 1, \dots 0)$ כאשר $1$ רק בקורדינאטה ה־$i$. נאמר ש־$\{e_1 \dots e_n\}$ הוא \textit{הבסיס הסטנדרטי}. 
	
	\textbf{משפט. }$V$ מ"ו מעל שדה $F$ ו־$V$ בסיס, אם יש לו בסיס $B$. אם $B_1, B_2$ בסיסים, אז $|B_1| = |B_2|$. 
	
	\textbf{הגדרה. }בעבור $V$ מ"ו עם בסיס $B$, $\mathrm{dim}V := |B|$ (מוגדר היטב מהמשפט). בשיעור הבא נוכיח את המשפט. 
\end{document}