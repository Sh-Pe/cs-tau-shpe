\documentclass[]{article}

% Math packages
\usepackage[usenames]{color}
\usepackage{ifxetex,ifluatex,amsmath,amssymb,mathrsfs,amsthm,witharrows}
\renewcommand{\qedsymbol}{$\blacksquare$} % end proofs with \blacksquare. Overwrites the defualts. 
\usepackage{cancel,bm}

% Deisgn
\usepackage[legalpaper, margin=0.3in]{geometry}
\usepackage[skip=4pt, indent=0pt]{parskip}
\usepackage[normalem]{ulem}
\renewcommand\labelitemi{$\bullet$}

% Hebrew initialzing
\usepackage{polyglossia}
\setmainlanguage{hebrew}
\setotherlanguage{english}
\newfontfamily\hebrewfont[Script=Hebrew]{David CLM}

% Math shortcuts

\newcommand\N     {\mathbb{N}}
\newcommand\Z     {\mathbb{Z}}
\newcommand\R     {\mathbb{R}}
\newcommand\Q     {\mathbb{Q}}

\newcommand\ml    {\ell}
\newcommand\mj    {\jmath}
\newcommand\mi    {\imath}

\newcommand\powerset {\mathcal{P}}
\newcommand\ps    {\mathcal{P}}
\newcommand\pc    {\mathcal{P}}
\newcommand\ac    {\mathcal{A}}
\newcommand\bc    {\mathcal{B}}
\newcommand\cc    {\mathcal{C}}
\newcommand\dc    {\mathcal{D}}
\newcommand\ec    {\mathcal{E}}
\newcommand\fc    {\mathcal{F}}
\newcommand\nc    {\mathcal{N}}

\newcommand\siff  {\longleftrightarrow}
\newcommand\ssiff {\leftrightarrow}
\newcommand\so    {\longrightarrow}
\newcommand\sso   {\rightarrow}

\newcommand\epsi  {\epsilon}
\newcommand\vepsi {\varepsilon}
\newcommand\vphi  {\varphi}

\newcommand\Neven {\N_{\mathrm{even}}}
\newcommand\Nodd  {\N_{\mathrm{odd }}}
\newcommand\Zeven {\Z_{\mathrm{even}}}
\newcommand\Zodd  {\Z_{\mathrm{odd }}}
\newcommand\Np    {\N_+}

\newcommand\open  {\big(}
\newcommand\qopen {\quad\big(}
\newcommand\close {\big)}
\newcommand\also  {\text{, }}
\newcommand\defi  {\text{ definition}}
\newcommand\defis {\text{ definitions}}
\newcommand\given {\text{given }}
\newcommand\case  {\text{if }}
\newcommand\syx   {\text{ syntax}}
\newcommand\rle   {\text{ rule}}
\newcommand\other {\text{else}}
\newcommand\set   {\ell et \text{ }}

\newcommand\ra    {\rangle}
\newcommand\la    {\langle}

\newcommand\oto   {\leftarrow}

\newcommand\QED   {\quad\quad\mathscr{Q.E.D.}\;\;\blacksquare}
\newcommand\QEF   {\quad\quad\mathscr{Q.E.F.}}
\newcommand\eQED  {\mathscr{Q.E.D.}\;\;\blacksquare}
\newcommand\eQEF  {\mathscr{Q.E.F.}}
\newcommand\jQED  {\mathscr{Q.E.D.}}

\newcommand\dom   {\text{dom}}
\newcommand\Img   {\text{Im}}
\newcommand\range {\text{range}}

\newcommand\trio  {\triangle}

\newcommand\rc    {\right\rceil}
\newcommand\lc    {\left\lceil}
\newcommand\rf    {\right\rfloor}
\newcommand\lf    {\left\lfloor}

\newcommand\lex   {<_{lex}}

\newcommand\bs    {\blacksquare}

\newcommand\az    {\aleph_0}
\newcommand\anz   {\aleph}
\newcommand\azz   {2^{\aleph_0}}

\newcommand\n     {$n$־יה\ }

\title{מתמטיקה בדידה – עוצמות 4}
\author{שחר פרץ}
\date{6 למרץ 2024}

\begin{document}
	\maketitle
	
	\subsection*{תזכורות: }
	\begin{itemize}
		\item $ 2^{\az} = |\N \to \{0, 1\}| $
		\item $ 2^{|A|} := |A \to \{0, 1\}|$
		\item $ \az \le \azz $
		\item $ \forall A. |\ps(A)| = 2^{|A|} $ ובפרט עבור $ |\ps(\N)| = 2^{|\N|} $
		\item משפט קנטור: $ \forall A. |A| < |\ps(A)| $
		\item $ \anz := |\R| $
		\item $ \anz = \azz $
	\end{itemize}
	
	הוכחה למשפט האחרון מהרשימה: נשתמש בקש"ב. בכיוון הראשון, $ |\R| \le |\ps(\Q)| $ ע"י הגדרת הפונקציה: 
	\[ f \colon \R \to \ps(\Q), \ f(r) = \{q \in \Q \mid q \le r\} \]
	נוכיח ש־$ f $ חח"ע. יהיו $ r_1, r_2 \in \R  $ שונים. בה"כ $ r_1 < r_2 $. מצפיפות הרציונליים בממשיים, קיים $ q_0 \in \Q $ כך ש־$ r_1 < q_0 < r_2 $. לכן, $ q_0 \in \{q \in \Q \mid q \le r_2\} = f(r_2)$ וגם $ q \neq \{q \in \Q\mid q \le r_1\}  = f(r_1) $ וסה"כ $ q_0 \in f(r_2) \setminus f(r_1) $ ולכן $ f(r_1) \neq f(r_2) $ וסה"כ $ f $ חח"ע. [להוסיף בהוכחה יותר פורמלית: $ \ps(\Q) = \ps(\N) = \azz$]
	
	בכיוון השני: צ.ל. $ \azz \le |\R| $. ניעזר בתכונה הבאה של $ \R $: לכל סדרה $ \{a_n \mid n \in \N\} $ של ספרות $ \{0, \dots, 9\} $, קיים מספר ממשי שהוא $ 0.a_0a_1a_2\dots $ (זה אולי נשמע טרוויאלי, אבל זה לא נכון על $ \Q $ לדוגמה). מתקיים לכל $ n \in \N $: אם $ a_n < 9 $ אז $ 0.a_0a_1 \dots a_n \dots \le 0.a_0 \dots (a_n + 1) $ (דוגמה: $ 0.134794... \le 0.1348 $). (למה נכתב $ \le $ ולא $ > $? יתכן שוויון עבור $ 0.09999\dots = 0.1 $ [כל הבלגן הזה כדי להמנע מלהכנס להגדרה של ממשיים, כמו שעשינו עם ארז]. \\
	נגדיר פונקציה $ g \colon (\N \to \{0, 2\}) \to \R $ ע"י $ g = \lambda h \in \N \to \{0, 1\}.                                                                                                   0.h(0)h(1)h(2) \dots $. לפי התכונה הנ"ל. הטווח של $ g $ הוא אכן $ \R $. נוכיח ש־$ g $ חח"ע. יהיו $ h_1, h_2 \in \N \to \{0, 2\} $. אז קיים $ n_0 \in \N $ כך ש־$ h_1(n_0) \neq h_2(n_0) $. בה"כ נניח ש־$ h_1(n_0) = 0 \land h_2(n_0) = 2 $. אז: 
	\[ g(h_1) = 0.h_1(0) \dots \underbrace{h_1(n_0)}_{=0} \dots < 0.h_1(0) \dots 1 < 0.h_1(0) \dots 2 = 0.h_2(0) \dots 2 \le g(h_2) \]
	זאת בהנחה שבחרנו את ה־$ n_0 $ הקטן ביותר האפשרי, כך ש־$ \forall n < n_0. h_1(n) = h_2(n) $. סה"כ $ g(h_1) < g(h_2) $ ולכן $ g $ חח"ע. 
	
	\textbf{"בואו נוכיח, ונעשה שיהיה לנו פשוט"} (נטלי). לא ההוכחה הכי פורמלית...
	
	סה"כ, "הוכחנו" ש־$ \azz \le |R| $, ולסיכום מקש"ב $ \anz = \azz $. 
	
	\subsection*{הערה: }
	דוגמאות לקבוצות מעוצמה $ \azz $: 
	\begin{itemize}
		\item $ \N \to \{0, 1\}, \Q\to \{0, 1\} $ וכו'
		\item $ \R $
		\item $ \ps(\N), \ps(\Q), \ps(\Z), \ps(\N \times \N), \ps(\Neven) $ וכו'
	\end{itemize}
	
	\subsection*{הערה 2: }
	השערת הרצף: לא קיימת קבוצה $ X $ כך ש־$ \az < |X| < \azz $ (במילים אחרות, $ \azz $ היא העוצמה הראשונה שגדולה ממש מ־$ \az $). 
	השערת הרצף התחילה בתור השערה, אבל הוא לא באמת השערה בימנו אנו: הוכיחו כי הטענה הזו בלתי תלויה באקסיומות, כלומר ש\textbf{אי אפשר להוכיח אותה באמצעות ZFC אבל גם אי־אפשר להפריכה}. עושים את זה באמצעות כלים מתמטיים יחסית מתקמים. 
	\subsection*{הערה 3: }
	לא בחומר: $ \anz_1 $ היא העוצמה הראשונה שגדולה יותר מ־$ \az $
	\section*{קצת תרגול בלכסון}
	\subsection*{א}
	הוכיחו ע"י לכסון שהקבוצה הבאה אינה בת מנייה: 
	$ A = \{f \in \N \to \{0, 1\} \mid \forall i \in \N. f(i) \cdot f(i + 1) = 0\} $ \\
	פתרון: נניח בשלילה ש־$ |A| = \az $. אז קיים זיווג $ F \colon \N\to A $. נבנה $ g \in A $ באופן הבא: 
	\[ g = \lambda n \in \N. \begin{cases}
		1 - F\left (\tfrac{n}{2} \right )(n) &n \in \Neven \\
		0 &n \in \Nodd
	\end{cases} \]
	נוכיח ש־$ g \in A $: ראשית, לכל $ n \in \Neven $ מתקיים $ f(\tfrac{n}{2}) \in \N \to \{0, 1\} $ ולכן $ F(n/2)(n) \in \{0, 1\} $ ולכן $ g(n) = 1 - f(n/2)(n) \in \{0, 1\} $, ולכל $ n \in \Nodd $ הגדרנו $ g(n) = 0 $, ולכן $ \{0, 1\} $ הוא טווח עבור $ g $, כלומר $ g \in \N \to \{0, 1\} $. בנוסף, לכל $ i \in \N $ מתקיים שאחד מבין $ i, i + 1 $ הוא אי זוגי, ולכן $ g(i) = 0 \lor g( i + 1 ) = 0 $ אז $ g(i) \cdot g(i + 1) = 0 $. סה"כ $ g \in A $. עכשיו נוכיח ש־$ \forall m \in \N. g \neq F(m) $. יהי $ m \in \N $. צ.ל. $  \n in \N $ כך ש־$ g(n) \neq F(m)(n)) $. נבחר $ n = 2m $. אז $ g(2m) = 1 - F(2m / 2)(2m) = 1 - F(m)(2m) \neq F(m(2m))$. לכן $ g(2m) \neq F(m)(2m) $ ולכן $ G \neq F(n)/ $. סה"כ $ g \not\in \Img F $ אך כן $ g \in A $ ולכן $ |A| \neq \az $
	\subsection*{ב}
	הוכיחו ע"י לכסון שקבוצת כל הפונקציות החח"ע מ־$ \N \to \N $ (נסמנה $ A $) אינה בת מנייה. 
	פתרון: נניח בשלילה $ F \colon \N \to A $ זיווג. נגדיר: 
	\[ g = \lambda n \in \N. \sum_{i = 0}^{n} (F(i)(i) + 1) + 1 \]
	(מי שלא היה: היא הסבירה הרבה על הפונקציה הזו בע"פ, אז תכתבו לי אם לא ברור למה זה עובד. בכללי זה מונוטוני עולה אז זה חח"ע)
\end{document}