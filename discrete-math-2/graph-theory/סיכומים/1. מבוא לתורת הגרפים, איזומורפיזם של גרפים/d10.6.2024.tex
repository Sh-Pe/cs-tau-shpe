\documentclass[]{article}

% Math packages
\usepackage[usenames]{color}
\usepackage{forest}
\usepackage{ifxetex,ifluatex,amsmath,amssymb,mathrsfs,amsthm,witharrows,mathtools}
\WithArrowsOptions{displaystyle}
\renewcommand{\qedsymbol}{$\blacksquare$} % end proofs with \blacksquare. Overwrites the defualts. 
\usepackage{cancel,bm}

% tikz
\usepackage{tikz}
\newcommand\sqw{1}
\newcommand\squ[4][1]{\fill[#4] (#2*\sqw,#3*\sqw) rectangle +(#1*\sqw,#1*\sqw);}


% code 
\usepackage{listings}
\usepackage{xcolor}

\definecolor{codegreen}{rgb}{0,0.35,0}
\definecolor{codegray}{rgb}{0.5,0.5,0.5}
\definecolor{codenumber}{rgb}{0.1,0.3,0.5}
\definecolor{deepblue}{rgb}{0,0,0.5}
\definecolor{deepred}{rgb}{0.5,0.03,0.02}

\lstdefinestyle{pythonstylesheet}{
	language=Python,
	morekeywords={}
	emphstyle=\color{deepred},
	backgroundcolor=\color{white},   
	commentstyle=\color{codegreen}\itshape,
	keywordstyle=\color{deepblue}\bfseries\itshape,
	numberstyle=\tiny\color{codenumber},
	basicstyle=\ttfamily\footnotesize,
	breakatwhitespace=false, 
	breaklines=true, 
	captionpos=b, 
	keepspaces=true, 
	numbers=left, 
	numbersep=5pt, 
	showspaces=false,                
	showstringspaces=false,
	showtabs=false, 
	tabsize=2, 
	morekeywords={object,type,isinstance,copy,deepcopy,zip,enumerate,reversed,list,set,len,dict,tuple,range,xrange,append,execfile,real,imag,reduce,str,repr},              % Add keywords here
	keywordstyle=\color{deepblue},
	emph={__init__,__add__,__mul__,__div__,__sub__,__call__,__getitem__,__setitem__,__eq__,__ne__,__nonzero__,__rmul__,__radd__,__repr__,__str__,__get__,__truediv__,__pow__,__name__,__future__,__all__,as,assert,nonlocal,with,yield,self,True,False,None},          % Custom highlighting
	emphstyle=\color{deepred},
	stringstyle=\color{deepgreen},
	showstringspaces=false
}
\newcommand\pythonstyle{\lstset{pythonstylesheet}}
\newcommand\pyl[1]     {{\pythonstyle\lstinline!#1!}}
\lstset{style=pythonstylesheet}


% Deisgn
\usepackage[labelfont=bf]{caption}
\usepackage[margin=0.6in]{geometry}
\usepackage{multicol}
\usepackage[skip=4pt, indent=0pt]{parskip}
\usepackage[normalem]{ulem}
\forestset{default}
\renewcommand\labelitemi{$\bullet$}
\usepackage{titlesec}
%\titleformat{\section}[block]
%	{\fontsize{15}{15}}
%	{\sen \dotfill \, \!\!\! \thesection \,\! \dotfill \she}
%	{1em}
%	{\MakeUppercase}

% Hebrew initialzing
\usepackage{polyglossia}
\setmainlanguage{hebrew}
\setotherlanguage{english}
\newfontfamily\hebrewfont[Script=Hebrew, Ligatures=TeX]{David CLM}
\usepackage[shortlabels]{enumitem}
\newlist{hebenum}{enumerate}{1}
\setlist[hebenum,1]{
	labelindent=\parindent,
	label={{\hebrewfont{\protect\hebrewnumeral{\value{hebenumi}}}}.}
}

% Language Shortcuts
\newcommand\en[1] {\selectlanguage{english}#1\selectlanguage{hebrew}}
\newcommand\sen   {\selectlanguage{english}}
\newcommand\she   {\selectlanguage{hebrew}}
\newcommand\del   {$ \!\! $}
\newcommand\ttt[1]{\en{\texttt{#1}}}

\newcommand\npage {\vfil {\hfil \textbf{\textit{המשך בעמוד הבא}}} \hfil \vfil}
\newcommand\ndoc  {\dotfill \\ \vfil \hfil \textbf{\textit{שחר פרץ, 2024}} \hfil \vfil}

\newcommand{\rn}[1]{
	\textup{\uppercase\expandafter{\romannumeral#1}}
}


%! ~~~ Math shortcuts ~~~

% Letters shortcuts
\newcommand\N     {\mathbb{N}}
\newcommand\Z     {\mathbb{Z}}
\newcommand\R     {\mathbb{R}}
\newcommand\Q     {\mathbb{Q}}
\newcommand\C     {\mathbb{C}}

\newcommand\ml    {\ell}
\newcommand\mj    {\jmath}
\newcommand\mi    {\imath}

\newcommand\powerset {\mathcal{P}}
\newcommand\ps    {\mathcal{P}}
\newcommand\pc    {\mathcal{P}}
\newcommand\ac    {\mathcal{A}}
\newcommand\bc    {\mathcal{B}}
\newcommand\cc    {\mathcal{C}}
\newcommand\dc    {\mathcal{D}}
\newcommand\ec    {\mathcal{E}}
\newcommand\fc    {\mathcal{F}}
\newcommand\nc    {\mathcal{N}}
\newcommand\kc    {\mathcal{K}}
\newcommand\sca   {\mathcal{S}} % \sc is already definded
\newcommand\rca   {\mathcal{R}} % \rc is already definded

\newcommand\Si    {\Sigma}

% Logic & sets shorcuts
\newcommand\siff  {\longleftrightarrow}
\newcommand\ssiff {\leftrightarrow}
\newcommand\so    {\longrightarrow}
\newcommand\sso   {\rightarrow}

\newcommand\epsi  {\epsilon}
\newcommand\vepsi {\varepsilon}
\newcommand\vphi  {\varphi}
\newcommand\Neven {\N_{\mathrm{even}}}
\newcommand\Nodd  {\N_{\mathrm{odd }}}
\newcommand\Zeven {\Z_{\mathrm{even}}}
\newcommand\Zodd  {\Z_{\mathrm{odd }}}
\newcommand\Np    {\N_+}

% Text Shortcuts
\newcommand\open  {\big(}
\newcommand\qopen {\quad\big(}
\newcommand\close {\big)}
\newcommand\also  {\text{, }}
\newcommand\defi  {\text{ definition}}
\newcommand\defis {\text{ definitions}}
\newcommand\given {\text{given }}
\newcommand\case  {\text{if }}
\newcommand\syx   {\text{ syntax}}
\newcommand\rle   {\text{ rule}}
\newcommand\other {\text{else}}
\newcommand\set   {\ell et \text{ }}
\newcommand\ans   {\mathit{Ans.}}

% Set theory shortcuts
\newcommand\ra    {\rangle}
\newcommand\la    {\langle}

\newcommand\oto   {\leftarrow}

\newcommand\QED   {\quad\quad\mathscr{Q.E.D.}\;\;\blacksquare}
\newcommand\QEF   {\quad\quad\mathscr{Q.E.F.}}
\newcommand\eQED  {\mathscr{Q.E.D.}\;\;\blacksquare}
\newcommand\eQEF  {\mathscr{Q.E.F.}}
\newcommand\jQED  {\mathscr{Q.E.D.}}

\newcommand\dom   {\text{dom}}
\newcommand\Img   {\text{Im}}
\newcommand\range {\text{range}}

\newcommand\trio  {\triangle}

\newcommand\rc    {\right\rceil}
\newcommand\lc    {\left\lceil}
\newcommand\rf    {\right\rfloor}
\newcommand\lf    {\left\lfloor}

\newcommand\lex   {<_{lex}}

\newcommand\az    {\aleph_0}
\newcommand\uaz   {^{\aleph_0}}
\newcommand\al    {\aleph}
\newcommand\ual   {^\aleph}
\newcommand\taz   {2^{\aleph_0}}
\newcommand\utaz  { ^{\left (2^{\aleph_0} \right )}}
\newcommand\tal   {2^{\aleph}}
\newcommand\utal  { ^{\left (2^{\aleph} \right )}}
\newcommand\ttaz  {2^{\left (2^{\aleph_0}\right )}}

\newcommand\n     {$n$־יה\ }

% Math A&B shortcuts
\newcommand\logn  {\log n}
\newcommand\cosx  {\cos x}
\newcommand\cost  {\cos \theta}
\newcommand\sinx  {\sin x}
\newcommand\sint  {\sin \theta}
\newcommand\tanx  {\tan x}
\newcommand\tant  {\tan \theta}
\newcommand\dx    {\,\mathrm{d}x}

\newcommand\seq   {\overset{!}{=}}
\newcommand\sle   {\overset{!}{\le}}
\newcommand\sge   {\overset{!}{\ge}}
\newcommand\sll   {\overset{!}{<}}
\newcommand\sgg   {\overset{!}{>}}

\newcommand\h     {\hat}
\newcommand\ve    {\vec}
\newcommand\lv    {\overrightarrow}
\newcommand\ol    {\overline}

\newcommand\mlcm  {\mathrm{lcm}}

\newcommand\limz  {\lim_{x \to 0}}
\newcommand\limxz {\lim_{x \to x_0}}
\newcommand\limi  {\lim_{x \to \infty}}
\newcommand\limni {\lim_{x \to - \infty}}
\newcommand\limpmi{\lim_{x \to \pm \infty}}

\newcommand\ta    {\theta}
\newcommand\ap    {\alpha}

\renewcommand\inf {\infty}
\newcommand  \ninf{-\inf}

% Combinatorics shortcuts
\newcommand\sumnk     {\sum_{k = 0}^{n}}
\newcommand\sumni     {\sum_{i = 0}^{n}}
\newcommand\sumnko    {\sum_{k = 1}^{n}}
\newcommand\sumnio    {\sum_{i = 1}^{n}}
\newcommand\sumai     {\sum_{i = 1}^{n} A_i}
\newcommand\nsum[2]   {\reflectbox{\displaystyle\sum_{\reflectbox{\scriptsize$#1$}}^{\reflectbox{\scriptsize$#2$}}}}

\newcommand\bink      {\binom{n}{k}}

\newcommand\cupain    {\bigcup_{i = 1}^{n} A_i}
\newcommand\cupai[1]  {\bigcup_{i = 1}^{#1} A_i}
\newcommand\cupiiai   {\bigcup_{i \in I} A_i}

\newcommand\sof[1]    {\left | #1 \right |}
\newcommand\cl [1]    {\left ( #1 \right )}

\newcommand\xot       {x_{1, 2}}
\newcommand\ano       {a_{n - 1}}
\newcommand\ant       {a_{n - 2}}

% Other shortcuts
\newcommand\tl    {\tilde}
\newcommand\op    {^{-1}}

\newcommand\bs    {\blacksquare}

%! ~~~ Document ~~~

\author{שחר פרץ}
\title{בדידה 9 $\sim$ נטלי שלום $\sim$ מבוא לתורת הגרפים}
\date{10 ביוני 2024}

\begin{document}
	\maketitle
	
	\section{מבוא לתורת הגרפים}
	\subsection{הגדרות}
	\textbf{הגדרה: }גרף, הוא זוג סדור $G = \la V, E \ra$ כך ש־$V$ קבוצה סופית ולא ריקה שאבריה נקראים צמתיים/קודקודים (Nodes/Vertices) \del, ו־$E$ היא קבוצה של זוגות של קודקודים שאיבריה נקראים קשתות (Edges) \del. לפעמים, קשתות נקראות צלעות. 
	
	\textbf{דוגמה: }
	\[ G = \la V, E \ra, \quad V = \{x, y, z, w\}, \quad E = \{\{x, y\}, \{z, x\}, \{y, z\}, \{y, w\}\} \]
	ציור לגרף: 
	\begin{forest}
		[y [x] [z] [w]]
	\end{forest}
	בשילוב עם עוד קו בין $x$ ל־$z$ שאין לי זמן לצייר כי אני לא ממש בדקתי איך להקליד גרפים וכך או אחרת אין לי זמן בהרצאה. 
	
	\subsubsection{גרף מכוון ולא מכוון}
	\textbf{בגרף לא מכוון}הקדתות הן קבוצות בגודל $2$ של קודקודים, ובגרף מכוון, הקשתות הן זוגות סדורים של הוקדקודים. נצייר עם חץ בקצה. 
	
	\textit{\textbf{בקורס הזה, אנחנו עוסקים רק בגרפים לא מכוונים} (אלא אם נאמר אחרת אבל לא יאמר אחרת)}. 
	
	\subsubsection{הגדרות נוספות}
	עבור קשת $e = \{u, v\}$ נאמר שהקשת $e$ \textbf{נוגעת} ב־$u$ וב־$v$, ונאמר ש־$u$ ו־$v$ הם \textbf{השכנים} (סמוכים). 
	
	לא נעסוק עם גרפים בהם יש קשת לקודקוד לעצמו (לולאה), או בהם יש יותר מקשת אחת בין זוג קדקודים (קשתות מקבילות). גרף בלי לואות או קצוות מקבילות, נקרא גרף פשוט. נעבווד רק על גפרים פשוטים ונקרא להם גרפים. 
	
	\textbf{סימון: }עבור הקודקוד $v$, מסמנים ב־$N(v)$ את קבוצת שכניו. לדוגמה, בקבוצה למעלה, $N(y) = \{x, z, w\}$. 
	
	\textbf{הדרגה}של קודקודים $v$ היא $d(v) = \sof{N(v). }$, לדוגמה, בדוגמה לעיל $d(z) = 2, \ d(u) = 0$. 
	
	\subsection{משפטים}
	\textbf{משפט: }
	\[ \sum_{v \in V}d(v) = 2 \cdot |E| \]
	\textbf{הוכחה: }כל קשת $\{u, v\} \in E$ נספרת גם ב־$d(u)$ וגם ב־$d(v)$, ולכן סכום הדרגות הוא פעמיים מספר הקשתות. 
	
	\textbf{מסקנה: }בגרף יש מספר זוגי של קודקודים בעלי דרגה אי־זוגית. 
	
	\textbf{למת לחיצות הידיים: }\textit{(שקול למסקנה)} יש בעולם מספר זוגי של אנשים, שבמהלך חייהם לחצו ידיים למסר אי־זוגי של אנשים. 
	
	\textbf{משפט: }בכל דרף בעל לפחות $2$ קודקודים, קיימים שני קודקודים בעלי אותה הדרגה. 
	\begin{proof}
		נניח בשלילה שלא קיימים כאלה. נסמן $n = |V|$. ישנם $n$ קודקודים, ולפי הנחת השלילה, יש $n$ דרגות שונות בגרף, שהן בהכרח $0, 1, \dots, n - 1$ (כי זה תחום ההגדרה של דרגות – הקטה ביתור הפאשרית היא $0$, והגדולה ביותר האפשרית בגרף עם $n$ קודקודים היא $n - 1$). קיבלנו שיש קודקוד שמחובר לכולם, וקודקוד אחר שלא מחובר לאף אחד, סתירה. 
	\end{proof}
	\subsection{תרגילים}
	\subsubsection{}
	
	\textbf{שאלה: }מה מספר הקשתות המסקימלי האפשרי בגרף עם $n$ קודקודים?
	
	\textbf{תשובה: }\textit{הסבר ראשון; }בגרף מלא, בו הכל מחובר להכל, יש קשת בין כל זוג קודקודיםם, מספר זוגות הקודקודים המקסימלי מתוך $n$ הוא $\binom{n}{2} = \frac{n(n - 1)}{2}$. \textit{הסבר שני; }נתבונן בקודקוד כלשהו. הוא יהיה מחובר לכולם, ודרגתו $n - 1$. הבא, כבר חושבה קשת אחת עליו ולכן נקבל $n - 2$ קשתות חדשות, סה"ר נקבל את הסכום $(n -1) + (n - 2) + \dots + 1$ שזה אותו הדבר לפי סכום סדרה חשבונית. ההנחה היא שאתם יודעים את הנוסחה הזו. \textit{דרך 3/}כל קודקוד הוא מדרגה $n - 1$ ושינם $n$ קודקודים, לכן ממשפט על סכום הדרגות נקבל: 
	\[ 2 \cdot |E| = \sum_{v \in V} d(v) = n(n - 1) \implies |E| = \frac{n(n - 1)}{2} \]
	
	\subsubsection{}
	\textbf{שאלה: }כמה גרפים קיימים בעלי $n$ קודקודים?
	
	\textbf{תשובה: }$2^{\binom{n}{2}}$
	 – עבור כל קשת, נבחר האם היא תהיה בגרף או לא. 
	 \subsection{חזרה להגדרות נוספות}
	 \textbf{הגדרה: }תת־גרף $G'$ של $G = \la V, E \ra$ הוא גרף שמקיים $G' = \la V', E' \ra, \ V' \subseteq V, \ E' = \{\{u, v\} \in E \mid u, v \in V'\}$, כלומר, לוקחים חלק מהקודקודים ב־$V$ ואת \textit{כל} הקשתות שנוגעות בקודקודים הללו. 
	 
	 \textbf{הגדרה: }הגדף המשלים של גרף $G = \la V, E \ra$ הוא גרף $\ol G = \la V, \ol E \ra$ כך ש־$\ol E := \ps_2(V) \setminus E$ (כאשר $\ps_2(V) = \{X \in \ps(V) \mid |X| = 2\}$). כלומר, קשת נמצאת בגרף המשלים אמ"מ היא לא נמצאת בגרף המקורי. 
	 
	 \section{איזומורפיזם של גרפים}
	 \subsection{הקדמה}
	 <הכנס גרף שלא ציירתי כאן>
	 
	 \textbf{הגדרה: }שני גרפים $G_1 = \la V_1, E_1 \ra, \ G_2 = \la V_2, E_2 \ra$ נקראים \textit{איזומורפיים}, אמ"מ קיים זיווג $f \colon V_1 \to V_2$ כך שלכל $a, b \in V_1$ יתקיים $\{a, b\} \in E_2 \iff \{f(a), f(b)\} \in E_2$. $f$ כזו נקראת \textit{איזומורפיזם}. 
	 
	 \textit{הערה: }קיום איזומורפיזם בין גרפים הוא יחס שקילות. אפשר לחשוב על כל מחלקת שקילות, בתוך גרף ללא שמות לצמתים. 
	 \subsection{דוגמאות}
	 \begin{itemize}
	 	\item \textbf{הגרף השלם על $\bm{n}$ קודקודים}, הוא גרף בו כל זוג קודקודים מחובר זה לזה בקשת. מחלקת השקילות של כל הגרפים השלמים על $n$ קודקודים, מסומנת ב־$\kc_n$. 
	 	\item $G = \la V, E \ra$ כאשר $V = \{0, \dots, n - 1\}$ נקרא \textbf{מעגל}, אם $E = \big\{\{i, i + 1\} \bmod n \in \mid i \in \{0, \dots, n - 1\} \big\}$ (המודולו בשביל הקודקוד האחרון). מסמנים ב־$\cc_n$ את מחלקת השקילות של גרף מעגל עם $n$ קודקודים (בעבור $n > 2$ – המעגל הכי קטן, הוא משולש). 
	 \end{itemize}
	 
	
\end{document}