\documentclass[]{article}

% Math packages
\usepackage[usenames]{color}
\usepackage{forest}
\usepackage{ifxetex,ifluatex,amsmath,amssymb,mathrsfs,amsthm,witharrows,mathtools}
\WithArrowsOptions{displaystyle}
\renewcommand{\qedsymbol}{$\blacksquare$} % end proofs with \blacksquare. Overwrites the defualts. 
\usepackage{cancel,bm}
\usepackage[thinc]{esdiff}

% tikz
\usepackage{tikz}
\newcommand\sqw{1}
\newcommand\squ[4][1]{\fill[#4] (#2*\sqw,#3*\sqw) rectangle +(#1*\sqw,#1*\sqw);}


% code 
\usepackage{listings}
\usepackage{xcolor}

\definecolor{codegreen}{rgb}{0,0.35,0}
\definecolor{codegray}{rgb}{0.5,0.5,0.5}
\definecolor{codenumber}{rgb}{0.1,0.3,0.5}
\definecolor{codeblue}{rgb}{0,0,0.5}
\definecolor{codered}{rgb}{0.5,0.03,0.02}
\definecolor{codegray}{rgb}{0.95,0.95,0.95}

\lstdefinestyle{pythonstylesheet}{
	language=Python,
	emphstyle=\color{deepred},
	backgroundcolor=\color{codegray},
	keywordstyle=\color{deepblue}\bfseries\itshape,
	numberstyle=\scriptsize\color{codenumber},
	basicstyle=\ttfamily\footnotesize,
	breakatwhitespace=false, 
	breaklines=true, 
	captionpos=b, 
	keepspaces=true, 
	numbers=left, 
	numbersep=5pt, 
	showspaces=false,                
	showstringspaces=false,
	showtabs=false, 
	tabsize=2, 
	morekeywords={object,type,isinstance,copy,deepcopy,zip,enumerate,reversed,list,set,len,dict,tuple,range,xrange,append,execfile,real,imag,reduce,str,repr},              % Add keywords here
	keywordstyle=\color{codeblue},
	emph={__init__,__add__,__mul__,__div__,__sub__,__call__,__getitem__,__setitem__,__eq__,__ne__,__nonzero__,__rmul__,__radd__,__repr__,__str__,__get__,__truediv__,__pow__,__name__,__future__,__all__,as,assert,nonlocal,with,yield,self,True,False,None,AssertionError,ValueError},          % Custom highlighting
	emphstyle=\color{codered},
	stringstyle=\color{codegreen},
	showstringspaces=false,
	abovecaptionskip=0pt,belowcaptionskip =0pt,
	framextopmargin=-\topsep, 
}
\newcommand\pythonstyle{\lstset{pythonstylesheet}}
\newcommand\pyl[1]     {{\lstinline!#1!}}
\lstset{style=pythonstylesheet}

\usepackage[style=1,skipbelow=\topskip,skipabove=\topskip,framemethod=TikZ]{mdframed}
\definecolor{bggray}{rgb}{0.85, 0.85, 0.85}
\mdfsetup{leftmargin=0pt,rightmargin=0pt,backgroundcolor=codegray,middlelinewidth=0.5pt,skipabove=4pt,skipbelow=0pt,middlelinecolor=black,roundcorner=5}
\BeforeBeginEnvironment{lstlisting}{\begin{mdframed}\vspace{-0.4em}}
	\AfterEndEnvironment{lstlisting}{\vspace{-0.8em}\end{mdframed}}

% Deisgn
\usepackage[labelfont=bf]{caption}
\usepackage[margin=0.6in]{geometry}
\usepackage{multicol}
\usepackage[skip=4pt, indent=0pt]{parskip}
\usepackage[normalem]{ulem}
\forestset{default}
\renewcommand\labelitemi{$\bullet$}
\graphicspath{ {./} }

% Hebrew initialzing
\usepackage[bidi=basic]{babel}
\PassOptionsToPackage{no-math}{fontspec}
\babelprovide[main, import]{hebrew}
\babelfont{rm}{David CLM}
\babelfont{sf}{David CLM}
\babelfont{tt}{Monaspace Argon}
\usepackage[shortlabels]{enumitem}
\newlist{hebenum}{enumerate}{1}

% Language Shortcuts
\newcommand\en[1] {\selectlanguage{english}#1\selectlanguage{hebrew}}
\newcommand\sen   {\selectlanguage{english}}
\newcommand\she   {\selectlanguage{hebrew}}
\newcommand\del   {$ \!\! $}
\newcommand\ttt[1]{\en{\small\texttt{#1}\normalsize}}

\newcommand\npage {\vfil {\hfil \textbf{\textit{המשך בעמוד הבא}}} \hfil \vfil \pagebreak}
\newcommand\ndoc  {\dotfill \\ \vfil {\begin{center} {\textbf{\textit{שחר פרץ, 2024}} \\ \scriptsize \textit{נוצר באמצעות תוכנה חופשית בלבד}} \end{center}} \vfil	}

\newcommand{\rn}[1]{
	\textup{\uppercase\expandafter{\romannumeral#1}}
}

\makeatletter
\newcommand{\skipitems}[1]{
	\addtocounter{\@enumctr}{#1}
}
\makeatother

%! ~~~ Math shortcuts ~~~

% Letters shortcuts
\newcommand\N     {\mathbb{N}}
\newcommand\Z     {\mathbb{Z}}
\newcommand\R     {\mathbb{R}}
\newcommand\Q     {\mathbb{Q}}
\newcommand\C     {\mathbb{C}}

\newcommand\ml    {\ell}
\newcommand\mj    {\jmath}
\newcommand\mi    {\imath}

\newcommand\powerset {\mathcal{P}}
\newcommand\ps    {\mathcal{P}}
\newcommand\pc    {\mathcal{P}}
\newcommand\ac    {\mathcal{A}}
\newcommand\bc    {\mathcal{B}}
\newcommand\cc    {\mathcal{C}}
\newcommand\dc    {\mathcal{D}}
\newcommand\ec    {\mathcal{E}}
\newcommand\fc    {\mathcal{F}}
\newcommand\nc    {\mathcal{N}}
\newcommand\sca   {\mathcal{S}} % \sc is already definded
\newcommand\rca   {\mathcal{R}} % \rc is already definded

\newcommand\Si    {\Sigma}

% Logic & sets shorcuts
\newcommand\siff  {\longleftrightarrow}
\newcommand\ssiff {\leftrightarrow}
\newcommand\so    {\longrightarrow}
\newcommand\sso   {\rightarrow}

\newcommand\epsi  {\epsilon}
\newcommand\vepsi {\varepsilon}
\newcommand\vphi  {\varphi}
\newcommand\Neven {\N_{\mathrm{even}}}
\newcommand\Nodd  {\N_{\mathrm{odd }}}
\newcommand\Zeven {\Z_{\mathrm{even}}}
\newcommand\Zodd  {\Z_{\mathrm{odd }}}
\newcommand\Np    {\N_+}

% Text Shortcuts
\newcommand\open  {\big(}
\newcommand\qopen {\quad\big(}
\newcommand\close {\big)}
\newcommand\also  {\text{, }}
\newcommand\defi  {\text{ definition}}
\newcommand\defis {\text{ definitions}}
\newcommand\given {\text{given }}
\newcommand\case  {\text{if }}
\newcommand\syx   {\text{ syntax}}
\newcommand\rle   {\text{ rule}}
\newcommand\other {\text{else}}
\newcommand\set   {\ell et \text{ }}
\newcommand\ans   {\mathit{Ans.}}

% Set theory shortcuts
\newcommand\ra    {\rangle}
\newcommand\la    {\langle}

\newcommand\oto   {\leftarrow}

\newcommand\QED   {\quad\quad\mathscr{Q.E.D.}\;\;\blacksquare}
\newcommand\QEF   {\quad\quad\mathscr{Q.E.F.}}
\newcommand\eQED  {\mathscr{Q.E.D.}\;\;\blacksquare}
\newcommand\eQEF  {\mathscr{Q.E.F.}}
\newcommand\jQED  {\mathscr{Q.E.D.}}

\newcommand\dom   {\text{dom}}
\newcommand\Img   {\text{Im}}
\newcommand\range {\text{range}}

\newcommand\trio  {\triangle}

\newcommand\rc    {\right\rceil}
\newcommand\lc    {\left\lceil}
\newcommand\rf    {\right\rfloor}
\newcommand\lf    {\left\lfloor}

\newcommand\lex   {<_{lex}}

\newcommand\az    {\aleph_0}
\newcommand\uaz   {^{\aleph_0}}
\newcommand\al    {\aleph}
\newcommand\ual   {^\aleph}
\newcommand\taz   {2^{\aleph_0}}
\newcommand\utaz  { ^{\left (2^{\aleph_0} \right )}}
\newcommand\tal   {2^{\aleph}}
\newcommand\utal  { ^{\left (2^{\aleph} \right )}}
\newcommand\ttaz  {2^{\left (2^{\aleph_0}\right )}}

\newcommand\n     {$n$־יה\ }

% Math A&B shortcuts
\newcommand\logn  {\log n}
\newcommand\cosx  {\cos x}
\newcommand\cost  {\cos \theta}
\newcommand\sinx  {\sin x}
\newcommand\sint  {\sin \theta}
\newcommand\tanx  {\tan x}
\newcommand\tant  {\tan \theta}

\newcommand\seq   {\overset{!}{=}}
\newcommand\sle   {\overset{!}{\le}}
\newcommand\sge   {\overset{!}{\ge}}
\newcommand\sll   {\overset{!}{<}}
\newcommand\sgg   {\overset{!}{>}}

\newcommand\h     {\hat}
\newcommand\ve    {\vec}
\newcommand\lv    {\overrightarrow}
\newcommand\ol    {\overline}

\newcommand\mlcm  {\mathrm{lcm}}

\DeclareMathOperator{\sech}   {sech}
\DeclareMathOperator{\csch}   {csch}
\DeclareMathOperator{\arcsec} {arcsec}
\DeclareMathOperator{\arccot} {arcCot}
\DeclareMathOperator{\arccsc} {arcCsc}
\DeclareMathOperator{\arccosh}{arccosh}
\DeclareMathOperator{\arcsinh}{arcsinh}
\DeclareMathOperator{\arctanh}{arctanh}
\DeclareMathOperator{\arcsech}{arcsech}
\DeclareMathOperator{\arccsch}{arccsch}
\DeclareMathOperator{\arccoth}{arccoth} 

\newcommand\dx    {\,\mathrm{d}x}
\newcommand\df    {\mathrm{d}f}
\newcommand\dfdx  {\diff{f}{x}}
\newcommand\dit   {\limhz \frac{f(x + h) - f(x)}{h}}

\newcommand\nt[1] {\frac{#1}{#1}}

\newcommand\limz  {\lim_{x \to 0}}
\newcommand\limxz {\lim_{x \to x_0}}
\newcommand\limi  {\lim_{x \to \infty}}
\newcommand\limni {\lim_{x \to - \infty}}
\newcommand\limpmi{\lim_{x \to \pm \infty}}

\newcommand\ta    {\theta}
\newcommand\ap    {\alpha}

\renewcommand\inf {\infty}
\newcommand  \ninf{-\inf}

% Combinatorics shortcuts
\newcommand\sumnk     {\sum_{k = 0}^{n}}
\newcommand\sumni     {\sum_{i = 0}^{n}}
\newcommand\sumnko    {\sum_{k = 1}^{n}}
\newcommand\sumnio    {\sum_{i = 1}^{n}}
\newcommand\sumai     {\sum_{i = 1}^{n} A_i}
\newcommand\nsum[2]   {\reflectbox{\displaystyle\sum_{\reflectbox{\scriptsize$#1$}}^{\reflectbox{\scriptsize$#2$}}}}

\newcommand\bink      {\binom{n}{k}}
\newcommand\setn      {\{a_i\}^{2n}_{i = 1}}
\newcommand\setc[1]   {\{a_i\}^{#1}_{i = 1}}

\newcommand\cupain    {\bigcup_{i = 1}^{n} A_i}
\newcommand\cupai[1]  {\bigcup_{i = 1}^{#1} A_i}
\newcommand\cupiiai   {\bigcup_{i \in I} A_i}
\newcommand\capain    {\bigcap_{i = 1}^{n} A_i}
\newcommand\capai[1]  {\bigcap_{i = 1}^{#1} A_i}
\newcommand\capiiai   {\bigcap_{i \in I} A_i}

\newcommand\xot       {x_{1, 2}}
\newcommand\ano       {a_{n - 1}}
\newcommand\ant       {a_{n - 2}}

\newcommand\mto       {\mapsto}
\DeclareMathOperator{\dist}{dist}

% Other shortcuts
\newcommand\tl    {\tilde}
\newcommand\op    {^{-1}}

\newcommand\sof[1]    {\left | #1 \right |}
\newcommand\cl [1]    {\left ( #1 \right )}
\newcommand\csb[1]    {\left [ #1 \right ]}

\newcommand\bs    {\blacksquare}

%! ~~~ Document ~~~

\author{שחר פרץ}
\title{קומבי 11 (כנראה) $\sim$ נטלי שלום $\sim$ איזומורפיה ומסלולים}
\date{17 ביוני 2024}

\begin{document}
	\maketitle
	\section{המשך – איזומורפיה}
	\subsection{תזכורות}
	\begin{enumerate}
		\item גרף (במקרה שלנו, גרף פשוט ולא מכוון או ממושקל) הוא $G = \le V, E \ra$ כאשר $B$ קבוצת הקודקודים (צמתים) – סופית, ולא ריקה. $E \subseteq \ps_s (V)$ קבוצת הקשתות (צלעות). 
		\item שני קודקודים נקראים שכנים, אם יש ביניהם קשת. 
		\item הדרגה של של קודקוד $v \in V$ היא $d(v) = \sof{N(v)}$, כאשר $N(v)$ היא קבוצת שכניו. 
		\item שני גרפים $G_1 G_2$ הם איזומורפים אם אם קיים זיווג $f \colon V_1 \to V_2$ משמר שכנויות, כלומר: 
		\[ \forall u, v \in V_1. \{u, v\}\in E_1 \iff \{f(u), f(v)\} \in E_2 \]
		הפונקציה הזו תקרא איזומורפיזם. 
		\item סימנו $=K_n$ מחלקת השקילות של הגרפים השלמים על $n$ קודקודים, ו־$=C_n$ מחלקת השקילות של המעגלים באורך $n$. 
		\item הגרף המשלים של $G = \la V, E \ra$ הוא $\ol G = \la V, \ol E \ra $ כך ש־$\ol E = \ps_2(V) \setminus E$.
	\end{enumerate}
	\textit{הערה: }גרף שלם = גרף מלא = קליקה
	\section{תרגילים}
	\subsection{}
	\textbf{שאלה: }האם $C_5$  איזומורפי ל־גרף המשלים שלו? [ציור בסיכומים אחרים, בו מספור של הקודקודים מ־0 עד 4]. נגדיר זיווג: 
	\[ 0 \mapsto 0, \ 1 \mapsto 2, \ 2 \mto 4, \ 3 \mto 1, \ 4 \mto 3 \]. 
	[\textit{הערה: }יש צורך לצייר בשביל לבחור סימון לכל הקודקודים]. 
	מותר לרשום את הזיווג גם בצורת פונקציה. \textbf{פתרון: }כן
	\subsection{תרגיל המשך}
	\textbf{תרגיל: }האם $C_6$ איזומורפי למשלים שלו?
	
	\textbf{פתרון: }דרך אחת תהיה להשתמש בדרגות. נראה דברים דומים בהמשך. אך נדבוק בדרך אחרת. ב־$C_6$ יש $6$ קשתות. $\sof{E} = \sof{\ps_2(V) \setminus E} = \sof{\ps_2(V)} - \sof{E} = \binom{6}{2} - 6 = \frac{6 \cdot 5}{2} - 6 = 15 - 6 = 9$
	כלומר, כמות הרשתות בין שני הגרפים שונה ולכן לא יתכן קיום איזומורפיה חח"ע ועל ביניהם. 
	
	\subsection{תרגיל המשך}
	\textbf{תרגיל: }הוכיחו שלא קיים גרף בעל $6$ קודקודים שאיזומורפי למשלים שלו. 
	
	\textbf{פתרון: }נניח בשלילה שקיים גרף כזה, אזי $|E| + |\ol E| = \binom{6}{2} = 15$ ומצד שני $|E| = |\ol E|$, כלומר $2 \cdot |E| = 15$ וסה"כ סתירה כי $15$ איז־זוגי. 
	
	\subsection{}
	נניח כי $f$ איזומורפיזם בין $G = \la V, E \ra$ ל־$G, = \la V', E' \ra$. הוכיחו: 
	\[ \forall v \in V. \underbrace{d_G(v)}_{\text{\textit{הדרגה בגרף} $G$}} = d_{G'}(f) \]
	\begin{proof}
		יהי $v \in V$. נסמן ב־$N$ את קבוצת שכיניו. נוכיח ש־$f[N]$ היא קבוצת השנים של $f(v)$ ב־$G'$.                                 
		\item יהי $w \in f[N]$. נוכיח שהוא שכן של $f(v)$ ב־$G'$
		\begin{itemize}
			\item קיים $u \in N$ כך ש־$w = f(u)$. כלומר $\{u, v\} \in E$ ומאחר ש־$f$ איזומורפיזם אז $\{\underbrace{f(u)}_{w}, f(v)\} \in E'$ ולכן $w$ שכן של $f(v)$ בגרף $G'$. 
			\item יהי $x$ שכן של $(v)$ ב־$G'$, נוכיח כי $x \in f[N]$. $f$ זיווג ובפרט על, לכן קיים $y \in V$ כך ש־$x = f(y)$ כלומר $\{x, f(v)\} = \{f(y), f(v)\} \in E'$ ומאחר ש־$f$ משמרת שכנויות נקבל ש־$\{y, v\} \in E$ כלומר $y \in N$ ולכן $x \in f[N]$. 
		\end{itemize}
		סה"כ מהכלה דו כיוונית $f[N] = d_{G'}(f(v))$, כלומר $d_G(v) = |N| = |f[N]| = d_{G'}f(v)$
	\end{proof}
	
	\subsection{}
	\textbf{שאלה: }תנו דוגמה לשני גרפים בעלי אותה סדרת דרגות שאינם איזומורפים עם: (א) 6 קודקודים (ב) 5 קודקודים
	
	\textbf{תשובה של שחר מהעתיד }(שראה פתרונות בסוף השיעור)\textbf{: }קו באורך $n$, ומנגד גרף של קו באורך $1$ ומעגל באורך $n - 2$ יפתור זאת. 
	
	נחזור לבעיה בסוף השיעור. 
	
	
	\section{הערות}
	\textit{הערה: }תנאים הכרחיים לאיזומורפיזם בין גרפים: 
	\begin{enumerate}
		\item אותו מס' קודקודים
		\item אותו מס' קשתות
		\item אותה סדרת דרגות (לא חשוב הסדר, אך יש חשיבות לחזרות)
	\end{enumerate}
	אלו \textbf{לא} תנאים מספיקים. 
	
	\textit{הערה: }הכרעת איזומורפיזם בין שני גרפים נתונים, היא בעיה קשה במדמח. 
	
	\section{מסלולים}
	\subsection{הגדרות}
	יהי $G = \la V, E \ra$ גרף. נגדיר: 
	\begin{enumerate}
		\item \textbf{מסלול } הוא סדרה של קודקודים $\la v_0, v_1, \dots v_k \ra$ כך שלכל $0 \le i \le k - 1$ מתקיים $\{v_i, v_{i + 1}\} \in E$ [=בין כל זוג קודקודים עוקבים יש קשת] ובנוסף לא עוברים על קשת יותר מפעם אחת (כלומר אסור ללכת שוב ושוב על אותה הקשת). [דוגמה מצויות שכמו תמיד אני אפנה אותכם לסיכומים אחרים כי לי אין כוח לצייר]. 
		\item \textbf{אורך של מסלול} הוא מס' הקשתות בו. לדוגמה, האורך של $\la v_0, \dots, v_k \ra$ הוא $k$. בפרט, סדרה באורך $1$ היא מסלול באורך $0$ (לדוגמה $\la v \ra$). 
		\item \textbf{מסלול פשוט} הוא מסלול שבו כל הקודקודים שונים שזה מזה. 
		\item \textbf{מעגל} הוא מסלול $\la v_0, \dots, v_m \ra$ שבו $v_0 = v_m$ ו־$m > 0$. 
		\item \textbf{מעגל פשוט} הוא מסלול שבו כל הקודקודים שונים זה מזה פרט לכך שהראשון והאחרון זהים. 
		\item \textbf{מרחק} בין שני קודקודים $a, b$ הוא אורך המסלול הקצר ביותר המחבר ביניהם, והוא מסומן ב־$\dist(a, b)$. אם אין מסלול המחבר ביניהם, מגדירים $\dist(a, b) = \inf$.
	\end{enumerate}
	
	\subsection{טענות והוכחות}
	
	\textbf{טענה: }[יהי יהיו תהינה בלה בלה בלה] אם קיים מסלול בין שני קודקודים שונים, אז בהכרח קיים מסלול פשוט ביניהם. 
	
	\textbf{תרגיל: } נתון גרף $G = \la V, E \ra$ שבו הדרגה המינימלית היא $m$. הוכיחו, שב־$G$ קיים מסלול פשוט באורך לפחות $m$. 
	
	\begin{proof}
		אם $m = 0$, אז ניקח קדוקוד כלשהו $v$ והוא מהווה מסלול פשוט באורך $0$ $\la v \ra$. 
		
		אם $m > 0$, נניח בשלילה שכל מסלול פשוט הוא באורך לכל היותר $m - 1$. ניקח מסלול $\la v_0, \dots v_k \ra$ מקסימלי בגרף. אז $k \le m - 1$. נסתכל על $v_k$. אם קיים לו שכן מחוץ ל־$\la v_, \dots v_{k -1 } \ra$, אז ניתן להוסיף אותו למסלול ולקבל מסלול ארוך יותר, בסתירה. אם לא קיים ל־$v_k$ שכן מחוץ לקבוצה זו, אז כל שכניו נמצאים בקבוצה הזו, ובה יש לכל $k$ קודקודים כלומר $d(v_k) \le k \le m - 1$ וזו סתירה לנתון שהדרגה המינימלית היא $m$. 
	\end{proof}
	
	
\end{document}