\documentclass[]{article}

% Math packages
\usepackage[usenames]{color}
\usepackage{forest}
\usepackage{ifxetex,ifluatex,amsmath,amssymb,mathrsfs,amsthm,witharrows}
\WithArrowsOptions{displaystyle}
\renewcommand{\qedsymbol}{$\blacksquare$} % end proofs with \blacksquare. Overwrites the defualts. 
\usepackage{cancel,bm}

% Deisgn
\usepackage[labelfont=bf]{caption}
\usepackage[margin=0.6in]{geometry}
\usepackage{multicol}
\usepackage[skip=4pt, indent=0pt]{parskip}
\usepackage[normalem]{ulem}
\forestset{default preamble={for tree={circle, draw}}}
\renewcommand\labelitemi{$\bullet$}

% Hebrew initialzing
\usepackage{polyglossia}
\setmainlanguage{hebrew}
\setotherlanguage{english}
\newfontfamily\hebrewfont[Script=Hebrew, Ligatures=TeX]{David CLM}
\usepackage[shortlabels]{enumitem}
\newlist{hebenum}{enumerate}{1}
\setlist[hebenum,1]{
	labelindent=\parindent,
	label={{\hebrewfont{\protect\hebrewnumeral{\value{hebenumi}}}}.}
}

% Math shortcuts

\newcommand\N     {\mathbb{N}}
\newcommand\Z     {\mathbb{Z}}
\newcommand\R     {\mathbb{R}}
\newcommand\Q     {\mathbb{Q}}

\newcommand\ml    {\ell}
\newcommand\mj    {\jmath}
\newcommand\mi    {\imath}

\newcommand\powerset {\mathcal{P}}
\newcommand\ps    {\mathcal{P}}
\newcommand\pc    {\mathcal{P}}
\newcommand\ac    {\mathcal{A}}
\newcommand\bc    {\mathcal{B}}
\newcommand\cc    {\mathcal{C}}
\newcommand\dc    {\mathcal{D}}
\newcommand\ec    {\mathcal{E}}
\newcommand\fc    {\mathcal{F}}
\newcommand\nc    {\mathcal{N}}
\newcommand\sca   {\mathcal{S}} % \sc is already definded
\newcommand\rca   {\mathcal{R}} % \rc is already definded

% combinatorics
\newcommand\p     {\mathcall{p}}
\newcommand\C     {\mathcall{c}}
\newcommand\s     {\mathcall{s}}

\newcommand\siff  {\longleftrightarrow}
\newcommand\ssiff {\leftrightarrow}
\newcommand\so    {\longrightarrow}
\newcommand\sso   {\rightarrow}

\newcommand\epsi  {\epsilon}
\newcommand\vepsi {\varepsilon}
\newcommand\vphi  {\varphi}
\newcommand\Neven {\N_{\mathrm{even}}}
\newcommand\Nodd  {\N_{\mathrm{odd }}}
\newcommand\Zeven {\Z_{\mathrm{even}}}
\newcommand\Zodd  {\Z_{\mathrm{odd }}}
\newcommand\Np    {\N_+}

\newcommand\open  {\big(}
\newcommand\qopen {\quad\big(}
\newcommand\close {\big)}
\newcommand\also  {\text{, }}
\newcommand\defi  {\text{ definition}}
\newcommand\defis {\text{ definitions}}
\newcommand\given {\text{given }}
\newcommand\case  {\text{if }}
\newcommand\syx   {\text{ syntax}}
\newcommand\rle   {\text{ rule}}
\newcommand\other {\text{else}}
\newcommand\set   {\ell et \text{ }}

\newcommand\ra    {\rangle}
\newcommand\la    {\langle}

\newcommand\oto   {\leftarrow}

\newcommand\QED   {\quad\quad\mathscr{Q.E.D.}\;\;\blacksquare}
\newcommand\QEF   {\quad\quad\mathscr{Q.E.F.}}
\newcommand\eQED  {\mathscr{Q.E.D.}\;\;\blacksquare}
\newcommand\eQEF  {\mathscr{Q.E.F.}}
\newcommand\jQED  {\mathscr{Q.E.D.}}

\newcommand\dom   {\text{dom}}
\newcommand\Img   {\text{Im}}
\newcommand\range {\text{range}}

\newcommand\trio  {\triangle}

\newcommand\rc    {\right\rceil}
\newcommand\lc    {\left\lceil}
\newcommand\rf    {\right\rfloor}
\newcommand\lf    {\left\lfloor}

\newcommand\lex   {<_{lex}}

\newcommand\bs    {\blacksquare}

\newcommand\az    {\aleph_0}
\newcommand\uaz   {^{\aleph_0}}
\newcommand\al    {\aleph}
\newcommand\ual   {^\aleph}
\newcommand\taz   {2^{\aleph_0}}
\newcommand\utaz  { ^{\left (2^{\aleph_0} \right )}}
\newcommand\tal   {2^{\aleph}}
\newcommand\utal  { ^{\left (2^{\aleph} \right )}}
\newcommand\ttaz  {2^{\left (2^{\aleph_0}\right )}}

\newcommand\n     {$n$־יה\ }

\newcommand\logn  {\log n}
\newcommand\cosx  {\cos x}
\newcommand\sinx  {\sin x}
\newcommand\tanx  {\tan x}
\newcommand\dx    {\,\mathrm{d}x}

\newcommand\en[1] {\selectlanguage{english}#1\selectlanguage{hebrew}}
\newcommand\sen   {\selectlanguage{english}}
\newcommand\she   {\selectlanguage{hebrew}}
\newcommand\del   {$ \!\! $}

\newcommand\seq   {\overset{!}{=}}
\newcommand\sle   {\overset{!}{\le}}
\newcommand\sge   {\overset{!}{\ge}}
\newcommand\sll   {\overset{!}{<}}
\newcommand\sgg   {\overset{!}{>}}

\newcommand\ttt[1]{\en{\texttt{#1}}}

\newcommand\tl    {\tilde}
\newcommand\op    {^{-1}}

\newcommand\h     {\hat}
\newcommand\ve    {\vec}
\newcommand\lv    {\overrightarrow}

\newcommand\sumnk {\sum_{k = 0}^{n}}
\newcommand\sumni {\sum_{i = 0}^{n}}
\newcommand\sumnko{\sum_{k = 1}^{n}}
\newcommand\sumnio{\sum_{i = 1}^{n}}
\newcommand\bink  {\binom{n}{k}}

\newcommand\nsum[2]   {\reflectbox{\displaystyle\sum_{\reflectbox{\scriptsize$#1$}}^{\reflectbox{\scriptsize$#2$}}}}

\title{שיעורי בית 18 -- עבודת פסח -- מבוא לקומבינטוריקה}
\author{שחר פרץ}

\begin{document}
	\maketitle
	\section*{1} %%1
	\begin{enumerate}[(a)]
		\item \textbf{שאלה: } כמה תתי קבוצות $S \subseteq [n] \times [m]$ מקיימות $S = A \times B$ עבור $A \subseteq [n], \ B \subseteq [m]$? \\
		\textbf{תשובה: }נוכיח כי לכל $C \subseteq [k]$ יתכנו $2^k$ אפשרויות לבחירת $C$ שונות. נתבונן באוסף של כל הסמפרים עד $k$, ומעקרון החיבור נסכום את האפשרויות לכל גודל קבוצה אפשרי, כלומר $\sum_{j = 0}^{k} \binom{k}{j} = 2^k$ כאשר נשתמש בבינום כי בקבוצה אין חשיבות לסדר, והשוויון יתקיים לפי משפט. \\
		לפי הטענה שהוכחנו, יתכנו $2^n, 2^m$ אפשרויות לבחירת $A, B$ בהתאמה, ומעקרון הכפל סך האפשרויות ל־$S$ יהיה
		$2^n \cdot 2^m = 2^{n + m}$. נשים לב שבחרנו מספר פעמים מקרים עבורם $A, B$ קבוצות ריקות, כי $A \times \emptyset = \emptyset \times B = \emptyset$, ולכן נוריד $2^m$ פעמים שחרנו את $B$ ביחד עם $\emptyset$ ונוריד $2^n$ נוספים באופן דומה, ונוסיף $1$ כדי לספור את הקבוצה הריקה פעם אחת, סה"כ $\bm{2^{n + m} - 2^n - 2^m + 1}$ אפשרויות. 
		\item \textbf{שאלה: }כמה מחרוזות טרינאריות באורך 30 מכילות בדיוק 12 מופעים של 2, ושההפרש בין מספר האחדים למספר האפסים הוא לכל היותר 2? \\
		\textbf{תשובה: }תחילה, נבחר $12$ מקומות מתוך $30$ המקומות לקבע בהם את $2$ – לזה יש $\binom{30}{12}$ אפשרויות. נתבונן בכל המקרים הזרים למספר הפעמים ש־$0$ ו־$1$ יופיעו ב־$18$ המקומות הנותרים וניעזר בכלל הכפל: 
		\begin{itemize}
			\item 9 פעמים 1 ו־9 פעמים 2 – $\binom{18}{9}$. 
			\item 10 פעמים 1 ו־8 פעמים 2 – $\binom{18}{10}$. 
			\item 8 פעמים 1 ו־10 פעמים 2 – $\binom{18}{8}$. 
		\end{itemize}
		סה"כ, מכלל החיבור: 
		\[ \binom{30}{12}\left (2\binom{18}{8} + \binom{18}{9}\right ) \approx 7.48 \cdot 10^{15} \]
		\item \textbf{שאלה: }כמה יחסים $R$ מעל הקבוצה $[n]$ ישנם המקיימים את התנאי $\forall x, y, z \in [n]. xRz \land yRz \implies x = y$? \\
		\textbf{תשובה: }לכל אפשרות של $z$, מן הנתונים תהיה אפשרות יחידה ל־$x \in [n]$ שיקיים $xRz$, ואם קיים $y$ אחר אז $y = x$ וזו סתירה. האפשרות השנייה היא, שלא קיים $x$ העומד עם $z$ ביחס. סה"כ $n + 1$ אפשרויות לכל $z$ שונה, ומשום שיש $n$ ערכי $z$ אז סה"כ מכלל הכפל $\bm{(n + 1)^n}$ אפשרויות. 
		
	\end{enumerate}
	\section*{2} %%2
	\begin{enumerate}[(a)]
		\item \textbf{שאלה: }כמה שלשות סדורות $\la A, B, C \ra \in (\ps([n]))^2$ ישנן המקיימות $|A \uplus B \cup C| = k$ וכן $A \cap B = \emptyset$? \\
		\textbf{תשובה: }
		נבחר $k$ איברים מתוך $n$ על מנת לעבוד איתם. לכך יהיו $\binom{n}{k}$ אפשרויות. נניח $A \uplus B = m$. יהיו $\binom{n}{m}$ אפשרויות לבחור את $A \uplus B$, ו־$2^m$ דרכים לבחור את $A$ (כי $\sum_{i = 0}^{m}\binom{m}{i} = 2^m$), ולאחר זאת $B = (A \uplus B) \setminus A$ כלומר יש דרך יחידה לבחור את $B$. סה"כ $ 2^m\binom{n}{m} $ דרכים עד כה. עבור $C$, מתוך $M$ האיברים של $A$ ו־$B$ נוכל לבחור באופן חופשי (כי אין זה ישנה את גודל האיחוד), אך ניאלץ לבחור את כל האיברים של מה שנותר מ־$k$ האיברים שבחרנו מתוך $n$ להיות גם חלק מ־$C$ (אחרת לא כילינו אותם). סה"כ, מכלל החיבור: 
		\[ \binom{n}{k}\sum_{m = 0}^{k}\binom{n}{m}2^m2^m = \bink \sum_{m = 0}^{n}\binom{n}{m}4^m1^m = \bm{\binom{n}{k}5^n} \]
		כאשר המעבר האחרון מהבינום של ניוטון. \\
		\textit{הערה: אפשר לסכום הן על $k$ אך גם על $n$, כי לכל $n > m > k$ יתקיים $\binom{n}{m} = 0$ ולכן האיברים הנוספים האלו לא באמת יכללו}. 

		\item \textbf{שאלה: }בכמה דרכים ניתן לפזר 40 כדורים כחולים זהים ו־20 כדורים לבנים זהים ל־5 תאים, כך שבכל תא מספר הכדורים הלבנים אינו עולה על מספר הכדורים הכחולים? \\
		\textbf{תשובה: }
		"נצמיד" ל־20 כדורים כחולים כדורים לבנים (זה חוקי כי כך לכל כדור לבן יהיה כדור כחול, וכי צריך להכניס את כל הכדורים הלבנים לתאים וכן את כל הכחולים). סה"כ יש לנו כדורים "מעורבים" מסוג אחד וכדורים כחולים מסוג שני. כמות הדרכים להכניס 20 כדורים זהים מסוג כלשהו ל־5 תאים היא $S(5, 20)$, וסה"כ מכלל הכפל נקבל $S(5, 20)^2 = \bm{\binom{24}{20}^2} \approx 10^{8}$. 
	\end{enumerate}
	\section*{3} %%3
	יהיו 3 מרצים, 4 מתרגלים ו־7 בודקים שלא החזירו לנו את שיעורי הבית מלפני ארבעה חודשים (להנגתם היו להם מבחנים...), שכולם יחדיו יקראו הצוות, הרוצים לשבת יחדיו; נחשב את כמות הדרכים ליישבם תחת אילוצים שונים. 
	\begin{enumerate}[(a)]
		\item \textbf{שאלה: }הצוות סביב שולחן עגול, המרצים יושבים זה לצד זה, המתרגלים זה לצד זה והמרצים זה לצד זה. כמה דרכים יש לישבם?\\
		\textbf{תשובה: }נחשב את כמות הדרכים ליישבם בשורה ונחלק בכמות האנשים בו. ראשית כל, נתייחס לכל המרצים, כל המתרגלים והבודקים כ־3 אובייקטים שונים. נמצא שיש $3! = 6 $ צורות שונות ליישב אוייבקטים אלו. נחשב את כמות הפרמוטציות בכל אחת מהקבוצות: $ 7! $ אצל הבודקים, $ 4! $ אצל המתרגלים, ו־$ 3! $ אצל המרצים. סה"כ, מכלל הכפל ומחלוקה בכמות ב־3 (כי יש שלוש קבוצות, היושבות במעגל), נקבל:
		\[ \tfrac{3! \, 4! \, 7!}{3} \cdot 6 = \bm{2 \cdot 3! \, 4! \, 7!} \approx (\pi - 3) \cdot 10^6 \]
		\item \textbf{שאלה: }יש להושיב את הצוות סביב 2 שולחנות עגולים זהים (כל אחד מהם מתאים ל־7 אנשים). כמה דרכים יש לישבם?\\
		\textbf{תשובה: }תחילה, נבחר 7 אנשים מהצוות, ונעביר אותם לשולחן הראשון. סה"כ $\binom{14}{7} = 3432$ אפשרויות. עבור כל אחד מהשולחנות האלו, ניקח את שבעת האנשים ונחשב את כמות הדרכים לסדרם ביניהם – $ 7! $. נחלק ב‏־$7$ משום שיושבים בשולחן עגול, ונקבל $ 6! $ אפשרויות, לכל שולחן בנפרד, ונחלק ב־2 כי אין סדר השולחנות משנה. מכלל הכפל: 
		\[ \bm{\frac{\tbinom{14}{7} \cdot 6!^2}{2}} \approx 8.8 \cdot 10^{8} \]
		\item \textbf{שאלה: }יהיו 280 סטודנטים בקורס. בכמה דרכים כולם יכולים לשבת סביב שולחן עגול, כך שבין כל 2 אנשי צוות ישבו לפחות 10 סטודנטים? \\
		\textbf{תשובה: }נתבונן באובייקט שיקרא "סטודנט מוגדל", שיהיה איש צוות וסביבו 5 סטודנטים. ישנם 14 אנשי צוות, וכי $ 280/14 = 20 \ge 0 $ אז סה"כ יש 14 סטודנטים מוגדלים. נתבונן בסדר בתוך בסטונדט המוגדל: הסטודנט המוגדל יחולק לשתי קבוצות של 5 סטודנטים וביניהם מורה, וסה"כ $ 5! \cdot 2 $ אפשרויות לסידור (נכפיל את זה ב־14 כדי להתחשב בכל הסטודנטים המוגדלים). את 20 הסטודנטים הנותרים אפשר למקם בכל מקום בין 14 הסטורנטים המוגדלים, כמו והיה צורך לחלקם ל־15 מקומות שונים, כלומר סה"כ $S(15, 14)$ אפשרויות. את הסטודנטים המודלים עצמם אפשר לסדר ב־$ 14! $ מקומות שונים. נחלק ב־$ 280 + 14 $ כדי לקחת בחשבון את הסידור במעגל. סה"כ מכלל הכפל: 
		\[ \bm{\frac{14! \cdot 14 \cdot 5!\, 2 \cdot S(15, 14)}{280 + 14}} \approx 8 \cdot 10^{19} \]
	\end{enumerate}
	
	\section*{4} %%4
	יש למצוא את מספר הפתרונות לכל משוואה/אי־שוויון.
	\begin{enumerate}[(a)]
		\item \textbf{משוואה:}
		$\underbrace{(x_1 + x_2 + x_3)}_a\underbrace{(x_4 + x_5 + x_6 + x_7)}_{b} = 33, \ \land \ x_1, \dots, x_7 \in \N$ \\
		\textbf{תשובה: }נמצא כי 3 ו־11 הם הגורמים הראשוניים הקטנים ביותר של $33$, ומכיוון ש־$a, b \in \N \land 11 \cdot 3 = 33 \land ab = 33$ אז $\la a, b \ra = \la 3, 11 \ra, \la 11, 3 \ra, \la 1, 33 \ra, \la 33, 1 \ra$ (כלומר ישנן ארבע אפשרויות וזה אומר שאפשר לבדוק הכל ידנית). נחשב את כמות האפשרויות לכל אחת מהן; אם $\la a, b \ra = \la 11, 3 \ra$, אז $x_1 + x_2 + x_3 = 11$, שיגרור $S(3, 11)$ פתרונות לשלושת משתנים אלו, וגם $ x_3 + x_4 + x_5 + x_6 = 3$ שיגרור $S(4, 3)$ פתרונות לשלושת המשתנים הללו. מכלל הכפל סה"כ $S(4, 3)S(3, 11)$ פתרונות במקרה זה. אם $\la a, b \ra = \la 3, 11 \ra$ אז באופן דומה נקבל $S(4, 11)S(3, 3)$ פתרונות. באופן דומה אם $\la a, b \ra = \la 33, 1 \ra$ אז יהיו $S(4, 1)S(3, 33)$ פתרונות. באופן דומה אם $\la a, b \ra = \la 1, 33 \ra$ אז יהיו $S(4, 33)S(3, 1)$ פתרונות.
		סה"כ מכלל החיבור נמצא שכמות הפתרונות היא: 
		\[ S(4, 11)S(3, 3) + S(4, 3)S(3, 11) + S(4, 1)S(3, 33) + S(4, 33)S(3, 1) = \bm{29000} \]
		\item \textbf{אי־שוויון: }$100 \le \sum_{i = 1}^{100}x_i \le 200$ \\
		\textbf{תשובה: }נחשב את כמות הפתרונות של $\sum_{i = 1}^{100}x_i \le 200$ ושל $\sum_{i = 1}^{100}x_i \le 99$ וניעזר בכלל החיסור. כמות הפתרונות של הראשון מבין אי־השוויונות תוכל להיות מחושבת ע"י הוספת איבר "עזר" ומציאת כמות הפתרונות של $\sum_{i = 1}^{101}x_i = 99$, שהיא $S(101, 99)$. באופן דומה, כמות הפתרונות של השנייה מביניהם היא $S(101, 200)$. סה"כ נקבל $\bm{S(101, 200) - S(101, 99)} \approx 1.2 \cdot 10^{82}$. 
		
		\item \textbf{אי־שוויון: }$\sum_{i = 1}^{1000}x_i > 500 \ \land \ x_1, \dots, x_{1000} \in \{0, 1\}$ \\
		\textbf{תשובה: }בהכרח לפחות 501 מתוך ערכי ה־$x_i$ השונים יהיו $1$ אחרת הסכום יהיה קטן מ־500. נניח שיש $n$ ערכי $x_i$ שהם $1$ (ת.ה. $ 501 < n < 1000 $). את ה־$1$ הראשון נמקם באחד מתוך $1000$ המקומות, את השני באחד מתוך $1000 - 1$ המקומות שנותרו, וכן הלאה; עד שנגיע ל־$1000 - n$ ויכלו ה־$1$־ים. סה"כ $\prod_{k = 0}^{n}1000 - k = P(1000, n)$. מכלל החיבור, סה"כ יהיו $\bm{\sum_{n = 501}^{1000}P(1000, n)} \gg 10^{1400}$ אפשרויות. 
	\end{enumerate}
	\section*{5} %%5
	\begin{enumerate}[(a)]
		\item \textbf{שאלה: }כמה פונקציות $f \in [n] \to [k]$ מונוטוניות עולות חלש ישנן? ניעזר בסימון $\forall i \in \N. f(i) =: f_i$ \\
		\textbf{תשובה: }מהנתונים, $f_1 \le f_2 \le \dots \le f_n \le k$. נסמן $x_1 = f_1, \ \forall 0 < x \le n. x_x = f_x - f_{x - 1} $ רקורסיבית, לכן $\sum_{i = 1}^{n}x_i = f_n = k$ באינדוקציה. משום שסדרות ערכי ה־$x$ חח"ע ביחס לערכי ה־$f_i$ (אפשר להוכיח באינדוקציה), אז כמות ערכי ה־$y$ הקיימים הם כמות ערכי ה־$f_i$ הקיימים, אשר מגדירים את הפוקנציה. ישנם $n - 1$ ערכי $x$, ומשום שידוע שכמות הפתרונות לסכום כזה ניתנת ע"י הפונקציה $S(i, j + 1)$, אז במקרה הזה התשובה היא $S(k, n - 1 + 1) = S(k, n)$. עד כה התעלמנו מהמקרה בו אחד המספרים הוא $0$, ואז הפתרון אינו תקין (כי $0 \notin [k]$). אם מספר שאינו $x_1$ הוא $0$, אז זה סתירה לאי השוויון כלומר לא ספרנו את הפתרון הזה, ואם $x_1 = 0$ אז יש עוד $S(k, n - 1)$ פתרונות לא תקינים שספרנו (הפתרון לאותה השאלה עם מספר אחד פחות). סה"כ מכלל החיסור $\bm{S(n, k) - S(n, k - 1)}$. 
		\item \textbf{שאלה: }כמה פונקציות $f \colon [n] \to [k]$ מונוטוניות חזק ישנן? \\
		\textbf{תשובה: }באופן דומה לסעיף הקודם, $f_1 < f_2 < \dots < f_n < k$. נגדיר באופן דומה $y_1 = f_1, \ \forall 0 < x \le n. x_x = f_x - f_{x - 1}$. בניגוד לפעם הקודמת, ישנה הגבלה נוספת והיא $f_i \neq f_{i - 1} \implies f_i - f_{i - 1} \neq 0 \implies y_i \neq 0$. מההגדרות לעיל $\sum_{i = 1}^{n}y_i < k \iff \sum_{i = 1}^{n + 1}y_i = k $ כאשר $y_{n + 1} = k - y_n \neq 0$. גם $y_1 = 0$ כי $0 \notin [n]$. עבור כל $n$ המספרים נגדיר ערך התחלתי ל-1. עבור ההגבלות האלו ידוע שהפתרון הוא $S(n + 1, k - n) = \binom{k - n + n - 1 + 1}{k - n} = \binom{k}{k - n} = \bm{\binom{k}{n}}$. 
		\item \textbf{שאלה: }פונקציה מונוטונית עולה בטירוף אמ"מ $\forall 1 \le i \le n - 1. f_i + i \le f_{i + 1}$. כמה פונקציות מונוטוניות עולות בטירוף ישנן? \\
		\textbf{תשובה: }נרצה למצוא את כמות הפתרונות למשוואה $f_1 + 1 \le f_2 \land \dots \land f_{n - 1} + n - 1 \le f_n \land f_n \in [k]$. באופן דומה לסעיפים קודמים, נגדיר $\forall 1 \le i < n. x_i = f_i - f_{i - 1}$. מתנאי הפונקציה העולה בטירוף, נסיק $x_1 \ge 1, x_2 \ge 1 + 2, \dots x_n \ge \sumni i$ או באופן כללי $x_k \ge \sum_{i = 0}^{k} i = \frac{k(k + 1)}{2}$. באופן דומה לסעיפים קודמים, גם כאן יהיו כמות הפתרונות תהיה כמות הפתרונות של המשוואה $x_1 + \dots + x_n < k$, אך נרצה להיפתר מההגבלה לעיל. בשביל לעשות זאת, נגדיר $x'_i = x_i - i$ ולכן $\sumni x'_i + \sumni i = \sumni x_i < k$ וסה"כ נעביר אגפים ונקבל $\sumni x'_i < k - \sumni i$. נוסיף איבר עזר $x'_{i + 1} \ge 1$. סה"כ $\bm{S(k - \frac{n(n + 1)}{2}, n - 1)}$ אפשרויות. 
		\item \textbf{שאלה: } כמה פונקציות $f \colon [n] \to [k]$ שאינן מונוטוניות עולות או יורדות ישנן? \\
		\textbf{תשובה: }הפתרון לשאלה (א) בעבור פונקציות מונוטוניות יורדות יוותר זהה, עם שינוי הסימן. סה"כ יש $ 2(S(n, k) + S(n, k - 1)) - k $ אפשרויות שאינן תקינות (החיסור כי יש $k$ מקרים של פונקציות קבועות, שאינן יורדות או עולות). כמות הפונקציות בכללי, היא $n^k$ (עם חשיבות לסדר ועם החזרה), ולכן סה"כ התשובה היא $\bm{n^k - 2S(n, k) - 2S(n, k - 1) + k}$. 
	\end{enumerate}
	\section*{6} %%6
	פונקציה $f \colon [n] \to [m]$ תוגדר כ"כמעט זיווג" אמ"מ	$|\{k \in [n] \mid f\op[\{k\}] > 1\}| = 1$. 
	\begin{enumerate}[(a)]
		\item \textbf{שאלה: }כמה כמעט־זיווגים $f \colon [n] \to [m]$ ישנם? \\
		\textbf{תשובה: }לפי הגדרת הכמעט זיווג, קיים ויחיד $k \in [n]$ כך ש־$|f\op[\{k\}]| > 1$. יהיו $m$ אפשרויות לבחירת $k$, ועבור $m - 1$ האיברים הנותרים ב־$[m]$, נצטרך לבחור להם באופן חח"ע איבר מ־$[n]$ – לכך, יהיו $n \cdot (n - 1) \cdot \dots \cdot (n - m) = P(n, m)$ אפשרויות. משום שפונקציה היא מלאה, יתרת המספרים ב־$[n]$ יאלצו ללכת ל־$k$. סה"כ $\bm{m \cdot P(n, m)}$ אפשרויות.  \\
		\textit{הערה: }הטענה תקפה גם אם $ \lnot m \ge n - 1 $. שכן במקרה זה $\binom{m}{n - 1} = 0 $ והמכפלה תתאפס, כי אכן אין שום תוצאות (כי $f|_{[n] \setminus \{k\}}$ חח"ע $f \colon |n - 1| \to |m|$ ולכן $n - 1 < m$). 
		\item \textbf{טענה: }$|X / S| = n$ כאשר יחס השקילות $S$ מוגדר באמצעות $fSg \iff \exists h \in [m] \to [n]. f = g \circ h \land h \text{bijection a is  }$. 
	\begin{proof} \ \\
		\textbf{כיוון ראשון: }נסמן $f^k = f\op[\{k\}]$ לכל $f$ פונקציה. יהיו $f, g$ פונקציות, וידוע קיום יחיד $k_f$ כך ש־$|f^{k_f}| > 1 $, ובאופן דומה על $k_g$. נניח בשלילה $|f^{k_f}| \neq |g^{k_g}|$ וגם $fSg$. בה"כ $|f^{k_f}| > |g^{k_g}|$, וזו סתירה, כי נגרר חוסר קיום פונקציה חח"ע בין הקבוצות שאמורה להיות חלק מ־$h$ (כי מהשוויון, לאחר ההרכבה שניהם אמורים להוביל לאותו האיבר), כלומר $h$ לא חח"ע. סה"כ הוכחנו $fSg \implies |f^{k_f}| = |g^{k_g}|$. \\
		\textbf{כיוון שני: }נניח $|f^{k_f}| = |g^{k_g}| $ כלומר קיימת פונקצית זיווג בניהם, נסמנה $j \colon f' \to g$ (כאשר $f'$ הוא $f$ מצומצמת על החלק הנשלח ל־$k_f$). נרצה להוכיח $fSg$, כלומר קיום $h$ זיווג כך ש־$f = g \circ h$. עבור החלקים הנשלחים ל־$k_f$, יתקיים $h(x) = j(x)$, ובכל השאר כבר ידוע מהגדרת $f, g $ ככמעט־זיווג, שקיים ערך יחיד היתאים למטרה. 
		
		סה"כ $fSg \iff |f^{k_f}| = |g^{k_g}|$, ולגודל הקבוצה $|f^{k_f}|$ לכל $f \colon [m] \to [n]$ יהיו $n$ אפשרויות, כלומר $\bm{|X / S| = n}$. 
	\end{proof}
	
	
	\end{enumerate}
	\section*{7} %%7
	נביא פתרון קומבינטורי (עם סיפור) למשוואות להלן. 
	\begin{enumerate}[(a)]
		\item \textbf{המשוואה: }$\binom{3n}{2} = 3\binom{n}{2}+ 3n^2$ \\
		\textbf{הבעיה: }יהיו $3$ סלים עם $n$ כדורים בפנים. נרצה לבחור שני כדורים מתוך הסלים.  \\
		\textbf{אגף ימין: }נתבונן בשני מקרים זרים: אם שני הכדורים באותו הסל, אז נבחר (שזה $\binom{3}{1} = 3$ אפשרויות) ומתוכו נבחר שני כדורים (בסל יש $n$ כדורים, לכן אלו $\binom{n}{2}$ אפשרויות) וסה"כ לכך $3\binom{n}{2}$ אפשרויות למקרה זה. במקרה השני, שני הכדורים בשלים שונים. נבחר שני סלים ($\binom{3}{2} = 3$ אפשרויות), ומתוך כל אחד מהם נבחר כדור (לבחירת כדור מכל אחד מהסלים $n$ אפשרויות, יחדיו $n^2$). מכלל החיבור סה"כ $3\binom{n}{2} + 3n^2$ אפשרויות. \\
		\textbf{אגף שמאל: }נאחד את שלושת הסלים לסל אחד, בו יהיו $3n$ כדורים, ונבחר מתוכו $2$ כדורים. סה"כ $\binom{3n}{2}$ אפשרויות. 
		\item \textbf{המשוואה: }$\sum_{k = 0}^{n}\binom{n}{k}2^k = 3^n$ \\
		\textbf{הבעיה: }מציאת כמות הקבוצות על $[n] \times \{0, 1\}$. \\
		\textbf{אגף ימין: }עבור כל תו במיקום ה־$n$, נוכל לבחור אם התו במיקום ה־$n$ לא יכלל, ואם הוא יכלל – האם ישוייך לו המספר $0$ או $1$. סה"כ אלו $3$ אפשרויות, ומכלל הכפל נקבל $3^n$ אפשרויות.  \\
		\textbf{אגף שמאל: }נניח שגודל הקבוצה הוא $k$. נבחר את המספרים שיופעיו – $\binom{n}{k}$ אפשרויות. לכל אחד מהם, נבחר האם ישוייך המספר $0$ או $1$ – $2^k$ אפשרויות. מכלל החיבור, נקבל $\sum_{k = 0}^{n}\binom{n}{k}2^k$. 
		\item \textbf{המשוואה: }$\sum_{k = 1}^{n}k\binom{n}{k}^2 = n\binom{2n - 1}{n - 1}$ \\
		\textbf{הבעיה: }יהיו $n$ כדורים. נוכל לצבוע כל כדור בצהוב או בכחול, וכדור שנצבע בשניהם יהיה כדור בצבע ירוק. על כדור צבוע בכחול (ובפרט בירוק) יחיד, נסמן נקודה. כמה אפשרויות לצביעה כזו ישנן? כמות הכדורים הצבועים בכחול וכמות אלו הצבועים בצהוב שווה.  \\
		\textbf{אגף ימין: }מתוך $n$ הכדורים שלנו נבחר אחד להיות זה שעליו הנקודה. נמספר את הכדורים (אין חשיבות לסדר המספרים), ונשכפל אותם. עותק אחד נשים בשק כחול, ועותק שני בשק צהוב. נוציא את הכדור שבחרנו לשים עליו נקודה מהשק הכחול (שכן הכדור עם הנקודה בהכרח צבוע בכחול) ונוציא $n - 1$ כדורים משני השקים ($\binom{2n - 1}{n - 1}$). בה"כ הוצאו פחות כדורים מהשק הכחול מאשר מהצהוב. כל הכדורים שהוצאו מהשק הכחול יהפכו לכחולים, ואלו המוספרים באופן זהה מהשק השני יהפכו לצהובים. מה שנותר מהשק השני שאינו צהוב, יהיה ירוק. נאסוף את כל הכדורים, ואת כמות הכדורים מכל קטוגוריה נחלק בשניים, וסיימנו. \textbf{[לשפר]}  \\
		\textbf{אגף שמאל: }נניח שיש $k$ כדורים צבועים. נבחר $\binom{n}{k}$ כדורים להיצבע בכחול, $\binom{n}{k}$ להצבע בצהוב. נגדיר $a$  ככמות הכדורים שנצבעו בשני הצבעים, כלומר הם ירוקים. מכאן $k - a$ כדורים צהובים וכמות זהה של כדורים כחולים, כלומר אנו עומדים בתנאי. מתוך $k$ הכדורים שנצבעו כחול, נבחר אחד מהם להיות זה עם הנקודה עליו ($k$ אפשרויות). מכלל החיבור, $\sum_{k = 1}^{n}k\binom{n}{k}^2$ אפשרויות כדרוש. 
		\item \textbf{המשוואה: }$\binom{m}{n}\binom{n}{k} = \binom{m}{k}\binom{m - k}{n - k}$ \\
		\textbf{הבעיה: }כמות הדרכים לבנות ממלכה עם $m$ אנשים, $n$ אנשים שבאצולה, ו־$m$ אנשים מלכים (ובפרט באצולה).  \\
		\textbf{אגף ימין: }נבחר $k$ מלכים מתוך העם של $m$ האנשים ($\binom{m}{k}$). עתה נותר לבחור את כל שאר האצולה שהם לא מלכים, כלומר עוד $n - k$ אנשי אצולה מתוך $m - k$ אנשים שהם לא כבר באצולה – סה"כ $\binom{m - k}{n - k}$. מכלל הכפל $\binom{m}{k}\binom{m - k}{n - k}$ אפשרויות.  \\
		\textbf{אגף שמאל: }מתוך העם, נבחר $n$ אנשי אצולה ($\binom{m}{n}$ אפשרויות). מתוך האצולה, נבחר עוד $k$ מלכים ($\binom{n}{k}$ אפשרויות). סה"כ מכלל הכפל $\binom{m}{n}\binom{n}{k}$ אפשרויות. 
		\item \textbf{המשוואה: }$\binom{m + n + 1}{n} = \sum_{k = 0}^{n}\binom{m + k}{k}$ \\
		\textbf{הבעיה: }ננסה לחלק $n$ כדורים ל־$m + 2$ סלים. \\
		\textbf{אגף ימין: }נניח שבסל הראשון נשים $n - k$ כדורים, אז ניצטרך לחלק את $k$ הכדורים הנותרים ל־$m - 1$ סלים, כלומר נצטרך $S(m - 1, n - k) = \binom{m + k}{k}$. סה"כ מכלל החיבור $\sumnk \binom{m + k}{k}$ אפשרויות. \\
		\textbf{אגף שמאל: }ידוע $\binom{m + n + 1}{n} = S(m + 2, n)$ שעונה באופן מיידי על הדרישה. 
		\item \textbf{המשוואה: }$\sum_{k = 0}^{n}k(k - 1)\binom{n}{k}2^k = 4n(n - 1)3^{n - 2}$ \\
		\textbf{הבעיה: }מציאת כמות הקבוצות על $[n] \times \{0, 1\}$ (כלומר, מספרים אליהם הספרה 0 או 1 צמודה), כאשר 2 מהספרות יקראו "לבנות". \\
			\textbf{אגף ימין: }
			 מתוך $n$ המספרים, נבחר שניים לבנים – עבור הראשון יהיו $n$ אפשרויות, ועבור השני $n - 1$. ידוע ששני המספרים האלו יכללו בקבוצה, ולכן נותר להחליט עבור $n - 2$ המספרים הנותרים, 1 מ־3 אפשרויות: או שאינן בקבוצה; או שהן במחרוזת ו־1 צמוד אליהן; או שהן במחרוזת ו־2 צמוד אליהן. 
			 סה"כ יהיו $ 4n(n - 1)3^{n - 2} $ מכלל הכפל, כדרוש. \\
		\textbf{אגף שמאל: } נניח שגודל הקבוצה הוא $k$, 
		לכן $0 \le k \le n$. נבחר את $\binom{n}{k}$ הספרות שיכללו בתוך הקבוצה, ועבור על המספרים נשייך מספר, 0 או 1 – כלומר $2^k$ אפשרויות. יהיו $k$ אפשרויות למי יהיה המספר הלבן הראשון, ו־$k - 1 $ למספר השני. סה"כ מכלל החיבור $\sum_{k = 0}^{n}k(k - 1)\binom{n}{k}2^k$ כדרוש. 
	\end{enumerate}
	\section*{8} %%8
	\textbf{שאלה: }תהי $|B| = n$. יהי $\alpha \in \R$, צ.ל. לחשב את $\sum_{A \subseteq B}\alpha^{|A|}$. \\
	\textbf{תשובה: }לפנות כל, מכיוון שהמאפיין היחיד המבתטא בסכום הוא גודל הקבוצה – ולא תוכנה – אז ננסה למצוא כמה פעמים כל גודל מכל קבוצה יופיע. עבור קבוצה קבוצה בגודל כללי $i$, לפי הגדרת הבינום יהיו $\binom{n}{i}$ תתי קבוצות בתוך $B$ המקיימות $|A| = n$. נסיק, שהסכום שלנו שווה ל־$\sum_{i = 0}^{n}\binom{n}{i}\alpha^i $ (כי לכל גודל קבוצה נרצה לספור את מספר הפעמים שהגודל מופיע בסכום המקורי). סה"כ, לפי הבינום של ניוטון: 
	\[ \sum_{A \subseteq B}\alpha^{|A|}
	 = \sum_{i = 0}^{n}\binom{n}{i}\alpha^i
	 = \sum_{i = 0}^{n}\left [ \binom{n}{i}\alpha^i1^{n - i} \right ]
	 = \bm{(\alpha + 1)^n} \]
	\section*{9} %%9
	\textbf{שאלה: }מה המקדם של $x^{17}$ בביטוי $\left (x + \tfrac{1}{x}\right )^{2024}$? \\
	\textbf{תשובה: }לפי הבינום של ניוטון: 
	\[ \left (x + \tfrac{1}{x}\right )^{2024} = \left (x + x^{-1} \right )^{2024} = \sum_{k = 0}^{n}\binom{n}{k}x^k x^{-(n  - k)}= \sum_{k = 0}^{2024}\binom{2024}{k}x^{2k - 2024} \]
	סה"כ המקדם יהיה 17 אמ"מ $ 2k - 2024 = 17 $, כלומר $ k = 1020.5 $, וזו סתירה כי $k \in \N$ ולכן לא קיים איבר כזה (או בניסוח שקול, המקדם הוא $0$). 
	\section*{10} %%10
	\textbf{שאלה: }מצא את כל הקבוצות המרוכזות בעצמן ב־$[n]$, כלומר את $|\{A \in [n] \mid |A| \in A\}|$. \\
	\textbf{תשובה: }
		נניח שגודל הקבוצה $A \in [n]$ הוא $k$. מכאן, $k = |A| \in A$. לכן, ניאלץ לבחור להוסיף את $k$ לקבוצה. עבור $k - 1$ האיברים שנותרו, נבחר איברים מ־$n - 1 $ המספרים שנותרו, כלומר סה"כ נקבל $\binom{n - 1}{k - 1}$ אפשרויות. מכלל החיבור, נמצא שכמות האפשרויות היא: 
	\begin{alignat*}{9}
		\; & \sumnk\binom{n - 1}{k - 1} &&= \sum_{k' = - 1}^{n - 1} \binom{n - 1}{k - 1 + 1} \\
		\; = &\sum_{k' = -1}{n'}\binom{n' + 1 - 1}{k} &&= \sum_{k' = 0}^{n'}\binom{n'}{k'} \\
		\; = &2^{n'} &&= \bm{2^{n - 1}}
 	\end{alignat*}
 	זאת בעבור $k' = k - 1, n' = n - 1 $, ומכיוון $ \forall n \in \N. \binom{n}{-1} = 0 $. סה"כ כמות האפשרויות היא $ \bm{2^n - 1} $. 
 	
 	\textit{הערה: השתמשתי במעברים שלי בזהויות מקורס מתמטיקה \en{A}. }
	\section*{11} %%11
	\begin{enumerate}[(a)]
		\item לפי הבינום של ניוטון: 
		\[ \sum_{k = 0}^{n}\binom{n}{k}3^{n - k} =  \sum_{k = 0}^{n}\binom{n}{k}3^{n - k}1^n = (3 + 1)^{n} = \bm{4^n} \]
		\item לפי הבינום של ניוטון: 
		\[ \sum_{k = 0}^{n}\binom{n}{k}3^{n - k}(-1)^k = (3 - 1)^k = \bm{2^n} \]
	\end{enumerate}
	\section*{12} %%12
	\begin{enumerate}[(a)]
		\item \textbf{שאלה: }כמה מחרוזות באורך $n$ מעל $\{A, B, C, D\}$ קיימות כך שהמספר המופעים של התו $A$ זוגי? \\
		\textbf{תשובה: }נניח ש־$A$ מופיע $k$ פעמים. נבחר את מקומותיו מתוך המחרוזת להיות $\binom{n}{k}$, ועבור כל $n - k$ התווים שנשארו יהיו $ 3 $ אפשרויות עם החזרה לכל תו, כלומר לכך יהיו $3^{n - k}$ אפשרויות. מכלל הכפל $\binom{n}{k}3^{n - k}$, ומכלל החיבור $שa := \sum_{k = 0 \ (even)}^{n}\binom{n}{k}3^{n - k}$. נתבונן בסכום השני, $\sum_{k = 0}^{n}\binom{n}{k}3^{n - k}(-1)^k$. נשים לב שעבור כל $k$ אי זוגי, $ (-1)^k = -1 $ בעוד לכל האי־זוגיים $ (-1)^k = 1 $. נסמן את סכום כל החלקים הזוגיים ב־$a$ והזוגיים ב־$b$, כך שהסכום הזה יהיה $a - b$. באופן דומה, הסכום $\sum_{n}^{k = 0}\binom{n}{k}3^{n - k}$ לא יחסיר אף פעם ולכן שווה ל־$a + b$. אנו מעוניינים רק ב־$a$, כלומר נוכל לחבר את הסכומים ולמצוא כי: 
		\[ a = \frac{(a + b) + (a - b)}{2} = \frac{\sum_{k = 0}^{n}\binom{n}{k}3^{n - k}(-1)^k + \sum_{k = 0}^{n}\binom{n}{k}3^{n - k}}{2} = \frac{4^n + 2^n}{2} = \bm{4^{n - 0.5} + 2^{n - 1}} \]
		כאשר המעבר המסומן בסימן קריאה מתבצע לפי הסעיף הקודם. 
		\item \textbf{שאלה: }כמה מחרוזות באורך $n$ מעל $\{A, B, C, D\}$ קיימות כך שמספר המופעים של התווים $A, B$ הם זוגיים, לכל אחד? \\
		\textbf{תשובה: }נניח שהתו $A$ מופיע $k$ פעמים ושהתו $B$ מופיע $m$ פעמים. לתו $A$ נבחר $\binom{n}{k}$ מקומות, ול־$B$ נבחר $\binom{n - k}{m}$ מקומות. לכל שאר שני התווים יוותרו $ n - m - k$ מקומות, כלומר $ 2^{n - m - k} $ אפשרויות ייתכנו. 
		נגדיר את הסימון המקוצר: 
		\[ \nsum{x = 0}{n} f(x) := \sum_{x = 0}^n f(x)\cdot (-1)^k + \sum_{x = 0}^n f(x)\cdot 1^k = 2 \cdot \sum_{x = 0 \ (even)}^{n} f(x) \]
		נבחין, שמהבינום של ניוטון, יתקיים: 
		\[ \nsum{k = 0}{n}\binom{n}{k}a^{n - k} = \sumnk\bink a^{n - k}1^k + \sumnk\bink a^{n - k}(-1)^k = (a + 1)^n + (a - 1)^n \]
		
		{\vfil \hfil \textbf{\textit{המשך בעמוד הבא}}  \hfil \vfil}
		
		\pagebreak
		בכל קונטקסט, נסמן $n' = n - k$. סה"כ, כמות האפשרויות מכלל החיבור היא: 
		\begin{alignat}{9}
			  Ans. \, = \; &\sum_{k = 0 \ (even)\ }^n \sum_{m = 0 \ (even)}^{n - k}\bink\binom{n - k}{m}2^{n - k - m}        %(1)
			&&=\sum_{k = 0 \ (even)\ }^n \bink \sum_{m = 0 \ (even)}^{n - k}\binom{m'}{m}2^{m' - m} \\
			= \; &\sum_{k = 0 \ (even)\ }^n \bink \frac{1}{2} \nsum{m = 0}{n - k}\binom{m'}{m}2^{m' - m}                     %(2)
			&&=\sum_{k = 0 \ (even)\ }^n \bink \frac{1}{2} \left [ (2 + 1)^{m'} + (2 - 1)^{m'} \right ] \\
			= \; &\frac{1}{2} \cdot \sum_{k = 0 \ (even)\ }^n \bink 3^{n - k} + \frac{1}{2}\sum_{k = 0 \ (even)\ }^n \bink   %(3)
			&&=\frac{1}{4}\nsum{k = 0}{n} \bink 3^{n - k} + \frac{1}{4}\nsum{k = 0}{n} \bink 1^{n - k} \\
			= \; &\frac{1}{4} \left [ (3 + 1)^n + (3 - 1)^n + (1 + 1)^n + (1 - 1)^n \right ]                                 %(4)
			&&=\frac{1}{4} \left ( 4^n + 2 \cdot 2^n \right ) = 4^{n - 1} + 2^{n + 1 - 2}\\
			= \; &\bm{4^{n - 1} + 2^{n - 1}}                                                                                 %(5)
		\end{alignat}
		\item \textbf{שאלה: }פתרו את 12א בטיעונים קומבינטוריים \\
		\textbf{תשובה: }
		ברור כי יש $2^n$ מחרוזות על $\{C, D\}$, ושיש $4^n$ מחרוזות בסה"כ. עבור מחרוזות בלי $A$ ו־$B$ (כלומר על $\{C, D\}$), יתקיים ש־$A$ יופיע 0 פעמים ובפרט מספר זוגי של פעמים. נטען שעבור $4^n - 2^n$ המקרים הנותרים, מחצית מהמקרים יהיו חוקיים. נוכיח באינדוקציה על $ 1 \le n $. \textit{בסיס: }עבור $n = 1 $, אם המחרוזת היא A אז $A$ מופיע פעם אחת ולכן המקרה שגוי, במקרה השני המחרוזת ריקה וזה תקין – 50\% כדרוש. \textit{צעד: }נניח שהטענה נכונה בעבור מחרוזת באורך לכל היותר $n$, ונוכיח עבור $n + 1 $. תהי מחרוזת באורך $n$, במחצית מהמקרים נוסיף $A$ או $B$ ובמחצית השנייה לא. [לבדוק בגרסה הסופית: משהו עקום כאן] \\
		סה"כ כמות הפתרונות היא $\frac{1}{2} (4^n - 2^n) + 2^n = \frac{1}{2}(4^n + 2^n)$ כדרוש. 
	\end{enumerate}
	
	\dotfill \\
	
	\hfil \textbf{שחר פרץ, 2024} \hfil
	
\end{document}