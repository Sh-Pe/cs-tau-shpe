%! ~~~ Packages Setup ~~~ 
\documentclass[]{article}


% Math packages
\usepackage[usenames]{color}
\usepackage{forest}
\usepackage{ifxetex,ifluatex,amsmath,amssymb,mathrsfs,amsthm,witharrows,mathtools}
\WithArrowsOptions{displaystyle}
\renewcommand{\qedsymbol}{$\blacksquare$} % end proofs with \blacksquare. Overwrites the defualts. 
\usepackage{cancel,bm}
\usepackage[thinc]{esdiff}


% tikz
\usepackage{tikz}
\usetikzlibrary{graphs}
\newcommand\sqw{1}
\newcommand\squ[4][1]{\fill[#4] (#2*\sqw,#3*\sqw) rectangle +(#1*\sqw,#1*\sqw);}


% code 
\usepackage{listings}
\usepackage{xcolor}

\definecolor{codegreen}{rgb}{0,0.35,0}
\definecolor{codegray}{rgb}{0.5,0.5,0.5}
\definecolor{codenumber}{rgb}{0.1,0.3,0.5}
\definecolor{codeblue}{rgb}{0,0,0.5}
\definecolor{codered}{rgb}{0.5,0.03,0.02}
\definecolor{codegray}{rgb}{0.96,0.96,0.96}

\lstdefinestyle{pythonstylesheet}{
	language=Python,
	emphstyle=\color{deepred},
	backgroundcolor=\color{codegray},
	keywordstyle=\color{deepblue}\bfseries\itshape,
	numberstyle=\scriptsize\color{codenumber},
	basicstyle=\ttfamily\footnotesize,
	commentstyle=\color{codegreen}\itshape,
	breakatwhitespace=false, 
	breaklines=true, 
	captionpos=b, 
	keepspaces=true, 
	numbers=left, 
	numbersep=5pt, 
	showspaces=false,                
	showstringspaces=false,
	showtabs=false, 
	tabsize=4, 
	morekeywords={as,assert,nonlocal,with,yield,self,True,False,None,AssertionError,ValueError,in,else},              % Add keywords here
	keywordstyle=\color{codeblue},
	emph={object,type,isinstance,copy,deepcopy,zip,enumerate,reversed,list,set,len,dict,tuple,print,range,xrange,append,execfile,real,imag,reduce,str,repr,__init__,__add__,__mul__,__div__,__sub__,__call__,__getitem__,__setitem__,__eq__,__ne__,__nonzero__,__rmul__,__radd__,__repr__,__str__,__get__,__truediv__,__pow__,__name__,__future__,__all__,},          % Custom highlighting
	emphstyle=\color{codered},
	stringstyle=\color{codegreen},
	showstringspaces=false,
	abovecaptionskip=0pt,belowcaptionskip =0pt,
	framextopmargin=-\topsep, 
}
\newcommand\pythonstyle{\lstset{pythonstylesheet}}
\newcommand\pyl[1]     {{\lstinline!#1!}}
\lstset{style=pythonstylesheet}

\usepackage[style=1,skipbelow=\topskip,skipabove=\topskip,framemethod=TikZ]{mdframed}
\definecolor{bggray}{rgb}{0.85, 0.85, 0.85}
\mdfsetup{leftmargin=0pt,rightmargin=0pt,innerleftmargin=15pt,backgroundcolor=codegray,middlelinewidth=0.5pt,skipabove=5pt,skipbelow=0pt,middlelinecolor=black,roundcorner=5}
\BeforeBeginEnvironment{lstlisting}{\begin{mdframed}\vspace{-0.4em}}
	\AfterEndEnvironment{lstlisting}{\vspace{-0.8em}\end{mdframed}}


% Deisgn
\usepackage[labelfont=bf]{caption}
\usepackage[margin=0.6in]{geometry}
\usepackage{multicol}
\usepackage[skip=4pt, indent=0pt]{parskip}
\usepackage[normalem]{ulem}
\forestset{default}
\renewcommand\labelitemi{$\bullet$}
\usepackage{titlesec}
\titleformat{\section}[block]
{\fontsize{15}{15}}
{\sen \dotfill \ (\thesection) \dotfill \she}
{0em}
{\MakeUppercase}
\usepackage{graphicx}
\graphicspath{ {./} }


% Hebrew initialzing
\usepackage[bidi=basic]{babel}
\PassOptionsToPackage{no-math}{fontspec}
\babelprovide[main, import, Alph=letters]{hebrew}
\babelprovide[import]{english}
\babelfont[hebrew]{rm}{David CLM}
\babelfont[hebrew]{sf}{David CLM}
\babelfont[english]{tt}{Monaspace Xenon}
\usepackage[shortlabels]{enumitem}
\newlist{hebenum}{enumerate}{1}

% Language Shortcuts
\newcommand\en[1] {\begin{otherlanguage}{english}#1\end{otherlanguage}}
\newcommand\sen   {\begin{otherlanguage}{english}}
\newcommand\she   {\end{otherlanguage}}
\newcommand\del   {$ \!\! $}
\newcommand\ttt[1]{\en{\footnotesize\texttt{#1}\normalsize}}

\newcommand\npage {\vfil {\hfil \textbf{\textit{המשך בעמוד הבא}}} \hfil \vfil \pagebreak}
\newcommand\ndoc  {\dotfill \\ \vfil {\begin{center} {\textbf{\textit{שחר פרץ, 2024}} \\ \scriptsize \textit{נוצר באמצעות תוכנה חופשית בלבד}} \end{center}} \vfil	}

\newcommand{\rn}[1]{
	\textup{\uppercase\expandafter{\romannumeral#1}}
}

\makeatletter
\newcommand{\skipitems}[1]{
	\addtocounter{\@enumctr}{#1}
}
\makeatother

\newcommand\bgr[1] {
	\begin{center}
		\en{\tikz\graph {#1}; }
	\end{center}
}

%! ~~~ Math shortcuts ~~~

% Letters shortcuts
\newcommand\N     {\mathbb{N}}
\newcommand\Z     {\mathbb{Z}}
\newcommand\R     {\mathbb{R}}
\newcommand\Q     {\mathbb{Q}}
\newcommand\C     {\mathbb{C}}

\newcommand\ml    {\ell}
\newcommand\mj    {\jmath}
\newcommand\mi    {\imath}

\newcommand\powerset {\mathcal{P}}
\newcommand\ps    {\mathcal{P}}
\newcommand\pc    {\mathcal{P}}
\newcommand\ac    {\mathcal{A}}
\newcommand\bc    {\mathcal{B}}
\newcommand\cc    {\mathcal{C}}
\newcommand\dc    {\mathcal{D}}
\newcommand\ec    {\mathcal{E}}
\newcommand\fc    {\mathcal{F}}
\newcommand\nc    {\mathcal{N}}
\newcommand\sca   {\mathcal{S}} % \sc is already definded
\newcommand\rca   {\mathcal{R}} % \rc is already definded

\newcommand\Si    {\Sigma}

% Logic & sets shorcuts
\newcommand\siff  {\longleftrightarrow}
\newcommand\ssiff {\leftrightarrow}
\newcommand\so    {\longrightarrow}
\newcommand\sso   {\rightarrow}

\newcommand\epsi  {\epsilon}
\newcommand\vepsi {\varepsilon}
\newcommand\vphi  {\varphi}
\newcommand\Neven {\N_{\mathrm{even}}}
\newcommand\Nodd  {\N_{\mathrm{odd }}}
\newcommand\Zeven {\Z_{\mathrm{even}}}
\newcommand\Zodd  {\Z_{\mathrm{odd }}}
\newcommand\Np    {\N_+}

% Text Shortcuts
\newcommand\open  {\big(}
\newcommand\qopen {\quad\big(}
\newcommand\close {\big)}
\newcommand\also  {\text{, }}
\newcommand\defi  {\text{ definition}}
\newcommand\defis {\text{ definitions}}
\newcommand\given {\text{given }}
\newcommand\case  {\text{if }}
\newcommand\syx   {\text{ syntax}}
\newcommand\rle   {\text{ rule}}
\newcommand\other {\text{else}}
\newcommand\set   {\ell et \text{ }}
\newcommand\ans   {\mathscr{A}\!\mathit{nswer}}

% Set theory shortcuts
\newcommand\ra    {\rangle}
\newcommand\la    {\langle}

\newcommand\oto   {\leftarrow}

\newcommand\QED   {\quad\quad\mathscr{Q.E.D.}\;\;\blacksquare}
\newcommand\QEF   {\quad\quad\mathscr{Q.E.F.}}
\newcommand\eQED  {\mathscr{Q.E.D.}\;\;\blacksquare}
\newcommand\eQEF  {\mathscr{Q.E.F.}}
\newcommand\jQED  {\mathscr{Q.E.D.}}

\newcommand\dom   {\mathrm{dom}}
\newcommand\Img   {\mathrm{Im}}
\newcommand\range {\mathrm{range}}

\newcommand\trio  {\triangle}

\newcommand\rc    {\right\rceil}
\newcommand\lc    {\left\lceil}
\newcommand\rf    {\right\rfloor}
\newcommand\lf    {\left\lfloor}

\newcommand\lex   {<_{lex}}

\newcommand\az    {\aleph_0}
\newcommand\uaz   {^{\aleph_0}}
\newcommand\al    {\aleph}
\newcommand\ual   {^\aleph}
\newcommand\taz   {2^{\aleph_0}}
\newcommand\utaz  { ^{\left (2^{\aleph_0} \right )}}
\newcommand\tal   {2^{\aleph}}
\newcommand\utal  { ^{\left (2^{\aleph} \right )}}
\newcommand\ttaz  {2^{\left (2^{\aleph_0}\right )}}

\newcommand\n     {$n$־יה\ }

% Math A&B shortcuts
\newcommand\logn  {\log n}
\newcommand\logx  {\log x}
\newcommand\lnx   {\ln x}
\newcommand\cosx  {\cos x}
\newcommand\cost  {\cos \theta}
\newcommand\sinx  {\sin x}
\newcommand\sint  {\sin \theta}
\newcommand\tanx  {\tan x}
\newcommand\tant  {\tan \theta}
\newcommand\sex   {\sec x}
\newcommand\sect  {\sec^2}
\newcommand\cotx  {\cot x}
\newcommand\cscx  {\csc x}
\newcommand\sinhx {\sinh x}
\newcommand\coshx {\cosh x}
\newcommand\tanhx {\tanh x}

\newcommand\seq   {\overset{!}{=}}
\newcommand\slh   {\overset{LH}{=}}
\newcommand\sle   {\overset{!}{\le}}
\newcommand\sge   {\overset{!}{\ge}}
\newcommand\sll   {\overset{!}{<}}
\newcommand\sgg   {\overset{!}{>}}

\newcommand\h     {\hat}
\newcommand\ve    {\vec}
\newcommand\lv    {\overrightarrow}
\newcommand\ol    {\overline}

\newcommand\mlcm  {\mathrm{lcm}}

\DeclareMathOperator{\sech}   {sech}
\DeclareMathOperator{\csch}   {csch}
\DeclareMathOperator{\arcsec} {arcsec}
\DeclareMathOperator{\arccot} {arcCot}
\DeclareMathOperator{\arccsc} {arcCsc}
\DeclareMathOperator{\arccosh}{arccosh}
\DeclareMathOperator{\arcsinh}{arcsinh}
\DeclareMathOperator{\arctanh}{arctanh}
\DeclareMathOperator{\arcsech}{arcsech}
\DeclareMathOperator{\arccsch}{arccsch}
\DeclareMathOperator{\arccoth}{arccoth} 

\newcommand\dx    {\,\mathrm{d}x}
\newcommand\dt    {\,\mathrm{d}t}
\newcommand\dtt   {\,\mathrm{d}\theta}
\newcommand\du    {\,\mathrm{d}u}
\newcommand\dv    {\,\mathrm{d}v}
\newcommand\df    {\mathrm{d}f}
\newcommand\dfdx  {\diff{f}{x}}
\newcommand\dit   {\limhz \frac{f(x + h) - f(x)}{h}}

\newcommand\nt[1] {\frac{#1}{#1}}

\newcommand\limz  {\lim_{x \to 0}}
\newcommand\limxz {\lim_{x \to x_0}}
\newcommand\limi  {\lim_{x \to \infty}}
\newcommand\limh  {\lim_{x \to 0}}
\newcommand\limni {\lim_{x \to - \infty}}
\newcommand\limpmi{\lim_{x \to \pm \infty}}

\newcommand\ta    {\theta}
\newcommand\ap    {\alpha}

\renewcommand\inf {\infty}
\newcommand  \ninf{-\inf}

% Combinatorics shortcuts
\newcommand\sumnk     {\sum_{k = 0}^{n}}
\newcommand\sumni     {\sum_{i = 0}^{n}}
\newcommand\sumnko    {\sum_{k = 1}^{n}}
\newcommand\sumnio    {\sum_{i = 1}^{n}}
\newcommand\sumai     {\sum_{i = 1}^{n} A_i}
\newcommand\nsum[2]   {\reflectbox{\displaystyle\sum_{\reflectbox{\scriptsize$#1$}}^{\reflectbox{\scriptsize$#2$}}}}

\newcommand\bink      {\binom{n}{k}}
\newcommand\setn      {\{a_i\}^{2n}_{i = 1}}
\newcommand\setc[1]   {\{a_i\}^{#1}_{i = 1}}

\newcommand\cupain    {\bigcup_{i = 1}^{n} A_i}
\newcommand\cupai[1]  {\bigcup_{i = 1}^{#1} A_i}
\newcommand\cupiiai   {\bigcup_{i \in I} A_i}
\newcommand\capain    {\bigcap_{i = 1}^{n} A_i}
\newcommand\capai[1]  {\bigcap_{i = 1}^{#1} A_i}
\newcommand\capiiai   {\bigcap_{i \in I} A_i}

\newcommand\xot       {x_{1, 2}}
\newcommand\ano       {a_{n - 1}}
\newcommand\ant       {a_{n - 2}}

% Other shortcuts
\newcommand\tl    {\tilde}
\newcommand\op    {^{-1}}

\newcommand\sof[1]    {\left | #1 \right |}
\newcommand\cl [1]    {\left ( #1 \right )}
\newcommand\csb[1]    {\left [ #1 \right ]}

\newcommand\bs    {\blacksquare}

%! ~~~ Document ~~~

\author{שחר פרץ}
\title{\textit{עבודה מסכמת במתמטיקה בדידה 2}}
\begin{document}
	\maketitle
	
	{\Large \sen\hfill \textbf{Combinatorics} \hfill\she}
	
	\section{}
	\begin{enumerate}[(A)]
		\item \textbf{שאלה: }כמה סידורים של חבילה מלאה של 52 קלפים יש שבהן ארבעת האסים, אינם מופיעים ברצף אחד אחרי השני? 
		
		\textbf{תשובה: }ראשית כל, נתבונן ב־$52!$ הסידורים האפשריים של החפיסה כולה. עתה נתבונן בקבוצת המשלים – כמות האפשרויות לחפיסות בהן ישנם 4 אסים רצופים. נתייחס לרצף כמו קלף גדול יחודי בפני עצמו, ולכן, מכיוון שארבעת האסים יחשבו כאחד, יהיו $49!$ אפשרויות לסדר חלק זה. לסדר הפנימי של האסים עצמם יהיה $4!$ אפשרויות. סה"כ מכלל הכפל $58 \cdot 48!$ אפשרויות בקבוצת המשלים. סה"כ: 
		\[ \ans = \bm{52 - 49! \, 4!} \]
		\item \textbf{שאלה: }כמה סידורים של חבילה מלאה של 52 קלפים יש בהן כל 4 קלפים מאותו הסוג (13 סוגים שונים) אינם מופיעים ברצף אחד אחרי השני? 
		
		\textbf{תשובה: }נגדיר $=a_i$ כמות האפשרויות לסידור בו $i$ רצפים של 4 תווים. מובן כי $0 \le i \le \frac{52}{4} = 13$ (לא ייתכנו רצפים בסדר גודל הארוך יותר מהחפיסה כולה). 
		
			כדי למצוא את $a_i$, נבחר את הרצף הראשון מבין $13$ האפשרויות. ואת השני מבין $12$ האפשרויות שנותרו, ונמשיך הלאה. באופן דומה לסעיף הקודם, לכל אחד מהסדרות האלו נתייחס קקלף "גדול" יחודי אחד, לכל אחת מ־$i$ הסדרות סדר פנימי של $4!$, וסה"כ סדר כולל של $(52 - 4i + i)!$ אפשרויות ($-4i$ על הקלפים שנוציא החוצה, ו־$+i$ ל"קלף גדול" כמוהו לסדרה עצמה). סה"כ: 
			\[ a_i = i(52 - 3i)!\, 4! \]
			בכלליות:
			
			ומעקרון ההכלה וההדחה, אם $=A_i$ קבוצת כל הרצפים באורך 4 מסוג נתון, ומשום שאין הגבלה על הכלליות בבחירת קלף מסויים, $\sof{\bigcap_{i \in I}A_i}$ זהה בערכו לכל $I \in [n]$ כך ש־$|I|$ קבוע בגודל $k$, ובפרט שווה ל־$a_k$ (המקרה הסמטרי של העקרון), ובשילוב עם עקרון המשלים (על קבוצת על הקומבינציות שגודלה $52!$), נקבל: 
			\begin{align*}
				\ans &= 52! - \sum_{\mathclap{\emptyset \neq I \in [n]}} (-1)^{|I| - 1}\sof{\bigcap_{i \in I}A_i} \\
				&= 52! - \sum_{k = 1}^{n} (-1)^{k - 1}\bink a_k \\
				&= \bm{52! - \sum_{k = 1}^{n} (-1)^{k - 1}\bink k\, (52 - 3k)!\, 4!}
			\end{align*}
	\end{enumerate}
	\section{}
	יהי סריג דו ממדי, ונגדיר מסלול חוקי אמ"מ בכל צעד מ־$\la x, y \ra$ ננוע אך ורק לנקודה $\la x + 1, y + r \ra$ לכל $r \in \N$. 
	\begin{enumerate}[(A)]
		\item \textbf{שאלה: }כמה מסלולים חוקיים קיימים מ־$\la 0, 0 \ra$ ל־$\la n, k\ra$? 
		
		\textbf{תשובה: }יהי מסלול $a := \{a_i\}_{i = 0}^n$ מ־$\la 0, 0 \ra$ ל־$\la n, k \ra$ כאשר $\forall i \in [n]. \exists x, y \in \N. a_i = \la x, y \ra$. נניח שהמסלול חוקי; אזי: 
		\[\forall i \in [n - 1]. \pi_1(a_i) - \pi_1(a_{i + 1}) = 1 \land \exists r \in \N. \pi_2(a_{i + 1}) - \pi_2(a_{i}) = r\]
		ולכן נוכל להגדיר מיפוי: 
		\[ \forall i \in [n - 1]. a_k \mapsto \pi_2(a_{i + 1}) - \pi_2(a_{i}) =: r_i \in \N \]
		חח"ע ועל לקבוצת המסלולים החוקיים. תמונת המיפוי תהיה $\N^{n - 1}$. מהגדרת המסלול, $a_n = \la n, k \ra$ ולכן: 
		\begin{align*}
			\sum_{i = 1}^{n - 1}r_i &= \sum_{i = 1}^{n - 1}\pi_2(a_i) - \pi_2(a_{i + 1}) \\ 
			&= \pi_2(a_1) \cancel{- \pi_2(a_2) + \pi_2(a_2)} \cancel{- \pi_2(a_3) + \pi_2(a_3)} - \cdots \cancel{+ \pi_2(a_i) - \pi_2(a_i)} + \dots + \pi_2(a_n) \\
			&= \pi_2(a_1) + \pi_2(a_n) = 0 + k = k
		\end{align*}
		בכך, התייחסנו לכל ההגבלות – חוקיות המסלול באורך $n$ (מובעת בהיותה חח"ע ועל לקבוצה המאפשרת זאת), והיותו נגמר ב־$\pi_2(a_n) = k$ (הכרחי ומספיק להיות סכום $\sum r_i = k$). נקבע את גודל הסדרות התמונה המקיימות זאת. ידוע שכמות האפשרויות לסכום מספרים יהיה $S(n - 1, k)$, ולכן סה"כ זהו פתרון הבעיה. נסכם: 
		\[ \ans = \bm{S(k, n - 1)} \]
		\item \textbf{שאלה: }כמה מסלולים חוקיים קיימים מ־$\la 0, 0 \ra \to \la n, k \ra$, כך שאף צעד בהם אינו מסתיים בנקודה $\la n, k \ra$?
		
		\textbf{תשובה: }באופן דומה לסעיף הקודם, כמות הצעדים מ־$\la 0, 0 \ra$ ל־$\la 2n, 2k\ra$ תהיה $S(2k, 2n - 1)$. נחפש את קבוצת המשלים. בהינתן מסלול שעובר בין הראשית ל־$\la 2n, 2k\ra$ הוא יכלל בקבוצת המשלים אמ"מ הוא עבור ב־$\la n, k \ra$, כלומר הוא למעשה מסלול $\la 0, 0 \ra \to \la n, k\ra$ ואז עוד מסלול $\la n, k\ra \to \la 2n, 2k \ra$. המסלול האחרון שקול לבעיה הראשונה בעבור טרנספורמציה איזומטרית של $\la x, y \ra \mapsto \la x - n, y - k\ra$ שלמעשה תבהיר כי פתרון שתי הבעיות הוא $S(k, n - 1)$, וכאשר נחבר אותם יחדיו, מכלל הכפל, גודל קבוצת המשלים הוא סה"כ $S(k, n - 1)^2$. אזי: 
		\[ \ans = \bm{S(2k, 2n - 1) - S(k, n - 1)^2} \]
		\item \textbf{שאלה: }כמה מסלולים קיימים מ־$\la 0, 0 \ra \to \la n, k \ra$ כך שכל צעד $\la x_1, y_1 \ra \to \la x_2, y_2 \ra$ מקיים $y_1 + 2 \le y_2$?
		
		\textbf{תשובה: }נבחין שקילות לאחד הנתונים: 
		\[ y_1 + 2 \le y_2 \iff \pi_2(a_i) - \pi_2(a_{i + 1}) \le -2 \iff \underbrace{\pi_2(a_{i + 1}) - \pi_2(a_i)}_{=r_i} \ge 2 \]
		ואכן ננסה למצוא את כמות הסדרות $\{r_i\}_{i = 1}^{n - 1}$ כך ש־$r_i \ge 2$, כך ש־$\sum r_i = k$, לפי השקילות שהוכחה בסעיף (א). לבעיה זו קיימת בעיה שקולה ידועה, היא חלוקת $k$ כדורים ל־$n - 1$ תאים, כשבכל תא לפחות 2 כדורים. אזי, ניאלץ להתחיל מלשים שני כדורים בכל תא, וסה"כ נבזבז $2n - 2$ כדורים. את $k - 2n - 2$ הכדורים נותרים נחלק בין התאים. סה"כ, קיבלנו: 
		\[ \ans = \bm{S(k - 2n - 2, n - 1)} \]
	\end{enumerate}
	\section{}
	יהיו $n$ כדורים ממוספרים.  יש לסדרם ב־$n$ תאים ממוספרים, כאשר בכל תא יימצא בדיוק כדור אחד. לכל $1 \le i \le n - 1$ עסור להכניס את הכדור ה־$i$ לתא ה־$i$, בעוד אין מגבלה על הכדור ה־$i$. כמות האפשרויות לסידורים כאלו תהיה $F(n)$. 
	\begin{enumerate}[(A)]
		\item \textbf{שאלה: }הביעו את $F(n)$ בעזרת $D_m$. 
		
		\textbf{תשובה: }נפלג למקרים. 
		\begin{itemize}
			\item אם הכדור ה־$i$ נמצא בתא ה־$i$, אז יש עוד $n - 1$ תאים נותרים בהם אי־אפשר שכדור יהיה בתא המתאים לו מבחינת מספר. כלומר, יהיו $D_{n - 1}$ אפשרויות. 
			\item אם הכדור ה־$i$ לא נמצא בתא ה־$i$, אז כל $n$ הכדורים לא נמצאים בתא המתאים להם, כלומר יש $D_n$ אפשרויות. 
		\end{itemize}
		סה"כ מכלל החיבור: 
		\[ \ans = D_n + D_{n - 1} \]
		
		\item 
	\end{enumerate}
	\section{}
	\begin{enumerate}[(A)]
		\item הוכיחו באופן קומבינרטורי: 
		\[ \sum_{i = 0}^{n - 1}(-1)^{i} \binom{n}{i} \binom{n + r - i - 1}{r} = \binom{r - 1}{n - 1} \]
		אין לי מושג... 
		\item מצאו ביטוי ללא סכימה לאגף שמאל של המשוואה: 
		\[ \sum_{k = 2}^{n} k(k - 1)\bink = n(n - 1) \cdot 2^{n - 2} \]
		\textbf{סיפור: }מתוך $n - 1$ איברים, קבוצה של לפחות שני איברים, ומתוכה נבחר שניים שונים ונסמנם בכחול ובירוק. כמה אפשרויות יש לכך? 
		
		\textbf{אגף ימין: }נבחר כדור כחול ($n$ אופציות) ולאחריו ירוק ($n - 1$ אופציות). עתה, בעבור $n - 2$ האיברים הנותרים, נשייך להם את המספר $1$ אם נרצה להכניסם לקבוצה ו־$0$ אם לאו – לכך, יהיו $|\{0, 1\}|^{n - 2}$ אפשרויות. סה"כ מכלל הכפל $n(n - 1)2^{n - 2}$ אפשרויות. 
		
		\textbf{אג ף שמאל: }נניח שגודל הקבוצה הוא $2 \le k \le n$ (בהכרח גודל הקבוצה גדול מ־2 כי קיים מה כדור כחול וירוק) – לבחירה מתוך קבוצה $\bink$ אופציות. לכן, מתוך $n$ האיברים שיש לנו, נבחר $k$ איברים לשים בקבוצה. מאילו, נבחר אחד כחול ($k$ אפשרויות) ואחד ירוק ($k - 1$ אפשרויות) וסה"כ מכלל הכפל $\binom{n}{k}k(k - 1)$ בעבור $k$ נתון, ומכלל החיבור $\sum_{k = 2}^{n}k(k - 1)\bink$ אופציות. 
		
	\end{enumerate}
	\section{}
	
	\setcounter{section}{0}
	{\Large \sen\hfill \textbf{Graph Theory} \hfill\she}
	
	\section{}
	נוכיח או נפריך קיום גרף מתאים: 
	\begin{enumerate}[(A)]
		\item 6 צמתים מדרגות $1, 3, 3, 3, 4, 5$. \textbf{נפריך קיום. }נניח בשלילה שקיים גרף כזה, אזי קיים גרף בעל 5 צמתים מדרגה זוגית ($1, 3 \times 3, 5$) בסתירה למשפט לפיו קיים מספר זוגי (ובפרט אינו $5$) של צמתים בעלי דרגה אי זוגית. 
		\item 6 צמתים מדרגות $1, 3, 3, 3, 5, 5$. \textbf{נפריך קיום. }נניח בשלילה קיום גרף כזה. אזי, קיים שני קודודים מדרגה $5$, היא פחותה ב־1 מכמות הצמתים בגרף כולו – ומשום זה לא יכול להכיל קשת בינו צומת לבין עצמה, הם יפנו לכל שאר הצמתים. אזי, הצומת $v$ שקיים מהנתונים ודרגתו $1$ יופנה משתי הצמתים הללו (שדרגתן $5$), וסה"כ $1 = d(v) \ge 2$ וזו סתירה. 
		\item 6 צמתים מדרגות $1, 3, 3,3, 4, 4$. \textbf{נוכיח קיום. } 
		
		\bgr{a$_1$ -- b$_4$ -- {c$_3$, d$_4$, e$_4$} -- f$_3$, c$_3$ -- d$_4$, e$_4$-- d$_4$}
		
	\end{enumerate}
	\section{}
	\begin{enumerate}[(A)]
		\item צ.ל. בכל עץ עם $n \ge 2$ צמתים יש לפחות שני עלים. 
		\begin{proof}
			נניח בשלילה קיום עץ בעל $n \ge 2$ צמתים, שיש לו פחות משני עלים. אזי, ל־$n - 1$ מהצמתים בו הם אינם עלים, ולכן דרגתם היא $d(v) \ge 2$. נסמן ב־$\tl v$ את הקודקוד היחיד שלא ידוע שמקיים זאת, בעבורו $d(\tl v) \ge 1$ (עם $d(\tl v) = 0$ אז הגרף אינו קשיר וזו סתירה). ממשפט על סכום הדרגות וכמות הצמתים ביחס לכמות קשתות בגרף, נקבל: 
			\[ \begin{WithArrows}
				&2(|V| - 1) = 2|E| = \sum_{v \in V}d(v) = d(\tl v) + \sum_{\mathclap{v \in V \setminus \{\tl v\}}}d(v) \ge 1 + 2(n - 1) = 2n - 1 \Arrow[down]{$\times 0.5$} \\
				&|V| - 1 \ge \frac{2n - 1}{2} \implies n = |V| \ge n + 0.5 \implies 0 \ge 0.5
			\end{WithArrows} \]
			וזו סתירה. 
			\item יהי $G = \la V, E \ra$ גרף. נוכיח שלכל גרף $H = \la [n], E_h \ra $, שאיזומורפי ל־$G$ מתקיים $G = H$ אמ"מ $V = \varnothing$. 
			\begin{proof}
				content...
			\end{proof}
		\end{proof}
	\end{enumerate}
	\section{}
	יהי $G = \la V, E \ra$ גרף. נניח $\forall v \in V. d(v) \ge k >1$. צ.ל. קיום מעגל פשוט באורך לפחות $k + 1$. 
	\begin{proof}
		נניח בשלילה שהמעגל הפשוט המקסימלי $U$ הוא באורך $m \le k$. נראה באינדוקציה על $j$ המסלול הארוך ביותר הכולל צומת יחיד במעגל, ש־$j$ לא חסום.
		\begin{itemize}
			\item בסיס: נניח $j = 0$ כלומר המעגל מכיל את כל הצמתים בגרף, אזי נתון מעגל באורך $m \le k$, וידוע שלכל אחד מ־$m$ הקודקודים דרגה $k$, וכבר במעגל מחוברים לשני קודקודים נוספים ומשום שהגרף פשוט לא תתיכן קשת בין צומת לעצמה, כלומר מבין $m$ הצמתים במעגל ל־$m - 3$ ייתכן החיבור, בעוד נותר לחבר ל־$k - 2$ צמתים נוספים. נבחין בסתירה כי $k - 2 \ge m - 2 > m - 3$, כלומר אין מספיק צמתים לחבר אליהם. כן בעבור כל קודקוד, כלומר יש צורך ב־$k$ צמתים נוספים, ואכן כל קודקוד מתחבר לקודקוד שמחוץ למעגל כלומר $j > 0$. 
			\item צעד: נניח באינדוקציה על נכונות הטענה על $j - 1$ ונוכיחה בעבור $j$. נתבונן בקצה המסלול באורך $j$ אותו נסמן ב־$J$, בו ימצא קודקוד $v$. ידוע $d(v) \ge k$. אם ישלח איזושהי צומת אל המעגל, נסיק כי $U \cup \{v\}$ מעגל פשוט באורך $m + 1$, סתירה לכך ש־$U$ הוא המינימלי. אם ישלח קשת אל אחד מהקודקודים הידועים המסלול שאינו $J$, בה"כ $\tl J$, אז $U \cup J \cup \{v\}$ מעגל פשוט באורך גדול מ־$m$ בסתירה לכך ש־$m$ אורך המעגל המינימלי. מכיוון שלא שלח קשת לקודקוד ב־$U$ או לאחד מהמסלולים $\tl J$ שיצאו מ־$U$, ניוותר עם שני מקרים: הראשון, בו שלח קשת לקודקוד שאיננו קשור למדובר עד כה, אז המסלול $J$ יתארך ויהיה ל־$j + 1$ ובכך אכן $j$ לא חסום וסיימנו, וסה"כ הוא בהכרח ישלח צומת לקודקוד ב־$J$. לכן, $J \ge d(v) \ge k$, וגם קודקודי $J$ מהווים מעגל (הרי הם כולם מקושרים במסלול, ועתה יש צומת המחברת בין הראשון לאחרון במסלול היא $v$) אבל המעגל הזה באורך $|J| \ge k + 1$ על אף שאורך המעגל המקסימלי הוא $m \le k$. מצאנו בכל מקרים סתירה, כדרוש. 
		\end{itemize}
		סה"כ, בעבור כל ערך $j$, יתקיים שבהכרח נצטרך ערך $j$ גדול יותר (לכן $j$ לא חסום). ניתן דעתנו על כך שהטענה זו מהווה סתירה, כי אם $j$ גדול לא חסום ובפרט גדול ככל רצוננו ומשום ש־$j \le |V|$, אז  $|V|$ לא חסום ויש כמות אין־סופית של קודקודים. בכך ההנחה בשלילה הוכחה כשגויה, ותמה ההוכחה. 
%		\begin{itemize}
%			\item בסיס $n = 0$, אזי הגרף בהכרח ריק ו־$d(v0)$ לא מוגדר בעבור שום קודקוד, והטענה נכונה באופן ריק. 
%			\item צעד, נניח את נכונות הטענה באינדוקציה מלאה על $k < n$ ונוכיחה על $n$. נניח בשלילה שהמסלול המקסימלי בגרף הוא לכל היותר $k$. נתבונן במסלול המקסימלי $\la v_i \ra ^j_{i = 0}$ כאשר $j \le k$. ידוע $d(v_j) \ge k$, ובפרט, משום שאורך המסלול הפשוט הוא $k$ נדע שיכלול $k - 1$ קודקודים שאינם $v_j$, ולכן קיים עוד לפחות קודקוד $v$, שאינו במעגל, ו־$v_j$ מקשר אליו. אם $v$ בעל מסלול מהוביל לאחד קודקודים במסלול, שלא דרך $v_j$, אז מצאנו מסלול באורך $k + 1$, וזו סתירה. אחרת, הוצאת הקשת בין $v$ לבין $v_j$ הגרף תפריד את הגרף לכזה שיכיל רכיב קשירות בו המעגל המקורי נמצא, ושאר הגרף. 
%		\end{itemize}
	\end{proof}
	\section{}
	יהי $G = \la [n], E_G \ra$ גרף. נוכיח שלכל $H = \la [n], E_H \ra$ שאיזימורפי ל־$G$ מתקיים $G = H$, אם ורק אם:
	משהו סגנון לחבר מעגלים
	\begin{itemize}
		\item[$\implies$]
		\item[$\impliedby$]
	\end{itemize}
	\section{}
	\section{}
	\section{}
	\section{}
	\section{}
	\section{}
	
\end{document}