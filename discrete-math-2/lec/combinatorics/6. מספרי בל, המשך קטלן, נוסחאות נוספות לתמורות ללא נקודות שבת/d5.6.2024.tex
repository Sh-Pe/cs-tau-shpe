\documentclass[]{article}

% Math packages
\usepackage[usenames]{color}
\usepackage{forest}
\usepackage{ifxetex,ifluatex,amsmath,amssymb,mathrsfs,amsthm,witharrows,mathtools}
\WithArrowsOptions{displaystyle}
\renewcommand{\qedsymbol}{$\blacksquare$} % end proofs with \blacksquare. Overwrites the defualts. 
\usepackage{cancel,bm}

% tikz
\usepackage{tikz}
\newcommand\sqw{1}
\newcommand\squ[4][1]{\fill[#4] (#2*\sqw,#3*\sqw) rectangle +(#1*\sqw,#1*\sqw);}


% code 
\usepackage{listings}
\usepackage{xcolor}

\definecolor{codegreen}{rgb}{0,0.35,0}
\definecolor{codegray}{rgb}{0.5,0.5,0.5}
\definecolor{codenumber}{rgb}{0.1,0.3,0.5}
\definecolor{deepblue}{rgb}{0,0,0.5}
\definecolor{deepred}{rgb}{0.5,0.03,0.02}

\lstdefinestyle{pythonstylesheet}{
	language=Python,
	morekeywords={}
	emphstyle=\color{deepred},
	backgroundcolor=\color{white},   
	commentstyle=\color{codegreen}\itshape,
	keywordstyle=\color{deepblue}\bfseries\itshape,
	numberstyle=\tiny\color{codenumber},
	basicstyle=\ttfamily\footnotesize,
	breakatwhitespace=false, 
	breaklines=true, 
	captionpos=b, 
	keepspaces=true, 
	numbers=left, 
	numbersep=5pt, 
	showspaces=false,                
	showstringspaces=false,
	showtabs=false, 
	tabsize=2, 
	morekeywords={object,type,isinstance,copy,deepcopy,zip,enumerate,reversed,list,set,len,dict,tuple,range,xrange,append,execfile,real,imag,reduce,str,repr},              % Add keywords here
	keywordstyle=\color{deepblue},
	emph={__init__,__add__,__mul__,__div__,__sub__,__call__,__getitem__,__setitem__,__eq__,__ne__,__nonzero__,__rmul__,__radd__,__repr__,__str__,__get__,__truediv__,__pow__,__name__,__future__,__all__,as,assert,nonlocal,with,yield,self,True,False,None},          % Custom highlighting
	emphstyle=\color{deepred},
	stringstyle=\color{deepgreen},
	showstringspaces=false
}
\newcommand\pythonstyle{\lstset{pythonstylesheet}}
\newcommand\pyl[1]     {{\pythonstyle\lstinline!#1!}}
\lstset{style=pythonstylesheet}


% Deisgn
\usepackage[labelfont=bf]{caption}
\usepackage[margin=0.6in]{geometry}
\usepackage{multicol}
\usepackage[skip=4pt, indent=0pt]{parskip}
\usepackage[normalem]{ulem}
\forestset{default}
\renewcommand\labelitemi{$\bullet$}
\usepackage{titlesec}
%\titleformat{\section}[block]
%	{\fontsize{15}{15}}
%	{\sen \dotfill \, \!\!\! \thesection \,\! \dotfill \she}
%	{1em}
%	{\MakeUppercase}

% Hebrew initialzing
\usepackage{polyglossia}
\setmainlanguage{hebrew}
\setotherlanguage{english}
\newfontfamily\hebrewfont[Script=Hebrew, Ligatures=TeX]{David CLM}
\usepackage[shortlabels]{enumitem}
\newlist{hebenum}{enumerate}{1}
\setlist[hebenum,1]{
	labelindent=\parindent,
	label={{\hebrewfont{\protect\hebrewnumeral{\value{hebenumi}}}}.}
}

% Language Shortcuts
\newcommand\en[1] {\selectlanguage{english}#1\selectlanguage{hebrew}}
\newcommand\sen   {\selectlanguage{english}}
\newcommand\she   {\selectlanguage{hebrew}}
\newcommand\del   {$ \!\! $}
\newcommand\ttt[1]{\en{\texttt{#1}}}

\newcommand\npage {\vfil {\hfil \textbf{\textit{המשך בעמוד הבא}}} \hfil \vfil}
\newcommand\ndoc  {\dotfill \\ \vfil \hfil \textbf{\textit{שחר פרץ, 2024}} \hfil \vfil}

\newcommand{\rn}[1]{
	\textup{\uppercase\expandafter{\romannumeral#1}}
}


%! ~~~ Math shortcuts ~~~

% Letters shortcuts
\newcommand\N     {\mathbb{N}}
\newcommand\Z     {\mathbb{Z}}
\newcommand\R     {\mathbb{R}}
\newcommand\Q     {\mathbb{Q}}
\newcommand\C     {\mathbb{C}}

\newcommand\ml    {\ell}
\newcommand\mj    {\jmath}
\newcommand\mi    {\imath}

\newcommand\powerset {\mathcal{P}}
\newcommand\ps    {\mathcal{P}}
\newcommand\pc    {\mathcal{P}}
\newcommand\ac    {\mathcal{A}}
\newcommand\bc    {\mathcal{B}}
\newcommand\cc    {\mathcal{C}}
\newcommand\dc    {\mathcal{D}}
\newcommand\ec    {\mathcal{E}}
\newcommand\fc    {\mathcal{F}}
\newcommand\nc    {\mathcal{N}}
\newcommand\sca   {\mathcal{S}} % \sc is already definded
\newcommand\rca   {\mathcal{R}} % \rc is already definded

\newcommand\Si    {\Sigma}

% Logic & sets shorcuts
\newcommand\siff  {\longleftrightarrow}
\newcommand\ssiff {\leftrightarrow}
\newcommand\so    {\longrightarrow}
\newcommand\sso   {\rightarrow}

\newcommand\epsi  {\epsilon}
\newcommand\vepsi {\varepsilon}
\newcommand\vphi  {\varphi}
\newcommand\Neven {\N_{\mathrm{even}}}
\newcommand\Nodd  {\N_{\mathrm{odd }}}
\newcommand\Zeven {\Z_{\mathrm{even}}}
\newcommand\Zodd  {\Z_{\mathrm{odd }}}
\newcommand\Np    {\N_+}

% Text Shortcuts
\newcommand\open  {\big(}
\newcommand\qopen {\quad\big(}
\newcommand\close {\big)}
\newcommand\also  {\text{, }}
\newcommand\defi  {\text{ definition}}
\newcommand\defis {\text{ definitions}}
\newcommand\given {\text{given }}
\newcommand\case  {\text{if }}
\newcommand\syx   {\text{ syntax}}
\newcommand\rle   {\text{ rule}}
\newcommand\other {\text{else}}
\newcommand\set   {\ell et \text{ }}
\newcommand\ans   {\mathit{Ans.}}

% Set theory shortcuts
\newcommand\ra    {\rangle}
\newcommand\la    {\langle}

\newcommand\oto   {\leftarrow}

\newcommand\QED   {\quad\quad\mathscr{Q.E.D.}\;\;\blacksquare}
\newcommand\QEF   {\quad\quad\mathscr{Q.E.F.}}
\newcommand\eQED  {\mathscr{Q.E.D.}\;\;\blacksquare}
\newcommand\eQEF  {\mathscr{Q.E.F.}}
\newcommand\jQED  {\mathscr{Q.E.D.}}

\newcommand\dom   {\text{dom}}
\newcommand\Img   {\text{Im}}
\newcommand\range {\text{range}}

\newcommand\trio  {\triangle}

\newcommand\rc    {\right\rceil}
\newcommand\lc    {\left\lceil}
\newcommand\rf    {\right\rfloor}
\newcommand\lf    {\left\lfloor}

\newcommand\lex   {<_{lex}}

\newcommand\az    {\aleph_0}
\newcommand\uaz   {^{\aleph_0}}
\newcommand\al    {\aleph}
\newcommand\ual   {^\aleph}
\newcommand\taz   {2^{\aleph_0}}
\newcommand\utaz  { ^{\left (2^{\aleph_0} \right )}}
\newcommand\tal   {2^{\aleph}}
\newcommand\utal  { ^{\left (2^{\aleph} \right )}}
\newcommand\ttaz  {2^{\left (2^{\aleph_0}\right )}}

\newcommand\n     {$n$־יה\ }

% Math A&B shortcuts
\newcommand\logn  {\log n}
\newcommand\cosx  {\cos x}
\newcommand\cost  {\cos \theta}
\newcommand\sinx  {\sin x}
\newcommand\sint  {\sin \theta}
\newcommand\tanx  {\tan x}
\newcommand\tant  {\tan \theta}
\newcommand\dx    {\,\mathrm{d}x}

\newcommand\seq   {\overset{!}{=}}
\newcommand\sle   {\overset{!}{\le}}
\newcommand\sge   {\overset{!}{\ge}}
\newcommand\sll   {\overset{!}{<}}
\newcommand\sgg   {\overset{!}{>}}

\newcommand\h     {\hat}
\newcommand\ve    {\vec}
\newcommand\lv    {\overrightarrow}
\newcommand\ol    {\overline}

\newcommand\mlcm  {\mathrm{lcm}}

\newcommand\limz  {\lim_{x \to 0}}
\newcommand\limxz {\lim_{x \to x_0}}
\newcommand\limi  {\lim_{x \to \infty}}
\newcommand\limni {\lim_{x \to - \infty}}
\newcommand\limpmi{\lim_{x \to \pm \infty}}

\newcommand\ta    {\theta}
\newcommand\ap    {\alpha}

\renewcommand\inf {\infty}
\newcommand  \ninf{-\inf}

% Combinatorics shortcuts
\newcommand\sumnk     {\sum_{k = 0}^{n}}
\newcommand\sumni     {\sum_{i = 0}^{n}}
\newcommand\sumnko    {\sum_{k = 1}^{n}}
\newcommand\sumnio    {\sum_{i = 1}^{n}}
\newcommand\sumai     {\sum_{i = 1}^{n} A_i}
\newcommand\nsum[2]   {\reflectbox{\displaystyle\sum_{\reflectbox{\scriptsize$#1$}}^{\reflectbox{\scriptsize$#2$}}}}

\newcommand\bink      {\binom{n}{k}}

\newcommand\cupain    {\bigcup_{i = 1}^{n} A_i}
\newcommand\cupai[1]  {\bigcup_{i = 1}^{#1} A_i}
\newcommand\cupiiai   {\bigcup_{i \in I} A_i}

\newcommand\sof[1]    {\left | #1 \right |}
\newcommand\cl [1]    {\left ( #1 \right )}

\newcommand\xot       {x_{1, 2}}
\newcommand\ano       {a_{n - 1}}
\newcommand\ant       {a_{n - 2}}

% Other shortcuts
\newcommand\tl    {\tilde}
\newcommand\op    {^{-1}}

\newcommand\bs    {\blacksquare}

%! ~~~ Document ~~~

\author{שחר פרץ}
\title{קומבי 8 $\sim$ נטלי שלום }
\date{5 ביוני 2024}

\begin{document}
	\maketitle
	\section{המשך ההוכחה מהשיעור הקודם}
	תזכורת להגדרה ולסימון: $=\cc_n$ מספר קטלן ה־$n$־י, מספר הסדרות המאוזנות באורך $2n$. 
	
	\textbf{טענה: }$\begin{cases}
		\cc_n = \sum_{i = 0}^{n - 1}\cc_i \cdot \cc_{n - i - 1}\\
		\cc_0 = 1
	\end{cases}$
	
	\textbf{טענה: }(הביטוי הסגור ל־$\cc_n$) $\cc_n = \binom{2n}{n} - \binom{2n}{n + 1}$. 
	
	\begin{proof}
		\ [המשך ההוכחה מהשיעור הקודם] [ההוכחה קומבינטורית]. הבעיה שרצינו לראות שקילות לה, היא מה מספר הדרכים במישור, להגיע מהנקודה $(0, 0)$ ל־$(n, n)$ ע''י צעדים של ימינה ומעלה, מבלי להיות מעל האלכסון $y = x$?
		
		אגף שמאל, כבור הוסבר בשיעור הקודם. 
		
		נסביר את אגף ימין (הביטוי הבינומי). נסתכל על כל ההכילוכים של ימינה ולמעלה מ־$(0, 0)$ ל־$(n, n)$. עלינו לבצע $n$ צעדים ימינה, ו־$n$ צעדים למעלה. נבחר מתוך $2n$ צעדים, אילו $n$ הם ימינה, סה''כ $\binom{2n}{n}$. 
		
		נסתכל על הליוך לא חוקי כלשהו. נתבונן בנקודה הראשונה בה עברנו את האלכסון, שערכה בהכרח יהיה $(k, k + 1)$. כמות הצעדים ימינה שנותרו עד הנקודה $(n, n)$, היא $n - k$. וכמות הצעדים למעלה $m - k - 1 $. \textit{(עקרון השיקוף)} מהנקודה הזו והלאה נשקף את המשך ההילוך. כלומר, כל צעד ימינה נחליף בלמעלה ולהפך. כלומר: ימינה $n - k - 1 $ ולמעלה $n - k $. נסיים בנקודה $(k + m - k - 1, k + 1 + n - k) = (n - 1, n + 1)$. כלומר, ההתאמה בין הילוכים לא חוקיים להילוכים כלשהם מ־$(0, 0)$ ל־$(n - 1, n + 1)$ היא חח''ע ועל. לכן, כמות ההילוכים הלא חוקיים היא $\binom{2n}{n + 1}$ (קצת כמו קודם, נבחר מתוך $2n$ צעדים, מהם ה־$n + 1$ צעדים למעלה). מעקרון המשלים נקבל את הדרוש. 
	\end{proof}
	
	שימושי לדעת, שניתן לפשט את הביטוי הסגור, ולקבל: 
	\[ \cc_n = \binom{2n}{n} - \binom{2n}{n + 1} = \frac{1}{n + 1} \binom{2n}{n} = \prod_{k = 2}^{n}\frac{n + k}{k} \]
	
	\section{מספרי בל}
	\textbf{הגדרה: }$\bc_n$ (מספר בל ה־$n$־י) הוא מספר החלוקות של הקבוצה $\{1, \dots, n\}$. דוגמה: 
	\[ \bc_3 = |\{\{\{1, 2, 3\}\}, \ \{\{1\}, \{2, 3\}\}, \ \{\{2\}, \ \{1, 3\}\}, \ \{\{3\}, \{1, 2\}\}, \ \{\{1\}, \{2\}, \{3\}\}\}| \]
	
	\textbf{טענה: }(נוסחת נסיגה עבור $\bc_n$): 
	\[ \begin{cases}
		\bc_n = \sumnko \binom{n - 1}{k - 1} \cdot \b_{n - k} \\
		\bc_0 = 1
	\end{cases} \]
	
	\begin{proof}
		נסתכל על המספר $n$. נניח שהוא נמצא בחלוקה, בקבוצה שגודלה $k$ ($1 \le k \le n$). כמות האפשרויות ליתר $k - 1 $ האיברים בקבוצה היא $\binom{n - 1}{k - 1}$. את $n - k$ האיברים הנותרים, נחלק בדרך כלשהי. לכך יש $\bc_{n - k }$ אפשרויות. סה''כ מעקרון הכפל והסכום, נקבל את הדרוש. 
	\end{proof}
	הנוסחה הסגורה למספרי בל תהיה: 
	\[ \bc_n = \frac{1}{e} \sum_{k = 0}^{\inf} \frac{k^{n}}{k!} \]
	לא נוכיח אותה, כי נדרשים כלים בחדו''א על־מנת להגיע אליה. 
	\section{הערות נוספות}
	\subsection{תמורות ללא נקודות שבת}. 
	\textbf{תזכורת: }מספר התמורות על $n$ איברים ללא נקודות שבת הוא
	\[\begin{cases}
		D_n = n! \sum_{i = 0}^{n}\frac{(-1)^{i}}{i!} \\
		D_1 = 0
	\end{cases}\]
	שתי נוסחאות נסיגה נוספות שניתן להוכיח, הן: 
	\begin{gather}
		D_n = n! - \sumnko \bink D_{n - k} \\
		D_n = (n - 1) (D_{n - 1} + D_{n - 2})
	\end{gather}
	נוכל להוכיח את שתיהן קומבינטורית. 
	
	\subsection*{תרגיל – שילוש מצולעים קמורים}
	\textit{הערה: }כדי להצליח לפתור דברים בעזרת קטלן, נרצה להוכיח התאמה חח''ע ועל לבעיה אחרת. 
	
	\textbf{תרגיל: }בכמה דרכים ניתן לשלש (לחלק למשולשים בעזרת אלכסונים שאינן חותכים) מצולע קמור בעל $n /= 2$ קודקודים ממוספרים? נסמן ב־$T_n$. 
	
	\textbf{םתרון: }נמספר את הקודקודים בסדר עולה. נתבונן בצלע $(1, 2)$, ונראה לאיזה משולש היא שייכת. נפריד למקרים. 
	\begin{itemize}
		\item אם היא נמצאת במשולש עם הקודקוד $3$ או $n + 2$ (הקודקודים הצמודים) אז נותרנו עם מצולע בעל $n + 1$ קודקודים שעלינו לשלש, ויש $T_{n - 1}$ אפשרויותץ 
		\item אחרת, אם הקודקוד השלישי הוא $4 \le k \le n + 1$ אז נותרנו עם מצולע בגודל $k - 1$ ומצולע בגודל $n - k + 4$. לכן: 
		\[ T_n = 2T_{n - 1} + \sum_{k = 4}^{n + 1}T_{k - 3}T_{n - k + 2} = \open \set j = k - 3 \close \quad 2t_{n - 1} + \sum_{j = 0}^{n - 2} T_{j}T_{n - k - 1} = \sum_{j = 0}^{n - 1} T_j \dot T_{n - j - 1} = \cc_n \]
		יש לנו גם את אותו תאי ההתחלה, $T_0 = \cc_0 = 1 $ ולכן סה''כ $T_n = \cc_n$. 
	\end{itemize}
	
\end{document}