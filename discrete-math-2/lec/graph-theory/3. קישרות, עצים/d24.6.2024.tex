\documentclass[]{article}

% Math packages
\usepackage[usenames]{color}
\usepackage{forest}
\usepackage{ifxetex,ifluatex,amsmath,amssymb,mathrsfs,amsthm,witharrows,mathtools}
\WithArrowsOptions{displaystyle}
\renewcommand{\qedsymbol}{$\blacksquare$} % end proofs with \blacksquare. Overwrites the defualts. 
\usepackage{cancel,bm}
\usepackage[thinc]{esdiff}

% tikz
\usepackage{tikz}
\newcommand\sqw{1}
\newcommand\squ[4][1]{\fill[#4] (#2*\sqw,#3*\sqw) rectangle +(#1*\sqw,#1*\sqw);}


% code 
\usepackage{listings}
\usepackage{xcolor}

\definecolor{codegreen}{rgb}{0,0.35,0}
\definecolor{codegray}{rgb}{0.5,0.5,0.5}
\definecolor{codenumber}{rgb}{0.1,0.3,0.5}
\definecolor{codeblue}{rgb}{0,0,0.5}
\definecolor{codered}{rgb}{0.5,0.03,0.02}
\definecolor{codegray}{rgb}{0.95,0.95,0.95}

\lstdefinestyle{pythonstylesheet}{
	language=Python,
	emphstyle=\color{deepred},
	backgroundcolor=\color{codegray},
	keywordstyle=\color{deepblue}\bfseries\itshape,
	numberstyle=\scriptsize\color{codenumber},
	basicstyle=\ttfamily\footnotesize,
	breakatwhitespace=false, 
	breaklines=true, 
	captionpos=b, 
	keepspaces=true, 
	numbers=left, 
	numbersep=5pt, 
	showspaces=false,                
	showstringspaces=false,
	showtabs=false, 
	tabsize=2, 
	morekeywords={object,type,isinstance,copy,deepcopy,zip,enumerate,reversed,list,set,len,dict,tuple,range,xrange,append,execfile,real,imag,reduce,str,repr},              % Add keywords here
	keywordstyle=\color{codeblue},
	emph={__init__,__add__,__mul__,__div__,__sub__,__call__,__getitem__,__setitem__,__eq__,__ne__,__nonzero__,__rmul__,__radd__,__repr__,__str__,__get__,__truediv__,__pow__,__name__,__future__,__all__,as,assert,nonlocal,with,yield,self,True,False,None,AssertionError,ValueError},          % Custom highlighting
	emphstyle=\color{codered},
	stringstyle=\color{codegreen},
	showstringspaces=false,
	abovecaptionskip=0pt,belowcaptionskip =0pt,
	framextopmargin=-\topsep, 
}
\newcommand\pythonstyle{\lstset{pythonstylesheet}}
\newcommand\pyl[1]     {{\lstinline!#1!}}
\lstset{style=pythonstylesheet}

\usepackage[style=1,skipbelow=\topskip,skipabove=\topskip,framemethod=TikZ]{mdframed}
\definecolor{bggray}{rgb}{0.85, 0.85, 0.85}
\mdfsetup{leftmargin=0pt,rightmargin=0pt,backgroundcolor=codegray,middlelinewidth=0.5pt,skipabove=4pt,skipbelow=0pt,middlelinecolor=black,roundcorner=5}
\BeforeBeginEnvironment{lstlisting}{\begin{mdframed}\vspace{-0.4em}}
	\AfterEndEnvironment{lstlisting}{\vspace{-0.8em}\end{mdframed}}

% Deisgn
\usepackage[labelfont=bf]{caption}
\usepackage[margin=0.6in]{geometry}
\usepackage{multicol}
\usepackage[skip=4pt, indent=0pt]{parskip}
\usepackage[normalem]{ulem}
\forestset{default}
\renewcommand\labelitemi{$\bullet$}
\usepackage{titlesec}
\usepackage{graphicx}
\graphicspath{ {./} }

% Hebrew initialzing
\usepackage[bidi=basic]{babel}
\PassOptionsToPackage{no-math}{fontspec}
\babelprovide[main, import]{hebrew}
\babelfont{rm}{David CLM}
\babelfont{sf}{David CLM}
\babelfont{tt}{Monaspace Argon}
\usepackage[shortlabels]{enumitem}
\newlist{hebenum}{enumerate}{1}

% Language Shortcuts
\newcommand\en[1] {\selectlanguage{english}#1\selectlanguage{hebrew}}
\newcommand\sen   {\selectlanguage{english}}
\newcommand\she   {\selectlanguage{hebrew}}
\newcommand\del   {$ \!\! $}
\newcommand\ttt[1]{\en{\small\texttt{#1}\normalsize}}

\newcommand\npage {\vfil {\hfil \textbf{\textit{המשך בעמוד הבא}}} \hfil \vfil \pagebreak}
\newcommand\ndoc  {\dotfill \\ \vfil {\begin{center} {\textbf{\textit{שחר פרץ, 2024}} \\ \scriptsize \textit{נוצר באמצעות תוכנה חופשית בלבד}} \end{center}} \vfil	}

\newcommand{\rn}[1]{
	\textup{\uppercase\expandafter{\romannumeral#1}}
}

\makeatletter
\newcommand{\skipitems}[1]{
	\addtocounter{\@enumctr}{#1}
}
\makeatother

%! ~~~ Math shortcuts ~~~

% Letters shortcuts
\newcommand\N     {\mathbb{N}}
\newcommand\Z     {\mathbb{Z}}
\newcommand\R     {\mathbb{R}}
\newcommand\Q     {\mathbb{Q}}
\newcommand\C     {\mathbb{C}}

\newcommand\ml    {\ell}
\newcommand\mj    {\jmath}
\newcommand\mi    {\imath}

\newcommand\powerset {\mathcal{P}}
\newcommand\ps    {\mathcal{P}}
\newcommand\pc    {\mathcal{P}}
\newcommand\ac    {\mathcal{A}}
\newcommand\bc    {\mathcal{B}}
\newcommand\cc    {\mathcal{C}}
\newcommand\dc    {\mathcal{D}}
\newcommand\ec    {\mathcal{E}}
\newcommand\fc    {\mathcal{F}}
\newcommand\nc    {\mathcal{N}}
\newcommand\sca   {\mathcal{S}} % \sc is already definded
\newcommand\rca   {\mathcal{R}} % \rc is already definded

\newcommand\Si    {\Sigma}

% Logic & sets shorcuts
\newcommand\siff  {\longleftrightarrow}
\newcommand\ssiff {\leftrightarrow}
\newcommand\so    {\longrightarrow}
\newcommand\sso   {\rightarrow}

\newcommand\epsi  {\epsilon}
\newcommand\vepsi {\varepsilon}
\newcommand\vphi  {\varphi}
\newcommand\Neven {\N_{\mathrm{even}}}
\newcommand\Nodd  {\N_{\mathrm{odd }}}
\newcommand\Zeven {\Z_{\mathrm{even}}}
\newcommand\Zodd  {\Z_{\mathrm{odd }}}
\newcommand\Np    {\N_+}

% Text Shortcuts
\newcommand\open  {\big(}
\newcommand\qopen {\quad\big(}
\newcommand\close {\big)}
\newcommand\also  {\text{, }}
\newcommand\defi  {\text{ definition}}
\newcommand\defis {\text{ definitions}}
\newcommand\given {\text{given }}
\newcommand\case  {\text{if }}
\newcommand\syx   {\text{ syntax}}
\newcommand\rle   {\text{ rule}}
\newcommand\other {\text{else}}
\newcommand\set   {\ell et \text{ }}
\newcommand\ans   {\mathit{Ans.}}

% Set theory shortcuts
\newcommand\ra    {\rangle}
\newcommand\la    {\langle}

\newcommand\oto   {\leftarrow}

\newcommand\QED   {\quad\quad\mathscr{Q.E.D.}\;\;\blacksquare}
\newcommand\QEF   {\quad\quad\mathscr{Q.E.F.}}
\newcommand\eQED  {\mathscr{Q.E.D.}\;\;\blacksquare}
\newcommand\eQEF  {\mathscr{Q.E.F.}}
\newcommand\jQED  {\mathscr{Q.E.D.}}

\newcommand\dom   {\text{dom}}
\newcommand\Img   {\text{Im}}
\newcommand\range {\text{range}}

\newcommand\trio  {\triangle}

\newcommand\rc    {\right\rceil}
\newcommand\lc    {\left\lceil}
\newcommand\rf    {\right\rfloor}
\newcommand\lf    {\left\lfloor}

\newcommand\lex   {<_{lex}}

\newcommand\az    {\aleph_0}
\newcommand\uaz   {^{\aleph_0}}
\newcommand\al    {\aleph}
\newcommand\ual   {^\aleph}
\newcommand\taz   {2^{\aleph_0}}
\newcommand\utaz  { ^{\left (2^{\aleph_0} \right )}}
\newcommand\tal   {2^{\aleph}}
\newcommand\utal  { ^{\left (2^{\aleph} \right )}}
\newcommand\ttaz  {2^{\left (2^{\aleph_0}\right )}}

\newcommand\n     {$n$־יה\ }

% Math A&B shortcuts
\newcommand\logn  {\log n}
\newcommand\cosx  {\cos x}
\newcommand\cost  {\cos \theta}
\newcommand\sinx  {\sin x}
\newcommand\sint  {\sin \theta}
\newcommand\tanx  {\tan x}
\newcommand\tant  {\tan \theta}

\newcommand\seq   {\overset{!}{=}}
\newcommand\sle   {\overset{!}{\le}}
\newcommand\sge   {\overset{!}{\ge}}
\newcommand\sll   {\overset{!}{<}}
\newcommand\sgg   {\overset{!}{>}}

\newcommand\h     {\hat}
\newcommand\ve    {\vec}
\newcommand\lv    {\overrightarrow}
\newcommand\ol    {\overline}

\newcommand\mlcm  {\mathrm{lcm}}

\DeclareMathOperator{\sech}   {sech}
\DeclareMathOperator{\csch}   {csch}
\DeclareMathOperator{\arcsec} {arcsec}
\DeclareMathOperator{\arccot} {arcCot}
\DeclareMathOperator{\arccsc} {arcCsc}
\DeclareMathOperator{\arccosh}{arccosh}
\DeclareMathOperator{\arcsinh}{arcsinh}
\DeclareMathOperator{\arctanh}{arctanh}
\DeclareMathOperator{\arcsech}{arcsech}
\DeclareMathOperator{\arccsch}{arccsch}
\DeclareMathOperator{\arccoth}{arccoth} 

\newcommand\dx    {\,\mathrm{d}x}
\newcommand\dt    {\,\mathrm{d}t}
\newcommand\dtt   {\,\mathrm{d}\theta}
\newcommand\df    {\mathrm{d}f}
\newcommand\dfdx  {\diff{f}{x}}
\newcommand\dit   {\limhz \frac{f(x + h) - f(x)}{h}}

\newcommand\nt[1] {\frac{#1}{#1}}

\newcommand\limz  {\lim_{x \to 0}}
\newcommand\limxz {\lim_{x \to x_0}}
\newcommand\limi  {\lim_{x \to \infty}}
\newcommand\limni {\lim_{x \to - \infty}}
\newcommand\limpmi{\lim_{x \to \pm \infty}}

\newcommand\ta    {\theta}
\newcommand\ap    {\alpha}

\renewcommand\inf {\infty}
\newcommand  \ninf{-\inf}

% Combinatorics shortcuts
\newcommand\sumnk     {\sum_{k = 0}^{n}}
\newcommand\sumni     {\sum_{i = 0}^{n}}
\newcommand\sumnko    {\sum_{k = 1}^{n}}
\newcommand\sumnio    {\sum_{i = 1}^{n}}
\newcommand\sumai     {\sum_{i = 1}^{n} A_i}
\newcommand\nsum[2]   {\reflectbox{\displaystyle\sum_{\reflectbox{\scriptsize$#1$}}^{\reflectbox{\scriptsize$#2$}}}}

\newcommand\bink      {\binom{n}{k}}
\newcommand\setn      {\{a_i\}^{2n}_{i = 1}}
\newcommand\setc[1]   {\{a_i\}^{#1}_{i = 1}}

\newcommand\cupain    {\bigcup_{i = 1}^{n} A_i}
\newcommand\cupai[1]  {\bigcup_{i = 1}^{#1} A_i}
\newcommand\cupiiai   {\bigcup_{i \in I} A_i}
\newcommand\capain    {\bigcap_{i = 1}^{n} A_i}
\newcommand\capai[1]  {\bigcap_{i = 1}^{#1} A_i}
\newcommand\capiiai   {\bigcap_{i \in I} A_i}

\newcommand\xot       {x_{1, 2}}
\newcommand\ano       {a_{n - 1}}
\newcommand\ant       {a_{n - 2}}

% Other shortcuts
\newcommand\tl    {\tilde}
\newcommand\op    {^{-1}}

\newcommand\sof[1]    {\left | #1 \right |}
\newcommand\cl [1]    {\left ( #1 \right )}
\newcommand\csb[1]    {\left [ #1 \right ]}

\newcommand\bs    {\blacksquare}

%! ~~~ Document ~~~

\author{שחר פרץ}
\title{גרפים 3 $\sim$ נטלי שלום $\sim$ קשירות, עצים}
\date{24 ליוני 2024}

\begin{document}
	\maketitle
	\section{תזכורות}
	\begin{itemize}
		\item גרף מסמנים  ב־$G = \la V, E \ra$
		\item $d(v)$ היא הדרגה של קודקוד $v$. 
		\item מסלול מקודקוד $a$ ל־$b$ הוא סדרה של קודקודים $\la v_0, \dots v_m \ra$ כך ש־$v_0 = a, v_m = b$ וכך ש־: 
		\begin{enumerate}
			\item בין קל שני קודקודים עוקבים סדרה, יש קשת, כלומר $\forall 0 \le i < m. \{v_i, v_{i + 1}\} \in E$
			\item אין קשת שחוזרת פעמיים, כלומר $\forall i \neq j. \{v_{i}, v_{i + 1}\} \neq \{v_j, v_{j + 1}\}$. קודקוד יכול לחזור פעמיים. 
		\end{enumerate}
		\item אורך של מסלול, הוא מס' הקשתות שבו. בפרט, סדרה באורך $1$ היא מסלול באורך $0$. 
		\item מעגל: מסלול $v_0, \dots v_m$ שבו $v_0 = v_m$ ו־$m> 0$, נקרא מעגל. 
		\item מסלול פשוט, הוא מסלול שבו שעל הצמתים שונים זה מזה. 
		\item מעגל פשוט, הוא מסלול שבו כל הצמתים שונים זה מזה, פרט לכך שהראשון והאחרון זהים. 
	\end{itemize}
	\subsection{תרגיל}
	נתון גרף $G = \la V, E \ra$ כך שהדרגה של כל קודקוד היא לפחות $d \ge 2$, ונתון $|V| \le d^2 - d$. הוכיחו, שיש ב־$G$ מעגל באורך $4$. 
	\begin{proof}
		נניח בשלילה שקיים $G$ כזה ללא מעגלים באורך $4$. יהי $v \in V$, אז קיימים לו לפחות $d$ שכנים, נסמן $d$ מתוכם: $A = \{u_1, \dots, u_d\}$. נבחין בכמה דברים: 
		\begin{itemize}
			\item לכל שני שכנים $u_i, u_j$ של $v$, לא ייתכן קיום שכן משותף שאינו $v$, כי אחרת, אם היה קיים $w$ כזה, אז קיים מעגל באורך $4$ שהוא $v, u_i, w, u_j, v$. 
			\item כל $u_i$ הוא שכן של לכל היותר מקודקוד אחד ב־$A$, כי אם $u_i$ שכן של $u_j, u_k$                      ($i \neq j \neq k$) אז נקבל מעגל $v, u_j, u_i, u_k, v$. 
		\end{itemize}
		לכן, לכל שכן של $v$ יש לפחות $d - 2$ שכנים שאינם $v$. נקבל: 
		\[ |V| \ge \underbrace{1}_{\mathclap{\text{הקודקוד $v$}}} + \underbrace{d}_{A} + \underbrace{d}_{A} + \underbrace{d(d - 2)}_{\mathclap{\text{השכנים של $A$ ללא $v$ ו־$A$}}} = d^2 - d + 1 \]
		סה"כ סתירה לחסם $|V| \le d^2 - d$. 		
	\end{proof}
	
	\section{קשירות}
	\textbf{הגדרה: }יהי $G = \la V, E \ra$ גרף. נגדיר יחס $\sim$ מעל $V$ באופן הבא: $\forall a, b \in V. a \sim b \iff \text{קיים מסלול בין $a$ ל־$b$}$
	
	\textbf{טענה: }$\sim$ יחס שקילות מעל $V$. 
	\begin{proof}[הוכחה (קצת מעפנה). ]
		רפלקסיביות, סימטריות – "קל``. 
		
		טרנזיטיביות: לא טרוויאלי. הבעיה: קשתות שיעשו overlapping. הוכחה לא קונסטקטיבית: על בסיס קיום הילוך ידוע קיום מסלול פשוט ביניהם, בלי לבנות את המסלול. הוכחה קונסטרקטיבית: 
		
		יהי $a = v_0, \dots v_m = b$, $b = u_0, \dots u_k = c$. ניקח את הקודקוד הראשון ב־$v_0, \dots v_m$ כך שקיים $0 \le j \le k$ שמופיע גם ב־$u_0, \dots u_k$. נניח שהוא $v_i = u_j$, אז ניקח את המסלול $v_0, \dots v_{i}, u_{j + 1},\dots, u_k$. 		
	\end{proof}
	\textbf{הגדרה: }כל מחלקת שקילות ביחס $\sim$ מעל $V$, נקראת \textit{רכיב קשירות}. 
	
	\textit{הערה: }מספר רכיבי הקשירות בגרף הוא תכונה שנשמרת תחת איזומורפיזם. 
	
	\textbf{הגדרה: }גרף נקרא \textit{קשיר} (connected) אם יש בו רכיב קשירות יחיד. 
	
	כלומר, גרף הוא קשיר אם בין כל שני קודקודים בו קיים מסלול. 
	
	\textit{הערה: }כל מחלקת שקילות (כל רכיב קשירות) הוא תת־גרף קשיר של הגרף המקורי. 
	
	\textbf{משפט: }בגרף קשיר, בין כל שני קודקודים קיים מסלול פשוט. 
	(הוכחנו, בשיעור הקודם, טענה האומרת שאם יש מסלול בין שני קודקודים אז יש מסלול פשוט ביניהם). 
	
	\textit{תזכורת: }הגרף המשלים של $G = \la V, E \ra$ הוא $\ol G = \la V, \ol E \ra$ כך ש־$\ol E = \ps_2(V) \setminus E$. 
	
	\textbf{משפט: }לכל כרף $G = \la V, E \ra$, לפחות אחד מבין הגרפים $G, \ol G$ הוא קשיר. 
	
	\begin{proof}
		אם $G$ קשיר, סיימנו. לכן, נניח ש־$G$ לא קשיר ונוכיח ש־$\ol G$ קשיר. יהיו $u, v \in V$. נוכיח שקיים ביניהם מסלול ב־$\ol G$. נפריד למקרים: 
		\begin{itemize}
			\item אם היה קיים ביניהם מסלול ב־$G$, אז מאחר ש־$G$ לא קשיר, אז קיים קודקוד $a$ כך שאין מסלול ב־$G$ בינו לבין $u$. ולכן גם אין מסלול ב־$G$ בין $a$ ל־$v$, ובפרט, הקשתות $\{a, u\}, \{a, v\}$ לא ב־$G$ ולכן הן שייכות ל־$\ol E$, ואז נקבל שיש מלסול ב־$\ol G$ בין $u$ ו־$v$: $u, a, v$. 
			\item אם לא היה קיים מסלול בין $u$ ל־$v$ בגרף $G$: אז בפרט $\{u, v\} \notin E$ כלומר $\{u, v\}\in \ol E$ וסה"כ המסלול $\la u, v \ra$ הוא מסלול בין $u$ ל־$v$. 
		\end{itemize}
		סה"כ בין כל שני קודקודים ב־$\ol G$ יש מסלול ולכן הוא קשיר. 
	\end{proof}
	
	\textbf{משפט: }\textit{(מספר הקשתות המינימלי בגרף קשיר)} עבור $G$ גרף קשיר, מתקיים $|E| \ge |V| - 1$. 
	
	\textit{הערה: }ההפך לא נכון. 
	
	אינטואיציה: נרצה לחבר אותם בקו ישר. 
	
	\begin{proof}
		לשם ההוכחה, נרצה נראה שתי למות (טענת עזר, שהיא לא ממש משפט): 
		\begin{itemize}
			\item \textbf{למה 1. }בגרף קשיר שבו $|E| < |V|$ ו־$|V| \ge 2$, קיים קודקוד מדרגה 1. 
			\begin{proof}[הוכחה של למה 1. ]
				דרגה $0$ לא תיתכן כי אז יהיה קודקוד מבודד [=קודקוד ללא שכנים] ואז לא יהיה מסלול בינו לבין יתר הקודקודים. לכן, אם נניח בשלילה שלא קיים קודקוד מדרגה 1, אז כל הקודקודים מדרגה לפחות $2$. ואז: 
				 		\[ 2 \cdot |E| = \sum_{v \in V} d(v) \ge 2 \cdot |V| \implies |E| \ge |V| \]
				 		וזו סתירה לנתון $|E| < |V|$. 
			\end{proof}
			\item 		\textbf{למה 2. }אם מסירים מגרף קשיר צומת [=קודקוד] בעל דרגה $1$ [כןכן צומת זה זכר] ואת הקשת שנוגעת בו, אז הגרף נישאר קשיר. 
			\begin{proof}[הוכחה של למה 2. ]
			יהי $G = \la V, E \ra$ קשיר, $u \in V$ קודקוד מדרגה $1$, ונסיר אותו. עלינו להוכיח שתת הגרף על $V \setminus \{u\}$ קשיר. יהיו $x, y \in V \setminus \{u\}$. מאחר ש־$G$ קשיר, היה ביניהם מסלול ב־$G$. לא ייתכן שהמסלול הנ"ל עבר ב־$u$ מכיוון שהיינו אמורים לחזור פעמיים על אותה הקשת היחידה שיוצאת מ־$u$. 
			\end{proof}
		\end{itemize}
		נעבור להוכחת המשפט (יאי 🎊). נוכיח באינדוקציה על $|V| = n$. 
		\begin{itemize}
			\item \textbf{בסיס: }בעבור $n = 1$ מתקיים $|E| = 0 \ge |V| - 1 = 0$
			\item \textbf{צעד: }נניח שהטענה נכונה עבור $n - 1$, כלומר שבגרף קשיר על $n - 1 $קודקודים מתקיים שמספר הקשתות הוא לפחות $n -2$. נוכיח עבור $n$. יהי $G$ גרף קשיר על $n$. נניח בשלילה שיש בו $|E| \le n - 2 < n = |V|$. לפי למה 1, קיים בו צומת מדרגה $1$. לפי למה 2, נוכל להסיר אותו, ולהיוותר עם גרף קשיר. נקבל גרף קשיר עם $n - 1$ קודקודים שבו $|E| \le n - 3$, לכן, מהנחת האינד', נקבל סתירה, שהרי $|E| \ge n - 2$. 
		\end{itemize}		
	סה"כ הטענה הוכחה. 
	\end{proof}
	
	\section{עצים}
	\textbf{הגדרה: }\textit{עץ} הוא גרף קשיר ללא מעגלים. 
	
	\textbf{הגדרה: }\textit{יער} הוא גרף ללא מעגלים. 
	
	\textbf{הגדרה: }\textit{עלה} הוא צומת ביער בעל דרגה $1$. 
	
	\textbf{משפט: }\textit{(מספר הקשתות המקסימלי ביער)} בגרף חסר מעגלים בעל $n$ צמתים, יש לכל היותר $n - 1$ קשתות. כלומר $|E| \le |V| - 1$. 
	
	\textit{הערה: }ההפך לא נכון. 
	\begin{proof}
		לצורך הוכחת המשפט, נראה למה. 
		
		\textbf{למה 3. }אם מסירים קשת מגרף חסר מהגלים, אז מספר רכיבי הקשירות בגרף גדל [בדיוק באחד, אך לא נצטרך להוכיח זאת]. "אני לא רושמת את ההוכחה" (tbh זה יחסית קל)
		
		נעבור להוכחת המשפט. באינדוקציה על $|V| = n$: 
		\begin{itemize}
			\item \textit{בסיס: }עבור $n = 1$, יתקיים $|E| = 0 \le |V| - 1 = 0$. 
			\item \textit{צעד: }נניח שהטענה נכונה עבור כל $1 \le k < n$, ונוכיח עבור $n$. יהי $G$ גרף חסר מעגלים על $n$ צמתים. נסיר ממנו את הקשת $\{x, y\}$. לפי למה 3, מספר רכיבי הקשירות (אותו נסמן ב־$t$) לאחר ההסרה גדל, כלומר $t \ge 2$. נסמן את רכיבי הקשירות של $G$ לאחר הסרת הקשת ב־$G_i = \la V_i, E_i \ra$ עבור $1 \le i \le t$. כל רכיב קשירות מקיים $|V_i| < |V| = n$, והוא חסר מעגלים. לכן, מה"א $|E_i| \le |V_i| - 1$. לכן, מספר הקשתות בגרף המקורי, (לפני ההסרה של $x, y$) הוא: 
			\[ |E| \le \underbrace{1}_{\{x, y\}} + \sum_{i = 1}^{t}|E_i| \le 1 + \sum_{i = 1}^{t}|V_i| - 1 = 1 - t + \underbrace{\sum_{i = 1}^{t}|V_i|}_{|V|} \underset{t \le 2}{\le} |V| - 1  \]
		\end{itemize}
	\end{proof}
\end{document}
